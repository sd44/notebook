\chapter{俄国的修正主义}

\section{19世纪90年代俄国马克思主义的三个派别}

19世纪90年代马克思主义崛起过程中,正统派逐渐演化出两个不同的马克思主义派别:
“\textbf{合法马克思主义}”和“\textbf{经济主义}”。正统派被定义为坚持普列汉诺夫
思想体系的\textbf{信条},马克思和恩格斯著作中的任何主张不仅要应用于新情况,而且
不需要作更改或修订。正统派的主要支持者是普列汉诺夫和列宁。“合法马克思主义”包括
司徒卢威、杜冈-巴拉诺夫斯基和布尔加科夫,杜冈-巴拉诺夫斯基是最为重要的经济理论
家。1900年之前,他们接受普列汉诺夫制定的政治方案,与此同时,对马克思主义的理论基
础采取了\textbf{批判}的态度。20世纪初,司徒卢威、布尔加科夫和其他不太重要的“合
法马克思主义”者,不仅逐渐脱离了马克思主义,而且还脱离了唯物主义和社会主义,转而
迎合自由主义、唯心主义和温和派。

“经济主义”是另一种形式的\textbf{修正主义},但是,无论是在起源上还是在试图加
以修正的内容上,“经济主义”同“合法马克思主义”是大相径庭的。\textbf{合法马克思
主义者是一些和发展中的工人运动很少联系的知识分子,而经济主义者则深深地涉足工人运
动。此外,合法马克思主义集中于修正理论而非实践,而经济主义则几乎完全相反。}从本
质上说,经济主义者们\textbf{不关心理论问题},而是寻求扩大社会民主党参与经济运动
的程度。他们认为,组织罢工、促进工会的形成和赢得\textbf{国家的法律让步},是至关
重要的。这些活动不仅是实现政治目标的手段,而且还可以取代资产阶级对民主革命的领导。
为了证明自己的观点,他们有时候的确需要求助于德国的修正主义,伯恩施坦把他们视为自
己的支持者。

“经济主义”反映了马克思主义中作为最终目标的社会主义,同用于实现这一目标的工
人运动手段之间存在的冲突。经济主义者因其愿意受限于工人运动、追随而非积极领导工人
运动而著称。从理论上讲,他们对政治经济学不感兴趣。但他们在俄国社会民主党的演进中
起着非常重要的作用。他们同正统马克思主义冲突的过程,也是 “列宁主义”——通常被理
解为\textbf{把阶级同政党联系起来}的理论——产生的过程。列宁最著名的著作
\textbf{《怎么办》},就是为了回应经济主义的挑战而撰写的(参见以下第十一章)。

基于本章的论题,我们对俄国修正主义的研究仅限于合法马克思主义,因为合法马克思
主义理论具有重要的意义。这不仅因为司徒卢威声称自己提出了国际修正主义涉及的根本问
题;而且在对正统派的批评中,合法马克思主义比德国修正主义更为深刻。但是,在政治上,
俄国的修正主义争论相对来说并不是什么重要的事件。19世纪90年代,合法马克思主义和正
统派视民粹主义为它们共同的敌人,它们正是由于这一原因而团结一致的。此外,在当时,
所有的合法马克思主义者都接受这样一种观点,即\textbf{俄国需要一场资产阶级民主革命,
而不愿意接受经济主义}。在1898-1903年,经济主义被正统派视为真正的威胁。到1901年,
合法马克思主义已经从社会民主党的行列中完全脱离出来,这进一步使普列汉诺夫和列宁不
太愿意与他们建立系统的密切联系。当然,正统派对俄国修正主义并没有采取完全沉默的态
度,但正统派的反应时而很分散,时而很克制。只是在1900年之后,马克思主义革命阵营中
才逐渐出现对修正主义的更为充分的批判。

合法马克思主义提出的对马克思主义的“修正”,包含的内容十分广泛,其中包括辩证
法、唯物史观、资本主义经济“运动规律”和价值理论。尽管后两个方面与政治经济学存在
密切的联系,但前两个方面对列宁后来的思想演变却起着至关重要的作用。

\section{辩证法和历史唯物主义}

1894年,司徒卢威出版了《俄国经济发展问题的评述》,该著作的初衷是对民粹主义进行马
克思主义式的批评。但事实上,它引起了以上第九章讨论的19世纪90年代的“大论
战”。……在马克思主义政治经济学方面,有以下三个重要的方面:首先,司徒卢威主张划
时代的变迁可能采取进化的方式;\textbf{对抗性力量可能会相互适应},结果不是矛盾的
激化,而是\textbf{矛盾的“缓和”}。其次,在批判民粹派时,司徒卢威描绘了资本主义
发展的一种不流血的形式。丹尼尔逊只看到资本主义消极的一面,司徒卢威则一直
\textbf{强调它积极的一面}。这反映在《俄国经济发展问题的评述》中的著名结束语上:
“\textbf{承认自己的不文明并向资本主义学习!}”这就给民粹派的指责留下了口实,即
至少一些马克思主义者事实上对资产阶级秩序进行了辩护。再次,在解释\textbf{农业经济
危险}的状态时,司徒卢威利用了一些非马克思主义的理论,特别是吸收了马尔萨斯主义的
理论。司徒卢威认为,\textbf{人口过剩}是造成19世纪90年代初发生饥荒的关键因素。

普列汉诺夫毫不费力地避开了进化式社会变迁的一般主张。\textbf{没有一个马克思主义者
曾经主张所有的进步都是通过革命实现的,}马克思主义辩证法的概念是抽象的,它的具体
表现形式可能多种多样的。

列宁也没有公开地做出让步,但在他后来对马克思主义的发展中,却体现出对司徒卢威
实质批判上作出了\textbf{重大让步}。1905年之后,列宁的思想发生了较大的转变,他开
始接受司徒卢威“\textbf{缓和了的矛盾}”和\textbf{非革命式发展概念}的合理性。列宁
把俄国历史解释为“\textbf{自上而下的重建}”的过程,在这个过程中,沙皇和占主导地
位的地主阶级力图适应那些维护自己生存必需的资产阶级的某些秩序。1905年的事件把资产
阶级卷入这一过程;资产阶级在执政联盟中已经初步获得一定的地位,在面临民众的激进主
义时,他们变成了彻底的保守派。列宁由此而逐渐相信俄国资产阶级的转变,很可能采取渐
进的“普鲁士式”的方式(参见以下第十一章)。实际上,20世纪初,列宁和司徒卢威就此
问题曾达成过真正的共识;他们的分歧与其说是政治上的,还不如说是分析上的。在司徒卢
威的政治立场日益转向右翼,并乐于接受改良时,列宁为了抵制改良,转向了左翼。但是,
人们还远没有弄清楚,列宁的观点在多大程度上实际地受惠于司徒卢威(参见本章以下第7
节)。在19世纪90年代的大部分时间里,他们的私人关系相对密切,并且列宁对司徒卢威从
未丧失的知识能力作过高度评价,尽管在1900年之后,列宁对他进行了谩骂和攻击。

但是,在资本主义经济发展问题上,列宁没有对司徒卢威的分析作出丝毫的让步,他认
为后者的分析严重低估了资本主义发展的代价,是一种\textbf{“客观主义”而非“唯物主
义”}。在这一点上,恩格斯本人完全赞同。更具体地说,列宁相信资本主义矛盾一开始就
在起作用。这与他倾向于把资本主义视为一个过程的观点完全一致(参见以上第九章),在
这种情况下,他比司徒卢威更接近于丹尼尔逊。并且,列宁在其它方面可能过于“矫枉过
正”。他低估了\textbf{人口增长}在引起农民生活条件恶化上的实际上的重要性。此外,
尽管马克思本人敌视马尔萨斯,但马克思还是认为,过多的人口增长可能对封建生产关系有
着天然的影响。

\section{消费不足和管理资本主义}

同司徒卢威的《俄国经济发展问题的评述》一样,杜冈-巴拉诺夫斯基在对消费不足论的批
评上并没有放过马克思。像民粹派经济学家一样,马克思被指责为“\textbf{西斯蒙第主
  义}”,即在资本主义经济中,\textbf{有效需求总是不足的},因为工人太穷,以至于不
能买回他们生产的全部净产品。

列宁超越了这一点,指出\textbf{消费不足可以被解释为比例失调的一种形式},包括第Ι
部类和第\Rnum{2} 部类的比例失调,因此,他的观点同杜冈-巴拉诺夫斯基的危机理论基本上是相
一致的。但是,列宁的观点不够成熟,因为他并没有令人信服地说明,为什么这种特定形式
的比例失调比其它形式的比例失调更易于发生。考茨基和后来的布哈林也没有就这个问题增
添任何新的内容。希法亭也是如此。……罗莎·卢森堡的观点较为深刻,她指责杜冈-巴拉
诺夫斯基的工作只不过是一种“算术练习”,它无法解释是什么在推动资本主义进行无休止
的投资。卢森堡所作的批判是以她的资本积累理论为基础的,如同以上第六章中看到的那样,
这种资本积累理论缺乏连贯性。(只有依据凯恩斯的观点,马克思主义者才能够令人信服地
阐明,卢森堡基于错误的理论基础的批判中蕴涵着的真理成分。)这并不让人感到意外;以
上第九章已经指出,马克思主义者缺乏的恰恰是\textbf{分析消费不足问题所需要的一个连
贯一致的有效需求理论。}

正是由于缺乏这样的理论,才出现了对杜冈-巴拉诺夫斯基的曲解。他被指责为是“调
和主义者”和倾向于认为管理资本主义具有可能性。在杜冈-巴拉诺夫斯基的文本中,这两
种指责都难以成立。事实上,杜冈曾主张,考虑到资本主义本质上是分权式的、无政府主义
的体制,它不可能消除比例失调,并且将产生长期的以周期形式表现的经济危机。但杜冈并
没有尝试用对比例失调的复杂性问题或对资本主义国家经济功能局限性的详细分析,来证明
他自己的结论。此外,杜冈-巴拉诺夫斯基明确地\textbf{拒绝接受所有的崩溃理论}。他
不相信比例失调引起的危机有越来越严重的趋势,并且对其它的崩溃理论进行了批评,特别
是马克思的\textbf{利润率下降理论}。

\section{利润率下降和无产阶级的贫困化}

比例协调论包含着对马克思利润理论的批判。\textbf{如果资本主义可以完全自动化,并不
  断地进行积累,那么利润就不可能仅仅来自对工人的剥削}。但是,杜冈-巴拉诺夫斯没把
这一点当会事,他实际的批判集中在\textbf{利润率下降规律}上。在这一问题上,他是最早
宣称马克思的理论存在严重的逻辑错误的学者之一。他认为,资本有机构成的不断提高,反
映了劳动生产率的提高,从而减少了工人阶级的必要劳动时间量。如果实际工资没有提高,
这将使得剥削率的增加足以提高利润率(或至少保持不变)。\textbf{他假设,技术进步包
  括了不变资本对直接劳动的替代。不变资本的价值等于或小于它替代的劳动力的价值},杜
冈-巴拉诺夫斯基坚持认为,“基于这些假设,\textbf{新的技术条件下的生产量不可能下
  降},要不然的话,机器生产代替手工生产就没有任何经济意义”。他的结论
是,\textbf{利润率要么不变,要么上升。}

杜冈-巴拉诺夫斯基的分析存在一些不足,但他的整个观点是合理的。马克思并没有忽
视对剩余价值率上升起“反作用的各种趋势”,这一点正如杜冈-巴拉诺夫斯基在一定程度
上所提示的那样,但马克思显然不认为这会破坏他整个理论的结构。这可能构成了列宁对杜
冈-巴拉诺夫斯基做出如下评价的基础,列宁认为,杜冈-巴拉诺夫斯基“只是随意地进行
修改……以便反驳马克思”,整个程序是“极为愚蠢和荒谬的”。此外,杜冈-巴拉诺夫斯
基的批判没有认识到马克思观点的复杂性,\textbf{马克思是在分析存在多种商品经济背景
下的问题,创新既提高了资本有机构成,又提高了剩余价值率,因此在初始价格下创新才是
有利可图的;只有在所有的资本家完全采用新技术时,新的价格才意味着利润率的下降。}

杜冈-巴拉诺夫斯基的观点的逻辑结构,局限于单一商品生产的世界,生产过程包含了
\textbf{生产自身和劳动力生产}(尽管他本人并没有认识到这一点,而且有时候他也使用
三部门模型,但是,\textbf{在他的模型中,资本有机构成是始终不变的})。因此,相对
价格是无法改变的(假设工资不变的话),杜冈-巴拉诺夫斯基\textbf{没有正视马克思的
分析}。然而,要证明一个假定的一般规律无效,只需要一个反例就行了,杜冈-巴拉诺夫
斯基提出这样的“特殊案例”对马克思进行反驳。他的见解是有力的。这已被下一代马克思
主义者的著名的“置盐定理”所证明,\textbf{“置盐定理”指出,只要实际工资保持不变,
成本节约型的创新确实会提高利润率。}

但是,杜冈-巴拉诺夫斯基的观点对正统派几乎没有产生什么影响。直到\textbf{1929
年}亨里克·格罗斯曼出版《\textbf{资本主义制度的积累规律和崩溃}》时,利润率下降规
律才开始在马克思主义经济危机理论中发挥重要作用(参见以下第十六章)。但是,提出沿
着这一思路发展下去将会是死胡同的观点的先驱者,非杜冈-巴拉诺夫斯基莫属。他的观点
在马克思主义的进一步演化中有着重大的意义,因为这种观点强化了他认为资本主义不存在
崩溃趋势的信念。正因为如此,\textbf{杜冈才主张社会主义需要一个非经济的基础},这
样,他预见到了后来“西方马克思主义”的发展(参见本章以下第7节)。

杜冈-巴拉诺夫斯基和其他一些合法马克思主义者都坚定地认为,\textbf{资本主义的
成熟包含着实际工资的提高}。像伯恩施坦和德国的修正主义者一样,\textbf{他们相信
“贫困化”仅限于资本主义发展的早期阶段};一旦资本主义作为占主导地位的生产方式建
立起来,就会出现工资上升的趋势。当然,历史站在他们这一边;资本主义发展过程中,实
际工资确实提高了。但是,历史事实并不能替代分析性的解释,而他们恰恰没有能够提供这
样一种分析性的解释。正统马克思主义也好不到哪里。普列汉诺夫和列宁解释说,马克思主
张贫困是相对的,而非绝对的,并且极力说明发达资本主义社会的统计数据与这种主张是相
一致的。一般说来,这与马克思论述贫困化时的许多观点是一致的,即\textbf{贫困化是一
个相对份额}的问题,当然,这与修正主义的观点并不冲突。问题是,\textbf{考虑到马克
思的逐渐扩大的失业后备军的理论和工会只具有有限能力的观点,并没有理论能够证明贫困
为什么不会变得越来越严重。无论是正统马克思主义者还是修正主义者,都没有提出令人满
意的工资理论。}

\section{价值和分配理论}

俄国修正主义者对劳动价值论进行了严厉批判。杜冈-巴拉诺夫斯基和布尔加科夫都发
现《资本论》第三卷中转形过程中存在的严重问题,他们认为,马克思在那里实际上是把利
润率作为一个\textbf{外生变量},而非\textbf{内生变量}。如布尔加科夫提出的:
\begin{quotation}\textbf{即使经济中的总价格恰好与总价值相一致,也不意味着价
值是由劳动、利润是由剩余价值决定的……}如果不能证明在单个情况下利润由剩余价值构
成,那么通过用总剩余价值除以总资本定义平均利润率就很奇怪了……这完全是以待决之问
题为论据,尽管它是这一理论的精神实质。
\end{quotation}

杜冈-巴拉诺夫斯基认为的马克思价值概念中存在的“内在矛盾”在于,“根据马克思
的观点,价值是对象化的劳动。但是,正如马克思明确地承认的那样,价格不等于劳动价值。
劳动如果不在价格中体现的话,就不可能把自身对象化为任何东西。因此,\textbf{价值不
是对象化的劳动}。”结果,“马克思生活在一个与现实世界没有任何联系的虚幻的世界中。
诸如土地价格之类的真实现象,被描述为想象的产物,而完全虚幻的概念,如在交换关系中
并没有发挥任何作用的交换价值,却被当作经济学的最高智慧。”

然而,在他们自己的价值分析中,也存在一些问题。杜冈-巴拉诺夫斯基的理论精微而
复杂,他利用了许多马克思主义者和非马克思主义者的观点。他区分了三种形式的劳动价值
论,\textbf{他反对的只是把劳动定义为价值实质的“绝对”形式的劳动价值论。}杜冈-
巴拉诺夫斯基\textbf{认可了阿奎那和蒲鲁东的“唯心主义”的劳动价值论,在他们的理论
中,物化劳动成为一种伦理标准和公平价格的基础;}他也认可了李嘉图的“相对”的劳动
价值论,这种理论认为\textbf{劳动是决定价值的两个因素之一(另一个因素是生产过程的
长短)}。他附和了新古典经济学分析中的劳动的负效用和拉斯金、霍布森的\textbf{人本
经济学,把劳动定义为“绝对成本”的基础,因为人是唯一的经济活动的主体。}杜冈-巴
拉诺夫斯基认为,资本家只认识到“金钱成本”或“相对成本”,因而忽视了人作为目的自
身与作为实现其它目的的手段之间的区别。其必然后果就是商品拜物教,即认为人类的属性
由无意识的客体决定的一种意识状态。

杜冈-巴拉诺夫斯基相信,要构建一个令人满意的价值理论,\textbf{客观主义的劳动
价值论必须辅之以主观效用价值论。}既然实际的经济生活有主观的和客观的方面,那么价
值理论也就必须有两个维度。经济行为既包含了(主观的)效用最大化目标,也包含了(客
观的)外部世界的变化。杜冈-巴拉诺夫斯基认为,李嘉图至少已经认识到了这一点,并假
定了最大化行为的存在,只是没有确切地阐述边际效用递减规律。然而,这一规律完善了李
嘉图的价值理论,而不是在效用价值论要求存在由劳动成本提供客观要素的严格意义上与之
相矛盾。杜冈-巴拉诺夫斯基推论说,\textbf{(对每一对商品来说),均衡要求它们的边
际效用的比率等于他们的劳动成本的比率。劳动并不是如马克思所坚持的那样,是价值的实
质,而是大多数商品平均价格最重要的决定因素。}


提出这种类似的综合方案的,还有保守的德国人米尔普福特和英国自由主义经济学的老
前辈马歇尔(是用迥然不同的形式来表述的),近来还有雷弗·约翰森和森岛通夫。对杜冈
-巴拉诺夫斯基来说,这是一条充满意想不到的困难的道路,而且布哈林毫不留情地批判了
他们。布哈林认为,新古典经济学的价值理论,建立在\textbf{社会与个人之间的关系的自
由主义概念}基础之上,这与马克思主义是不一致的。这种概念不可能只为马克思主义增添
新的内容,而不损害马克思主义整体的连贯性。这种观点因新古典式的批评而得以强化,
\textbf{新古典主义者认为单单效用理论就足够了,不需要客观主义者提供的成本理论。效
用既是需求也是供给的基础。}此外,对杜冈-巴拉诺夫斯基产生最重要的影响的奥地利新
古典主义,恰恰是一种主观主义形式的理论,而这种理论与杜冈自己的积累理论发生了严重
的冲突。\textbf{奥地利学派的理论从资本设备对消费品生产做出的贡献上寻找商品价值的
根源。从而积累依赖于消费需求,正如布哈林指出的,这与杜冈-巴拉诺夫斯基分析再生产
时的立场相矛盾。此外,布哈林也可能已经注意到——假如他在这个问题上没有和杜冈一样犯
糊涂的话——新古典主义的需求概念,排除了消费不足的可能性,而这是杜冈-巴拉诺夫斯基
周期性经济危机理论的基础。}

在杜冈-巴拉诺夫斯基对效用理论和劳动成本相统一的探索中,还存在着其他的缺陷。尽
管他认识到不同的资本密集程度意味着劳动价值的比率不可能等于均衡价格的比率,但他并
没有把这种思考纳入劳动价值与边际效用相联系的方程中。此外,他拒绝了马克思本人的价
值概念,但是,又没有提出令人信服的理由去证明为什么还要保留源于它的其他概念。相反,
他坚持的是一种\textbf{定义不清的折中主义},搅乱了在系统地阐述经济理论时要把“客
观主义”和“主观主义”方法区分开来这一真正的问题。

在分配问题上,杜冈-巴拉诺夫斯基接近于马克思而不是新古典主义。他拒绝接受马克思
的剩余价值理论,因为\textbf{利润}还受到生产中使用的\textbf{不变资本量的影响},而
不只是受可变资本的影响——这种判断源自他对利润率下降的批评。但是,杜冈-巴拉诺夫斯
基\textbf{保留了剩余劳动的概念},因为他正确地注意到,这是一个简单的事实,它
是“\textbf{如此的明显,以至于无需证明}”,而且从逻辑上看,剩余价值独立于任何价值
理论。他把分配问题视为解释工人阶级创造的剩余劳动的受益者是谁,以及获得了多大的数
量的问题。杜冈-巴拉诺夫斯基严厉地(也是合理地)\textbf{批评了新古典主义的生产率
  理论和“节欲”利润理论。}同马克思一样,他把地租和利润解释为特定的阶级社会的范畴,
而不是土地和资本的生产率发展的必然(非历史的)结果。与马克思不同,杜冈-巴拉诺夫
斯基的剥削理论具有明显的\textbf{伦理本质:非生产者对剩余劳动的占有是不道德的,}因
为它违背了人类平等这一社会主义的根本原则。在这里,杜冈-巴拉诺夫斯基显示出他受到
康德哲学和马克思主义之前的或“空想”社会主义理论的影响,杜冈认为,这些社会主义理
论在某些方面甚至比马克思本人的社会主义理论更“科学”。

狭义上的杜冈-巴拉诺夫斯基的分配理论并不成熟。他呼吁综合马克思主义和生产率分析,
因为两者都包含真理的成分,并且支持那种认为资本家通常会在讨价还价中获胜的工资理论。
所有这一切,连同他的价值理论都是难以令人满意的折中,并且杜冈对价值理论和分配理论
是相互独立的这种观点的坚持,很难和他对资本主义经济是一个\textbf{高度整合的体系}——一
个部门的变化必然引发广泛的反应——这一事实的强调相调和。然而,今天,人们可能明白,
他也许正在尝试提出后来被\textbf{斯拉法}严格地加以阐述的观点,即\textbf{分配数量
的决定在逻辑上先于商品价格的决定。}

\section{农业经济学}

如果俄国的修正主义者可以宣称自己在经济思想的许多分支上具有原创性的贡献,那么在农
业问题上他们则严重地依赖于\textbf{德国人}。……鉴于在这一领域的突出地位,列宁是
反对修正主义的主角,但是,在这样做时,列宁与修正主义者批评家一样,严重地依赖于德
国的资料,尤其是考茨基的《土地问题》。

德国的修正主义者认为,\textbf{马克思关于大规模农业在经济上具有优越性的信条是
错误的,农民在资本主义发展过程中能够存活下来,}所以德国社会民主党必须做出让步,
以确保农民在政治上对他们的支持。考茨基成功地抵制了纲领上的变化,在《土地问题》中
寻求从理论上对修正主义者的反击。但是,他是通过对迄今为止仍然被正统马克思主义所坚
持的粗糙的观点的修改来达到这一点的,并且承认农业发展与工业发展截然不同。
\textbf{考茨基认识到,农业领域的无产阶级化通常是复杂的,工人保留了少量把他们束缚
于土地上的份地。农民可能通过“劳动过度”和“消费不足”来抵制资本主义农业的侵蚀
(换言之,他们将比产业工人劳动更加努力而消费得更少)。}

列宁运用《土地问题》来反对布尔加科夫对农业经济学的修正主义分析,这种分析只不
过是对德国修正主义者的观点的重新表述。在这样做时,列宁的观点进一步偏离了马克思本
人对原始积累的分析。列宁有关农民家庭而不是个体分化的观点得到进一步的巩固。把这个
观点同“劳动过度”和“消费不足”的机制都能够使农民抵制无产阶级化的观点结合在一起
了。然而,他借助这些因素支持俄国的马克思主义,坚持认为它们具有一般性:俄国的农业
与西欧的几乎没有什么不同。列宁使用\textbf{考茨基反对修正主义的观点}强调他对民粹
主义的批评。而且不像考茨基那样只满足于理论上的反击,列宁随后调整了布尔什维克的农
业政策,以便使修正主义者注意到的农业生产特殊性造成的影响趋于最小化(参见以下第十
一章)。

\section{俄国修正主义的重要意义}

人们严重地低估了俄国修正主义的变种——合法马克思主义的重要性。最常见的错误是,把合
法马克思主义只看作是著名的德国修正主义的分支。……在修正主义者当中,伯恩施坦赞扬
杜冈引入伦理问题,是为马克思主义的唯物主义“冷漠的历史主义注入了生活的气息”,并
把康德而不是黑格尔置于他的社会主义的核心,同时他也批评杜冈-巴拉诺夫斯基在废弃剩
余价值理论上的无力。对正统派而言,卡尔·考茨基详尽地批评了杜冈-巴拉诺夫斯基的主
要著作,认为即使杜冈与德国修正主义者都是错误的(在归根结底的含义上),他仍然是一
位比德国修正主义者更为优秀的理论家。

根据布尔加科夫对农业经济学的批判,列宁似乎修改了他的布尔什维克土地纲领。杜冈-巴
拉诺夫斯基的著作更加重要。他的比例失调论为马克思主义对民粹派的停滞理论进行反驳,
提供了强有力的支持,并且在隔了一代人之后,\textbf{在分析层面支持了苏联经济学家的
观点,即认为通过限制消费增长而不是通过消费导致的增长来加速工业化。}布哈林谴责托
洛茨基、普列奥布拉任斯基和后来斯大林的“\textbf{应用杜冈主义}”,并不是没有道理
(参见以下第十五章)。杜冈-巴拉诺夫斯基的经济分析,对理解俄国革命过程的动态学也
是至关重要的。他在俄国历史分析中强调的那些属性,被托洛茨基作为自己“不断革命论”
的核心,正是这一理论被证明最具先见之明地预测了1917年的事件(参见以上第九章和以下
第十二章)。

所有这一切,很少被人们认识到。俄国的修正主义最多被视为是马克思主义理论的重要批判
者,而没有认识到它对马克思主义自身发展产生的影响。……1905年之后,列宁用资产阶级
转型的\textbf{“普鲁士”模式}对俄国现代化进行的讨论,与司徒卢威的讨论实际上是相同
的。但是,在激进社会主义者的圈子里,司徒卢威是不受欢迎的人,很自然的,当孟什维克
的评论家质疑布尔什维克的政治经济学时,列宁无意向他们提供更多的理由。因为这一点,
不可能从文本上去证明俄国修正主义对正统马克思主义产生的影响。然而,即使这种影响真
的不存在,但主要马克思主义者的理论中至关重要的观点被俄国修正主义者预见到,却是一
个事实。

正是因为合法马克思主义同俄国正统派的决裂,加之他们的思想没有被正统派明确地认
可,人们就会假设他们对西方马克思主义的发展是无足轻重的。这是一个错误,这种错误在
杜冈-巴拉诺夫斯基身上表现得最为明显。他确实同俄国马克思主义断绝了关系,在他不再
是社会主义者时,也没有追随其它合法马克思主义者。他仍然严厉地批判资本主义,而且正
是他的经济理论和他的新康德主义一起,解释了他这样做的原因。正如在第九章中所指出的,
\textbf{他相信“用更多的机器生产更多的机器”与人类价值并不相容。}然而,杜冈-巴
拉诺夫斯基把它作为对资本主义必然本质的表述,他因此强调了在马克思那里可以找到的
\textbf{异化和拜物教的主题},而别的马克思主义者或者贬低或者完全忽视了这一主题。
\textbf{谴责资本主义,不是因为它行将崩溃,而是因为它违背人性。如果资本主义将来会
被取代的话,其逻辑依据必然是伦理方面的,完成这种取代的手段只能是建立在选择的基础
上的行动。}“人类将不会把它视为盲目的、自然的经济力量带来的礼物而接受社会主义。
人类必须为实现这一新的社会秩序而有意识地工作和奋斗。”\textbf{社会主义不再是“科
学的”;它的基础是一种新形式的乌托邦,但是,这种乌托邦在马克思主义那里已有所体现。}

这是一个在许多方面与同时代德国的修正主义相比成果更为丰硕的观点,因为后者总是把社
会主义看作是资产阶级自由主义发展的必然的终点。在20世纪下半叶,与杜冈-巴拉诺夫斯
基类似的思想,成为“西方马克思主义”的核心内容。越来越多的马克思主义者,例如卢卡
奇、葛兰西和法兰克福学派都拒绝接受\textbf{政治经济学具有第一位}的重要性,
而\textbf{坚持上层建筑分析}。社会哲学、认识论和美学来到舞台的中央,对资本主义的批
判建立在这些基础之上。“青年马克思”而不是支配了第二国际的思想的“成熟马克思”,
成为灵感的主要来源。杜冈-巴拉诺夫斯基可以理所当然地宣布他自己开辟了这条道路,即
使人们并没有认识到这一事实。此外,具有讽刺意味的是,在这一点上,杜冈-巴拉诺夫斯
基与列宁分享了共同的立场。尽管是经济分析方面的经典马克思主义者,列宁的意识理论和
政党组织理论将普列汉诺夫马克思主义中的唯意志论因素发挥到了极致。1917年革命后不久,
它开始被视为“列宁主义”的真正本质,并对“西方马克思主义”产生了影响,引起了
对1914年之前正统理解的否定。

\chapter{列宁的政治经济学:1905-1914}

\section{列宁经济思想的分期}
\textbf{1900年以前},原创性并不是列宁政治经济学的特征。正如我们在以上第九章看到的,
列宁政治经济学中显然存在某些新特征,但整个理论框架则是由普列汉诺夫提供的。同
样,\textbf{第一次世界大战期间},列宁形成的经济理论,主要建立在其他人著作的基础之
上,特别是希法亭和布哈林的著作(参见以下第十三章)。这后一阶段,列宁的思想无疑是
最为重要的,因为它为布尔什维克革命提供了理论基础。在这两个阶段之间,列宁创造了新
的和富有想象力的政治经济学,它为理解整个俄国历史发展和革命的马克思主义所面临的问
题提供了新的观点。此外,这一阶段的许多主题,对理解他在生命的最后十年的作为至关重
要,尽管写于1916年的《帝国主义论》的论题,仍然占据着主导地位。

同孟什维克决裂之后,列宁开始了他第二阶段的政治经济学研究。最初,社会民主主义者没
有认识到\textbf{经济问题}是导致他们分裂的根源。在《俄国资本主义的发展》中,列宁得
出了并不被其他理论家认同的结论。\textbf{特别是他坚持的沙皇时代的社会形态中资本主
  义是占主导地位的生产方式的观点,在起草党的纲领时引起一些争议。然而,这一论题只
  被看作是论述问题的着重点上的差异,}当然,列宁不会寻求同普列汉诺夫的策略的彻底决
裂:尽管资本主义在俄国占据主导地位,\textbf{但资本主义关系仍然十分落后},两阶段革
命论的适用性仍然是不容置疑的。事实上,直到1917年初,列宁并没有打算放弃这一“算术
学”。1903-1904年,对他(和其他人)来说,同孟什维克决裂的根本原因仍然是不清晰的。
他的解释集中在先前的“经济派”在党内已经获得稳固的地位,造成了对《怎么办》中作出
概述的党的组织方案的抵制,而这一点曾被所有的火星派接受。

从马克思主义的视角看,无论是列宁本人的立场,还是他对孟什维克坚持的立场的解释,显
然都是难以令人满意的。无论是《怎么办》中的理论内容,还是对机会主义的说明,
都\textbf{缺乏唯物主义的基础},直到1905年,列宁才填补了这一空白。通过对俄国社会民
主党土地纲领考察,可以最好地理解列宁是怎么逐渐做到这一点的。

\section{俄国马克思主义的土地纲领}
一开始,俄国社会民主党认识到,如果能够获得农民的支持,反对沙皇专制的革命将得到极
大的帮助,而且成功的革命将包括农业关系的变化,这会有利于资本主义的进一步发展。然
而,直到19世纪90年代末,土地纲领中的措辞大多仍然是一些一般性的术语,不存在对什么
样的具体的经济措施最能促进俄国农业现代化的分析,也不存在对前资本主义关系为何能够
存活下来的分析。\textbf{党没有提出一个明确的指向},把农民的不满引导到对社会民主党
的支持上。只是随着列宁对农业的分析,这方面的具体政策才开始形成。

列宁的具体政策完成于1899年,主要围绕所谓的“割地”\footnote{割地指俄国1861年农民改革中农
民失去的土地。按照改革的法令,如地主农民占有的份地超过当地规定的最高标准,或者在
保留现有农民份地的情况下地主占有的土地少于该田庄全部可耕地的1/3(草原地区为1/2),
就从1861年2月19日以前地主农民享有的份地中割去多出的部分。份地也可通过农民与地主间
的特别协议而缩减。割地通常是最肥沃和收益最大的地块,或农民最不可缺少的地段(割草
场、牧场等),这就迫使农民在受盘剥的条件下向地主租用割地。改革时,对皇族农民和国
家农民也实行了割地,但割去的部分要小得多。要求归还割地是农民斗争的口号之
一,1903年俄国社会民主工党第二次代表大会曾把它列入党纲。1905年俄国社会民主工党第
三次代表大会提出了没收全部地主土地, 以代替这一要求。}展开的。这一指向除了要
求\textbf{废除地主},设计了其他一些旨在\textbf{创造新的资产阶级秩序}的措施之外,
还包括了一些具体的经济方面的要求,即\textbf{没收地主的割地重新分配给农民}。……消
除旧秩序的残余,主要地可以通过把割地还给农民来实现,因为它会破坏旧秩序的经济基础。
在政治上,这一政策可以满足农民渴望土地的要求,因而能够获得农民对资产阶级民主革命
中无产阶级领导权的支持。

显然,这一纲领存在严重的错误,至少在政治方面是如此。在1905和1917年,农民完全获得
了土地,他们并没有把行动仅限于部分的没收上。考虑到这一点,“cut-off”(收回土地再分配)方案具有极强的人为特征。然而,这种政策,明显地是从列宁的经济分析
中得出的,并且也与其它的社会民主党成员的思想相符合。\textbf{对俄国的马克思主义者
  来说,有些原因使他们很难证明全面夺取土地是合理的。这毕竟是革命的民粹主义者的政
  策,并且普列汉诺夫强调,在反对民粹主义时要坚持正统马克思主义的独特的纯洁性,人
  们广泛认同这一观点。}当时,人们也相信,等同于农场的大规模农业是一种历史的进步。
和这些观点交织在一起的,是说不清道不明的\textbf{对俄国农民的落后和野蛮的恐惧}(参
见以上第八章)。在经济分析层面,列宁指出了农民内部存在的分化,而且通常认为这种分
化限制了农村的阶级斗争。人们相信,农民资产阶级\textbf{从不可能}全力以赴地向土地所
有权发起攻击,他们只热衷于支持那些消除前资本主义剥削的措施。因此,有强有力的理论
证据表明,应当把土地纲领仅限于割地。

1902年开始的农村骚乱,使得社民党土地纲领的实施受到了更为严重的抵制。1905年之后,
党的两翼都抛弃了这一政策。孟什维克和布尔什维克都改变了自己的观点,\textbf{支持全
  面的没收土地的政策(尽管他们的政策截然不同)}。然而,只有布尔什维克提供了对这一
变化的经济基础的说明。1905年以后,列宁的政治思想并不是原创性的;在他之前,许多人
已经提到\textbf{无产阶级与农民之间的结盟}。但是,只有列宁提供了成熟的经济理论支撑
这一策略,以实现资产阶级民主革命。正如他的经济分析构成了割地的土地纲领的基础一样,
他独自发展的政治经济学为政策的变化,再一次提供了理论支撑。相比之下,孟什维克只是
基于纯粹的政治理由改变了他们的政策。从历史唯物主义的观点来看,孟什维克的观点是肤
浅的,尽管他们自称为正统,布尔什维克才更接近于对经典马克思主义方法的坚持。

列宁在19世纪90年代对资本主义发展所做的分析显然需要修改,而与其他马克思主义者相比,
他可能是最适合做这件事的人。列宁的优势不仅在于他广博的农业领域的知识,而且还在于
他敏锐地意识到\textbf{俄国自由主义和修正主义的本质}。一开始,列宁在他的著作中,就
对自由主义表露出极其罕见的批判立场。从情感方面来说,这源自他痛苦的个人经历,但是,
他的立场又被如下认识强化了:\textbf{他认为俄国革命所需要的激进的本质,很可能会远
  离资产阶级的支持。}这在19世纪90年代引起了他与普列汉诺夫及其支持者的摩擦,这一摩
擦最终因列宁对自由主义者的贬损变的温和而得以解决。但是,没有确切的证据表明,列宁
真正改变了自己的观点。因此,在同孟什维克决裂之后,在修改他自己关于革命的经济理论
时,列宁不再有所顾虑,在赋予自由派资产阶级以实际性作用方面,他受到的压力也更小
了。

此外,列宁作为著名的农业方面专家,\textbf{反对修正主义认为的马克思主义的“运动规
  律”不适用于农业部门的观点}的任务,也落在了他的肩上。他在这样做的时候,没有做出
任何明显的让步,而是像考茨基那样,提出了说明了农业部门的小生产者比工业部门的小生
产者更容易生存下来的理由。从而,列宁对能够从制度上加强资本主义发展的手段的重要性
特别敏锐(参见以上第十章)。

这两件事被证明是有联系的。第一件事促使列宁思考\textbf{如何创造一个没有资产阶级盟
  友的资产阶级秩序}。社会民主党手中掌握的革命性的国家权力,可以通过\textbf{土地的
  国有化}为第二个问题提供解决办法。但是,这两个因素只是整个俄国资本主义经济发展的
新的和独特的视角的一部分,俄国资本主义的发展被认为包含着这两种在部分上互补、但在
本质上却存在着深刻冲突的趋势。

\section{布尔什维克的政治经济学}

1905年,农民的激进主义证实了列宁的疑虑,即他的政治经济学存在严重的缺陷。他逐渐认
识到,\textbf{他先前的著作不仅低估了农业中封建残余的力量,而且对它们的经济基础的
  说明也存在问题。}他现在主张,\textbf{不能从割地的角度,而应当从地主经济本质的角
  度理解这一经济基础。}因此,彻底的资产阶级革命不能只限于对地产进行外科手术式的打
击,而是需要\textbf{对整个地主阶级的剥夺}。只有采取这种极端的措施,农业资本主义才
能得到全面的发展,而且社会民主党只有毫无保留地支持农民夺取土地,农村小生产者才能
成为无产阶级获取领导权时的同盟。

但是,列宁并没有把这个问题看作是富于活力的农业资本主义与垂死挣扎的封建主义之间的
简单冲突,事情更为复杂。如他在《俄国资本主义的发展》中表明的,无论是地主经济还是
农民经济,都正在\textbf{日益资本主义化},在这一点上,他并没有推翻自己的观点。相反,
他主张地主的现代化程度被高估了,它更重要的是代表了资产阶级转型中的一种\textbf{特
  殊类型},这一点在他1899年的著作中并没有被揭示出来,只是在部分程度上有所说明。地
主的资本主义化只是“自上而下的重建”的一个特征,为了维护自己的生存,沙皇和占主导
地位的地主阶级与资产阶级秩序的某些因素相妥协。\textbf{旧体制的重要特征被保留了下
  来,因为正是旧体制自己的力量领导了这一变迁过程。}特别是因为,从经济层面上
看,\textbf{农民与地主之间的半封建关系主要决定于后者},因而大规模的资本主义发展受
到限制因此,广泛而又深刻的农业的资本主义化,要求农民革命反对沙皇政府推动的现代化,
根据列宁的观点,农民革命的客观的经济内容,可以被看作是努力建设更为完善的资产阶级
秩序的合力之一。

1905年的革命,使列宁能够对城市资产阶级和社会民主党在这两种对抗形式的资产阶级转型
中发挥的作用进行说明。十月党人和士官生的妥协行为,证实了列宁此前对自由主义分子的
疑虑,说明资产阶级自身是\textbf{“自上而下”而非“自下而上”革命过程}中的行动
者:\textbf{资产阶级寻求的是与正在进行现代化的沙皇政府的妥协,而非推翻它们。资产
  阶级对人民革命的支持,只不过是增强它在既定的统治阶层内部拥有更多权力的一种手
  段。}列宁认为,无产阶级的革命力量,凸显了资产阶级的保守性。农民革命强化了这一点,
因为通过土地购买,资产阶级已经“地主化”了,这如同地主阶级变成了工业资本家一样。
所有这一切都表明\textbf{资产阶级成为激进革命的敌人},证明\textbf{1905年革命缓和了
  中产阶级与封建上层建筑之间的摩擦。}列宁断言,如果建立了一个纯粹的资产阶级秩序,
它将包括\textbf{反对资产阶级的资产阶级革命}。

这意味着列宁同普列汉诺夫的革命“算术学”的\textbf{决裂}(参见以上第八章)。列宁认
为,指明需要经过两个阶段革命的更为抽象的“代数学”仍然是正确的;但是,如果要实现
正统派最初的目标,\textbf{无产阶级和资产阶级的结盟是不可能的}。确实,孟什维克对普
列汉诺夫的策略的坚持是反动的。它意味着\textbf{放弃}无产阶级的领导权,\textbf{屈
  从}于资产阶级从上层统治阶级那里获得让步的目标,从而\textbf{捍卫}了沙皇统治的重
建。在孟什维克演化的过程中,列宁日益警觉地发现这一逻辑本身在孟什维克演化中产生的
结果,他不愿意为了维护党的统一而进行妥协的意向进一步加强。它不再只是一个反对改良
主义的问题,或者是主张坚持某种特定类型的党组织的问题,列宁现在为自己的政治立场找
到了经济基础(参见本章第6节)。客观地看,而且不考虑孟什维克的意图,他们成
了\textbf{戏剧中的演员},这一戏剧把工人阶级整合到再造后的旧体制中。因此,把列宁在
社会民主党内部的斗争看作只是争夺个人领导权,是肤浅的。1905年革命之后的布尔什维主
义,是建立在对资本主义发展和资产阶级革命复杂本质的唯物主义分析基础之上的。

在列宁看来,俄国社会民主党面临着一个无法回避的选择。它要么在影响“普鲁士”式道路
的资本主义发展中扮演一个卑微的角色,要么坚持正统派的精髓,在彻底推翻旧的沙皇体制
(尽管在进行现代化)的革命中掌握\textbf{对工人阶级和农民的领导权}。\textbf{每一个
  策略都包含着对资产阶级革命的支持,但这是两种不同类型的资产阶级革命,它们之间具
  有对抗的性质。}列宁认为,普列汉诺夫认为这两种资产阶级革命构成一个正在展示次要的
历史变化的统一体,是错误的。它们可以是完全不同类型的革命,这取决于哪个阶级占据支
配地位,取决于资本主义发展采取何种道路。

\section{俄国历史和“普鲁士道路”}

列宁认为,现代俄国历史中的主导力量是“普鲁士道路”。这是一个自上而下的重建过程,
在这一过程中,旧体制把资本主义体制的因素融入自身中,以确保自己能够在充满敌意的国
际环境中生存下来。在彼得大帝改革之后,这条道路前行的一个重大步骤,就是1861年的农
奴解放。列宁意识到,\textbf{作为农业资本原始积累的一个因素,俄国的农奴解放与马克
  思集中研究的英国资本原始积累的形式具有完全相反的特征。}在俄国农奴解放中,农业生
产者没有与土地分离,他们仍然\textbf{被束缚在土地上}。加之\textbf{对地主阶级进行的
  补贴},农奴解放并没有触动传统的剥削关系,\textbf{只是推动了大地产向资本主义组织
  形式的演变}。然而,\textbf{农奴解放也为农业经济中资本主义发展提供了基础}。但是,
由于它要服从“普鲁士道路”的要求,农业资本主义的活力受到限制。列宁认为,后来的事
件,可以被看作是\textbf{以农民为基础的资本主义推翻建立在大地主转型基础之上的资本
  主义的斗争。}

 在列宁看来,1905年与1907年之间的农民革命,是这种对抗的最明显的表现。革命虽然因
镇压而失败了,但是,它进一步\textbf{中断了“普鲁士道路”的演进过程}。政府当局,特
别是政府首脑斯托雷平认识到,长期的生存依赖于获得强大的社会支持。因此,\textbf{农
  村和城市的资产阶级获得了一定的特权。保护村社的法律被撤消了,许多措施被用来激励
  富农巩固他们的土地。农民大众付出了代价,他们的贫困化继续支持着大地主向更为资本
  主义化的形式缓慢地过渡。当大量农民破产时,这种过渡可能进行得相对平稳,但是一个
  更广泛的、具有实质意义的农村有产阶级——农民资产阶级——才构成社会安全的长久的基础。}

列宁认为,沙皇专制也把\textbf{城市资产阶级}纳入到它的政治权力的结构中了,尽管它们
的地位比较低下。现有的体制改变了它的绝对专制主义的特色,并且,新的议会(杜马)为
官僚阶层、地主阶级和资产阶级之间达成未来的妥协提供了一个制度平台。这些让步并不能
让所有的中产阶级都感到满意,因为卡德茨(立宪民主党)想要取得主导地位。然而,由于
不愿意支持人民革命,他们被迫接受特权的重新分配,并且试图通过宪政手段在未来获得更
多的利益。列宁相信,这势必引起\textbf{他们与官僚和地主之间的摩擦},社会民主党为了
自己的利益将会\textbf{充分利用这种摩擦},但是,如果(像孟什维克那样)认为的这是资
产阶级民主革命得以圆满完成的根基,那就犯了根本性的错误。

列宁认为,虽然无产阶级从流产的革命中获益不多,但是现有政权有可能同意进行一些政治
和经济改革,以努力把无产阶级吸纳进来,这就如同西欧曾经发生的事情那样。大多数发达
资本主义国家已经走上了“\textbf{普鲁士}”式的现代化发展道路,而且也确实巩固了这种
形式的资产阶级革命。俄国在这一过程也有可能成功。但是,布尔什维克的任务是阻止它的
实现,用更富有活力的“\textbf{美国式道路}”取代它,通过革命建立“\textbf{工农民主
  专政}”。

\section{布尔什维克的策略和“美国式道路”}

在《俄国资本主义的发展》中,列宁集中关注了“商品生产的逻辑”对农业经济的影响,他
在新政治经济学中并没有抛弃这一观点;事实上,他把它提高到更为突出的位置。他通
过19世纪90年代的分析,作出三个重要方面的修改来达到这一点的。首先,正是地主阶级,
或者说更广泛的“普鲁士道路”才是关键的因素,它们现在被视为农业经济现代化的限制因
素,而不只是1861年解放农奴的特定形式。其次,进一步提高了如下认识,资本主义的发展
显著地受到\textbf{实践中的土地占有形式}的影响,而且\textbf{土地的国有化是一种最有
  利于资本主义发展的土地占有形式}。最后,列宁主张,通过“\textbf{工农民主专
  政}”的革命策略,农业资本主义可以获得迅速发展。

\textbf{自1905年开始,列宁倾向于认为,地主的生产活动主要是封建性质的,而不是资本
  主义性质的。}尽管拥有数量可观的土地,但地主的生产活动并不能代表\textbf{“大规
  模”农业生产}。地主的生产过程,通常是农民在传统的方式下使用他们自己的生产资料进
行的小规模农业生产的加总,而且这种方式还被一套半封建关系笼罩着。尽管受到国家补贴
的支持,但这种生产活动只是\textbf{缓慢地接近于资本主义形式},而且是通过同时造成农
民大众的贫困来实现的。这和列宁19世纪90年代的观点是相反的。他不再认为,农业经济中
的资本主义和地主经济中的资本主义一样,可以获得很好的发展。相反,前者受到后者的显
著地影响。

列宁进而主张,稳固地建立以农民为基础的资本主义,要求\textbf{剥夺地主}。这种剥夺,
将结束封建剥削,并为生产资料的发展提供更多的资源。正如列宁认识到的,最初可能会产
生一些负面效应,因为同落后的地主一起将会失去一部分资本主义化程度较高的地主,而且
农村无产阶级的数量也可能会下降。但是,这些特征将被他描述的“美国式道路”全面的合
理性抵消,\textbf{“美国式道路”是指资本主义从占有大量土地的自由农民阶级发展而来。
  所以,任何的延迟,只不过是后退了一小步,却前进了一大步。}

列宁认为,这其中的包含着比仅仅从封建关系中解脱出来更多的内容。“\textbf{普鲁士道
  路}”不仅带来了,而且本身也要求大量农民维持受压迫状态,在这种状态下,\textbf{农
  民的普遍的落后和野蛮被保持下来}。与此\textbf{相对},通过没收增加的可用的资源,
扩大了农民的眼界,需求将会增加,市场化程度将会提高,农村与城市文明的差别将会减小,
对工业品的需求扩大,农业部门的规模相对缩小。总之,通过创造出一种类似于北美
的\textbf{自由的、广泛的和迅速发展的资本主义具有的良性循环},将会逐渐削弱“农村生
活的愚昧状态”。

在同修正主义的论战中,列宁认识到,在欧洲发达的资本主义地区,有时候会同时存在相对
落后的农民,农民通过“\textbf{消费不足}”和“\textbf{过度劳作}”减缓自身融入资本
主义关系的速度。因此,他倾向于通过“普鲁士模式”在欧洲现代化中的主导地位来解释这
一点。但是,他的著作也显示出,他越来越强调土地国有化是俄国摆脱传统造成的一切障碍
的一种手段。\textbf{列宁并没有把土地国有化视为一种社会主义措施,因为与之相伴的不
  是生产的社会化,}尽管它确实提高了商品交换和流通。土地租赁建立在与马克思《资本论》
中概括的级差地租理论相一致的商业原则的基础之上。

同马克思和考茨基一样,列宁宣称土地国有化的主要优势在于它消除了绝对地租。在马
克思经济学中,\textbf{与商品生产中各种形式的级差地租不同,绝对地租要求土地私人所
有权的存在,而这种所有权也可以通过土地国有化来消除。}这是有益的,因为绝对地租剥
夺了农业资本家的资源,因而阻碍了积累。然而,\textbf{马克思的理论是有缺陷的。马克
思所说绝对地租,实际上不过是级差地租的一种特殊形式,而且马克思对绝对地租大小的解
释建立在错误的观点之上。}因而,列宁对马克思理论的应用是缺乏充分的根据的。但是,
这并没有损害列宁观点的实质。像马克思一样,他用其他一些思考充实了他的分析。土地国
家所有提供了一种灵活性,它便于在技术进步下经济规模变化时投入的重新组合。它也使得
不同的但相互作用的生产过程的有效配置与公共资源的适当管理变得更加容易。最后,公共
主管当局作为(级差)地租的接受者,拥有了一种为所需要的投资提供资金的手段。

列宁对土地国有化的支持,可以被看作是对马克思分析原始积累时提出的“\textbf{清
扫领地}”这一“古典”解决方法进行优化的一次尝试。在列宁的整体方案中,这完全是行
得通的。在马克思看来,英国农业中三个阶级构成的阶级结构,是最适合于资本渗透的历史
案例。列宁认识到这一点,并且寻求一种更彻底的“清扫领地”的方式,\textbf{他正确地
认识到,农业越不发达,这种方法就越合适。}

但是,合适的程度并不是这里涉及的唯一问题。除此之外,还存在如何实现土地国有化的问
题。它不可能遭到无产阶级的反对。但是,农民是怎么样看的呢?他们会认为所有土地都国
有化符合他们的利益吗?列宁对此深表怀疑,并且\textbf{同意没收地主的土地可能是农业
  革命可以推至的极限}。然而,对农民政治代表所能接受的方案的考察,使他相
信\textbf{有些}农村地区支持全面的土地国有化。当然这更符合马克思主义者的愿景,
即\textbf{上升的阶级要求采取的措施与进步发展客观要求的措施相一致。}总之,这些因素,
是列宁把革命的民粹主义解释为资产阶级意识形态的基础。正如以上第九章显示的,他从来
没有把民粹主义思想看作是完全的乌托邦,更不用说是反动的。但是,\textbf{1905年之
  后,}列宁对民粹主义思想有了更加清晰的认识,同孟什维主义相比,\textbf{民粹主义持
  有的是一种对实现“美国式道路”资本主义发展历史要求的主观想象的观点。}

列宁设想通过无产阶级和农民的结盟实现土地的国有化,这种结盟把城市的无产阶级革命和
农民的土地革命结合在一起。这两支力量都是需要的农民的作用在于以一种有效的、“平
民”的方式摧毁农村旧体制的经济基础。无产阶级则会削弱城市中心并全面地领导革命力量,
农民很难做到这一点。\textbf{资产阶级在最好的意义上是袖手旁观,在最坏的意义上则是
  被“民主专政”镇压的反革命力量},在马克思主义的意义上,这一术语就是\textbf{不受
  法律约束的阶级统治}。\textbf{“民主专政”以苏维埃形式组织起来,它的作用是确保革
  命成功,镇压反对者,建立未来法律上的平等和彻底民主的基础,实施限制剥削的措施,
  如果有可能的话,实行土地国有化。}

但是,在列宁看来,专政应该仅限于\textbf{民主革命要求的范围之内}。因为有资产阶级成
员,所有的措施都同资产阶级秩序的继续这一目标相适应。\textbf{客观条件妨碍了社会主
  义革命的完成,而且农民从一开始也会制约它的实现。这是两种力量的专政,而不是无产
  阶级的专政。}革命成功后,将通过召开立宪会议使情况稳定下来,在立宪会议中,资产阶
级统治将在\textbf{非特定的意义上}出现。从那时开始,无产阶级的任务就是\textbf{作为
  社会主义力量团结起来},民主专政下实施的措施将最大可能地实现这一点。从而,普列汉
诺夫的两阶段革命论保留了下来,但是,是以一种新颖的形式,而且是以一种令“俄国马克
思主义之父”不能信服的方式保留下来。

\section{列宁主义的力量}

正如我们将看到的那样,\textbf{普列汉诺夫保持怀疑是明智的},尽管他提出的用以捍卫孟
什维克主义的观点,在实际问题面前通常都表现得软弱无力。然而,就当时的情况来说,必
须承认的是,列宁1905-1914年之间形成的理论代表了俄国马克思主义真正的进步。在这一时
期,\textbf{列宁提出的政治经济学与革命问题存在着密切联系},这种品质是社会民主党一
直以来所十分缺乏的。无论是列宁还是其它的人的\textbf{有关19世纪90年代的经济理论主
  要集中关注农业资本主义的发展问题,因而无法与普列汉诺夫理论体系中的政治结构相吻
  合,普列汉诺夫的理论体系专注于和城市资产阶级结盟的问题。}正统派依据马克思著作采
取的这一立场,试图表明俄国正在发展的资本主义与实现资产阶级民主革命的逻辑没有联系。
通过对区分不同类型资本主义新观点的阐述,\textbf{列宁填补了这一裂缝。他把资本主义
  发展的不同形式与资产阶级民主革命的多样性相联系,并且正确地指出,普列汉诺夫只有
  俄国革命的“抽象”概念。}

有充分的理由认为,\textbf{古典马克思主义是列宁主义的核心。}从方法论层面看,列宁比
孟什维克更接近历史唯物主义的奠基人。当然,列宁的著作也包含了一些独特的因素,尤其
是他强调自发的无产阶级意识和社会民主党的意识之间的区别,以及与这个问题相联系的合
适的党组织问题。但是,1905年之后,列宁提出的政治经济学为这些思想的存在提供了从经
济基础层面的解释,从而保证了这些观点和马克思主义分析视角之间的一致性。布朗基主
义——雅各宾倾向和军事化组织是适当的,因为两种不同的资本主义发展道路在争夺统治权,
取代普鲁士道路的关键是捣毁并夺取国家权力机器。因而,在这方面,孟什维克对布尔什维克
的批评是不中肯的。只有当第一阶段革命成功,社会主义被提上议事日程时,这种批评才是
合适的。

令人惊讶的是,甚至是列宁著作中存在的某些明显的混乱,按照他1905年之后的理论理解,
也呈现出合理性的特征。例如,在以上第七章可以看到,\textbf{列宁对沙皇俄国特征的说
  明是不稳定的}。他有时认为它是亚细亚式的,有时又认为它是封建式的,甚至偶尔称之为
资产阶级式的。这种不稳定性,在列宁晚期的著作反复出现。但是,现在讲得通
了,\textbf{因为列宁的理论强调沙皇专制的转型——因而也是模糊的——本质:“普鲁士道
  路”恰恰是把绝对专制主义的、封建的和资产阶级的特征结合在一起的,}当这些特征达到
新的统一的时,沙皇俄国也就走到了尽头。

支持列宁“普鲁士道路”概念的理由有很多。把这一概念应用于沙皇俄国的历史,明显地克
服了普列汉诺夫与此不同的观点中存在的疑难,普列汉诺夫的观点集中关注亚细亚形式与欧
洲形式的融合(参见以上第八章)。此外,\textbf{它也可以用来反击普列汉诺夫对土地国
  有化的批评,这种批评认为,土地国有化为恢复亚细亚特征奠定了基础。此外,列宁揭示
  了普鲁士式的现代化道路的无效率,这由地产的危险境况所证明。}俄国的贵族与东部德国
的容克相比更缺乏向农业资本主义转变的能力。这既反映了他们为沙皇政府服务的取向,也
反映出大土地所有者数量的巨大。列宁十分正确地指出,\textbf{大规模本身并不意味着效
  率,它只是资本主义关系中的次要因素。}尽管落后,但农业经济\textbf{明显地具有活
  力},而且地主把土地租给农民,通常比直接把它作为商业企业的一部分来管
理,\textbf{更为有利可图}。然而,\textbf{税收落到了农业中的小生产者身上},其中的
许多收益被浪费在直接或间接的对官僚化土地贵族的支持上。列宁也可以求助于马克思
对1848年革命的论述,来反击孟什维克对“民主专政”的批判。尽管普列汉诺夫也深受马克
思这方面见解的影响(参见以上第七章和第八章),但是,出于对马克思晚期著作的思考,
列宁的解释改变甚微。

关于“美国式道路”的政治经济学,列宁1905年后的分析与他早期著作中的分析相比,在与
关键问题的联系上,不如早期著作中的分析联系那么紧密,这的确是事实,原因在
于\textbf{规模经济的缺乏成为产生大规模资本主义农业的障碍}(参见以上第九章)。但是,
列宁后来对土地国有化的强调是完全适当的,而且在部分程度上缓和了这一错
误。\textbf{如果}他的方案实现的话,它将成为一种\textbf{富于效率的、相对人道的消除
  传统农民的手段}。不仅私有产权的缺乏削弱了资本渗透的阻力,而且对级差地租的控制也
可以被用来控制劳动力向城市的流动。此外,国家收益的增加挤出了地主的消费,否则的话
它就会阻碍积累。因此,尽管技术条件将会保留小农,但是他们的数目将会锐减,那些留存
下来的小农将比在私人地主那里工作更富裕。\textbf{马克思的绝对地租理论是错误的——这
  一点是由孟什维克的农业专家马斯洛夫提出的,这一事实并不会削弱列宁方案中的合理内
  核。}并且在批评孟什维克的土地归市有或市营的替代方案时,列宁指出他的批评者的主要
局限,这种替代方案不仅是一种历史上人为的,而且也缺乏经济基础。

更一般地说,撇开俄国情况的具体特点不论,列宁对不同类型资产阶级转型的明确区分,
预见到现代史学的一个重要主题。\textbf{人们越来越认识到“普鲁士”道路对处于上升期
的资本主义是至关重要的,它以一种变化了的形式保存下来的前资本主义因素,有力地影响
了20世纪的政治学,这表明“现代化付出的代价至少和革命的代价一样大,很可能还要高昂
得多”。}甚至是18世纪晚期法国的资产阶级革命——马克思认为的资产阶级革命的典范——的
性质也是模糊不清的。迄今为止,它并不像人们认为的那样能够很好地与经典(有些人可能
会说庸俗)马克思主义相吻合。此外,\textbf{列宁对在实现理想化的资产阶级秩序时资产
阶级的反革命本质的强调,往往被认为是一种典型的情况,而非特殊的例外。}在此意义上,
正是列宁的著作而不是马克思或孟什维克的著作,更具有一般性。

但是,列宁并没有由此推断马克思主义作为一种整体的理论是不够格的,他只是
\textbf{断言激进的资产阶级革命必须由非资产阶级来推进。在这里存在的不只是语义上的
冲突,这一主张体现的是一种真正的矛盾。}因为它源自列宁的政治经济学,所以在这种形
式的列宁主义中存在着深刻的理论问题。

\section{列宁政治经济学中的矛盾和难题}
列宁1905-1914年的政治经济学源自\textbf{有悖常情的逻辑}。农民政治学为列宁相信的封
建关系先前被严重低估的观点提供了主要的证据。\textbf{上层建筑中发生的事件引起了对
  经济基础的评价的变化。}列宁不是借助于独立地重新分析农业经济的本质,来重新表述他
加以改进的经济学。他也没有试图准确地解释他自己的错误的根源,这些错误是他
在19世纪90年代得出的与现在不同的结论的基础。相反,\textbf{地主只是被重新划分为主
  要是封建性质的。}这个过程被他早期著作的本质所掩盖。在把资本主义等同于它出现的过
程本身时,列宁能够求助于大量的指标(参见以上第九章),每一个指标的重要性都被一个
事实掩盖起来,即\textbf{没有提供有关这些指标重要性的等级序列。列宁的马克思主义因
  而接近于一个对证伪具有免疫力的循环体系。}

同样类型的“逻辑”,在列宁的政治经济学中也是明显的。继马克思和普列汉诺夫之后,列
宁认为农民是无产阶级和资产阶级因素构成的混合物,\textbf{而不是一个自成一类的范畴}。
这为解释农民的行为和信念提供了巨大的灵活性。列宁充分利用了马克思主义的这一一般特
征,\textbf{把民粹主义解释为激进的资产阶级意识形态,并且把土地革命视为资本主义发
  展“美国式的道路”的体现。因为不存在可用的标准去明确地质疑这些解释,因而列宁的
  观点受到了保护,避免受到孟什维克的批判。}

但是,列宁有关农民的立场存在一个缺陷,这个缺陷可以从不同的角度来识别,他对原始积
累的分析最为特殊。正如我们在本章第5节中看到的,\textbf{列宁支持土地国有化政策,是
  因为它使得小生产者不再能控制他们的生产资料,而使资本主义获得最大程度的发展。}然
而,列宁认识到,\textbf{农民可能只是没收地主的土地,并把它们作为自己的财产进行重
  新分配。尽管这样做不如国有化有益},但列宁还是坚持认为,这种做法也具有高度的进步
性,因为它破坏了地主财产;在他看来,这是关键问题。他指出:“农民从地主那里得到
的……土地越多,……资本主义发展的就越迅速”,以此来支持他的观点。在1861年农奴解
放的背景下,这是可以理解的,但是,一般说来,这明显会让人生疑,并且列宁的土地国有
化的主张又突出了这一点。孟什维克对此也不得要领。但他们指责列宁已经接近于民粹主义
的革命策略而这只有在他们随后能够坚持民粹主义代表了一种切实可行的观点,并坚持列宁
相信的简单的没收可以促使资本主义产生是错误的时候,这一观点才能产生极大的力量。但
是,这一点孟什维克做不到。和列宁一样,他们也相信“商品经济的逻辑”。因而列宁在两
个方面都有问题。\textbf{无论是完全剥夺地主土地所有制,还是作为唯一的所有者,农民
  都将建立有利于原始积累的条件。}

民粹主义思想的历史内涵可以交由历史事件去说明。\textbf{那些村社势力最强大的地区,
  也是1917年(包括1905年)农民革命最有力的地区。资产阶级因素完全被淹没
  了};那些巩固了他们的土地,或者完全同村社相分离的富农,又被迫成为村社的成员,而
且他们的土地连同地主的土地都被重新分配。\textbf{用马克思的术语看,土地革命是反动
  的:它重新确立了传统制度(列宁分析中的明确的封建残余)的重要性,恢复了它们的活
  力。}孟什维克对农民激进主义的怀疑,被证明是有充分根据的,但这并不能削弱如下事实:
他们无法用马克思主义的术语从理论上证明自己的观点。

这反过来显示了列宁主义中存在的另外两个问题。它表明与实现列宁的目标有关的农业中的
力量是\textbf{富农},也表明合适的革命策略是\textbf{同农民资产阶级结盟}。而且,这
样的联盟,势必引起贫农和中农的反对,他们是恢复村社的主要受益者。但是,这将意味着
两种阶级斗争——\textbf{反封建主义的和反资产阶级的——不可能同时进行};在农村,它们将
是相继进行的。然而,列宁与所有的俄国马克思主义者一样,\textbf{固执地认为它们必须
  同时进行。}而且理由充分:只有这种立场与把农民视为小资产阶级相一致(因此大量的农
民注定要无产阶级化),而且只有以此为基础,社会民主党才能反击民粹派的指责,指出他
们——客观地说——是资产阶级力量。

此外,有关1905年和1907年革命本质的经验证据表明,\textbf{“普鲁士式道路”和“美国
  式道路”并不像列宁所认为的那样,是对抗性的。建立后者的一个前提条件是存在一个萌
  芽状态的资产阶级,他们的力量足够强大,以至于能够主导土地革命的进程。}因此,尽管
视斯托雷平与列宁为竞争对手是明智的,但他们之间仍然有\textbf{共同之处}。为了实现自
己的目标,列宁需要斯托雷平在大的范围内实现他自己的目标,结果是村社的关系彻底被打
碎,\textbf{农民资产阶级}更加稳固地成长起来。

因此,\textbf{列宁的马克思主义要求普鲁士道路获得一定程度的成功。}但正是对这种可能
性的认识——列宁超越了这一点,设想了它的完全成功——为任何一种马克思主义的政治经济学
提出了深刻的问题。它说明“矛盾”并不一定必然发挥赋予它们的力量,因为它们可能以皮
特·司徒卢威所表明的方式得以“\textbf{缓解}”。因此,1905年之后,列宁自己的分析承
认了\textbf{修正主义核心观点的合理性}(参见以上第十章)。这不是他这样做的唯一的一
点。在《怎么办》中,修正主义被视为是对革命的马克思主义的真正威胁,因为它表达了对
工人阶级自身的“自发”意识的崇拜。而且在后期的著作中,列宁反复提到资本主义社会的
平衡和整合机制。

在这里,对马克思主义来说还存在另外一个问题。\textbf{“普鲁士式道路”的核心是封建
  贵族转变为依据资本主义经济原则行事的地主阶级。}那么,通过什么标准来定义它的阶级
利益呢?是通过旧模式的结构,还是新模式的要求?是通过一些适合于描述转型过程中的转
变时刻的变量的加权平均值来定义,还是通过转型自身的本质属性来定义?类似
地,\textbf{“普鲁士式道路”被认为包括了资产阶级融入到现代化的旧制度中。}正如列宁
注意到的,对资产阶级自身而言,这是一个次优的解决方法,对人民激进主义而言,它可以
勉强被接受。但是,如果这是事实,我们就有了\textbf{阶级利益和阶级行动的区别}。不仅
经济结构无法清楚地界定阶级利益,现在对这个问题来说又加入了一个新认识,一个被结构
性地定义了的阶级利益,可能并不是阶级行动的主要基础。

面对这种情况,列宁得出的明显的结论就是:\textbf{如果一个激进的资产阶级民主革命想
  要实现,它必然是通过除资产阶级以外的其它阶级实现的。并且列宁认识到,这可能包括
  无产阶级对资产阶级自身的反对。}但他不相信这提出了一个矛盾。\textbf{革命是要建立
  资产阶级秩序,而不是资产阶级统治,更不要说是建立特定一代的资产阶级个人的统治。
  换言之,资产阶级革命是依据革命的结果而不是革命的主角来思考的。}然而,\textbf{列
  宁对资产阶级的反革命行动可能导致革命偏离轨道视而不见},这不是因为它可能成功,而
恰恰是因为它将会失败。在这种情形下,成功的革命将包括剥夺资产阶级和无产阶级控制工
业。在工人阶级掌权的情况下,通过什么样的手段维持城市经济中资产阶级的组织和配置原
则呢?在城市地区,农民不可能把革命限制在资产阶级阶段。在长期看,他们——在马克思
的“\textbf{麻袋里的马铃薯}”(《马克思恩格斯文集》第2卷,第566-567页,人民出版
社2009年12月)的比喻意义上——也不可能对反对他们的有组织的城市力量进行更多的抵抗。
因此,同普列汉诺夫的政治经济学一样,列宁的政治经济学同样\textbf{未能克服任何一种
  在非社会主义革命中赋予无产阶级革命的领导权的理论中存在的内在矛盾}(参见以上第八
章)。

在第一次世界大战之前,在社会民主党中,只有托洛茨基认识到了这一问题。它成为“不断
革命论”的支点,从而资产阶级民主革命将\textbf{嵌入}于社会主义革命中。1917年,列宁
对俄国问题的分析采取了同样的立场。他得出结论的思路与托洛茨基不同,但是考察后者的
立场有助于理解前者的实质。所以,在以下第十三章转向列宁的政治经济学之前,我们将首
先讨论托洛茨基著作中“不断革命论”的概念及其理论基础,考察他的不平衡和综合发展政
治经济学。



\chapter{托洛茨基论不平衡和综合发展}

\section{引言}

在《资本论》第一卷序言中,马克思写道:“\textbf{工业较发达的国家向工业较不发达的
  国家所显示的,只是后者未来的景象}”。\pagescite[][8]{capital} 正像我们在以上第
八章和第十一章看到的那样,普列汉诺夫和列宁都坚持这一观点。他们的经济学遵循《资本
论》的结构,\textbf{集中从商品生产关系的视角考察资本主义的发展},而且他们的政治策
略也都与\textbf{加速俄国的西方化进程}相一致。\textbf{列夫·托洛茨基}与他们不同,
他\textbf{否定了马克思的观点}。他系统地阐述了自己的政治经济学,这种政治经济学使他
能够比其他任何一位理论家都更好地理解沙皇俄国现代化的结构和矛盾,理解俄国革命进程
的本质。\textbf{托洛茨基把最早出现在民粹主义中的观点和马克思主义的概念整合在一起,
  证明无论是过去还是未来,俄国都不可能走西方发达国家曾经走过的道路。}

由于托洛茨基1917年后的立场与他早期的观点密切相关,因此,我们也有必要考察20世纪20
年代他的理论从俄国向所有资本主义落后国家传播的情况。这为理解斯大林主义在前苏联的
经济论战的获胜,提供了最可靠的基础。关于斯大林主义,则将在以下第十五章讨论。

\section{俄国不断革命的政治学}

在俄国马克思主义中,托洛茨基的\textbf{《总结与展望》}代表了迄今为止发现
的\textbf{最为激进的革命社会主义的宣言。}他1905年到1907年革命失败期间在狱中完成的
这部著作认为,只要采取社会主义革命的形式,进一步的群众暴动将会成功。\textbf{除非
  嵌入}于无产阶级专政中,否则民主革命是不可能取得胜利的。托洛茨基指出,无论是孟什
维克还是布尔什维克的“两阶段”革命论,都对阶级动力作了错误的说明,\textbf{阶级动
  力将会因任何激进力量重新恢复活动而被释放出来。}无产阶级领导权必定是永久性的,而
不是像正统派认为的那样,只限于击溃沙皇专制。\textbf{落后的俄国将不会、也不能经历
  一个资产阶级共和国时期。}

托洛茨基并没有否认俄国面临的问题是民主革命问题。他敏锐地意识到俄国落后的现状,认
为\textbf{在俄国不存在社会主义革命成功的任何物质上的先决条件}。根除土地问题中的中
世纪性质、推翻沙皇专制、实施对剥削的限制措施,既是革命的驱动力,也是历史的中心任
务。但是,它们的实现,主要依赖于一个\textbf{革命的工人政府的建立}。资产阶级无力领
导民主革命,或者说无法和无产阶级的领导进行合作。资产阶级的抵制,迫使无产阶级夺取
政权,并且据此实施集体主义的经济措施。\textbf{成功的民主革命,将“长入”到一个连
  续的或“不断的”社会主义革命过程中。}\textbf{俄国将不会经由民主革命实现社会主义;
  相反的,民主革命的任务只有通过社会主义革命才能实现。}因此,托洛茨基颠倒了普列汉
诺夫和列宁的马克思主义的革命顺序。

托洛茨基认为,农民只能选择追随无产阶级或选择支持反动势力;他们\textbf{不可能
采取一种独立的立场。}但是,正是俄国农业的落后性质,使得农民最终接受无产阶级的领
导。土地问题在沙皇时代肯定无法解决,只有通过无产阶级专政才能得到根本解决,这将确
保农民对反对地主和沙皇国家的支持。\textbf{无产阶级专政的土地基础,就是广泛的资本
主义关系的缺乏,而不是像布尔什维克和孟什维克认为的那样,要依靠资本主义关系的成
熟。}

因此,落后俄国的工人阶级有可能\textbf{先于}工业发达国家的无产阶级获取政权。然而,
在托洛茨基看来,在孤立的状况下,是无法维持这一政权的。\textbf{最终,必然和农民发
生冲突,因为农民对无产阶级专政的支持只限于完成土地革命。维护无产阶级统治所要求的
集体主义措施,将导致与农民的分道扬镳,无产阶级掌握政权措施的结果,将会同时削弱这
种统治的非无产阶级基础。}在此意义上,土地问题既是俄国社会主义革命的最大帮手,也
是它的一个主要挑战者。

不断革命论结果陷入矛盾之中,\textbf{只有革命超越了民族的界限,并且成为世界范围内
  持续的“不断的”革命时,这个矛盾才可能解决}。但是,俄国无产阶级的选择再次受到限
制。正如它作为民主革命的主角参与到俄国革命中,结果被推动去超越民主革命一
样,\textbf{民族革命也将被迫成为超越国界的革命,因为欧洲国家将努力反对俄国的革
  命。}为捍卫民族的胜利,俄国无产阶级\textbf{势必把阶级斗争扩大到西方,它的革命也
  将“长入”世界革命。}在这里,托洛茨基试图对他的分析同正统马克思主义的分析加以调
和。\textbf{社会主义革命的先决的物质条件存在于工业发达国家,而俄国革命的命运提出
  了需要十足的无产阶级政权的问题。}就历史过程的结果而言,正统派是对的,但
是,\textbf{他们的错误在于实现这一目的的方式上:革命将会从东到西发展,而不是相
  反。}西方成功的革命,将消除对俄国革命的军事威胁,使得西方的资源可以用于俄国社会
主义建设之中。

这些就是不断革命论的政治学的内核。但是,托洛茨基并没有将他的分析限于这一层面,他
也不可能这样做。正如他所设想的,不断革命是一个必然的过程,无论革命党的纲领是什
么,\textbf{历史事件的逻辑将要么绕过它们,要么吞没它们。}把俄国革命带向成功的任务,
必然要求俄国马克思主义者理解革命的真正本质。但是,\textbf{过程本身超出了他们的控
  制,因为它根植于物质条件之中。}与考茨基和普列汉诺夫一样,托洛茨基的马克思主义从
本质上看是决定论式的。因此,我们转向托洛茨基的政治经济学,这种政治经济学为托洛茨
基不同寻常的政治学观点提供了理论基础。

\section{俄国的不平衡和综合发展}

托洛茨基对俄国历史整体论述的重要观点,显然是符合正统马克思主义传统的。在指出传
统俄国具有“半亚细亚”性质方面,他追随了马克思、恩格斯和普列汉诺夫。但无论如何,
他也没有把自己同列宁明显地区分开来。

托洛茨基对当时俄国工业的性质、促使这种性质形成的制度,以及这些制度对城市社会结构
的影响作了考察,这些是至关重要的。\textbf{托洛茨基很少关注农村的经济条件,他认为
  即使在1905年革命之前,俄国农业资本主义也不怎么明显。}这使得托洛茨基对俄国工业发
展的分析,更接近于杜冈-巴拉诺夫斯基的分析,而不是接近于列宁的分析(参见以上第九
章)。但是,与列宁一样,托洛茨基也关注资本主义发展形式产生的革命潜力。与列宁相同,
但不同于杜冈-巴拉诺夫斯基,托洛茨基的经济分析主要是\textbf{服务于解决政治问
  题}的。

\textbf{在托洛茨基看来,地理条件有助于说明俄国社会的亚细亚特征,这意味着俄国传统
  的城市主要是行政和军事堡垒。城市的商业活动很少,制造业更不集中,作为农业的附属
  物分布在农村。}因此,俄国的城市与欧洲中世纪晚期和现代早期的城市迥然不同。俄国资
本主义的发展并没有造成与欧洲的趋同。相反,它与早期的差别一起,使得20世纪俄国
的\textbf{城市人口主要由无产阶级构成},与欧洲其他资产阶级和小资产阶级仍占据重要地
位的城市相比,无产阶级的人数更多。俄国城市的平民特征,反映了沙皇专制工业化过程
的“\textbf{后发}”特征、反映了在这个过程中对外国资本的广泛使用和国家职能的扩大。
这最大程度地\textbf{限制了俄国本土资产阶级的发展},资本所有权主要控制在国家或外国
人手中。尽管这提高了西方城市的资产阶级特征,但它却削弱了俄国城市的资产阶级特征。
同时,工业化没有能够显著地增加城市小资产阶级的数量。工厂生产往往是大规模的生产,
绕过了欧洲工业发展早期阶段存在的现象。反过来说明,\textbf{这不仅创造了大量的无产
  阶级,而且这些无产阶级还具有鲜明的现代特征。}

托洛茨基承认城市人口只占少数,但他认为,城市人口的规模并不能反映它的经济与政治力
量。\textbf{城市创造了与其人口总数不成比例的大部分的国民收入,并且国家结构的神经
贯穿于每个城市。托洛茨基认为,所有的现代革命都是、并且必然是城市领导的。}托洛茨
基显然比列宁更像马克思,坚持认为农民的生产关系和生活方式使得农民绝对不可能发挥独
立的政治作用。\textbf{由此得出结论:在俄国任何形式的革命都将是无产阶级革命,}尽
管无产阶级处于少数派的地位,但它有能力摧毁现存的国家机器,并取而代之的是自己设计
的国家机器。

这一结论本身,同其他的俄国马克思主义者的结论类似。无产阶级领导权毕竟是普列汉
诺夫早在19世纪80年代就清楚地说明的原则,而且列宁对这一原则也作了反复强调。托洛茨
基开辟的新领域在于:强调无产阶级无与伦比的力量以及在革命过程中必须解决的问题的重
要性。\textbf{他低估了资产阶级、农民和整个经济发展水平对革命产生的限制作用。}所
有这些问题,都与他提出俄国工业化独特性的观点密切相关。

我们已经对托洛茨基有关农民以及无产阶级革命可能在农村引起普遍不满的观点作了考
察。对于俄国资产阶级来说,它不仅只是代表了城市人口中的一小部分,而且它的利益与现
存秩序相联系。资产阶级在经济上与地主和国家融合在一起,而且在与工人阶级的利益发生
冲突时,资产阶级需要一个专制的政府作为自己的保护者。所以,它会抵制最有利于无产阶
级和农民的民主要求(八小时工作制和没收地主土地)。这将阻止资产阶级领导革命,并迫
使无产阶级把革命扩大到打破资产阶级革命的界限。换言之,\textbf{资产阶级的保守性,
不仅对革命产生了限制性的影响,而且也是社会主义革命的催化剂。}

托洛茨基从来没有否认,俄国完成社会主义革命的物质条件\textbf{还不够成熟}。但是,他
确实认为,这并不会削弱俄国发动社会主义革命的能力。在这方面,\textbf{重要的不是俄
  国经济发展的总体指标,}而是他在分析中所揭示的这一发展所催生的\textbf{阶级力量的
  汇聚}。此外,他相信,俄国革命在国际范围内的扩张是极有可能的。俄国工业同国外资本
的联系,使得欧洲关注俄国工业的命运。\textbf{无产阶级由于资产阶级反对革命而被迫进
  行的剥夺,引起欧洲国家的干预,因而也迅速地在整个欧洲大陆的范围内提升了工人阶级
  的力量。}西欧的无产阶级将会帮助俄国的工人阶级。尽管托洛茨基从来没有详细地说明其
他国家的资源是怎样与俄国社会主义联系起来的,但是,这种联系使他坚信,他的主张是和
正统马克思主义连接在一起的。可以认为,俄国的阶级斗争最终同作为社会主义发展前提条
件的生产力发展是相符合的。

贯穿托洛茨基所有这些观点的核心概念是:“\textbf{不平衡和综合发展}”。在托洛茨基看
来,俄国的现代化是一个不平衡的过程。一些部门的发展不仅超过另一些部门的发展,而且
它们还具备了在世界范围内也算得上是最先进的特征。同时,它们又综合在单一的社会形态
中,这种社会形态中的其他部门具有不同的历史发展进度。这不是偶然的:\textbf{存在着
  功能整合的特点。}因此,落后的农业中剥削的加剧阻碍了它自身的发展,但是却有利于创
造现代资本主义工业的“飞地”。传统农业问题的恶化,使得农民在革命中跟随在无产阶级
后面。反过来,无产阶级的领导,将保证传统农民暴动的成功。相似的情况对阶级团结并不
总是必要的:在情况存在差异的形势下,阶级之间的结盟可能源自它们之间的相互依赖。

尽管普列汉诺夫和列宁是复杂的理论家,在他们的著作中可以发现这些思想的不同方面,但
总体说来,他们的观点与托洛茨基的存在明显差异。例如,列宁试图说明俄国资本主义发展
的\textbf{广泛性与欠发达}的本质;而托洛茨基则强调在前资本主义农业的背景下,它
的\textbf{集中程度及其发达形式}。对列宁来说,工农联盟是基于俄国的落后这一共同的条
件,托洛茨基则恰好相反。这种差异的后果之一,就是“\textbf{落后}”这一观点本身成了
被怀疑的对象。尽管托洛茨基本人经常使用这一术语来描述俄国,并且他设想的革命被认为
是“\textbf{对落后的革命}”,但是,托洛茨基分析的逻辑表明,这些定义是不恰当
的。\textbf{他的观点的核心是,一个姗姗来迟的现代的俄国,造就了一种在欧洲既是发展
  得最为现代的、又是发展得最为迟缓的经济结构。正是这种不平衡的形式,提供了理解俄
  国的历史和未来的线索。 }

对托洛茨基来说,前途就是不断革命。但是,\textbf{他确实含蓄地承认,不平衡和综合发
  展具有与不断革命相抵触的性质。}特别是\textbf{俄国国家机器的现代化},使它变得比
传统的专制主义更加强大。作为变革的主要的代理人,\textbf{它使有产阶级对它的服从达
  到极端的程度,并且引进了一些与完善其统治方式相关的知识和技术。}在这一意义上,俄
国的革命是\textbf{正在进行现代化的国家与现代无产阶级之间的斗争}。尽管托洛茨基从来
没有使用这些术语来说明问题,但是从他的分析中可以得出这些结论,并且应当能够被用来
证明他的信念:\textbf{不断革命是俄国唯一可能的前途}。实际上,这种不明确性,在他后
来的思想中注定要发挥巨大的作用,我们将在以下第7节考察这一点。但是,在1917年之前,
托洛茨基具有第二国际马克思主义的决定论特征,尽管他对第二国际马克思主义的其他特征
的是有疑义的。

\section{托洛茨基的马克思主义和马克思主义的遗产}

把对俄国社会主义发展特殊道路的描述,建立在不平衡发展的逻辑基础上,说明托洛茨基的
马克思主义明显地受到\textbf{民粹主义}的影响。尽管托洛茨基承认民粹主义并不是没有洞
察力,但是却没有证据表明,民粹主义是托洛茨基得出他的结论的重要影响因素。像俄国其
他许多马克思主义者一样,托洛茨基也经历过赞同民粹主义思想的阶段,而且也很典型的
是,\textbf{一旦接受了马克思主义,他就再也没有转向民粹主义思想。}事实上,他有关农
民革命能力的观点,使他的理论明显地呈现出反民粹主义的特征。因此,在说明托洛茨基马
克思主义的具体特征时,我们必须寻找其他方面的影响。

其他方面的影响主要表现在两个方面:亚历山大·赫尔凡德(其笔名\textbf{帕尔乌斯}更为
著名)的思想和1905年的事件。托洛茨基相信后者证明了前者的思想,并超越了前者的思想。
在1904-1906年期间,帕尔乌斯和托洛茨基联系密切,并且详细地讨论了俄国革命的所有问题。
考虑到托洛茨基的学识,这种影响是否是单向的是值得怀疑的。托洛茨基把相关的见解发展
为系统的理论体系。在这样做时,他超越了帕尔乌斯,\textbf{帕尔乌斯不相信俄国的工人
  政府能够构成社会主义革命的第一阶段,而不只是作为彻底解决民主任务的手段。}托洛茨
基并没有直接接受帕尔乌斯思想中的某些因素,尤其是关于\textbf{资本主义经济发展已经
  使民族国家过时了、这些发展将造就帝国主义战争时代的观点。}后来,托洛茨基接受了这
种观点(我们将在以下一节看到),这使得他的马克思主义更接近于主流马克思主义。然而,
即使那时,他也因为同正统派的彻底决裂,而不断地受到批评。

毫无疑问,还有一些其他因素影响了托洛茨基。把“不断革命”的概念和俄国的情况联系起
来,最早是由\textbf{大卫·梁赞诺夫}在1903年提出的,这一观点主要建立在这样的判断上,
即\textbf{革命的社会主义运动在俄国十分发达,欧洲社会主义革命应当参与到俄国革命中
  去。}梁赞诺夫通过对1848-1850年间马克思策略发展的分析,来支持\textbf{不断革命思
  想}的针对性(参见以上第七章)。我们并不知道托洛茨基(或帕尔乌斯)在多大程度上受
到梁赞诺夫的实际影响,关于俄国革命的激进性质和无产阶级掌握革命领导权的必要性的观
点,是20世纪早年托洛茨基所生活的环境的一部分,但是这些观点很少准确地被表述过。正
统马克思主义也没有为资产阶级革命过程中无产阶级的统治将受到怎样的限制提供令人满意
的说明(参见以上第八章和第十一章)。当1905年革命爆发时,无产阶级激进主义的上升曲
线向托洛茨基表明,实际的约束可能已经被克服了。

作为“不断革命论”的基础的不平衡和综合发展的思想,也不是由托洛茨基本人,或是由他
同帕尔乌斯一起带入俄国马克思主义中的。在普列汉诺夫和列宁的早期著作中,只是简单地
预见到这一概念(参见以上第八章和第九章)。更重要的是,\textbf{梁赞诺夫认为,沙皇
  专制不只是旧社会的残余,沙皇专制还适应了资本主义发展的需要。他也承认,土地条件
  能够为俄国的无产阶级革命提供支持。}考茨基也指出俄国的国外投资产生的社会和政治后
果,认为俄国革命必然显示一种历史的独特性。最为重要的是杜冈-巴拉诺夫斯基的著作,
他的著作准确地聚焦于沙皇时代工业化的不同方面,为托洛茨基的分析奠定了唯物主义基础
(参见以上第九章)。但是,托洛茨基在《总结与展望》中广泛地引用考茨基的论述时,从
来没有表明他在理论上受过杜冈-巴拉诺夫斯基的恩惠。考虑到合法马克思主义者脱离了社
会民主党的阵营,这一点并不让人感到意外,但是,考虑到《19世纪的俄国工厂》这部著作
的分量,认为托洛茨基不曾读过,或者这部著作不曾给他留下什么印象,却是令人难以置信
的。

然而,托洛茨基的思想被他同时代的人视为同正统派是彻底决裂的。毫无疑问,他的思
想最初出现时,这种评价是正确的。在我们考察托洛茨基反对他的批评者并捍卫自己的观点
之前,关注一下马克思著作中的两个特征是有意义的,托洛茨基可以从中寻求支持。这两个
特征虽然不会消除认为托洛茨基严重偏离正统马克思主义的指控,但确实可以使人认为这种
指控只在一般意义的性质上是成立的。在此基础上,可以认为,尽管托洛茨基是一位激进的
创新者,但他同时也在突出马克思思想的深层结构。

\begin{quotation}
  人们在自己生活的社会生产中发生一定的、必然的、不以他们的意志为转移的关系,即同他
  们的物质生产力的一定发展阶段相适合的生产关系。这些生产关系的总和构成社会的经济
  结构,即有法律的和政治的上层建筑坚立其上并有一定的社会意识形式与之相适应的现实基
  础。物质生活的生产方式制约着整个社会生活、政治生活和精神生活的过程。不是人们的
  意识决定人们的存在,相反,是人们的社会存在决定人们的意识。社会的物质生产力发展到
  一定阶段,便同它们一直在其中活动的现存生产关系或财产关系 ( 这只是生产关系的法律
  用语) 发生矛盾。于是这些关系便由生产力的发展形式变成生产力的桎梏。那时社会革命
  的时代就到来了。

  我们判断一个人不能以他对自己的看法为根据,同样,我们判断这样一个变革时代也不能以
  它的意识为根据,相反,这个意识必须从物质生活的矛盾中,从社会生产力和生产关系之间的
  现存冲突中去解释。无论哪一个社会形态,在它们所能容纳的全部生产力发挥出来以前,是
  决不会灭亡的;而新的更高的生产关系,在它存在的物质条件在旧社会的胎胞里成熟以前,是
  决不会出现的。所以人类始终只提出自己能够解决的任务,因为只要仔细考察就可以发
  现,任务本身,只有在解决它的物质条件已经存在或者至少是在形成过程中的时候,才会产
  生。(马恩全集,第一版,第十三卷,P8-9,政治经济学批判序言)

  
  必须考虑到, 新的生产力和生产关系不是从无中发展起来的,也不是从空中,又不是从自己
  产生自己的那种观念的母胎中发展起来的,而是在现有的生产发展过程内部和流传下来的、
  传统的所有制关系内部,并且与它们相对立而发展起来的。如果说,在完成的资产阶级体制
  中,每一种经济关系都以具有资产阶级经济形式的另一种经济关系为前提,从而每一种设定
  的东西同时就是前提,那么,任何有机体制的情况都是这样。这种有机体制本身作为一个
  总体有自己的各种前提,而它向总体的发展过程就在于:使社会的一切要素从属于自己,或者
  把自己还缺乏的器官从社会中创造出来。有机体制在历史上就是这样向总体发展的。它变
  成这种总体是它的过程即它的发展的一个要素。\pagescite[][235-236]{karlvol46a} 
  
\end{quotation}

正如《政治经济学批判》中概括的那样,\textbf{历史唯物主义包含了不平衡和综合发展的
  概念。转型时代是两种生产方式交织在同一种社会形态中的时代。它们的发展是不平衡的,
  进步的生产方式获得了超前的发展,而另一种生产方式的发展则滞后了,最终带来的危机
  只能通过社会革命才能解决。}在《政治经济学批判大纲》中,马克思坚持认
为,\textbf{这有利于建立在单一生产方式基础之上的新的有机体的形成。但是在其它地方,
  他承认不平衡和综合发展可能是更加普遍的现象}:例如,马克思认为,西欧的封建制源自
古代奴隶制与征服了罗马的德意志野蛮部落的社会组织之间的相互渗透。马克思也注意
到,19世纪的英国这一“典型”的资本主义模式,并不要求贵族必然丧失他们在政治上的支
配地位。在另一个极端,他认为法国农民代表了“文明中的野蛮”。这是他鄙视农村生活落
后的一种表现,同时也反映了资本主义对农村渗透的迟缓。

这样的例子还有很多,它们与马克思对历史唯物主义做出的更具一般性的阐释之间存在
\textbf{矛盾},在这种阐释中,不平衡和综合发展的概念是隐含的和受到限制的。因此,
托洛茨基的马克思主义,\textbf{可以被视为是为了提供一个更具一般性的框架而进行的第
一次有意识的尝试,}在这个框架中,马克思承认的不平衡和综合发展的复杂性与历史唯物
主义的核心思想保持了一致。托洛茨基自己从来没有明确地宣称这一点,但他这样做不是没
有道理,特别是在他把分析拓展到世界经济层面和非俄国的外围资本主义之后(我们将在以
下一节讨论这一点)。

事实上,托洛茨基为了对抗更为正统的批判而进行的辩护中,包含了不同类型的观点。
他宣称,\textbf{马克思主义代表了一种研究社会关系的科学,而不是文本的注释},因此
他自己的思想应当在这种思想是否准确地说明了俄国革命的阶级动力的意义上加以评价。为
达到这一目的,托洛茨基阐明了这些思想中的某些因素。\textbf{他否认他曾断言俄国可以
直接从专制主义转向社会主义,或者说,不断革命意味着可以跨越民主革命阶段。不断革命
的概念没有混淆革命过程的必然阶段。}它不像批评家宣称的那样,是一种把不同发展类型
混合在一起的理论,而是俄国历史过程本身就是这样的。社会经济的发展并不像正统派描述
的那样理性,\textbf{这给革命提出了无法回避的真正的问题}。

托洛茨基确实声称,不断革命论和马克思主义之间有着\textbf{明确的传承关系},因为这一
思想源自马克思1848年革命期间的观点。这不是一种冒险主义,尽管在19世纪中期不断革命
并不成熟,但是通过考察1789年、1848年和1905年革命的结构,我们可以看到资产阶级革命
嵌入社会主义革命是这一时代的内在属性。这已在一定程度上被拉萨尔、考茨基、卢森堡,
甚至普列汉诺夫所承认。\textbf{托洛茨基把不断革命论,同对孟什维克理论的毁灭性批判
  和同对布尔什维克观点稍微不太有利的评价联系在一起了。孟什维主义把自己的理论建立
  在一个正式的与俄国发展的独特性无关的历史类比的基础之上。资产阶级民主是不可能实
  现的,因为不存在独立的革命的中产阶级。}因而坚持孟什维克主义,就意味着采取了一种
客观上反革命的立场。对布尔什维克来说也是一样的,差别只在于\textbf{布尔什维克的保
  守性只在革命成功后才显现出来。一旦投身于无产阶级,农民就无法抑制自身的激进主义。
  并且,没有认识到实施反资产阶级措施的革命,随后就会被动地为资产阶级的统治开辟道
  路。}

1906年到1917年间,托洛茨基强化了他的主张,表述了合法马克思主义的一个观点。他在
《总结与展望》中作出的最初的说明,主要来自俄国的立场,\textbf{并没有在国际资本主
  义的结构内说明俄国的现代化。}他也没有以一种严格的方式,把俄国革命与欧洲革命联系
起来,\textbf{没有理由证明,当俄国的革命成功时,欧洲的条件将有利于俄国革命的扩
  展。}托洛茨基认为,社会主义的先决的物质条件的确存在于先进国家,但他也注意到,对
俄国来说,阶级斗争在决定革命的爆发方面至关重要,但他\textbf{对西方的阶级冲突的发
  展几乎没有说出什么内容}。然而,到1917年时,托洛茨基做了许多有损他这些观点作用的
事情,因此他的观点更接近于正统马克思主义,接近于布哈林和列宁的思想(参见以下第十
三章)。托洛茨基对自己的理论的进一步补充和完善,是在1917年之后作出的,而
到20世纪30年代初,托洛茨基可以宣布,他已经建立了与资本主义发展的最后阶段相联系的
世界历史理论。

\section{帝国主义与世界经济}

托洛茨基认为,第一次世界大战的爆发标志着一个时代结束。资本主义的进步作用已经终结,
资本主义民族国家已经成为生产力进一步发展的桎梏。其结果是欧洲帝国主义的出现,反过
来造就了一个一体化的世界经济。但是,托洛茨基认为,\textbf{一旦资本实现了对世界统
  一的支配,主要国家就将被迫卷入一场争夺霸权的斗争。在1914年之后,在世界历史上,
  唯一的选择要么是社会主义、要么是野蛮。}卢森堡较早的时候已经表达了这种观点(她也
深受帕尔乌斯的影响),而且布哈林、列宁差不多在同一时间得出了与托洛茨基类似的结论
(参见以上第六章和以下第十五章)。\textbf{纯粹民族革命的时代已经过去,资本主义成
  为一种全球体系,所有的民族国家已经联系在一起,并受到它们作为其组成部分的整体结
  构的支配。}对托洛茨基来说,无论对先进国家还是落后国家,都是如此。\textbf{彻底地
  摆脱这些关系,是任何一个孤立的社会主义革命无法完成的任务。挽救一个国家的革命的
  唯一出路,在于把革命扩大到世界经济中的其他国家,并逐渐削弱国际资本。}托洛茨基现
在主张,革命者应当根据他们对资本主义体系整体的影响,来评价自己的革命行动。求助于
任何特定国家的经济发展水平都是时代的错误。现在,正是全球性的生产力和它受到的民族
国家生产关系的制约,在发挥作用。这些力量将越来越多地被用作战争的手段,威胁并破坏
过去的进步发展。

因此,对托洛茨基来说,尽管俄国革命一直具有世界历史意义,但直到1917年,这种观点才与
一个更加综合的视角联系起来,在这种视角中,他能够对正统派对他的思想进行的最猛烈的
批评进行反击:在世界经济已经得到成熟发展的意义上,俄国进行社会主义经济建设的时机
才是成熟的,不存在其它的评判标准,因为资本主义的本质是一种\textbf{全球体系}。事实
上,托洛茨基明确地把资本主义作为一种世界经济纳入自己的分析之中,这意味着所有的历
史唯物主义的基本观点现在都可以被证明。对于认为这些主张没有得到运用的异议,可能反
驳的是,马克思本人并没有指明准确的参照点。第二国际的理论家——包括考茨基、希法亭、
普列汉诺夫和列宁(但不包括卢森堡)——都倾向于认为,马克思的资本主义模型是与单个民
族的资本主义相联系的,但这种看法无法从《资本论》中找到支持。在那里,马克思分析
了“\textbf{资本一般}”,而不是特定的民族资本主义国家,也不是一系列民族国家的资本
主义。而且《共产党宣言》明确地把资本主义视为一种全球体系。此外,历史唯物主义是一
种世界历史理论。那么,它怎么可能只限于诸如民族国家这样的特定的历史制度呢?

但是,托洛茨基主张,“\textbf{后发优势}”确实仍然在发挥作用。\textbf{帝国主义战争
  代表了资本主义的总危机},因而开启了向全面的社会主义过渡的新纪元。精确地看,这种
过渡从哪里开始,并不是一个十分重要的问题。但是,俄国的不断革命恰好适合于领导其它
国家走上这条道路,\textbf{因为在一个充满了不平衡和综合发展的结构中,会更强烈地感
  受到战争造成的破坏性影响。此外,由于俄国既是一个半殖民国家,同时也是一支帝国主
  义力量,因此,无产阶级革命将会通过消除一个剥削区域而严重削弱西方的经济,并加剧
  国际范围内的阶级斗争。}

所有这一切与托洛茨基早期的观点并不矛盾,而是为它们提供了一个更加稳固的基础。他把
俄国资本主义视为帝国主义关系国际体系的一部分,而且他相信革命的传播效果得到加强。
托洛茨基的观点的发展,引起了对同后一个问题相联系的问题的思考。他认识到\textbf{先
  进国家也展现出不平衡和综合发展的重要特征。只有美国接近于纯粹的资产阶级社会。其
  它的西方国家仍然存在从来没有被实现的民主革命的一面}(参见以上第四章、第九章和第
十一章)。因此,欧洲的社会主义革命必须完成民主革命的任务,从而不会完全不同于俄国
的不断革命。最终,可以预见俄国革命具有强烈的\textbf{象征意义},即使对先进国家也是
如此。

但是,托洛茨基强调了\textbf{俄国与边缘资本主义地区的相似性},这些地区包括1917年之
前的巴尔干地区、20世纪20年代一般意义上的殖民地与半殖民地国家。因此,他把俄国的经
济发展视为\textbf{典型的落后的资本主义国家经济发展的代表},对所有这些地区和国家而
言,不断革命的模式都是适用的。这意味着基于与俄国相同的原因,民族主义运动、反帝国
主义运动都是社会主义斗争的一部分;它们同样只能在无产阶级的领导下才能获得成功。而
且任何地区的胜利,都将在地缘上扩大工人阶级的力量,通过减少剥削区域沉重打击国际资
本主义,加速革命向以大都市为中心的帝国主义国家的蔓延。毫不奇怪,托洛茨基逐渐
把20世纪看作是不断革命的时代。

\section{托洛茨基的马克思主义的力量}

毫无疑问,托洛茨基的主要思想成就,是他对沙皇俄国工业化导致的政治发展的理解。
\textbf{他比其他马克思主义者更准确地预言了推翻沙皇专制引发的一系列事件的顺序}:
1917年的历史与他十年前的预测大体一致。此外,托洛茨基比其他激进的布尔什维克更敏锐
地认识到,这样一场革命将要面临的困难,因而在20世纪20年代这些困难出现时,他更有可
能不会低估这些困难,即使他并没有为克服这些困难做好更充分的准备。尽管在革命后的年
代,托洛茨基也存在各种各样的错误和反复,但他的理论的深度和连贯性都远远超过布哈林
和斯大林。

对托洛茨基政治经济学的广泛支持,来自具有\textbf{最强烈的理论倾向的西方经济史学家
  亚历山大·格申克龙},他对沙皇俄国工业发展的详尽分析,就是以类似托洛茨基使用的方
式建立起来的。他对专制统治下工业化逻辑的描述与托洛茨基相类似。他把俄国发展的独特
性作为大陆模式的一部分,充实了托洛茨基对1789-1905年欧洲革命序列的分析。此外,通过
对不平衡和综合发展对理解落后地区的发展的极端重要性的强调,格申克龙对托洛茨基把他
的理论应用到俄国的西方地区提供了支持。当然,这种理论的普遍化对马克思自己的分析提
出了重要挑战。\textbf{尽管资本主义代表了一种正在普遍化的生产方式,但是在托洛茨基
  看来,作为历史转型的发动机,它并不像马克思本人暗示的那样强大有力。}在这一点上,
经验证据明显地支持了托洛茨基:在英国和法国——马克思眼中资产阶级革命的典型样本,旧
制度的主要因素在整个19世纪一直存在。这已经成为现代欧洲史学的主题。

也可以为把托洛茨基的不平衡和综合发展的一般理论应用于俄国的东方地区提供支持。19世
纪欧洲资产阶级带来了资本占支配地位的全球经济,但是,它并没有像《共产党宣言》所宣
称的“\textbf{按照自己的面貌为自己创造出一个世界}”。相反地,正如托洛茨基理解的那
样,\textbf{它产生了一个复杂的分工的等级结构,在这一结构中,前资本主义经济形式通
  常使自己适应资本的要求,而不是被它消灭。因此,托洛茨基的著作,可以被视为
  是1945年以后马克思主义的一个重要分支的最早的例子之一。}

托洛茨基更为具体的\textbf{“时代概念”的有效性显然是值得怀疑的}:1914年并没有引发
大国之间的一系列帝国主义战争;先进资本主义国家稳定的资本积累,并没有像托洛茨基认
为的那样已经结束,而且正式的殖民帝国已经瓦解。\textbf{资本主义已被证明能够适应转
  型形式,它的灵活性比托洛茨基预计的要大。}然而,20世纪上半叶托洛茨基的观点,与西
方资本主义的经济和政治的历史是一致的,整个资本主义体系见证了一系列最漫长、最深重
的危机。这不能被视为是纯粹偶然的情况造成的结果。在托洛茨基设想的形式中,存在一个
结构性的问题。\textbf{为了有效地运行,资本主义要求建立一个国际组织,一个类似于民
  族国家在自己疆域内形成的组织。}在19世纪的大部分时间里,这一组织的重要雏形是由英
国的海上霸权提供的。自20世纪中期开始,一个更加全面的美国霸权变得越来越明显。但是
在这两段时间之间,由单一民族国家行使国际霸权的经济基础并不存在,并且类似的替代组
织也不存在。结果,欧洲军国主义势力高涨,国际经济遭到大规模的破坏,爆发了彰显帝国
主义野心的两次世界大战。

此外,也就是托洛茨基本人,在20世纪20年代,为理解这一中间阶段为什么是一个过渡期提
供了基础。他承认,相对于美国而言,欧洲战争削弱了重要的好战国家的经济和军事实
力;\textbf{到20世纪20年代,在同欧洲相联系时,美国已开始展示它的经济力量,它的长
  期利益在于瓦解所有的殖民帝国,以便实施“门户开放”政策。}托洛茨基没有意识到这对
他提出了一个新问题,因为他坚持西方资本主义陷入了一场仅靠自身无法从中解脱的危机。
相反地,他坚信先进资本主义国家之间不断发生的帝国主义战争,相对于世界范围内的社会
主义革命而言是唯一的选择。但是,与大多数持同样的总体的观点的马克思主义者相比,在
个别孤立的要点上,他对未来持有的观点具有更少的决定论的特征。

\section{托洛茨基马克思主义的不足}

按第二国际时期正统马克思主义的标准,托洛茨基思想中的非决定论特征,是他的马克思主
义存在缺点的潜在来源。不平衡和综合发展以及不断革命的概念,源自\textbf{历史“特殊
性”}的观念,托洛茨基从来没有停止证明它们的重要性。这样,马克思主义的科学地位的
主张势必受到削弱。因此,托洛茨基体系的主要困难在于,他在承认历史特殊性的支配作用
的同时,试图提出历史发展的\textbf{一般规律},尽管托洛茨基具有简化对自己论点中的
关键之处进行解释的倾向,在一定程度上掩盖了这种特征的真实程度。

我们可以从他的著作的许多特征中看出这一点。托洛茨基对不断革命论的最初表述,是把它
当作\textbf{必然的事件的序列}。然而,在1917年革命取得明显胜利时,托洛茨基因为采纳
了布尔什维克的\textbf{唯意志论}而削弱了它。这是对他先前立场的\textbf{重大逆转},
他先前的立场倾向于建立一个\textbf{结构松散的群众政党},本质上是孟什维克式的。他并
不是没有意识到孟什维主义中包含的机会主义的危险,但是,他相信当新的群众起义爆发
时,\textbf{客观事件自身的力量将会迫使孟什维克采纳不断革命论。}结果,“取代”的威
胁——\textbf{布尔什维克党取代阶级成为革命斗争的主体}——被视为一个更大的危险。二月革
命之后,大多数孟什维克采取的阶级调和政策,\textbf{最终使他相信列宁在意识形态的纯
  洁性和集中化的党组织问题上的毫不妥协是正确的。如果只是从隐含意上理解,托洛茨基
  现在不得不接受不断革命并不是俄国经济发展的必然结果。}而列宁描述的通过“普鲁
士”道路完成俄国的现代化显然是可能的。自此以后,托洛茨基就这一问题提出的“解决办
法”,强调“主观”因素是革命成功必然的前提条件。事实上,他的观点简化为俄国之外革
命力量的失败是领导不当的结果,并且如果没有列宁,俄国革命本身也将会夭折。这
是\textbf{带有复仇性质的唯意志论}。

托洛茨基在把不断革命论普遍化时也存在类似的缺陷,不断革命论源自对俄国经济发展
的深刻的分析。在把它应用于更广泛的地区时,他没有为这种理论提供类似的分析,他以帝
国主义在整个东方再生产出了类似俄国的那些特性为由,撇开了实际的民族的特殊性。毫无
疑问,马克思遗产中的某些因素推动了这种情况的出现,尤其是在资本主义渗透之前,亚细
亚生产方式是整个非西方世界的特征的观点,推动了这种情况。但是,无论是什么原因,托
洛茨基在把自己的\textbf{理论普遍化时存在的肤浅性},在1945年以来席卷整个世界的实
际革命中显露无疑:\textbf{没有哪一个革命遵循了不断革命的模式}。东欧是通过红军的
占领苏维埃化的;中国的革命是通过城市知识分子组织的大规模农民起义获得成功的;古巴
革命、越南、埃塞俄比亚和其它地方的革命,\textbf{很少展现出无产阶级领导的迹象}。

此外,借助于托洛茨基首先明确表达的这一概念,马克思主义者揭示了欧洲之外的国家的结
构和历史的多样性。按照托洛茨基的方法,甚至\textbf{比托洛茨基更接近于这种方法},马
克思主义者展现了内在于托洛茨基最好的著作中的\textbf{非决定论特征}。在他们的手中,
世界经济成了不同模式接合的拼盆,而马克思主义自身被简化为理解这个拼盘必须
的“\textbf{工具箱}”。就算是同历史唯物主义创始人的最初主题的连续性被保持了,它也
只是通过诉诸于专制政党的所谓的权力,寻求在不存在苏联过去那样情况的地方复制苏联的
历史。因而马克思主义成了东欧政治结构和第三世界民族运动的借口。

在所有这些问题上,和托洛茨基还存在另一处明显的联系。\textbf{在他早期著作中,
最主要的缺陷在于把不断革命与社会主义联系的方式上。}托洛茨基声称,只要革命被扩大
到国际范围,俄国不断革命的成功可以作为社会主义建设的第一步。但是,他从没有为这种
观点提供令人信服的论据,而且根据普列汉诺夫清晰表达的相反的观点(参见以上第八章),
这种“疏忽”尤为麻烦。这种省略与托洛茨基的后革命社会的观点有关。尽管它不反对对未
来共产主义社会进行人道主义式的描述,但是,在他关于向社会主义过渡的分析中,
\textbf{这种人道主义是普遍缺乏的},在过渡中,生产关系的根本变化,等同于废除私有
财产和结束商品生产,而对生产的专制管理仍然不变,因此,直到20世纪40年他的生命结束
时,托洛茨基仍然把苏联描述为“工人国家”(尽管是一种退化形式的),这恰恰是因为没
有恢复私有制。

\textbf{托洛茨基试图通过把革命扩大到国际范围复活马克思主义的尝试,也因其总是
倾向于简化他对争论中的关键问题的解释而受到诟病。}资本主义民族国家之间的矛盾和生
产力进一步发展的经济基础问题,没有被加以详细说明。到底它是一个规模经济问题、消费
不足问题,是垄断的超额利润产生的剩余资本问题,还是其它一些问题,他从来没有明确地
指明。在某些方面,托洛茨基的视角与卢森堡的有点相像。有时候,他暗示资本主义经济面
临的主要问题是长期的市场短缺问题。如果这是托洛茨基的观点,这还容易理解;在一战前,
帕尔乌斯持这一观点,并影响了卢森堡和考茨基(参见以上第四和第五章)但是,托洛茨基
从来没有明确阐述帝国主义对抗背后的经济机制。因此,他对危机的分析差不多成了一种循
环论证:战争是矛盾的唯一证据,而矛盾又被用来解释战争。

这与他后革命的立场存在某些联系。\textbf{托洛茨基没有对不同的帝国主义经济进行
区分;尤其是在对战争中的帝国主义关系进行的说明中,缺乏不平衡和综合发展的概念。}
他相反表明,所有的“大国”都面临着几乎同样的问题:为有效利用它们的生产力找到充足
的市场。这意味着他的视角与列宁的截然不同,列宁强调帝国主义力量的不平衡发展,并把
它作为自己的战争理论的基础。在20世纪20年代,理论上的这种差异,成为布尔什维克党内
爆发激烈争论的基础,因为对国际经济的理解成为政策制定的决定性因素(参见以下第十五
章)。

毫无意外,建立在这一脆弱的基础之上的世界革命的方案,\textbf{不可能是有效的}。相比
之下,倒是1917年托洛茨基为俄国制定的策略得到了证明。但是,正如我们注意到的,这种
战略的成功是通过布尔什维克党获得的,托洛茨基并没有低估这一点的重要性。但是,他确
实声称列宁逐渐接受了不断革命论的有效性,并且在关键时刻使它成为了布尔什维克的信条。
这未免有点夸张,因为\textbf{列宁更大程度上是受到了布哈林的影响,布哈林运用不同于
  托洛茨基的理论,得出了与他相一致的结论。}在转向十月革命之前列宁观点的重大转变之
前,我们在以下一章,先探讨布哈林的著作。


\chapter{帝国主义和战争:布哈林和列宁论垄断资本主义,1914~1917}

\section{战争对俄国马克思主义的影响}

1914年8月第一次世界大战的爆发,对列宁关于资本主义的分析,并没有立即带来根本性的影
响,他也许认为俄国的落后阻塞了除民主革命外的任何其他变革。尽管在列宁战争年代的著
述中,资本主义作为一种世界体系的本质表现得更为突出,但写于1916年的《帝国主义论》
中的大部分原理,可以在他战前的著述中找到,直到1917年开头几个月,他仍然热衷于
对“\textbf{民主专政}”的分析。同样,普列汉诺夫和孟尔什维克仍然致力于对战前经济的
分析,坚持要\textbf{同自由资产阶级建立同盟}。只有托洛茨基对他的不断革命论进行了重
大调整,但正如我们在以上第十二章中看到的那样,这一变化强化而非弱化了他先前的结
论。

然而,战争并没有改变政治路线。大部分孟什维克,包括普列汉诺夫在内,都支持民族防御。
只有由马尔托夫领导的孟什维克国际主义者和托洛茨基的支持者们,完全否定通过“防御主
义”把马克思主义者都调和到和平纲领的基础上。这使得他们更接近于许多布尔什维克,但
列宁的立场更为极端。\textbf{列宁采纳的是一种“革命失败主义”立场,旨在将帝国主义
  的冲突转变为因军事失败引起的一系列内战。}与此同时,列宁痛斥了竭力相信保卫“大
国”的人,\textbf{对“第二国际”作了谴责,强调建立一个新的忠于革命和革命战争的社
  会主义者组织的必要性。}列宁将国际社会主义的破裂,视为类似于俄国1903年的分裂事件,
但这一次他很快采取了一种不妥协的立场,\textbf{将马克思主义作为一个整体而不只是它
  在俄国的分支。}在这方面,他常常与其他激进分子包括某些布尔什维克发生争执。

在整个战争年代,列宁一直信奉这些原则,而且随着战争的推延,他不断地修改他的经
济学和革命策略。虽然列宁对战争开始时的敌对行为感到诧异,后来他有时也强调这种对立
的政治原因,但他很快就将战争视为他在1914年以前已观察到的基本经济变化的反映。战争
既不是偶然发生的,也不只具有短暂的重要意义,相反,\textbf{它标志着国际资本主义发
展的一个最新阶段的到来,开辟了欧洲无产阶级革命的新时代。}这些问题,不再像在1914
年之前那样,被当作是次要的主题;\textbf{俄国社会的具体问题,不再是列宁思考的主题。}

最初,列宁将希法亭对金融资本的分析(以上第五章作了概括),视为理解这个新阶段
的关键,尽管他认为《金融资本》的作者并没有意识到这个新阶段可能造成的革命性的后果。
但随着时间的推移,列宁开始修改他的经济学,用一种\textbf{更为激进}的方式解释其政
治意义。毫无疑问,在这里,尼古拉·布哈林成为列宁理论上的促进因素,布哈林在第一次
世界大战前就已经对一种全新的政治经济学作了阐述,这种经济学认为,\textbf{俄国唯一
可能发生的革命就是社会主义革命。}到1915年时,布哈林对他的理论基础作了完善,在随
后的两年间,列宁将这些理论吸收到自己的分析中。他并非不加批判地照搬,他对布哈林的
著作始终有些怀疑,到1917年时,他有关适合于俄国革命的形式及其可能的后果的观点发生
了质的飞跃,这在很大程度上受到了布哈林的影响。

\section{布哈林论世界经济和帝国主义国家}

战争使布哈林达到对更高的上层建筑的分析。他将边际主义经济学理论,特别是将奥地利学
派的经济学理论,视为源于金融资本所有制关系变化产生的“\textbf{有闲阶级}”的意识形
态,他试图揭露其自相矛盾的肤浅的推理。他对边际主义方法论和实质的批判追随了希法亭、
博特凯维兹和帕尔乌斯(参见以上第三章),但是他对新古典主义意识形态性质的说明是独
创性的,清晰地与\textbf{资本集中相联系的所有权与控制权分离}的观点相吻合。产生
于19世纪70年代的新奥地利经济学是\textbf{个人主义、主观主义}的,并以有利于布哈林认
为的\textbf{“食利者”或寄生性的持股阶级的消费}为导向。即便如此,布哈林未能克服所
有意识形态分析的根本的困难,也就是说,在对物质基础的分析上,他没有提供比其他人的
分析更为准确的判断标准。布哈林\textbf{忽视}了其它形式的边际主义,尤其是\textbf{关
  注生产和分配的美国边际主义}提出的问题。这个问题可能确实只具有第二位的重要性。马
克思认为的资产阶级经济学理论的重大转变,发生在19世纪30年代而不是19世纪晚期,这是
站不住脚的,因为新古典主义分析仍然在发展新的形式。

布哈林仍然坚持其观点,但是战争促使他对\textbf{资本主义经济体系}重新作出思考,而不
是继续对资产阶级意识的研究。他主要的仍然是依靠希法亭的《金融资本》,尽管在强调的
内容上发生了重大转变。大部分内容写于1915年的《帝国主义和世界经济》,把现代资本主
义概念化为一种\textbf{世界体系}。\textbf{希法亭集中关注的是先进国家的经济结
  构。}布哈林追随罗莎·卢森堡——虽然不是她的消费不足经济学——\textbf{将所有国家的经
  济看作是世界市场的构成部分,它们都遵循世界市场规律。}这些规律正是马克思在《资本
论》中分析的资本主义商品生产的规律。“大国”都已经成为“有组织”的实体中的一
员,\textbf{在这个实体内,价值规律是无效的。“生产的无政府状态”已经转移到世界经
  济中,马克思的范畴对此仍然是适用的。它们影响了各个国家的个别经济,但是通过国际
  关系进行的。}与希法亭不同的是,布哈林认为,\textbf{危机的纯粹的内部基础,已经
  因“国家资本主义”内“组织化”程度的进一步提高而消除了。}

这意味着经济单位被进一步政治化了;它们的分离变成了一个\textbf{国家边界问题},它们
的\textbf{竞争是通过国家体系进行调节的。}这种竞争是不可避免的。像托洛茨基一样(参
见以上第十二章),布哈林相信\textbf{生产力的发展已经超越了在任何一个民族国家内部
  可以有效运行的界限。}现代资本主义企业,因为大规模生产的需要和利润率的下降被
迫\textbf{国际化},这两者都反映了资本有机构成的不断提高。资本有机构成的提高采取了
不同的形式,其中包括帝国主义的兼并。同时还包括一国内的资本的联合,在这个过程中,
资本的集中和积聚产生了垄断:随着金融资本出现发展起来的“\textbf{国家资本主义托拉
  斯}”,甚至出现了更为纯粹的国家资本主义形式。世界市场的参与主体和国家体系保持一
致,\textbf{竞争在国家之间以对抗的形式出现}。所以,尽管布哈林可能赞同克劳塞维茨战
争“是政治通过另一种手段的继续”的观点,但是他会补充说,政治本身已融入经济之中。

在布哈林看来,\textbf{同时作用于“民族化”和“国际化”的资本的两种力量,确定了现
  代资本主义的主要矛盾。}它们证明了布哈林把希法亭阐述的现象综合进来的观点。在《帝
国主义和世界经济》一书中,布哈林在学术上受惠于希法亭是显而易见的,正是在对希法亭
思想的重新构思中,布哈林形成了他自己三大重要贡献中的第一个。他的第二个独创性的观
点在于,\textbf{主张资本的国家集中化已经超越了金融资本,并形成一系列“新的利维
  坦”或准极权主义的国家资本主义。战争本身就是新结构的产物,同时加速了新结构的成
  熟。}每个民族国家的资产阶级,代表了一种具有新性质的\textbf{统一体}。议会已经成
为一种时代错误,因为已不再迫切需要一个平台来调和资产阶级不同集团的部门利益。同样
的原因,布哈林认为,\textbf{自由主义的自由已经不复存在。}因此,\textbf{资产阶级的
  革命不再重要,民主的目标不可能在资本主义条件下实现,通向社会主义的议会道路被关
  闭了。}

此外,在布哈林看来,所有的国内机构都处在\textbf{国家政权的管理}之下,包括有组织的
劳工和落后的农业资本主义。对于前者来说,\textbf{工人阶级领袖通常只愿意被整合到国
  家之中,因为他们把国家权力的增长等同于向社会主义的进步。他们用民族国家资本主义
  的理由取代了革命的马克思主义,}布哈林坚持认为,第二国际的崩溃是一个必然的结果。
这个过程有其自身特定的物质基础,因为日益扩大的垄断资本的利润被用来为“工人贵
族”和官僚政治官员的特权提供资金。然而,机会主义者的日子已经屈指可数。\textbf{世
  界资本主义固有的矛盾最先带来的只能是一系列的世界战争。}不断增加的人民大众的苦难,
最终将打破改良主义者的幻想并导致革命。在布哈林看来,\textbf{危机变成了战争的同义
  语}。他所理解的同卢森堡和托洛茨基一样,唯一的选择就是:\textbf{社会主义或野
  蛮}。

资本的“民族化”和“国际化”这两种并行不悖的趋势,为这两种可能奠定了基础。布哈林
甚至比希法亭还肯定地认为\textbf{战争是必然的}:与其说它是由环境决定的可以实施也可
以放弃的“政策”,不如说它是现代资本主义本质结构的必然结果。然而,他同时且再一次
比希法亭更加肯定地认为,\textbf{国家资本主义关系的“组织化”,已经为社会主义经济
  计划打下了基础,这在全球范围内都是如此。}纯粹的国家层面的经济发展指标,与评价社
会主义的可能性无关,\textbf{资本主义是一种涵盖了全球的世界经济。但是由于它分成了
  资本主义国家单位,不存在实现进一步“组织化”的可能。}在考茨基的超帝国主义概念
中,\textbf{先进资本主义国家会协调他们的帝国主义剥削}(以上第六章作了概述)。布哈
林承认,这一点在理论上是可能的,但在实际中是无法实现的,因为国际资本主义不同部分
的经济发展状况的\textbf{多样性},使这种协定行不通。特别是,具有更为发达的生产力和
较低的生产成本的生产者,或者有更强大的国家力量受其影响的生产者,他们\textbf{缺乏
  长期遵守国际卡特尔规定的兴趣。}从而,从实践上看,考茨基的超帝国主义是无法实现
的。

列宁用类似的观点对考茨基作了驳斥(参见以下第4节),而且驳斥得更加有力。布哈林理论
的一个普遍特征在于,\textbf{低估了世界经济不同部分经济发展的不平衡}(参见以下
第3节)。相比之下,列宁关于帝国主义战争理论,主要建立在国家资本主义发展不平衡基础
之上。因此,布哈林对考茨基的“超帝国主义”的批评具有矫揉造作的特征。他认
为,\textbf{只有无产阶级的国际主义和社会主义才能为生产力的进一步发展奠定基础。}孤
立地看,布哈林对考茨基的批评是很合理的,但作为他自己的帝国主义理论的一部分,这种
批评与他的经济学的一般特征是相冲突的。因此,列宁对超帝国主义的分析虽然缺少独创性,
但列宁的批评与他有关帝国主义的总体见解是一致的,布哈林的观点不具备这种一致性。

布哈林对马克思主义作出的第三个原创性发展在于,\textbf{他认识到资本主义体系的特定
  本质决定了它被推翻的形式。}在这个问题上,他完全不同于希法亭,后者认为简单地接管
现有的国家机器作为无产阶级权力工具,毫无疑问是可行的。在布哈林看来,这是错误
的“伯恩施坦主义”,它已经传染给了德国社会民主党(参见以上第四章和以下第十四章)。
由于\textbf{国家组织已经与资本主义经济融合在一起},所以现有的国家必然无法满足新生
产方式的需要。\textbf{“帝国主义国家”是一种只适合于资本主义最新阶段的特定的历史
  形式。相反,无产阶级不是控制而是必须打碎所有现存的政治形式,并通过建立适合于新
  社会的结构实施自己的专政。}

在这里,布哈林恢复了马克思对国家的敌意的观点,并提供一种经济学的论证,在这种经济
学中,能够发现与潘涅库克和其他“左翼共产主义者”所坚持的类似的政治学观点。布哈林
这样做,对区分改良和革命提供了一种清晰的要素。同时,\textbf{他含蓄地否认了“经济
  基础”和“上层建筑”之间存在任何直接的因果关系。在国家资本主义
  下,“政治”是“经济”不可分割的一部分,反之亦然。}在资本主义世界市场中发挥作用
的、有组织的民族国家的“主体”是帝国主义国家,世界市场的关系与国家体系的关系几乎
相同。


\section{布哈林、托洛茨基和列宁}

正如在以上第十二章看到的,托洛茨基关于俄国资本主义发展的理论包含了类似的看
法:\textbf{历史唯物主义需要作重大的修正}。托洛茨基关于\textbf{帝国主义理论的结
  论},撇开他的\textbf{消费不足论}的特征,是可以与布哈林的结论\textbf{相容}的。除
此之外,这两位理论家的观点是大相径庭的。布哈林强调\textbf{现代资本主义的纯粹性},
他没有托洛茨基理解的那种“\textbf{综合发展}”的概念。除了在同考茨基的争论中,布哈
林\textbf{没有强调不平衡发展}。他当然知道世界经济中的某些国家比其他国家先进,国家
内部的经济条件也不尽相同。但是,他不认为这些差异在解释问题上有什么价值。相反,
他\textbf{强调帝国主义对现代化的影响,认为大城市中心的组织化削弱了经济落后的重要
  意义。}这样,原始农业和先进工业通过帝国主义吞并和有组织的国家资本主义结
构\textbf{融合}起来。布哈林认为,\textbf{当前资本主义的矛盾源于它的现代性,而不是
  它的不完善和不完全发展。}从而俄国资本主义的“特殊性”,在他的信念——反对沙皇专制
的革命必然成为社会主义革命——中,起不到任何作用。

列宁的立场与布哈林的立场之间的关系较为复杂,不只在于这种关系总在变化。列宁在三个
基本问题上,是赞同布哈林观点的。列宁认为,战争源于现代资本主义的本质;资本主义已
经达到其“最高”阶段,并产生了能在先进的西方国家中最终实现社会主义的革命形势;第
二国际中出现的机会主义并非偶然,而是源自帝国主义的本质。尽管如此,在战争开头两年,
列宁同样持保留态度,他们都关注了布哈林和希法亭不同的地方。这清晰地体现在列宁对布
哈林《帝国主义和世界经济》所作的序言中,特别体现在列宁第二年写的《帝国主义是资本
主义的最高阶段》中。在后一本著作中,列宁强调希法亭和霍布森的重要性,声称关于帝国
主义问题的论述,“恐怕都没有超出这两位作者所阐述的。

毫无疑问,列宁认为布哈林描绘了一副夸张而又存在过度简化的现代资本主义的图景。至少
到1916年中期,列宁才接受用“\textbf{国家资本主义}”描述大都会城市中心的特征,他从
来不认为垄断消除了单个资本主义经济的内部矛盾。事实上,他坚持相反的看法:因
为\textbf{垄断只是部分程度上的垄断,垄断部门和竞争部门的利益冲突仍然存在,一旦发
  生危机,垄断就阻碍了市场机制达到一个新的均衡。}此外,列宁认为,国家资本主义
的\textbf{不平衡发展}是至关重要的。\textbf{正是俄国整体上的落后,阻碍了社会主义革
  命的发生,而不同的单个资本主义经济相对地位的变化是战争爆发的原因。}在列宁看来,
世界被“大国”完全瓜分和资本主义不平衡发展,成为军事对抗的经济基础。而不是像布哈
林认为的那样,\textbf{战争是国家内经济消除了竞争,并把它完全转移到国家之间的相互
  关系中造成的结果。}

另外,列宁认为,布哈林的国家理论接近于\textbf{无政府主义}。同时,他的无产阶级专政
的观点与考茨基的非常接近,考茨基\textbf{将社会主义无产阶级专政的目标看作是对现存
  国家机器的完全控制}。1917年前,列宁也否认社会主义革命在俄国爆发的可能性,他严厉
地批评了布哈林更为一般性的观点,这种观点认为民主问题——尤其是被压迫民族的民族自决
权问题——由于经济的发展变成了多余的。此外,他反对布哈林将\textbf{农民}排除在无产阶
级联盟之外,同时认为布哈林在一些次要问题上也有失误。

因此,尽管他们相互尊重,但列宁和布哈林的关系在战争期间是很不融洽的。在1916年末
和1917年初,这两位理论家达成重要的和解,主要是\textbf{列宁转向了布哈林的观点}而非
相反的情况。这是非同寻常的事情:因为19世纪90年代之后,列宁很难在重要的理论问题上
认同其他人的观点,而且在这些年的早些时候,他的灵活性也只是昙花一现(参见以上第十
一章)。在1903年与孟什维克决裂后,列宁在布尔什维克的理论问题上也是极其独断的,他
强制性地驱逐了很多反对者。现在,在一些至关重要的观点上,他开始背离布尔什维克的理
论。在我们考察这一点之前,必须先研究一下列宁的《帝国主义是资本主义的最高阶段》,
这本著作写于列宁仍在与布哈林发生争执的1916年。

\section{列宁的《帝国主义论》和向布哈林的转向}

列宁的以“通俗的论述”为副标题的《帝国主义是资本主义的最高阶段》,没有什么原创性,
列宁仅仅是对一些重要的机构和关系进行了概述,他认为在当时这些机构和关系能够勾画
出“资本主义的最高阶段”的特征。很多研究成果是列宁关于帝国主义研究的基础,这个小
册子实际上只是列宁先前一些观点的系统化,先前的观点是以一种相对晦涩的方式表达出来
的,从而很难进行评价(参见以下第6节)。列宁主要参考了希法亭的《金融资本》,相对于
希法亭来说,英国自由主义者J.A.霍布森的著作也影响了列宁。但《帝国主义是资本主义的
最高阶段》中,只有一次顺便提及布哈林。列宁有关帝国主义的笔记,清楚地指出希法亭存
在不足之处,笔记显示,在当时,只在现\textbf{代资本主义独特的寄生性}这一观点上,列
宁受到布哈林的影响。然而,这反映的可能是《有闲阶级的经济理论》而不是《帝国主义和
世界经济》的影响。列宁的政治经济学的主要变化,实际上发生在《帝国主义是资本主义的
最高阶段》出版之后。但是这部著作在列宁逝世后取得了经典地位,并因此掩盖了在列宁思
想的演变过程中,它只具有第二位重要性的事实。

列宁把帝国主义和“垄断资本主义”(虽然有时候以一种更为狭义的方式使用这一术语)
视为同义语。从“纯粹的经济的概念”看,帝国主义有五个特征:(列宁专题文集-论资本
主义,2009年版,P176)
\begin{enumerate}
  \item 生产和资本的集中发展到这样高的程度,以致造成了在经济生活中起决定作用的垄
断组织。
  \item 银行资本和工业资本已经融合起来,在这个“金融资本的”基础上形成了金融寡头。
  \item 和商品输出不同的资本输出具有特别重要的意义。
  \item 瓜分世界的资本家国际垄断同盟已经形成。
  \item 最大资本主义大国已把世界上的领土瓜分完毕。帝国主义是发展到垄断组织和金融
资本的统治己经确立、资本输出具有突出意义、国际托拉斯开始瓜分世界、一些最大的资本
主义国家已把世界全部领土瓜分完毕这一阶段的资本主义。
\end{enumerate}

列宁文本的大部分内容,致力于对这一定义不同方面的经验证据的论述,对考茨基超帝国主
义理论的批评,以及对改良主义的根源在于帝国主义创造的“工人贵族”这一信念的阐述。

列宁认为,“资本主义的最高阶段”的出现,不早于20世纪初。然而,无论是对帝国主义同
竞争性资本主义的联系,还是对帝国主义不同构成部分的相互关系,列宁都没有予以详细的
说明,而对帝国主义根本特征背后的因果机制的分析,也往往是晦涩的。比如,在资本输出
这一重要问题上,列宁提到了大城市中心积累的“过剩资本”,但却没有解释为什么缺乏内
部投资机会。他以赞同的态度提到霍布森,提到他接受了长期消费不足的观点。但是,列宁
在19世纪80年代与民粹主义的争论中,已阐明了这种理论中存在的种种困难(参见以上第九
章)。在《帝国主义是资本主义的最高阶段》或其他战争时期的著述中,他没有说明他对自
己早期观点已经作了修改,而且希法亭也分析资本输出,但忽视了\textbf{消费不足}(参见
以上第五章)。希法亭和布哈林都求助于马克思的\textbf{利润率下降理论},但这一点在列
宁的分析中没有出现。

在列宁那里,相对清晰的地方是他对帝国主义后果的分析。尤其是他\textbf{强调世界被瓜
  分完毕和国家资本主义发展的不平衡,使得一再发生的重新瓜分世界的战争不可避免。}在
这一点上,他的立场比希法亭在《金融资本》中的立场更为坚定。在解释第一次世界大战爆
发的根源时,列宁坚持认为,20世纪早期英帝国和法帝国控制的殖民地的面积,同它们的经
济实力相比较要大得多。在过去的半个世纪,资本主义迅速发展的德国发现了自己的不
足。\textbf{为了获得充足的市场,同时也为了确保得到原材料和保护海外投资的领域,}德
国被迫诉诸军事侵略以便能重新瓜分其他国家的殖民地。无论战争的结果如何,列宁认
为,\textbf{在世界被瓜分完毕的背景下,持续的不平衡发展将动摇和平的基础。}

如本章上一节看到的那样,列宁赋予这些现象的重要意义,使得他对考茨基超帝国主义的批
评远比布哈林的批评有力。同布哈林和希法亭一样,\textbf{列宁认为资本的集中化是一种
  持续存在的力量,这种力量在理论上形成了发达资本主义国家联合剥削世界的可能性。}与
布哈林和希法亭一样,列宁也承认国际卡特尔的存在,这意味着考茨基的思想并不是没有依
据。但是,列宁认为,\textbf{任何超帝国主义的协定都是暂时的,}因为\textbf{世界经济
  不同部分的不平衡发展}必然会破坏它,并带来进一步的重新瓜分世界的战争。

布哈林和列宁都否定了考茨基关于经济领土的扩张仅限于\textbf{正式的农业区殖民地}的
观点。列宁有关资本输出目的地的资料表明,殖民地在资本输出中并没有占据重要的位置。
而且,列宁明确地承认,帝国主义的统治无需正式的政治控制,兼并不必仅限于农业区,或
仅限于欧洲大陆以外的地区。同时,他十分正确地注意到了这一问题的复杂性。他关注的中
心问题是\textbf{资本主义不平衡发展背景下的垄断,垄断的需要可以通过一些不涉及直接
的政治控制、对农业区的统治、对欧洲以外地区的殖民的方式来实现。}在其他地方,列宁
注意到,很多真正的帝国主义国家并不是纯粹的资本主义国家。有一段内容可能来自托洛茨
基,这段内容指出了\textbf{已经融入到帝国主义中的非资产阶级的利益的重要性}。这无
疑与列宁早期著述中对“普鲁士”式现代化道路的重要性的认识完全一致(参见以上第十一
章),

然而,在关于垄断和帝国主义扩张对欧洲工人运动影响的问题上,列宁的分析不如布哈林。
虽然他们思想实质是一致的,都源于\textbf{恩格斯}的思想(参见以上第一章),但列宁的
分析不如布哈林缜密,这不单纯是列宁用语粗糙的问题。由于没有能接受\textbf{布哈林帝
  国主义国家是资本主义先进国家中活跃的“主体”}的观点,列宁的“\textbf{经济主
  义}”更为明显。在列宁的帝国主义理论中,不存在“相对自治”程度比较高的国家进行的
斡旋,而根据布哈林的分析,某些国家的“相对自治”程度是相当高的。正如我们将要在
第6节中看到的那样,两种分析中没有哪种特别令人信服,但布哈林的分析要好一些。

《帝国主义是资本主义的最高阶段》完成后到1917年十月之间的著述表明,列宁至少在布哈
林著作具有的三个其它特征上,作出了自己的类似的判断。首先,\textbf{列宁明显地接受
  了国家垄断资本主义的发展的观点},即便是他没有对此进行详细的阐述;第
二,\textbf{他承认布哈林对无产阶级专政的分析是正确的},并对布哈林的分析进行了较大
程度的扩充;最后,\textbf{列宁开始认为俄国革命必须超越资产阶级民主的界限。}这造成
了他和他自己政党中大多数人的冲突,而且只有经过艰苦的努力,得到布哈林与托洛茨基的
后继者们的支持,他才能够战胜“老布尔什维克”的抵制。即使在那时,他仍然要面对由加
米涅夫领导的布尔什维克的右派的持续的负隅顽抗。

列宁观点转变的原因,部分地是由于他在战争开头两年所处的地位的矛盾造成的,而最根本
的原因是他\textbf{有关现代资本主义的观点同他的政治学之间存在矛盾}。列宁清楚地表明
了自己的信念,在俄国已经出现的资本主义最发达的形式是\textbf{寄生性的和反动的},它
们之间的战争威胁到文明本身。那么,\textbf{在什么意义上,巩固资本主义基础的革命,
  应当被视为进步的呢?}此外,他的“\textbf{革命失败主义}”的逻辑认为,现存政权的
军事失败将有助于革命。那么,无产阶级如何凭借已经开始崩溃的国家机器来确保自己的专
政呢?进一步讲,正如我们在以上第十一章所看到的,列宁设想的“民主专政”将采取一种
苏维埃结构,这意味着要摧毁沙皇专制的国家机器。那么,在无产阶级自己的社会主义革命
中,我们为什么将苏维埃从“打碎”现存的国家机器——无论它的阶级本质是什么,它本身都
不是苏维埃的形式——中排除掉呢?

因此,在1914年到1916年之间,列宁主义是一种\textbf{高度不稳定的混合物}。但是,第一
次世界大战中好战的政权也是如此,现实的发展使列宁意识到帝国主义的概念落后于历史的
发展。政府管制和中央计划的加强清楚地表明,布哈林理论中的极端主义至少指向了正确的
方向。列宁的著作开始使用“\textbf{国家资本主义}”这个术语,他认为这个术语适用于最
新阶段的资产阶级社会。与此同时,他认为即使是这些新的控制手段,也\textbf{无法维持
  物质再生产}。全面战争带来的沉重负担和对世界经济联系的破坏,使许多国家濒临破产。
列宁认为,这种危机在俄国特别明显,从1917年开始,他坚持认为,只有\textbf{社会主义
  革命}才能扭转向混乱状态急剧转变的局面。只有\textbf{资产阶级的民主措施}是不够
的:\textbf{国家资本主义的管理机器,必须完全脱离资本的统治并服务于人民群众。}

列宁相信,战争的进一步后果,是被剥削者在反对资本主义时变得越来越团结。不但俄国的
无产阶级,而且贫困的农民和被征服的民族,都逐渐\textbf{认识到未能制定出社会主义政
  策就停止的革命,无法满足他们最迫切的要求。}在列宁看来,\textbf{现代资本主义既能
  联合不同的财产形态,也能在被压迫阶级中建立新的团结。}在帝国主义时代,只有在无产
阶级的领导下,通过革命向社会主义过渡,大量的小资产阶级才能够实现他们自己的利益。

列宁思想中的这种巴枯宁主义主题,融合了当时另一种产生于纯粹的思想过程的无政府主义
思潮。由于布哈林的国家理论与无产阶级专政相联系,列宁对此既感到不安又充满敌意,他
开始认真地研究这个问题。研究的最终结果是《\textbf{国家与革命}》的发表,这部著作
写于十月革命前的几个月,但是这部著作中的结论是在1917年2月沙皇专制被推翻之前得出
的,这些结论是对布哈林的超越。列宁坚持认为,资产阶级国家不仅必须被“打碎”,而且
必须由类似于1871年的巴黎公社那样的大众参与的民主机构来取代它。\textbf{社会主义的
秩序必须通过国家资本主义从上层发展而来的管理机器和从下层产生的大众的民主机构的联
盟来实现。}

列宁刚突破他先前理解的正统思想,他的新思想就得到明显的经验证明。苏维埃和工厂委
员会在1917年雨后春笋般地涌现出来,将俄国置于欧洲革命的前沿。列宁当时认为,首先在
俄国出现的社会主义能够引发世界革命。作为无产阶级力量的具体化身,作为新的国际主义
的大本营,通过\textbf{集结革命战争}的手段,俄国革命能够扩大到国际范围。这将使一
切都变得更容易,因为这样的革命将剥夺英国和法国的重要剥削来源,打破战时同盟。

从而,布哈林政治经济学的重要因素被吸收到列宁主义中。然而,这种结合是不完全的,在
重要的理论问题上,两个人之间仍然存在分歧。列宁的观点比布哈林的观点更为具体。当他
开始接受适用于社会主义革命的“最高纲领”是当时最强大的法宝时,他并没有像布哈林那
样,放弃\textbf{与民主革命相关的“最低纲领”}。相反,列宁认为后者仍然具有重要的
意义,但这时他坚持,“最低纲领”只能通过社会主义革命才能实现。正如我们在以上第十
二章看到的,这也是托洛茨基不断革命论的核心思想。那么,列宁是在布哈林的影响下,成
了另一个托洛茨基主义者吗?


\section{列宁和“不断革命论”}

列宁的观点,在1914年后\textbf{更接近于托洛茨基的观点},如我们在以上第十一章看到的,
他这样做时,已经是从一个很接于托洛茨基的立场出发。\textbf{“普鲁士道路”把前资本
  主义形式和资本主义形式整合在一起,因此“综合发展”的概念已经隐含其中。}虽然把自
己限制在有利于资本主义发展的各种措施中,但“\textbf{民主专政}”被认为能够与反对资
产阶级的严厉措施相兼容。1917年,列宁走得更远,他进一步主张,任何有效的民主变革,
都必须辅之以不断革命论一直坚持的社会主义建设。在这个基础上,列宁增加了\textbf{无
  产阶级专政理论},这种专政理论似乎否定了他自己的\textbf{政党学说},而政党学说正
是托洛茨基以前反对布尔什维主义时集中批判的内容。革命本身是有国际目标的,但最初它
必须只限于采取特定的集体主义手段,不包含\textbf{直接的彻底的向社会主义转变},托洛
茨基认为这是不可能的。

毫不奇怪,列宁战争期间的著作中,有大量的段落带有不断革命论的味道。当然,很多
布尔什维克对列宁回到俄国后第一次表明他的新立场的“\textbf{四月提纲}”的反应,就
是指责列宁的托洛茨基主义。托洛茨基在1917年加入布尔什维克党后,表明了自己的观点,
即他与列宁之间存在的分歧是次要的。直到1940年去世,托洛茨基一直坚持这一立场,其他
很多人也赞同他的观点。列宁在1919年之前没有读过《结论与展望》的事实,可以被用来为
他对托洛茨基的学说存在明显的误解进行辩解。

尽管如此,列宁和托洛茨基之间仍然存在理论上的差异,其中有些差异在革命之后成为
他们之间发生冲突的主要原因。最明显的是,\textbf{列宁1917年的著作中表达的“两阶
段”革命的观点,并不同于托洛茨基两个革命相嵌在一起的概念。另外,列宁并不认为俄国
的社会主义革命是不可避免的。}整个1917年,他都认为通过军事专政落实 “普鲁士”式的
解决方法,仍然是极有可能的。真正将两位理论家区分开来的,是两者对\textbf{革命的原
因及其可能造成的后果问题}存在不同看法。正如我们已看到的,列宁通过对发达资本主义
的理解,而不是通过以托洛茨基的方式重新解释俄国的“落后”,\textbf{开始接受俄国社
会主义革命是必然的}。在列宁看来,认识到俄国社会形态内部的“综合发展”,无助于解
释社会主义革命会成功的原因,“不平衡发展”的因果作用也仅限于国际冲突中。列宁的策
略转变,包含了与俄国的内部条件相关的观点,但是它们与俄国工业化的“特殊性”无关,
而与战争造成的影响息息相关。

与托洛茨基一样,列宁为俄国社会主义革命提供了一个国际主义的理由,即\textbf{如
果它不能在世界范围内扩大,它就可能失败。}然而,在这一点上,列宁与托洛茨基存在一
个重大的区别:列宁逐渐认为,大部分农民在向社会主义过渡时会保持合作,因为
\textbf{现代资本主义造成了所有被压迫者的新的团结,为自治和经济计划提供了手段,}
即使在俄国落后的条件下也是如此。但是这也意味着俄国革命面临的主要威胁来自外部,不
存在不可逾越的内部制约因素。在这里,与托洛茨基存在的观点上的分歧,支配了1924年列
宁去世后有关经济政策的争论(参见以下第十五章)。列宁将世界革命看作一个长期的进程,
其中有挫折也有进步,而且在不同的国家显示出很大的差异。1917年,列宁尽管坚信俄国革
命不会在很长时间内孤立无援但他有时又认为社会主义革命在相当长的时间内可能被继续孤
立。

正如苏联的编辑们主张的,所有这些都不能证明列宁是“一国社会主义”理论家,但他的著
作的确很容易作出这样的解释,托洛茨基的著作却不具有这一特征。20世纪20年代时,布哈
林和斯大林提出,列宁对“一国社会主义”持支持态度,但他们并不像托洛茨基那么固执
(参见以下第十五章)。列宁使用了国际主义术语证明十月革命的正当性,但是\textbf{他
  也诉诸于国内形势,表明俄国可以在没有外部援助和根本性的内部冲突的情况下向社会主
  义前进。}

\section{对布尔什维克政治经济的评价}

列宁对1917年俄国实际形势的评价极其准确。二月革命呈现出一种旨在加强“防御主义”的
“沙文主义革命”的特征,他早就认为这是可能的;孟什维克的调和主义政策证明,把他们
看作\textbf{机会主义者}是正确的;而且由于群众及他们在政府中的领导人和苏维埃有不
同的目标,如列宁预计的国内危机进一步加剧。在这种情形下,列宁主张,唯一的选择要么
是\textbf{右翼的军事专政},要么是为“\textbf{和平、面包和土地}”进行进一步的革命。
由于难以和其他社会主义政党进行合作,革命只能在布尔什维克的领导下继续推进(参见以
上第八章)。

早在1914年,列宁就提出,理解国际关系的宽广视角具有其合理的内核。正如以上第十一章
所注意到的,存在着制约国际“秩序”形成的结构性难题;“大国”有其帝国主义目标,由
此导致战争为革命性的变革提供了沃土。自1914年以来的30年,西方资本主义经历了一系列
资本主义历史上最严重的危机。布尔什维克的理论家们正确地找到了危机的根
源,\textbf{危机的根源在于世界经济的本质,世界经济不但融合为一体,而且发生十分危
  险的分裂。}

然而,尽管布尔什维克的结论基本上符合实情,但他们得出结论时使用的经济理论不太可靠。
列宁在《帝国主义是资本主义的最高阶段》中表达的观点,招致了蜂拥而至的批评。考虑到
列宁借鉴了希法亭的观点,另外他的结论与布哈林和托洛茨基的有类似之处,所以,这些负
面评价大部分也适用于他们的作品。当然,这四个理论家都存在重要的缺陷,但是,在许多
批评者提出的所谓“缺陷”中,很多实际上是错误的。对列宁的批评尤为如此,批评家们往
往要么没有将他有关帝国主义的著述视为一个整体来考察,要么未能对这些著作中存有歧义
的地方给予以足够的重视。与这些错误联系在一起的,通常是规范的学术实践被中止了,这
种中止在评价一个存在较大争议的人物时是不能被宽恕的。

由于对“政治学”的相对独立性、对民族主义没有给予以足够的重视,有人认为列宁在理解
国际关系时的分析,是一种粗糙的经济主义。此外,批评者还认为,19世纪后期,殖民化和
资本输出是由不同国家各自的武力决定的。经济动机在领土兼并中往往不重要,而且对殖民
地的资本输出落后于对其他地区的资本输出。还有人认为,消费不足不是资本国际化的推动
力,垄断也不是,因为英国作为最庞大的帝国,它的垄断最不发达。最后,有些批评者宣称,
根据列宁的观点,20世纪后半叶的去殖民化,对西方资本主义的发展应当是灾难性的。

所有这些批评,都没有对列宁实际阐述的内容给予足够重视,相反,将焦点集中在反驳被建
构出来的对列宁思想的刻板印象上。列宁对\textbf{国家垄断资本主义}的接受,必然地证明
了他在《帝国主义论》中表达的观点。尤其是,这意味着政治学和经济学在他的视角中是融
合在一起的,这同布哈林是一样的。即便如此,列宁坚持认为,“政治”因素在理解帝国主
义现象时是十分重要的,自第一次世界大战爆发以来列宁一直这么认为。如果有什么不同的
话,倒是托洛茨基对当时事件的分析,由于忽略了与“政治”相关的经济方面的问题而显得
不足。列宁认为,第一次世界大战是一场帝国主义战争,但是他从没有否认被压迫族群
的“民族主义”战争的重要性;他的确认识到,1914年战争的直接原因在于塞尔维亚的民族
主义。 对布哈林\textbf{低估了被征服民族的民族情绪重要性}的指责,可能是有充分根据
的,但是,这是一种列宁本人也多次作过的批评。布哈林和列宁都指出了\textbf{自由—民主
  规范,在先进资本主义地区的减弱,而重新强调了前资本主义价值的重要性。}

列宁和布哈林都没有提出帝国主义兼并的一般理论,更不用说国际关系的\textbf{一般
理论}。他们关注的是他们各自理解的资本主义特定的历史阶段,列宁明确地将其追溯至20
世纪初。由于认识到“帝国”在历史上无处不在,他们的理论中丝毫没有提到19世纪的殖民
化能够单独用“资本主义的最高阶段”的经济动机来解释。\textbf{资本主义的运动规律尽
管不是很清楚,但无论如何还是在于资本的垄断化,而不在于工人阶级的消费不足,后一点
是霍布森(而不是列宁)强调的。}

列宁和布哈林都认识到,\textbf{垄断资本主义寻求的控制,并不总是需要通过正式的
殖民化来实现。主要的资本主义国家的兼并目标和他们对国外的投资,也并不只限于欧洲以
外的地区。}列宁批评考茨基和卢森堡将帝国主义的扩张等同于\textbf{对外围地区的殖民
化}。他专心于解释已经被“瓜分”完毕的世界可能产生的后果,而没有对世界是怎样被瓜
分的进行说明。列宁坚持认为,导致不稳定的主要因素,在于新兴资本主义国家的需要而不
是那些老牌资本主义国家的需要,老牌资本主义国家的殖民历史牵涉到的力量,不同于在资
本主义的“最高阶段”占支配地位的力量。在这一点上,他与布哈林有极大的不同,但是布
哈林强调了现代资本主义下发展起来的新的、特定历史时期的竞争形式。托洛茨基甚至没有
从事明确的经济学研究;他强调生产力的发展无法再局限于民族国家的范围之内(参见以上
第十二章)。这三个理论家视角的共同之处,在于他们都具有希法亭著作中有的欧洲中心论
的特征。发达资本主义国家的经济结构,它们与欧洲战争的联系,以及在列宁看来的欧洲民
族运动在革命中的重要意义,都应该加以强调。直到20世纪20年代末,俄国马克思主义者才
开始发展以强调外围为特征的后来的帝国主义理论。

列宁和布哈林的现代资本主义理论,否认殖民地对城市中心的生存是绝对必要
的。\textbf{大国需要对全世界的开发区进行垄断控制;如果去殖民化没有威胁到这种控制,
  就没有理由期盼资本主义的消亡。}而且正如已注意到的,列宁和布哈林都认识
到\textbf{经济控制是比正式的殖民统治更具一般性的范畴。}

所有这些虽然都反映出不能轻易地接受对列宁政治经济学的批评,但它们也指出了列宁的理
论本身存在的局限。列宁关注的帝国主义的各种因素,从理论上看仅仅是被\textbf{松散地}联
系在一起。明显的错误几乎没有,通常也很难找到错误的陈述。这一特征对托洛茨基而言更
为真实,他的帝国主义经济学从未以任何形式被详细地阐述过。布哈林更是如此,但是正如
我们已经在本章第2节看到的那样,\textbf{这是以忽视经济发展的复杂性和夸大经济管理
的发达形式为代价的},列宁对这种缺陷作过批评。

尽管如此,依然可以对列宁和布哈林对现代资本主义的分析提出具体的异议。最明显的是,
他们对垄断资本主义的经济分析是极其肤浅的。布哈林的分析事实上是自相矛盾
的:\textbf{他相信世界市场使工资、价格和利润率发生均等化,与此同时声称,民族国家
  经济中的竞争已经被消除。}列宁没有能将帝国主义的不同特征清晰地联系起来,对资本输
出背后力量的分析尤其模糊。他们都没有考虑在把垄断与超额利润联系起来时明显存在的问
题。他们从来没有问过,在给定的工资水平和技术变动率下,垄断是如何提高总的剩余价值
的,更别说回答这一问题了。这一缺陷在布哈林那里表现的更为明显,他与列宁不
同,\textbf{将国家垄断视为完全的垄断,缺乏对利润从竞争性部门转出的理解。}简言之,
无论是列宁还是布哈林,都没有提出一个适用于垄断资本主义的价格和工资理论。

这削弱了他们有关“\textbf{工人贵族}”的理论,这种理论认为“工人贵族”是\textbf{工
  人阶级改良主义}在经济方面的中坚力量。关于帝国主义剥削和议会参与有利于工人阶级融
入资本主义的这种一般性观点,并不存在问题。\textbf{问题在于断言只有少数工人运动受
  到这种融入过程的深刻影响,对这种融入的机制也没有能够详细说明;假设无产阶级群众
  革命的完整性没有发生变化。}事实上,列宁在分析俄国的现状时是自相矛盾的,他坚持认
为高工资的钢铁工人最具抗争性,贫穷的纺织技工是最不具革命性的。而且他从来没有解释
过“工人贵族”的主题是如何与孟什维克的流亡者联系起来的,孟什维克的流亡者和德国工
人运动中的官僚主义者持类似的防御主义立场,他也从未解释过这一理论是如何与他早期的,
与此不同的有关无产阶级的调和主义的言论联系起来。


存在的问题还有,布尔什维克认为先进的资本主义带来的政治上的可能性的思想是存在矛盾
的。\textbf{无论是布哈林还是列宁,都无法令人信服地解释垄断资本主义条件下实施的集
  权式的经济控制和无产阶级国家的分权概念之间的联系。}“打碎”现有国家机器的观点具
有理论和现实意义,但由直接的民主机构取代现有的国家机器和利用垄断资本主义发展出来
的\textbf{等级控制机制},都是存在困难的。用官僚手段管理经济可能会受到地方选举机构
的监督,但是\textbf{把管理从属于大众的控制完全是另一回事,因为它与官僚组织的本质
  相冲突。}类似的矛盾也出现在1917年列宁对政党进行的分析中。后来,像1905年那样,列
宁抛弃了《怎么办》中的模式,并且向在布尔什维克纲领后摇摆的革命群众\textbf{敞开了
  党的大门}。如何使“先锋主义”与这一实践活动相一致,列宁的任何著述中都没有正视这
一问题。

列宁也没有正视在革命成功后建立的政治体制中,农村经济如何与城市中心的经
济\textbf{相结合}的问题。在打碎现有的国家政权后,在消灭了地主和榨取农村的剩余价值
并最终养活了城市的金融资本家后,列宁需要找到替代物。他认识到俄国的工业处于一种行
将崩溃的状态,而且农村的生产者将会在土地革命的过程中实现对农业的经济控制,所以问
题无疑是明显的。然而,总的来说,他太急于解释农民因客观条件产生的需要,他相信这种
需要已经为所有被压迫的人奠定了新的利益共同体的基础。

托洛茨基至少一直认为,\textbf{无产阶级和农民之间存在着时代的差异}。此外,他的不
平衡和综合发展概念,指出了布哈林和列宁的经济理论中存在的重要缺陷。同希法亭一样,
他们把当时德国的资本主义作为他们分析的出发点。但德国“独特”的历史产生了一些在其
他地方无法被复制的新特征,包括\textbf{工业资本和银行资本在一定基础上的联合},这
种基础在布尔什维克理论家看来,是社会主义的经济管理可以依靠的(参见以上第五章和第
十二章)。列宁注意到,德国资本主义有着独特的特征,但他错误地以为,这些特征是更为
“先进”的属性、而\textbf{不具特定国家的属性}。布哈林和列宁都认为,战时紧急状态
下产生的集权式的经济控制,仅仅是在所有条件下都在有力地发挥作用的长期趋势的加速发
展。因此,\textbf{他们过分地强调了国家资本主义普遍的重要性和持久性。}

从他们的视角看,发生在先进资本主义国家间的战争是\textbf{地方性}的,这种特定的缺陷
是无关紧要的,因为任何和平时期国家经济作用的下降都不过是短暂的插曲,当战争再次爆
发时则相反。从布哈林和列宁对资本主义潜在适应性的正常感觉来看,这是另一个更为重要
的出发点。像托洛茨基一样(参见以上第十二章),他们偶尔也关注“\textbf{超帝国主
  义}”的可能性:单个资本主义国家的军事支配改变了国际关系的性质;打破了其他的殖民
帝国,从而\textbf{削弱了垄断影响的领域};所有这一切都是为了自己的利益。但他们并没
有认真地对待这一点,而是继续强调一再发生的帝国主义战争的必然性。这样,他们就无法
预测20世纪后半叶美国霸权下主要资本主义国家之间的长期和平。当然,这本身也有可能只
是暂时性的。民族资本主义在继续不平衡地发展,随着美国相对经济实力的下降和扮演全球
警察角色的成本的上升,美国的\textbf{支配地位被逐渐削弱}。只有在这种非常有限的意义
上,列宁的观点才能被证明是经的住考验的。

\part{社会民主主义和共产主义:1917~1929}

\chapter{修正主义的复兴}

\section{引言}

1914年8月第一次世界大战的爆发,打破了本已岌岌可危的欧洲社会主义运动的统一。所
有重要交战国的马克思主义政党中的大多数人,都或多或少地热情支持他们各自国家的战争
动员。比如,奥托·鲍威尔和鲁道夫·希法亭,理所当然地参加了奥地利军队,甚至没有考虑
其他任何可供选择的行动。在德国,社会民主党右翼的爱国热情,产生了令人震惊的沙文主
义诡辩,把霍亨索伦帝国的军事胜利视为迈向社会主义道路的必经之路。最初,几乎没有反
对的声音,只有以罗莎·卢森堡和卡尔·李卜克内西为核心的一个较小的、坚定的国际主义者
团体,支持列宁和布尔什维克的革命立场。但是,随着战争的拖延,和平主义趋势出现了,
和平主义者反对好战的德国,但却不愿公开呼吁推翻它。卡尔·考茨基和爱德华·伯恩施坦,
作为对手已经很久了,在支持中间立场时他们发现自己再次团结起来。

这个社会民主政府镇压斯巴达克派起义,纵容1919年1月对李卜克内西和卢森堡实施的谋杀,
并使战后德国资本主义不受威胁。1918年德国共产党(KPD)成立后,区分布尔什维克俄国
的支持者和反对者的鸿沟,变得更加难以逾越了。现在,中间派的立场站不住脚了,1923年
独立社会主义党(德国独立社会民主党USPD)——考茨基和希法亭是其领导人——作为“第二半
国际”的中坚力量,开始和大多数组织联合。社会民主主义或共产主义战线已经确立,德国
工人阶级运动开始在改良主义者和属于第三国际革命者之间发生分裂。

在理论上,马克思主义也必然获得相应的发展。早在1914年前,政治争论就蔓延到经济争论
中,对资本主义发展存在的冲突的分析,被用于为互相竞争的政治策略辩护。一方面苏联的
资源增加了无产阶级革命成功的巨大的道德权威,产生了日益僵化(尽管是不断变化的)的
正统,在斯大林获得权力后,这种正统变成了一种\textbf{教条主义}。与理论力量差不多
完全来自俄国的共产主义者相对的,是德国和奥地利的主流社会民主主义者。德国社会民主
党在魏玛共和国获得一定程度的尊重,有时候他们参与到政府中,鲁道夫·希法亭在20世纪
20年代曾两度担任无可挑剔的保守的财政部长。随着社会主义革命前景逐渐暗淡,德国社会
民主党更加坚定地致力于在现有的体制内进行渐进的改革,越来越多的人接受了爱德华·伯
恩施坦战前的异端邪说。只有少数离经叛道的人,仍然呆在两大阵营之外。

\section{“有组织的资本主义”和新修正主义}

\begin{quotation}
  \textbf{金融资本的社会化职能},使战胜资本主义变得非常容易。一旦金融资本把最重要
  的生产部门置于自己的控制之下,为了获得对最重要的生产部门的直接支配权,只要社会
  通过执行机关——被无产阶级夺取的国家——占有金融资本就足够了。由于所有其它的生产部
  门都依赖于这些最重要的生产部门,所以,即便没有任何进一步的社会化,对大产业的控
  制就已经提供了最有效的社会控制形式……

  甚至在今天,占有柏林六大银行就意味着占有了大产业的最重要的部门。

  换句话说,由于金融资本已经完成了社会主义所必需的那种程度的剥夺,放弃由国家采取
  的突然的剥夺,\textbf{代之以通过社会提供的经济利益进行逐渐的社会化是可能的。}
\end{quotation}
然而,在当时,希法亭对帝国主义的精辟论述和其主要结论中富有战斗性的语调令人振奋,
使得人们无法认识这些段落中包含的\textbf{潜在的修正主义的涵义}(参见以上第五章)。

不过,希法亭改良主义的经济基础,的确是与当时德国资本主义的本质有着较为紧密的联系。
在德国,从来不存在一个全面的自由放任时期,政府一直在经济生活中扮演着重要的角色,
竞争总是受到托拉斯、卡特尔和限价协定的限制。战争的全面爆发加速了这些发展,因为国
家现在对大的经济部门负责,鼓励新的生产者形成联盟,并巩固业已存在的联盟。他们全心
全意地参与战争的努力,为工会赢得了政府和雇主一定程度的认可,给日益强大的国家主义
政权增添了一抹新的社团主义色彩。欧洲其他国家的情况莫不如此,战后德国仅仅是部分程
度上“恢复正常”,战时的很多变化被证明是不可逆转的。

20世纪20年代初,在劳工运动和资本家的圈子里,对进一步的“合理化”的支持是普遍
的。“在所有的资本主义工业国,战后集权的趋势在德国表现得最为明显。”罗伯特·布兰
迪在1933年写道:“该国的背景最有利于继续发展,德国支持集权的态度,看起来在几乎每
一个重要的经济活动中都加强了而不是削弱了。”这样,\textbf{在德国,“老曼彻斯特体
制”没有未来可言},布兰迪断言,“有两个事实至关重要。首先,回到‘\textbf{自由竞
争}’资本主义经济体制的愿望,在德国\textbf{几乎完全消失}了;其次,这种回归的可能
性,也几乎完全消除了”。甚至工会也支持合理化,他们将“合理化”视为“在技术和组织
上从旧秩序向新秩序进化的表达”,这种有序的进化将在全面的经济计划中达到顶峰。他们
期待,工人阶级的政治力量将足以确保工人从合理化产生的收益中,获得一个大的而且不断
增加的份额。

希法亭在战争初期就认识到这些变化的含义,他强化了自己思想中的改良主义因素,这
在《金融资本》中已有所体现。1915年,他在奥地利社会民主党刊物《斗争》上发表文章,
对他认为的社会主义和工会运动迫使资本主义社会进行调整作了描述。在斗争过程中,工人
阶级的力量不断壮大、自我意识增强,但是\textbf{在改善自身地位时,工人阶级也克制了
自己的革命冲动,}因为他们现存的条件不再是完全无法忍受的。“这就落入一种两难的境
地:\textbf{工人运动的反对革命的影响削弱了资本主义的革命趋势}。”其他的稳定因素,
已经出现在“资本主义高度发展的最新阶段”。自19世纪90年代中期以后,萧条的时间缩短,
长期的失业不再像以前那么严重。在德国和美国也不再有传统意义上的失业后备军。金融资
本使得整个体制不再那么容易受到经济危机的侵扰,“\textbf{包含着从无政府资本主义的
经济秩序向有组织的资本主义经济秩序转变的萌芽}”。作为金融资本产物的国家权力的增
长,在同样的方向上发挥作用:
\begin{quotation}
  取代社会主义胜利所出现的,可能是一个\textbf{有组织的社会},但它不是民主地组织起
  来的,而是\textbf{按统治与被统治关系组织起来的,顶层是资本主义垄断和国家的联合
    起来的权力,下面则是按等级划分的劳动群众作为生产的雇员在工作。}不是社会主义战
  胜资本主义社会,我们拥有的将是一个比过去更好地满足群众的直接物质需要的,有组织
  的资本主义。
\end{quotation}

希法亭推断,对战争的体验只会加强这些趋势。那些后来被视为“战争社会主义”的东西,
仅仅是战前的有组织的资本主义的强化形式。

影响更为深远的结论,是由后来成为奥地利共和国总统的法学理论家\textbf{卡尔·伦纳}得
出的。在写于1916年的文章中,伦纳指责马克思主义者忽视了资本主义结构发生的一个根本
性变化。\textbf{国家的经济职能已经达到前所未有的高度。}“这是一个国家渗透进私有经
济直到其基本细胞的问题,不是使少数工厂国有化,而是通过有意志和有意识的调节和指导,
控制经济中全部的私有成分”。这种情况的出现经历了四个阶段。第一个阶段是从1878年到
大约1890年,国家进行干预以\textbf{保护弱者}对抗来自内部和外部的竞争。紧随其后的
是“\textbf{有组织的私人企业经济}”时期,这一时期出现了针对反对国家限制竞争的卡特
尔组织。第三阶段是“\textbf{帝国主义国有经济}”时期,
\begin{quotation}
  国家权力通过用\textbf{保护强者来代替传统的“保护弱者”来为资本服务}…… 国家权
  力和经济开始融合;国家统治的领域和国有经济的领域是一致的,国有经济被看作是国家
  权力的一种手段,而国家权力则被看作是强化国有经济的一种手段,尽管它们公开宣称是
  相互分离的。
\end{quotation}

这一过程在第四阶段即最后一个阶段达到顶点,即“\textbf{国家经济时代}”,在这个时代,
私营部门“由国家来决定,并完全成为国家控制的组织”。\textbf{从而 “马克思曾经经历
  并加以描述的资本主义社会已不复存在}……将这一过程的起点和终点放在一起看,人们可
能会说自由放任的资本主义已经转变为\textbf{国家资本主义},或正处在向国家资本主义转
变过程中。”

在这一点上,伦纳比希法亭走得更远。1924~1926年,希法亭在他编辑的社会民主党的新
的理论刊物《社会》上发表的一系列文章中,确实重新回到了\textbf{有组织的资本主义}
这一问题上来。\textbf{战争及其后果使工业集中程度得到了进一步的实质性的提高。}希
法亭认为,新建立的世界经济的稳定,是基本结构长期变化的结果,而不是转瞬即逝或偶然
情况下产生的结果。不会再有巨大的经济崩溃。资本主义经济的不稳定性降低了,危机对工
人阶级的影响下降了。这是\textbf{托拉斯计划投资的结果},是对它们支出上反周期时机
准确把握的结果,是大银行联合货币当局进行信贷管制的结果。失业对无产阶级来说不再是
威胁,社会保险的扩大降低了对其生存的威胁,而科学管理技术,既提高了劳动强度又分化
了工人阶级。总的说来,改革在让工人们获得一定利益的同时,也使工人们变得更为保守,
所以\textbf{革命性变革的经济基础已不复存在。}这一时期,奥托·鲍威尔、希法亭的同事
埃米尔·莱德勒,甚至意大利共产主义者安东尼奥·葛兰西也都表达过类似的观点。

有组织的资本主义的第二个重要含义和国际关系有关。希法亭认为,不再有令人信服的
理由,期待再度出现帝国主义的敌对行为,而布哈林和列宁提出来的并把它作为战争根源的
经济矛盾是虚构的(参见以上第十三章)。希法亭把一种新的“\textbf{现实主义的和平主
义}”视为战后外交政策的主流。无论是在政治上还是在意识形态上,盎格鲁—撒克逊在全世
界影响的扩大是以牺牲德国和法国为代价的。英国和美国有强烈的意愿阻止战争。英国需要
喘息的空间,以便解决它的殖民地问题,并在亚洲和近东对抗民族解放运动,而美国成为商
品和资本的主要输出国,需要和平的环境去获取收益。甚至法国也需要来自英国和美国的金
融援助,它没有资格进行战争叫嚣。在所有这些国家中,民主的发展进一步削弱了军国主义,
因为数以百万的工人和农民保卫世界和平的呼声日渐高涨。

甚至是德国资本主义,似乎也变得没有以前那么好战了。希法亭提出,化工生产现在成
了德国工业中的主导部门,已经超过俾斯麦时期主导经济的煤和钢铁产业。与老工业相比,
\textbf{工资在化工产业总成本中占的比例更低,因而资本家的利益不再和工人的利益存在
剧烈的冲突。}这改变了最有影响的资本家们对国际前景的看法。战前,老的重工业和大地
主结盟,与军队和帝国主义官僚机构形成强大的同盟,支持侵略性的扩张。战争的失败打碎
了德国的军事力量,但它仍然是第一流的经济大国。“因此,德国资本主义进行扩张的强烈
愿望,必须寻找其他出路,它们在不同类型的国际资本主义利益共同体中找到了。”德国的
外交政策变得越来越温和,是因为可察觉到的资本的利益已经发生了变化。德国实业家最终
接受了英国资本家早在战前就已表现出来的\textbf{世界主义的世界观}。虽然希法亭没有
引述任何权威论述,而且非常谨慎地避免使用“超帝国主义”一词,但他现在赞同考茨基在
1915年提出的\textbf{和平的世界卡特尔的观点}。

最后,他指出政治和经济之间关系发生的重要变化。\textbf{不再有哪个阶级能单独实
施对国家的控制}。无产阶级已经获得了一定程度的真正的政治权力,而且意识到了自身的
力量。\textbf{无产阶级不再将国家视为压迫者手中的工具,而是认为,在一定程度上国家
也可以成为无产阶级为自己谋求福利的工具。}然而,有组织的资本主义并没有消除阶级斗
争。事实上,\textbf{越来越规范的生产和无组织的财产关系基础之间的矛盾},比以往更
为尖锐。为了解决这一矛盾,按统治与被统治关系组织起来的经济必须转化为\textbf{民主
地组织起来的经济}。由于其自身的性质,有组织的资本主义适合于通过渐进的立法改革实
现和平过渡,改革将改善经济体制的机能。原子化的竞争是过去的事情,与它相伴的是
\textbf{非人性化的市场力量的暴政}。卡特尔、托拉斯、资本家联盟、合作组织和工会对
经济生活产生了越来越大的影响。从而对经济进行社会控制已经成为现实。当前期,少数公
司资本家——《金融资本》中的“柏林六大银行”——享有其阶级利益。一旦无产阶级获得完全
的政治权力,它就可以\textbf{利用这些机构实现自己的目的}。这样,希法亭否定了他以
前的观点,不再只把国家视为资本家进行统治的一个执行机构,而将其视为整个社会的代理
人。他认为,国家政策取决于争夺国家控制权的不同阶级之间的相对力量,在这一斗争中,
工人阶级将变得越来越强大。

在英国\textbf{基尔特社会主义}的影响下,像\textbf{弗里茨·纳夫塔利}那样的社会
民主党理论家,主张扩大经济民主,完善德国工人在1918年赢得的政治民主。弗里茨·纳夫
塔利主张,工人阶级必须使用其新的政治力量,扩大自己的经济权力并挑战持续存在的经济
独裁。有组织的资本主义的出现、公共企业的扩展、争取工作环境法律保护的工人运动取得
的重大(即使是有限的)胜利,这一切都表明“\textbf{资本主义在折断之前可以弯曲}。”一
些胜利将打击资本权贵的专制统治。这对德国工人阶级具有重要的战略意义。纳夫塔利写道:
虽然德国社会民主党的社会主义目标没有改变,但实现这一目标的方法,必须随着情况的变
化而调整。经济民主既是社会主义的必要条件,也是实现社会主义的最好方法。可以在以下
几个领域做出改进:第一,也是最重要的,是\textbf{扩大国家对经济生活的控制},这可
以通过反垄断立法、反周期信贷和公共投资政策来实现;第二,\textbf{增加工业领域的工
会代表},这可以通过扩大建立于1918年的劳资协同经营制度的影响范围、加强集体谈判和
培育工会企业来实现;第三,\textbf{进一步发展社会保险网络},以便将收入分配带出市
场领域;最后,支持消费者合作,提倡建立更为民主的教育体系。党的口号应该是
“\textbf{通过经济的民主化实现社会主义}”。

卡尔·考茨基把20世纪20年代的大部分时间,投入到完成其对历史唯物主义进行的百科全书式
的阐释中,他打算把它作为正统马克思主义的指南。考茨基\textbf{否定了}他自己1899年提
出的用来反击伯恩施坦的主张,那时他认为资本主义生产方式面临着无法克服的经济限
制。\textbf{崩溃不是必然的趋势}。经济危机已经变得更加温和,对资本利益的威胁变得更
小,工人阶级的生活水平已有很大的提高。国家正在丧失其作为压迫工具的特征,并向成
为“解放被剥削者的工具”转变。越来越紧张的国际局势也不是不可避免的。正如普选权
(俾斯麦会很震惊)已经成为德国无产阶级手中强大的武器一样,\textbf{国际联盟不再是
  战胜国的工具,它正在转变为国际合作的平台}。无论从国家层面还是从国际层面看,老式
的阶级国家正在向“福利国家”转变。就这样,在与伯恩施坦断绝关系近30年后,考茨基接
受了修正主义的许多重要观点。

有组织的资本主义重要的理论家们,\textbf{否认}他们的观点与马克思主义断绝了联
系。\textbf{在生命的最后阶段,鲁道夫·希法亭仍然将自己视为一个马克思主义者},他围
绕历史唯物主义的一般问题,而不只是狭隘的经济问题,进行着更加自由的创作。希法亭丝
毫没有放弃《金融资本》中的观点,当需要他表达有关政治经济学的看法时,他只是概述这
本书的一些重要观点。考茨基继续致力于社会主义的实现,虽然现在\textbf{他强调社会主
  义的实现取决于工人阶级的品德、智力和政治权力而不是经济崩溃。}事实上,资本主义经
济越成功、越发达,无产阶级就会越强大,社会主义也就会越来越近。\textbf{只要社会改
  革削弱了阶级对抗,改革就成为社会主义无望的原因。}考茨基相信资本和劳动之间的冲突
正在加剧,尽管他承认存在着深刻的经济变革。危机的一再发生,带来了大规模的失业和随
之而来的苦难,工人阶级生活水平的提高,远远落后于他们的剥削者的生活水平的提高。从
而,“\textbf{资本家在经济上变得越来越强大,无产阶级在政治上变得越来越强大。}”

\section{弗里茨·斯滕伯格论帝国主义}

1929年之前,对新正统主义观点提出的挑战,是随着唯一严肃的非共产主义者\textbf{弗里
  茨·斯滕伯格}大部头《\textbf{帝国主义}》的出版出现的。通过应用、扩充和改造罗
莎·卢森堡的《资本积累论》(参见以上第六章),斯滕伯格认为,当前的繁荣只是暂时现象,
资本主义经济不久就会遭到自身矛盾的报复。无论是旧的还是新的修正主义,都仅仅是对暂
时的经济稳定的一种意识形态的反映。革命性政治活动的必要性,不久就会重新显现。

战后返回欧洲,提倡介于资本主义和苏维埃共产主义之间的“\textbf{第三条道路}”。

\textbf{像卢森堡一样,斯滕伯格的批判集中在马克思本人经常做出的一个简化假设上},
该假设认为,纯粹的资本主义经济是一个没有农民、工匠和其他小商品生产者的经济。斯滕
伯格坚持认为,这个假设是一个最重要的错误,因为非资本主义生产自资本主义历史开始之
初就扮演了重要的角色。罗莎·卢森堡已经正确地强调了马克思的推理中存在的这一缺陷,
但她的批评还不够深入。马克思所有的著作都遵循了这一假设:\textbf{工资理论、危机理
论和革命理论,以及对再生产的分析}。简言之,纯粹资本主义的假设消除了这个体系的真
实本质。

斯滕伯格指出,\textbf{资本主义生存的基本条件是对过剩人口的需要},他以此开始自己的
批判:没有过剩人口,就没有剩余劳动,剩余价值也无法被生产出来。马克思对失业后备军
的解释集中在机械化造成失业的影响上,\textbf{机械化提高了资本有机构成},减少了生产
过程中相对于死劳动而言对活劳动的需求。在斯滕伯格看来,\textbf{技术进步既不必然也
  不足以保证过剩人口的产生。}之所以不必然,是因为即使是不存在技术变化,\textbf{非
  资本主义生产者也构成劳动力的潜在来源}。之所以不足以,是因为在特定的历史环境
下,\textbf{机器取代劳动的效果}可能会因为海外市场的大规模扩张而得到“\textbf{过度
  补偿}”,这种扩张因\textbf{机器}的使用而变得可能。\textbf{每单位产出可能要求更
  少的活劳动},但是世界非资本主义地区需求的增长会增加产出的数量,从而\textbf{就业
  仍然会提高。}(作者注:Der imperialismus, pp. 23-30.这毫无疑问是一个有充分根据
的对马克思进行的批判:参见马克思:《资本论》第一卷,《马克思恩格斯文集》第5卷,人
民出版社2009年版,第25章。 M.C. Howard and J. E. King ,The Political Economy of
Marx (Harlow: Longman, 1985) 2nd edn. pp. 226-7.)

更为正式的说法就是,斯滕伯格指出了\textbf{过剩人口的六个方面来源}。其中两个来
源——\textbf{人口的增长、机械化——内在于资本主义体系},这也是马克思所强调的。而更为
重要的是四个“外生”因素。在这四个来源中,马克思注意到了其中的两个:\textbf{对工
  匠的剥夺和农民迁徙到城市}。但他在很大程度上\textbf{忽略}了另外两个过剩人口
的“\textbf{外源性}”来源:\textbf{从非资本主义地区来的移民,(最重要的)对这些地
  区的资本输出。}同机械化一样,资本输出首先增加了资本主义工业化国家内部的就业,因
为需要更多的劳动力生产生产资料。\textbf{之后,随着殖民地开始与殖民母国进行生产竞
  争,就业下降了。}因此,在先前的非资本主义地区融入世界经济之前,\textbf{过度补
  偿}是帝国主义渗透的早期阶段的一个主要特征。“非资本主义地区资本化的进程越快,比
如,印度依靠本国工人进行工业建设,那么起主动作用的帝国主义国家中大量工人阶级的
处境就越困难;那时,机器取代工人的步伐也越快。”

这一结论包含在卢森堡的分析中,但是可能她没有非常清楚地表达出来。这使得斯滕伯
格能够依据过剩人口的主要来源,提出资本主义发展的分期标准。在大致对应于马克思的
“原始积累”阶段的第一个时期,头两个外生因素(当地工匠和农民)是最重要的。接着,
机械化成为劳动力的主要来源。最后,\textbf{外源性来源开始占主导地位,帝国主义时代
开始了。}斯滕伯格进一步区分了早期和晚期帝国主义。在帝国主义活动展开的初期,技术
进步和资本输出的过度补偿效果仍然是强大的,与此同时,本土的非资本主义生产者迅速枯
竭。从而,失业率处于它的最低点,而工人阶级的生活水平处于他们的最高点。这是工人阶
级的“蜜月期”。它以大多数资本主义发达国家大多数工人实际工资的真正的增长为特征。
事实上,\textbf{修正主义就是“蜜月期理论”。}

斯滕伯格坚持认为,蜜月期不会太久,因为实际工资持续不断地增加,会威胁到资本积
累自身。从而,资本主义会从体制外面寻求解脱。\textbf{蜜月期被资本的加速输出所终止,
这赋予过剩人口国际上的含义。}比如,\textbf{英国工人的工资不再仅单独由英
国决定,而是受到英国资本主义向全球扩张和利用殖民地和半殖民地地区劳动力的可能性的
影响。}由此可以得到两个重要的结论:\textbf{第一,如果不存在可能提高殖民地工人生
活水平的有效的无产阶级的团结,欧洲的工资将被迫下降到亚洲的水平;第二,帝国主义不
能被视为是一个偶然的或一个可以避免的政策选择。它是作为整体的资本主义体系的一种内
在必然。}

斯滕伯格支持上述看法中的第二个观点,关注了帝国主义在实现剩余价值方面发挥的作用。
罗莎·卢森堡早已经证明,马克思的再生产模型是难以令人满意的。斯滕伯格坚持认为,它们
必须被加以修改,以包含资本主义现实的两个重要特征:\textbf{第\Rnum{1}部类的资本有
  机构成比第\Rnum{2} 部类的高,两大部类的资本有机构成都在提高。}无论是罗莎·卢森堡、
奥托·鲍威尔还是布哈林,都\textbf{没有能够以这种方式}成功地扩展马克思的分
析。\textbf{假如他们这样做了,他们会发现第\Rnum{2} 部类必然有无法销售的剩余产
  品,}如此一来,一个封闭的资本主义体系确实会在\textbf{实现剩余价值}方面面临严重
的困难,并被迫从海外寻找非资本主义的消费者。 罗莎·卢森堡在这方面对马克思的批评,
是“正确的且在所有的要点上是合理的”。但是她在一定程度上夸大了问题的重要性:并不
是所有积累的剩余价值都实现不了,只是其中一部分,即第\Rnum{2} 部类生产的消费品的剩
余无法实现。由此导致的危机的破坏性,比人们从卢森堡的分析中预计到的要小一些。然
而,\textbf{这些危机足够真实,它们“最终的决定性原因”,是“建立在对抗性的分配关
  系基础之上的生产的增长与的消费的增长之间必然的比例失调”,而这种对抗性的分配关
  系,“以一种非常复杂的方式,以生产资料和消费资料产业投资支出的比例失调表现出
  来。” }

斯滕伯格指出,\textbf{卢森堡认为帝国主义总是导致阶级斗争的加剧是错误
  的。}直到1914年,在帝国主义早期,过度补偿的影响意味着与此相反的事实是正确的:实
际工资已经提高,社会改革已经实现,资本和劳动之间的敌对已经弱化。\textbf{但是,帝
  国主义仅仅是为阶级战争提供了暂时的缓解},就如同它只是代表了一种资本主义社会经济
矛盾有限度的解决方法一样。通过同化越来越大的领土,\textbf{它越来越难获得新的非资
  本主义市场,从而开始自掘坟墓。}世界进入了一个新的经济危机周期,为进入正在萎缩的
市场而产生的国际竞争加剧,实际工资下降的压力不断增加。这一点在英国最明显(斯滕伯
格在总罢工那年正在进行此文的写作。) 只有在美国,工人阶级继续受益。这儿的 “蜜月
期”比欧洲的来得晚,修正主义者从美国经验中概括的一般结论是错误的。斯滕伯格断言,
由于对特定历史时期的具体情况绝对化,他们犯下了资产阶级科学中最大的错
误。\textbf{帝国主义晚期的修正主义没有未来,因为它是危机深化、生活水平下降和帝国
  主义战争继续的阶段。}

\section{对斯滕伯格的评论}

斯滕伯格的《帝国主义》是一部\textbf{质量参差不齐}的著作,它在结构和范围上令人印象
深刻,而在实质性分析方面较为薄弱。对\textbf{资本主义发展阶段的逻辑历史的说明},将
这些不同阶段经济和政治思想上的变化联系起来的尝试,使得斯滕伯格的著作比这一时期任
何其他著作都更接近于马克思的《资本论》。此外,\textbf{他的过剩人口来源理论,触及
  到半个世纪后发达资本主义经济明显面对的“去工业化”背景下重新出现的问题。}斯滕伯
格自己并未提供任何坚实的分析,支持其有关“蜜月期”和帝国主义晚期生活水平会下降的
武断的说法。\textbf{马克思主义者后来争论说,国际工资的差异引起国际贸易中的不平等
  交换,富国对剩余价值的榨取阻碍了贫穷国家的工业发展。其他人用发达国家和落后国家
  之间大的生产率差距来解释工资差异,这种生产率差距制约了后者在国际市场上的竞争
  力。}斯滕伯格的著作忽略了所有这些问题。

他的经济危机理论更难令人满意。正如亨利克·格罗斯曼很快注意到的那样,他的经济危机理
论建立在“\textbf{一个未加分析的用图表表示的数字例子}”上。斯滕伯格对卢森堡和奥
托·鲍威尔进行了严厉的批评,然而,和他们不同,他并没有尝试通过一个典型的再生产图式
去证明他关于无法销售的剩余产品的论断的真实性。因此,我们有充分的理由假
定,\textbf{在以上第六章对罗莎·卢森堡的分析的批评,同样适用于斯滕伯格的危机理
  论。}奥托·鲍威尔的妻子\textbf{海琳},在奥地利社会党的理论刊物上发表的对《帝国主
义》的评论中,明确说明了这一点,她指责\textbf{斯滕伯格忽视了信贷在发达资本主义经
  济流通过程中发挥的作用。}海琳认为,\textbf{信贷从一个部门到另一个部门的扩展,可
  以克服比例失调。}斯滕伯格的观点,如果可以应用的话,也\textbf{仅仅适用于信贷尚未
  出现的相对原始的资本主义阶段。}

海琳的评论,在对任何狭隘的帝国主义的经济理论的批判上,具有独立的意义。她坚持认
为,\textbf{国际贸易是劳动的社会分工的一种表现形式,不是经济矛盾明确无误的迹象。
  由于进口必须用它们的收入支付,出口的增长并没有提供任何实现问题的解决方法,}比如
英国,事实上是个净进口国。在理解帝国主义斗争问题上,政治因素非常重要:
\begin{quotation}
  世界战争源自军事集团的罪行和轻率行为,源自哈布斯堡的王朝利益和罗曼诺夫王朝的政
  治威望这一事实——所有这一切都很难与通常的研究经济利益而不是社会经济状态的唯物史
  观的基本形式相调和。在与封建主义结盟时,资本主义可以好战,可以寻求战争并通过战
  争提高其利润率。但是它\textbf{同样也可以是爱好和平的}。
\end{quotation}

它可以通过\textbf{国际卡特尔},通过\textbf{国家对出口信贷担保的保护},获得比
通过军事庇护能够得到的更多的利润。总之,\textbf{帝国主义既无法提供治愈经济危机的
良药,也并不意味着资本主义体系会衰落到一种不可避免的野蛮状态。更为重要的是,帝国
主义的对抗是个十分便利的骗人把戏,其背后隐藏的是统治阶级向工人隐瞒国内阶级斗争的
现实的意图。}

\section{一个评价}

海琳·鲍威尔对斯滕伯格批评的主要一击让人无法反驳:\textbf{斯滕伯格的经济论证不足
以证明他的预言式的结论。}但是无论是她自己还是其他修正主义者,都无法通过有力的分
析来支持她的论断。揭示斯滕伯格和卢森堡的含糊不清和不足之处,表明在不存在帝国主义
或部门之间的比例失调造成的危机的情况下,信用提供了实现剩余价值的可能是一回事;
\textbf{证明信用可能在长期带来平稳、持续和有利的增长是另一回事}。指出侵略性经济
民族主义理论的缺陷,揭示盎格鲁—萨克逊(甚至某些德国人)实业家和政治家中存在的和
平主义倾向是毫不困难的。\textbf{表明战争的经济动机已经逐渐消失完全是一件更艰难的
任务。}希法亭、考茨基和海琳·鲍威尔都\textbf{无法对列宁主义关于资本主义对抗和战争
之间必然联系的观点,提供令人信服的反驳}(参见以上第十三章)。从某种意义上来说,
他们的修正主义走的还不够远。

马克思主义自身的某些品性,使得修正主义具有这样一个肤浅的基础。现在看来,希法
亭在《金融资本》中强调的那些资本主义的特征,\textbf{从很大程度上,是因为德国在相
对落后的条件下进行的工业化造成的:而不是代表一个作为后发的产物的“发展的更高阶
段”。}因此,斯滕伯格指责修正主义把特定历史时期的具体情况绝对化是正确的。但是其
他的马克思主义者也犯了同样的错误。到目前为止,甚至是最为成熟的历史唯物主义者,都
接受单向的历史理论,这就很难从特定的暂时的形式辨识出普遍的发展模式。

布哈林认为,\textbf{有组织的资本主义的概念是自相矛盾的,因为资本主义本质上是无政
  府性的。}他认为,\textbf{竞争仅仅是从国家的层面转移到了国际领域,这使得世界经济
  并不比以前更加统一,和谐、和平的资本积累依然像以前一样遥远}(参见以上第十三章)。
正统马克思主义者和独立的马克思主义著述者持类似的批判意见。但是,古典马克思主义几
乎没有为否定有组织的国际资本主义经济提供什么根据。马克思在他主要的经济学著作中,
将资本主义的分类抽象成相互分离、相互竞争的\textbf{领土国家}:他提出的是\textbf{单
  一的资本主义模式,而不是资本主义体系的模式。}他的国家理论仅提到国内统治,忽略了
它与其他国家之间的关系,没有为国家体系的\textbf{国际关系}提供任何分析。强化这一疏
漏的是马克思主义者对国际主义的阶级利益的信念。“所有国家的工人团结起来”,这一口
号断言无产阶级沙文主义已成为过去,意在声称资产阶级民族主义同样的古老。如果这是个
非常普遍的想法,它就是一个严重的错误。

通过与更多的正统马克思主义相比较,可以知道\textbf{新修正主义并没有对资本主义社会
  持特别和谐的观点。}他们的政治策略仍然与\textbf{阶级斗争}联系在一起,认为没有阶
级斗争就没有向社会主义进化的可能。的确,希法亭几乎将20世纪20年代的德国描绘成一个
具有双重政权的国家。这显然是错误的。然而,更为重要的是,\textbf{他相信社会主义可
  以通过已建立的民主机构的框架和平地实现}。马克思本人倾向于认为,自由主义的民主是
与资本主义经济基础最相适应的政治上层建筑,直到1917年列宁的“四月提纲”发表时,布
尔什维克甚至还这么认为,并把他们反对俄国专制主义的整个政治策略,建立在这一基础之
上(参见以上第八章、第十一章和第十三章)。

20世纪20年代修正主义的错误很快被揭示出来:\textbf{大萧条、纳粹对政权的夺取以及帝
  国主义战争的再次爆发,成为对它们最有效的批判。}但是它们的错误深深地根植于马克思
主义的理论素材中,而不是德国或社会民主党特有的过失。确实,在某些方面,修正主义的
分析比正统马克思主义更有说服力。\textbf{海琳·鲍威尔的新熊彼特主义式的认识,即返祖
  性的前资本主义思想应当为帝国主义侵略负主要责任,现在已被广泛地接受。}修正主义者
的错误在于,认为1918年后欧洲的政治重建已经消除了帝国主义侵略,并且夸大了战后经济
稳定的持久性。\textbf{社会主义不可能通过和平的和零敲碎打的改革渐进地实
  现。}正如19世纪90年代的伯恩施坦和他的同事一样,20世纪20年代的修正主义已经成了一
种痴心妄想。

\chapter{向社会主义过渡:1917~1929年的共产主义经济学}

\section{引言}

十月革命开启了马克思主义政治经济学发展的新篇章。\textbf{向社会主义过渡作为一个
实践问题},提上了议事日程。无论在马克思和恩格斯的著作中,还是在第二国际理论家的
著作中,\textbf{几乎找不到任何的指南},布尔什维克思想家们不得不发展一种能够指明
如何实现这种过渡的经济学。因为夺取政权发生在世界资本主义的外围,任何的创新都是必
需的。由于\textbf{俄国革命在某种意义上是“反《资本论》的革命”(葛兰西,《反《资
本论》的革命》)},对俄国来说,马克思有关过渡问题的仅有的遗产相关性极其有限。当
然,很多革命者完全意识到这些难题,他们用他们所处时代的新的解释,证明自己行为的合
理性(参见以上第十二章和第十三章)。托洛茨基在20世纪20年代中期直截了当地重申:
“\textbf{如果世界资本主义……能够找到一种新的动态平衡……这就意味着我们的基本历
史判断出现了错误。这也意味着资本主义还没有完成其历史‘使命’,而且(帝国主义)还
不构成资本主义崩溃的一个阶段。}”因此,俄国革命应当被视为一场“\textbf{早熟}”的
革命,向社会主义的过渡注定要失败。

\textbf{面对1917年以后的困难,布尔什维克完全可能没有任何社会主义式的解决方法},这
也许是我们在理解共产主义经济学具有的理论上的不稳定和冲突的特征时,需要考虑的最重
要的因素。\textbf{革命政权继承了一个濒临崩溃的经济。}1914年后成年男性人口中的三分
之一,被动员起来参与到战争中,落后的俄国经济已经极其脆弱,很难承受一场全面的战争。
革命和内战更是毁灭性的。假设1913年工业产出指数为100,1917年下降到75,1921年为31,
而农业生产在1917年下降为90,四年后下降到60。在俄国内战期间,在西方资本主义国家的
封锁下,俄国的对外贸易事实上几乎完全中止。随后的复苏十分迅速,工业和农业指数
在1928年分别上升到133和125。然而,将1913~1928年作为一个整体来看,俄国仍然远远落
后于西方。与1870年至1913年2.5\%的年增长率相比,这一时期的年产出增长率只有极低
的0.8\%,前一个时期人口的年增长率为0.9\%,但在1913年后下降到0.3\%。

1929年之前,苏联的经济史可以分为三个不同的阶段,每一阶段都产生了各自的过渡理
论。\textbf{革命后的头八个月,以法律上和事实上的经济关系的鸿沟不断扩大为特征。农
  民夺取了土地,并根据传统的公社原则进行重新分配,使得新政府颁布的正式的土地国有
  化法令成为多余,并减弱了先前农民内部分化程度和减低了生产率。屈指可数的工业的国
  有化,大多是自发的地方行为的结果。此外,实行了“工人控制”,其中私人资本家受工
  厂委员会和当地布尔什维克官员的监督。}列宁顶着左翼的批评,捍卫了这种制度,将其描
述为“\textbf{国家资本主义}”(相当含糊的一个术语,尤其是考虑到革命成功之前对它的
使用),并把它视为\textbf{至少在早期阶段的主要的过渡模式。}

1918年6月以后,内战的爆发激起了一股\textbf{国有化}的浪潮,并强制实行\textbf{紧缩
  经济}。\textbf{试图征用全部的农业剩余,}农民留下的仅够种子和勉强生存。工业品的
配置不是通过货币中介完成,而是进行\textbf{直接的配置,工资以实物发放},\textbf{把
  军事纪律强加给城市劳动力。}最终取消了对公用事业、住房、铁路交通和基本食物配给的
收费。\textbf{经济管理的集权达到了前所未有的程度,并呈现出侵占、恐怖和独断的特
  征。}布尔什维克理论家明确地将其视为向社会主义的适当的过渡。

1921年初,对农民的余粮征收,被一种新的\textbf{农业产出税}取代,这是第三个阶段的标
志,即\textbf{新经济政策(NEP)阶段}的来临。\textbf{恢复了农民留下农业剩余进行交
  易的权利,以及雇佣劳工的权利。}他们只为购买工业品而出售粮食,因此新经济政策预示
着恢复了农业和工业之间的市场交易。\textbf{富裕的农民——臭名昭著的富农——从这一制度
  中获益颇多,}一个新的小资产阶级——\textbf{耐普曼}——开始利用这一机会从事有利可图
的零售贸易。而经济的“制高点”——银行业、大规模工业和对外贸易——仍然归国家所有,并
受政府直接管理支配,用货币购买投入,为货币出售产品,\textbf{私人企业再次获准雇佣
  技工和拥有小作坊}。\textbf{新经济政策在相当大的范围上允许中央计划和市场的同时存
  在。列宁将其视为“过渡性的混合体制”。说它是“混合的”,是因为其中有社会主义的
  因素,有简单的商品生产和社会主义生产;说它是“过渡性”的,是因为它固有的不稳定
  性,它要么终结于资本主义的复辟,要么终结于完全社会化经济的实现。}在这个阶段,人
们普遍认为,过渡将随着国有部门经济比重的增加,通过逐步对市场形式的超越而实现。但
是,在如何完成这种过渡上,理论家们之间存在着重大的差异。

新经济政策下的\textbf{根本难题是城市和农村的关系}。四分之三的人口是农民,工业的扩
张要求将大部分的农业剩余转移到城市地区。随着战时共产主义的结束,这只能通过引导农
民为市场提供充足的粮食而自发地实现,而这又要求农民可以以有吸引力的价格获得工业领
域生产的商品。但是\textbf{工业领域的“商品荒”}是整个新经济政策实施期间的重要特征,
农业品和工业品的相对价格一再成为尖锐冲突的源泉,就像在1923年的“\textbf{剪刀差危
  机}”中表现出来的那样,\textbf{当时工业品价格急剧上升,农民不愿意在市场上出售他
  们的农产品,出现了担心粮食严重短缺的恐慌。}不久之后,价格的“剪刀差”消失了,但
在这十年行将结束的时候,新经济政策的问题变得更为明显了。1928年的“\textbf{粮食危
  机}”,当时农产品的销售数量远远低于对它的需求数量,成为压垮新经济政策的最后一根
稻草。此后,开始实施让人联想起\textbf{战时共产主义}的措施,1929年后,它们在斯大
林“自上而下的革命”中达到了顶峰。

本章重点研究这些时期产生的过渡理论。尽管布尔什维克执掌政权的第一个十年,是与更为
广泛的富有成果的理论发展联系在一起的。在新成立的由\textbf{大卫·梁赞诺夫}管理的马
克思恩格斯研究院的推动下,马克思主义思想史的研究蓬勃发展;出现了\textbf{马克思主
  义数理经济学}的萌芽;马克思主义者第一次认真地对\textbf{计划问题}进行了分析;依
附理论的萌芽,作为革命的马克思主义的焦点向东方转移的过程的一部分,开始出现。苏联
的历史学家进一步深入研究了历史唯物主义在亚洲的应用;像\textbf{康德拉季耶夫}这样的
统计学家,对\textbf{资本主义增长周期的本质}提出了富有启发性的观点,甚至新民粹主义
都试图在解决农业的社会主义改造问题上,作出自己的贡献。尽管如此,过渡问题还是主要
思想家关注的重点。

\textbf{过渡理论多种多样。在某种程度上,这是马克思主义的本质功能,通过把逻辑归因
  于历史,产生了一种把过渡问题看作是一个受到高度约束的社会工程问题的倾向。}人们认
为,无产阶级一旦作为统治阶级掌握政权,必然会走社会主义的道路。换言之,\textbf{过
  渡具有法则般的特征},因此,理论可能会落后于实践,而没有成为纯粹被动的理论或者是
对不怎么令人满意的现实的辩解。这有助于解释,为什么像布哈林和普列奥布拉任斯基这样
的理论家,为了使他们的理论同经济发展形式发生的根本变化相一致,\textbf{会在很短的
  时间内坚持明显存在差异的观点},在他们打算接受的经济发展形式发生根本变化时可能更
是如此,而且他们既不会陷入怀疑论中,也不缺乏忠实的追随者。\textbf{理论著作同历史
  转变一样,具有了辩证的特征。}

然而,革命之后精神生活的复杂性,不仅仅体现在理论家们立场的急剧变化上。激烈的
争论,成为每个阶段的特点。现成的过渡经济学的缺乏与此有关,因为事实是\textbf{社会
主义的特征从来没有被清晰的界定过}。从而,在什么构成最终目标这一问题上,存在着大
量的不同的观点,最终目标的萌芽特征在过渡时期应该是可能被观察到的。布尔什维克既简
单化了又复杂化了这个问题。随着\textbf{将党视为无产阶级的先锋队和无产阶级利益的代
表,它的支配地位成了健康发展的判断标准。}事实上,早在20世纪20年代,保持
\textbf{布尔什维克的政治垄断地位},成为无产阶级专政继续存在的长期信仰的主要支柱。
托洛茨基在1904年宣称的,布尔什维克的逻辑使得\textbf{党“取代”了阶级}这一断言已
得到证明,尽管托洛茨基本人现在也否认了这种指责。

几乎所有的党员都赞同保持布尔什维克专政的重要性,但是\textbf{革命的孤立和农民对它
  进行的“小资产阶级的包围”,增强了对压制反革命力量的可能性的疑虑。然而,这些焦
  虑被一些事件改变了。一旦在激烈而艰苦的内战中,政权获得胜利并保存了它的力量,由
  军事失败而丧失政权的可能性就消失了(虽然这种担心在1926年后又卷土重来)}。相反,
党内的每一派别都对其他派别持怀疑态度。这种相互之间的不信任,是有其充分的理由的:
可能表达利益的其他方式都被否定了,\textbf{非无产阶级的阶级利益也只能由布尔什维克
  政党来代表}。托洛茨基表达了一种普遍的情绪,他写道:“\textbf{无论是阶级还是政党,
  都不能通过他们所说的进行评判……这也完全适用于政党中的不同群体。”“在手握共产
  主义旗帜的情况下倒退到热月的状态。在这里存在着恶魔般的历史欺骗”}是可能的。这样,
理论的分歧呈现出一种革命与反革命的阶级对抗的恐怖特征。

因党内每一派别都认识到\textbf{党的专政实际上是不稳固的},这一特征得以加强。尽管很
少有布尔什维克相信当前的形势将完全无法持续下去,但主要的理论家认为他们
的\textbf{生存的确是岌岌可危的}。策略选择的空间非常有限,而其他派别的计划似乎对成
功过渡的可能性产生了威胁。在这样的情形下,占统治地位的派别对党的纪律的合法界限作
了更为严格的解释,一系列的反对派别感受到压力更大的限制。

1924年1月列宁的去世(加之他在去世前生病的18个月中影响有限),加剧了这一冲突,
因为这一冲突开始与争取继承权的斗争融合在一起。这导致对列宁的神化和将其作品提升到
类似于犹太法典《塔木德经》的地位,这都使得像普列奥布拉任斯基和托洛茨基这样的具有
独创性的理论家难以适应。然而,\textbf{它的确反映出布尔什维克党转变为一个大的官僚
机构,它有能力管理一个现代国家,但同时也能接受更为粗俗的意识形态。}

\section{作为过渡模式的国家资本主义和公社国家}

十月革命后,布尔什维克政府试图立即实施列宁在1917年就已经形成的思想(参见以上第十
三章)。\textbf{苏维埃政权发布法令,承认农民夺取土地是合法的,准许工人对工业的控
制,并对那些被认为能够使国家政策有效发挥作用的城市经济因素进行国有化。开启了与德
国政府的和平谈判,拒绝偿付沙皇俄国的债务,并开始建立一个新的共产国际的初步工作。}

列宁认为,向社会主义过渡的最初阶段,可以从\textbf{“国家资本主义”组织和公社国家
的结合开始}。私人资本家和资产阶级专家,仍然像以前一样工作,但现在要置于无产阶级
政治权力的指导和监察之下。列宁认为,这足以重建生产和分配,直到国际革命能够提供向
社会主义建设更有系统进展的环境。这里有几个假设,\textbf{没有这些假设上述主张就毫
无意义}。列宁在1917年早期就坚持认为,依据革命大众广为接受的纲领明确地夺取权力,
可以\textbf{消除内战扩大的可能性}。同时,他坚持认为,\textbf{民主化将振兴经济和
武装力量},足以抵御德国帝国主义和进行一场必要的革命战争。

这些信念很快被一些事件证明是错误的,由这些想法决定的过渡形式同时遭到破
坏。\textbf{革命进一步瓦解了经济和武装力量。}在食物供给不足的压力下,在从被剥夺的
大地主那里获得土地的期望下,\textbf{城市人口减少,士兵逃离军队。}布尔什维克政府被
迫在惩罚性的布列斯特——立托夫斯克和约下向德国\textbf{投降}。\textbf{这种混乱导致无
  法通过批准地方行使主动权,来扭转工人控制的局面。这破坏了整体和谐,超越了被布尔
  什维克领导人视为权宜之计的措施,扩大了对私人资本家的压制,从而对“国家资本主
  义”的过渡模式产生了直接的和不利的影响。}到1918年3月,列宁很快\textbf{背弃诺言,
  以经济上的必要性为名义,将工人阶级组织的自治从属于等级制的控制,并以此取代了在
  《国家与革命》中被明确阐明的理论。}

政权力量的薄弱,也使其无力确保私人资本家和现有官僚们的合作。1918年,反革命力
量的增长,导致列宁的方案所依靠的人员逃离了无产阶级的大本营。\textbf{随着1918年中
期内战的爆发,布尔什维克被迫将大规模的国有化当作一种安全措施。同时,随着求助于强
制措施为城市居民和红军征集食物供应(没有足够的现实的资源通过自愿交换来获得它们),
政权与农民之间的联盟变得紧张起来。}

所有这些都表明,列宁1917年革命策略中存在\textbf{严重的误判和内在矛盾}(参见以上第
十三章)。特别是他的“国家资本主义”的过渡模式,被证明\textbf{无法与阶级斗争的动
  力学相协调}。列宁正是在这一问题上遭到党内批评。包括布哈林在内的左派,在这一阶段
呼吁采取更多激进的措施,并热情支持\textbf{战时共产主义政策}的发展。他们认为,这一
系列的行动是一些事件强加给政权的,但是,\textbf{由于革命发展的内在逻辑,左派
  将1918年经济政策的激进化看作是向社会主义过渡不可避免的特征。}布哈林的思想被证实
是有感染力的。到1920年时,他将这些思想在《过渡时期经济学》中系统化了,所有的重要
的布尔什维克似乎都接受了这些思想的普遍效力。然而,\textbf{布哈林作了太多的妥协
  (其他的左派难以接受):将迄今为止他一直坚持的完全民主化的“公社国家”的主张,
  让位于一个中央集权的党的专制的概念,尽管这个概念以无产阶级“自治”的面目进行了
  伪装。}

\section{直接向社会主义过渡的战时共产主义}

起始于1918年中期实施了几乎三年的政策,以\textbf{战时共产主义}的名称著
称,\textbf{它代表了一种形式简陋的指令经济。几乎所有的工业都被国有化了,资源通过
  行政机关而非市场进行配置,私人贸易受到压制,而货币关系很大程度上被削弱。契卡
  (肃反委员会或安全警察)和红军将农产品从农民那儿夺走,并将其自由分配给工业,并
  作为消费配额分配给指定的群体。苏维埃内部的民主受到了有力的压制,党内的纪律大大
  加强,对反革命分子实施“红色恐怖”,这些反革命分子包括无政府主义者和曾经支持十
  月革命但却抵制布尔什维克党专政的社会革命团体。}

布哈林的《\textbf{过渡时期经济学}》是对\textbf{战时共产主义}作出的最为成熟的理论
表述。它最核心的论点就是\textbf{独裁是必要的},因而也是普遍适用的、是向社会主义过
渡的范式。它既代表了无产阶级专政也代表了国家社会主义形式,是\textbf{“翻过来”的
  现代资本主义}。国家资本主义的结构——布哈林认为它带来了革命——在无产阶级的政治控制
下将会被重组。由于这代表了一种\textbf{新的阶级专政},针对敌对群体不可避免地存
在\textbf{强迫和恐怖行为}。根据布哈林的论点,无产阶级的民主组织同样是不适合的。集
中虽然随着往后国家的消亡,最终会以新的形式出现,但在无产阶级专政期间,为了胜利地
结束内战,集中是必要的。在布哈林看来,过渡时期的政治结构是一种真正的民主;在党的
领导下,无产阶级自愿的自律是其阶级统治最完美的表现。

布哈林重申了在《帝国主义和世界经济》以及《有闲阶级的经济理论》中(参见以上第
十三章)的立场,并创造了“\textbf{消极的扩大再生产}”一词来描述现代资本主义的危
机。他认为,经济生活中的国有化、军国主义和战争,\textbf{使得扩大再生产朝向紧缩的
方向发展}。对资源的非生产性利用,达到了阻碍经济积极增长的程度,带来了革命并导致
世界范围内出现崩溃。但是,无产阶级统治的建立,\textbf{将使得消极的再生产延长到后
资本主义时代}。作为经济崩溃的产物,随着反革命力量被粉碎,革命会使其得到深化。

布哈林进一步认为,建立在国家控制和强制下的无产阶级专政,已经摆脱了经济法则的
宰制。从马克思主义的特定意义上理解,\textbf{政治经济学只适用于商品生产体系。推翻
资本主义之日,也是无产阶级经济学终结之时。}布哈林这时认为,“从必然王国向自由王
国的飞跃”还没有实现。决定论而非唯意志论仍然居统治地位,但它代表了对无产阶级阶级
利益的自觉追求。这毕竟是由社会主义革命所代表的具有划时代意义的过渡的另一种表现形
式,事实是,纪律严明的组织和强制是最重要的。

布哈林设想,随着内战的成功结束,战时共产主义的结构也会发生改变,但他没有看到
根本性重建的需要。在党的先锋队的领导下,中央集权的无产阶级国家会坚持一种行政性地
组织在一起的城乡间的交易体系,随着工业生产能力的恢复,以替代强制性的征用。这是因
为\textbf{消极的再生产,必然集中于联系更为紧密的城市经济中,也因为农业生产的小资
产阶级结构阻碍了有效的国有化,因此,最初对农民的剩余进行强制性的剥夺是必需的}。
最终农村经济被社会化,阶级差别逐渐消失,不平等减少,国家随着参与式民主和和谐的计
划经济的建立而消亡,虽然布哈林从未确切地详细说明所有这一切是如何发生的。

正如对其存在的必要性出色地加以合理化一样,布哈林的《过渡时期经济学》有
其\textbf{政治目标}:它力求反驳西方无政府主义者和社会民主党对党的专政及其统治方法
进行的批评,在反驳中,\textbf{《过渡时期经济学》把党的专政及其统治方法提升到任何
  一次成功的无产阶级革命中都必然出现的现象的高度。}从这个意义上来说,布哈林的书与
列宁的《无产阶级革命和叛徒考茨基》,以及托洛茨基的《恐怖主义与共产主义》是互补的。
它部分地解释了为什么重要的布尔什维克从未完全否定布哈林的过渡时期理论。随后的几年,
在新经济政策建立起来后,他们仍承认战时共产主义措施是权宜之举,而且将它们视为整个
过渡过程中一个必不可少的部分。托洛茨基甚至声称,如果革命扩大到了国际范围,1921年
的“退却”将是不必要的(虽然他并不总是坚持这一点)。普列奥布拉任斯基承认,战时共
产主义措施为新经济政策的过渡途径提供了基础。至于斯大林20世纪30年代早期
的“\textbf{二次革命}”,可以说已经有了一个较早的理论基础,在很大程度上是布哈林的
《过渡时期经济学》\textbf{奠定了这个基础}。此外,很多著作中的主要思想在继续提
出20世纪20年代完全不同的过渡范式。\textbf{至高无上的国家、各种不同形式经济结构的
  统一、党的主导作用、对工人阶级“自律”的需要,以及完全取代市场关系的最终目标,
  所有这些都体现在布哈林随后的著作中。}

在战时共产主义为什么最终被证明是不可持续的这个问题上,布哈林和其他理论家都相
当的坦率。\textbf{战时共产主义的矛盾是破坏了列宁“国家资本主义”计划的矛盾的反作
用形式。国家资本主义屈从于阶级斗争的压力,而战时共产主义的崩溃是因为它忽视了阶级
合作的需要。}内战期间,当恢复“旧体制”仍然是一种明显的可能时,对农民的强制没有
引起他们一致的抵抗,因为布尔什维克的失败意味着地主的回归。但是,随着20世纪20年代
对白军的胜利变得越来越稳固,农民对布尔什维克的反对日益明显。然而,内战结束时,并
没有立即放弃战时共产主义,这表明领导阶层非常专注于这种过渡模式。直到1921年3月,
列宁最后得出结论:\textbf{要么是经济政策的根本改变,要么是他的政府被暴力推翻。}

\section{工农联盟的政治经济学:布哈林的间接过渡理论}

俄国1914年至1921年间的社会经济崩溃,在现代史上是绝无仅有的。赫伯特·乔治·威尔斯
对1921年的印象是:它是“\textbf{无法修复的断裂}”之一。除了产出的缩减,俄国的城市
人口大量减少,俄国的战时伤亡很可能是所有交战国中最大的,同时,大范围的饥荒迫在眉
睫。\textbf{党外的无产阶级机构已经丧失实权,工人阶级在战胜反革命的过程中几乎毁灭
  殆尽。}

1921年3月,列宁得出结论,布尔什维克专政的继续存在和经济的复苏,需要从战时共
产主义的过渡道路上“退却”。但是,他希望新经济政策能把改进了的“国家资本主义”发
展模式——这一模式让人联想到革命刚刚结束的时期——纳入其中。\textbf{他竭力(不是很成
功)鼓励外国资本进入苏维埃国家的合资企业中,他还力求(较为成功)外交关系的正常化,
同时敦促共产主义者“学会贸易”}

所有这些,只不过是为了物质经济的恢复,不能代表一个连贯一致的模式。它还具有党
内左派指出的\textbf{具有明显的复辟主义危险性的特征}。此外,在1922年末和1923年,
列宁对政权中的\textbf{官僚主义作风和具有沙皇专制特色的行政权力滥用},越来越感到
不安。在持续的孤立和“小资产阶级的包围”中,他逐渐认识到,理想的社会主义的前途,
主要希望在于保持党的精英们对\textbf{共产主义价值观的忠诚}。同时,他认为真正迈向
社会主义的经济进步,是有可能在新经济政策下实现的,\textbf{“退却”通过稳固“工农
联盟”而成为一种进步},在这个联盟中,无产阶级可以通过工业的进步恢复活力,农民通
过合作社的发展得以改造。

正是在这一基础上,1923年后,布哈林提出了他的第二种过渡模式。与他1920年的《过渡时
期经济学》相比,布哈林现在提倡通过\textbf{迂回}的途径,\textbf{依靠非社会主义形式
  的增长实现社会主义}。结果,他的新方案遭到左翼反对派的强烈批评,但是,在1923年后
的几年里,它如正统的列宁主义一样对占统治地位的党的派别产生了支配性的影响。然而,
布哈林的新经济政策模式,并不表明与《过渡时期经济学》的完全决裂。两者都建立在接受
过渡时期布尔什维克先锋主义的基础之上;每一个模式都得到相同的历史唯物主义解释的辩
护,在这种解释中,辩证法实际上等同于现代社会学中的功能主义均衡论;布哈林1920年提
出的许多具有实质意义的观点,在他改进过的过渡思想中有了新的表述。\textbf{长期目标
  没有改变:包括农业在内的完全社会化的经济,以及市场关系的消除。}从现代意义上讲,
布哈林从来就不是一个“市场社会主义者”。因此,他并没有完全否定他早期的方案,认为
在当时的情形下,它是一种适当的政策。在布哈林看来,\textbf{主要的不同是情况发生了
  变化。}

布哈林认为,在新经济政策的情况下,向社会主义的进步取决于两个主要因素:\textbf{大
  工业的扩张和合作社的发展}。这些主张相对而言争议较少。\textbf{这时候,几乎所有的
  布尔什维克都假定,国有部门的发展与社会主义关系的扩大具有相同的含义;没有人否认
  合作社既能削弱农民的个人主义,又可以从贸易活动中挤出私人资本。在工业如何扩张和
  合作社为什么是实现农业社会化足够强大的力量问题上,布哈林受到最严厉的批评。}

布哈林声称,\textbf{国有工业取决于农民需求的增长,这些需求最终为消费品提供了市
  场。}在这种联系中,布哈林批评了杜冈-巴拉诺夫斯基对扩大再生产的分析,\textbf{在
  杜冈的分析中,消费需求被认为是无关紧要的},布哈林也\textbf{呈现出}他在《有闲阶
级的经济理论》(参见以上第五章、第十一章和第十三章第7节)中予以抨击的\textbf{奥地
  利边际主义对他的影响}。布哈林承认,合作社在过去被正确地解释为一种有助于资本主义
发展的组织。但是无产阶级专政必然改变它们的性质。正如小资产阶级农业通过帝国主义国
家融入现代资本主义一样,无产阶级国家将把俄国农业关系整合进正在发展的社会主义综合
体中。这将有助于对合作社的进一步的激励,不管这些合作社是涵盖了生产活动还是仅限于
流通。因此,对布哈林而言,就像在《帝国主义和和世界经济》以及《过渡时期经济学》中
那样,政治仍然是一个关键变量。

布哈林认为,在无产阶级和农民之间存在长期的工农联盟的基础;但工农联盟是个微妙
的问题。\textbf{任何人为地加速工业增长的尝试都会扰乱经济的比例,造成国有部门的
“销售危机”,同时,由于要求农业部门提供更多的资源,会威胁到工人和农民之间的政治
联盟。现实要求党承认苏维埃社会主义是“落后”的,进一步的发展可能不得不以“蜗牛般
的速度”进行(参见本章以下第7节)。}

然而,在布哈林看来,这不是俄国例外论的问题。国内的情形反映了全球经济的整体结
构。小资产阶级在整个世界的阶级构成中占据主体地位,尽管如此,基于与《帝国主义和世
界经济》(参见以上第十三章)中概括的相同的原因,向社会主义转化的时机已经成熟。布
哈林认为,随着革命的传播,俄国的情形将会在其它国家出现,\textbf{无产阶级和农民的
工农联盟将具有普遍的重要意义}。而且,\textbf{与帝国主义战争和革命性的内战必然地
联系在一起的消极的再生产,将会产生类似于俄国新经济政策开始时出现的那种经济崩溃。}因
此,\textbf{甚至对发达国家来说,向社会主义的过渡也必然是间接式的。}布哈林继续指
出,认为社会主义革命在形式上类似于资产阶级革命是错误的。与封建社会的资产阶级不同,
无产阶级是被剥夺的阶级;它只能在推翻资本主义后才能开始自己的文化发展。这进一步强
化了共产主义政党按照列宁主义高度集中的原则组织起来的必要性,这一点已经在所有的布
尔什维克的实践中被证明了。然后,布哈林再一次在新经济政策具有的范式意义的基础上,
就如同《过渡时期经济学》在内战的条件下仍具有普遍意义一样,给出了他对俄国向社会主
义过渡的理解。

\section{苏维埃关于西方和过渡问题的观点}

在这一点上,有必要偏离主题考察一下苏维埃对世界资本主义经济前景的看法。对过渡
的争论来说,这是一个具有重要意义的问题,\textbf{因为西方持续存在的不稳定使得革命
的希望和潜在的危机同时存在。}存在希望是因为,如果没有这种不稳定,在发达资本主义
国家发生社会主义革命的可能会很小;存在对革命而言潜在的危险是因为,经济上的灾难和
帝国主义国家敌对的增强,唤醒了对苏维埃国家重新进行军事侵略的幽灵。有时候,对西方
的看法包含一个更为复杂的视角,我们会在本章以下第8节看到,但这是问题的实质。然而,
事实上,共产主义者对这些问题进行的分析,并不怎么令人印象深刻。这一点令人感到奇怪,
这不仅是因为这些问题具有极端的重要性,而且也因为大量的知识精英当时正集中在莫斯科。
除了本章关注的——布哈林、普列奥布拉任斯基、托洛茨基和某个时期的列宁——\textbf{还有
很多如康德拉季耶夫、福克纳和恰亚诺夫这样才华横溢的孟尔什维克和新民粹主义者,他们
在整个20世纪20年代仍然可以自由地工作和出版著作。}

似乎有两个原因,使他们未能对世界资本主义进行充分的评价。\textbf{首先,他们都专注
  于国内经济发展问题。}很容易理解为什么他们的精力都集中在社会过渡、经济增长和社会
主义计划等这些令人兴奋的新问题上,而不是放在被认为最终注定要灭亡的社会秩序内部矛
盾的研究上;\textbf{第二,20世纪20年代列宁主义正统的重要影响力不断增加,逐渐阻碍
  了与官方布尔什维克主张的列宁思想有冲突的任何原创性思想的出现。}因此,康德拉季耶
夫著名的\textbf{长波理论},被党内占统治地位的派别和左翼反对派视为异端邪说。

类似的因素也使得苏联以外的共产主义者,压制了原创性思想的发展。例如,德国政党
教条式的独裁主义,反对而不是赞同像弗里茨·斯滕伯格和亨利克·格罗斯曼这样的马克思主
义者(参见以上第十四章和以下第十六章),而且在20世纪20年代的法国、意大利和美国,
都不存在相对来说高水平的共产主义运动。英国的政党较为幸运,拥有一位年轻且极具才华
的成员——\textbf{莫里斯·多布}。但是,多布同样也着迷于苏维埃社会主义内部的经济问题,
把对当代资本主义进行详细剖析的任务留给了像英籍印裔人拉贾尼·帕尔姆·杜德这样天分稍
逊的同志。

当时,对20世纪20年代国际资本主义进行最有影响的研究的非俄国经济学家,是流亡在莫斯
科的匈牙利人\textbf{尤金·瓦尔加},即使他算不上是一位敏锐的思想家,也该是有原创性
的思想家,他的重要地位在很大程度上通过默认获得的。20世纪20年代早期,瓦尔加重视资
本主义的\textbf{暂时“稳定”问题},这在实际上相当于只是承认世界经济已经
从1920年至1921年的深度萧条中\textbf{复苏}了,而且道威斯计划正在限制赔款危机带来的
破坏。瓦尔加后来认为,\textbf{复苏将是短暂的,因为工业合理化正在不断提高资本有机
  构成,减少就业和工人的消费,并无情地导致进一步的危机。}鉴于1929年后发生的一切,
可以说这一观点是极具先见之明的,但是,它只得到了并不严密的论证,从而很容易受到对
消费不足持有反对意见的传统马克思主义者的批评。总之,\textbf{共产主义者对待消费不
  足的态度是以明显的不安为特征的,这一点可由党的教科书在这一问题上的前后不一致看
  出来。然而,在缺乏任何一种利润率下降模型时(参见以下第十六章),布尔什维克经济
  学家无法提出更好的资本主义危机理论。}

回到布哈林,\textbf{他是唯一一位认真地对待资本主义长期稳定的可能性的重要的共
产主义理论家,资本主义的长期稳定的确是在新经济政策下实现渐进主义的苏维埃工业化的
必要的前提条件。}但布哈林没有始终如一地坚持这一立场,他的观点取决于“\textbf{有
组织的资本主义}”的概念,但是这一概念既因与像希法亭和考茨基这样的革命的社会民主
党敌人的联系而受到玷污,而且也存在分析上的缺陷(参见以上第十四章)。1929年后,布
哈林的党内的敌人很容易就这两点败坏他的名声。到那时,\textbf{布哈林向社会主义过渡
的思想因为国内的原因被斯大林主义者放弃了(参见本章以下第9节)}。但是,这种情况的
出现也存在国际上的原因。1926年后,当苏维埃的外交和第三国际的政策遭到几次重大挫折
后,俄国不久之后将遭到军事进攻的担心进一步增强。因此,更大的力量被放在进行更为迅
速的工业化的需要上,理论家们在调整对西方的看法方面感受到了相当大的压力。

1926年后,斯大林偏离了布哈林的思想,尤其是在1928年同布哈林的思想发生了明显的
决裂,这常常被描绘为“\textbf{向左转}”。这是适当的,因为\textbf{斯大林的思想更
加接近于左翼反对派持有的某些思想,其主要理论家是托洛茨基和普列奥布拉任斯基。}

\section{社会主义原始积累:普列奥布拉任斯基的二元经济过渡理论}

左翼反对派将布哈林的新过渡理论(本章第4节对该理论已作概述),看作是修正主义的
形式,认为它在经济上是错误的,在政治上是危险的。托洛茨基对左翼反对派对布哈林的整
体批判作了最佳阐述(参见本章以下第8节),但是,左翼反对派最具原创性的经济学家是
叶·普列奥布拉任斯基。

布哈林一样,在革命之后最初的几年,普列奥布拉任斯基是战时共产主义的热情支持
者。20世纪20年代初,他也承认\textbf{战时共产主义终究不是过渡的长期模式}。不亚于布
哈林,他也致力于支持新经济政策(同时也不否定战时共产主义的必要性),相信它构成了
社会主义建设的一般范式。然而,他的理论关注的焦点与布哈林的存在着很大的不同,他更
多关注革命后经济中的具体情况。

在普列奥布拉任斯基看来,\textbf{国有部门的扩张是社会主义增长的主要变量},最为重
要的是国有部门与资产阶级和小资产阶级关系占支配地位的非国有经济相比时的增长率。
1917年十月是一场\textbf{双元革命}。正如托洛茨基所言:
\begin{quotation}
  为了建立苏维埃国家,需要把两个历史性质完全不同的因素接合起来并相互渗透:农民战
  争——资产阶级发展的曙光期所特有的运动,和无产阶级暴动——标志着资产阶级衰落的运动。
  这就是1917年的本质。
\end{quotation}

尽管普列奥布拉任斯基承认,无产阶级和贫穷农民之间结成联盟是可能的,但他认
为,\textbf{只有通过无产阶级力量的不断增长和限制富农与“耐普曼”的影响,这种联盟
  才是成功的。双方在激烈争夺对农民群众的领导权。}它不只是一个国内力量为复辟俄国资
本主义效力的问题。\textbf{世界经济}也有着严重的危害。苏维埃工业是无效率的,只能通
过\textbf{外贸垄断}来保持运营,而且面临着\textbf{资本主义世界市场的持续压力}。外
部资产阶级力量和俄国新兴的资产阶级一起形成与农民的联盟,反对无产阶级专政。

因此,普列奥布拉任斯基主张,\textbf{国有工业必须在经济增长中占支配地位。}这
会直接增强社会主义关系的力量,同时,\textbf{通过向农业集体化(没有大规模的机械化
是不可能的)提供资源,可以间接地产生同样的效果。合作社有助于这一过程,但它并不
是对生产关系向社会主义转变的真正的替代:这是历史唯物主义的逻辑。}这里产生了一个关
键的经济问题,布哈林理解的社会主义积累——也就是,\textbf{依靠国有部门内部产生的剩
余进行积累——不可能为工业按要求的比例发展提供充足的资源}。与马克思对资本主义起源
的分析相似,普列奥布拉任斯基提出了“\textbf{社会主义原始积累}”的概念。
\textbf{扩大社会主义积累所需的资源,还必须从非社会主义经济中榨取。}在当时的情况
下,这就意味着它们主要来自\textbf{农业部门}。因此,\textbf{工业和农业的交换不可
能遵循价值规律,就如同苏维埃经济和世界资本主义经济之间的贸易无法遵循该规律一样。
从内部看,为了国有部门社会主义原始积累的利益,必然会存在不等价交换。外部经济关系
不得不通过外贸垄断来开辟,这同样是为社会主义原始积累服务。}

普列奥布拉任斯基经常使用社会主义原始积累“规律”这一术语,因为他认为俄国革命
如果要持续下去,不存在能够替代这一增长策略的其它选择。\textbf{社会主义原始积累规
律和价值规律之间存在持续的张力。}两大规律事实上是对完全不同的生产方式的抽象表达,
\textbf{内部和外部的资本主义力量迫切要求进行等价交换和自由的国际贸易,而社会主义
的存活必然要求中止这两点。}在这种冲突中,苏维埃工业有一个重要的优势,尽管在当时
它不如资本主义现代工业有效率,但普列奥布拉任斯基相信它具有\textbf{组织上的优势}。
现有的社会主义关系使得对国有经济的不同部门,社会主义工业、国内私营经济与世界市场
之间的关系进行有计划的协调成为可能。\textbf{经济计划}是无产阶级专政的王牌;如果
正确地使用了它,俄国社会主义原则上没有理由不继续前进。


随后,当普列奥布拉任斯基开始考察苏联在20世纪20年代后半期的具体情况时——正如他所说
的,为社会主义原始积累的代数学提供算术例子——普列奥布拉任斯基变得\textbf{极为悲观}。
他在这里证明,无论与布哈林还是与托洛茨基相比,他自己都是\textbf{更为优秀的经济学
  家},布哈林和托洛茨基倾向于把他们的辩论局限于\textbf{理论概括}的层
面。\textbf{普列奥布拉任斯基把他的经验研究和对马克思再生产图式进行的卓越的概念扩
  展结合在一起。}对罗莎·卢森堡对再生产图式纯粹的抽象所作的批判印象深刻,普列奥布
拉任斯基正式地改造了它们,以便于对单一社会形态中存在多种生产方式的情形进行分析,
他认为这种情形是最为典型的。对于苏联来说,跨部门交换是在不同的经济体系中进行的,
每一种经济体系都有其自己的部门。对社会主义原始积累的经验分析要求适当地关注分配原
理,以及那些与非生产性消费和流通活动中私人资本的重要性相联系的原理。通过条理清晰
地展开自己的推理,\textbf{普列奥布拉任斯基推断,在俄国持续孤立的状况下,社会主义
  原始积累是一个无法解决的难题。彻底地解决这一问题有赖于将革命扩展到国际范
  围;“一国社会主义”是不可能的。}

\section{布哈林和普列奥布拉任斯基的比较}

1924~1928年,布哈林和普列奥布拉任斯基就他们之间在向社会主义过渡问题上的分歧,展
开了激烈的争论。这也是左翼反对派和布尔什维克党中占统治地位的派别之间发生的更广泛
的冲突的一部分。不同群体坚持的一般的理论立场,将在本章以下第8节和第9节考察。这里,
我们集中关注布哈林和普列奥布拉任斯基在经济理论方面发生的冲突。

他们之间的分歧是围绕\textbf{如何发展工业}展开的。在布哈林看来,工业的扩张取决
于\textbf{农民需求的增长,特别是消费品市场上农民的需求。}布哈林反对罗莎·卢森堡和
杜冈—巴拉诺夫斯基的观点,他将卢森堡的\textbf{长期消费不足}视为一派胡言,认为杜冈—巴
拉诺夫斯基相信工业可以独立于消费需求而发展同样也是错误的。布哈林对卢森堡的批评远
比对杜冈—巴拉诺夫斯基的批评严厉(参见以上第五章和第十章):事实上,布哈
林\textbf{严重歪曲}了杜冈—巴拉诺夫斯基有关第\Rnum{2} 部类发生收缩的可能性的观点。
作为对普列奥布拉任斯基进行攻击的基础,布哈林对杜冈—巴拉诺夫斯基经济学的抨击无论如
何都不十分中肯。\textbf{苏联经济不是一种竞争性的资本主义经济,社会主义原始积累被
  设想为只是国有部门相对迅速的增长,而不是消费的绝对下降,更不是农民消费的绝对下
  降。}普列奥布拉任斯基坚定地认为,\textbf{工人阶级的生活水平作为无产阶级恢复活力
  过程的必要组成部分,必然会上升。}

杜冈—巴拉诺夫斯基强调\textbf{比例协调}对均衡增长来说是至关重要的。普列奥布拉任斯
基同意“比例协调铁律”限制了国家积累。的确,确保比例失调被限制在可管理的范围之内
的问题,困扰着普列奥布拉任斯基。他正确地认识到主要的困难在于“\textbf{商品荒}”。
与战前的情况相比,革命已经从根本上改变了国内生产的工业品的供求关系。从总体上看,
需求增加了,供给减少了。因此,在新经济政策实施期间,存在对国有部门产出的过度需求,
而不是像布哈林想象的那样,存在潜在的购买力不足。1925年,当普列奥布拉任斯基认识
到\textbf{未来的增长要求大量的固定投资}时,他的观点得到了进一步的加强。迄今为
止,\textbf{工业的复苏建立在恢复对现有生产能力的充分利用的基础之上,想要保持持续
  的发展,必须迅速地提高现有的生产能力。}在普列奥布拉任斯基看来,短期内,与商品荒
相联系的问题将会加剧,但\textbf{长期地看,投资不足会使这些问题变得难以解决,并迫
  使工农联盟破裂。}

布哈林没有能预见到这一问题,而且他的经济分析就妥善处理这一问题来说,设计得也不好,
因为处理这个问题要求缩减农民需求的增长。无论如何,布哈林强调的完全与此相反。在他
看来,经济发展有一个必然的特定的顺序。农业产出的增加提高了农民的购买力,进而推动
了轻工业的发展,相应地扩大了对重工业产品的需求。布哈林坚持认为,他不仅仅概括了过
去的经验联系,而且指出了一个必然的具有一般性的经济原则。布哈林倾向于将其他任何一
种增长方案贬斥为“\textbf{应用杜冈主义}”。普列奥布拉任斯基把扩大生产能力的投资作
为工业复苏的手段的认识,与布哈林的分析完全相反:\textbf{重工业的扩张必将快于轻工
  业的扩张}。只有到那时才能在工业消费品产出方面出现意义重大的\textbf{长期扩张},
这种扩张对消除商品荒来说是必要的。然而,普列奥布拉任斯基只是将对不平衡增长的呼吁
视为一个技术经济学问题。只有到了20世纪30年代,重工业在苏联经济中获得特权地位才成
为一个原则性问题。

考虑到这些差别,就很容易理解为什么普列奥布拉任斯基强调\textbf{总计划}的迫切必要性。
总计划对过渡来说,而不仅仅对社会主义的未来来说,是必不可少的,因为它使得比例失调
引致危机之前,可以被预则并加以纠正。由于布哈林对苏维埃经济问题的真实本质持完全不
同的观点,\textbf{他更倾向于支持市场的自治}。\textbf{这事实上是他反对工业与农业之
  间不等价交换的经济根源。}布哈林好像完全误解了普列奥布拉任斯基有关限制价值规律的
影响是现实的需要的理论观点。他也不理解普列奥布拉任斯基提议增加工业品批发价
格,\textbf{并不意味着新的“剪刀差危机”的开始},在这种危机中贸易条件会不利于农民,
减少他们在市场上交易粮食的激励。对普列奥布拉任斯基来说,整个问题很大程度上是一个
务实的问题。\textbf{保持较低的工业品批发价格(这是20世纪20年代的官方政策),意味
  着流通中的私人资本以工业和农业为代价积累自身,但是,商品荒却持续存在:受益者
  是“耐普曼”(中间商),}他们从工业中购买到便宜的商品,并以更高的价格卖给农民。
普列奥布拉任斯基认为,\textbf{最终需要降低所有工业品的价格,以避开来自世界市场的
  压力},在这一点上是非常明确的。

在所有这些问题上展开的论战,因每个理论家不时地使用\textbf{情绪化的语言}而加
剧。1924年,普列奥布拉任斯基提到了\textbf{“剥削”农民的需要},在苏维埃经济结构
中对农民的“剥削”的地位类似于“殖民地”在资本主义原始积累中的地位另一方面,布哈
林谈到工业化会因被号召“使自己发财吧”的“农民的抱怨”,而以“蜗牛般的速度”爬行。
每个人都抓住其它人用语上的不当。这不只是因为缺少了同志般的友善。布哈林学派、占统
治地位的党的派别中的斯大林主义者和左翼反对派,都相信他们抨击的那些表达形式揭示了
他们对手的学说中隐藏的内容。这些怀疑有一定的正确之处。斯大林和布哈林有充分的理由
主张“非正统”的不断革命论涉及了更深层次的问题,不断革命论与他们对列宁主义的解释
存在冲突。他们的反对者,在官方的思想和政策中看到了可能会\textbf{导致资本主义复辟
的重大的经济危机的萌芽},同样是正确的。

\section{一国社会主义或不断革命}

作为经济学家,普列奥布拉任斯基远比布哈林优秀,但是他也是一个较为专业化的思想家,
依赖于托洛茨基的思想,普列奥布拉任斯基提供了有关社会主义原始积累的整体观点。这是
一种合理的劳动分工。如我们在以上第十二章已看到的那样,托洛茨基进行严格的经济分析
的能力非常有限。然而,作为一个\textbf{创新型的马克思主义者},能与之匹敌的人几乎
没有。托洛茨基比包括列宁在内的其他任何布尔什维克理论家,都更有条件对革命后的俄国
的情形进行具有思想上的连贯性的解释。\textbf{不断革命论正确地分析了十月革命的阶级
动力学。同时,它最小化了对可能的社会主义收益的预期,这种收益是布尔什维克夺取政权
后可能实现的。不断革命论也依赖于不平衡和综合发展这一更具一般性的概念,后者突出了
20世纪20年代苏联的形势中存在的矛盾(参见以上第十二章)。}

然而,托洛茨基的独创性思想,在一个关键之处\textbf{误入歧途},布哈林和斯大林很快察
觉到这一错误(以及其他一些实际上是他们自己的想像虚构出来的错误)。最显而易见的
是,\textbf{布尔什维克的政权在革命孤立的条件存活了下来。}1924年,斯大林用列
宁\textbf{帝国主义对抗背后不平衡发展的概念}解释了这一点。西方资本主义国家之间存在
的分歧,阻碍了它们对苏联进行\textbf{统一的持续的军事进攻}。虽然托洛茨基承认这是事
实,但他不承认他的错误源于他自己有关帝国主义的观点,\textbf{这种观点(自相矛盾地)
  低估了发达资本主义国家之间不平衡发展的重要性(参见以上第十二章和第十三章)。}因
此,斯大林声称,托洛茨基从来没有真正理解现代资本主义的本质。在斯大林看来,在未来,
在苏联建设社会主义时,帝国主义国家之间\textbf{持续存在的分歧},可以被俄国的外交和
第三国际消除军事威胁加以利用。扩大革命尽管很重要,但不是生存所必需的。

与此相联系的是对托洛茨基低估农民的重要性的指责。经常用荒谬的语言来表述的这种
指责,主要建立在布哈林工农联盟的理论基础之上,使用这种理论的术语,布哈林和斯大林
指出,即使是在革命仍然保持孤立的情况下,内部的冲突并不必然毁灭革命。革命的国际主
义没有被忽视,但托洛茨基的不断革命论很大程度上夸大了它的重要性。

布哈林和斯大林也指出列宁的著作中存在类似的主题,他们注意到列宁的俄国革命进程
的思想与托洛茨基的不同(参见第十三章)。布哈林对这一主题进行了阐述,他将1917年革
命看成是无产阶级和农民的\textbf{联合革命},而不是像托洛茨基一直认为的那样,是
\textbf{两种革命的融合}。此外,布哈林认为,在无产阶级领导下,不同阶级之间的关系
本质上是和谐的。无产阶级专政作为一种新的国家形式,已经从总体上改变了阶级关系,苏
维埃社会代表了真正的新的统一体。布哈林甚至承认民粹主义思想在这个时候在一定程度上
看来是正确的。他的含意是清晰的:\textbf{托洛茨基未能详细说明俄国革命的性质,左翼
反对派将农民视为恢复资本主义的力量的观点是不合时宜的。同时,这时的布哈林认为,一
国社会主义的可能性从一开始就内在于布尔什维克的努力中。}

布哈林通过重新评价国际革命的经济含义强化了这种认识,先前所有的布尔什维克,包括托
洛茨基,都以一种草率的方式对待国际革命问题。正如在本章第4节提到的,\textbf{布哈林
  强调小资产阶级关系在世界经济中的主导地位,坚持认为不可避免的革命的经济成本,将
  很大程度上减少任何社会主义革命可能继承的物质遗产。}联系他关于无产阶级文化剥夺本
质的新观点,布哈林得出结论:\textbf{国际革命不可能为苏联提供多少经济帮助。}在所有
这些观点中,存在着对马克思主义的重大修正,尤其是当它和布尔什维克先锋主义结合时,
并且其大部分内容也被托洛茨基和普列奥布拉任斯基所接受。

布哈林和斯大林都没有忽视,根据国际革命的可能性以及它最可能采取的形式,来重新组织
他们的观点。欧洲资本主义仍然是在战后已经“\textbf{稳定}”下来的有组织的实
体。\textbf{这一观点意味着西方不存在即将来临的革命,而且对苏联的武装干涉在可以预
  见的将来并不构成威胁。}与此同时,布哈林坚持认为,资本主义在外围的发展已经停止
了;\textbf{帝国主义剥削现在具有了十足的寄生性。这为殖民地地区反对帝国主义的斗争,
  而不是为无产阶级革命创造了基础。}以列宁的著作中关于民族自决权(布哈林在这个观点
上与列宁和解了)的观点为基础,布哈林认为,甚至\textbf{殖民地的资产阶级都可能具有
  了进步的作用}。因此,布哈林得出结论,第三国际不可能切实支持托洛茨基把不断革命论
推广到所有落后的资本主义地区的企图(参见以上第十二章)。

所有这些被布哈林派所坚持(到1927年时也被斯大林派所坚持)的思想,被托洛茨基认为是
脱离了真正的列宁主义(他越来越把它等同于自己的思想);是\textbf{革命发生了退化的
  征兆},是为铺平资本主义复辟道路的热月反动提供了意识形态的外衣。1926年后,托洛茨
基把其反对者的思想的核心,视为是\textbf{一国社会主义的教条}。这有一定的合理
性。20世纪20年代中期,布哈林和斯大林低估了苏维埃经济结构中固有的内部矛盾,对其发
展前景过分乐观。此外,不应该太在乎斯大林学说的字面含义:\textbf{他和布哈林都没有
  精确地定义社会主义},甚至到那时\textbf{他们也从没有宣称“完全的”社会主义可以单
  独在俄国实现}。相反“一国社会主义”学说充当的是\textbf{击败左翼反对派这一党的派
  别的辩论工具},是拒绝托洛茨基向社会主义过渡的提议的象征。

这些提议,必然包含改变战争爆发前的立场,因为事实明显地未能完全证明托洛茨基早
期的思想。托洛茨基开始重新评价国际资本主义的状况。尽管国际资本主义比托洛茨基在
1917年预计的更富弹性,但它仍处在衰退过程中;问题只是比他原来预想的复杂了一点。
\textbf{“资本主义发展曲线”具有双元结构:既有长期趋势又有周期性波动。}后者的表
现和以前很相像,前者明显地变得平缓或者是开始下降。托洛茨基认为,\textbf{任何资本
主义的稳定都将是非常短暂的。}资产阶级社会不再是欧洲的进步力量,可以预计
\textbf{革命的情形会频繁再现}。只有在外围地区,资本主义明显地处于上升期,美国的
情况可能也是如此。

托洛茨基认为,这为苏联结束政治上的孤立提供了机会。而且这意味着\textbf{为东方
的社会主义革命开辟了道路。殖民地和半殖民地落后资本主义体系中的无产阶级可以复制俄
国工人阶级的成就},因为经济结构有助于已经扩展到俄国之外的不断革命(参见以上第十
二章)。只要“十月的教训”体现在第三国际的政策中,只要\textbf{抛弃布哈林和斯大林
的“孟什维主义”},成功就是必然的。

托洛茨基关于欧洲资本主义处于下降期的思想,也为他提供了一个解决社会主义原始积
累中固有困难的明显的方法。\textbf{他认为政治上的孤立,并不意味着经济上的孤立。}
资本主义对市场的需求,可以被用于把俄国经济重新融入世界市场中。这\textbf{必然通过
计划来实现——当然不能放弃对外贸的垄断},这可以提供暂时的和永久的利益。对消费品的
进口可以被用来克服“商品荒”,根据比较优势形成的专业化,将极大地提高国有工业的效
率。

\textbf{融入世界市场}的想法,也构成托洛茨基反对一国社会主义可能性的重要经济思想,
因为他正确地理解了这一学说,认为它意味着在苏联自有资源的基础之上进行自力更生式的
经济发展。像普列奥布拉任斯基一样,他强调苏联的工业\textbf{无法在遵循价值规律的基
  础上参与国际竞争,}这对于缩小效率上的差距来说是必须的。如果不能做到或不能迅速做
到这一点,苏联经济会愈加难以抵御内部和外部资本主义要求在非管制的基础之上开放经济
的压力。国有部门将难逃一劫,苏联社会主义也将遭遇同样的命运。更为抽象的
是,\textbf{托洛茨基将“历史的基本规律”描述为“胜利最终属于能为人类社会提供更高
  的经济水平的经济制度”。社会主义革命向其它国家的扩展,将有利于生产力比单纯依赖
  国内环境时获得更充分的发展},没有这种革命的扩展,就没有一个国家,更不要说落后的
苏联有望超越国际资本主义取得的经济成就。因此,托洛茨基能够坚持其最初主张中的要
点:\textbf{如果资本主义的稳定被证明是持久的,那么俄国的革命注定要消亡。}

在托洛茨基看来,解决苏联现实情况中存在的矛盾的关键是\textbf{政治改革}。\textbf{布
  尔什维克放弃对权力的垄断是不可能的,甚至将党内的派系合法化(自
  从1921年“退却”到新经济政策后就被禁止了)也是不可能的。}但是,托洛茨基认
为,\textbf{通过承认批评是合法的,并把组织置于普通党员的控制之下,重新恢复党的无
  产阶级特征是非常有必要的。}这是巩固无产阶级力量的关键因素,类似的措施也可以有效
地、普遍地扩大到国家制度中。总之,对托洛茨基来说,\textbf{必须反对“官僚主义”。}他
似乎认为,一旦做到了这一点,他的思想就会明显地获得胜利,因为它们才是真正的列宁主
义,因而也代表了无产阶级的真正利益。结果,\textbf{社会主义原始积累规律就将在经济
  政策中占据主导地位,第三国际的资源就将被正确地动员起来终结政治上的孤立。}

正如托洛茨基清晰地表明的那样,\textbf{他的整个立场建立在他将国际资本主义视为
正在衰落的体系基础之上。}然而,缺乏对西方资本主义经济失灵确切原因分析这一缺陷,
一直存在于托洛茨基革命之后的著作中。他对这一问题进行的分析是深入的,但却不够严谨。
此外,\textbf{尽管他相信资本主义无法保持长期的稳定与他的经济融合主义和政治上的不
妥协是吻合的,但是,他是以几乎无法相互兼容的不同的方式表述后两种观点的。}在托洛
茨基看来,苏联经济必将\textbf{和平地融入}资本主义世界市场中,同时,共产党也将为
通过作为革命的序幕的罢工展示自己做好准备。

1927年底,托洛茨基相信已经没有多少时间可以用来挽救革命了:“热月式的危险即将
来临”。这反映出他认为苏联的内部矛盾激化了。确实如此。这时,当斯大林将左翼反对派
成员开除出党并将他们在国内流放时,\textbf{“商品荒”以严重的粮食收购危机的形式显
露出来。1928年期间,这打破了布哈林派和斯大林主义者之间的联盟,使得斯大林开始越来
越多地使用被其击败了的对手的语言。}

\section{斯大林主义的解决方法}

“斯大林个人的不幸……在于(他的)理论才能和集中在他手中的国家权力之间存在着巨大
的差距。”这是托洛茨基在1927年9月写下的内容,这恰当地描绘了斯大林的情况。
\textbf{他对马克思主义学说发展作出的贡献的确非常少。}然而到1928年时,以斯大林为
首的党派对党进行了有效的控制,而且党几乎完全控制了整个国家。此外,构成斯大林追随
者的核心的官员们,越来越处于其个人统治之下。\textbf{正是这种权力,而不是斯大林的
理论深度,为他提供了在这个十年结束时突破困境的能力。}

自1928年初开始,让人联想到\textbf{战时共产主义的粮食征用}被用以克服农民粮食销售的
不足;1929年至1933年之间,这些措施\textbf{被扩大为实行强制的集体化}。这通
过\textbf{毁灭一切农民独立性的残余,缓解了快速工业化中来自农村的制约;早已存在的
  来自苏联无产阶级(他们的生活水平在1929年后大幅下降)的抵制变得不再可能。}在很短
一段时间内,党的专政就变成了\textbf{个人极权主义}。整个社会被迫成为“\textbf{生产
  前线}”,生铁的产出被官方视为是向社会主义前进正确无误的象征。

左翼反对派和布哈林派都将斯大林的解决方法,看作是证明了他们对对方的批判。在布哈林
看来,粮食危机的爆发是因为左翼反对派的“\textbf{超工业化}”政策造成的,这种政策
在1926年后,对斯大林派和计划部门的经济学家产生了越来越大的影响。布哈林长期以来坚
持认为,\textbf{左翼纲领将导致“二次革命”和完全官僚化的警察国家}。另一方面,很多
反对派,包括普列奥布拉任斯基,在斯大林“向左转”后与他达成了\textbf{心神不安的和
  解}。对他们而言,他们对建立在布哈林理论观点基础之上的过去的政策的后果的预言被证
实了。斯大林主义的令人畏惧的工业化的含义,将在本书第二卷中加以讨论。

\chapter{亨利克·格罗斯曼和资本主义的崩溃}

\section{引言}

以上第十四章,我们看到,重新恢复活力而且明显处于稳定状态的世界资本主义经济,给
20世纪20年代马克思主义分析带来的影响。一些正统的社会民主党人,比如考茨基和希法亭,
披上了\textbf{早期修正主义者}的外衣,认为在“有组织的资本主义”时代,大规模的经
济危机已不大可能。一些\textbf{新卢森堡主义者}(最著名的是弗里茨·斯滕伯格)反对这
种观点,他们坚持认为,\textbf{由实现问题产生的危机是不可避免的}。托洛茨基主义者
则不合时宜地坚持,\textbf{必然发生的规模更大的帝国主义战争迫在眉睫}(参见第十四
章和第十五章)。一个共同的线索把这些迥然不同的思想流派统一在一起了。那就是他们几
乎完全忽视了马克思《资本论》第三卷,特别是其中加以阐述的\textbf{利润率下降}趋势。
从这一点看,现代马克思主义危机研究中的一个重要因素,几乎完全被忽视了。本章将关注
一位马克思主义者作出的——对《资本论》第三卷中的分析作出发展,并把它应用于危机理
论——第一次认真尝试。尽管存在着\textbf{严重的缺陷},但这种分析造成的影响(从长期
来看)却是\textbf{极其深远}的。

1929年,在华尔街大崩盘发生前夕,一本622页的著作出版了,它是由一位迄今为止人
们只知道他是一个经济史学家和统计学家的波兰学者\textbf{亨利克·格罗斯曼}撰写的。在
《资本主义制度的积累规律和崩溃(包括危机理论)》中,格罗斯曼提出了一种崩溃理论,
他自称这是第一个和马克思《资本论》第三卷中的分析精神实质相一致的崩溃理论。在格罗
斯曼看来,马克思的历史观取决于生产方式辩证转变的理论。格罗斯曼认为,这意味着资本
主义“\textbf{积累有其无法超越的,绝对的经济限度}”。对于格罗斯曼来说,任何能够
称作马克思主义政治经济学的研究,都必须说明这些限度的本质。不要期望新古典理论能做
哪怕是一丁点这样的事情,\textbf{新古典经济学把自己限制在只关注个人动机而忽略了造
成过度储蓄的客观条件的陈词滥调中。}但是,马克思自己也未能对资本主义崩溃趋势做出
清晰的说明,格罗斯曼认为,后来的大多数马克思主义著述家都明显地否定了这种趋势的存
在。只有罗莎·卢森堡可以免于遭到这种批评,然而她的理论却是错误的。

格罗斯曼对自己的分析方法有着非常清楚的认识,他声称这种方法是马克思本人的。经
济崩溃的趋势必须从资本主义生产的内在本质,而不是从商品流通和交换的外在表现来推断。
因此,他的著作致力于表明:
\begin{quotation}概括经验上可观察的世界经济的趋势,这些趋势被视为资本主义最新
发展阶段的典型特征(在各种各样论述帝国主义的著作中被加以阐明的:垄断组织、资本输
出、瓜分原材料生产区的斗争,等等),是如何成为次要的表面现象的,这些现象是由作为
资本主义基本根源的资本积累的本质引起的。
\end{quotation}

这使得对诸如价格波动等经济危机的症状和存在于生产过程本身的经济危机产生的根本原
因进行区分成为可能。基于方法论的理由,格罗斯曼假设所有市场都保持供给和需求的均衡,
价格(假定等于劳动价值)和货币价值不变。\textbf{他有意回避了对这种背景下的信贷和
竞争的考察。经济危机必须由“资本的内在本质”来解释}。

这本著作以一个冗长的文献综述作为开篇,它占据了整个第一章和第二章的前六节。在
第二章的其余部分,格罗斯曼提出了他自己的经济崩溃理论,这种理论建立在对\textbf{利
润率下降}进行的马克思主义分析的基础之上,并被用于建立一个\textbf{周期性危机的模
型}。第三章评价了起“反作用的趋势”,首先是在假定的封闭经济中讨论这些趋势,然后
在世界市场中讨论。后者构成了格罗斯曼帝国主义理论的基础。在结论一章中,格罗斯曼阐
明了他的分析对阶级斗争和革命性变革的前景具有的意义。


\section{格罗斯曼的崩溃和危机模型}

奥托·鲍威尔1913年出版的著作的第六章,在对卢森堡《资本积累论》进行的评论中,提出
了资本有机构成不断提高的4年期资本积累的数字例子,格罗斯曼通过对这部分内容回顾,
\textbf{把鲍威尔的模型扩展到36年},并且证明这一积累过程无法无限期的持续下去。
\textbf{最终资本主义体系将不能生产足够的剩余价值,以满足(1)模型所要求的积累率,
和(2)资本家的消费。当后者下降到零时,崩溃就发生了;在这一点出现之前的很长一段
时期,严重的经济危机是很有可能发生的。}


\textbf{格罗斯曼的计算常有些细小错误,但不会影响结论。}

格罗斯曼的模型可以用表16.1加以概括。和鲍威尔一样,假定不变资本以每年10\%的比
率增长,可变资本以每年5\%的比率增长。剥削率始终保持在100\%,这样,每一年的剩余价
值等于使用的可变资本。格罗斯曼假设劳动力价值为单位变量,从而每单位可变资本代表一
名工人。假设劳动力以每年5\%的比率增长,必须积累足够的剩余价值作为追加的不变资本
($a_c$),以保证其10\%的年增长率,必须留出足够的剩余价值作为追加的可变资本
($a_v$),以保证其5\%的年增长率。资本家的消费是一个剩余量:一旦保证了积累,他们
可以消费掉剩下的一切。正如鲍威尔已经发现的那样,\textbf{均衡增长要求用于积累的剩
余价值的比例持续地上升,用于消费的比例稳定地下降。}到第35年时,消费量几乎接近于0,
如果此时生产继续到下一年,消费量就有可能降为\textbf{负值}。实际上,格罗斯曼认为,
这是不可能的,并且“\textbf{这一体系必然崩溃……[因为]资本家们遇到了麻烦,并且担
心继续运转的生产体系产生的成果会全部落入工人阶级手中。”所以即使“忍饥挨饿”,资
本家阶级也不愿再继续进行积累。}

% Please add the following required packages to your document preamble:
% \usepackage{booktabs}
% \usepackage{graphicx}
% \usepackage{lscape}
\begin{landscape}
\begin{table}[tbp]
\centering
\caption{格罗斯曼的积累模型}
\resizebox{\linewidth}{!}{%
\begin{tabular}{@{}llllllllll@{}}
\toprule
\textbf{年份} & \textbf{不变资本(a)}                                                             & \textbf{可变资本(v)}                                                          & \textbf{资本家的消费 (k)} & \textbf{增加的不变资本 ($a_c$)}                                                     & \textbf{增加的可变资本 ($a_v$)}                                                   & \textbf{总价值} & \textbf{消费的剩余价值 的比例(\%)} & \textbf{积累的剩余价值 的比例(\%)} & \textbf{利润率 $(k+a_c+a_v)/(c+v)\%$ } \\ \midrule
1           & 200000                                                                       & 100000                                                                    & 75000               & 20000                                                                     & 5000                                                                    & 400000       & 75                       & 25                       & 33.3                           \\
2           & 220000                                                                       & 105000                                                                    & 77750               & 22000                                                                     & 5250                                                                    & 430000       & 74.05                    & 25.95                    & 32.6                           \\
3           & 242000                                                                       & 110250                                                                    & 80539               & 24200                                                                     & 5511                                                                    & 462500       & 73.04                    & 26.96                    & 31.2                           \\
4           & 266000                                                                       & 115762                                                                    & 83374               & 26600                                                                     & 5788                                                                    & 497524       & 72.02                    & 27.98                    & 30.3                           \\
5           & 292600                                                                       & 121550                                                                    & 86213               & 29260                                                                     & 6077                                                                    & 535700       & 70.93                    & 29.07                    & 29.3                           \\
6           & 321860                                                                       & 127627                                                                    & 89060               & 32186                                                                     & 6381                                                                    & 577114       & 69.7                     & 30.3                     & 28.4                           \\
7           & 354046                                                                       & 134008                                                                    & 91904               & 35404                                                                     & 6700                                                                    & 612062       & 68.58                    & 31.42                    & 27.4                           \\
8           & 389450                                                                       & 140708                                                                    & 94728               & 36945                                                                     & 7035                                                                    & 670866       & 67.32                    & 32.68                    & 26.4                           \\
9           & 428395                                                                       & 147743                                                                    & 97517               & 42839                                                                     & 7387                                                                    & 723881       & 66                       & 34                       & 25.6                           \\
10          & 471234                                                                       & 155130                                                                    & 100251              & 47123                                                                     & 7756                                                                    & 781494       & 64.63                    & 35.37                    & 24.7                           \\
11          & 518357                                                                       & 162886                                                                    & 102907              & 51835                                                                     & 8144                                                                    & 844129       & 56.67                    & 44.33                    & 20.6                           \\
15          & 758925                                                                       & 197988                                                                    & 112197              & 75892                                                                     & 9899                                                                    & 1154901      & 56.67                    & 44.33                    & 20.6                           \\
19          & 1111139                                                                      & 240654                                                                    & 117509              & 111113                                                                    & 12032                                                                   & 1572447      & 49.66                    & 50.34                    & 17.8                           \\
20          & 122252                                                                       & 265325                                                                    & 117612              & 122225                                                                    & 12634                                                                   & 1727634      & 46.63                    & 53.37                    & 17.1                           \\
21          & 1344477                                                                      & 265325                                                                    & 117612              & 134447                                                                    & 13266                                                                   & 1875127      & 44.33                    & 55.67                    & 16.4                           \\
25          & 1968446                                                                      & 322503                                                                    & 109534              & 196844                                                                    & 16125                                                                   & 2613452      & 33.96                    & 66.04                    & 14                             \\
27          & 2381819                                                                      & 355559                                                                    & 99601               & 238181                                                                    & 17777                                                                   & 3092937      & 25.2                     & 74.9                     & 12.9                           \\
30          & 3170200                                                                      & 411602                                                                    & 73882               & 317200                                                                    & 20580                                                                   & 3993404      & 17.97                    & 82.03                    & 11.5                           \\
31          & 3487220                                                                      & 432182                                                                    & 61851               & 378722                                                                    & 21609                                                                   & 4351584      & 14.31                    & 85.69                    & 11                             \\
33          & 4219536                                                                      & 476480                                                                    & 30703               & 421953                                                                    & 23824                                                                   & 5172496      & 4.2                      & 95.8                     & 10.1                           \\
34          & 4641489                                                                      & 500304                                                                    & 11141               & 464148                                                                    & 25015                                                                   & 5642097      & 0.45                     & 99.55                    & 9.7                            \\
35          & 5105637                                                                      & 525319                                                                    & 0                   & 510563                                                                    & \begin{tabular}[c]{@{}l@{}}应有 26265 \\ 实际 14756\\ 短缺 11509\end{tabular} & 6156275      & 0                        & 104.61                   & 9.3                            \\
36          & \begin{tabular}[c]{@{}l@{}}现有 5616200 \\ 使用 5499015\\ 过剩 117185\end{tabular} & \begin{tabular}[c]{@{}l@{}}应用 551584 \\ 使用 540075\\ 失业 11509\end{tabular} & 0                   & \begin{tabular}[c]{@{}l@{}}应有 561620 \\ 实际 540075\\ 短缺 21545\end{tabular} & \begin{tabular}[c]{@{}l@{}}应有 27003\\ 实际 0\\ 短缺 27003\end{tabular}      & 6696350      & 0                        & 109.35                   & 8.7                            \\
            &                                                                              &                                                                           &                     & \multicolumn{2}{l}{\textbf{总短缺 48548}}                                                                                                              &              &                          &                          &                                \\ \bottomrule
\end{tabular}%
}
\end{table}
\end{landscape}

根据格罗斯曼的理论,如果我们考察一下第35年结束时的情形,经济崩溃的含义就变得
非常清晰了。假定追加的不变资本是对剩余价值的第一项扣除,在第36年,
$c=5616200$(与前一年相比增加了10\%)。由于资本家的消费事实上不可能为负,所以
\textbf{设消费为0}。所有在第35年生产的余下的剩余价值(525319-510563=14756)被用作
追加的可变资本的积累,使其从525319升到540075;\textbf{与上一年相比,增长了2.81\%。
由于劳动力增长了5\%,达到551584,现在就有11509个失业的工人。}不变资本的剩余也出
现了。第36年要求的(虽然格罗斯曼没有明确地加以说明,但是根据技术条件可以推断)不
变资本对可变资本的比率是:
\[ 5616200 \div 551584=10.18 \]但是只有540075单位的可变资本可供使用。给定
$10.18$的资本有机构成,能够被使用的不变资本为:
\[ 540075 \times 10.18= 5499015 \]

这产生了117185单位的过剩生产能力。这种情况正好符合《资本论》第三卷中
以“\textbf{人口过剩时的资本过剩}”为标题的那一节中说明的情况。这也意味着,在利润
率尽管在下降、但仍然等于8.7\%时,产生了“资本的过度积累”。

当然,这个数字例子只是\textbf{说明性}的。在一般意义上,\textbf{哪一年发生崩
溃,取决于资本的有机构成、不变和可变资本的增长率,以及剥削率的大小。}举个例子来
说,如果最初的有机构成是8而不是2,积累早在第5年就不可能持续下去了;如果不变资本
的增长率是20\%而不是假定的10\%,崩溃将发生在第8年。一个大于100\%的剥削率将把崩溃
发生的时间推迟到35年之后。可变资本增长率的下降可能会导致不同的结果,这取决于它们
对剩余价值生产的影响。如果由于实际工资的下降,可变资本的增长不那么迅速,资本积累
就会从中受益。然而,如果可变资本增长的减速是因更低的人口增长率引起的,崩溃就会提
前来临(这些结论的代数形式在本章附录中进行讨论)。\textbf{格罗斯曼指出,资本家会
通过削减工资或资本输出对过度积累做出反应。如果这两种增加剩余价值的方法都不足以避
免崩溃的威胁,资本家就会降低积累率;他们会早至第21年时就这样做,因为在这一年他们
的消费开始绝对的下降。随后,在任何情况下,独立于技术进步的劳动替代效应的失业都会
随之发生。}

格罗斯曼预计,资本主义体系\textbf{不会一次性彻底地崩溃,而是会发生一系列一次
比一次更为严重的危机。}只有当起反作用的趋势(以下将要讨论)不再起作用时,“最终
的危机”才会爆发。危机起到一种“\textbf{修复过程}”的作用,恢复了继续积累的前提
条件。\textbf{商业循环的周期性,无论是资产阶级经济学家还是马克思主义经济学家都还
未能加以解释,}它只能“以一种\textbf{纯粹演绎的方式}……并作为已经形成的再生产机
制中基本要素的必然结果”加以解释。从而,它取决于有机构成、剥削率、不变资本和可变
资本增长率的大小。

格罗斯曼没有明确地说明怎样从他的数字例子中推出周期性波动。然而,他的确从一个
非常不同的视角提出了一个危机模型,这种视角包含\textbf{放松积累率和劳动力增长率保
持平衡这一最初的条件}。现在,假定资本家每年增加5\%而不是10\%的不变资本,同时,相
应地减少对可变资本的积累,他们剩余的储蓄被留下来为未来的积累提供资金。这将对劳动
力市场和货币市场产生影响。\textbf{失业将会因积累率的下降而上升。这将迫使实际工资
下降并提高剥削率,从而导致积累率的重新上升。}此外,由于资本家把他们过剩储蓄作为
“\textbf{借贷资本}”借出,贷款的\textbf{利率就会下降},这再一次成为积累的促进因
素。与此同时,利润率继续下降。伴随着资本积累率的升高,资本家的\textbf{净金融资产
的价值会下降。当它降至零的时候,就会发生过度积累的危机。}整体的效应是提高了周期
的幅度,结果产出和失业的波动越来越猛烈。

% Please add the following required packages to your document preamble:
% \usepackage{booktabs}
% \usepackage{graphicx}
\begin{table}[htbp]
\centering
\caption{格罗斯曼的危机模型(简单的)}
\resizebox{\textwidth}{!}{%
\begin{tabular}{@{}lllllllll@{}}
\toprule
年份 & \begin{tabular}[c]{@{}l@{}}不变资本\\ c\end{tabular} & \begin{tabular}[c]{@{}l@{}}可变资本\\ v\end{tabular} & \begin{tabular}[c]{@{}l@{}}剩余价值\\ s\end{tabular} & \begin{tabular}[c]{@{}l@{}}资本家的消费\\ k\end{tabular} & \begin{tabular}[c]{@{}l@{}}增加的\\ 不变资本\\ $a_c$\end{tabular} & \begin{tabular}[c]{@{}l@{}}增加的\\ 可变资本\\ $a_v$\end{tabular} & \begin{tabular}[c]{@{}l@{}}潜在的借贷资本\\ L\end{tabular} & 失业 \\ \midrule
1 & 200000 & 25000 & 25000 & 2500 & 10000 & 56 & 12444 & 0 \\
2 & 210000 & 25056 & 25056 & 2505 & 10500 & 57 & 11994 & 1194 \\
3 & 220000 & 25113 & 25113 & 2511 & 11025 & 61 & 11516 & 2449 \\
4 & 231000 & 25174 & 25174 & 2517 & 11576 & 72 & 11009 & 3766 \\
5 & 243101 & 25246 & 25246 & 2524 & 12515 & 57 & 10510 & 5141 \\
6 & 255256 & 25303 & 25303 & 2530 & 12762 & -461 & 10011 & 16603 \\
7 & 260018 & 24842 & 24842 & 2484 & 13201 & 38 & 9211 & 7974 \\
8 & 281219 & 24880 & 24880 & 2488 & 14060 & -154 & 8386 & 9576 \\ \bottomrule
\multicolumn{9}{l}{* 任一年的失业的增加都等于上一年的v的5\%减去上一年的$a_v$。}  
\end{tabular}%
}
\end{table}
% Please add the following required packages to your document preamble:
% \usepackage{booktabs}
% \usepackage{graphicx}
\begin{table}[htbp]
\centering
\caption{格罗斯曼的危机模型(复杂的)}
\resizebox{\textwidth}{!}{%
\begin{tabular}{@{}llllllllll@{}}
\toprule
年份 & \begin{tabular}[c]{@{}l@{}}不变资本\\ c\end{tabular} & \begin{tabular}[c]{@{}l@{}}可变资本\\ v\end{tabular} & \begin{tabular}[c]{@{}l@{}}剩余价值\\ s\end{tabular} & \begin{tabular}[c]{@{}l@{}}资本家的消费\\ k\end{tabular} & \begin{tabular}[c]{@{}l@{}}增加的\\ 不变资本\\ $a_c$\end{tabular} & \begin{tabular}[c]{@{}l@{}}增加的\\ 可变资本\\ $a_v$\end{tabular} & \begin{tabular}[c]{@{}l@{}}借贷资本\\ L\end{tabular} & \begin{tabular}[c]{@{}l@{}}资本家积\\ 累的财富\\  *\end{tabular} & 失业 \\ \midrule
1 & 200000 & 25000 & 25000 & 2500 & 10000 & 56 & 12444 & 12444 & 0 \\
2 & 210000 & 25056 & 25056 & 2505 & 14700 & 535 & 7316 & 19760 & 1194 \\
3 & 224700 & 25591 & 25591 & 2559 & 20223 & 1056 & 1753 & 21513 & 1971 \\
4 & 244923 & 26647 & 26647 & 2664 & 26941 & 1565 & -4523 & 16990 & 2193 \\
5 & 271864 & 28812 & 28812 & 2881 & 35320 & 2238 & -12167 & 4823 & 2175 \\
6 & 307184 & 30450 & 30450 & 3045 & 44077 & 2477 & -19249 & -14426 & 1456 \\
7 & (253361) & (32927) & (32927) &  &  &  &  &  & 74 \\ \bottomrule
\multicolumn{10}{l}{* 通过加总(正的或负的)当年和以前年份储蓄的借贷资本量计算。}  
\end{tabular}%
}
\end{table}


表16.2和16.3概括了这个过程。在这里,资本有机构成的初始值为8。表16.2忽略了工
资和利率变化对积累率的影响,表16.3考察了这个问题。在表16.2中,不变资本以每年5\%
的稳定比率增加(格罗斯曼计算有点小错误),由第一年的200000上升到第二年的210000;
可变资本增加的更为缓慢,从25000上升到25056。这保证了有机构成处于前面——不变资本每
年增加10\%,可变资本每年增加5\%时——能够达到的水平:
\[ 220000 \div 105000 (\div 8 \div 2) = 210000 \div 25056 = 8.38 \]

随后几年的数据以此类推:在某些情况下,它们意味着负的可变资本的积累。假定资本家每
年消费他们剩余价值的十分之一。由于剥削率稳定在100\%,第一年生产的剩余价值为25000,
其中2500被消费了,10056被用于积累。剩下12444单位的剩余价值被留做\textbf{潜在的借
贷资本(此时它们还没有被借出)}。从表16.2可以看到资本家每年可以\textbf{留出的新的
借贷资本量稳步下降。}(尽管如此,到第8年结束时,资本家已经拥有了高达85081的资金储
备)。由于劳动力继续以每年5\%的比率增长,而可变资本的积累(从而就业增加)变得相当
缓慢,失业持续地增加。第一年的时候为0,第二年就达到了1194,因为在增加的1250名工
人中只有56个找到了工作。(回想格罗斯曼设定劳动力的价值为单位价值,因此每单位可变
资本代表每年只有一名工人就业)。到第8年结束时,24880名工人有工作,9576名工人失业
了,此时的失业率为:

\[ 9576 \div (9576 + 24880) = 27.8\% \]

如果资本家的确借出了他们的过剩储蓄,利率将下降;并且,随着失业的上升,真实工资将
下降。两个因素都将鼓励进行更快的积累。表16.3说明了它们对模型的影响,如果第一年不
变资本以5\%的比率积累,那么在接下来的四年中将分别为6\%、8\%、9\%和9.5\%。可变资本
积累的决定,正如表16.2中表明的那样,由保证有机构成处于假定不变资本始终以10\%,可
变资本始终以5\%的速度增长时的有机构成水平的需要决定。失业水平的计算同以前一样。该
体系将在第6年结束之前耗尽其剩余价值。\textbf{在第4年要求收回贷款的资本家,最终将
  成为净债务人;积累停止; 经济则进入严重的萧条期。}这可以被解释为\textbf{暂时的
  崩溃或新的高涨的前奏} (然而,格罗斯曼并没有具体说明什么样的机制能够导致积累重新
恢复)。

格罗斯曼在结束他对危机理论的讨论时,重新回到了他在开始时讨论的方法论主题。格罗斯
曼的周期模型\textbf{源于生产过程},在这种模型中,劳动力和货币市场发生的事件依据积
累的效果来推断。价格、工资和利率的变化是过度生产的结果而不是它的原因。危机的产生
是因为\textbf{剩余价值生产的不足},这是“\textbf{支配马克思整个概念框架的基本法
  则}”。格罗斯曼指出,这一点不仅被把注意力过多地放在价格波动和个别资本家的计算错
误上的资产阶级经济学家所忽视,也被格罗斯曼认为的马克思主义的继承者所忽视。鲍威尔、
考茨基和希法亭都受到杜冈-巴拉诺夫斯基\textbf{比例失调论}的影响,\textbf{这使得他
  们认为,通过对经济活动实施中央调控,资本主义生产方式自身可以克服危机。}他们在最
好的意义上可以被称为“\textbf{新和谐主义者}”,后来,考茨基则干脆完全抛弃了马克思
主义。作为一名革命者,\textbf{罗莎·卢森堡错误地把流通而不是把生产作为崩溃的根源},
并且把这种困难的根源\textbf{归因于剩余价值的过剩而不是不足}。布哈林在从战争中推出
崩溃趋势之前,只是简单地罗列了一系列所谓的经济矛盾,但它们都是外生的、非经济的因
素。即使是列宁,格罗斯曼相当赞同列宁,对他进行的严厉批评也最少,也没有能够对他自
己资本主义“成熟过度”的重要概念进行解释,因此,列宁的分析同其它人的一样,在资本
过度积累问题上存在缺陷。

\section{崩溃理论的政治学}

在所有这一切中,有很多内容需要进行批判。然而,首先要谈的是格罗斯曼\textbf{对反作
用趋势的讨论和崩溃理论的政治学}。他将反作用趋势解释为\textbf{在周期下降阶段发挥
了增加利润率作用的因素,这些因素通过提高利润率,提供了经济反弹的机会,而不是造成
该体系最终的崩溃。}它们或者通过\textbf{降低不变资本的价值},或者通过\textbf{增加
剩余价值的生产},达到这样的目的,它们既可以内在于、也可以外在于资本主义机制。格
罗斯曼的分析建立在马克思\textbf{《资本论》第三卷}的分析基础之上,但是,格罗斯曼
的分析更为详尽,更加强调“世界资本主义的内部矛盾变得越来越尖锐,崩溃趋势越来越接
近于\textbf{绝对崩溃点}。”

在国内市场起反作用的趋势可以分为三类。\textbf{第一类是不利于增加资本有机构成的力
  量。}这些力量包括生产生产品的\textbf{第Ι部类产业的技术进步},它通过使已有的和
新生产的生产资料的贬值,降低构成不变资本的要素的价值。包括\textbf{交通和通讯技术
  的进步},这削减了不变资本处于流通过程中的时间,从而减少了资本的使用年
限。\textbf{第二类指的是导致剩余价值生产增加的因素。}包括\textbf{第\Rnum{2} 部类生产工资
  品的产业的技术进步},这\textbf{降低了构成可变资本的要素的价值。}包括通过增加劳
动强度和把工资降低到劳动力价值之下的方法降低实际工资。这是格罗斯曼首次提
到\textbf{剥削率的上升};它出现在本书的中间,只占据了一页内容,而且完全没有对格罗
斯曼的任何一个数字例子产生影响。最后,也就是\textbf{第三类,指的是作为总剩余价值
  的一部分的租金和商业利润的下降,这相应地增加产业利润。但是,这种利润的增加部分
  程度上因为对非生产性工人构成的“新中间阶层”进行支持的成本的增加而被抵消。}

格罗斯曼把更多的注意力放在了世界市场上起反作用的趋势及其对帝国主义理论的意义上。
他严厉地批评了以前所有的帝国主义理论,因为他们都\textbf{忽视了资本主义扩张推动了
  剩余价值生产的增加。它与剩余价值实现方面的困难无关,这种困难是深层次功能失调的
  后果而不是它的原因,这种困难是由格罗斯曼分析的有利可图的投资机会的匮乏引起的。}罗
莎·卢森堡是这种错误最著名的受害者,紧随其后的是弗里茨·斯滕伯格(参见以上第十四
章)。鲁道夫·希法亭错把资本主义历史的一个发展阶段——这个阶段银行暂时支配了产业资本,
当作资本主义生产方式自身所具有的一般性历史趋势。格罗斯曼认为,积累越来越倾向于自
我融资,而且“在最终的意义上,更多的是产业控制银行而不是相反”。希法亭的资本输出
理论也是不充分的,因为它无法解释在国际范围利润率差异一直存在的情况下,为什么资本
输出是一个近期才出现的现象。即便是列宁,也\textbf{错把抵消利润率下降的特殊手段
  (即垄断的增长)当成利润率下降的根本原因,并且没有能明确地把过度积累视为帝国主
  义理论的基础。}

格罗斯曼的分析指出了\textbf{可能会提高利润率的进入世界市场的三种不同的方式}。第一
种和\textbf{国际贸易中的不等价交换}有关。格罗斯曼观察到,\textbf{存在国际意义上的
  劳动价值向生产价格的转化}。马克思曾提及这一点,但后来,除了奥托·鲍威尔,所有的
人都忽视了这一点,但鲍威尔并没有把它和积累理论联系起来。格罗斯曼利用一个数字例子
说明了这一点,在他的例子中,欧洲的有机构成和剥削率高于亚洲,利润率低于亚洲:
\begin{gather*}
  \text{亚洲} 16c + 84v + 21s = 121 \\
  \text{欧洲} 84c + 16v + 16s = 116
\end{gather*}

在这里,欧洲的资本有机构成是4,亚洲的是0.25,欧洲的剥削率为100\%,亚洲的为25\%;
利润率分别是16\%和21\%。如果欧洲向亚洲输出资本,利润率就会在全球范围出现均等化,
两组商品都以118.5的价格出售,共同的利润率为18.5\%。欧洲的利润率上升,亚洲的利润
率下降,\textbf{价值从亚洲转移到欧洲。从而“技术上和经济上更加发达的国家以落后国
家为代价占有了超额剩余价值。}

这些利润的获得与亚洲国家\textbf{是否是资本主义国家没有关系}。这在世界市场上产生了
持续的、激烈的竞争,并导致发达资本主义国家之间激烈的对抗,因为\textbf{一个国家的
  所得是另一个国家的所失}:从而,\textbf{经济民族主义成为发达资本主义国家的一个永
  恒特征},而不(像考茨基认为的那样)只是一个过渡阶段。

帝国主义帮助提高利润率的第二种方式,是\textbf{围绕原材料的垄断控制展开的斗争}。
在这种斗争中,\textbf{资本家在新重商主义潮流中,把国家作为他们的代理人。这又是一
场零和游戏,}其中每个国家都企图以其它国家为代价,来削弱自己的崩溃趋势(通过使不
变资本贬值),其它国家被迫向它“纳贡”。因此,试图对商品市场进行\textbf{国际调节
是不现实的。}最后是资本输出问题。如前所述,格罗斯曼对所有先前的马克思主义资本输
出理论提出了异议。先前的著述家的资本输出理论依赖于发达和落后国家之间利润率的差异。
因此,他们既忽视了\textbf{利润率国际均等化}的趋势,也忽视了由于最新技术的使用,
\textbf{殖民地的资本有机构成实际上可能更高}。对格罗斯曼而言,\textbf{过度积累}再
次成为决定性因素:“\textbf{不是国外的利润率更高,而是国内投资机会的匮乏才是资本
输出的根本原因。}

帝国主义不仅因为危机变成全球性的而加剧了危机,而且还会酿成战争:“\textbf{为
投资渠道而战是世界和平的最大威胁}”。因此,格罗斯曼认为,\textbf{国际关系日趋和
谐的观念是荒谬的。单个国家无法克服资本主义国家之间的矛盾。}格罗斯曼坚持认为,鲁
道夫·希法亭提出的“\textbf{总卡特尔}”仅仅是一种幻想,因为希法亭误解了马克思主义
的价值理论。\textbf{消除竞争和商品交换将意味着消灭资本主义自身。}因此,鲁道夫·希
法亭的“总卡特尔”完全不可能成为一个资本家的协会。另一方面,任何真正的资本主义经
济基本上都是充满危机的。从而,\textbf{受管制的资本主义经济,“在理论上是不可能
的”。}

格罗斯曼推断说,由于\textbf{工资水平}承受了越来越大的压力,\textbf{日益严重
的社会紧张是不可避免的。}19世纪后期实际工资能够随着生产率的提高和劳动强度的增加
而上升。这已经变得不再可能了,因为过度积累要求资本家迫使工资下降。\textbf{伯恩施
坦和修正主义者认为,绝对贫困只是历史上某一特定时期的现象,仅限于资本主义工业化的
早期阶段。}在格罗斯曼看来,这与真实情况恰恰相反。他认为,\textbf{贫困化发生在资
本主义工业化的晚期阶段,发生在“过度积累为工会活动设定了客观的限制”时。这将加剧
阶级斗争。}在格罗斯曼看来,\textbf{崩溃理论}和经常归咎于它的宿命论不是一回儿事。
经济崩溃“虽然是一种客观必然,发生的时间也可以准确地推算”,但\textbf{它并不是一
个自发的过程,并不是一个只能被动地等待着它的到来的过程。如果不是对它的抵制,贫困
化将延长资本主义制度的寿命,而工会的抵制将加速它的崩溃。}正如1926年的英国总罢工
表明的那样,工业中的冲突从根本上看是政治性的。罢工和裁员正在为资本主义的生死闹得
不可开交。格罗斯曼的结论呼应了考茨基的爱尔福特纲领和《阶级斗争》中的思想(参见以
上第四章)。格罗斯曼宣称,\textbf{社会主义的最终目标不是一个从外部带给工人阶级的
理想,而是日常阶级斗争的必然结果。}


\section{格罗斯曼的批评者}

格罗斯曼的著作在分析方面存在的缺陷,被有关他的著作的评论无情地揭露出来,在所有
的评论中,只有两篇对格罗斯曼的观点表示完全的赞同(其中一篇是由一位法国社会学家撰
写的)。一个又一个的评论家\textbf{反对奥托·鲍威尔最初的假定条件,}认为这种假定条
件过于严格,以至于事实上\textbf{无法用它来构筑资本主义现实经济的模型。}第一,为
什么不变资本应当以每年10\%的比率,可变资本以每年5\%的比率持续增长,而不考虑这种
类型的增长对整个资本主义体系的生存问题造成的影响?\textbf{资本有机构成不是由技术
给定,而是取决于投资决策的利润率。}在面对格罗斯曼认为资本家们将要面临的继续发展
的后果时,\textbf{他们将会改变自己的行为以避免毁灭,比如降低资本积累的整体速度或
者降低有机构成的增长率。}从而,格罗斯曼对鲍威尔的例子的使用,在海琳·鲍威尔看来,
“只是在玩数字游戏”。

第二个问题与剥削率有关,格罗斯曼认为剥削率将保持不变,而且只有在降低实际工资
的情况下剥削率才会上升(他在危机模型中说明了这一点)。但是,在格罗斯曼的模型中,
正是提高了资本有机构成的技术进步,同时提高了\textbf{生产工资品或者为生产工资品的
产业}提供投入的产业的劳动生产率。如果实际工资保持不变或下降,\textbf{剥削率必然
提高},否则实际工资将以和劳动生产率相同的速度稳步提高。因此,格罗斯曼\textbf{忽
视了马克思主义的相对剩余价值概念},正如一些批评者指出的那样,即使是\textbf{剥削
率能够上升,仍可能会因剩余价值生产增加的太慢而无法避免崩溃;}但是格罗斯曼的分析
中没有对为什么会必然如此进行哪怕是一丁点的证明。

与这些问题相联系的,是方法论上的一个重要批判。格罗斯曼\textbf{把技术变迁降低
不变资本和可变资本构成要素价值的影响看作是次要的因素},它们只对导致经济崩溃的基
本力量产生“\textbf{校正}”的作用。但是,这些影响是资本积累过程中\textbf{固有的
一部分}。它们应当被纳入格罗斯曼的正式模型中,而不是仅仅附在最后面。

有人进一步反对说,格罗斯曼应该运用更具一般性的代数形式,而不是依赖于数字例子提出
他自己的观点;而且格罗斯曼的确使用了的那点儿代数,也并不是特别有帮助。格罗斯曼还
应该解释\textbf{为什么单个资本家在利润率仍然为正时就停止了投资:“从整个资本主义
  的视角看,无知(在积累问题上的)……不是它崩溃的基础”。}此外,格罗斯曼的模型是
一个总体模型,这\textbf{掩盖了不同经济部门之间的差异},在这方面,他的模型不如奥
托·鲍威尔的成熟格罗斯曼假定所有商品都按其劳动价值出售,\textbf{忽视了价值向生产价
  格的转化};然而,这并不能使他的分析不受损害,这是不言自明的。格罗斯曼很快便开始
抨击其他的马克思主义者忽视了价值和价格的差异,而没有认识到这种批评同样适用于他自
己的著作。

格罗斯曼受到的另外的批评指出,他误解了马克思,\textbf{错误地赋予马克思一个崩溃理
  论而不是经常性危机的理论,从而贬低了无产阶级的革命作用}格罗斯曼没有注意到马克思
危机理论中的\textbf{消费不足和比例失调}的线索,错误地理解了马克思的“\textbf{资本
  绝对过剩}”的概念,当\textbf{失业后备军下降到零时,这种过剩就会发生,从而进一步
  的积累无法增加剩余价值的生产。}格罗斯曼的理论是一种\textbf{相对积累过剩}的理论,
在这种理论中,\textbf{剩余价值量可以持续地增加},因此这种理论完全不同于马克思的理
论。

最后,批评转向了格罗斯曼的分析的经验有效性问题。格罗斯曼的理论不能解释19世纪
初发生的危机,那时\textbf{过度积累}还远没有成为一个严重的问题。格罗斯曼的理论也
不能解释拥有\textbf{不同程度过度积累的国家同时波动的现象}。格罗斯曼也没有解释这
样一个事实,即\textbf{危机总是发生在特定部门,而不是发生在作为一个整体的整个经济
中。}


\section{评价}

在所有这些批评中,有很多实质性的内容。格罗斯曼没有提出一个明确的积累理论;也就是
说,在他的著作中,\textbf{完全没有解释为什么资本家会以指定的比率增加不变资本和可
  变资本}(这一批评,公正地说,\textbf{同样适用于马克思和许多后来的马克思主义著述
  家})。格罗斯曼还\textbf{低估了技术变迁对不变资本和可变资本构成要素价值的影响,
  这种影响往往既降低了资本的有机构成又提高了剥削率。从而,利润率可能会增加}:事实
上,如果技术进步的同时,实际工资不变或下降,那么利润率必然会增加。这两个缺陷破坏
了格罗斯曼提出的崩溃理论的整个基础:如果资本家需要比格罗斯曼假设的更少的剩余价值
(相对于他们的资本而言),如果资本家能够生产出比格罗斯曼允许的更多的剩余价值,那
么就\textbf{没有什么明显的理由能够说明为什么资本家不能同时积累和消费。}最后,
表16.2和16.3表示的危机模型包含一个矛盾。\textbf{格罗斯曼假定实现所有生产出来的剩
  余价值不存在困难。但是,那部分留作借贷资本的剩余价值并不能被任何相应的投资所抵
  消。因此,有效需求不足以实现全部剩余价值;超额储蓄使得商品无法售出,并减少了利
  润。}对这个问题的忽视,暴露出格罗斯曼成了\textbf{萨伊定律}的俘虏,同样地格罗斯
曼也成了“古典和谐论”和他自己严厉批评了的新古典经济学家的受害者。

因此,格罗斯曼\textbf{完全未能建立}资本主义崩溃必然性的理论。然而,很难不得出这样
一个结论,即对格罗斯曼的许多敌意大都出于\textbf{政治动机}。作为一个独立马克思主义
者,格罗斯曼无疑处于更加暴露的位置,因为他同样地反对斯大林主义,反对修正主义者的
社会民主主义,反对在其它方面和他有很多共同点的左派共产主义者对议会制的抵制。因此,
他没有什么盟友。正统列宁主义者警告说,他的理论有消极论和宿命论的危险。左派共产主
义者,如安东·潘涅库克和卡尔·柯尔施也以同样的理由攻击他,与此同时,法兰克福研究所
的重要经济学家弗里德里希·波洛克,对格罗斯曼坚持经济学的首要性和无情的积累规律的存
在提出了异议。格罗斯曼的著作出版后,他和法兰克福研究所的关系明显地冷淡了。唯一的
全心全意的支持格罗斯曼的是\textbf{保罗·马蒂克},保罗·马蒂克是一名定居在美国的德国
议会共产党人,同时也是世界产业工人联盟分支中的一员。但在,至少是在20世纪30年代,
保罗·马蒂克在国际马克思主义界几乎没有什么影响力。

因此,格罗斯曼的直接影响是微乎其微的,尽管很快就爆发了资本主义历史上最严重的
危机。格罗斯曼在经济思想史领域虽然仍在继续着他高产的工作,但是他重新修订自己的著
作并完成另外两卷的计划却没有实现,他计划在这两卷中论述简单再生产和马克思的经济学
方法。计划中的著作的英文和法文翻译工作也没有成为现实,而日文版对当时出现的特殊的
宇野学派的马克思主义也没有产生什么影响。在格罗斯曼逝世后,和他的观点类似的观点才
开始广泛传播,首先是在德国,而后——当保罗·马蒂克终于赢得公众时——是美国。今天,那
种认为\textbf{“资本的逻辑”阻止剩余价值生产并以此达到避免自发危机的积累}的观念,
成了马克思主义政治经济学的一个主要分支的基本教义。本书第二卷将对导致这种情况出现
的背景进行描述。

\section{附录:格罗斯曼崩溃模型的代数公式}

\chapter*{结束语}

\phantomsection
\addcontentsline{toc}{part}{结束语}

\textbf{1883年}的马克思主义经济学似乎相对简单一些。其实质内容包含在少量的基本文
献中,事实上,这些文献包括《资本论》第一卷以及作为补充的《共产党宣言》和《反杜林
论》。这些著作,只能由马克思和恩格斯本人来解释,他们对他们的追随者\textbf{在知识
上的支配}是毋庸置疑的。他们集中关注的一些问题虽然很重要,但却\textbf{相当狭隘}。
\textbf{马克思主义政治经济学是有关价值和剥削、资本和剩余价值、积累和危机,以及资
本主义产生和即将来临的对它的超越的理论构成的,所有这些理论都基于对当时英国领先地
位的考察而形成的。}

到1929年,这种在某种程度上可能使人产生误解的\textbf{简单},被\textbf{深刻而且明显
  的复杂性}所取代。\textbf{首先,可以被定义为马克思主义经济学的文献大量增加
  了。}《资本论》第二卷和第三卷,以及随后《剩余价值学说史》的出版,使得马克思
的\textbf{大量的成熟的著作},可供他的追随者们使用,这些著作提供了一些新的洞见和新
的问题。马克思的一些早期著作也可以看到了,这些著作在经济学问题上提供了非常不同的
视角。相关的文献,除了马克思和恩格斯的著作,现在还包括许多其他作者的著作。世纪之
交之前,考茨基、普列汉诺夫已经建立起正统的马克思主义,这种正统——根据个人的观
点——后来被希法亭、卢森堡、托洛茨基、布哈林和列宁的贡献加以丰富或提出挑战。1917年
后,列宁的著作在共产主义者中享有经典的地位,甚至比马克思主义创始人的地位还
高。\textbf{列宁的理论是依据一个落后的资本主义地区的情况形成的——这一事实是很重要
  的。}1917年后,马克思主义越来越成为一种世界性的运动,而不是一种几乎完全集中在欧
洲的运动。同时,\textbf{由于无产阶级革命在外围地区而非发达资本主义国家的中心地带
  取得了成功,在运动内部出现了对马克思主义的“经济主义”解释的质疑。}

这表明了复杂性的第二个维度,作为一种思想体系和一种政治运动,马克思主义成为
\textbf{多中心的}。早在19世纪90年代,正统马克思主义的理论和实践就受到了德国和俄
国修正主义者的抨击,政治经济学问题是他们批判的中心。随后俄国社会民主党分裂,接着
是1905年后德国党的激进派和中间派之间发生的冲突,第一次世界大战期间第二国际的破产,
最后,是由布尔什维克革命引起的决裂。\textbf{经济学越来越多地被作为粗糙的政治武器,
其它派别的马克思主义者而不是资产阶级通常被视为自己的敌人。}经济学争论变得越来越
激烈、\textbf{更具个人化特征和论战性},甚至在争论中(正如1920年代的斯滕伯格和格
罗斯曼一样)主要的参与者既不为修正主义者的也不为革命党的路线进行辩护。

然而,说马克思主义理论在1917年之后“崩溃了”或“消亡了”是不正确的。就政治经
济学而言,马克思主义在1929年依然是充满活力的。这可以从其日趋复杂的第三个原因中推
知:争论的问题的范围扩大了。在马克思逝世后的12年间,《资本论》第二卷和第三卷的出
版,激起了有关价格和利润分析以及危机理论的新争论,就危机理论而言,在大萧条前夕,
资本主义经济崩溃的问题仍悬而未决。在随后的10年间,资本主义的结构变化推动了德国和
俄国马克思主义者对金融资本、垄断、国家支出、军国主义和帝国主义理论的发展,俄国马
克思主义者还就不同类型的前资本主义社会的资本主义工业化问题进行了争论。1917年后,
俄国的理论家不得不正视一个巨大的困难,即把马克思主义经济学加以改造,以使其适应落
后的社会主义国家的实际境况,而曾经是他们的同志的德国和奥地利的理论家们(现在大多
数被视为自己的敌人),则纠缠于西欧和美国的国家管制或“有组织的”资本主义问题。

\textbf{尽管存在着许多重大的缺陷,但到1929年时,不能认为马克思主义经济学正在“退
  化”。相反,它继续进入新的领域、提出新的思想、面对新的对手,遭遇新的问题,它的
  多样性、它的分裂、它的不和谐是它巨大力量的来源。}不再有(也尚未有)单一的认为马
克思主义者都要遵循的正统马克思主义。在西方,自诩为铁板一块的社会主义经济科学已经
被抛弃了。甚至是在苏联,在斯大林已经开始投下其长期的不祥的阴影时,20世纪20年代仍
然存在的知识分子的自由程度令人印象深刻。然而,人类却站在深渊的边缘。接下来的10年
将见证资本主义世界经济的崩溃、苏联新经济秩序的建立,以及希特勒和斯大林控制的广大
地区对批判性思想的镇压。这一重大的世界历史剧,赋予马克思主义经济学提出的根本问题
以新的紧迫性。\textbf{如何解释国家经济权力明显的无法阻挡的增长?它预示着什么?如
  何理解法西斯主义,如何描述苏联生产方式的特征,它对历史唯物主义而言又意味着什么?
  发达资本主义国家的前景是什么,外围地区的社会主义希望又是什么?}

1929年,马克思主义经济学面临的任务是极其艰巨的。在本书第二卷中,我们将看到这些任
务是如何成功地完成的。



%%% Local Variables:
%%% mode: latex
%%% TeX-master: "../main"
%%% End:
