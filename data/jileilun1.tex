
\chapter*{原序}

这本著作是作为国民经济学的通俗入门书而写的。我在许久以前就准备出版本书。但因为党
校的教学工作及革命运动工作的耽搁,未能如期完成。今年一月国会选举之后,我打算再着手
把马克思的经济学通俗化,至少可以将原理那部分完成;但是,当时遭遇着意想不到的困难。
显然,想在具体的事情下,说明资本主义的生产总过程,以及分析它的客观历史限界,我感到没
有什么把握。等到进行了精密的考察之后,我就抱着这样一种见解:即此处不仅存在著说明的
问题,而且还存在著理论上牵涉马克思资本论第二卷的内容,以及有关现今帝国主义政策的实
际和它的经济根源的问题,倘若我能够成功地把到些问题给以科学的正确的处理,那么,这本
著述将不仅具有纯理论上的兴趣,而且在我们对帝国主义进行实际斗争中也将具有一些意义
吧!

1912年1月罗莎·卢森堡

\part{再生产间题}
\chapter{我们研究的目的}

卡尔·马克思伙促使我们注意社会资本的再生产问题,他在这一点上对经济理论作出了一个
不朽的贡献。值得注意的是在经济学说史中,对这个问题作确切闺述的尝试的,我们只见到
两次:一为重农学派之父魁奈的尝试,那是这个问题研究的开始;二为马克选的尝试,那是这
个问题研究的最后阶段。不久,这个问题经尝出现於资产阶级经济学中。但资产阶级经济学
者从未把这个开题和与此相关的和交织起来的次要间题划分开来。而充分领会到纯粹从它
本身看的各个方面他们从未确切地把这个问题表述出来,更不必说解决它了。但这个问题
既然是一个极重要的问题,他们的尝试毕竟能帮助我们对经济科学的发展趋向得到一些了
解。

精确地说,什么是总资本的再生产问题呢?从字面上讲,“再生产”是生产当程的重复和更新。
初看起来,很难察觉再生产的观念在哪些方面与我们所都能懂得的重复观念有所不同一一那
么,为什么要用这个陌生的新术语呢?在我们所要考察的这一种重复里,在生产过程的不断
反复进行中,存在着若干特点:第一再生产的经常性重复是经常性消费的一般先决条件,而经
常性消费是不论在哪一种历史形态下的人类文明的前提。从这样来看,再生产的概念反映
着人类文明历史的一个方面。除非若干先决条件如工具、原料和劳动已经在前一生产周期
中建立起来,生产是不可能反复进行的,也就是说,再生产是不可能发生的。但在人类文明
的原始阶段,在人类控制自然的初期,这种重新从事生产的可能性.或多或少地依靠机会。只
要打猎和捕鱼是社会生存的主要基础,频繁的饥荒阻断了生产的经常性的重复。有些原始人
民很早就认识到为了使再生产成为经常的、反复进行的过程,采取若干措施是必要的;他们
把这些措施结合在宗教性的礼仪中,在这样方式下他们就把这些措施接受下来,当作传统的
社会义务。所以,正如斯宾塞和吉伦的详尽调查告诉我们的,澳洲黑人的图腾崇拜基本上不
是別的,而是社会集团为了获取和保存动、植物食物而采取的一些办法罢了。从远古时期年
复一年地实行这些预防办法,久而久之,它们就僵化为宗数礼仪了。但只有随着耘耕的发明,家
畜的驯养,和以消费为目的的畜牧业的发展,构成再生产要素的消费和生产的循环才有可能。
就再生产应以一定程底的社会对自然的控制为条件,或用经济术语来说,以一定标準的劳动生
产率为条件而论,它的含义不止于简剖的重复。

在另一方面,在一切社会发展的阶段上,生产过程是以两个不同的、而又密切联系的因素——技
术条件和社会条件——的持续为基础,是以人与自然以及人与人的确切关系为基础再生产在同
样程度上依赖於这两项条件。我们刚才看到再生产是如何与人类劳动技术条侍相结合的以及
在什么程度上它是单纯地由于一定的劳动生产率水平的结果;但在每种情形下,当时流行的
社会生产形态所起的决定性作用也并不较小。在原始农业共产主义公社中,再生产以及全部
经济生活是由全体劳动羣众和他们的民主机构来决定的。关于重复使用落动的决定——梦动的
组织——作为劳动的主要前提的原料、工具和人力的供应——再生产的安排,和再生产规模的确
定,都是计划合作的结果,在这计划合作中,本公难范围内的每一个人都参加的。在以奴隶劳
动或徭役为基础的经济制度下,再生产是通过人对人的统治关系来实行和周密地加以调节的。
这里,再生产的规模是决定于占统洽地位的上层分子所保有的对多寡不等的其他阶展的劳动
的处置权力。在以


