\chapter{瓦尔拉斯与一般均衡理论}

\section{瓦尔拉斯的一般均衡体系}

瓦尔拉斯对边际分析的运用仅仅是他对现代经济学做出的部分贡献。与杰文斯和门格尔相比,
他关于边际主义的作品在很多方面更加成熟。瓦尔拉斯的一般均衡理论研究对经济学专业具
有重大影响,它使瓦尔拉斯与马歇尔一起成为新古典经济学两个分支之一的创始人代表。

\subsection{什么是一般均衡理论}

\textbf{一般均衡理论}是对经济体的一种分析,它\textbf{同时考虑所有部门}。因此,我
们要考虑任何一种突变对系统的直接和间接影响,并且考虑在直接影响下的交叉市场的影响。
拥有经济体部门之间这种相互关系的概念,相对来说比较简单,但是,将其正规地写下来却
是一种相当复杂的思想。瓦尔拉斯的贡献是以一种正规的方式使一般均衡系统模型化。

\subsection{一般均衡理论的早期先驱者}

因为拥有一般均衡的概念相对来说比较容易,所以,1874年当瓦尔拉斯出版《纯粹政治经济
学纲要》时,一般均衡已经不是一个新的观点就不足为奇了。早期经济学家对于经济体由很
多相互关联的部分所组成,已经有了清楚的认识。例如,魁奈在其经济表中就呈现了这种情
形,描绘了年产品在经济体不同部门之间的流动。类似地,亚当·斯密在对市场过程的生动
描述中,显示了对经济体不同组成部分之间关系的深刻洞察。但是,尽管这些经济学家解释
了不同部门的相互关联性,却没有正规地将其模型化。

1838年安东尼,奥古斯丁·古诺在将经济体相关性正规化方面取得了重大进展,同时也分析
了某些微观经济问题。他能够以数学形式来表达一些关于厂商的理论问题,并利用微积分来
证明当边际成本等于边际收益时利润最大化。因此,他为厂商理论做了像杰文斯与门格尔为
选择理论所做的事情:用边际术语来阐述这一理论。此外,古诺抽象的数学倾向,使他能够
在相当大的程度上理解经济体中的各种关系,并且有助于他领先于瓦尔拉斯。古诺正确地断
定“要获得完整而严格的方案来解决经济系统一些组成部分的问题,考虑整个系统是绝对有
必要的”。

然而古诺认为,数学分析还没有充分发展到能够阐明一般均衡模型的程度。冯·杜能也运用
微积分来解决经济理论中的问题,并且像古诺一样,这种数学代向使他看到了用联立方程体
系来呈现一般均衡模型的可能性。也许因为与瓦尔拉斯相比,他们是更优秀的数学家(瓦尔
拉斯因为未能通过人学考试中的数学部分而没有进入颇负盛名的法国综合理工学校,而古诺
则被认为是一位富有才华的数学家),因而,当面对为了使问题易于处理而必须面对的很多
假设,以及无法度量相关概念的问题时,古诺与杜能并没有试图投身于对一般均衡理论复杂
关系的研究。

无论出于什么理由,在其他人惧怕前行的地方,瓦尔拉斯则向前迈进;因此,\textbf{恰恰
是瓦尔拉斯首先运用数学符号,通过阐明关于经济体的一个模型,使人们对一般均衡的认识
清晰而精确。}凭借这一成就,他理应被誉为现代经济理论的一位重要先驱者,尤其应突出
强调他对抽象模型的构建和对数学的应用。

\begin{mybox}{瓦尔拉斯、马歇尔以及复杂性}
  \begin{multicols}{2}局部均衡分析经常被看做是对瓦尔拉斯一般均衡方法进行补充的一
种方法;它们只不过是从相反的目标开始。马歇尔最初着眼于小问题,而瓦尔拉斯最初着眼
于那种大问题,但是,\textbf{两者最终将结合在一起}。

    复杂系统分析的现代研究提出,上述观点\textbf{可能是错误的}。根据这一最新研究,
\textbf{一般均衡所要求的信息处理,远远超过人类大脑的计算能力。}如果确实如此,这
两种方法就是不兼容的,因为一旦系统实现了超越个体决策能力的均衡,一个人就不可能从
均衡状态下逐步变弱衰退。如果是那样的话,系统就获得了它自己的生命,而不与个人的决
策直接相关。

    为了实现对总体经济的分析,人们必须通过局部均衡来靠近它,然后修正那种局部均衡,
使之“较少局部”和“更少局部”。人们最终能够通过扩展马歇尔的分析来考察总体经济。
但是,通过一般均衡分析将不能做到这一点。

    Robert Clower与Axel Leijion--hufvud 对凯恩斯宏观经济学的阐释,遵循了这种推论
方针,他们指出,凯恩斯经济学是马歇尔总体经济分析方法的开始。
  \end{multicols} % \tcblower % This is the lower part.
\end{mybox}

我们将用文字来描述瓦尔拉斯模型,并论述模型提出的一些理论问题。然而,在着手进行论
述之前,考察一般均衡模型与局部均衡模型之间的区别将是有益的。


\subsection{局部均衡分析与一般均衡分析}

就模型与理论的真正精髓而言,它们都假定某些因素保持不变,以使这些因素不影响模型中
变量的行为。在自然科学中,实验方法被证明是富有成效的,研究者可以重复进行实验,除
了两个变量之外的其它所有变量都保持不变。一个变量——例如水的温度——被允许变动,以此
来观察对另一变量的影响。如果在多次实验中观察到水在华氏212度沸腾,那么,我们就能
断定,当某些因素保持不变时——在这个例子中压力保持不变是关键——水将在这一温度下沸腾。

经济学家根据模型的抽象程度来区分局部均衡模型与一般均衡模型。与一般均衡分析相比,
\textbf{在局部均衡分析中更多的因素被假定为保持不变。}局部均衡分析只允许少量的变
量变动;所有其他变量都假定不变。\textbf{一般均衡分析允许更多的变量变动。然而,一
般均衡分析并不允许所有的变量都变动从而影响模型,只有那些被看做是经济学范围内的变
量才被允许变动。}例如,一般均衡模型将个人的品位与偏好、生产产品的可利用技术,以
及经济与社会的制度结构都假设为既定的。因为作为一门社会科学,经济学的范围在历史上
就被正统理论限制在可以计量的变量上,所以,一个数学上的一般均衡模型看来是可行的。

大多数局部均衡模型遵循着阿尔弗雷德·马歇尔的传统,将其自身限制在对某一特定家庭、
厂商或者行业的分析上。假定我们希望分析牛肉行业成本降低对牛肉价格的影响。运用局部
均衡方法,我们将从处于假定均衡下的行业开始,通过使成本降低来破坏均衡,然后推论出
新的均衡位置。在这一分析中,经济体的所有其他变量都假定不变,从而对牛肉行业没有影
响。牛肉行业成本降低将导致牛肉供给增加和牛肉价格下降,直至一个新的均衡水平。

现在假定我们减少了对模型的限制,将猪肉行业与牛肉行业都包含在分析中。和牛肉行业较
低成本的直接影响是,因为牛肉供给增加,牛肉价格降低。然而,牛肉价格的下降也会影响
对猪肉的需求。如果牛肉价格相对于猪肉价格来说下降了,对猪肉的需求将会减少,因为牛
肉的需求增加了:消费者将用牛肉替代猪肉。猪肉需求减少会导致猪肉价格下降,这将导致
对牛肉需求减少,以及牛肉价格的进一步下降。牛肉价格的这种下降,又将进一步减少对猪
肉的需求,并再次降低猪肉的价格。两种产品价格与需求之间的相互作用,将会一直持续,
所引起的价格与数量变动将变得越来越小,直到两个行业中\textbf{新的均衡条件确立}为
止。

\begin{figure}[ht] \centering
  \begin{tikzpicture}[thick] \draw (0,0) node[below left] {$O$}; \draw[very
thick] (0,5) node[above] {牛肉价格} -- (0,0) -- (5,0) node[right] {牛肉数量};

    \coordinate (d) at (0.8,4.7); \draw (d) -- +(-40:5cm) node [below right]
{$D$};

    \coordinate (d1) at (0.8,4.1); \coordinate (d2) at (0.8,3.3); \coordinate
(d3) at (0.8,2.5); \draw (d1) -- +(-40:4.5cm) node [below right] {$D_1$}; \draw
(d2) -- +(-40:4cm) node [below right] {$D_2$}; \draw (d3) -- +(-40:3cm) node
[below right] {$D_3$};

    \coordinate (s) at (0.6,0.6); \draw (s) -- +(40:5cm) node [above right]
{$S$};

    \coordinate (s1) at (2.2,0.6); \draw (s1) -- +(40:5cm) node [above right]
{$S_1$};

    \begin{scope}[xshift=8cm] \draw (0,0) node[below left] {$O$}; \draw[very
thick] (0,5) node[above] {猪肉价格} -- (0,0) -- (5,0) node[right] {猪肉数量};

      \coordinate (d) at (0.8,4.7); \draw (d) -- +(-40:5cm) node [below right]
{$d$};

      \coordinate (d1) at (0.8,4.1); \coordinate (d2) at (0.8,3.3); \coordinate
(d3) at (0.8,2.5); \draw (d1) -- +(-40:4.5cm) node [below right] {$d_1$}; \draw
(d2) -- +(-40:4cm) node [below right] {$d_2$}; \draw (d3) -- +(-40:3cm) node
[below right] {$d_3$};

      \coordinate (s) at (0.6,0.6); \draw (s) -- +(40:5cm) node [above right]
{$s$};
    \end{scope}
  \end{tikzpicture}%
  \caption{\label{fig:pigcow}行业的相互依赖}
\end{figure}

\cref{fig:pigcow}中,作为牛肉行业成本降低的结果,牛肉的供给曲线从 $S$移动到
$S_1$,。牛肉价格的下降引起对猪肉的需求由 $d$直接减少到 $d_1$;,这又降低了猪肉
的价格。猪肉价格的下降导致对牛肉的需求由 $D$减少到 $D_1$。这两种产品价格与需求之
间连绪隐相互作用被表示为需求曲线向下移动,直至实现最终均衡。

\textbf{局部均衡分析}是一种尝试,它通过孤立经济体的一个部门,例如一个行业,并且
忽视这个部门与经济体其余部分的相互作用,将复杂的问题还原为易于处理的形式。对于有
前后关联的论点来说这是有益的。然而,\textbf{清晰程度的增强与分析上的整齐,是在损
害理论上的严密和完整的情况下实现的。}

如果通过向模型中添加第三个和第四个行业,来向更加一般的均衡模型靠近,那么,分析将
会变得如此复杂,以至于图形表达将会引来更多的困惑而不是清晰。瓦尔拉斯的伟大贡献在
于,他认识到从数学上才能最好地了解和表达行业间复杂的相互依赖。对于非关联的论点来
说,他的一般均衡分析是有益的。

\subsection{瓦尔拉斯的主张}

假设我们对牛肉行业的价格与产量感兴趣。牛肉的需求与供给能够表示为将价格与供给量和
需求量相联系的方程。尽管模型中有三个变量——价格、供给量、需求量——但在均衡下只有两
个未知数,因为均衡下需求量等于供给量。于是可以得出,牛肉行业均衡价格的问题就由一
个供给方程、一个需求方程以及两个未知数组成。

现在我们从这个局部均衡模型变动到更加复杂的一般均衡模型。即使在一个一般均衡模型中,
也有必要忽视复杂经济体的某些方面,所以,我们\textbf{假定}经济体只由\textbf{厂商
与家庭}两个部门组成,忽视\textbf{政府部门与国外部门}。此外,还假定厂商不相互购买
中间产品,\textbf{家庭的偏好不变,技术水平不变,存在充分就业,以及所有的行业都是
完全竞争的。}下图呈现了这样一个经济体的图示。
\begin{figure}[ht] \centering
  \begin{tikzpicture}[thick] \node (ss1) at (0,0) {家庭} ; \node (aa1) at
(7cm,0) {厂商};

    \path[->] (ss1) edge [bend left=60] node [below] {对最终产品的需求} (aa1)
([xshift=2mm]ss1.north west) edge [<-,bend left=70] node [above] {最终产品的供给}
([xshift=-2mm]aa1.north east) (aa1) edge [bend left=60] node [above] {对要素的需
求} (ss1) ([xshift=-2mm]aa1.south east) edge [<-,bend left=70] node [below] {要
素的供给} ([xshift=2mm]ss1.south west);
  \end{tikzpicture}
  \caption{\label{fig:interdepend}经济部门的相互依赖}
\end{figure}

家庭以其既定的偏好与有限的收入,为了获得最终产品而进入市场,并且因这些产品表现出
一种对美元的需求。……正是在图11.2上半部分表示的这些市场中,最终产品的价格、供给
量以及需求量得以确定。这些市场如果处于均衡状态,每种特定商品的供给量与需求量必定
相等。

要素市场由图11.2下半部分表示。在这些市场中,厂商向家庭索要土地、劳动以及资本,存
在一种由厂商到家庭的美元收入流。当家庭在这些市场上供给生产要素时,要素价格得以确
定。这里的均衡要求所有的市场都出清,以使每种要素的供给量等于需求量。家庭从要素市
场获得其收入,并在市场上将收入花掉得到最终产品。因为家庭在有限收入既定的前提下使
消费最终产品获得的满足最大化,所以,家庭这样分配其支出,即花在任何一种特定产品上
的最后1美元,与花在任何其他产品上的最后1美元相比,产生相等的边际效用(戈森第二定
律 \cref{eq:second})。厂商与家庭之间的收入流代表了经济体的国民收入;要使其处于均
衡状态,家庭必须花掉他们获得的全部收入。收入的分配是在要素市场上确定的,它取决于
各种要素的价格以及每个家庭所出售的要素数量。

当市场经济中的厂商朝一个方向看时,它们面对着最终产品的价格;当它们朝另一个方向看
时,又面对着各种生产要素的价格。考虑到这些价格以及可利用的技术,厂商将投入进行组
合,以一种使其利润最大化的方式生产。这就要求它们的投入组合方式,能以最可能低的成
本生产既定产量,并且要求它们在利润最大化的产量水平上生产。竞争力量将导致一种长期
均衡状态,在这种状态下,最终产品的价格正好等于其平均生产成本。要使国民收入水平处
于均衡,厂商必须在要素市场上花掉得自于最终市场的收入。

从经济体的这一稍微抽象的例子中,首先能够得到的也是最明显的经验是,经济体的不同部
分是相互关联的。把系统中的一个变量看成决定另一个变量,会令人误解。如果存在均衡,
那么,所有的变量同时被决定。假定通过变动一种最终产品的价格使均衡受到破坏,这将对
整个系统产生影响,因为消费者将改变其支出模式,厂商将改变其产量。这些变化将在要素
市场上被感知,因为厂商会改变它们对投入的需求,从而形成一组新的投入价格,以及一种
不同的收入分配。

斯密、魁奈还有其他人,已经认识到市场经济不同部分的相互依赖。但是,要想非简单地表
述一切事情都取决于其它所有事情,就有必要详细说明不同部门之间的关系。瓦尔拉斯的天
赋使他能够通过运用数学,为更加准确的陈述奠定基础。当运用数学符号在明确的瓦尔拉斯
数学模型中考察经济体时,用语言分析其模型时不明显的问题就凸显出来了。

家庭对最终产品的需求,能被表示为将价格与每个家庭的需求量相联系的方程。既定的某种
最终产品的市场需求,也能被表示为一个方程,该方程通过加总家庭需求方程而获得。用类
似的方式,即加总将价格与供给量相联系的厂商供给方程,能够得到最终产品的市场供给。
最终产品的市场均衡要求每种最终产品的供给量等于需求量。能够类似地得到要素市场的市
场需求方程与供给方程,均衡条件是所有的市场都出清。对于家庭来说,能够得到一个方程,
方程一边表示家庭的收入(对所有要素来说,所出售的每种要素的价格乘以出售的数量,再
加总),另一边表示家庭的支出(对所有购买的产品来说,所购买的每种最终产品的价格乘以
购买的数量,再加总)。处于均衡状态下的家庭,收入必定等于支出,并且支出的方式必定
能使效用最大化。厂商利润最大化的均衡条件,以及通过竞争的力量使厂商的平均成本等于
其价格的均衡条件,都同样能用方程来表示。

因此,我们得到了一个表明经济体各部门相互依赖的联立方程系统。瓦尔拉斯对市场经济运
行的阐述,引起了一些新问题。例如,一般均衡方案可行吗?由经济体不同部门中的市场所
产生的必要的均衡条件,与整个经济体的一般均衡一致吗?生产是如何适合这一模型的?由
市场确定并由一般均衡方案给出的未知数有:(1)最终产品的价格,(2)要素的价格,
(3)最终产品的供给量与需求量,(4)要素的供给量与需求量。只有一套导致整个经济体
均衡的价格与数量的组合,还是有很多可能的均衡?如果某个问题确实存在一种解决方案,
那么,这种解决方案有经济上的意义吗?或者说它将产生负的价格与数量吗?这种解决方案
是一种稳定均衡还是一种不稳定均衡?系统是确定的吗?实际情况是,存在多种可能性。市
场运行的实际过程可能导致数学函数的改变,从而不能引起最后的均衡。另一种可能性是,
能够获得最后的均衡,但是,其位置取决于系统中变量所遵循的路径。这就暗示了其有可能
产生不同的最终均衡值。最后,均衡是如何实现的?谁制定了价格?如果存在非均衡交易,
将会发生什么情况?瓦尔拉斯知道这些问题中的一些,而另一些在1874年之后将近六十年的
时间中,都没有得到确定或解决。

瓦尔拉斯没有令人满意地回答这些问题中的任何一个。因此历史上的看法想必是,如果他是
现代新古典经济学之父,那么,他并没有走完通向乐土的路程。另一种说法是他许诺了很多,
却只交付了一个包含很多漏洞的抽象框架、尽管存在这样的负面看法,但即使是最苛刻的批
评,也必定赞同他的确提出了一种模型,该模型就市场运行提供了重大见解,并且能够作为
更多理论发展的基础。当人们考察瓦尔拉斯去世近一百年来经济学的发展时,能够肯定地说
他对经济学具有巨大的影响。

\begin{mybox}{一般均衡、复杂性以及人类大脑的极限}
  \begin{multicols}{2}在为大学本科生讲授经济学时,经济学教授通常使用非常适合几何
表达的两种产品例子,例如,通过无差异曲线分析个人选择。在这样的例子中,突出的理性
假设在直觉上是有道理的。然后,我们这些经济学教师就挥挥手,将分析\textbf{扩展到n
种产品},却没有指出每多增加一种产品,对于决策者来说,做出这一改变在计算上所必需
的要求也呈\textbf{指数级倍数}增加。在某些方面,它相当于说明了一个人是\textbf{怎
样会跳}的,然后就假定这个人\textbf{会飞}。

    事实是,为了获得大量商品条件下的一般均衡,人们将需要比目前拥有更多计算能力的
大脑,即使那样,为了保持理性,他们将花掉所有的时间来处理信息。要点是,当存在一种
思考成本时,过多“理性”就没有意义了。所以,\textbf{当人们失去理性时,他们倒有可
能是理性的。}

  复杂系统分析方面的新近研究表明,当存在这种决策复杂性时,总系统的性质会发生变化,
为了了解复杂系统,人们必须以一种根本上不同的方式来处理问题。如果这是正确的,在将
来,经济分析的瓦尔拉斯一般均衡理论基础,有可能被经济思想的一些其它基础彻底取代。
  \end{multicols} % \tcblower % This is the lower part.
\end{mybox}

\subsection{回顾瓦尔拉斯}

瓦尔拉斯在经济理论史上的较高地位,部分地依赖于他独立发现边际效用理论,更多地则依
赖于他使市场经济各部门的相互依赖概念化。尽管在他之前的一些人已经觉察到家庭、厂商、
最终产品价格、生产要素价烙、所有最终产品和中间产品的供给量和需求量之间的相互依赖,
但是,没人能够像瓦尔拉斯那样,通过将其描述为一个联立方程系统来准确地表达这种感知。
现在,人们就有可能看出家庭的均衡和最终产品市场的均衡,与厂商的均衡和要素市场的均
衡是一致的。杰文斯与门格尔试图在边际效用、最终产品价格以及生产要素价格之间发现一
种简单的因果关系,与瓦尔拉斯的一般均衡模型相比,他们的尝试似乎是天真的。瓦尔拉斯
清楚地表明数学作为一种经济分析工具的力量,尽管在完全进入20世纪之前,他的这一预言
没有得到充分的认同,并且今天一些人还就数学的适当运用进行着争论。

与杰文斯或者门格尔的分析相比,瓦尔拉斯的边际分析更加成熟。他并没有看到从主观效用
到价值的一种简单的因果关系方向;取而代之的是,他看到了一个复杂的相互关联的系统。
因为瓦尔拉斯关注于各部门的相互依赖,在某种意义上,只能向后去研究需求,所以,他没
有像杰文斯和门格尔那样掉进一些陷阱中。当杰文斯和门格尔满足于在效用、最终产品价格
以及生产要素价格之间寻找一种单方向的因果关系时,瓦尔拉斯的一般均衡模型却表明,它
们完全是相互关联的。在瓦尔拉斯的系统中,所有的价格相互决定,不可能在任何一个方向
上确定价值的因果关系。最终产品的价格影响着生产要素的价格,并且受到后者的影响。在
一般均衡模型中,一切事情都取决于其它所有事情。目前还不能完全确定这一成熟的说明是
因理解产生的,还是因瓦尔拉斯专注于一般均衡胜于专注效用的副产品。对瓦尔拉斯来说,
效用只不过是为了得出他所希望的需求曲线而需要假定的某种东西。因此,瓦尔拉斯没有为
需求分析提供充分的效用基础,只是暗示了这一基础。

\subsection{瓦尔拉斯、边际生产力以及经济体的相互依赖}

\textbf{瓦尔拉斯的一般均衡理论不仅依赖于需求从而依赖于效用,而且依赖于供给从而依
赖于边际生产力递减。}瓦尔拉斯的阐述中也有很多含糊之处。\textbf{在前三版的第20讲
中,其模型使用了不变的生产系数,这就是说,因为一种要素不能独立于另一种要素而变动,
所以不存在边际产品。}因此,他对一般均衡理论的早期说明,并不具备完整的一般均衡模
型的第二个基础。尽管如此,他还是陈述说能将分析扩展到包括可变的生产系数。对此,读
者只能单凭信仰来认同这种可能性。

瓦尔拉斯认识到这个问题,并在19世纪后期向一位同事咨询他怎样才能扩展其分析,以包括
可变的生产要素。因此,在1900年《纯粹政治经济学纲要》\textbf{第四版中,他将可变生
产要素(从而将供给的边际生产力基础)融合进来。}然而,瓦尔拉斯是在菲利普·维克斯蒂
德正式形成边际生产力概念,并宣扬其重要性之后六年才将边际生产力融合进来。正因为如
此,\textbf{瓦尔拉斯对供给方面边际分析的贡献受到质疑。}正如在边际效用中的情形那
样,他的兴趣在于其一般均衡理论所需要的供给函数,而不在于为供给函数做铺垫的生产函
数。

瓦尔拉斯了解其模型的一些不足。在1874年之后的\textbf{将近六十年中,其他问题并没有
得到确定或解决,一些问题至今仍未解决。}为了对这样的一些问题有所了解,我们来考察
下面的问题。
\begin{enumerate}
\item \textit{一般均衡方案可行吗?}

  一些人认为通过简单地计算方程和未知数,能够推论出存在一般均衡。亚伯拉罕·韦尔多
(Abraham Wald,1902--1950)在1933年表示情况并非如此,对一般均衡的推论相当复杂。
只有在\textbf{1954年},杰勒德·德布鲁(Gerard Debreu,1921--)和肯尼斯·阿罗(Keyneth
Arow,1921--)才能够证明一般均衡方案的存在。

\item \textit{如果的确存在一种解决方案,那么,这种解决方案有经济上的意义吗?或者
说它将产生负的价格与数量吗?}

仅仅能够从数学上证明一般均衡的存在,并不意味着它与现实世界有任何关系。因为一般均
衡与现实世界的联系相当间接,所以\textbf{还不能完全确定数学与现实世界是相关的。它
被称为不存在的世界的天体力学。}

\item \textit{生产是如何适合瓦尔拉斯系统的?}

  尽管瓦尔拉斯系统看上去包括生产,但是仔细考察会发现它\textbf{主要是一个交换模型,
生产与系统之间具有不适宜的联系。只要存在规模收益不变,这一点就是没有问题的;但是,
如果存在规模收益递增,模型就有着严重的问题。}

\item \textit{由经济体不同部门中的市场所产生的均衡条件,与整个经济体的一般均衡一
致吗?}

瓦尔拉斯认为,他已经回答了这个复杂问题,但是实际上他没有。存在严格的条件,在那些
条件下才能实现这种一致性。

\item \textit{由市场确定并由一般均衡方案给出的未知数有:(1)最终产品的价格,(2)
要素的价格,(3)最终产品的供给量与需求量,(4)要素的供给量与需求量。只有一套导
致整个经济体均衡的价格与数量的组合,还是有很多可能的均衡?}

瓦尔拉斯意识到多重一般均衡的可能性,一般均衡分析还必须应对这种可能性。一般均衡理
论家能表明存在\textbf{唯一均衡的条件},但是,他们不能表明那些条件就是我们在经济
体中能预期到的条件。当人们试图将预期包括进模型中,就像人们在所谓的太阳黑子模型
(sunspot models)中所做的那样时,事情将变得更加复杂。这些模型充满了多重均衡。
\textbf{多重均衡(multiple equilibria)的可能性,是将一般均衡模型应用于现实世界
的最大局限之一。}多重均衡是怎样产生影响的?在多重均衡下,\textbf{即使市场解决方
案可能是均衡的,它也不一定是最佳的方案};可能存在一种更好的方案。此外,如果存在
一种更好的均衡,那么,相对于那种更好均衡的非均衡,实际上可能优越于市场所实现的均
衡。

\item \textit{均衡是稳定的还是不稳定的?}

  均衡不一定都是稳定的;\textbf{如果模型失去了均衡,它会恢复均衡吗?}这个问题相
对迅速地得到了回答,并表明了稳定所必需的条件。没有表明的是那些条件是否适合现实。
一些事件实际上可能会破坏稳定。市场运行的实际过程可能导致数学函数的改变,这将不会
引起最终均衡。另一种设想是可以实现最终均衡,但是,其位置可能取决于系统中的变量所
遵循的路径。因此,有可能产生不同的最终均衡值。

\item \textit{均衡是怎样实现的?谁制定了价格?如果存在非均衡交易,将会发生什么情
况?}

瓦尔拉斯致力于解决这个问题,该问题在现代宏观经济学争论中扮演着重要的角色。他提出
了众多的计划,包括书面的和口头的保证以及一个\textbf{卖者喊价过程}(tatonnement
process),在这一过程中,一位拍卖者(因此得名瓦尔拉斯拍卖者Walrasian Auctioneer)处
理所有的出价,确定在哪些价格上将使全部市场出清,并且只有在那时才允许交易。

\textbf{唐纳德·沃克}(Donald Walker)深度考察了这些计划,并断定模型有致命的缺陷,
原因是\textbf{瓦尔拉斯没有赋予模型充分可行的特征。}沃克的推论严重损害了宏观经济
学的新古典分支,该分支将其分析建立在假想的拍卖者的合理性上。

\end{enumerate}

这些问题都是实质性的,但是,它们并没有损害到瓦尔拉斯的成就。现代经济学中很多最有
心智的人都是在他所设计的框架内引出问题。一是到20世纪50年代,经济学家们都在专心于
有关一般均衡存在与稳定的问题。今天,他们还在从事着其他问题的研究。尽管瓦尔拉斯的
阐述在数学上并不是很完美,然而,自20世纪50年代以来,它一直是高级研究的框架。

瓦尔拉斯取得成功的根源是运用了数学,这也是一般均衡理论存在一些失误的原因。高度抽
象的模型提供了经济体相互依赖的见解,然而,瓦尔拉斯并没有试图从经验上度量其模型中
的概念。\textbf{它们未被设计成可度量的;它是没有实证应用的理论。}概念度量的困难,
在整个现代一直是对一般均衡理论的一种批评。因此,尽管一般均衡理论表明了经济体内部
均衡下存在的关系,但是,它没有解释当瓦尔拉斯认为不变的因素实际上发生变化时经济体
中发生了什么。

大多数学者的结论是,尽管一般均衡模型用来回答有关可供选择的经济政策后果问题时,具
有极大的潜力,然而,这一潜力仍然有待实现。弗兰克·哈恩(Frank Hahn)是一位一般均衡
理论家,他写道:

\begin{quotation}正是亚当·斯密首先认识到,需要解释为什么这种社会安排没有引起混乱。
成千上万贪婪而自私的个体,追求他们自己的目标,其间基本上不受国家的控制,从“常
识”来看,这些个体似乎是无政府状态的一剂可靠处方。斯密不仅引出了一个显然重要的问
题,而且让我们在回答这个问题的道路上开始出发。阿罗与德布鲁做了经典陈述的一般均衡
理论(1954年与1959年),接近道路的终点。既然我们已经到达那里,我们发现它并不如我
们期待的那样仿人深受启迪。
\end{quotation}

\subsection{瓦尔拉斯方法与马歇尔方法}

简要比较一下瓦尔拉斯的方法与马歇尔的方法是有益的。瓦尔拉斯对方法与形式感兴趣。他
在寻找有关经济体模型最一般的数学说明。马歇尔把经济理论看做是分析的发动机,它必须
涉及现实世界,否则,就应当被忘掉,或者只是隐隐约约地留在人们心中,相关时才被用于
分析。

不可能存在两种不同的方法。正如我们将在关于现代微观经济学的章节中看到的那样,马歇
尔的经济学支配着很多大学本科课程,但是,瓦尔拉斯的经济学已经成为主流的研究生微观
经济学。尽管瓦尔拉斯方法获得了胜利,然而,它的问题也是突出的,并使现代微观经济学
容易受到更多的批评。

\subsection{瓦尔拉斯论政策}

瓦尔拉斯将其纯粹经济学看做是用来阐述经济政策的工具。\textbf{他将自己看做是一个社
会主义者,却激烈地反对诸如圣西门一类的乌托邦社会主义者的观点。他认为,经济理论未
能严格地证明最佳的资源配置发生在完全竞争的条件下。}在其《纯粹政治经济学纲要》第8、
第22、第26以及第27讲中,他考察了这些问题并断定“自由竞争所支配的市场中的生产……
将引起需要得到最大可能的满足”,以及“在某些限度内,自由获得了最大的效用”。

因此他提倡,\textbf{国家应当通过立法致力于形成完全竞争市场的系统}。同时,瓦尔拉
斯\textbf{并不是一位十足的自由放任支持者:他发现很多适宜于政府干预的领域。}将他
描述为一位\textbf{市场社会主义倡导者}是有道理的。他提出\textbf{地租代表着不劳而
获的收入},因此应当属于政府所增加的收入,在这一点上他遵循了穆勒的观点。瓦尔拉斯
推论,在完全竞争市场中,随着地租作为私人收入的一种来源被废除,收入分配将不再包含
大的不公平。一般而言,他试图采取介于左翼社会主义者与强硬的自由放任支持者之间的一
条政策路线。他试图证明竞争市场中的一般均衡导致了\textbf{社会效用最大化},他的这
一尝试在很大程度上已经被经济学家们所忽视或者忘记。\textbf{纳特·威克塞尔稍后证明
只有当所有的个人都拥有相同的效用函数和相等的收入时,瓦尔拉斯的结论才能成立。}

瓦尔拉斯模型含义中的社会主义观点,在20世纪30年代被理论家们加以扩展,并且超越了为
人所知的社会主义--资本主义争论,我们将在第13章中予以评论。

\section{维尔弗雷多·帕累托}

维尔弗雷多·帕累托是瓦尔拉斯的弟子,一般均衡理论的早期支持者。他贯彻了瓦尔拉斯在
一般均衡理论中所使用的推理,并且将分析加以扩展,以考察不同政策的福利含义。帕累托
试图将瓦尔拉斯经济学延伸到政策中。帕累托是现代福利经济学的创始人之一;另一位是亚
瑟·C·庇古,他扩展了马歇尔经济学的福利含义。

帕累托致力于研究如何评价经济体或者一个经济体内部\textbf{特定市场结构的资源配置效
率问题}。亚当·斯密断定,完全竞争市场导致合意的结果,尤其是经济体较高的长期增长率。
19世纪70年代开始的对微观经济学日益浓厚的兴趣,引出了有关资源配置效率的间题,并且
导致了影响经济体的不同经济政策价值评价标准的发展。

亚当·斯密对自由放任的提倡,并不是基于一个理论上严密的模型。它更多地聚焦于市场的
宏观经济后果,外加最少的政府干预。19世纪90年代,帕累托开始运用新的边际工具来评价
微观经济行为,并且成为主要在一般均衡框架中起作用的福利经济学分支的创始人。帕累托
也代表了一种\textbf{欧洲大陆(尤其是法国与意大利)方法},与基于阿尔弗雷德·马歇尔构
造的局部均衡结构的\textbf{英国框架}相对立。福利经济学的英国路线开始于享利·西奇威
克(Henry Sidgwick,1838一1900),他是一位对经济学做出贡献的政治哲学家。西奇威克于
1883年出版了他的《政治经济学原理》(Principles of Political Economy)。正是马歇尔
在剑桥的继承者\textbf{庇古},通过扩展和改进西奇威克和马歇尔关于市场失灵与外部性
的见解,才成为\textbf{福利经济学局部均衡分支的创始人}。

帕累托对评价资源配置效率问题的回答是简单的,如果资源配置的某种改变,使一个人的状
况变好了,而其他人的状况没有变坏,那么,这种改变将会增进福利。\textbf{一种理想的
或者最优的稀缺资源分配即帕累托最优(Paretoopti-mum),它被界定为不可能使某个人的状
况变好而其他人的状况不变坏的分配。帕累托认识到,他关于最优的概念,对于现实世界的
问题来说,并不是特别适当,在他的著作《心灵与社会》中,他解释了在现实世界的福利分
析中,进行人与人之间比较的上必要性。}然而,他将他的帕累托最优标准看做是对瓦尔拉
斯一般均衡理论的一种有益的分析性扩展。

竞争性市场将导致一种帕累托最优状态——离开这种状态,没有人的状况变好的同时,而不使
其他人的状况变坏,当这种观点得到确定时,帕累托最优政策就获得了一种特殊意义。这是
从一般均衡分析中产生的重要结论之一,并且它加深了我们对于市场的理解。这一判断构成
了对市场的理论支持,市场被用于社会主义--资本主义争论的形式方面,我们将在第13章中
予以考察。但是,这一判断遗漏了关于市场用途的更广泛争论的其他重要方面,它使市场过
程看上去很\textbf{机械}。因此,\textbf{它引导着福利经济学远离现实世界中的问题,
不是使经济学向着马歇尔希望加以利用的那样作为分析的发动机方向发展,而是向着与现实
几乎没有直接关系的一套形式主义的演绎证明发展。现实情况是,任何政策都帮助了一些人
而损害了为一些人;因此,对于符合帕累托最优标准的政策,如果经济学家只是给出政策是
有利的判断,那么,他们的分析就必定与现实世界相分离了。}

\section{总结}

瓦尔拉斯的一般均衡分析使人印象深刻,但是,在瓦尔拉斯的效述中也存在很多问题,直至
今天,也只有部分问题已经得到解决。同样的话也能用来概括它的竞争者,即马歇尔的局部
均衡分析。尽管它们都存在缺陷,但是,瓦尔拉斯与马歇尔两人的成就是相当大的。他们提
供了将边际主义者在供给方面的研究与在需求方面的研究相结合的手段,作为这样的开端者,
他们理应被称作新古典经济学之父。

%%% Local Variables:
%%% mode: latex
%%% TeX-master: "../../main"
%%% End:
