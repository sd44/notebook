\chapter{约翰·斯图亚特·穆勒与古典经济学的衰落}
\label{cha:mill}


穆勒是一位天才的与众不同的思想家,对经济学、政治学和哲学都作出了重大贡献。他巨大
的思想力,是通过他父亲詹姆斯·穆勒的高广度和强度的天才精英教育而实现的……其心理
代价是穆勒二十岁时表现出来的精神抑郁,所幸的是,经过一段时间的消沉之后,穆勒重新
恢复了健康,并成为他所处时代以及所有时代中最主要的知识分子之一。

穆勒实质上是一位社会哲学家,他下决心要提高个人在社会中的作用。与他父亲和李嘉图的
悲观主义不同,他提出了保守的乐观主义,预期了一个良好社会的发展。他经济思想的主要
来源有:早年在斯密、李嘉图、他的父亲及边沁古典经济学方面所受的影响;傅立叶和圣西
门的社会主义著作;孔德的著作(他使穆勒将经济学仅仅视为人类社会行为的一个方面);
最后是穆勒的朋友哈里特·泰勒,她后来成了他的妻子,人文社会主义。穆勒既是一位古典
自由主义者,也是一位社会改革家。他认为他的主要任务是写一本对李嘉图的学术思想进行
清晰说明的书,并将19世纪第二个二十五年期间出现的新观点融合进去。然而,穆勒是一位
对国际贸易理论以及对“供给--需求”分析作出了重要贡献的原创思想家。

穆勒的《政治经济学原理》代表了对古典经济理论的最大修正,也代表了古典经济学理论的
顶点。当时对李嘉图有很多批评,穆勒对这些批评作出了回应。这些批评有三个主要来源。
第一,越来越多的证据显示,李嘉图学说与从英国经济体运转中获得的经验结果不一致。马
尔萨斯人口理论是李嘉图体系的一个实质性前提,与之相反,不断增多的证据表明,随着人
口增加,实际人均收入提高了,而不是减少了;并且,农业中技术的进步,使农业正经历着
收益递增而不是收益递减。第二,学者们开始彻底研究李嘉图的理论结构,尤其是他的劳动
价值论,并且发现李嘉图对需求的处理,以及对利润在价格决定中作用的处理是欠缺的。第
三,许多人道主义者和经济学家,忽视经济思想的技术内容,猛烈抨击李嘉图理论结构所代
表的新兴资本主义经济体的基础。

经济思想后来的很多发展都出自于对萨伊、李嘉图等长期供需均衡的批评。此外,由法国、瑞士、
德国以及英国经济学家创造的越来越多的社会主义文献集,对古典观点——依靠资本主义经济
体不受阻碍地运转,能完美地实现经济和谐——提出了质疑。

\textbf{批评中更加技术性的内容,}多是那些把经济学作为主业而不是作为副业来研究的人
提出的。他们的主要目标是\textbf{部分地否定马尔萨斯的人口学说、历史上的农业收益递
  减以及工资基金学说,并且用他们的理论来代替劳动价值理论。}在他们的价值理论
中,\textbf{利润是决定价格的因素,需求与效用在相对价格决定中的作用被扩大了。}这种
分析形势最终接触了边际效用学派和阿尔弗雷德·马歇尔经济学的果实,前者始于19世纪70年
代。

\section{李嘉图之后经济学的发展}

\subsection{对古典经济学的早期批判}

很多早期古典经济学的批评家将19世纪西欧资本主义的运转视为不和谐。大部分批评者都提
倡用非暴力手段消除社会冲突,尽管所开出的治疗药房随着经济学家的不同而不同。早期社
会主义者间接影响了正统理论的发展,却直接影响了穆勒,尤其是对英国立法和工人运动的
形成产生了重要作用。“实际上,19世纪30年代的大多数理论发展,尤其是涉及把利润的性
质当做一种收入来源的理论,都或多或少是有意识地反对传播社会主义意识形态的结果。”

\subsection{经济学的范围与方法}

李嘉图从斯密理论与历史描述的松散结合转向抽象演绎的理论模型,很少直接论述方法问题。
但是,他后来的追随者就经济学的适当方法达成了几乎完全一致的看法。他们的\textbf{新
  李嘉图方法将经济学看做是一门基于某些简单假设的学科。因此,经济学家的任务是纠正
  体系的逻辑,以确保结论是根据既定假设得出来的。}在后李嘉图时期,当经济理论与可利
用的经验数据之间出现冲突时,这样一种方法观点及其有助于经济理论的发展,其原因在于,
他使经济学家可以忽视数据。

在此期间,涉及经济学范围和方法的最佳并且是最清楚的两个论述,是由拿骚·西尼尔和穆勒
作出的。在《政治经济学大纲》中,西尼尔将政治经济学界定为处理“财富的性质、生产以
及分配”。西尼尔认为,经济学家已经浪费了时间去试图收集更多的经验信息,应当确立经
济学家努力的方向,即提高经济理论的逻辑一致性。

西尼尔指出,作为一门科学,经济学的基础依赖于四个基本主张:(1)\textbf{理性原理,
  即人们是理性的和精于计算的,并且希望以最小的牺牲获得财富;}(2)马尔萨斯人口学
说;(3)农业收益递减原理;(4)历史上的产业收益递增原理。将经济学视为纯粹演绎性
学科,对经济理论的发展具有重要影响;但在考察这些后果之前,我们先着眼于细腻而方法
论观点中另一个引人注意的方面。

\textbf{西尼尔}是明确主张经济学应当是实证科学的最早的经济学家之一。他认为作为一个
科学家,\textbf{经济学家应当注意区分规范判断与实证经济分析}。例如他在以下两者之间
所做的区别:(1)支配财富性质与生产的一般法则;(2)支配收入分配的原理,它反映了
经济体的特定习俗和制度结构。\textbf{穆勒后来就生产法则(是像引力定律一样的自然法
  则)和作为其体系基础的分配法则(不固定,来自于特定的社会与制度安排)作了区分。}西
尼尔主张,作为一个科学家,经济学家能够指出各种经济行为的后果,或者达到既定目标的
可能途径,但是,他或她不应当离开实证科学分析的领域,不应对既定活动形式上的要求作
出价值判断简单地说,经济学家应当是他或她关心“是什么”“而不是应当是什么”。经济
学家的“结论,无论其一般性和真相是什么,都不允许他增添一个字符的建议。”

认同李嘉图所实践、细腻而所阐述的方法,对李嘉图之后的经济学产生了不幸的影响。19世
纪30到40年代变得明显的理论与现实之间的冲突,在很大程度上被忽视了;尽管经验证据与
李嘉图理论体系的几个基本前提相矛盾,然而,经济学家还是顽强地坚持着李嘉图的模型。

\subsection{马尔萨斯的人口理论}

李嘉图的《原理》出版后,深切关注人口问题的经济学家为避免马尔萨斯人口论中过量人口
严重后果问题,主张家庭使用一些避孕措施。

西尼尔认为,历史证明表明食物供给人口比人口增加得更快。

\subsection{工资基金学说}

作为一种\textbf{短期工资理论},工资基金学说简单地主张,\textbf{工资率取决于劳动的
  供需}。他们实际上\textbf{不是}现代经济学中使用的供需。\textbf{工资基金的规模是
  积累的资本中支付给劳动的部分,它使劳动需求固定。}给定工资基金的规模,短期工资率
就通过用劳动市场上的人数除工资基金而得到。于是在短期中,工资基金在量上是固定的,
劳动数量是固定的,工资率被唯一地确定下来。

随着马尔萨斯人口理论的远去,工资基金学说不得不承担起作为短期和\textbf{长期工资理
  论}的重担。这是它\textbf{不可能做到}的,原因在于,工资基金学说中没有一处说到有
关劳动的长期供给的内容。在这一时期的经济学家的著作中,似乎对工资基金学说的看法与
对劳动工会的态度之间没有什么联系:主张工资基金学说的很多经济学家其实明确赞成组成
劳动工会。不过,在通俗文献中,工资基金学说是以反对通过工会的经济主张而知名的。穆
勒于1869年对工资基金学说作出了著名的否定,并且后来的经济学家对穆勒的这一否定赋予
了重要性。

\subsection{历史上的收益递减}

对英国的经济来说,所有可以利用的数据都表明,李嘉图基于历史上的收益递减的预测是错
误的。马克·布劳格被一些人认为是对这一时期进行研究的最精明的现代学者。他曾说到“理
论与现实之间的分离,可能从来没有比在李嘉图经济学全盛时期更加彻底。”……这种分离
根植于李嘉图的方法论中,为李嘉图所实践、西尼尔所揭示的方法论,专门强调从一组既定
的假设中进行推理的演绎过程;因此,它使李嘉图主义者忽视他们的模型与事实之间的矛盾,
而使自己忙于完善其理论结构的优雅型。大多数非正统经济思想的一个共同点是:他们都认
为,正统经济理论精确地显示出李嘉图经济学的缺陷——正统模型与事实之间的冲突以及完善
其理论结构的演绎过程和内在一致性所受到的困扰。

\subsection{未采取的道路:查尔斯·巴比奇与收益递增}

巴比奇是以机械计算器的发明者而知名,但他也是一位敏锐的经济观察家,他写了一本比李
嘉图著作要好得多的书,描述大量生产的性质与含义。内森·罗森伯格指出,巴比奇捕捉到了
通过重复劳动和大量生产能够实现的成本节省,并且看到收益递增将成为行业的驱动力,巴
比奇能被视为现代产业经济学复杂方法之父。

\subsection{下降的利润率}

按照李嘉图的观点,农产品的成本增加时,边际土地上的利润随着边际内土地地租的提高而
下降,这一趋势将会持续下去,直到利润率接近为零,并且因收入由资本家向地主的再分配
而产生静止状态。但是,这一断言的正确性也只能通过经验证据而不是理论得到确定。度量
不同时期经济体利润率变化的统计问题相当有难度,19世纪也没有这种度量所要求的统计工
具。当然,一些经济学家也质疑今天是否拥有这些统计工具。尽管缺乏对历史上农业收益递
减以及对利润率下降和静止状态最终到来的经验证明,但是李嘉图主义者——尤其是穆勒——仍
然坚持这些预测。

\subsection{利润(利息)理论}

李嘉图断定,利润率的变化对于解释不同时期相对价格的变化所起的作用并不重要。他认为,
尽管相对价格在理论上取决于劳动成本和资本成本,\textbf{资本成本为利润,然而利润在
  实际中所起的作用如此无关紧要,以至于能将它们忽略掉。可见,李嘉图的价值理论实际
  上就是生产成本理论,只有劳动一种成本。}

李嘉图式社会主义者利用李嘉图的劳动价值理论来表明劳动承受着剥削……\textbf{恰恰是
  为了纠正李嘉图价值理论的逻辑缺陷,并支持盛行的意识形态反对李嘉图式社会主义者的
  进攻,使得经济学家们将他们的注意力转向了利润。}

在后李嘉图阶段的早期,对利润理论和价值理论最重要的贡献是由\textbf{拿骚·西尼尔}作
出的,他最先试图发展一种\textbf{利息节欲理论}(abstinence theory of interest)。
西尼尔在他的价值理论中,比李嘉图更强调需求方面的效用;在论及供给方面时,他又强调
以无效用作为实际生产成本。利用古典经济学的基本心理假设,他主张人们都是理性和精于
计算的。如果我们要生产资本产品,一些人就一定要放弃消费,除非因这一痛苦而给资本家
以报酬,否则他们不会放弃消费。因为资本与劳动两者对最终产品的生产都有必要,所以他
们的价格必定足以支付这两种实际生产成本。因此,西尼尔发展了一种\textbf{生产成本价
  值理论},其中\textbf{工资是对劳动的返还,利润是对资本供给者的返还}。

古典经济学没有对利润与利息进行区别。西尼尔致力于发展一种利息理论,该理论成为19世
纪末期发展起来的庞巴维克理论的前身。西尼尔遵循着古典惯例,他所进行的讨论只涉及供
给方面。他仅考察了储蓄供给曲线的决定力量,而利息理论也应当解释对投资的需求。作为
一种反对社会主义者的主张,西尼尔的利息结余理论存在几个缺点。他主张储蓄的供给曲线
是完全有弹性的,并且储蓄所招致的痛苦成本或者说无效用,对于富人和穷人来说是相同的。
因为他专门\textbf{将利息看成是对痛苦成本或者说是对放弃消费的无效用的支付},所以通
过遗产或赠送而获得的资本的利息收入,就没有牺牲社会公正或经济公正。因此最后,
就\textbf{利息的社会公正}而言,西尼尔的利息理论所提出的问题,可能比它回答的问题还
要多。

\section{穆勒思想的背景}

\subsection{穆勒的经济学方法}

他把经济学当做使用演绎方法的假设科学。经济学家作出某些假设,然后从这些假设中演绎
出结论。因为实验方法对经济学家来说不可用,所以他们必须依靠演绎技术,而不能运用自
然科学中非常富有成效的归纳技术。然而,穆勒小心地指出,经济学家\textbf{从其演绎模
  型中得出的结论,应当通过与事实的比较予以经验。}按照穆勒的观点,应用演绎模型预测
的结果与历史事实之间缺乏一致性,这显示出重要的却未被注意到的“干扰原因”。这些原
因可能会导致新的更多的假设,这些假设再通过演绎推理产生新的结论,或者它们可能是经
济学家未考虑到的非经济因素的结果。

经济学家假定了一个\textbf{抽象的经济人},完全受着拥有财富欲望的激励。然而,穆勒认
识到,尽管这种抽象产生了一些有用的结论,\textbf{但它最终还要与更加复杂的人类社会
  与人类活动模型相结合。}

西尼尔区分了实证经济学与规范经济学,目的是将规范性的判断从经济研究中消除掉,穆勒
吸取了这一划分,目的却是将社会哲学问题并入到李嘉图模型中。

在《自传》中,穆勒论述了其生产法则与分配法则概念的来源,他的主要灵感源自于印证圣
西门主义者的社会主义著作,穆勒证实哈里特·泰勒是使它确信区分生产法则与分配法则之重
要性的人。李嘉图理论对静止状态的预测,尤其是其中有关工资将停留在维持生活的最低水
平上的预测,遭到穆勒更加乐观态度的反对,穆勒深信随着时间的变化,社会将以一
种\textbf{贤明而人性}的方式来行动,所以将会产生\textbf{更加平等而公正的收入分配}。
穆勒赞成\textbf{高遗产税},但反对累进税,原因在于他担心累进税的阻碍效应。他也提倡
形成生产者合作,并且认为当工人不仅获得工资,而且从这些合作中获得利润和利息时,他
们将会有更大的动力来提高生产力。此外,他认为农业收益递减可以通过增加对人的教导以
及晚婚和节育来降低人口率。

现代正统经济理论揭示了生产法则与收入的功能性分配之间的密切关系。最终产品与服务市
场零售价格的决定力量,与各种生产要素价格的决定力量密切相连。投入与产出之间的实物
关系被经济学家称作生产函数,它决定了不同生产要素的边际实物生产力,并且一种生产要
素的市场价格部分地由这一生产力来决定。然而,\textbf{关于个人收入分配的决定力量,
  现代正统理论只有很少的阐述。个人收入分配取决于一组更加宽泛的非经济变量,如一个
  社会的法律、习俗以及制度安排,因此按照正统经济学的观点,个人收入分配在经济学学
  科之外。}此外,正统理论家不愿意考察与个人收入分配相关的问题,原因在于它涉及\textbf{规范
性问题和价值判断}。如果将穆勒的理论转换成现代经济理论术语,那么穆勒将会主张在不
同要素的边际生产力与个人收入分配之间,只存在一种松散的联系。社会不能修正生产函数,
但是社会确实有能力实现一种与其自身价值判断相一致的个人收入分配。

\subsection{穆勒的折衷主义}

穆勒的巨大实力是他的折衷主义,这也是两位最重要的后穆勒主义英国经济学家马歇尔与凯
恩斯的实力。

\subsection{杰里米·边沁的影响}

穆勒及其同时代的人致力于将理论与政策相结合,对他们影响最大的是英国人\textbf{杰里
  米·边沁}的著作。1780年,边沁第一部重要著作出版之后,他便成为了一群改革者的智力
领袖,这群人以哲学激进分子或者说功利主义者而著称。马尔萨斯之前,边沁就主
张\textbf{节欲};边沁主义者后来曾提倡一连串改革,包括\textbf{全民投票(含女性)、
  监狱改革、言论和出版自由、文职以及工会合法化等}。边沁从一个简单前提开始,
即\textbf{人们受到两种强烈欲望的激励:获得愉快和避免痛苦。}如果社会能够度量愉快和
痛苦,那么就能创造出导致多数人的最大幸福的法律。按照边沁的观点,度量愉快与痛苦的
最佳办法要借助于\textbf{货币指标}。因此,边沁及其追随者希望设计出导致多数人的最大
利益的法律,从而使社会改革成为一门精密科学。

然而,穆勒也部分地否定被他父亲所接受的边沁主义的某些方面。在他生命的剩余时间里,
在避开边沁理论结构某些方面的同时,他也继续分享边沁对社会改革的关注。边沁主义体系
中的两个部分格外扰乱他。首先是其中的教条主义,尤其明显的是,哲学激进主义者坚决主张快乐主义
的愉快——痛苦计算,能够被用来分析所有的人类行为。受到孔德和其他人的影响,穆勒不能
接受这样一种狭隘的观点,这一观点似乎忽视了将人类和其他动物区分开的很多因素。第二
个干扰是,在某些方面,激进主义者并不足够激进。

\subsection{自由放任、干预抑或社会主义}

这种分类很难适用于穆勒。也许,表现像穆勒这样敏感而复杂的思想家特色的最好方式是认
为在公共政策方面,他代表了古典自由主义与社会主义之间的一个中点。他的社会主义不是
马克思主义的,并且穆勒显然与马克思有很少的联系。然而,他的确区分了\textbf{革命社
  会主义者(左翼革命的)与哲学社会主义者(右翼进化的)},他自己的观点似乎更倾向于
后者。

穆勒在他的短文《论自由,1959)中,试图表述政府与公民之间的适当关系。一些强烈的古典
自由主义成分包含在内:当政府反对个人意愿时,唯一正当的权力行使是“阻止伤害其他人。
他自身的利益,无论物质上的还是精神上的,都不是充足的理由。”然而,《原理》中他又
有些异常。在对实际社会行为的讨论中,穆勒又被迫放弃了这种强烈的自由主义观点,他在
一般规律的例外上发现了异常。在《原理》中,他有一处作了有力的自由主义陈
述,“\textbf{总之,自由放任应当是通用的惯例:除非为一些巨大利益所要求,否则对它
  的每种违背都是某种罪恶。}”另一处他又逐渐从严格的自由放任观点中退出,声称“政府
的\textbf{唯一目的}是保护个人和财产,这是\textbf{不能接受}的。政府的目标就像社会
联盟的目标一样广泛。\textbf{它们由所有的有益行为以及所有对罪恶的免疫力组成},这些
都是政府的存在所能直接或间接给予的。”换句话说,穆勒承认政府干预的缺失不一定必然
导致最大的自由,因为对自由来说,存在很多其他限制,这些限制只有立法或者政府才能消
除。

穆勒对地主的谴责是尖锐的,他的政策建议是从地主那里拿走地租和土地价值的额外增加额。
穆勒并不强调劳动者与社会其他成员,尤其是资本家之间的阶级冲突;然而,其全部社会哲
学和他所提倡的主要项目,例如\textbf{全民教育、借助遗产税的收入再分配、工会的组成、
  工作日的缩短以及对人口增长率的限制},都暗示着除了那些与土地所有权相联系的冲突与
不和谐之外,体系中还存在着其它的冲突与不和谐。

穆勒对私人财产的论述,反映出他将古典自由主义与社会改革相混合。财产权不是绝对
的,\textbf{当社会断定财产权与公共利益发生冲突时,它能够废除或者改变这些权利。}

就像他否定社会主义者所认为的私人财产是社会罪恶的主要原因一样,穆勒也不能接受社会
主义者所认为的竞争是导致社会困难的一个原因。“对竞争的每一种限制都是一种罪恶,对
竞争的每一种扩充,即使暂时伤害性地影响到一些劳动者阶层,但最终也总是有益的。”穆
勒对竞争的这些赞同观点,与他对工会的支持及其通过使用垄断力量改善劳动者状况的其他
尝试之间的矛盾,给他带来而一些难题。\textbf{工会}“远不是对自由劳动市场的一种妨碍,
反而是\textbf{自由市场的一种必要手段};是使劳动的销售者能够在竞争体制下,正当地招
股自身利益的一种不可缺少的办法”。

\subsection{一种不同的静止状态}

穆勒的折衷主义和他带给经济学的人道主义,没有一处比在对经济体长期趋势的论述中得到
更好的反映了。尽管经验证据相反,穆勒仍然坚持基本的李嘉图模型,该模型预测了利润率
下降与静止状态。但是,穆勒的静止状态不是李嘉图所预想的那种消沉。与到目前为止几乎
所有的正统经济学家相反,穆勒不能确信经济处于成长中的国家,例如他所处时代的英国是
否是一个理想的居住地。穆勒发现,一个繁荣的成长的经济体存在很多应受责难的方面,例
如“踩踏、挤压、推搡以及践踏”。在著名的有关静止状态一章中,穆勒以批评的眼光来看
待他自己的社会,并勾画出他对未来的希望。穆勒对\textbf{美好社会}所设定的标准
是\textbf{个体幸福、福利的提高}。他明确地表示,这些东西\textbf{不一定}用物质产品
来度量。产量增加和人口增加本身\textbf{不一定}是有益的,\textbf{也不能}自动地有益
于社会。按照穆勒的观点,静止状态可能是非常渴望的一种社会形态,随着经济活动的步伐
放缓,更多的注意力将被集中在个体上以及他的非经济和经济福利上。“只有在世界上落后
的国家,增加人口还是它们一项主要目标;在那些最发达的国家,经济上需要的是一种更好
的分配。……在没有一个人贫穷,没有一个人希望更加富有的同时,也没有任何理由担心,
其他人向前的努力会把自己向后推。”

\subsection{穆勒的社会哲学}

当他转向政策问题时……穆勒意识到市场力量的运行不一定导致一种和谐的经济与社会秩序,
他尤其认识到地主与社会之间的冲突,以及现有秩序在收入分配中的不公平。

穆勒关注生活质量,他发现唯物质的以增长为导向的经济体中的很多东西,使人们从自我实
现和提高转向低级追求。他接受李嘉图关于经济体的长期趋势导致一种静止状态的分析,但
是由于他乐观的人道主义,他预见到了一个新的更好的社会,这个社会不再以严格的物质追
求为方向,也不是李嘉图所说的黑暗世界。

\section{穆勒的经济学}

\subsection{理论的作用}

由于受到正统与非正统思想家著作的影响,穆勒总是批评地对待技术性的经济理论。理查德·琼
斯《略论财富的分配》1831年版,一般性地批评了古典观点,并特别批评了李嘉图的地租理
论,因为他们的分析忽视了\textbf{经济体的历史与制度环境}。\textbf{琼斯被称作历史学
  派的先驱者},他质疑李嘉图的分析在所有时间和所有地点的应用,并提倡用一种更加经验
性的方法来解释制度结构的变化。穆勒的《原理》第2篇第4章“关于竞争与习俗”部分,含
蓄地认可了琼斯的这一批评,这表明穆勒认识到,抽象的经济理论必须通过历史上盛行的制
度的了解来加以调和。因此,穆勒主张\textbf{竞争与习俗两种力量支配着收入分配}。“在
所有场合,他们(英国经济学家)都倾向于表达自己的思想,好像他们认为竞争实际上其作
用一样,而不管竞争起作用的趋势显示出什么。”

穆勒采取了一种相对的历史观点,指出市场经济中竞争发生作用是一种有较少经验的历史现
象,如果我们回过头看就会发现,习俗传统在解决有关收入分配的经济问题上起着主要作
用。……他认识到李嘉图体系假定经济体中存在一组参与者即生意人,他们被获取利润这一
强烈欲望激励着,正是通过他们的活动,资源得到配置,市场达到了均衡。然而,也有不存
在这一类参与者的经济体,即使是市场经济,如果其中“缺乏有魄力的竞争者,那些拥有资
本的人也宁愿把它留在原地,以一种更宁静的方式通过资本获得较少的利润”。此处,以及
在他书中的其他地方,穆勒一直在考虑一个问题,即应当给与抽象理论多大的重要性,又应
当给予制度--历史的材料多大的重要性。这一问题被不同的非正统经济学家一次又一次地
提出来,今天仍然有待于我们去解决。

一些社会力量,如习俗修改其至否定了基于竞争性过程的预测,面对这样的社会力量,经济
学家为什么还要不断地使用竞争性模型?“如果我们认为只有通过竞争原理,政治经济学才
能具备科学的特性,那么这就可以得到部分的理解。”但只有当我们接受对科学的下面这种界
定时,这一奇怪的结论才有意义:经济理论或模型要想是科学的,就要能够得出严格的且准
确的结论。换句话说,科学要求必须作出准确的预测,并且,预测发生的概率应当等于1。这
一观点将当时流行的对科学的看法由自然科学延伸至经济学。然而今天,我们能够将预期发
生概率小于1的研究领域当做科学研究领域来接受。因而,现代物理学承认,阻止试验在完全
一致的条件下重复进行的随机现象是可能发生的。在关于竞争与经济科学的陈述中,穆勒似
乎接受了科学的狭义概念。然而,在他的大部分著作中,他更倾向于今天对科学的看法。


穆勒对理论作用的看法——因为在实践中,在某一既定社会背景下,\textbf{诸如习俗一类的
  其他因素可能会更改理论预测},所以,不能不加批评地接受理论上的结果一一使他与李嘉
图有所区别,而更接近于斯密的观点。在对亚当斯密的考察中,我们发现,斯密的经济政策
主张,并不是应用于机械社会中的抽象理论工具,而是一种前后关联的分析,它反映出斯密
对纯理论主张如何在某一既定社会背景下起作用这一问题的看法。

穆勒通过与共产主义做比较,来考察资本主义与私人财产的优点,我们能从中感受到他的折衷
主义,这种折衷主义也体现了斯密式的前后关联的分析。穆勒认为他应当选择纯理论上的共
产主义,用来与现有的货本主义做对比,但是,他立刻又声称这并不是一个合适的选择依据。
因为拿现有的资本主义(并且事实上,是经过社会改革的资本主义)与共产主义做比较,正如
很可能要显示的那样,打破了有利于私人资本主义制度的平衡。

斯密与穆勒的前后关联的分析,基本上根植于他们宽泛的经济学方法——把经济活动仅仅看
成是所有活动的一个组成部分。这一点与李嘉图以及追随李嘉图路线的大批主流经济学家比
较狭罕的注意力尖锐对立。
\clearpage
\subsection{价值理论}

穆勒提出的价值理论或者相对价格理论,是对李嘉图劳动价值理论的根本否定,尽管穆勒强
调其理论的特征不是对李嘉图教条的背离,而是过去理论的延续。他提出了一种\textbf{生
  产成本价值理论},其中,\textbf{货币成本基本上代表了实际成本或者是劳动与节制的无
  效用。}在这点上,穆勒与西尼尔两人有可以比较的价值理论。穆勒认为,价值理论的目的
是解释相对价格,绝对价值是以价值的不变度量为基础的,他放弃了李嘉图对绝对价值的探
索。在对地租的论述中,他认识到土地的机会成本并不总是为零,在土地具有可替代用途的
情形中,地租就是社会生产成本。尽管穆勒没有按照马歇尔的方式对短期与长期作出区别,
但是,他看上去的确对这一区别有含糊的认识,并且将他的主要任务看成是解释相对价格在
长期中是如何决定的。尽管他没有清楚地阐明供给与需求,但是,他的价值理论清楚地反映
出如下认识,即\textbf{需求的数量与供给的数量是价格的函数}。由于这一原因,我们可以
用熟知的马歇尔形式来呈现穆勒的长期价格理论,这样做对两个人都是公平的。

对一件产品来说,要想拥有交换价值或者价格,就必须是有用的和难以获得的;但
是,\textbf{仅在不寻常的情况下使用价值才决定交换价值或者价格。}

\begin{figure}[ht]
  \centering
  \subcaptionbox{供给完全无弹性\label{fig:mulea}}{%
  \begin{tikzpicture}[scale=0.6]
    \draw[very thick] (0,5) -- node[left, text width =1em] {价格} ++ (0,-5) -- node[below] {数量} ++ (5,0);
    \draw (2.6,4.6) node[above] {供给} -- (2.6,0);
    \draw (0.7,4) -- +(-40:5) node[below right] {需求};
  \end{tikzpicture}%
  }\hfill
  \subcaptionbox{供给完全弹性\label{fig:muleb}}{%
  \begin{tikzpicture}[scale=0.6]
    \draw[very thick] (0,5) -- node[left, text width =1em] {价格} ++ (0,-5) -- node[below] {数量} ++ (5,0);
    \draw (0,2.3)  -- (4.5,2.3) node[right] {供给};
    \draw (0.5,4) -- +(-47:4.5) node[below right] {需求};
  \end{tikzpicture}%
  }\hfill
  \subcaptionbox{边际成本递增\label{fig:mulec}}{%
  \begin{tikzpicture}[scale=0.6]
    \draw[very thick] (0,5) -- node[left, text width =1em] {价格} ++ (0,-5) -- node[below] {数量} ++ (5,0);
    \draw (0.7,0.15) -- +(40:4.5) node[right] {供给};
    \draw (0.7,4) -- +(-40:5) node[below right] {需求};
  \end{tikzpicture}
  }
  \caption{用马歇尔形式呈现穆勒的长期价格理论}
  \label{fig:mulema}
\end{figure}

第一组,在\textbf{供给受到绝对限制}的地方,供给曲线\textbf{完全没有弹性(垂直的),
  价格取决于供给和需求}(见\cref{fig:mulea}),它包括酒、艺术品、绝版的书籍、古币、
土地的位置价值,以及随着人口密度提高所有潜在的土地。他也利用这个例子来分析垄断者
能够人为地限制供给的垄断情形。第二组商品是\textbf{制造业产品},供给曲线拥
有\textbf{完全弹性(水平的)},穆勒断定这些产品的生产成本决定了它们的价格。穆勒假定,
所有制造业都处于成本不变的情况下(见\cref{fig:muleb});即随着产量提高,边际成本不
变。穆勒的第三组商品是\textbf{农业}中生产的产品,他假定随着产量扩大,\textbf{边际
  成本递增},这些商品的价格取决于最不利条件下的生产成本(见\cref{fig:mulec})。因此,
他将边际收益递减原理应用到农业生产中,但没有应用到制造业产品上。尽管穆勒仔细地解
释了任何商品在具有价格之前,效用(需求)与获取的困难(供给)两者都必须存在,然而,其
结论中的术语使供求定理在三组产品中的基本适用性变得不明显。

穆勒清楚地看出了通过需求与供给力量,如何导致了市场均衡价格,也看出了:
\begin{quotation}
  适当的数学推导是等式的推导。需求与供给,即需要的数量与提供的数量将变得相等。如
  果在某一时刻不相等,竞争将使它们相等,在这一过程中,所借助的方式是价值的调整。
  如果需求增加,价值提高;如果需求减少,价值下降;此外,如果供给减少,价值提高;
  如果供给增加,价值下降。
\end{quotation}

当需要的数量等于供应的数量时,就实现了最终的均衡。

尽管穆勒没有使用数学公式/表格或者“供给--需求”曲线,他对价格决定的分析也显著地超
越了李嘉图的分析,这尤其是因为他所提出的概念性的工具明显地与“供给--需求”函数一
致。他未能涵盖的唯一一组商品是那些成本递减、拥有向下倾斜的长期供给曲线的商品。

在讨论下列问题时,称勒对价值理论也作出了一些原创性的贡献,这些问题包括非竞争性的
商品(他意识到劳动市场上的流动性远远不充分),厂商按照固定比例生产两种或更多产品
时的定价(羊毛与羊肉),当土地有可替代的用途时作为价格决定因素的地租,以及规模经济。
他对价值理论发展表示出的满意,在他下面的观点中显示出来:“幸亏,对于现在或任何未
来的经济学家来说,现在所保留的(1848)有关价值的定律中,没有什么需要整理的,有关这
一主题的理论是完全的。”

很多在穆勒之后进行创作的经济学家因这段陈述而感到愉快,这也可能是为什么马歇尔提出,
他自己对微观经济理论的贡献很快将会陈旧的原因。然而可以证实的是,从穆勒以来,我们
对供给与需求在竞争性市场下的资源配置中所起作用的一般理解,没有发生基本的改变。当
然,出现了很多允许更多技术性分析和更多见解的进展;但是,穆勒能够运用原始的技术工
具,完全不借助数学符号,完成对市场的重要分析,并很少有分析性的错误。穆勒微观经济
理论的主要遗漏,直到20世纪30年代才得以弥补,这一遗漏是指他\textbf{未能分析不完全
  竞争市场}。一些人可能会说,这一遗漏依旧在在,尚待弥补。

\subsection{国际贸易理论}

经济分析史学家因穆勒对国际贸易理论的页献而给予其赞扬。特别是穆勒对国际贸易收益在
贸易国之间分割的分析,可能是他对技术性的经济理论最重要最持久的贡献。借助比较优势
主张,李嘉图支持并扩展了斯密对非规制国际贸易收益的分析。……尽管李嘉图能够通过利
用比较优势来表明各国从贸易中获得的收益,然而,他没有说明酒和布的国际价格是什么,
以及贸易收益如何在两个国家之间进行分配。显然,英国更喜欢用1码布交换尽可能多的酒,
和葡萄牙更喜欢为了1码布而放弃尽可能少的酒。李嘉图简单地提出进出口交换比率或者说国
际价格将大约位于两个国内价格的中间。

穆勒考虑了贸易收益如何分割,并且给出了一个令人惊讶的正确答案,这样说是考虑到他没
有利用数学工具,同时,弹性的概念还没有被开发出来这一事实。马歇尔与埃奇沃斯两人后
来利用数学辅助和图表技术,更加精确地呈现了穆勒的主张,两个人都认可并且赞扬了穆勒
的贡献。穆勒断定,\textbf{进出口交换比率取决于两个国家对进口产品的需求}。在刚才引
用的例子中,如果英国对进口酒的需求远大于葡萄牙对进口布的需求,那么,交换比率和贸
易收益将有利于葡萄牙,国际价格将接近于2加仑酒换1码布。葡萄牙不必放弃更多的酒来获
得布。\textbf{对进口品需求的相对力量,取决于“双方消费者的倾向与状况”},国际价格
或者说进出口交换比率将是这样一个价值,它使“每个国家所需的从邻国进口的商品数量,
将恰好足够相互支付”。穆勒进一步展示了他所谓的“消费者的倾向与状况”是什么含义,
这清楚地表明他正在讨论需求曲线的位置与弹性,尽管他从未明确地形成需求弹性的概念,
但是,他描述了有弹性、无弹性、单一弹性的情形。

穆勒对贸易理论的其他贡献不太重要,但仍显示出他的分析能力。他将运输成本引进到对外
贸易的分析中,并且说明即使在具有比较成本差异的情况下,运输成本也有可能导致贸易不
会发生。他还分析了关税对于进出口交换比率的影响,指出价格与收入两者的变化如何引起
国家之间的贸易均衡,并且显示了由国家之间单方面转移支付所引起的贸易调整。穆勒之后,
直到Gotthard Bertil Ohlin,(1899一1979)和凯恩斯对古典国际贸易理论作出重大变动之
前,这期间经历了将近一百年的时间。

\subsection[穆勒的货币理论与超额供给]{穆勒的货币理论与超额供给:萨伊定律的重新思
  考}
马尔萨斯、查尔摩斯(Chalmers)、西斯蒙第都对萨伊定律进行过抨击,穆勒关注着这些批评,
并在他的一篇题为“关于消费对生产的影响”(Of the Influence of Consumptionon
Production)一文中反驳了这些批评。这篇文章写于大约1830年,但直到1844年才在《略论政
治经济学某些有待解决的问题》中发表,同时也收于《原理》第3篇第14章“关于超额供
给”中。穆勒捍卫萨伊定律,反对很多消费不足主义者的主张,即认为如果财富节省少一些、
非生产性消费支出多一些,经济体的经济状况就会变好。\textbf{20世纪之前,没有人能够
  比得上他对萨伊定律的捍卫。}穆勒承认,随着市场对可变的供给与需求状况作出反应,有
可能存在个别商品的超额供给,但是他认为,将这一分析引入宏观经济学,并推论说所有商
品的超额供给会永远存在,则是不合逻辑的。在对萨伊定律的捍卫中,穆勒对三种可能的经
济体进行了区分:物物交换的经济体、货币是一种商品且不存在信用的经济体以及存在信用
货币的经济体。他公开将货币引入对可能的一般性生产过剩的讨论中,通过这种方式,穆勒
大大地改进了支持萨伊定律的主张,这些主张以前是由李嘉图、詹姆斯·穆勒及萨伊本人提出
的。

穆勒非常清楚地表明,在一个物物交换的经济体中,永远不会存在总需求不足,因为提供商
品的决策就预示着对商品的需求。在一个简单的物物交换的经济体中,个人或厂商仅仅出于
需要而生产产品,并与其他产品交换。……如果引入货币,但是\textbf{货币的唯一功能只
  是作为交换的媒介,那么,结论是相同的。}然而,如果货币部分地执行\textbf{价值贮
  藏}的功能,那么,销售者可能不会立刻回到市场上去购买,人尽管产生了足够的总购买力
来实现充分就业,但是在现阶段并不能得到行使,从而会导致\textbf{总的超额供给}。

在研究这些问题时,穆勒通过发展经济周期心理理论,将亨利·桑顿成熟的货币分析再次引入
古典观点中。穆勒表明,\textbf{当信用被引入时,就存在商品超额总供给的可能性。}扩张
和繁荣期间信用的过度发行,紧跟着可能就是信用收缩,这是工商业界悲观主义的结果。
\begin{quotation}
  这种时候,才真正存在商品超过货币需求的过剩;换名话说,存在一种货币的供给不足。
  由于大量信用突然灭绝,每个人都不喜欢失去现款,很多人不惜代价急于得到现款。因此,
  几乎每个人都是销售者,简直没有任何购买者。
\end{quotation}

按照穆勒的观点,将信用货币引入经济体,会使超额总供给成为可能,这并不是因为马尔萨
斯总量供过于求意义上的生产过剩,而是因为工商业界预期的改变。穆勒说,任何这样的超
额供给将\textbf{持续较短时间},随着价格在经济体中发生变化,紧跟着的将是充分就业。
穆勒对萨伊定律所引起的问题以及货币在经济体中作用的讨论,其实际效应是反对马尔萨斯
的指责,捍卫古典体系的基本部分,并且基于信用货币与经济信心之间的相互作用,发展出
一种简单的经济周期波动心理理论。

\subsection{通货流派与银行流派}

穆勒有关货币理论的观点,是在时代背景中得到发展的,也反映出他的如下方法论,
即\textbf{对实际问题的反应指导着理论研究,而不是理论发展独立于政策问题。}当时的背
景是金银争论的延续,以及如何处理周期性的萧条和正在发生的金融中断。

金银争论的延续被称为通货流派/银行流派争论。\textbf{通货流派}(Currency School)维持
金银主义者的观点,\textbf{主张纸币本位与金本位的混合应当受到强硬规则的制约},像严
格的金本位一样来实行。他们认为这一政策是阻止货币通货膨胀发行的唯一方
法。\textbf{银行流派}(Banking School)则主张,需要\textbf{更加富有弹性的货币政
  策},只要银行遵循了\textbf{真实票据学说},就不需要对纸币发行实行控制。

穆勒认为,银行流派在市场平静的正常时期是正确的,但是他并不认为真实票据学说总是适
宜的。他认为可能会发生\textbf{投机性的金融繁荣}。在这种时候,通货流派关于纸币发行
与黄金相结合的政策就是适当的政策。

\subsection{工资基金:穆勒的放弃}

工资基金学说被一些经济学家和很多知名经济学家用来反对组成工会。根据\textbf{工资基
  金理论},工资率是由劳动力的规模和工资基金的规模决定的,劳动者提高工资的任何尝试,
无论借助什么手段,都将是无效的。\textbf{穆勒认为,他对经济思想的唯一贡献是区分了
  固定的生产法则与由制度和文化决定的分配法则},并且,他作出这种区分的理由是使他的
人道主义能够缓和李嘉图主义者的保守结论。

尽管穆勒接受了工资基金学说,但是他仍然支持组成工会。这点上,他遵循了亚当·斯密的
推理。斯密曾经指出,单个的无组织的劳动者与雇主就工资率进行讨价还价时,处于一种竞
争性的劣势。在穆勒看来,工会与罢工似乎是劳动者试图用来平衡雇主企业力量的一种适当
的工具。穆勒坚持工资基金学说这一情况,有可能被解释为他对无规制的人口增长后果的强
烈关注。《原理》第七版出版之前,穆勒在对威廉·桑顿的书评中,表现出他几乎完全赞同
了桑顿的主张,并断定那些认为工会无法提高工资的说法是无效的。

工资基金学说断言,劳动需求被工资基金的规模完全固定住了。穆勒现在放弃了这一观点,
他认为,\textbf{尽管能够用来支付工资的最大基金数量是固定的,但是既定的劳动力与工
  资率也可能并不用尽这一固定量。按照这一推理,工资率就不是最后被决定的;存在一系
  列的可能工资。工会因此能够通过讨价还价的过程来提高工资。}

然而,1871年出版的《原理》第七版在这点上并没有作出变动,原因在于,穆勒认为,这些
新的发展“要合并到关于政治经济学的一般论述中还不太成熟”。这相当令人困惑,因为
在1862年,在他的《原理》第五版中,穆勒已经推断工资率取决于雇主与雇员的讨价还价能
力,劳动者提高其能力的一个重要方法就是通过联合。\textbf{这一矛盾是穆勒在流露其人
  道主义情感时,又试图停留在古典经济学一般框架中的又一个简单例子};前者提倡以更加
平等的收入分配为中心的社会改革,后者则是他年轻的时候从其父亲那里学习来的。

\section{总结}

通过对李嘉图《原理》【〔1817年版)之后五十余年中正统经济理论发展的考察,可以揭示出
引人注意的矛盾与相反趋势。经济学的日益专业化,社会主义和人道主义著作的增加,以及
理论与事实之间的冲突都掀起了对李嘉图分析的批判。经济学家变得更加了解他们的学科,
并开始忙于研究经济学的范围与方法问题,以及实证经济思想与规范经济思想之间的区别。
随着经济发展和更多数据变得可用,理论与事实之间越来越大的分离变得明显了,并就主要
的李嘉图集成提出了重要质疑,诸如马尔萨斯人口理论、历史上的收益递减原理以及随着时
间变化利润率下降的预测。

后李嘉图时期最引人注意和令人惊异的方面,是经济学家面对相矛盾的经验证据,依然坚持
李嘉图模型预测的这种韧性。这在很大程度上可以通过他们对非常抽象和演绎的李嘉图模型
的热情接受得到解释。对严格的劳动价值理论中固有的逻辑难题的逐渐了解,以及对李嘉图
式社会主义者所施加的批评的回应,导致了利息的节制理论和生产成本价值理论的发展,其
中,劳动成本与资本成本两者都是组成部分。

这就是约翰·斯图亚特·穆勒脱颖而出的环境,他在很小的年龄就被用李嘉图传统加以培养,但他
对资本主义经济的不公正有强烈而深刻的感受。他尝试着将古典自由主义的不感情用事与社
会改革的人道主义相结合,来推进一个较少关注商业事务、更多关注个人提高与自我实现艺
术的社会和经济。

穆勒对社会改革的关注,使他坚持强调下列两者的区别,即不变的生产法则与可变的分配法
则的区别,后者受到制度的决定,并支配着个人收人分配。他努力建立理论与政策应用之间
的一致性,这使他更多地与斯密的传统而不是与李嘉图的传统结盟。他的折衷主义使他难以
在意识形态上分类;他的著作中包含着古典自由主义与自由放任的明显张力,然而,他也经常
提人对经济体实行政府干预。对穆勒来说,地主与社会其他成员之间的利益冲突是体系中的
不和谐因素。但是,他否定社会主义者对私人财产和竞争的谴责,指出了一些可以保留这些
制度的好处、同时消除其明显罪恶的调整措施。此外,他的乐观主义形成了关于静止状态的
一种新观点,摆脱了阴郁的李嘉图色彩。

穆勒对经济理论作出了持久的重要的贡献。尽管他并不承认,但他最终还是否定了李嘉图的
劳动价值理论,就此发展出了一种包含劳动成本与资本成本的长期生产成本价值理论。



%%% Local Variables:
%%% mode: latex
%%% TeX-master: "../../main"
%%% End:
