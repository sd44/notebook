\chapter{阿尔弗雷德·马歇尔与新古典经济学}

阿尔弗雷德·马歇尔被认为是新古典微观经济学之父这一头衔的两个竞争者之一(另一个是莱
昂·瓦尔拉斯)。以斯密、李嘉图以及约翰·斯图亚特·穆勒的工作为基础,马歇尔发展了一种
分析框架,该分析框架今天依旧作为通用的大学本科经济理论以及大多数经济政策的结构性
基础。对马歇尔的观点进行真正彻底的考察,将几乎包括今天所有的局部均衡微观经济理论。

\section*{马歇尔作为新古典主义之父的宣言}

马吹尔带着大学本科数学训练的经历,以及改善穷人生活质量的强烈人道主义情感走进了经
济学的世界。他早年的教育和家庭环境,使他愿意在英国的教堂担任神职。但是,他在剑桥
的大学本科学习展示了他对数学的强烈偏好与天赋。因此,毕业之后他就留在剑桥讲授数学。
然而不久,他便对学习形而上学、伦理学以及经济学感兴趣。到了19世纪60年代后期,他对
经济学产生了强烈的兴趣,以至于决定做一个学者型教师而不是牧师。他开始在剑桥讲授经
济学;,在两位早期数理经济学家古诺与汉…履能著作的影响下,他开始将李嘉图和约翰.斯
图亚特*穆勒的经济学转化为数学。

马软尔是在有利的历史时期开始其经济学研究的。我们已经记录了古
典理论基础的瓦解。马尔萨斯的人口学说主张实际工资将随着人口的增加
而下降,但是,英国的经济史持续表明实际情况与此相反。约翰,斯图亚
286
ei
第10章了阅尔弗雷德。蕊歌未与蒜古典经济学
特:穆勒已经不满意于工资基金理论,到1869年他明确地放莽了该理论。
德国历史学派和某些英国经济学家,例如,,T.E克利夫.莱斯利
(T.E.CliffeLeslie,1827--1882)与湛尔特,贝格认特(WaltBagehot,,
1826一1877),反对古典经济理论的一些基本原则。1871年,杰文斯与门格
尔对古典理论几乎专门强调供给进行了择击。由古典理论而来的政策也受
到了围攻。例如,日益增多的英国工厂工人的贫困生活和工作状况,与自
由放任的主张不相适应。因此,出现阿尔弗雷德.马软尔这样一个拥有巨
大学识与智慧的人的时机成熟了,从1867年至1890年,马软尔仔细地铸就
着供求分析的原理。

杰文斯草率地出版了其著作,宣称朱毁了古典价值理论,彻底改革了
经济理论;而在1890年在《经济学原理》(PrinciplesofEconomics)中谨慎
地提出其观点之前,在二十多年的时间里,马葡尔只是向他的学生与同事
提出过他的这些观点。正如凯恩斯恰当描述的那样:“杰文斯看到壶中的水
沸腾了,像个孩子似地欢呼起来,马软尔也看到壶水沸腾了,却悄悄坐下
来,造了一台发动机。”®马软尔建造的分析发动机既反映了他的个性,也
反映了养育他的环境。他早年的宗教信仰,后来显示为成熟的人道主义,
引发了他对穷人的深切关心,也使他乐观地确信经济研究能够提供改善整
个社会福利的方法。他的学识使他熟悉那些历史导向的经济学家们的反击,
这些经济学家反对如下观点,即经济理论是一种适用于所有时间和地点的
绝对真理的主体。1885年在当选剑桥教授的一次就职演讲中,马软尔谈到
了这一批评:“就经济学说能够独自宣称具有普遍性这一点来说,并没有什
么教条。它不是具体真理的主体,而是发现具体真理的发动机。”信

马软尔试图将他早期的数学训练与他的历史学背景结合起来,建造一
台适用于变化时代的“发动机”。然而,由于意识到约翰.斯图亚特.称勒
1848年得出的“价值理论是完全的”这一结论的草率性,马软尔预见到,
随着新理论的不断出现以适应不断变化的社会,他自己对经济学的贡献将
恋往陈旧。他也明白杰文斯因为用一种价值完全取决于需求的理论替代十
@@约翰:梅纳德.凯恩斯.传记文集.美国;于午线出版公司,1956:538
人@”参见亚瑟.C:庇古编辑的《阿尔弗雷德,马软尔纪念集》第159页,该书由美国凯拼与
米勤曼出版公司1956年出版。
1
!
:
!
287:
i(OYY)经济思想风
yg8
典生产成本价值理论,而拥有独创和被信任的权利。昌然马区尔希户他日
己的观点可能既是原创的也是持久的,但是他最希望的是能够被理解一一
不仅被他的经济学家同事们,而且普遍被社会,尤其是那些做生意的人所
理解。因此,尽管到1870年他已经开始设计出其理论的基本数学结构,并
在稍后发展了用图表说明供求分析的基本方法,然而在1890年之前,他并
没有实际公布他的发现一一当时,他仅在脚注和附录中使用了数学与图表。
马菊尔,这位前所未有的集理论家、人道主义者、数学家以及历史学家于
一身的人,试图为他所处时代对方法论的争论指明道路;同时,用边际主
义者的新工具调和古典分析的最佳之处,以此来解释价格的决定力量与资
源的配置。

尽管在经济理论的发展中马吹尔是一位杰出的人物,然而,他对理论
上和方法上的问题拒绝采取严格的立场,这给随后的几代经济学家造成了
很多麻烦。为了得出折衷的看法,他有时显得含糊和优柔骞断。他似乎经
常在说那得看情况,李嘉图是正确的但也是错误的;抽象的理论是有益的
也是有害的;历史的方法是有帮助的,但理论也是必需的;从一种观点来
看,支付给生产要素的报酬是决定价格的因素,但从另一种观点来看,它
又是价格被决定的因素。一些读者将这种关于理论与方法问题的机动性视
为真正智慧的标志,但是,一些读者尤其是比较抽象化的数理经济学家,
恼怒于他们所认为的马软尔经济学中优凶赛断的部分。不过,他的风格引
发了大量著作,这些著作都试图揭开马软尔的“真正用意”是什么。

尽管一百多年以前,马软尔为经济思想做出了他的贡献,但是,他仍
然吸引荐很多经济思想史学家们的注意力。我们特别建议熟读格罗尼维根
1995年出版的关于马吹尔的优秀传记。
经济学的部加
马软尔的《经济学原理》第一篇第1章以一个宽沁而灵活的对经济等
的定义开始:“政治经济学或经济学是对人类日常生活的研究;它考察个人
和社会行为中与获得和使用物质财富密切相关的部分。”人9
DD阿尔强雷德,马葡尔.经济学原理.第8版[1920].英国:寿殉米兰出版公司,1948:1
I
2
oa
这一定义中引起注意并有些讽刺意义的方面是,它是用两种不同的术
语一一政治经济学(politicaleconomics)和经济学(econimics)提到的。考
虑到马菊尔对经济学的宽泛界定,人们认为他将使用更加宽泛的政治经济
学术语。马吹尔使用两种术语,反映出他所处时代的一些方法问题。政治
经济学这一术语在当时比经济学更加普遍,它暗示了经济学与政治学是有
关系的,并且,经济学作为社会科学中的一门学科,与规范性的判断密切
相连。然而,马葡尔的一位同事和朋友约翰.内维尔凯恩斯(约翰*梅
纳德.凯恩斯之父)尤其对方法问题感兴趣,他在1891年出版了一本名为
《政治经济学的范围与方法》(TheScopeandMethodofPoliticalEconomy)的
书,其中清楚地区分了经济学的三个分支:实证经济学,它包括经济学的
科学分支,规范经济学,它考虑社会的目标应当是什么;以及经济学艺术,
它使实证科学分支的见解与规范分支所决定的目标相联系。约坦.内维
尔-凯恩斯声称,在实证分支的讨论中经济学与经济科学的术语要更好于
政治经济学,原因在于,这些名称强调了经济学的科学特征。与李嘉图和
约翰,斯图亚特.穆勒不同,马菊尔选择将他的著作称为《经济学原理》,
而不是《政治经济学原理》,并且最终停止使用政治经济学这一术语,而先
同经济学这一术语。就他选择经济学这一术语来说,具有讽刺意味的是,
他比几乎任何一个同时代的人都更多地实践了经济学艺术,而不是经济学
科学。他育焦于实用的理论,对纯粹的经济学科学并不感兴趣。这种变化
有两种可能的理由。第一种可能是马软尔希望他的方法与政治经济学的方
法有所不同。第二种可能是马菊尔试图在他所任教的剑桥,为经济学启得
独立研究领域的认同,而政治经济学这一术语暗示着不同领域之间的交释,
这不利于他的目的。

这一定义另一个引人注意的方面是它的宽泛程度与灵活性一一一些人
可能会说它的优柔寡断。根据定义,经济学怎样能够与政治科学、社会学、
心理学、人类学以及历史学相区分?马吹尔的松散定义并不是缘于玖忽和
不集中的思想,而是来自于一种有意识的不情愿,即不情愿明显地把经济
学从其他社会科学中分离开来。他指出,大自然也画不出这么清晰的分界
线,经济学家将学科的范围界定得太罕,这将做不成任何事情。在题为
“经济学的范围与方法”的附录C中,马砍尔考虑了(以其特有的折吏方
'
t
:
'
289,,
;
~gp
3NS经济思想区
1290
pn
i
ed
式)与允许每种学科独立发展相对照,发展一种统一的社会科学的优铀点
和可行性。使社会科学成为一体的观点吸引着他,但是,他也知晓伟大的
孔德与赫伯特.斯宾寨(HerbertSpencer,1820一1903)两人的努力都未能
实现这一目标。另一方面,他观察到借助于专业化,自然科学已经取得了
巨大进步。他最后断定,缺乏一些具体问题时这个事情不能得到解决:
与任何其他社会科学相比,经济学都取得了更大的进步,原因在于,
它比任何其他社会科学更清楚而精确。然而,其范围的每一次扩大,都会
使这种科学的精确性有所损失;损失是否大于因范围扩大而带来的收益,
这个问题是不能呆板决定的,中
马软尔指出,每个经济学家都应当界定经济学的克围,以符合他或她
自己的意愿,因为一些经济学家在比较窑的经济学范围内,更有可能全力
以赴地工作,另一些则在比较宽泛的框架内工作。他警告说,那些选择宽
泛的经济学定义,并向社会科学的其他领域扩展其分析的人,行为必须极
其谨慎,但如果他们仔细工作,他们就为经济学及其他社会科学提供了巨
大的帮助。

马葡尔在他对经济学范围的讨论中,提出了另外一个引人注意的问题,
即社会的需要与社会经济活动之间关系的复杂性。能把经济学描述为研究
经济活动满足社会需要的方式的学科吗?马葡尔否定了这个定义,原因在
于,它暗示着需要是独立的既定量,相对于需要而言,经济活动是第二位
的。在《经济学原理》第三篇第2章对需要与活动之间关系的论述中,马
吹尔试图纠正他认为不正确的结论,这一结论是由杰文斯和门格尔以及他
们的前韭得出的,他们似乎视“消费理论为经济学的科学基础"。他在最宽
泛的可能的背景下评价需求(需要)与供给(活动)的相对重要性。他的
观点是,我们的需要并不发生在我们内部并独立于我们的活动,相反,我
们的很多需要是我们活动的直接副产品。把这一思想应用于21世纪,就是
说将一个“雅皮士”家庭购买一辆小型货车的愿望,视为经济分析的起点
是错误的,原因是这一需要可能起因于家庭对其社会角色的感知。马吹尔
QD同上,第780页。
第10章同志弗雷乱.蕊软示与新古典经济学
提出,经济学家应当从对需求的初步研究开始,继续到活动与供给,表返
回到需求。他主张这将使他们能够理解需要与活动之间复杂的相互联系。
在经济分析中,如果被迫在需要的最高地位与活动的最高地位之间做出选
择的话,马鞭尔将会选择活动,这反映出他对强调供给的古典经济学的亲
和力,并使他与杰文斯与门格尔形成对照,他们强调需求:
因为在研究需要的学问中,很多让人最感兴趣的东西都来自于研究努
力与活动的学问。这两者相互补充,少了任何一方,另一方都是不完整的。
但是,如果两者中的一个超过了另一个而被称为人类历史的阐释者的话,
那么,无论是在经济方面还是在其他方面,它都是研究活动的学问而不是
研究需要的学问,人0
马吹尔基于宗教的人道主义关心,使他将消除贫困作为经济学的首要
任务。他认为,解决这些问题的关键存在于事实中,以及经济学家的理论
中,他最大的美梦是他正在建造的分析发动机能够揭示出贫穷的原因,并
最终识别出如何加以补救。在他对经济理论史进行评论的附录B中,他对
古典理论家,尤其是李嘉图没有认识到贫穷滋生贫穷进行了严厉的批评。
贫穷之所以滋生贫穷,是因为穷人没有足够的收入获得能使他们挣更多钱
的健康与培训。与古典理论家相对照,马葡尔一心一意地相信和极大提高劳
动阶层福利的可能性。

他对经济学范围的讨论,首先显示出他希望对以历史为导向的经济学
家的批评进行回应,这些经济学家要求比较宽泛的经济学定义;其次显示
出他希望讨论如下问题,即经济学应当做为一门狭窄的抽象学科来发展,
还是应当发展成一门统一的社会科学;再次显示出他希望回答边际效用经
济学家的问题,他们坚持认为消费理论应当优先于成本与供给理论;最后
显示出他希望对古典经济学中曾经困扰约翰.斯图亚特,穆勒的部分提出
异议,因为它为消除贫困提供了很小的希望。通常地,马菊尔试图就这些
问题提出一种折训的看法,而很少采取清晰的态度。
DD同上,党90页。
t
1
!
;
~291
-----nat
eg
eHOC)sryofEronomeSoicgth
了沁吹人尔论方法
马歇尔受到的训练和他的背景也反映在他对方法论的看法上。他的数
学才能使他充分了解数学作为经济学家所掌握的一种工具的力量,他对李
嘉图的密切研究也显示出通过构建抽象模型所获得的洞察力。他对历史和
历史上经济学家的广泛阅读,使他相信他们方法的价值,以及他们对古典
理论进行择击的正确性。他认识到古典经济学,尤其是李嘉图经济学的主
要缺陷是未能认识到社会变革。但是他看到,抽象理论与历史分析的结合
能够纠正这一缺陷,在附录B中,他沈扬亚当斯密是方法的模范。在附
录C“经济学的范围与方法”与附录D“抽象推理在经济学中的应用”中,
他给予历史方法以及德国历史学派高度的赞扬。马葡尔自己的方法是试图
将理论的、数学的以及历史的方法相混合。他承认一些经济学家更喜欢依
赖某种单一的方法,他并不反对这一点。对于马软尔来说,不同方法的运
用并不能表明冲突或者对立,原因在于,所有的经济学家都从事着一项共
同的任务。每种方法都从其角度将经济体的运转描述得更加清楚,并因此
增强我们的理解能力。

马软尔试图调和他所处时代对方法的争论,这使他容易受到来自各方
面的攻击。德国与英国历史导向的经济学家认为,马软尔的经济学方法太
抽象并具有刚性。20世纪美国经济学家托尔斯坦*凡拖仑以及追随他的所
谓的制度学派,主导了对马吹尔方法的强烈反击。抽象数学方法的提倡者,
恼怒于他对历史方法的赞扬,以及他关于理论与数学局限性的直率评论。
1906年,在写给专心于在经济研究中使用数学和统计学方法的朋友A.L.鲍
雷(A.L.Bowley)的一封信中,马软尔做出了切中抽象数学方法核心部分
的评论:-
我找不到任何对你有所用处的关于“数学-经济学”的注释:对于我
过去是怎样考虑这一主题的,我只有非常模糊的记忆。现在我从不阅读数
学,实际上,我已经忘记怎样将很多东西结合在一起成为一个整体。

但是我知道最近几年对于我所从事的主题,我有越来越多的感觉,即
一条涉及经济假设的优秀数学定理,非常不太可能是优秀的经济学:我越
"
'

Ek
:292
ps
和
第10章和癌饼弗雷德.蔬舱内与某古典红济学
来越多地依靠这些规则一一《1)将数学作为一种迷记语言来使用,而不是
作为分析的发动机来使用;(2)在你做完之前坚持使用它们;(3)转换成
英语;(4)然后用实际生活中重要的例子来说明;(5)烧掉数学;(6)如
果你在(4)上没有成功,就烧掉(3)。我经常遵循最后这一条。D
马黄尔的《经济学原理》包括第(3)步与第(4)步,并且和十以一种
不打算供他的经济学家同事使用,而是打算供受过教育的读者使用的风格
创作的。他所使用的数学,要么被放在脚注中,要么被放在数学附录中。
尽管马软尔竭尽全力避免使用经济学行话,并用来自最近的或历史上的经
济经验中的实例来说明每一条原理,然而在其背后,唯一的东西就是强大、
紧密、高度抽象的理论结构。

正如马歇尔没有提出关于经济学的整齐而精简的定义一样,他通常也
避免给出很多经济概念的准确定义。古典经济学赋予了土地、劳动、资本
这些所谓的生产要素比其适当的含义更加准确的含义。在经济体中,土地、
劳动以及资本经常是混合在一起的,以至于只有高度的抽象才能将它们分
离。马软尔因此指出:“我们……把制造一件商品所需要的东西编排为任何
方便的组别,并把它们称作商品的生产要素(factorsofproduction)。”没
有规定硬性而不变的定义:手边的问题指示着如何界定要素。类似地,在
分析供给时,马吹尔不得不着手解决成本问题。如果供给取决于厂商的正
常成本,那么,哪个厂商将被选择为正常厂商?这里,马葡尔又一次表现
出了他的灵活性,他陈述道:“为了这个目的,我们将不得不研究代表性生
产者(expensesofarepresentativeproducer)在那一总量上的费用,”®他的
平均厂商或代表性厂商的概念,并不是诸如算术平均、众数、中值一类的
统计量。相反,他提出应当通盘考虑一个行业来找出下列厂商的准确位置,
即由正常或平均才能的人管理的厂商,既不是行业中的新来者,也不是原
有的已经建立的厂商,而是其成本扬示出它们能够正常地使用可利用技术
的厂商。
”参见庇百的《纪念集》第427页。
@@参见马网尔的《原理》第339页。
@同上,第317页。
”CNY到济思想
马软尔表面上的含糊、易变,以及偶然缺乏理论上的冰确,并不尽由
他混乱的头脑产生的,认识到这一点很重要。他采取一种仔细考虑的方法
论立场。马软尔对微观经济理论的理解以及他的数学才能,本来能够使他
以一种更加简练的形式来呈现其大约七百页的《经济学原理》。事实上,他
在其数学附录中做到了这一点。但是,经济体实际上远比数理经济学所能
显示的要复杂得多。在其生涯的早期,马吹尔就设计出了关于市场经济体
的纯理论;到了大约1870年它已经相当全面。《经济学原理》数学注释21
是一般均衡模型的一页纸版本,它说明了最终产品需求、最终产品供给、
生产要素需求以及生产要素供给之间的关系。1908年,马舱尔写信给约
翰.贝艾.克拉克说:“我的终身都已经并将继续献给以现实的形式,尽我
所能地多呈现我的注释21。”C在其《经济学原理》中,马欣尔明确地为缺
乏精确进行辩护。在简要、清楚说明了经济体长期均衡条件之后,马吹尔
继续指出
在我们生活的世界里,这一切都是不真实的。现实世界中,每种经济
力量都在其周围起作用的其他力量的影响下,不断地改变着自己的作用。
在这里,生产数量、生产方法以及生产成本的变化,彼此互相更改着;它
们总是影响着需求的特征与范围,并且也受到后者的影响。此外,所有这
些相互影响都需要时间来产生效果,而且通常,没有两种影响是齐头并进
的。因此,在现实世界中,任何一种关于生产成本、需求以及价值之间关
系的简易学说,都必然是错误的;越是通过娴熟的说明,赋予学说清晰的
外表,它就越有害。如果一个人相信自己的判断力和实际直觉,那么,与
那些自以为研究了价值理论,并断然认为它很容易的人相比,就更有可能
成为一个好的经济学家,包
理解复杂性:马歌尔方法的实施
有两点原因使马众尔将对经济体的研究视为复杂而困难的。一方面,
QD参见庇古的《纪念集》第417页。
外参见马舱尔的《原理》第368页。
1
1
'
:294
We
第10章阿尔弗雷德.蕊欣尔与新古典经济学
每件事情似乎都取决于其他的事情:经济体的所有组成部分之间,和存在一
种复杂的通常又敏感的关系。田一方面,“时间是经济研究中遇到的那些困
难中的一个主要原因,它使得人类有必要借助其有限的力量循序渐进。”?
原因并不会瞬间带来最终结果;随着时间的变化,它们才显现出来。但是,
当一种原因,例如,需求增加的影响被感觉到时,经济体中的其他变量可
以独立地改变(例如,供给可能增加),因此,经常难以孤立某一单个原因
并确定其影响。如果自然科学的实验方法(借此有可能使得除一种影响之
外的所有影响保持不变,然后观察重复实验的结果),能够为经济学家所使
用,那么,这个问题将不会存在。但是,因为实验方法对经济学家不可行,
所以,必须使用一种可替代的选择。当马软尔小心地发展其基本思想体系
时,他提供了这种可替代的选择。

按照这一体系,因为经济学家不能使所有的变量保持不变,这些变量
可能影响某种既定原因下的结果,所以,他们必须通过假设,在理论层面
上做到这一点。为了在分析经济体中复杂内部关系时取得一些进展,我们
假设某种因素发生变化时,“其他条件保持不变(ceterisparibus)”。在任
何分析的开始,很多因素都保持不变;然而,随着分析的进行,更多的因
索被允许变化,目的是达到较强的现实性。其他条件保持不变的方法使对
复灯问题的处理变得可行,代价是缺少一定的现实性。

马软尔首次也是最重要的一次对其他条件保持不变方法的运用,是发
展局部均衡(partialequilibrium)分析形式。为了分解一个复杂的问题,我
们将要分析的经济体的一个部分孤立起来,忽视但不否定经济体所有部分
的相互依赖。例如,我们分析孤立于所有其他影响的单个家庭或厂商的行
为。我们在分析既定的行业中形成特定价格的供求条件时,暂时忽视所分
析行业的产品与其他行业的产品之间复杂的替代与互补关系。局部均衡方
法的一项重要用处是,就既定原因的可能结果来说,取得一个最为接近的
结果。因此,这对于处理政策问题尤其有益一一例如,预测对进口手表征
收关税的影响。在局部均衡分析方法中,能够运用简单的供求分析来预测
这样一种政策的直接含义。马软尔的步骤是首先在局部均衡的框架中,将
OD同上,第366页。
295;
NH经济轧想史
ry2EcorordeHorsgh
一个问题限制得非常狭窄,使大多数变量保持不变,然后通过允许其他条
件变化,缓慢而仔细地扩展分析的范围。他的方法被适当地称为一次一事
汰(one-thing-at-a-timemethod)。
HH|间问题
经济分析中的一个主要困难是,原因要花费时间才能产生出丝示。正
确解释了既定原因短期结果的任何分析或结论,就长期来说,都将可能是
不正确的。马吹尔对其他条件保持不变方法的运用,与他处理时间的方法
是相应的。在有时称作即期或很短时期的市场周期中,很多因素保持不变。
随着时间周期延伸到短期、长期以及也被称为特别长的时期的长周期,越
来越多的常数被允许变化。时间段稍微影响需求分析,就会给供给的分析
造成非常多的混乱。

为了研究由时间引起的问题,马吹尔定义了四个时间周期。他承认他
所做的区别纯粹是人为的,因为“大自然没有在实际生活的经济环境中面
出这样的分界线”。2马软尔的时间概念不是一个用时钟的小时数度量的序
时时间;而是一个分析上的构造。不同的时间周期是根据厂商经济学与供
给经济学来界定的。市场周期(marketperiod)如此之短,以至于供给是不
变的,或者说完全没有弹性。不存在价格对供给数量的反作用行为,因为
对厂商来说周期太短了,使之不能够对价格的变化做出反应。在短期
(shortrun〉中,厂商能够改变生产与供给,但不能改变设备规模。在此存
在一种反作用行为,因为较高的价格引起较大的供给数量,供给曲线向上
倾斜。在短期中,厂商的总成本可以划分为两个部分:随着产量变化的成
本,马吹尔称之为特别成本、直接成本或者最初成本(primecosts),现代
教科书中称之为可变成本;以及不随产量变化的成本,马吹尔称之为补充
成本(supplementarycostis),现代教科书经常称为不变成本。短期中可变
成本与不变成本的区别显然是从马菊尔对经济世界的观察中得出的。在分
析厂商行为时,它成为一种重要的分析工具。在长期(longmn)中,设备
规模能够改变,全部成本都成了可变的。与在短期中相比,供给曲线更加
QD同上,党378页。
ee
第10章陪尔弗雷德.蕊组水与半吝典经济沧
富有弹性,因为厂商能够通过改变设备规模进行充分的调整以运应变化的
价格。一个行业的长期供给曲线能采取三种一般的形式:它能够向右上方
倾斜(成本可能增加);它能够是完全富有弹性的(成本可能不变);或者
在非常状态下,它能够向右下方倾斜(成本可能下降)。长周期(secular
period)或者说特别长的时期,允许技术与人口发生变化,马吹尔分析从一
代人到另一代人的价格运动时,用到了这种形式。

显然,马软尔的时间周期不是用日期来度量,而是指厂商与行业的供
给条件。一个设备规模只能很缓慢地改变的资本非常密集的行业,例如钢
铁行业,在序时时间上,其短期可能与男一个设备规模能很快改变的行业
的长期一样长。尽管马歌尔为微观经济理论几乎每个部分都增添了内容,
然而,他的主要关注点以及他最伟大贡献的来源,是他的时间对供给影响
的分析。他发现,价格分析中的主要困难在于确定时间的影响,在晚年他
声称这一领域需要做更多的工作。1908年在写给约翰.贝蒋.克拉克的一
封信中,他列举了仍然需要做大量工作的五个主题,其中最重要的是“详
细阅述时间的影响”,儿
了廿歌泵的过必
在19世纪最后二十五年时间里,关注和需求与供给在价格理论或者价值
理论中相对重要性的经济学家们,发生了一场争论。就像约输,斯图亚
特.穆勒的《原理》所提出的那样,上古典经济学强调供给;然而杰文斯、
门格尔、瓦尔拉斯强调需求,杰文斯与一些人走得更远,他们断言价值完
全取决于需求。难以评价这一争论对马软尔相对价格理论的内容与形式的
影响。他声称,他自己的价值与分配观点的主要成分,在1870年之前就设
计出来了,但是,“如果他使自己费心地去权衡和度量对创新的要求权,那
么,他将是轧春的。”®马菊尔烦恼于对其供求分析的批评,这些批评指出
他在试图调和古典观点与边际效用学派。他声称,他在寻找真理,而不仪
仅是和和平,此外,在杰文斯、门格尔、瓦尔拉斯开始就供求分析进行创作
QD参见庇古的《纪念集》第417贝。
G)同上,第148页。
29),
-1

人

总让
StoryofGeorormeeDispth
之前,他就已经明确地表达了他的供求分析。

马黄尔认为,对时间影响的正确理解,以及对经济变量之间相互依赖
的意识,将会解决是生产成本还是效用决定了价格这一争论。最终产品的
需求曲线向右下方倾斜,因为个人将在较低的价格上购买更大量的产品。
供给曲线的形状取决于所分析的时间周期。时期越短,需求在决定价格中
的作用就越重要;时期越长,供给在决定价格中的作用就越重要。在长期
中,如果存在不变成本,并且供给完全富有弹性,那么,价格将单独地取
决于生产成本。然而,一般来说,争论需求还是供给决定了价格是徒劳的。
马软尔运用下面的类推来表明,因果关系并不是一件简单的事情,任何斌
图找到单一原因的尝试注定要失败;
我们争论价值是由效用决定的,还是由生产成本决定的,在相当程度
上像在争论是剪刀的上片还是下片剪下了一张纸。的确,当剪刀的一边拿
着不动时,移动另一边才能把纸前下来,我们大致可以说,,纸是由剪刀的
第二边剪下来的。但是,这一表述并不十分准确,只有当它仅仅是对所发
生事情的一种通俗而不是科学的解释时,才可以这样说,人0
或许其至更为重要的是,马吹尔坚持认为边际分析馈很多经济竺家和锯
误地使用。他说,他们认为似乎正是边际价值(无论成本、效用还是生产
力)以某种方式决定了全部价值。例如,按照马吹尔的观点,在分析最终
产品的价格时,说边际效用或边际成本决定价格就是错误的。边际分析员
是提出“我们必须借助边际量来研究支配全部价值的那些力量"。@边际效
用或边际成本并不决定价格,因为它们的价值连同价格一起,与在边际量
上使用的要素相互决定。这里,马吹尔再一次提出了一个非常适当的类比
来表明他的观点。杰文斯孤立了价格决定中的本质因素:效用,成本,价
格。但是,他错在试图寻找单一原因,并将过程视为一连串的因果关系,
即生产成本决定供给,供给决定边际效用,边际效用决定价格。马吹尔认
为这是错误的,原因是它忽视了这些因素之间的相互关系和相互的因果联
@参见马吹尔的《原理》第348贝。
(5)同上,第410页。
!
1
!
!
:298
第10章辐永缆银德.蕊散粹与圭古典经济学
系。如果我们把三个球放进一个碗中,第一个是边际效用,第二个是生产
成本,第三个是价格,那么,说任何一个球的位置决定了其他两个球的位
置显然是错误的。但是,球相互决定彼此的位置是正确的。因此,需求、
供给以及价格在边际量上彼此相互作用,并相互决定它们各自的价值。

在《经济学原理》附录1和第五篇最后一段中,马软尔试图将他的价
格理论,置于李嘉图价值理论以及是效用还是生产成本决定了价格的争论
背景中。马欣尔认为,他自己的价格理论从根本上说与李嘉图有关。尽管
对此边际效用经济学家几乎没有赞同的,可是马软尔仍然指出,李嘉图认
识到需求的作用,但仅给予了有限的注意,因为它的影响易于明白;李嘉
图反倒是把他的精力放在更加困难的成本分析上。马吹尔认为,李嘉图的
生产成本价值理论既包括劳动成本,也包括资本成本。大多数经济理论史
家认为,这是对李嘉图过度宽宥的刚释。按照马软尔的观点,李嘉图价值
理论的主要缺陷是,他没有能力处理时间的影响,加上没有能力清楚地表
达他的观点,从而使其缺陷更加严重。马加尔耕定了杰文斯和其他边际效
用经济学家所声称的,他们有效地粉碎了李嘉图的价值理论,并用一种几
乎专门强调需求的正确的版本加以奉代。马葡尔把他自己的贡献仅仅视为
对李嘉图思想的扩充与发展,并认为他这样对待李嘉图,使得李嘉图价值
理论的根本基础完整无扎。在我们考察马菊尔的其他观点之前,我们将推
迟对其价值理论的评价。
也歌人尔论需求
马吹尔指出,需求对价格决定的影响相对来谨容易分析,他的这一主
张可能是完全正确的。然而,需求理论中仍然存在一些问题,马菊尔未能
令人满意地解决。他似乎意识到这些困难并通过假设回避了它们。他对需
求理论最重要的贡献是,他清楚地阅明了需求的价格弹性(priceelasticity)
的概念。价格与需求的数量彼此反方向地发生关系;需求曲线向右下方倾
斜。价格变化与需求量变化之间关系的程度,通过价格弹性系数揭示出来。
价格弹性系数汶.
eo。-咒求数量变化的自分比__A9/9
D价格变化的百分比Ap/p
299
汪针圭圭霜浊出污护护扩这沪针续[者搞抽抽抽抽半
Wp
NM经济四想史
|AhlyoAEamesPspA
30()
因为价格与需求的数量反方向地发生关系,所以,计算出来的需求的
价格弹性系数为负数。按照惯例,用一个正数来表示系数,所以,添加一
个负号到等式的右边。产品的价格习以需求的数量将等于购买者的总支出,
或者也可以说是销售者的总收益(pxg=7TE=TR)。如果价格下降1%,需
求的数量增加1%,那么,总支出或总收益将保持不变,系数将是一个等于
1的值。如果价格下降且总支出或总收益增加,那么,系数将为大于1的
值,商品就被说成是价格有弹性(priceelastic〉,的。如果价格下降了一个始
定的百分数,需求的数量增加了一个较小的百分数,那么,总支出或总收
益将下降,系数将为小于1的值,商品就被说成是价格无弹性(priceinelas-
tic)的。马葡尔也将弹性的概念用于供给方面,借此给予经济学田一种极
其有用的工具。尽管在较早的文献中就已经提出了价格弹性的概念,然而,
正是马葡尔运用其数学能力,能够精确地将它表示出来;因此,他被认为
是价格弹性的发现者。

按照马软尔的观点,个人淘望商品是因为通过商品消费能够获得效用。
马吹尔所使用的效用函数的形式是加法了晴数;也就是说,他通过将消费每
种产品获得的效用相加得到总效用。消费产品A获得的效用,只取决于所
消费的A的数量,而不取决于所消费的其他产品的数量。因此,震代与互
补关系就被忽视了。加法效用函数被规定为:

U=fqnt+hqe+hqgct+**+fnAn

当时,实践中所使用的效用函数明确地认识到了互补与蔡代关系,被
表示为:

U=f(qa,Ge,Gc,'**,qn)

弗朗西斯Y:埃奇沃斯与欧文.费委是两个与马软尔同时代的人,他
们提出了现在所使用的更为一般化的效用函数。马菊尔运用加法效用函数,
最重要的含义涉及收入效应,对此我们将简要地加以论述。

马软尔假定通过价格系统可以度量效用。如果一个人花了2美元购买男
一单位的产品A,花了1美元购买男一单位的产品B,那么,A提供的效用
一定是B所提供的两倍。他也认为效用的群体间比较是可能的,原因是在
群体比较中个人的特性被冲刷掉了。|

在马座尔的框絮中,需求理论最重要的任务是解释需求曲线的形状。
让铺南盾六间洁轴缚汇挫汉潮畏革站凡间计间相洁间计寺由护浊近才
Bi
第10音阿尔缆甸德.蕊软术与鞘三内经济凶
如果一件商品的边际效用随着更多的商品被消费而下降,那么,能够因此
靳定个人将为更大的数量支付较低的价格,从而断定需求曲线是负和斜率吗?
马欣尔认同边际效用递减〔区森第一定律),并且曾明了为消费很多商品的
个人带来最大效用的均衡条件(苹森第二定律):
MOa_MUeVwMU,(10.1)
PpP,P,
在均衡状态下,消费者将按照下列方式支出,即兹在任何一种最终产咒
上的最后一美元,与花在任何其他产品上的最后1美元具有相同的边际效用。
这些边际效用对价格的比率,将等于货币的边际效用,这也因此揭示了货币
的边际效用。货币的边际效用是从最后1美元支出中获得的边际效用。如果储
蓄被认为是一种产品,那么,货币的边际效用就是从最后1美元收入中获得的
效用。单个产品的边际效用等于它的价格乘以货币的边际效用:

MU,=P,:MUy(10.2)

为了了和解所遇到的一些问题,以及马软尔对这些问题的解决办法,我
们来完成需求曲线的推导。如果我们从效用最大化的个人开始,随后降低
一种产品的价格,那么,我们就能得出价格与需求数量之间的关系。利用
等式(10.1)与《10.2),我们看到,降低产品A的价格Ph,,在某些条件
下,将引起需求数量的增加。降低产品A的价格将具有两种效应。替代效
应(substitutioneffect)反映出如下事实,即现在产品A比它的替代品相对
便宜了,因此,个人对产品A的消费将增加。蔡代效应将总是导致在较低
的价格上更多的消费,在较高的价格上较少的消费。价格变化所产生的收
入效应(incomeeffect)比较复杂。降低产品A的价格提高了个人的实际收
入。以较低的价格,个人能够购买到与以前一样数量的产品A,并有剩余收
入能够花在产品A或其他产品上。例如,如果产品A的价格为1美元,以
前购买10个单位,那么,将产品A的价格降为0.9美元,将增加1美元的
实际收入。正常产品(normalgood)是指其消费随着收入的增加而增加。如
果产品A是一种正常产品,其需求曲线将向右下方倾斜。借助替代效应与
收入效应,降低其价格将增加需求的数量。

如果产品A是一种低档产品(inferiorgood),就会产生其他复杂的因
过。低档产品的消费随着收入的增加而减少。在消费者的预算中,碎牛肉
302
ee0HN经济思想史
可能是一种比较适当的低档产品。随着收入增加,碎牛肉的浓旨量将会下
降,因为更好的牛肉块蔡代了碎牛肉。如果产品A是低档产品,那么,巾
于蔡代效应,其价格下降将会导致消费增加,但是,收入效应又会导致消
费减少。如果替代效应大于收入效应,需求曲线将为负和斜率;但是,如果
收入效应大于兰代效应,需求曲线将为正斜率。向上倾斜的需求曲线的可
能性,极其扰乱需求理论。理论上的可能性存在,但是还没有经验信息表
明向上倾斜的需求曲线实际存在。

马软尔首先陈述了需求的一般法则:“随着价格下降,需求量增加,随
着价格上升,需求量下降。”%然后他注释说,罗伯特吉芬(Robert
Giffen)所掌握的资料指出,比较贫穷的人对面包的需求曲线可能向右上方
颌斜。换句话说,对于这些人来说,面包价格上升引起肉的消费和比较昂
贵的食物的消费减少,面包的消费增加。因此,在理论文献中,收入效应
比替代效应更强的低档产品,被称为吉芬产品(Giffengoods)。此外,关
于所谓的吉苏悖论,尽管存在相当多的理论文献,然而,没有可接受的统
计资料表明产生过向上倾斜的需求曲线。

我们回到推导需求曲线的理论问题上,看马软尔是如何处理这些问题
的。因为他使用加法效用函数进行研究,所以,在推导需求曲线的正式数
学处理中,他忽视了替代与互补关系一一尽管他的确从特征上论述了这些
问题。马葡尔简单地假设价格微小变动的收入效应可忽略不计;也就是说,
对于任何单个商品价格的微小变动,货币的边际效用保持不变。这样的话,
如果我们降低等式(10.1)中产品A的价格,产品A的需求量增加,边际
效用下降,直至MU,/P的比率变得与其他商品的比率相等,并且所有商
品的比率再次等于不变的货币的边际效用。还可以从另一个角度研究马史
尔的步骤。利用等式(10.2),由于边际效用递减原理,产品A价格的下降
(假定货币的边际效用不变),必定导致其消费的增加。

马软尔通过假定货币的边际效用不变来消除这些理论上的难点有两点
理由:第一,他没有理论工具来清楚地区分蔡代效应与收入效应;第二,
他宣称一种产品价格微小变动的收入效应是如此之小,以至于将其忽略不
QD同上,第99页。
Pomp
wi
0pe
oe
wy
第10章阅外纺沁德。蕊鞭淮与亲古内经济学
计,没有什么不利有影啊。
消费者剩余
马软尔认为,对于价格的微小变动,抽币的边际效用不变,他的这一
观点使他(或者因此他认为)得出了现在为人所知的福利经济学领域的某
些结论。这又是一例,表明马软尔第一个奖入新的经济理论领域后,紧跟
着产生了解释和扩展其分析的大量文献。消费者剩余(consumers”surplus)
的概念首先由马软尔提出,时至今日,福利经济学文献中仍然在讨论这一

利用等式(10.2)MU=P。*MUr,并假定货币的边际效用不变,则
产品A的价格与产品A的边际效用直接相关。马歌尔断定,产品A的价格
是产品A对消费者的边际效用的一种度量。由于边际效用递减,需求曲线
向右下方倾斜。其向下的斜率表明,与一种商品稍后的消费单位相比,消
费者愿意为该种商品较早的消费单位多支付些。然而,在市场上消费者能
够以一种价格购买其消费的所有单位。因为这一价格度量了所消费的最后
一单位的边际效用,所以,消费者以低于他们愿意支付的价格得到了和较早
的单位,即边际内单位。消费者愿意支付的总量与他们实际支付的总量之
间的差额构成了消费者剩余。

马吹尔希望运用消费者剩余概念得出有关福利的结论;因此,他关注
的是消费者作为一群人时的剩余,而不是个别消费者的剩余。他运用市场
需求曲线,而不是个别需求曲线来研究。给定如图10.1所示的市场需求曲
线,我们能分析消费者剩余。如果市场价格为0C,需求量将为OH。因为
DD'是一条市场需求曲线,所以,存在愿意支付高于0C价格的购买者。第
OM个消费者本来愿意支付等于MP的价格,但是只支付了相当于MR的价
格。因此,RP就代表了消费者剩余。所以,其他边际内购买者都得到了一
个消费者剩余。总的消费者剩余等于CDA,它是消费者购买商品的支出
(或者说0C4H)与他们本来愿意的支出(或者说0D4H)之间的差额。

因此,CD4是对消费者购买商品时获得的货币收益的度量。可以稍币
不同地表达这一结果,即实行完全价格歧视的歼断者,将使消费者移动到
其需求曲线下方,在这一过程中获取总收益0D4H;但是,在所有消费者都
303
eeehe
二
hl
erm
eHstoryofGeonorZODergHh
和
1
!
:304
po
Dy
to
toe
以单一价格0C进行购美的竞争性市场上,消费者的总文出为0CA有HH。因此,
CD4是消费者节省的数量,或者说是他们的货币收益。然而,马软尔希望
用效用来度量收益,只有存在一种不变的度量标准将价格转化为效用时,
货币收益才能被表示为一种效用收益。当我们从价格0D到MP青到H4向
下移动时,如果货币的边际效用保持不变,那么,马软尔的消费者剩余就
是表示从产品消费中获得效用的一种可以接受的方法。
D
____\Pr
|
价格!

|A

CR
|I
I
l|
!|D
!I
|I

0
MH

数量
图10.1消费者刺余
马葡尔运用价格来度量效用取决于两个假设(1)和存在一种忽视蕉代己
互补关系的加法效用函数;(2)价格微小变动的收入效应忽略不计一一也就
是说,货币的边际效用不变。运用一个更加一般化的非加法效用函数,埃奇
沃斯和欧文*费雪分别提出并说明了,尽管能够运用加法效用函数来度量效
用,但是,如果考虑到替代与互补效应,这种方法将是不可行的。此外,对
于马葡尔和其他人提出的需求理论中的享乐主义因素,也存在普遍的批评。
马软尔通过进行一些较小的术语变化,例如效用的满足来回应这些批评,但
他还是基本上坚持如下观点,即价格能够被用来度量效用。马吹尔明了与度
量消费者剩余相关的问题,这使他在福利经济学的应用中,仪在价格发生敏
小变化时才使用这种度量。对于一些价格变化《例如,图10.1中Hh附近的
价格),货币边际效用不变的假设看上去并不是不现实,尤其是所讨论的商品
第10章阿尔弗雷德.蕊吹示与新上症典经济学
支出仅代表了消费者总支出的一小部分时。对于大多数商品来说,其价格做
小恋动的收入效应有可能很小,以至于能够被忽略掉。
和%啤与各矢|
马软尔运用消费者剩余的概念来分析税收的福利后果。通过考察最简单
的情形即一个成本不变的行业,就能理解这一分析的实质。该行业用图10.2
中所示的具有完全弹性的供给曲线来表示,假定行业处于均衡状态,需求为
DD',供给为SS',价格为HA4,消费者剩余为SD4。现在征收了一项等于5s的
税收,使供给曲线移动到ss’'。消费者剩余的损失为Ssa4,税收为SsaK。消费
者剩余的损失超过了收入的增加,数量上为Kah4。因此,对成本不变行业所征
收的税是不合意的。这一分析也能类似地用来表明,对成本不变行业的补贴
也是不合意的,原因是净成本将超过净收益。假定需求为DD',供给为ss",
价格为ha。一个数量上为Ss的补贴将使供给曲线向下移动到SS'。消费者剩
全的增加量为Ssa4,它等于补贴总支出SsLh减去haL所余的那一部分。
saLgs!
价格I
4'
.SKI|5
I|Dp'
|t
|1
11
11
0及F
数量
图10.2税收、补贴以及消费者剩余
然后,马和软尔将这一分析扩展到收益递减行业(向上倾斜的供给曲线)
与收益递增行业(向下倾斜的供给曲线)o2假定存在收益递减,如果供给
QD同上,弟468~476页。
!
305);
i.
~、
Sn
人OY经济息想网
306
曲线倾斜得很了汗,使得税收超过了消费者剩余的损失,那么,征税将会吉
致福利增加。同样的方式,对成本递减行业补贴将增加福利,原因在于消
费者剩余的增加将超过补贴成本。马菊尔因此断定,对社会来说,对某些
收益递减行业征税,并用集中的收入补贴收益递增行业,可能会从中获得
利益。因为全部分析依赖于效用能够用消费者剩余来度量这一不确定的观
点,所以,它在制定政策时的实践价值就是有问题的。马哑尔提出这一分
析的目的,不是为税收和补贴提供一套精确的规则,而更多地是用以表明
非规制的市场并不总是导致最佳的资源配置。亚瑟.C,庇古采用了这些开
创性的建议来形成一种扩展的福利经济学理论。
也钦泵论供给
马葡尔为成本与供给分析葛定了基础,这一分析是当前公认的大等本
科课程讲授的内容。他对供给理论最重要的贡献是他的时间周期概念,尤
其是短期与长期的概念。他正确地认识到市场周期、短期以及长期中行业
供给曲线的形状,虽然他对这些形状的经济原因的解释经常是不完善的、
混乱的,甚至有时是错误的。

市场周期没有什么难点;此时供给完全没有弹性。现代微观经济理论
将厂商与行业短期供给曲线的形状解释为取决于收益递减原理。马软尔指
出,为了分析的目的,在短期中将厂商的成本划分为不变成本与可变成本
是有益的。然而,马默尔没有基于收益递减原理,确立下列两者之间的准
确关系,即不变成本和可变成本的区别与厂商短期成本曲线的推导。他通
常是在长期分析的背景下,主要将收益递减原理运用于土地上。

他的确在短期中运用了不变成本与可变成本的区别来表明:即使发生
亏损,只要弥补了总可变成本,厂商将会在短期内继续经营。在这些情况
下,厂商实际上通过经营使其亏损最小:停业将导致亏损等于总不变成本,
但是,只要总收益超过总可变成本,通过经营而产生的亏损就低于总不变
成本。因此,在一个完全竞争行业中,厂商的短期供给曲线,相当于其边
际成本曲线在其平均可变成本曲线以上的部分。以其特有的现实性,马拘
尔继续推断,当价格下降到平均成本以下发生亏损时,短期中厂商的实际
供给曲线不可能是其边际成本曲线。他说,厂商不愿意按照不能弥补其全
0
第10章同志弗雷德。马内相当精在肉经济学
部成本,即不变成本与可变成本的价格来销售产品,原因在于,他们担心
“扰乱市场”"。扰乱市场意味着今天以低价销售,并阻止市场价格明天上升,
或者按照使行业中其他厂商抱怨的价格销售。因此,当发生亏损时,真正
的短期供给曲线,并不是边际成本曲线介于平均可变成本曲线与平均成本
曲线之间的部分,而是位于边际成本曲线左侧的一条供给曲线。在这一论
述中,马葡尔去掉了完全竞争市场的假设,原因是在完全竞争的严格定义
中,没有厂商会关心向市场大量抛售商品,或关心其行为对行业中其他厂
商造成的后果。当放弃完全竞争的假设时,在马吹尔对市场运行的论述中,
能够部分地找到琼,罗宾逊(JoanRobinson,1903一1983)《不完全竞争》
(JmperfectCompetetion)与爱德华.H.张伯伦(EdwardH.Chamberlin,
1899一1967)《垄断竞争理论》(TheoryofMonopoliaticCompetition)的灵感。

尽管根据现代标准,马软尔对厂商长期成本曲线和供给曲线,以及行业
长期供给曲线的论述明显地不完善,但是,他在这些领域的早期努力,引发
了20世纪20年代和30年代一系列令人关注的文章,其中最为重要的是克拉
彭(Clapham)、奈特、斯拉法以及维纳〈Viner)的文章。马吹尔指出了决定
厂商成本与供给曲线形状与位置的长期力量。首先是内在于厂商的力量。随
着厂商规模增加,内部规模经济(internaleconomies)导致成本降低,内部不
经济导致成本增加。马欧尔对内部规模经济的经济原因的论述,,令人相当添
意,但是,他对内部不经济的论述最少,并且,他并没有真正面对下列问题,
即规模经济与不经济之间的关系及其对厂商最佳规模的影响。

不过,马软尔对外部经济(extermaleconomies)与不经济的论述,引出
了大量关于其分析中暗含的理论问题的文献。马吹尔希望对厂商和行业向
上倾斜的短期供给曲线,与一些行业出现的随着时间的变化,成本与价格
下降的历史证据进行调和。他的这种调和是以外部经济的概念为基础的。
随着行业的发展,外部经济一一马吹尔从未解释这些是外在于厂商还是外
在于行业一一导致厂商和行业成本曲线与供给曲线向下移动。存这些情况
下,行业的长期供给曲线将向下倾斜:将在较低的价格上提供较大数量的
供给。外部经济的主要原因是一个行业中所有厂商成本减少,当所有厂商
坐落在一起并分享他们的思想时,就会出现这种情况。地方化也会给当地
带来节省成本的辅助行业和熟练的劳动。
307!
人1
!(toyofEconrormeeMoisgAs
马歌尔对成本与供给的考察,提出了1900年至1940年之间被加以考察
的很多重要的理论问题。成本与供给曲线形状的经济原因是什么?为什么
一些行业的短期供给曲线上升而成本与价格在长期中下降?内部经济和外
部经济与竞争性市场相窑吗?
也歌尔论分本
马钦尔对生产要素价格决定力量与收入分配的解释,己其分析中的其
余部分是一臻的。在此,与在其他地方一样,他经常宽宏大度地认同对其
理论的批评例如,那些对他的分配的边际生产力理论的反击。用于解灵
最终产品价格的基本供求分析,以及短期与长期之间的区别,也同样被用
来解释地租、工资、利润以及利息。生产要素的需求是一种引致需求,它
取决于要素的边际产品价值。然而,边际产品难以分离,原因在于,技术
通常要求一各要素的增加伴随着更多其他要素的增加。马钦尔通过在边际
量上计算他所谓的净产品,解决了边际产品的度量问题。如朵一名新增劳
动者需要一把铁锤,那么,劳动的净产品就是劳动者新增的总收益减去铁
锤的新增成本。然后马葡尔指出,将要素定价理论称作分配的边际生产力
理论是错误的,原因是边际生产力仅度量了对要素的需求,而要素价格是
由边际量上的需求、供给以及价格相互作用决定的。在对他的边际生产力
家念及其关于劳动与工资的度量进行解释之后,马葡尔提价对边际生产力
理论进行谨慎的阅释:

这一学说有时作为一种工资理论被提出来。但是,任何这类主张都站
不住脚。一个工人的所得趋向于等于其劳动的净产品,这一学说本身没有
什么实际意义:因为为了估算净产品,除了他自己的工资外,我们还得考
虑他所生产的产品的全部费用。

就这一学说包含着一种工资理论而言,这种异议是有效的,尽管如此,
就这一学说阅明了决定工资的原因之一而言,这一异议又是无效的了中
他说,要率结合的比例取决于它们的边际产品与价格。关心利省最大
(DD同上,党518页。
和
;308
ne

TT

en

人

Eo

0
化的企业家,和希望在最低的可能成本上生产一个规定水平的产量,这将使
得厂商按照下列方式使用生产要素,即各生产要素的边际物质产品与其价
格的比率相等。如果厂商按照别的方式使用生产要素,那么,它就有可能
在边际量上进行替换,并实现较低的成本。马敬尔没有详述产品用尽问题
和欧拉定理;他认同维克斯蒂德-弗拉克斯的结论,即在长期竞争均衡下,
当每种要素获得其边际产品价值时,总产品被用尽。马软尔对个别生产要
素收益的分析一一工资、地租、利润、利息一一不是特别能引起注意。然
而,他对准租金概念的发展,连同他的要素价格与分配理论却值得注意。
准和有租金
运用其准租金〈quasi-rent)的概念,马吹尔不仅提出了关于市场系统
运行的见解,而且使古典经济学家与边际效用经济学家之间争论的一个方
面更加清楚。上古典经济学家主张,除了土地之外,支付给生产要素的报醒
决定价格。最终产品的价格取决于边际量上的生产成本。因为在边际量上
不存在地租,所以,(为约翰-斯图亚特.穆勒所主导的)古典学说认为工
资、利润以及利息是决定价格的因素,价格因此是由供给方面来决定的。
边际效用经济学家加入了对古典成本学说的早期批评他们声称,支付给
生产要素的报酬是价格被决定的因素。马欣尔的分析表明,无论一种要素
支付是决定价格还是价格被决定,都取决于所考察的时间周期(它极其重
要地影响着要素供给曲线的形状),以及进行分析的特定角度。我们来考察
称作地租、芽资、利润以及利息的报酬。

土地的收益在历史上被称为地租。在分析地租时,李嘉图假定土地的
供给完全没有弹性,并且土地不存在可以兰代的其他有用途。由于使用土地
而支付给地主的报酬,价格被决定而不是决定价格。谷物的高价格是高地
租的原因。尽管对这一理论,有一些来自二流经济学家的批评,然而,从
约翰.斯图亚特,穆勒时代到马向尔时代,基本的李嘉图地租分析保持不
变。马吹尔认识到问题要更加复杂。当从整个经济体的角度来看时,地租
是价格被决定的因素,而不是一种生产成本。然而,如果从个别农场主或
厂商的角度看,地租是一种生产成本,从而是决定价格的因素。希望租用
干地来种植落麦的农场主,必须支付足够的价格,以防止土地用于可替代
nn
2
洲罗全
i
309!
经济上思想哟
FO)aporyofBaonomesSonghy
的用途。除非燕麦种植者愿意支付的地租,高于大麦种植者或者房地产开
发商愿意支付的地租,否则燕麦种植者将不能在竞争性市场上租用到土地。
因此,从个别农场主或厂商的角度看,地租是一种必须支付的生产成本,
就像必须支付劳动与资本成本一样。

马软尔也提出,在某些情况下即使从整个经济体的观点来看,地租也
是决定价格的因素。对于一个拥有不花费任何成本的未开晨土地的经济体
来说,就像19世纪的美国,地租可以被视为决定价格的因素。马菊尔推论
说,原始拓荡者将其土地开晨收益中的一部分,不仅视为种植的直接收益,
而且视为随着人口流向开发区域的边缘地带而发生的土地价格的增值。因
此,这种预期的土地价格增值,是必须支付的必要的供给价格的一部分,
以使个人能够肪受边缘生活的艰苦与危险。上升的土地价格等于上升的地
租的资本化价值,从而被认为是一种社会成本。从经济体的角度看,这些
情况下地租是决定价格的因素。另一方面,从经济体的角度看,在一个全
部土地都已经开星的国家,土地供给曲线完全没有弹性,地租因而是价格
被决定的因素。对于拥有未开晨土地的国家,土地的供给曲线向右上方信
斜;地租越高,越多的土地将得到开垦,地租是决定价格的因素。在给埃
奇光斯的一封信中,马软尔评论说:
不说“地租不进入生产成本”是聪明的,因为那将使很多人困敬。但
是,说“地租的确进入生产成本”是不道德的,原因是,这种说法确实会
以一种和否定微妙真相的方式被加以应用,中
马软尔继续表明在短期中被称作工资、利润以及利明的收益如何其有
地租的一些特征。在长期均衡下,支付给特定劳动类型的工资(例如,一
名会计师),将恰好足以支付这一职业的人,使他们远离其他职业,留在现
在的用处上。这一长期工资是社会为了引出供给的数量而必须支付的供给
价格,工资因此是决定价格的因素。假设对会计师服务的需求增加了,会
计师的工资因此提高。会计师的供给在短期中比在长期中缺乏弹性。工资
的提高不会极大地影响供给的数量,所以,短期工资将上升到长期工资水
(QD参见庇古的《纪念集》436页。
4
:310
0
i
Pd
Wo
第10章同尔弗雷德-马软尔与新三肉经济学
平之上。较高的短期工资与使个人留在职业中所必需的价格没有关系,所
以它是价格被决定的因素,而不是决定价格的因素。理解这些问题的关键
在于供给曲线的弹性。在很短的时期中,某一特定种类劳动的供给曲线可
以被看做是完全无弹性的。由于劳动的供给数量保持不变,需求增加将导
致较高的工资。在短期中,随着在其他职业中工作的经过合格培训的个人
进入本职业,工资将略微下降。在长期中,供给曲线将变得更加有弹性,
工资将下降为长期均衡值即必要的供给价格。因此,在短期和市场周期中,
工资像地租一样,是决定价格的因素。马葡尔将这些支付称作准租金。“因
此,即使地租也不是作为一种独立的东西,而是作为大类中的首要种类来
看待的”2运用其准租金概念,马软尔说明了关于生产要素的报酬是决定
价格还是价格被决定的争论。这完全取决于时间周期:在长期中,工资决
定价格,但是在短期中工资是价格被决定的因素,就像地租一样。

马吹尔也将其准租金的概念应用于短期中对利润的分析。在完全竞争
市场的长期均衡下,每个厂商将只获得正常的利润率。正常利润是生产成
本,必须由厂商来支付以便持有资本,就像必须支付正常工资以吸引和持
有劳动力一样。如果一个厂商在长期中没有获得正常利润,资本将离开该
厂商,投奔能获得正常利润的其他厂商和行业。因此,在长期中,正常利
润是一种必要的生产成本,从而是决定价格的因素。但是,在短期中,称
作利润的收益可以被视为一种准租金,因此是价格被决定的因素。短期中
厂商的成本被划分为可变成本与不变成本。在短期中,厂商的收益必须足
以支付所有可变要素的机会成本,否则它们将离开厂商。剩下的是不变要
素的收益,在短期中不变要素在供给上是完全无弹性的。对于不变要素来
说,利润在短期中是一种准租金,是价格被决定的因素。如果总收益超过
总成本,就获得正常利润之上的利润;但是,在竞争盛行的地方,长期中
它们将会消失。如果总收益超过总可变成本但低于总成本,就会发生亏损;
但是,在长期均衡下这些亏损也会消失。像工资一样,利润可以是决定价
格的因素,也可以是价格被决定的因素,它取决于所考察的时间周期。

准和租金的概念被用于短期中的利息分析。长期中,存在一种正常的利
DD参见蕊吹尔的《原理》第412册。
4

1

!
311'
~L
Bo
:312
息率,它是一种必要的生产成本,从而决定价格。虽然一项旧的资本投资
依据市场供求状况有可能获得高于或低于正常利息率的收益,但是,因为
资本在短期中是不变的,或者说是沉没的,所以其收益就是一种准租金。

在最为宽泛的角度上,准租金的分析能够用来指出古典经济学家与边
际效用经济学家之间的一些本质区别,前者强调供给方面,后者强调需求
方面。如果生产要素的供给不变,那么,任何要素的收入都是一种准租金,
要素价格被决定,要素收益受到需求水平相当大的影响。在长期中,要素
的供给不辕定,最终产品的长期均衡价格必须足以支付生产中发生的全部
社会必需成本。在这些情况下,支付给生产要素的报酬就决定价格,并且,
最终价格的分析必须对供给的作用给予更大的关注。从分析上看,被称作
工资、利润、地租以及利息的收益,在不同的时间周期中有很多共同点。
不可否认地,尽管大自然没有明显地区分时间周期,然而,马葡尔普遍化
的时间周期理论及与之相伴的准租金学说,深刻地洞察了由相对价格的决
室力量所引起的问题。
稳定衡与不稳定均衡
马软尔将需求表看做是个人愿意为一种商品规定的数量所文付的最融
价格,因此,数量是自变量,,价格是因变量。另一方面,将供给表看做是
销售者愿意提供一种商品规定的数量时所能接受的最低价格。同样地,数
量是自变量,价格是因变量。在其《经济学原理》第五篇第三部分第6段
中,马菊尔解释了实现市场均衡的过程。因为他把数量看做是自变量,所
以导致均衡的调整在很大程度上是对数量调整方面的论述。如果在某一既
定数量上,需求价格超过供给价格,“那么,销售者获得的将多于他们在市
场充足时把那样数量的产品带入市场所获得的利益;这时,一种积极的力
量在发生作用,它倾向于增加销售的数量。”2了

图10.3是马向尔的均衡实现过程的图形示意。在数量R,上,需求价格
Rd超过了供给价格Rs,;因此,更大的数量将会被销售者带和市场。在数
量R,上,供给价格R,s,超过了需求价格R,d,,销售者减少带人人市场的数量。
QD参见蕊葡尔的《原理》第345页。
第10章涪尔弗雷德.蕊和欢浴与新古典经济学
当销售者对需求价格与供给价格的相对水平做出反应时,通过数量的受化
就导致了均衡。所实现的这种均衡是一种稳定的均衡(stableequilibrium),
原因在于,离开均衡的任何移动将形成使市场恢复均衡的力量。

瓦尔拉斯与当前的经济理论,在分析市场力量时遵循了一套不同的行
为假设。这些经济学家将价格看做是自变量。对他们来说,需求表显示了
个人愿意在不同价格上购买的数量,供给表表示销售者在不同价格上愿意
提供的数量。

哪一种是正确的:像瓦尔拉斯一样,将价格看做是自变量,还是像马
欣尔一样,将数量看做是自变量?因为这个问题涉及关于购买者与销售者
在市场上的行为方式的假设,所以,只能通过经验研究来决定。但是,这
样两种描述市场行为的方式,它们在分析上的结果能够从理论上推导出来。
马软尔断言并不存在理论上的差异,然而他错了。

这个问题因一个历史异常而更加使人困惑:尽管现代理论在将价格看
做是自变量方面遵循了瓦尔拉斯,然而,在将价格置于供求图形的纵轴方
面又遵循了马风尔。数学上的惯例是把因变量放在纵轴上。写成p=a-bg
的线性需求曲线方程,暗示着价格是因变量,然而,现代理论的行为假设
将价格看做是自变量。
DS’
a___3.
Pp,-Te-1
1AI
i|
失格个1I
|I
silINa,
Pi一一一一一一一~-放一一一+一一一|
1|||
|!|
||p
I||
5|!'
i||
|||
I|
0R,HnR,
数量
10.3”实现均衡
313:
Hp
,314
如果和需求曲线向下倾斜、供给曲线向上倾锋那么,瓦尔拉期写己罗
尔的确会得到相同的结论。再次提到图10.3,我们看到在马葡尔的分析中,
数量变化将导致一个均衡的数量O08。然而,瓦尔拉斯与现代理论利用价格
作为自变量,按照下列方式分析了导致均衡的力量。在价格P,上,需求的
数量是Pd,低于供给的数量P,s,,因此存在超额供给。销售者之间的竞
争将促使价格下降,直至达到一个使市场出清的价格,即在这一价格上供
给的数量等于需求的数量。在低于均衡价格的价格0P,上,存在超额需求;
供给的数量Pis,低于需求的数量Pd;,。因此,购买者之间的竞争促使价格
上升直至市场出清。

由图10.3中的供求曲线所表示的市场,将达到一种稳定的均衡。利用
瓦尔拉斯与现代理论的分析,除了0Pe之外的任何价格,将促使能够让价格
恢复到0Ps的力量开始运转。0Ps是均衡价格,因为如果价格在OP上,,它
就将停在那里。均衡是稳定的,如果菜种东西引起价格偏离0P:,它将恢复
到0Ps。然而,在马软尔的分析中,均衡也是稳定的均衡。数量08是一个
稳定的均衡数量,因为对于除了08之外的任何数量,需求价格或者高于或
者低于供给价格,市场力量将使数量恢复到0H。

当供给曲线向下倾斜时,不稳定均衡(unstableequilibrium〉就可能存
在。在不稳定均衡下,如果价格或数量达到其均衡值,它们将停在那里;
但是,如果系统被扰乱,将恢复不到这些均衡值上。横放着的鸡蛋处于稳
定均衡;如果受到干扰,它将恢复到初始的静止位置。但是,竖放着的鸡
蛋处于不稳定均衡;如果这么放着,它将竖着,但是如果受到干扰,它将
不会恢复到初始的均衡状态。图10.4中的面板(a)代表了运用马积尔分
析的稳定均衡,其中数量为自变量。

在大于0的数量上,供给价格超过需求价格;也就是说销售者接受的
最低供给价格超过了购买者愿意支付的最高价格,销售者将因此减少所所
供的数量直至O08。如果数量小于O08,销售者将扩大所提供的数量,原因
是需求价格超过了供给价格。然而,如果价格是自变量,图10.4中的面板
(a)就代表了一种不稳定均衡。在低于0Pe的价格上,供给的数量超过需
求的数量,销售者之间的竞争将促使价格进一步下降。如果价格高于OP:,
超额需求将促使价格进一步上升。
轩沪本寺计轩十放计村测沪轩二扩|
区轩轩轩所
oe

0

人

op
第10章同尔弗雷德.马敬未与者百典经济学
价格价格
Pl~Pe'
1SsTr________~
|1
|Ny
|S
ID
|
0H数量0FH’数量
图10.4稳定均衡与不稳定均衡
图10.4中的面板(b)说明,数量是日变量时的稳定均衡,在价格厦
自变量时能成为不稳定均衡。面板(a)与面板〈(b〉的比较表明,当供给
曲线向右下方倾斜时,均衡的稳定性将取决于供求曲线的相对斜率,以及
所使用的行为假设。马吹尔与丽尔拉斯的确拥有一个共同因素,它使得他
们两人都断定,在向上代斜的供给曲线下,能够实现稳定的均衡。他们两
人的分析都是一种静态的框架。对于瓦尔拉斯来说,现有时期供给的数量
与需求的数量取决于现有时期的价格;对马软尔来说,现有时期供给的价
格与需求的价格取决于现有时期的数量。因此,马吹尔与瓦尔拉斯两人都
假定了静态行为。
经济波动、货夏以及价格
尽管马软尔最为关注的是微观经济理论,然而,通过人研究抽币力量对
价格总水平的影响,他对宏观经济学也做出了重要贡献。虽然马软尔的一
些最早著作(1871)关注于货币数量理论,但是,在1923年名为《货币、
信用与商业》(Money,Credit,andCommerce)的著作出版之前,他没有发
表关于货币的任何系统性作品。他关于宏观经济主题的观点虽然没有公开,
但是,在他的演讲中和在呈给政府各委员会的证词中得到了充分的发展。
其《经济学原理》前五版的副标题是“卷1”,但在第六版中(1910),他
315
-0
i
i
人
tlyofEcoromcsMoispA
将这一副标题变成“介绍卷”"。1895年《经济学原理》第三版出版时,与
软尔宣布预计再出三卷:工业与贸易的现代条件;信用与就业;政府的经
济职能。他于1919年出版了《工业与贸易》(IndusiryandTrade),但没能
创作其他两卷。马菊尔在宏观经济学方面的创作,主要涉及经济稳定与不
稳定,以及价格总水平的决定力量。

马软尔基本认同约翰斯图亚特:穆勒关于经济体稳定性的观点:永
远不可能存在总需求不足,原因是储琵决策就包括投资决策;不可能存在
普遍的生产过剩。这种推论方式始于亚当斯密,由詹姆士:穆勒、李嘉
图以及萨伐进行了详细阐述;现在,在文献中以萨伊定律而为人所知。马
欣尔时代的经济活动中当然存在波动,一些经济学家,特别是美国的约
翰.A霍布斯倡导消费不足理论。马吹尔认为,对经济波动原因的理解
“不是通过研究消费而获得的,就像一些草率的经济学家所断言的那样”。
马软尔对经济波动原因的解释遵循了约翰*斯图亚特:穆勒的观点,穆蔓
强调经济信心的影响。在高涨期间,经济信心增强,信用快速扩张;在下
降期间,厂商变得翡观,信用快速收缩。穆勒对萨伊定律的认同使他断言
经济萧条不能归因于体制内的任何基本问题。马吹尔提出了两项公共政策
来应对经济萧条与失业。第一项是控制市场,目的是使信用在经济信心高
涨期间不至于过度扩张,因为过度扩张会导致不景气。如果确实发生了经
济萧条,政府可以通过保证厂商没有风险来帮助恢复经济信心。马软尔对
这个办法并不是完全满意,原因是它很难在不产生一些不利后果的情况下
得以实现。例如,保证厂商没有风险,既给有能力的厂商也给没有能力的
厂商上了保险,从而扰乱了奖赏有能力者处罚无能力者的市场过程。

尽管马吹尔对理解经济波动原因的贡献是不充分的,然而,他对价格
总水平决定力量的解释却是突出的。他认识到其微观经济分析是基于下列
假设,即存在充分就业且价格总水平没有重大变化。他对价格总水平决定
因素的分析,是在其供求分析的框架内构建的一种货币数量理论,关于这
部分内容,我们将在第15音中予以考察。
QD同上,党712页,脚注。
第10章同示弗雷德-马软忒与靖厂典经济学
癌.针
从马吹尔开始研究经济学以来已经过去了一个多世纪,人惟省如此,他
在微观经济学方面的贡献,还在为正统的大学本科理论提供基本原理。像
大多数经济学家一样,他的研究建立在过去伟大的理论家所做工作的基础
上。然而,与很多其他伟大的思想家与改革家不同,马软尔不是强调他与
过去经济学家的区别,而是承认借鉴了他们的思想。他把他的工作看做是
对斯密、李玄图以及约翰斯图亚特.穆勒工作的继续,并总是宽宏大量
地六释他们的工作。其作品的特点是谦撑,这是有巨大影响的思想家著作
中少有的一种特质。

马软尔凭借坚实的数学背景,满怀帮助低收入群体的人道主义愿望走
上经济学道路。他认为有可能区分经济学中的规范因素与实证因素,并信
于他认为是一种实证的免于价值判断的科学,这一科学基于下列信念,即
如果我们了解了是什么,社会就能更好地选择应该是什么。他致力于研究
很多方法上的与理论上的问题,其中一些问题在19世纪30年代以来的经济
学文献中予以了讨论。|

古典正统理论没有取得一种统一的方法。亚当.斯密在《国富论》中
综合了理论、历史以及描述方法;他最薄弱的环节是理论。尽管李嘉图没
有特别关注方法,然而,他在没有运用数学的情况下,呈现了一种几乎完
全处于抽象、演绎以及理论模式中的方法。李嘉图的弱项是历史与描述的
方法。约输,斯图亚特穆勒追随着斯密,试图构造一种理论、历史以及
描述方法彼此加强和补充的结构。然而,他们具有很多共同因素。他们都
认为,经济理论普遍正确,能同等地应用于历史上的不同时期和不同市场
结构的社会。他们也普遍假定,从家庭与厂商层面开始,才能最好地实现
对整个经济体的理解。人类的本性与行为先于文化。他们的另一个共同因
素是下面这种极为重要的信仰,即在自由市场中经济促突能够得到和谐解
决。无论自由市场多么不充分,与政府干预经济相比,它们都更加可取。
这一和谐自然秩序中的唯一明显缺陷是地主与工业家之间的冲突。除此之
外,通过没有政府指导的市场,稀缺资源将得到有效配置,市场自由发挥
作用将保证资源得到充分利用。在相对价格决定力量的古典分析中,长其
317
iL2
0
'3168
1
a
|
i
be
ba
中价格一般被假定为取决于成本方面或供给万面。

这些古典思想并不为每个人所接受。在后李嘉图时期,批评古典价值
理论,并进一步提出效用与需求而非成本与供给是决定相对价格的关键因
素的文献得到发展。一些经济学家运用李嘉图的劳动价值理论来表明劳动
受到隙前,并因此对古典体系中经济过程的和谐运转表示怀疑。奥十斯
塔.孔德、赫伯特.斯宾塞所提出的体系,对古典理论的方法基础表示怀
疑,古典理论狭窄地界定了经济学的范围,并且将人类行为视为先于文化
与社会。德国与英国的一些经济学家皇击古典理论的抽象特征,试图阐述
一种更加宽泛的以历史为导向的方法来理解经济。最终,古典理论的基本
结构受到了杰文斯、瓦尔拉斯、门格尔的攻击,他们几乎专门强调需求与
边际效用的作用,并希望以此来蔡代生产成本价值理论。

马软尔经济学是这些方法与理论争议的产物。马吹尔一贯拒绝采取一
种盲从的方法来对待这些问题,因此,他的结论不能令任何一方独断的思
想家满意。他认为,狭窄地界定经济学的范围有其优点,但是他也有理由
希望,社会科学中一种统一的方法被证明是富有成效的。因为每种方法都
有其收益与成本,所以他认为,浪费时间去争论经济学的特有方法是无意
义的。经济学家应当运用适合其训练与性情的方法,不同的方法应当被看
做是相互补充的,而不是相互排斥的。关于价格是由供给单独决定还是由
需求单独决定的争论,同样毫无意义。马软尔指出,价格是大量复杂而相
互作用力量的结果。将价格的决定过程视为下面这种简单的因果关系链条
是错误的,即效用决定需求,然后需求决定价格,或者说,成本决定供给,
然后供给决定价格。无论是效用方面还是成本方面的边际价值都不决定价
格。我们定位于边际量,是为了考察正在运转的力量,并提高我们对它们
的认识。但是,当我们转到边际量上时会发现,效用、成本以及价格相互
决定彼此的价值,并不存在简单的因果链条。边际量、局部均衡、其他条
件保持不变、时间周期、代表性厂商以及生产要素,全部都是抽象的理论
构件,它们帮助我们分解分析中的复杂问题。然而,这种理论进步的取得
是以现实的缺失为代价的,因此,经济学家必须用描述性的和历史的材料
来补充纯理论。

尽管马软尔对他所处时代很多方法与理论问题试图采取不明朗的态度,
然而,他通常偏向于古典理论中的某些成分。他比杰文斯、门格尔以及瓦
第10章阿水弗雷人德.怠软水与圭古典经济学
尔拉斯都更加宽泛地界定经济学的范围,并且更加偏好斯密和约划.斯图
亚特:穆勒的方法。他声称,尽管在长期中价格取决于一套复杂的力量,
但是,古典经济学家强调成本与供给的重要性基本上是正确的。机会成本
的概念提供了在短期中当供给相对不变时,进行资源配置的一些见解,但
是在长期中,通过考虑实际生产成本、劳动的努力以及资本家的等待或者
节欲,能够获得关于定价过程更加基本的见解。马吹尔从未能彻底摆脱边
沁的享乐主义心理,虽然他充分了解它所招致的批评。

现在的局部均衡微观经济理论的基本框架来自于马软尔的《经济学原
理)》。尽管从那时起,存在着很多对微观经济理论的重要贡献,然而,大部
分是对方法而不是对实质性分析的补充。一个主要的例外是由琼.罗宾远
和爱德华.H.张伯伦在20世纪30年代发起的对市场结构理论的贡献,他
们的很多观点是由马软尔提出的。马软尔体系的一个重大缺点是,他未能
考察收入与就业水平的决定力量。但是,当约翰梅纳德.凯恩斯于20世
纪30年代承担这一任务时,又是在应用于总变量的马吹尔供求分析框架内
予以阐明的。
