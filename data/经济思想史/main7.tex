\chapter{阿尔弗雷德·马歇尔与新古典经济学}

阿尔弗雷德·马歇尔被认为是新古典微观经济学之父这一头衔的两个竞争者之一(另一个是莱
昂·瓦尔拉斯)。以斯密、李嘉图以及约翰·斯图亚特·穆勒的工作为基础,马歇尔发展了一种
分析框架,该分析框架今天依旧作为通用的大学本科经济理论以及大多数经济政策的结构性
基础。对马歇尔的观点进行真正彻底的考察,将几乎包括今天所有的局部均衡微观经济理论。

\section{马歇尔作为新古典主义之父的宣言}

马歇尔带着大学本科数学训练的经历,以及改善穷人生活质量的强烈人道主义情感走进了经
济学的世界。……到了19世纪60年代后期,他对经济学产生了强烈的兴趣,以至于决定做一
个学者型教师而不是牧师。他开始在剑桥讲授经济学;在两位早期数理经济学家古诺与
冯·杜能著作的影响下,他开始将李嘉图和约翰·斯图亚特·穆勒的经济学转化为数学。

1871年,杰文斯与门格尔对古典理论几乎专门强调供给进行了择击。由古典理论而来的政策
也受到了围攻。例如,日益增多的英国工厂工人的贫困生活和工作状况,与自由放任的主张
不相适应。因此,出现阿尔弗雷德·马歇尔这样一个拥有巨大学识与智慧的人的时机成熟了,
从1867年至1890年,马歇尔仔细地铸就着供求分析的原理。

杰文斯草率地出版了其著作,宣称摧毁了古典价值理论,彻底改革了经济理论。……正如凯
恩斯恰当描述的那样:“杰文斯看到壶中的水沸腾了,像个孩子似地欢呼起来,马歇尔也看
到壶水沸腾了,却悄悄坐下来,造了一台发动机。”马歇尔建造的分析发动机既反映了他的
个性,也反映了养育他的环境。他早年的宗教信仰,后来显示为成熟的人道主义,引发了他
对穷人的深切关心,也使他乐观地确信经济研究能够提供改善整个社会福利的方法。他的学
识使他熟悉那些历史导向的经济学家们的倾向,这些经济学家反对如下观点,即经济理论是
一种适用于所有时间和地点的绝对真理的主体。1885年在当选剑桥教授的一次就职演讲中,
马歇尔谈到了这一批评:“就经济学说能够独自宣称具有普遍性这一点来说,并没有什么教
条。它不是具体真理的主体,而是发现具体真理的发动机。”

马歇尔试图将他早期的数学训练与他的历史学背景结合起来,建造一台适用于变化时代
的“发动机”。然而,由于意识到约翰·斯图亚特·穆勒1848年得出的“价值理论是完全
的”这一结论的草率性,马歇尔预见到,随着新理论的不断出现以适应不断变化的社会,他
自己对经济学的贡献将变得陈旧。马歇尔,这位前所未有的集理论家、人道主义者、数学家
以及历史学家于一身的人,试图为他所处时代对方法论的争论指明道路;同时,用边际主义者
的新工具调和古典分析的最佳之处,以此来解释价格的决定力量与资源的配置。

尽管在经济理论的发展中马歇尔是一位杰出的人物,然而,他对理论上和方法上的问题拒绝
采取严格的立场,这给随后的几代经济学家造成了很多麻烦。为了得出折衷的看法,他有时显
得含糊和优柔寡断。他似乎经常在说那得看情况,李嘉图是正确的但也是错误的;抽象的理
论是有益的也是有害的;历史的方法是有帮助的,但理论也是必需的;从一种观点来看,支
付给生产要素的报酬是决定价格的因素,但从另一种观点来看,它又是被价格决定的。一
些读者将这种关于理论与方法问题的机动性视为真正智慧的标志,但是,一些读者尤其是比较
抽象化的数理经济学家,恼怒于他们所认为的马歇尔经济学中优柔寡断的部分。不过,他的
风格引发了大量著作,这些著作都试图揭开马歇尔的“真正用意”是什么。

尽管一百多年以前,马歇尔为经济思想做出了他的贡献,但是,他仍然吸引荐很多经济思想
史学家们的注意力。我们特别建议熟读格罗尼维根1995年出版的关于马歇尔的优秀传记。

\subsection{经济学的范围}

马歇尔的《经济学原理》第一篇第1章以一个宽泛而灵活的对经济学的定义开始:“政治经济
学或经济学是对人类日常生活的研究;它考察个人和社会行为中与获得和使用物质财富密切
相关的部分。”

这一定义中引起注意并有些讽刺意义的方面是,它使用了两种不同的术语——政治经济学和经
济学。人们因此认为他将使用的是更加宽泛的政治经济学术语,这反映出他所处时代的一些
方法问题。政治经济学这一术语在当时比经济学更加普遍,它暗示了经济学与政治学是有关
系的,并且,经济学作为社会科学中的一门学科,与规范性的判断密切相连。然而,马歇尔
的一位同事和朋友约翰·内维尔·凯恩斯(约翰·梅纳德.凯恩斯之父)尤其对方法问题感兴趣,
他在1891年出版了一本名为《政治经济学的范围与方法》的书,其中清楚地区分了经济学的
三个分支:实证经济学,它包括经济学的科学分支;规范经济学,它考虑社会的目标应当是
什么;以及经济学艺术,它使实证科学分支的见解与规范分支所决定的目标相联系。约翰·内
维尔·凯恩斯声称,在实证分支的讨论中经济学与经济科学的术语要更好于政治经济学,原因
在于,这些名称强调了经济学的科学特征。与李嘉图和穆勒不同,马歇尔选择将他的著作称
为《经济学原理》,而不是《政治经济学原理》,并且最终停止使用政治经济学这一术语,
而赞同经济学这一术语。这里具有讽刺意味的是,他比几乎任何一个同时代的人都更多地实
践了经济学艺术,而不是经济学科学。他聚焦于实用的理论,对纯粹的经济学科学并不感兴
趣。这种变化有两种可能的理由。第一种可能是马歇尔希望他的方法与政治经济学的方法有
所不同。第二种可能是马歇尔试图在他所任教的剑桥,为经济学赢得独立研究领域的认同,
而政治经济学这一术语暗示着不同领域之间的交叠,这不利于他的目的。

马歇尔的松散定义并不是缘于疏忽和不集中的思想,而是来自于一种有意识的不情愿,即不情
愿明显地把经济学从其他社会科学中分离开来。他指出,大自然也画不出这么清晰的分界线,
经济学家将学科的范围界定得太罕,这将做不成任何事情。在题为“经济学的范围与方
法”的附录C中,马歇尔考虑了(以其特有的折衷方式)与允许每种学科独立发展相对照,发展
一种统一的社会科学的优缺点和可行性。使社会科学成为一体的观点吸引着他,但是,他也知
晓伟大的孔德与赫伯特·斯宾塞(Herbert Spencer,1820一1903)两人的努力都未能实现这一
目标。另一方面,他观察到借助于专业化,自然科学已经取得了巨大进步。他最后断定,缺
乏一些具体问题时这个事情不能得到解决:
\begin{quotation}
  与任何其他社会科学相比,经济学都取得了更大的进步,原因在于,它比任何其他社会科学
  更清楚而精确。然而,其范围的每一次扩大,都会使这种科学的精确性有所损失;损失是否大
  于因范围扩大而带来的收益,这个问题是不能呆板决定的。
\end{quotation}

马歇尔指出,每个经济学家都应当界定经济学的范围,以符合他或她自己的意愿,因为一些
经济学家在比较窄的经济学范围内,更有可能全力以赴地工作,另一些则在比较宽泛的框架
内工作。他警告说,那些选择宽泛的经济学定义,并向社会科学的其他领域扩展其分析的人,
行为必须极其谨慎,但如果他们仔细工作,他们就为经济学及其他社会科学提供了巨大的帮
助。

马歇尔在他对经济学范围的讨论中,提出了另外一个引人注意的问题,即社会的需要与社会
经济活动之间关系的复杂性。能把经济学描述为研究经济活动满足社会需要的方式的学科吗?
马歇尔否定了这个定义,原因在于,它暗示着需要是独立的既定量,相对于需要而言,经济活
动是第二位的。在《经济学原理》第三篇第2章对需要与活动之间关系的论述中,马歇尔试图
纠正他认为不正确的结论,这一不正确结论是由杰文斯和门格尔以及他们的前辈得出的,他
们似乎视“消费理论为经济学的科学基础"。他在最宽泛的可能的背景下评价需求(需要)与供
给(活动)的相对重要性。他的观点是,我们的需要并不发生在我们内部并独立于我们的活动,
相反,我们的很多需要是我们活动的直接副产品。把这一思想应用于21世纪,就是说将一
个“雅皮士”家庭购买一辆小型货车的愿望,视为经济分析的起点是错误的,原因是这一需
要可能起因于家庭对其社会角色的感知。马歇尔提出,经济学家应当从对需求的初步研究开始,
继续到活动与供给,再返回到需求。他主张这将使他们能够理解需要与活动之间复杂的相互
联系。在经济分析中,如果被迫在需要的最高地位与活动的最高地位之间做出选择的话,马
歇尔将会选择活动,这反映出他对强调供给的古典经济学的亲和力,并使他与杰文斯与门格
尔形成对照,他们强调需求:
\begin{quotation}
  所以,“消费理论是经济学的科学基础这句话是不对的。因为在研究欲望的学问中,很多让
  人最感兴趣的东西都来自于研究努力与活动的学问。这两者相互补充,缺一就不完全。但
  是,如果两者之中有可以称为人类历史——不管是经济方面还是任何其他方面的——解释者,
  它就是研究活动的学问,而不是研究欲望的学问。(马歇尔《经济学原理》朱志泰译,
  p125-126)
\end{quotation}

马歇尔基于宗教的人道主义关心,使他将消除贫困作为经济学的首要任务。他认为,解决这些
问题的关键存在于事实中,以及经济学家的理论中,他最大的美梦是他正在建造的分析发动
机能够揭示出贫穷的原因,并最终识别出如何加以补救。在他对经济理论史进行评论的附
录B中,他对古典理论家,尤其是李嘉图没有认识到贫穷滋生贫穷进行了严厉的批评。贫穷之
所以滋生贫穷,是因为穷人没有足够的收入获得能使他们挣更多钱的健康与培训。与古典理
论家相对照,马歇尔一心一意地相信极大提高劳动阶层福利的可能性。

他对经济学范围的讨论,首先显示出他希望对以历史为导向的经济学家的批评进行回应,这
些经济学家要求比较宽泛的经济学定义;其次显示出他希望讨论如下问题,即经济学应当做
为一门狭窄的抽象学科来发展,还是应当发展成一门统一的社会科学;再次显示出他希望回
答边际效用经济学家的问题,他们坚持认为消费理论应当优先于成本与供给理论;最后显示
出他希望对古典经济学中曾经困扰穆勒的部分提出异议,因为它为消除贫困提供了很小的希
望。通常地,马歇尔试图就这些问题提出一种折衷的看法,而很少采取清晰的态度。

\subsection{马歇尔论方法}

他认识到古典经济学,尤其是李嘉图经济学的主要缺陷是未能认识到社会变革。但是他看到,
抽象理论与历史分析的结合能够纠正这一缺陷,在附录B中,他赞扬亚当斯密是方法的模范。
在附录C“经济学的范围与方法”与附录D“抽象推理在经济学中的应用”中,他给予历史方
法以及德国历史学派高度的赞扬。马歇尔自己的方法是试图将理论的、数学的以及历史的方
法相混合。他承认一些经济学家更喜欢依赖某种单一的方法,他并不反对这一点。对于马歇
尔来说,不同方法的运用并不能表明冲突或者对立,原因在于,所有的经济学家都从事着一
项共同的任务。每种方法都从其角度将经济体的运转描述得更加清楚,并因此增强我们的理解
能力。

马歇尔试图调和他所处时代对方法的争论,这使他容易受到来自各方面的攻击。德国与英国
历史导向的经济学家认为,马歇尔的经济学方法太抽象并具有刚性。20世纪美国经济学家托
尔斯坦·凡勃仑以及追随他的所谓的制度学派,主导了对马歇尔方法的强烈反击。抽象数学方
法的提倡者,恼怒于他对历史方法的赞扬,以及他关于理论与数学局限性的直率评
论。1906年,马歇尔在写给专心于在经济研究中使用数学和统计学方法的朋
友A.L·鲍雷(Arthur L.Bowley)的一封信中,做出了切中抽象数学方法核心部分的评论:
\begin{quotation}
  我还没能找到任何对你有所用处的关于“数学--经济学”的注释,而且我对我过去在这个
  问题上的想法只有非常模糊的记忆。现在我已不阅读数学,实际上,我已经忘记如何去将很
  多东西结合为一个整体。

  在最近几年的研究工作中,我越来越感觉到,用优秀的数学定理来处理经济假说,不太可
  能作出优秀的经济学:我越来越多地依靠这些规则——(1)将数学作为一种速记语言来使用,
  而不是作为分析的发动机来使用;(2)使用这个办法一直到把想法记下为止;(3)把它
  们翻译成英文;(4)举例说明为什么这些想法在真实生活里是重要的;(5)烧掉数学;(6)如果你
  做不到(4),就烧掉(3)。我经常遵循最后这一条。
\end{quotation}

马歇尔的《经济学原理》包括第(3)步与第(4)步,并且是以一种不打算供他的经济学家
同事使用,而是打算供受过教育的读者使用的风格创作的。他所使用的数学,要么被放在脚
注中,要么被放在数学附录中。尽管马歇尔竭尽全力避免使用经济学行话,并用来自最近的
或历史上的经济经验中的实例来说明每一条原理,然而在其背后,唯一的东西就是强大、紧
密、高度抽象的理论结构。

正如马歇尔没有提出关于经济学的整齐而精简的定义一样,他通常也避免给出很多经济概念
的准确定义。古典经济学赋予了土地、劳动、资本这些所谓的生产要素比其适当的含义更加
准确的含义。在经济体中,土地、劳动以及资本经常是混合在一起的,以至于只有高度的抽
象才能将它们分离。马歇尔因此指出:“我们——把制造一件商品所需要的东西编排为任何方
便的组别,并把它们称作商品的生产要素。”没有规定硬性而不变的定义:手边的问题指示
着如何界定要素。类似地,在分析供给时,马歇尔不得不着手解决成本问题。如果供给取决
于厂商的正常成本,那么,哪个厂商将被选择为正常厂商?这里,马歇尔又一次表现出了他
的灵活性,他陈述道:“为了这个目的,我们将不得不研究在那个总生产量之下一个代表性
生产者(expenses of a representative producer)的费用。(同上,380页)”他的平均厂商
或代表性厂商的概念,并不是诸如算术平均、众数、中值一类的统计量。相反,他提出应当
通盘考虑一个行业来找出下列厂商的准确位置,即由正常或平均才能的人管理的厂商,既不
是行业中的新来者,也不是原有的已经建立的厂商,而是其成本扬示出它们能够正常地使用
可利用技术的厂商。”

马歇尔表面上的含糊、易变,以及偶然缺乏理论上的冰确,并不是由他混乱的头脑产生的,
认识到这一点很重要。他采取一种审慎的方法论立场。马歇尔对微观经济理论的理解以及他
的数学才能,本来能够使他以一种更加简练的形式来呈现其大约七百页的《经济学原理》。
事实上,他在其数学附录中做到了这一点。但是,经济体实际上远比数理经济学所能显示的要
复杂得多。在其生涯的早期,马歇尔就设计出了关于市场经济体的纯理论;到了大约1870年
它已经相当全面。《经济学原理》数学注释21是一般均衡模型的一页纸版本,它说明了最终
产品需求、最终产品供给、生产要素需求以及生产要素供给之间的关系。1908年,马歇尔写
信给约翰·贝茨·克拉克说:“我的终身都已经并将继续献给以现实的形式,尽我所能更多呈
现我的注释21。”在其《经济学原理》中,马欣尔明确地为缺乏精确进行辩护。在简要、清
楚说明了经济体长期均衡条件之后,马歇尔继续指出

\begin{quotation}
  但是,在我们生活的世界里,这一切都是不真实的。在现实世界中,每种经济力量都在围
  绕着它起作用的其他经济力量的影响下,不断改变着自己的作用。在这里,生产数量、生产方法
  和生产成本的变动始终是相互制约的;它们总是影响着需求的性质和承担,并且也受到
  后者的影响。此外,所有这些相互影响都需要时间来表现自己,而一般说来,没有两种影响是
  齐头并进的。因此,在现实世界中,任何一种关于生产成本、需求以及价值之间关系的简易
  学说,都必然是错误的;越是通过巧妙叙述而使它的外观越易懂,则它就越有害。如果
  一个人相信自己的判断力和实际直觉,那么,他与那些自以为研究了价值理论并断然认为它
  很容易的人相比,就更有可能成为一个好的经济学家。(《经济学原理》下卷,朱志泰译,
  66页,有修改。)
\end{quotation}

\subsection{理解复杂性:马歇尔方法的实施}

有两点原因使马歇尔将对经济体的研究视为复杂而困难的。一方面,每件事情似乎都取决于
其他的事情:经济体的所有组成部分之间,存在一种复杂的通常又敏感的关系。另一方
面,“时间因素是经济研究中所遇到的那些困难的一个主要原因,而这些困难使能力有限的人
循序渐进就成为必要。(同上,64页)”原因并不会瞬间带来最终结果;随着时间的变化,
它们才显现出来。但是,当一种原因,例如,需求增加的影响被感觉到时,经济体中的其它
变量可以独立地改变(例如,供给可能增加),因此,经常难以孤立某一单个原因并确定其影
响。如果自然科学的实验方法(借此有可能使得除一种影响之外的所有影响保持不变,然后观
察重复实验的结果),能够为经济学家所使用,那么,这个问题将不会存在。但是,因为实验
方法对经济学家不可行,所以,必须使用一种可替代的选择。当马歇尔小心地发展其基本思想
体系时,他提供了这种可替代的选择。

按照这一体系,因为经济学家不能使所有的变量保持不变,这些变量可能影响某种既定原因
下的结果,所以,他们必须通过假设,在理论层面上做到这一点。为了在分析经济体中复杂内
部关系时取得一些进展,我们假设某种因素发生变化时,“其他条件保持不变(ceteris
paribus)”。在任何分析的开始,很多因素都保持不变;然而,随着分析的进行,更多的因素
被允许变化,目的是达到较强的现实性。其他条件保持不变的方法使处理复杂问题变得可行,
代价是缺少一定的现实性。

马歇尔首次也是最重要的一次对其他条件保持不变方法的运用,是发展局部均
衡(partialequilibrium)分析形式。为了分解一个复杂的问题,我们将要分析的经济体的一个
部分孤立起来,忽视但不否定经济体所有部分的相互依赖。……暂时忽视所分析行业的产品
与其他行业的产品之间复杂的替代与互补关系。局部均衡方法的一项重要用处是,就既定原
因的可能结果来说,取得一个最为接近的结果。因此,这对于处理政策问题尤其有益——例如,
预测对进口手表征收关税的影响。在局部均衡分析方法中,能够运用简单的供求分析来预测
这样一种政策的直接含义。马歇尔的步骤是首先在局部均衡的框架中,将一个问题限制得非常
狭窄,使大多数变量保持不变,然后通过允许其他条件变化,缓慢而仔细地扩展分析的范围。
他的方法被适当地称为一次一事法。

\subsection{时间问题}

经济分析中的一个主要困难是,原因要花费时间才能产生出结果。正确解释了既定原因短期
结果的任何分析或结论,就长期来说,都将可能是不正确的。马歇尔对其他条件保持不变方
法的运用,与他处理时间的方法是相应的。在有时称作即期或很短时期的市场周期中,很多因
素保持不变。随着时间周期延伸到短期、长期以及也被称为特别长的时期的长周期,越来越
多的常数被允许变化。时间段稍微影响需求分析,就会给供给的分析造成非常多的混乱。

马歇尔定义了四个时间周期,不同的时间周期是根据厂商经济学与供给经济学来界定的。市
场周期如此之短,以至于供给是不变的,或者说完全没有弹性。不存在价格对供给数量的反作
用行为,因为对厂商来说周期太短了,使之不能够对价格的变化做出反应。在短
期(shortrun〉中,厂商能够改变生产与供给,但不能改变设备规模。在此存在一种反作用行
为,因为较高的价格引起较大的供给数量,供给曲线向上倾斜。在短期中,厂商的总成本可以
划分为两个部分:随着产量变化的成本,马歇尔称之为特别成本、直接成本或者最初成
本(prime costs),现代教科书中称之为可变成本;以及不随产量变化的成本,马歇尔称之为
补充成本,现代教科书经常称为不变成本。短期中可变成本与不变成本的区别显然是从马歇
尔对经济世界的观察中得出的。在分析厂商行为时,它成为一种重要的分析工具。在长
期(long run)中,设备规模能够改变,全部成本都成了可变的。与短期相比,供给曲线更加富
有弹性,因为厂商能够通过改变设备规模等进行充分的调整以适应变化的价格。一个行业的
长期供给曲线能采取三种一般的形式:它能够向右上方倾斜(成本可能增加);它能够是完
全富有弹性的(成本可能不变);或者在非常状态下,它能够向右下方倾斜(成本可能下
降)。长周期(secular period)或者说特别长的时期,允许技术与人口发生变化,马歇尔分
析从一代人到另一代人的价格运动时,用到了这种形式。

显然,马歇尔的时间周期不是用日期来度量,而是指厂商与行业的供给条件。一个设备规模
只能缓慢更新并资本非常密集的行业,例如钢铁行业,在序时时间上,其短期可能与另一个
设备规模能很快更新的行业的长期一样长。尽管马歇尔为微观经济理论几乎每个部分都增添
了内容,然而,他的主要关注点以及他最伟大贡献的来源,是他的时间对供给影响的分析。
他发现,价格分析中的主要困难在于确定时间的影响,在晚年他声称这一领域需要做更多的工
作。

\subsection{马歇尔的迷惘}

在19世纪最后二十五年时间里,关注和需求与供给在价格理论或者价值理论中相对重要性的
经济学家们,发生了一场争论。就像穆勒的《原理》所提出的那样,古典经济学强调供给;
然而杰文斯、门格尔、瓦尔拉斯强调需求,杰文斯与一些人走得更远,他们断言价值完全取
决于需求。马歇尔烦恼于对他供求分析的批评,这些批评指出他在试图调和古典观点与边际
效用学派。他声称,他在寻找真理,而不仅仅是和和平,此外,在杰文斯、门格尔、瓦尔拉
斯开始就供求分析进行创作之前,他就已经明确地表达了他的供求分析。

马黄尔认为,对时间影响的正确理解,以及对经济变量之间相互依赖的意识,将会解决是生
产成本还是效用决定了价格这一争论。最终产品的需求曲线向右下方倾斜,因为个人将在较低
的价格上购买更大量的产品。供给曲线的形状取决于所分析的时间周期。时期越短,需求在决
定价格中的作用就越重要;时期越长,供给在决定价格中的作用就越重要。在长期中,如果
存在不变成本,并且供给完全富有弹性,那么,价格将单独地取决于生产成本。然而,一般来说,
争论需求还是供给决定了价格是徒劳的。马歇尔运用下面的类推来表明,因果关系并不是一
件简单的事情,任何试图找到单一原因的尝试注定要失败;

\begin{quotation}
  我们争论价值是由效用所决定的还是由生产成本所决定,这在相当程度上像在争论一张纸是
  由剪刀的上边裁出还是由剪刀的下边裁出。的确,当剪刀的一边拿着不动时,移动另一边
  才能把纸剪下来,我们大致可以说,纸是由剪刀的第二边剪下来的。但是,这一表述并不
  十分准确,只有当它当作对现象的一种通俗的解释,而不是科学的解释时,才可以这样
  说。(同上,45页)
\end{quotation}

马歇尔说,边际分析经济学家似乎认为正是边际价值(无论成本、效用还是生产力)以某种方
式决定了全部价值。例如,按照马歇尔的观点,在分析最终产品的价格时,说边际效用或边
际成本决定价格就是错误的。边际分析只是提出“我们必须借助边际量来研究决定全体的价
值的那些力量”(同上,112页)。边际效用或边际成本并不决定价格,因为它们的价值连同
价格一起,与在边际量上使用的要素相互决定。这里,马歇尔再一次提出了一个非常适当的
类比来表明他的观点。杰文斯孤立了价格决定中的本质因素:效用、成本、价格。但是,他
错在试图寻找单一原因,并将过程视为一连串的因果关系,即生产成本决定供给,供给决定
边际效用,边际效用决定价格。马歇尔认为这是错误的,原因是它忽视了这些因素之间的相
互关系和相互的因果联系。如果我们把三个球放进一个碗中,第一个是边际效用,第二个是生
产成本,第三个是价格,那么,说任何一个球的位置决定了其他两个球的位置显然是错误的。
但是,球相互决定彼此的位置是正确的。因此,需求、供给以及价格在边际量上彼此相互作
用,并相互决定它们各自的价值。

在《经济学原理》附录1和第五篇最后一段中,马歇尔试图将他的价格理论,置于李嘉图价值
理论以及是效用还是生产成本决定了价格的争论背景中。马歇尔认为,他自己的价格理论从
根本上说与李嘉图有关。尽管边际效用经济学家几乎都不赞同这点,马歇尔仍然指出,李嘉
图认识到需求的作用,但仅给予了有限的注意,因为它的影响易于明白;李嘉图反倒是把他的
精力放在更加困难的成本分析上。马歇尔认为,李嘉图的生产成本价值理论既包括劳动成本,
也包括资本成本。大多数经济理论史家认为,这是对李嘉图过度宽宥的阐释。按照马歇尔的
观点,李嘉图价值理论的主要缺陷是,他没有能力处理时间的影响,加上没有能力清楚地表
达他的观点,从而使其缺陷更加严重。杰文斯和其他边际效用经济学家声称他们有效地粉碎
了李嘉图的价值理论,并用一种几乎专门强调需求的正确的版本加以奉代,马歇尔否定了他
们。马歇尔把他自己的贡献仅仅视为对李嘉图思想的扩充与发展,并认为这一贡献使得李嘉
图价值理论的根本基础完整无缺。在我们考察马歇尔的其它观点之前,我们将推迟对其价值
理论的评价。

\subsection{马歇尔论需求}

马歇尔指出,需求对价格决定的影响相对来说容易分析,他的这一主张可能是完全正确的。
然而,马歇尔未能令人满意地解决需求理论中存在的一些问题。他似乎意识到这些困难并通
过假设回避了它们。他对需求理论最重要的贡献是,他清楚地阐明了需求的价格弹性概念。
价格与需求的数量彼此反方向地发生关系,需求曲线向右下方倾斜。价格变化与需求量变化
之间关系的程度,通过价格弹性系数揭示出来。价格弹性系数为:
\[e_D = - \frac{需求数量变化的百分比}{价格变化的百分比} = - \frac{\Delta
    q/q}{\Delta p /p}\]

因为价格与需求的数量反方向地发生关系,所以,计算出来的需求的价格弹性系数为负数。
按照惯例,用一个正数来表示系数,所以,添加一个负号到等式的右边。产品的价格乘以需
求的数量将等于购买者的总支出,或者也可以说是销售者的总收益($p \times q = TE
=TR$)。如果价格下降1\%,需求的数量增加1\%,那么,总支出或总收益将保持不变,系数
将是一个等于1的值。如果价格下降且总支出或总收益增加,那么,系数将为大于1的值,商品
就被说成是价格有弹性(price elastic),的。如果价格下降了一个既定的百分数,需求的
数量增加了一个较小的百分数,那么,总支出或总收益将下降,系数将为小于1的值,商品就
被说成是价格缺乏弹性(price inelastic)的。马歇尔也将弹性的概念用于供给方面,借此给
予经济学另一种极其有用的工具。尽管在较早的文献中就已经提出了价格弹性的概念,然而,
正是马歇尔运用其数学能力,能够精确地将它表示出来;因此,他被认为是价格弹性的发现
者。

按照马歇尔的观点,个人渴望商品是因为通过商品消费能够获得效用。马歇尔所使用的效用
函数的形式是加法函数;也就是说,他通过将消费每种产品获得的效用相加得到总效用。
消费产品A获得的效用,只取决于所消费的A的数量,而不取决于所消费的其他产品的数量。
因此,替代与互补关系就被忽视了。加法效用函数被规定为:
\[U = f_1q_A + f_2q_B + f_3q_C + \cdots + f_nq_N\]

当时,实践中所使用的效用函数明确地认识到了互补与替代关系,被表示为:
\[U = f(q_A, q_B, q_C, \cdots, q_N)\]

弗朗西斯·Y·埃奇沃斯与欧文·费雪是两个与马歇尔同时代的人,他们提出了现在所使用的更
为一般化的效用函数。马歇尔运用加法效用函数,最重要的含义涉及收入效应,对此我们将
简要地加以论述。

马歇尔假定通过价格系统可以度量效用。如果一个人花了2美元购买另一单位的产品A,花
了1美元购买另一单位的产品B,那么,A提供的效用一定是B所提供的两倍。他也认为效用的
群体间比较是可能的,原因是在群体比较中个人的特性被冲刷掉了。

在马歇尔的框絮中,需求理论最重要的任务是解释需求曲线的形状。如果一件商品的边际效
用随着更多的商品被消费而下降,那么,能够因此断定个人将为更大的数量支付较低的价格,
从而断定需求曲线是负斜率吗?马歇尔认同边际效用递减(戈森第一定律),并且阐明了为消
费很多商品的个人带来最大效用的均衡条件(戈森第二定律):
\begin{equation}
  \label{eq:1}
  \frac{MU_A}{P_A} = \frac{MU_B}{P_B} = \frac{MU_N}{P_N} =MU_M
\end{equation}

在均衡状态下,消费者将按照下列方式支出,即花在任何一种最终产品上的最后1美元,与
花在任何其他产品上的最后1美元具有相同的边际效用。这些边际效用对价格的比率,将等于
货币的边际效用,这也因此揭示了货币的边际效用。货币的边际效用是从最后1美元支出中获
得的边际效用。如果储蓄被认为是一种产品,那么,货币的边际效用就是从最后1美元收入中
获得的效用。单个产品的边际效用等于它的价格乘以货币的边际效用:
\begin{equation}
  \label{eq:2}
  MU_A = P_A \cdot MU_M
\end{equation}

如果我们从效用最大化的个人开始,随后降低一种产品的价格,那么,我们就能得出价格与
需求数量之间的关系。利用\cref{eq:1}与\cref{eq:2},我们看到,降低产品A的价
格 $P_A$,在某些条件下,将引起需求数量的增加。降低产品A的价格将具有两种效应。替代
效应反映出如下事实,即现在产品A比它的替代品相对便宜了,因此,个人对产品A的消费将增
加。替代效应将总是导致在较低的价格上更多的消费,在较高的价格上较少的消费。价格变
化所产生的收入效应比较复杂。降低产品A的价格提高了个人的实际收入。以较低的价格,个
人能够购买到与以前一样数量的产品A,并有剩余收入能够花在产品A或其他产品上。例如,
如果产品A的价格为1美元,以前购买10个单位,那么,将产品A的价格降为0.9美元,将增
加1美元的实际收入。正常产品(normal good)是指其消费随着收入的增加而增加。如果产
品A是一种正常产品,其需求曲线将向右下方倾斜。借助替代效应与收入效应,降低其价格将
增加需求的数量。

如果产品A是一种低档产品(inferior good),就会产生其它复杂的因素。低档产品的消费随
着收入的增加而减少。在消费者的预算中,碎牛肉可能是一种比较适当的低档产品。随着收
入增加,碎牛肉的消费量将会下降,因为更好的牛肉块替代了碎牛肉。如果产品A是低档产品,
那么,由于替代效应,其价格下降将会导致消费增加,但是,收入效应又会导致消费减少。
如果替代效应大于收入效应,需求曲线将为负斜率;但是,如果收入效应大于替代效应,
需求曲线将为正斜率。向上倾斜的需求曲线的可能性,极其扰乱需求理论。理论上的可能性
存在,但是还没有经验信息表明向上倾斜的需求曲线实际存在。

马歇尔首先陈述了需求的一般法则:“随着价格下降,需求量增加;随着价格上升,需求量
下降。”然后他注释说,罗伯特·吉芬(Robert Giffen)所掌握的资料指出,比较贫穷的人对面
包的需求曲线可能向右上方倾斜。换句话说,对于这些人来说,面包价格上升引起肉的消费
和比较昂贵的食物的消费减少,面包的消费增加。因此,在理论文献中,收入效应比替代效
应更强的低档产品,被称为吉芬产品(Giffen goods)。此外,关于所谓的吉芬悖论,尽管存在
相当多的理论文献,然而,没有可接受的统计资料表明产生过向上倾斜的需求曲线。

我们回到推导需求曲线的理论问题上,看马歇尔是如何处理这些问题的。因为他使用加法效
用函数进行研究,所以,在推导需求曲线的正式数学处理中,他忽视了替代与互补关系——尽
管他的确从特征上论述了这些问题。他简单地假设价格微小变动的收入效应可忽略不计;也
就是说,对于任何单个商品价格的微小变动,货币的边际效用保持不变。这样的话,如果我
们降低\cref{eq:1}中产品A的价格,产品A的需求量增加,边际效用下降,直至 $MU_A /
P_A$的比率变得与其他商品的比率相等,并且所有商品的比率再次等于不变的货币的边际效
用。还可以从另一个角度研究马歇尔的步骤。利用等式 \cref{eq:2},由于边际效用递减原
理,产品A价格的下降(假定货币的边际效用不变),必定导致其消费的增加。

马歇尔通过假定货币的边际效用不变来消除这些理论上的难点有两点:第一,他没有理论工
具来清楚地区分替代效应与收入效应;第二,他宣称一种产品价格微小变动的收入效应是如
此之小,以至于将其忽略不计,没有什么不利影响。

\subsection{消费者剩余}

马歇尔认为,对于价格的微小变动,货币的边际效用不变,他的这一观点使他(或者因此他认
为)得出了现在为人所知的福利经济学领域的某些结论。这又是一例,表明马歇尔第一个闯入
新的经济理论领域后,紧跟着产生了解释和扩展其分析的大量文献。消费者剩余的概念首先
由马歇尔提出,时至今日,福利经济学文献中仍然在讨论这一概念。

利用等式\cref{eq:1} $MU_A=P_A \cdot MU_M$,并假定货币的边际效用不变,则产品A的价
格与产品A的边际效用直接相关。马歇尔断定,产品A的价格是产品A对消费者的边际效用的一
种度量。由于边际效用递减,需求曲线向右下方倾斜。其向下的斜率表明,与一种商品稍后的
消费单位相比,消费者愿意为该种商品较早的消费单位多支付些。然而,在市场上消费者能
够以一种价格购买其消费的所有单位。因为这一价格度量了所消费的最后一单位的边际效用,
所以,消费者以低于他们愿意支付的价格得到了较早的单位,即边际内单位。消费者愿意支
付的总量与他们实际支付的总量之间的差额构成了消费者剩余。
\begin{figure}[ht]
  \centering
  \begin{tikzpicture}[thick, scale=0.8]
    \draw (0,0) node[below left] {$O$};
    \draw[very thick] (0,5) node[above] {价格} -- (0,0)  --  (7,0) node[right] {数量};

    \coordinate (d) at (0,4.2);
    \coordinate (d1) at (5, 0.7) ;
    \draw (d) node [left] {$D$} -- (d1) node [below right] {$D'$};

    \coordinate (p) at ($(d)!0.35!(d1)$);
    \draw[dashed]  (p) node [right] {$P$} -- ( p -| 0,2);
    \draw[dashed, name path=pm]  (p) -- ( p |- 2,0) coordinate [label=below:$M$] (m);

    \coordinate (a) at ($(d)!0.65!(d1)$);
    \draw[dashed, name path=ac]  (a) node [right] {$A$} -- ( a -| 0,2) coordinate [label=left:$C$] (c);
    \draw[dashed]  (a) --  ( a |- 2,0) coordinate [label=below:$H$] (h);

    \path [name intersections={of=ac and pm,by={[label=below right:$R$]r}}];
  \end{tikzpicture}%
  \caption{\label{fig:conssurplus}消费者剩余}
\end{figure}

马歇尔希望运用消费者剩余概念得出有关福利的结论;因此,他关注的是消费者作为一群人
时的剩余,而不是个别消费者的剩余。他运用市场需求曲线,而不是个别需求曲线来研究。
给定\cref{fig:conssurplus}所示的市场需求曲线,我们能分析消费者剩余。如果市场价格
为 $OC$,需求量将为 $OH$。因为 $DD'$是一条市场需求曲线,所以,存在愿意支付高
于 $OC$价格的购买者。第 $OM$个消费者本来愿意支付等于 $MP$的价格,但是只支付了相当
于 $MR$ 的价格。因此,$RP$就代表了消费者剩余。所以,其他边际内购买者都得到了一个
消费者剩余。总的消费者剩余等于 $CDA$,它是消费者购买商品的支出(或者说 $OCAH$)与他
们本来愿意的支出(或者说 $ODAH$)之间的差额。

因此,$CDA$是对消费者购买商品时获得的货币收益的度量。可以稍微不同地表达这一结果,
即实行完全价格歧视的垄断者,将使消费者移动到其需求曲线下方,在这一过程中获取总收
益 $ODAH$;但是,在所有消费者都以单一价格 $OC$进行购买的竞争性市场上,消费者的总
支出为 $OCAH$。因此,$CDA$是消费者节省的数量,或者说是他们的货币收益。然而,马歇尔
希望用效用来度量收益,只有存在一种不变的度量标准将价格转化为效用时,货币收益才能
被表示为一种效用收益。当我们从价格 $OD$到 $MP$再到 $HA$向下移动时,如果货币的边际
效用保持不变,那么,马歇尔的消费者剩余就是表示从产品消费中获得效用的一种可以接受
的方法。

马歇尔运用价格来度量效用取决于两个假设:(1)存在一种忽视替代与互补关系的加法效用
函数;(2)价格微小变动的收入效应忽略不计——也就是说,货币的边际效用不变。运用一个
更加一般化的非加法效用函数,埃奇沃斯和欧文·费雪分别提出并说明了,尽管能够运用加法
效用函数来度量效用,但是,如果考虑到替代与互补效应,这种方法将是不可行的。此外,对
于马歇尔和其他人提出的需求理论中的享乐主义因素,也存在普遍的批评。马歇尔通过进行
一些较小的术语变化,例如效用的满足来回应这些批评,但他还是基本上坚持如下观点,即价
格能够被用来度量效用。马歇尔清楚与度量消费者剩余相关的问题,这使他在福利经济学的
应用中,仅在价格发生微小变化时才使用这种度量。对于一些价格变化(例
如,\cref{fig:conssurplus}中 $HA$附近的价格),货币边际效用不变的假设看上去并不是
不现实,尤其是所讨论的商品支出仅代表了消费者总支出的一小部分时。对于大多数商品来
说,其价格微小变动的收入效应有可能很小,以至于能够被忽略掉。

\subsection{税收与福利}

马歇尔运用消费者剩余的概念来分析税收的福利后果。通过考察最简单的情形即一个成本不
变的行业,就能理解这一分析的实质。该行业用\cref{fig:constantcost}中所示的具有完全
弹性的供给曲线来表示,假定行业处于均衡状态,需求为 $DD'$,供给为 $SS'$,价格
为 $HA$,消费者剩余为 $SDA$。现在征收了一项等于 $Ss$的税收,使供给曲线移动
到 $ss'$。消费者剩余的损失为 $SsaA$,税收为 $SsaK$。消费者剩余的损失超过了收入的
增加,数量上为 $KaA$。因此,对成本不变行业所征收的税是不合意的。这一分析也能类似
地用来表明,对成本不变行业的补贴也是不合意的,原因是净成本将超过净收益。假定需求
为 $DD'$,供给为 $ss'$,价格为 $ha$。一个数量上为 $Ss$的补贴将使供给曲线向下移动
到 $SS'$。消费者剩余的增加量为 $SsaA$,它等于补贴总支出 $SsLA$减去 $AaL$所余的那一部分。
\begin{figure}[ht]
  \centering
  \subcaptionbox{\label{fig:constantcost}成本不变行业}{%
    \begin{tikzpicture}[thick, scale=0.8]
      \draw (0,0) node[below left] {$O$};
      \draw[very thick] (0,5) node[above] {价格} -- (0,0)  --  (7,0) node[right] {数量};

      \coordinate (d) at (0,4.2);
      \coordinate (d1) at (6, 1) ;
      \draw [name path = dd1](d) node [left] {$D$} -- (d1) node [below right] {$D'$};

      \coordinate [label=left:$s$] (s) at (0,3);
      \coordinate [label=right:$s'$] (s1) at (5,3);
      \draw [name path=ss1] (s) -- (s1);

      \coordinate [label=left:$S$] (S) at (0,2);
      \coordinate [label =right:$S'$](S1) at (5,2);
      \draw [name path=SS1] (S) -- (S1);

      \path [name intersections={of=dd1 and ss1,by={[label=above:$a$]a}}];
      \path [name intersections={of=dd1 and SS1,by={[label=below left:$A$]A}}];

      \draw[dashed, name path=ah]  (a) -- ( a |- 0,0) coordinate [label=below:$h$] (h);
      \path [name intersections={of=ah and SS1,by={[label=below left:$K$]K}}];

      \draw[dashed]  (A) -- ( A |- 0,0) coordinate [label=below:$H$] (H);
      \draw[dashed]  (A) -- ( A |- s1) coordinate [label=above:$L$] (L);
    \end{tikzpicture}%
  }\hfill
  \subcaptionbox{\label{fig:diminishing}规模收益递减行业}{%
    \begin{tikzpicture}[thick, scale=0.8]
    \draw (0,0) node[below left] {$O$};
    \draw[very thick] (0,6.5) node[above] {价格} -- (0,0)  --  (6.5,0) node[right] {数量};

    \coordinate (d) at (0,5.5);
    \coordinate (d1) at (5.5, 1) ;
    \draw [name path = dd1](d) node [left] {$D$} -- (d1) node [below right] {$D'$};

    \coordinate [label=left:$s$] (s) at (0,3);
    \coordinate [label=right:$s'$] (s1) at (5,5.5);
    \draw [name path=ss1] (s) -- (s1);

    \coordinate [label=left:$S$] (S) at (0,1);
    \coordinate [label =right:$S'$](S1) at (5,3.5);
    \draw [name path=SS1] (S) -- (S1);

    \path [name intersections={of=dd1 and ss1,by={[label=above:$a$]a}}];
    \draw[dashed, name path=ac]  (a) -- ( a -| 0,0) coordinate [label=left:$c$] (c);
    \draw[dashed, name path=ah]  (a) -- ( a |- 0,0) coordinate [label=below:$h$] (h);

    \path [name intersections={of=dd1 and SS1,by={[label=below left:$A$]A}}];
    \draw[dashed, name path=AC]  (A) -- ( A -| 0,0) coordinate [label=left:$C$] (C);


    \path [name intersections={of=ah and AC,by={[label=above left:$K$]K}}];

    \draw[dashed,name path=AH]  (A) -- ( A |- 0,0) coordinate [label=below:$H$] (H);
    \draw[dashed, name path = AL] (a) --  (a -| A) node [right] {$L$};


    \path [name intersections={of=ah and SS1,by={[label=below right:$E$]E}}];
    \draw [dashed] (E) -- (E -| 0,0) node [left] {$F$};
    \draw [dashed] (E) -- (E -| H) node [below right] {$Q$};

    \coordinate [label=above:$T$](T) at (intersection cs:first line={(A) -- (A|-s1)}, second line={(s) -- (s1)});
    \draw [dashed] (T) -- (T -| 0,0) node [left] {$R$};
    \draw [dashed] (T) -- (A);
  \end{tikzpicture}%
  }
  \caption{\label{fig:taxessurplus}税收、补贴以及消费者剩余}
\end{figure}

然后,马歇尔将这一分析扩展到收益递减行业(向上倾斜的供给曲线,
如\cref{fig:diminishing})与收益递增行业(向下倾斜的供给曲线)。假定存在收益递减,如
果供给曲线倾斜得很陡,使得税收超过了消费者剩余的损失,那么,征税将会导致福利增加。
同样的方式,对成本递减行业补贴将增加福利,原因在于消费者剩余的增加将超过补贴成本。
马歇尔因此断定,对社会来说,对某些收益递减行业征税,并用集中的收入补贴收益递增行
业,可能会从中获得利益(同上,172--176页)。

(蛋蛋注:如\cref{fig:diminishing},仿照《经济学原理》下卷图31,以下由我改编自马歇尔原文)
\begin{quotation}
  因对每单位征收$aE$的税,在新均衡位置下生产 $Oh$单位,则总税收为 $cFEa$,而消费者
  剩余的损失则为 $cCAa$;进一步精简之下,判定总税收与消费者剩余损失的谁大谁小,要依
  $CFEK$和 $aKA$而定。图中 $CFEK$明显比$aKA$大得多,如此征税可能导致福利增加。
\end{quotation}

因为全部分析依赖于效用能够用消费者剩余来度量这一不确定的观点,所以,它在制定政策
时的实践价值是有问题的。马歇尔提出这一分析的目的,不是为税收和补贴提供一套精确
的规则,而更多地是用以表明非规制的市场并不总是导致最佳的资源配置。亚瑟·C·庇古采
用了这些开创性的建议来形成一种扩展的福利经济学理论。

\subsection{马歇尔论供给}

马歇尔为成本与供给分析奠定了基础,这一分析是当前公认的大等本科课程讲授的内容。他
对供给理论最重要的贡献是他的时间周期概念,尤其是短期与长期的概念。他正确地认识到
市场周期、短期以及长期中行业供给曲线的形状,虽然他对这些形状的经济原因的解释经常
是不完善的、混乱的,甚至有时是错误的。

市场周期没有什么难点;此时供给完全没有弹性。现代微观经济理论将厂商与行业短期供给
曲线的形状解释为取决于收益递减原理。马歇尔指出,为了分析的目的,在短期中将厂商的成
本划分为不变成本与可变成本是有益的。然而,马歇尔没有基于收益递减原理,确立下列两者
之间的准确关系,即不变成本和可变成本的区别与厂商短期成本曲线的推导。他通常是在长
期分析的背景下,主要将收益递减原理运用于土地上。

他的确在短期中运用了不变成本与可变成本的区别来表明:即使发生亏损,只要弥补了总可
变成本,厂商将会在短期内继续经营。在这些情况下,厂商实际上通过经营使其亏损最小:停
业将导致亏损等于总不变成本,但是,只要总收益超过总可变成本,通过经营而产生的亏损
就低于总不变成本。因此,在一个完全竞争行业中,厂商的短期供给曲线,相当于其边际成本
曲线在其平均可变成本曲线以上的部分。以其特有的现实性,马歇尔继续推断,当价格下降
到平均成本以下发生亏损时,短期中厂商的实际供给曲线不可能是其边际成本曲线。他说,
厂商不愿意按照不能弥补其全部成本,即不变成本与可变成本的价格来销售产品,原因在于,
他们担心“扰乱市场”。扰乱市场意味着今天以低价销售,并阻止市场价格明天上升,或者
按照使行业中其他厂商抱怨的价格销售。因此,当发生亏损时,真正的短期供给曲线,并不
是边际成本曲线介于平均可变成本曲线与平均成本曲线之间的部分,而是位于边际成本曲线
左侧的一条供给曲线。在这一论述中,马歇尔去掉了完全竞争市场的假设,原因是在完全竞争
的严格定义中,没有厂商会关心向市场大量抛售商品,或关心其行为对行业中其他厂商造成的
后果。当放弃完全竞争的假设时,在马歇尔对市场运行的论述中,能够部分地找到琼,罗宾
逊(Joan Robinson,1903一1983)《不完全竞争》与爱德华.H.张伯伦(Edward
H.Chamberlin, 1899一1967)《垄断竞争理论》的灵感。

尽管根据现代标准,马歇尔对厂商长期成本曲线和供给曲线,以及行业长期供给曲线的论述
明显地不完善,但是,他在这些领域的早期努力,引发了20世纪20年代和30年代一系列令人
关注的文章,其中最为重要的是克拉彭(Clapham)、奈特、斯拉法以及维纳的文章。马歇尔指
出了决定厂商成本与供给曲线形状与位置的长期力量。首先是内在于厂商的力量。随着厂商
规模增加,内部规模经济导致成本降低,内部不经济导致成本增加。马歇尔对内部规模经济
的经济原因的论述,令人相当满意,但是,他对内部不经济的论述最少,并且,他并没有真
正面对下列问题,即规模经济与不经济之间的关系及其对厂商最佳规模的影响。

不过,马歇尔对外部经济(extermaleconomies)与不经济的论述,引出了大量分析其中所暗含
理论问题的文献。马歇尔希望对厂商和行业向上倾斜的短期供给曲线,与一些行业出现的随
着时间的变化,成本与价格下降的历史证据进行调和。他的这种调和是以外部经济的概念为
基础的。随着行业的发展,外部经济——马歇尔从未解释这些是外在于厂商还是外在于行
业——导致厂商和行业成本曲线与供给曲线向下移动。在这些情况下,行业的长期供给曲线将
向下倾斜:将在较低的价格上提供较大数量的供给。外部经济的主要原因是一个行业中所有
厂商成本减少,当所有厂商坐落在一起并分享他们的思想时,就会出现这种情况。地方化也
会给当地带来节省成本的辅助行业和熟练的劳动。

马歇尔对成本与供给的考察,提出了1900年至1940年之间被加以考察的很多重要的理论问题。
成本与供给曲线形状的经济原因是什么?为什么一些行业的短期供给曲线上升而成本与价格
在长期中下降?内部经济和外部经济与竞争性市场相容吗?

\subsection{马歇尔论分配}

马歇尔对生产要素价格决定力量与收入分配的解释,与其分析中的其余部分是一致的。在此,
与在其它地方一样,他经常宽宏大度地认同对其理论(例如分配的边际生产力理论)的批评。

用于解释最终产品价格的基本供求分析,以及短期与长期之间的区别,也同样被用来解释地
租、工资、利润以及利息。生产要素的需求是一种引致需求,它取决于要素的边际产品价值。
然而,边际产品难以分离,原因在于,技术通常要求一种要素的增加伴随着更多其它要素的
增加。马歇尔通过在边际量上计算他所谓的净产品,解决了边际产品的度量问题。如果新增
一名劳动者需要配备一把铁锤,那么,劳动的净产品就是劳动者新增的总收益减去铁锤的新
增成本。然后马歇尔指出,将要素定价理论称作分配的边际生产力理论是错误的,原因是边
际生产力仅度量了对要素的需求,而要素价格是由边际量上的需求、供给以及价格相互作用
决定的。在对他的边际生产力概念及其关于劳动与工资的度量进行解释之后,马歇尔提倡对
边际生产力理论进行谨慎的阅释:
\begin{quotation}
  这个原理有时被当作工资理论提出来。但任何这种主张都站不住脚。一个工人的报酬趋
  向等于其劳动纯产品,就其本身来说,没有什么实际意义:因为为了估算净产品,除了
  他自己的工资外,我们还得考虑他所生产的产品的全部费用。

  不承认它是一种工资理论,这是对的;但不承认这个原理是阐明决定工资的那些原因中的
  一个原因所起的作用,这是不对的。(同上,226页)
\end{quotation}

他说,要素结合的比例取决于它们的边际产品与价格。关心利润最大化的企业家,希望在最
低的可能成本上生产一个规定水平的产量,这将使得厂商按照下列方式使用生产要素,即各
生产要素的边际物质产品与其价格的比率相等。如果厂商按照别的方式使用生产要素,那么,
它就有可能在边际量上进行替换,并实现较低的成本。马歇尔没有详述产品用尽问题和欧拉
定理;他认同维克斯蒂德-弗拉克斯的结论,即在长期竞争均衡下,当每种要素获得其边际产
品价值时,总产品被用尽。马歇尔对个别生产要素收益的分析——工资、地租、利润、利息
——不是特别能引起注意。然而,他对准租金概念的发展,连同他的要素价格与分配理论却
值得注意。

\subsection{准租金}

运用其准租金的概念,马歇尔不仅提出了关于市场系统运行的见解,而且使古典经济学家与
边际效用经济学家之间争论的一个方面更加清楚。古典经济学家主张,除了土地之外,支
付给生产要素的报酬决定价格。最终产品的价格取决于边际量上的生产成本。因为在边际量
上不存在地租,所以,(为约翰·斯图亚特·穆勒所主导的)古典学说认为工资、利润以及利息
是决定价格的因素,价格因此是由供给方面来决定的。边际效用经济学家加入了对古典成本学
说的早期批评,他们声称,支付给生产要素的报酬是被价格决定的。马歇尔的分析表明,
无论一种要素支付是决定价格还是被价格决定,都取决于所考察的时间周期(它极其重要地影
响着要素供给曲线的形状),以及进行分析的特定角度。我们来考察称作地租、工资、利润以
及利息的报酬。

土地的收益在历史上被称为地租。在分析地租时,李嘉图假定土地的供给完全没有弹性,并
且土地不存在可以替代的其他用途。由于使用土地而支付给地主的报酬,被价格决定而不
是决定价格。谷物的高价格是高地租的原因。尽管对这一理论,有一些来自二流经济学家的
批评,然而,从约翰·斯图亚特·穆勒时代到马歇尔时代,基本的李嘉图地租分析保持不变。
马歇尔认识到问题要更加复杂。当从整个经济体的角度来看时,地租是被价格决定的,
而不是一种生产成本。然而,如果从个别农场主或厂商的角度看,地租是一种生产成本,从
而是决定价格的因素。希望租用土地来种植燕麦的农场主,必须支付足够的价格,以防止土
地用于可替代的用途。除非燕麦种植者愿意支付的地租,高于大麦种植者或者房地产开发商
愿意支付的地租,否则燕麦种植者将不能在竞争性市场上租用到土地。因此,从个别农场主
或厂商的角度看,地租是一种必须支付的生产成本,就像必须支付劳动与资本成本一样。

马歇尔也提出,在某些情况下即使从整个经济体的观点来看,地租也是决定价格的因素。对
于一个拥有不花费任何成本的未开垦土地的经济体来说,就像19世纪的美国,地租可以被视
为决定价格的因素。马歇尔推论说,原始拓荒者将其土地开垦收益中的一部分,不仅视为种
植的直接收益,而且视为随着人口流向开发区域的边缘地带而发生的土地价格的增值。因此,
这种预期的土地价格增值,是必须支付的必要的供给价格的一部分,以使个人能够免受边缘
生活的艰苦与危险。上升的土地价格等于上升的地租的资本化价值,从而被认为是一种社会
成本。从经济体的角度看,这些情况下地租是决定价格的因素。另一方面,从经济体的角度
看,在一个全部土地都已经开星的国家,土地供给曲线完全没有弹性,地租因而是被价格决
定的。对于拥有未开垦土地的国家,土地的供给曲线向右上方倾斜;地租越高,越多的土地将
得到开垦,地租是决定价格的因素。在给埃奇沃斯的一封信中,马歇尔评论说:
\begin{quotation}
  不说“地租不进入生产成本”是聪明的,因为那将使很多人困惑。但是,说“地租的确进入
  生产成本”是不道德的,原因是,这种说法确实会以一种否定微妙真相的方式被加以应
  用。
\end{quotation}

马歇尔继续表明在短期中被称作工资、利润以及利息的收益如何其有地租的一些特征。在长
期均衡下,支付给特定劳动类型的工资(例如,一名会计师),将恰好使从事这一职业的人远
离其他职业,留在现在的职业上。这一长期工资是社会为了引出供给的数量而必须支付的供
给价格,工资因此是决定价格的因素。假设对会计师服务的需求增加了,会计师的工资因此
提高。会计师的供给在短期中比在长期中缺乏弹性。工资的提高不会极大地影响供给的数量,
所以,短期工资将上升到长期工资水平之上。较高的短期工资与使个人留在职业中所必需的
价格没有关系,所以它是被价格决定的,而不是决定价格的因素。理解这些问题的关键在于
供给曲线的弹性。在很短的时期中,某一特定种类劳动的供给曲线可以被看做是完全无弹性
的。由于劳动的供给数量保持不变,需求增加将导致较高的工资。在短期中,随着其它职业
的人经过培训后进入本职业,工资将略微下降。在长期中,供给曲线将变得更加有弹性,工资
将下降为长期均衡值,即必要的供给价格。因此,在短期和市场周期中,工资像地租一样,是
决定价格的因素。马歇尔将这些支付称作准租金。

\begin{quotation}
  我们已经面临着这部分经济学的中心理论了,即:“凡被正确地看成是‘自由’资本
  或‘流动’资本或新投资的利息的东西,被当作旧投资的一种租(即准租)是更加正确些。
  不过,在流动资本和在某特殊生产部门中所‘沉淀的’资本之间,在新旧投资之间,不存
  在着严格的界限。每组投资可以逐渐变成另一组投资。即使地租也不是当作一种自在的
  东西,而是当作一大类中的一个主要的种看待的……(同上,115页)
\end{quotation}

运用其准租金概念,马歇尔说明了关于生产要素的报酬是决定价格还是被价格决定的争论。
这完全取决于时间周期:在长期中,工资决定价格,但是在短期中工资是被价格决定的,就
像地租一样。

马歇尔也将其准租金的概念应用于短期中对利润的分析。在完全竞争市场的长期均衡下,每
个厂商将只获得正常的利润率。正常利润是生产成本,必须由厂商来支付以便持有资本,就像
必须支付正常工资以吸引和持有劳动力一样。如果一个厂商在长期中没有获得正常利润,资
本将离开该厂商,投奔能获得正常利润的其他厂商和行业。因此,在长期中,正常利润是一种
必要的生产成本,从而是决定价格的因素。但是,在短期中,称作利润的收益可以被视为一
种准租金,因此是价格被决定的因素。短期中厂商的成本被划分为可变成本与不变成本。在
短期中,厂商的收益必须足以支付所有可变要素的机会成本,否则它们将离开厂商。剩下的是
不变要素的收益,在短期中不变要素在供给上是完全无弹性的。对于不变要素来说,利润在短
期中是一种准租金,是被价格决定的结果。如果总收益超过总成本,就获得正常利润之上的
利润;但是,在竞争盛行的地方,长期中它们将会消失。如果总收益超过总可变成本但低于
总成本,就会发生亏损;但是,在长期均衡下这些亏损也会消失。像工资一样,利润可以是
决定价格的因素,也可以是被价格决定的结果,它取决于所考察的时间周期。

准租金的概念被用于短期中的利息分析。长期中,存在一种正常的利息率,它是一种必要
的生产成本,从而决定价格。虽然一项旧的资本投资依据市场供求状况有可能获得高于或低
于正常利息率的收益,但是,因为资本在短期中是不变的,或者说是沉没的,所以其收益就是
一种准租金。

在最为宽泛的角度上,准租金的分析能够用来指出古典经济学家与边际效用经济学家之间的
一些本质区别,前者强调供给方面,后者强调需求方面。如果生产要素的供给不变,那么,
任何要素的收入都是一种准租金,要素价格被决定,要素收益受到需求水平相当大的影响。
在长期中,要素的供给不固定,最终产品的长期均衡价格必须足以支付生产中发生的全部社
会必需成本。在这些情况下,支付给生产要素的报酬就决定价格,并且,对最终价格的分析
必须更大关注对供给的作用。从分析上看,被称作工资、利润、地租以及利息的收益,在不
同的时间周期中有很多共同点。不可否认地,尽管大自然没有明显地区分时间周期,然而,
马歇尔普遍化的时间周期理论及与之相伴的准租金学说,深刻地洞察了由相对价格的决定力
量所引起的问题。

\subsection{稳定均衡与不稳定均衡}

马歇尔将需求表看做是个人愿意为一种商品规定的数量所支付的最高价格,因此,数量是自
变量,价格是因变量。另一方面,将供给表看做是销售者愿意提供一种商品规定的数量时所
能接受的最低价格。同样地,数量是自变量,价格是因变量。在其《经济学原理》第五篇第三
部分第6段中,马歇尔解释了实现市场均衡的过程。因为他把数量看做是自变量,所以导致均
衡的调整在很大程度上是对数量调整方面的论述。如果在某一既定数量上,需求价格超过供
给价格,“那么,销售者所获得的将多于他们在市场充足时把同等数量产品带入市场所获得
的利益;这时,一种积极的力量在发生作用,它倾向于增加销售的数量。”


\begin{figure}[ht]
  \centering
  \begin{tikzpicture}
    \draw (0,0) node[below left] {$O$};
    \draw[very thick] (0,5) node[above] {价格} -- (0,0)  --  (6,0) node[right] {数量};

    \coordinate (d) at (0.75,4);
    \coordinate (d1) at (4,0.8);
    \draw (d) node [above left] { $D$} -- (d1) node[below right] { $D'$};

    \coordinate (s1) at (4,4);
    \coordinate (s) at (0.75,0.8);
    \draw (s1) node [above right] { $S'$} -- (s) node[below left] { $S$};

    \coordinate [label=above:$A$](A) at (intersection cs:first line={(d)--(d1)}, second line={(s)--(s1)});
    \draw [dashed] (A) -- (A -| 0,0) coordinate [label=left:$P_e$] (Pe);
    \draw [dashed] (A) -- (A |- 0,0) coordinate [label=below:$H$] (H);

    \coordinate [label=above:$s_2$](s2) at ($(s)!0.8!(s1)$);
    \draw [dashed] (s2) -- (s2 -| 0,0) coordinate [label=left:$P_2$] (P2);
    \draw [dashed] (s2) -- (s2 |- 0,0) coordinate [label=below:$R_2$] (R2);

    \coordinate [label=above:$d_1$](d1) at (intersection cs:first line={(s2)--(P2)}, second line={(d)--(d1)});
    \draw [dashed] (d1) -- (d1 |- 0,0) coordinate [label=below:$R_1$] (R1);

    \coordinate [label=above left:$s_1$](s1) at (intersection cs:first line={(d1)--(R1)}, second line={(s)--(s1)});

    \draw [dashed] (s1) -- (s1 -| s2) node [right] (d2) {$d_2$};
    \draw [dashed] (s1) -- (s1 -| 0,0) coordinate [label=left:$P_1$] (P1);

    \draw [very thick, -{Stealth[length=3mm, sep = 5pt, blue]}] (P2) -- (Pe);
    \draw [very thick, -{Stealth[length=3mm, sep = 5pt, blue]}] (P1) -- (Pe);

    \draw [very thick, -{Stealth[length=3mm, sep = 5pt, blue]}] (R1) -- (H);
    \draw [very thick, -{Stealth[length=3mm, sep = 5pt, blue]}] (R2) -- (H);

  \end{tikzpicture}%
  \caption{\label{fig:equili}实现均衡}
\end{figure}

\cref{fig:equili}是马歇尔的均衡实现过程的图形示意。在数量 $R_1$上,需求价
格 $R_1d_1$超过了供给价格 $R_1s_1$;因此,更大的数量将会被销售者带入市场。在数
量 $R_2$上,供给价格 $R_2s_2$,超过了需求价格 $R_2d_2$,销售者减少带入市场的数量。
当消费者对需求价格与供给价格的相对水平作出反应时,通过数量变化导致了均衡。所实现
的这种均衡是一种稳定的均衡,原因在于,离开均衡的任何移动将形成使市场恢复均衡的力
量。

瓦尔拉斯与当前的经济理论家,在分析市场力量时遵循了一套不同的行为假设。这些经济学家
将价格看做是自变量。对他们来说,需求表显示了个人愿意在不同价格上购买的数量,供给表
表示销售者在不同价格上愿意提供的数量。

哪一种是正确的:像瓦尔拉斯一样,将价格看做是自变量,还是像马歇尔一样,将数量看做
是自变量?因为这个问题涉及关于购买者与销售者在市场上的行为方式的假设,所以,只能通
过经验研究来决定。但是,这样两种描述市场行为的方式,它们在分析上的结果能够从理论
上推导出来。马歇尔断言并不存在理论上的差异,然而他错了。

尽管现代理论在将价格看做是自变量方面遵循了瓦尔拉斯,然而,在将价格置于供求图形的
纵轴方面又遵循了马歇尔。数学上的惯例是把因变量放在纵轴上。写成 $p=a-bq$的线性需求曲
线方程,暗示着价格是因变量,然而,现代理论的行为假设将价格看做是自变量。

如果和需求曲线向下倾斜、供给曲线向上倾斜那么,瓦尔拉期与马歇尔的确会得到相同的结
论。再次提到\cref{fig:equili},我们看到在马歇尔的分析中,数量变化将导致一个均衡的
数量 $OH$。然而,瓦尔拉斯与现代理论利用价格作为自变量,按照下列方式分析了导致均衡
的力量。在价格 $P_2$上,需求的数量是 $P_2d_1$,低于供给的数量 $P_2s_2$,因此存在超
额供给。销售者之间的竞争将促使价格下降,直至达到一个使市场出清的价格,即在这一价
格上供给的数量等于需求的数量。在低于均衡价格的价格 $OP_1$上,存在超额需求;供给的
数量 $P_1s_1$,低于需求的数量 $P_1d_2$。因此,购买者之间的竞争促使价格上升直至市场
出清。

当供给曲线向下倾斜时,不稳定均衡(unstable equilibrium)就可能存在。在不稳定均衡下,
如果价格或数量达到其均衡值,它们将停在那里;但是,如果系统被扰乱,任何脱离均衡状
态的运动会产生使变量进一步远离均衡位置的力量。

\begin{figure}[ht]
  \centering
  \subcaptionbox{\label{fig:unsequ1}以价格为自变量时,不稳定均衡}{
  \begin{tikzpicture}[thick]
    \draw (0,0) node[below left] {$O$};
    \draw[very thick] (0,5) node[above] {价格} -- (0,0)  --  (5,0) node[right] {数量};

    \coordinate (d) at (0.8,4.5);
    \draw [name path = dd] (d) -- +(-50:5cm) node [below right] {$D$};

    \coordinate (s) at (0.5,3.2);
    \draw [name path = ss] (s) -- +(-20:5cm) node [right] {$S$};

    \path[name intersections={of=dd and ss,by={[label=below left: $X$]X}}];
    \draw[dashed] (X) -- (X -| 0,0) coordinate [label=left:$P_E$] (PE);
    \draw[dashed] (X) -- (X |- 0,0) coordinate [label=below:$H$] (H);

    \ExtractCoordinate{H};
    \draw [very thick, -{Stealth[length=3mm, sep = 5pt, blue]}] (\XCoord + 20pt, 0) -- (H);
    \draw [very thick, -{Stealth[length=3mm, sep = 5pt, blue]}] (\XCoord - 20pt, 0) -- (H);

    \ExtractCoordinate{PE};
    \draw [very thick, -{Stealth[length=3mm, sep = 5pt, blue]}] (PE) -- (0, \YCoord + 20pt);
    \draw [very thick, -{Stealth[length=3mm, sep = 5pt, blue]}] (PE) -- (0, \YCoord - 20pt);
  \end{tikzpicture}%
  } \hfill
  \subcaptionbox{\label{fig:unsequ2}以数量为自变量时,不稳定均衡}{%
  \begin{tikzpicture}[thick]
    \draw (0,0) node[below left] {$O$};
    \draw[very thick] (0,5) node[above] {价格} -- (0,0)  --  (5,0) node[right] {数量};

    \coordinate (s) at (0.8,4.5);
    \draw [name path = ss] (s) -- +(-45:5cm) node [below right] {$S$};

    \coordinate (d) at (0.5,3.2);
    \draw [name path = dd] (d) -- +(-20:5cm) node [right] {$D$};

    \path [name intersections={of=dd and ss,by={[label=below left:$X$]X}}];
    \draw[dashed] (X) -- (X -| 0,0) coordinate [label=left:$P_E{}'$] (PE);
    \draw[dashed] (X) -- (X |- 0,0) coordinate [label=below:$H'$] (H);

    \ExtractCoordinate{H};
    \draw [very thick, -{Stealth[length=3mm, sep = 5pt, blue]}] (H) -- (\XCoord + 20pt, 0) ;
    \draw [very thick, -{Stealth[length=3mm, sep = 5pt, blue]}] (H) -- (\XCoord - 20pt, 0) ;

    \ExtractCoordinate{PE};
    \draw [very thick, -{Stealth[length=3mm, sep = 5pt, blue]}] (0, \YCoord + 20pt) -- (PE);
    \draw [very thick, -{Stealth[length=3mm, sep = 5pt, blue]}] (0, \YCoord - 20pt) -- (PE);
  \end{tikzpicture}%
  }
  \caption{\label{fig:unsequ}稳定均衡与不稳定均衡}
\end{figure}

\cref{fig:unsequ1}代表了运用马歇尔分析的稳定均衡,其中数量为自变量。在大于 $OH$的
数量上,供给价格超过需求价格;也就是说销售者接受的最低供给价格超过了购买者愿意支
付的最高价格,销售者将因此减少所所供的数量直至 $OH$。反之亦然。然而,如果价格是自
变量,这就代表了一种不稳定均衡。在低于 $OP_E$的价格上,供给的数量超过需求的数量,
销售者之间的竞争将促使价格进一步下降。如果价格高于 $OP_E$,超额需求将促使价格进一
步上升。

\cref{fig:unsequ2}说明,自变量为价格时的稳定均衡,在自变量为数量时,同样能成为不
稳定均衡。\cref{fig:unsequ1}与\cref{fig:unsequ2}的比较说明,当供给曲线向右下方倾
斜时,均衡的稳定性将取决于供求曲线的相对斜率,以及所使用的行为假设。马歇尔与瓦尔
拉斯的确拥有一个共同因素,它使得他们两人都断定,在向上倾斜的供给曲线下,能够实现
稳定的均衡。他们两人的分析都是一种静态的框架。对于瓦尔拉斯来说,现有时期供给的数
量与需求的数量取决于现有时期的价格;对马歇尔来说,现有时期供给与需求的价格取决于
现有时期的数量。因此,马歇尔与瓦尔拉斯两人都假定了静态行为。

\subsection{经济波动、货币以及价格}

尽管马歇尔最为关注的是微观经济理论,然而,通过研究货币力量对价格总水平的影响,他
对宏观经济学也做出了重要贡献。虽然马歇尔的一些最早著作(1871)关注于货币数量理论,
但是,在1923年名为《货币、信用与商业》的著作出版之前,他没有发表关于货币的任何系
统性作品。他关于宏观经济主题的观点虽然没有公开,但是,在他的演讲中和在呈给政府各
委员会的证词中得到了充分的发展。马歇尔在宏观经济学方面的创作,主要涉及经济稳定与
不稳定,以及价格总水平的决定力量。

马歇尔基本认同约翰·斯图亚特·穆勒关于经济体稳定性的观点:永远不可能存在总需求不足,
原因是储蓄决策就包括投资决策;不可能存在普遍的生产过剩。这种推论方式始于亚当·斯密,
由詹姆斯·穆勒、李嘉图以及萨伊进行了详细阐述;现在,在文献中以萨伊定律而为人所知。
马歇尔时代的经济活动中当然存在波动,一些经济学家,特别是美国的约翰·A·霍布斯倡导消费
不足理论。马歇尔认为,对经济波动原因的理解“不是通过研究消费而获得的,就像一些草
率的经济学家所断言的那样”。马歇尔对经济波动原因的解释遵循了约翰·斯图亚特·穆勒的
观点,穆勒强调经济信心的影响。在高涨期间,经济信心增强,信用快速扩张;在下降期间,
厂商变得悲观,信用快速收缩。穆勒对萨伊定律的认同使他断言经济萧条不能归因于体制内
的任何基本问题。马歇尔提出了两项公共政策来应对经济萧条与失业。第一项是控制市场,
目的是使信用在经济信心高涨期间不至于过度扩张,因为过度扩张会导致不景气。如果确实
发生了经济萧条,政府可以通过保证厂商没有风险来帮助恢复经济信心。马歇尔对这个办法
并不是完全满意,原因是它很难在不产生一些不利后果的情况下得以实现。例如,保证厂商
没有风险,既给有能力的厂商也给没有能力的厂商上了保险,从而扰乱了奖赏有能力者、处罚
无能力者的市场过程。

尽管马歇尔对理解经济波动原因的贡献是不充分的,然而,他对价格总水平决定力量的解释却
是突出的。他认识到其微观经济分析是基于下列假设:即存在充分就业且价格总水平没有重
大变化。他对价格总水平决定因素的分析,是在其供求分析的框架内构建的一种货币数量理
论,关于这部分内容,我们将在第15章中予以考察。

\section{总结}

从马歇尔开始研究经济学以来已经过去了一个多世纪,尽管如此,他在微观经济学方面的贡
献,还在为正统的大学本科理论提供基本原理。马歇尔不是强调他与过去经济学家的区别,
而是承认借鉴了他们的思想。他把他的工作看做是对斯密、李嘉图以及约翰斯图亚特·穆勒工
作的继续,并总是宽宏大量地阐释他们的工作。其作品的特点是谦逊,这是有巨大影响的思
想家著作中少有的一种特质。

马歇尔凭借坚实的数学背景,满怀帮助低收入群体的人道主义愿望走上经济学道路。他认为
有可能区分经济学中的规范因素与实证因素,并忙于他所认为的——经济学是一种实证的免于
价值判断的科学,这一科学基于下列信念,即如果我们了解了是什么,社会就能更好地选择
应该是什么。

古典正统理论没有取得一种统一的方法。亚当.斯密在《国富论》中综合了理论、历史以及描
述方法;他最薄弱的环节是理论。尽管李嘉图没有特别关注方法,然而,他在没有运用数学
的情况下,呈现了一种几乎完全处于抽象、演绎以及理论模式中的方法。李嘉图的弱项是历史
与描述的方法。约输·斯图亚特·穆勒追随着斯密,试图构造一种理论、历史以及描述方法彼
此加强和补充的结构。然而,他们具有很多共同因素。他们都认为,经济理论普遍正确,能
同等地应用于历史上的不同时期和不同市场结构的社会。他们也普遍假定,从家庭与厂商层
面开始,才能最好地实现对整个经济体的理解。人类的本性与行为先于文化。他们的另一个
共同因素是下面这种极为重要的信仰,即在自由市场中经济冲突能够得到和谐解决。无论自
由市场多么不充分,与政府干预经济相比,它们都更加可取。这一和谐自然秩序中的唯一明
显缺陷是地主与工业家之间的冲突。除此之外,通过没有政府指导的市场,稀缺资源将得到
有效配置,市场自由发挥作用将保证资源得到充分利用。在相对价格决定力量的古典分析中,
长期价格一般被假定为取决于成本方面或供给万面。

这些古典思想并不为每个人所接受。在后李嘉图时期,批评古典价值理论,并进一步提出效
用与需求而非成本与供给是决定相对价格的关键因素的文献得到发展。一些经济学家运用李
嘉图的劳动价值理论来表明劳动受到剥削,并因此对古典体系中经济过程的和谐运转表示怀
疑。奥古斯塔·孔德、赫伯特·斯宾塞所提出的体系,对古典理论的方法基础表示怀疑,古典理
论狭窄地界定了经济学的范围,并且将人类行为视为先于文化与社会。德国与英国的一些经
济学家抨击古典理论的抽象特征,试图阐述一种更加宽泛的以历史为导向的方法来理解经济。
最终,古典理论的基本结构受到了杰文斯、瓦尔拉斯、门格尔的攻击,他们几乎专门强调需
求与边际效用的作用,并希望以此来替代生产成本价值理论。

马歇尔经济学是这些方法与理论争议的产物。马歇尔一贯拒绝采取一种盲从的方法来对待这
些问题,因此,他的结论不能令任何一方独断的思想家满意。他认为,浪费时间去争论经济
学的特有方法是无意义的。经济学家应当运用适合其训练与性情的方法,不同的方法应当被看
做是相互补充的,而不是相互排斥的。关于价格是由供给单独决定还是由需求单独决定的争
论,同样毫无意义。马歇尔指出,价格是大量复杂而相互作用力量的结果。将价格的决定过程
视为下面这种简单的因果关系链条是错误的,即效用决定需求,然后需求决定价格,或者说,
成本决定供给,然后供给决定价格。无论是效用方面还是成本方面的边际价值都不决定价格。
我们定位于边际量,是为了考察正在运转的力量,并提高我们对它们的认识。但是,当我们
转到边际量上时会发现,效用、成本以及价格相互决定彼此的价值,并不存在简单的因果链
条。边际量、局部均衡、其他条件保持不变、时间周期、代表性厂商以及生产要素,全部都
是抽象的理论构件,它们帮助我们分解分析中的复杂问题。然而,这种理论进步的取得是以
现实的缺失为代价的,因此,经济学家必须用描述性的和历史的材料来补充纯理论。

他声称,尽管在长期中价格取决于一套复杂的力量,但是,古典经济学家强调成本与供给的
重要性基本上是正确的。机会成本的概念提供了在短期中当供给相对不变时,进行资源配置
的一些见解,但是在长期中,通过考虑实际生产成本、劳动的努力以及资本家的等待或者节
欲,能够获得关于定价过程更加基本的见解。马歇尔从未能彻底摆脱边沁的享乐主义心理,
虽然他充分了解它所招致的批评。

现在的局部均衡微观经济理论的基本框架来自于马歇尔的《经济学原理》。尽管从此之后,
存在着很多对微观经济理论的重要贡献,然而,大部分是对方法而不是对实质性分析的补充。
一个主要的例外是由琼·罗宾逊和爱德华·H·张伯伦在20世纪30年代发起的对市场结构理论的
贡献,他们的很多观点是由马歇尔提出的。马歇尔体系的一个重大缺点是,他未能考察收入
与就业水平的决定力量。但是,当约翰·梅纳德·凯恩斯于20世纪30年代承担这一任务时,又
是在应用于总变量的马歇尔供求分析框架内予以阐明的。



%%% Local Variables:
%%% mode: latex
%%% TeX-master: "../../main"
%%% End:
