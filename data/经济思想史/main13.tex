\chapter{现代非正统经济思想的发展}

争议是经济学的组成部分,今天,争议仍然持续着。新古典时期的终结,部分原因在于新古典
经济学的缺陷,这些缺陷已经被持不同意见的或者说非正统的经济学家所指出。新古典经济
学以三种方式回应这些指责:(1)忽视批评,视其为毫无理由的;(2)将全部批评或部分批
评融入最重要理论的范围、内容以及方法中;(3)发展一种方式来传播存有争议的问题。在
这一过程中,在重要的研究生学院和研究中心,尤其是在美国,经济学专业中的主流实践活
动,从新古典经济学演化为折衷的形式上的模型构建。

由于经济学专业中的这些变化,与新古典经济学占支配地位时相比,今天的非正统更加困
难。“新古典经济学家”这一术语是由托尔斯坦·凡勃仑创造的,目的是为他发表不同意见提
供目标。它使得凡勃仑能够将阿尔弗雷德·马歇尔与古典经济学合并在一起加以考虑,从而反
对马歇尔在正统与非正统之间寻找共同基础的尝试,前者具有高度抽象的理论结构,后者则
强调历史与制度因素。凡勃仑时代的主流经济学家并不将他们自己称为新古典主义;非正统经
济学家这样称呼他们,并且这一术语很快成为对主流经济学家信信的讽刺,而不是对他们实
际信仰的描述。这一点一直持续到今天,很多非正统经济学家仍然在批评“新古典经济学”"。
这一术语给他们的抨击提供了明确的目标。但是,正如我们在对现代微观经济学与宏观经济
学的论述中所主张的,现代经济学对新古典经济学的抨击经常偏离目标。

现代经济学的实质是其折衷的、形式化的模型构建方法。它对假设并没有做出严格的限制,
它也不具有新古典经济学所具有的意识形态内容。因为现代形式主义建模经济学的假设与核
心价值如此折衷和杂乱,因此难以对其进行择击。能够对这一庞大的阿米巴理论躯体的哪一
部分准确地进行攻击呢?现代主流经济学家对大多数异议的回应是谈论现代经济学的另一部
分内容,以此来表明主流是如何对待这个问题的,这就使得隔离与划分持不同意见的群体比
较困难。不过,一些非正统群体十分独特,有正当理由将其具体论述,而且读者应该还记得本
书前面有关分类限制的附文一一进行分类是为了阐明问题并获得见解,但不可避免地对被划
分的群体造成不公平。
\enlargethispage{\baselineskip}
现代非正统思想家大致分成五个持不同意见的群体:激进主义者,现代制度主义者,后凯恩斯
主义者,公共选择提倡者以及新奥地利学者。根据他们的政治观点,我们按照从自由到保守
的不同程度进行排列,将这些群体组织在\cref{tab:heterodox}中。阅读该表时请记住,我们
对非正统经济学家进行论述,目的是为现代美国非正统思想的多样性提供证据,并简单地介
绍一些引人瞩目的读物;决不是做到详细而无遗漏。作为一种概要,表中模糊了既定流派内部
的重要差异。
\clearpage

% Please add the following required packages to your document preamble:
% \usepackage{booktabs}
% \usepackage{graphicx}
% \usepackage{tabularx}

\newgeometry{vmargin=1cm,hmargin=1cm}
\begin{landscape}
  \thispagestyle{empty}
  \begin{table}[htbp]
    \small
  \centering
  \caption{非正统经济学家}
  \label{tab:heterodox}
 \setlength{\extrarowheight}{7pt}
  \begin{tabularx}{\linewidth}{@{}lXXXXX@{}}
    \toprule
    &  \multicolumn{1}{c}{激进主义者}& \multicolumn{1}{c}{制度主义者}
    &后凯恩斯主义者 & 公共选择提倡者 & 新奥地利学者 \\ \midrule
    对个人理性的看法 & 个人追求阶级信仰;在资本主义社会,自我实现极其困难 &个人主义心理状态是不正确的;人们通过文化学到适度的举止&不确定性使得个人难以理性化&个人在生活的各方面都是理性的,包括政治方面;寻租很重要&激进的个人主义;与自有哲学密切相连\\
    政策观点  & 政府代表了统治阶级;对于重大改革来说,政府形式的重大变革是必须的 & 赞成更多的政府干预
   & 倾向于赞成政府干预 & 政府是个人政治利益的反映;政府干预越少越好;强烈反对寻租形式的政府干预 & 基于道德基础强烈反对政府干预;认为它侵犯了个人的权利 \\
    生产理论  & 一些人主张劳动价值论,一些人否定劳动价值论;倾向于认为资本家从工人那里榨取了剩余价值 & 厂商用单凭经验的方法来决定价格;集中研究对定价的制度约束   & 厂商利用成本加成定价;利润由再投资的需要决定 & 个人层面上的利润最大化;一般认同主流观点,尽管认为寻租导致了垄断 & 利润最大化的厂商 \\
    分配理论 & 分配的阶级理论,基础是统治者的权力 & 由制度与法律结构决定的分配;市场不太重要
  & 由利润--工资混合决定的宏观经济分配理论 & 分配的边际生产力理论,为寻租所修订 & 边际生产力理论;集中于财产权利 \\
    健在的主要倡导者 & 安华·萨克,萨缪·鲍尔斯,赫伯特·金迪思 & 约翰·亚当斯,沃伦·萨缪尔斯
    & 保罗·戴维森,杨·克雷格尔 & 戈登·塔洛克, 詹姆斯·布坎南,罗伯特·托利森 & 彼得·博斯克,唐纳德·戈登,伊思雷尔·柯兹纳\\
    主要杂志 & 激进政治经济学评论 & 经济问题杂志 &后凯恩斯主义经济学杂志 & 公共选择 &卡托论文集 \\
    \Longunderstack{强调其观点的\\ 主要研究生院} & 麻省理工大学,社会研究新学员,犹他州大学 & 科罗拉多州立大学,内布拉斯加大学
   & 密苏里大学堪萨斯分校,利维研究所,巴德学院 & 乔治梅森大学 & 奥本大学,乔治梅森大学\\
    宏观经济观点 & 没有大规模的国家干预,经济体倾向于失业与危机  & 反对基于整体理性的模型;相关模型必须具有更多的制度结构;通常支持凯恩斯的政策
    & 反对IS--LM;不确定性使建模较为困难;相信多重均衡 & 采取一种从本质上说属于微观经济学的观点;认为独立的宏观经济学不存在 & 市场过程很重要;主流模型未对市场进行必要的关注 \\ \bottomrule
  \end{tabularx}%
\end{table}
\end{landscape}
\restoregeometry


\section{激进主义者}

\begin{quotation}
伯格:人类不应当相互蚕食。

伍希:多好的黏乎乎的麻醉品啊!你不能命令人们该吃什么,不该吃什么。人类一直就在相互
蚕食,并一直将会如此。这是天性。你不可能改变人类的天性。

伯格:我爱我的人类朋友。

伍希:我也一样爱我的人类朋友——身上沾着肉汁的人类。

\bigskip

保守的经济学家就像伍希一样,他们认为,人类生来具有某些想法一一例如,吃人,或者拥有
奴隶,或者做一个竞争的资本家——没有办法来改变这些观点……

与保守者相反,激进的经济学家认为,所有的观点与偏好——例如,我们对凯迪拉克的渴
望——都是由我们所生活的社会塑造的。……因为我们的意识形态是由社会环境决定的,所以
激进经济学家主张,社会经济结构的变革将最终改变支配性的意识形态。……因而,大多数人
持有新的更好观点的全新而更好的社会,是有希望存在的。
\end{quotation}

上面这一段话引自于亨特(Hunt)与谢尔曼(Sherman)激进式经济学入门教科书的开篇语。
引语从一个重要方面表明了激进主义者看待经济体的方式。他们认为、正统学者接受大多数
现状;激进主义者希望改变它,而不是接受它。


\subsection{现代激进主义者20世纪的前辈}

20世纪30年代至70年代期间,许多激进主义者的重要著作得到出版,包括莫里斯·多布的《政
治经济学和资本主义》(1937)、琼·罗宾逊的《论马克思主义经济学》(1942)、保罗·斯
威奇(Paul Sweezy)的《资本主义发展理论》(1942),以及保罗·巴兰(Paul Baran)与
保罗·斯威奇的《垄断资本》(1966)。

琼·罗宾逊毫无疑问是最杰出的女性经济学家,她以其令人印象深刻的《不完全竞争经济学》,
意外地作为一名主流经济学家出现在经济学界,此书的出版年份与爱德华·H·张伯伦的《垄断
竞争理论》的出版年份相同(1933)。在这本书中,她在运用边际分析,阐明并扩展马歇尔
关于介于纯粹竞争与纯粹垄断之间市场的提示时,展示了作为一名微观经济理论家的超凡技
能。在凯恩斯的《通论》出版之前的几年里,对张伯伦--罗宾逊不完全竞争市场分析存在着
相当大的关注。来自于剑桥和牛津的经济学家组成了一个小团体,帮助约翰·梅纳德·凯恩斯
发展出后来成为《通论》的思想,作为团体中的重要成员,琼·罗宾逊获得了更多的声
望。1937年,她出版了《就业理论引论》,这是一部杰出的关于凯恩斯思想的介绍性著作。
罗宾逊的学术与政治生涯体现了一种对正统的逐渐远离。她的《论马克思主义经济学》一直
以来都是一部优秀的分析马克思思想的简短著作。20世纪50年代,她提出了一种新的资本分
析理论,该理论否定了大部分的主流新古典资本与边际生产力理论。她进一步远离正统,创
作了一本介绍性的经济学教科书,引在将她的思想传达给更广泛的读者,但是,此书在商业
上并未获得成功。

随着琼·罗宾逊的年龄越来越大,每年对于她有可能获得诺贝尔经济学奖都有相当多的猜测。
当一年又一年过去,而这项荣誉并没有授予她这样一位如此杰出的经济学家时,专业中的很
多人士,从最正统的到非常异端的都感到奇怪。尽管如此,琼·罗宾偿仍然是后凯恩斯主义一
位重要的先驱,并且对现代激进经济学产生了重要影响。

激进的理论家们受到激励的、最具煽动性的著作,可能是巴兰与斯威奇的《垄断资本》
(1966)。这部著作:;(1)将米哈尔·卡莱斯基的一些思想,以及垄断竞争与寡头理论的一
些因素引入到激进经济学中,(2)含蓄地放弃了劳动价值理论。他们认为,随着资本集中——垄
断资本的增强,利润率将随着时间变化而上升。他们说,资本主义时期的危机将会发生,但
不是由于利润率的下隆,而是由于消费不足。他们预言,资本主义对消费不足的反应将会产
生更大的厂商、更浪费的消费,以及为了稳定失败的制度而引起的更多的政府支出。

\subsection{当代激进经济学}

激进经济学在20世纪60年代末期与70年代,凭借自身资格演变为一种思想流派,部分地适应
了因越南战争而引起的社会紧张。1968年,一群年轻的经济学家组成了激进政治经济学联合
会。该组织出版了《激进政治经济学评论》,它是激进经济思想方面的重要杂志,并被
《每月评论》、《科学与社会》以及《剑桥经济学杂志》所增补。尽管激进观点各不相同,
然而,有关新古典经济学和市场导向经济学何处错误的某些观点,将它们连接在一起。根据
艾琳·艾泼鲍姆〈Eileen Applebaum,1977)的看法,激进主义的观点包括以下三种。

\begin{enumerate}[(1)]
\item 激进经济学家认为:“主要的社会经济问题只能通过对社会的激进式调整得到解
  决。”他们认为贫困、种族歧视、性别歧视、环境破坏以及工人异化“并不是制度的病态
  畸形,而是直接源自于资本主义的正常运转”。

\item 激进主义者认为,新古典理论与现实世界经验之间存在着主要矛盾。主流经济学家看
  到社会和谐的地方,激进主义者则看到冲突。

\item 激进主义者根据其传统,将社会看做一个“存在于具体历史环境中的综合的社会制
  度”"。他们认为,主流经济学简单地认同现存制度,例如市场,将其视为既定的,不考虑
  改变这些制度的广泛建议。他们将主流经济学所提倡的增量看做几乎不值得考虑。
\end{enumerate}

从这些观点可以看出,对经济体的激进主义分析与主流分析有着重大的差异。激进主义的前
提是,西方社会中的问题是资本主义制度结构不可避免的结果。激进主义者在其分析中强调,
技术体现了个人之间的社会关系,任何分析都必须研究为什么资本主义会存在,而不是将其
视为既定的。

很多激进主义者通过阶级分析解释了资本主义的存在,他们认为,任何有用的经济学都必须
结合阶级分析。大多数激进主义者也相信,资本主义包含了内在了矛盾,它们将不可避免地
导致制度瓦解,尽管这一过程会因严厉的政府或者诸如学校一类的制度而被放慢,政府是为
了服务于资本家阶级的利益而存在的,制度则是政府的臂膀。

20世纪70年代中期以来,在主流经济争论中,激进主义者发挥了较小的作用,其原因是多方
面的。一些激进主义者转向内部,争论学术性的问题,例如,利润率下降趋势以及转化问题。
但是,20世纪70年代的一些激进作品,对主流分析产生了影响。其中一例是史蒂文·马格林的
《老板做什么》一书。马格林认为,技术不是既定的,而是被社会中特殊的一群人加以选择
的。在资本主义中,这一群体就是管理者或者“老板”",他们选择能给他们提供最大作用的
技术。在陈述他的观点时,马格林重新思考了亚当·斯密关于别针工厂的例子,斯密用这个例
子表明劳动分工的优势。马格林认为,将所有的工人集合在一个屋榴下,老板(组织者)掌
握了对工人的控制,维护了他们自身在生产过程中的作用,使得他们从工人那里榨取更多的
剩余价值。

在主流经济学中获得认可的第二种激进主义观点是涉及教育的经济学。对学校的主流分析主
张,个人投资学校并借此以增加未来收入的形式获得收益。投资使他们和社会变得富裕。为
了表明收益是什么,曾进行过重要的经验研究,在这一研究的基础上,新古典经济学断定,
我们在学校上投资不足。萨缪·鲍尔斯(Samuel Bowels)与赫伯特·金迪斯(Herbert
Gintis)对此并不赞同,他们认为学校不一定提高了社会福利。他们的假设是受过教育的人
的较高收入,有时只不过是被允许进入垄断行业所获得的收益。他们主张,学校不一定提高
了工人的真正价值;它仅仅是提供了一张工会会员证,允许个人进入一系列没有此证就不能进
入的行业。鲍尔斯与金迪斯断言,因为经济计量研究不能区分这两种假设,所以教育对社会
的贡献问题依然悬而未决。

激进思想对主流经济学的另一项侵袭,来自于一位“比较可接受的”激进主义者(他是如此可
接受,以至可以不被看做是激进主义者)。这项侵袭就是迈克尔·皮奥雷(Michael Piore)的
二元劳动市场分析。皮奥雷认为,将劳动市场看做单一市场是错误的,因为主要的结构约束
与社会约束限制了劳动的流动。例如,被录用为运务员(shipping clerk)的工人将发现,
不管他多么有能力,也几乎不可能被提拨到管理者的位置上。他由此得出结论,不同工作的
相对合意性,不一定是根据支付排序的,因为一个具有向上跃迁可能性的职位,最初所支付
的可能少于没有这种可能性的职位。皮奥雷说,因为每项工作是由单独的劳动工种完成的,
所以,将劳动市场看成竞争性的新古典分析并不符合现实。取而代之的是,人们应当将劳动
市场看做结构上受到约束的市场来加以分析,他将这种市场称做二元劳动市场。二元劳动市
场如今已经成为主流凯恩斯分析的一部分。

\section{现代制度主义者、准制度主义者\\以及新制度主义者}

与当代制度主义者相比,19世纪末期20世纪初期到20世纪30年代的制度主义者在经济学中发
挥了比较重要的作用,原因在于,后者参与了实现美国经济中重大的政策变革。作为非主流
非正统的美国经济思想流派,制度主义者的历史是最长的。在第12章中,我们介绍了20世纪早
期制度主义者中的三个主要人物:托尔斯坦·凡勃仑、韦斯利·克莱尔·米切尔以及约翰·R·康芒
斯。他们产生了一种思想流派,它一直持续到现在,广泛地影响了非正统经济学家。并
且,“制度主义者”这一称号经常被加以延伸,不仅仅用来形容这三个人的追随者。由于这
一原因,我们把对制度主义者的描述分成三个部分:(1)遵循凡勃仑、米切尔以及康芒斯思
想的传统制度主义经济学家;(2)我们称之为准制度主义者的人——其观点类似于制度主义者
经济学家,但又过于打破旧习,从而不符合传统制度主义模式;(3)新制度主义者——在新古典
精选理论范围内进行创作的经济学家,但他们认为无论在理论上还是理论的实际应用上,制
度都必须比现在更好地与现实相结合。

\subsection{凡勃仑、米切尔以及康芒斯的传统制度主义追随者}

美国的制度主义在20世纪20年代末30年代初达到巅峰,但是,到了30年代末就已经开始衰落。
克拉伦斯·E·克尔斯(Clarence Ayres,1891--1972)在其《经济进步理论》(1944)中,将
新古典方法对制度主义方法的胜利描述为彻底的。从那时起,制度主义者就处在学科之外:
他们提醒人们注意经济学家不应当忽视处于经济分析范围之外的重要问题,他们只是在这一
方面得到信任。

处于现代经济分析范围之外,并不一定使制度主义者感到受委屈,认识到这一点很重要。正
如我们在导论中论述的那样,主流不等同于权利。制度主义者坚定地认为,经济问题、文化
问题以及社会问题之间的相互作用太大了,因而,孤立地聚焦于推进了大量现代经济思想的
经济力量是没有正当理由的。在评价主流经济学时,他们赞同肯尼斯·伯丁(Kenneth
Boulding),将新古典经济学称为不存在世界的天体力学,他们认为,现代经济学的大多数研
究是精巧的游戏设计。

威斯康星大学曾经是十足制度主义的安身之处,现在则在开设主流课程,其制度主义的主要
残余表现是,它仍坚定地集中于经验研究和问题的复杂性方法。大多数现代制度主义者继续
以凡勃仑、米切尔以及康芒斯的观点为主,并在《经济问题杂志》(Journal of Economic
Issues)上表达这些观点。尽管他们坚持反对主流经济学,却在极大程度上很少被主流所注
意。在第二次世界大战后美国制度主义思想的领袖中,例如,艾伦·格仑奇、华莱士·彼得森以
及克拉伦斯·E·艾尔斯,没有人取得像早期人物那样卓越的成就,毫无疑问,部分原因在于他
们的观点主要是对早期制度主义者观点的详细描述。我们将克拉伦斯·E·艾尔斯作为后凡勃
仑--康芒斯--米切尔的一个例子来简要考察他的观点。

艾尔斯是第二次世界大战后美国最突出的制度主义者。他的大部分学术生涯是在美国奥斯丁
的得克萨斯大学度过的,在威斯康星大学转向主流之后,这里成为了接受制度训练的经济学
家的主要来源地。在其最重要的著作《经济进步理论》(1944)中,艾尔斯接受并详细阐述
了技术性职业与礼仪性行为之间基本的二分法,书中渗透了凡勃仑的大量研究。他通过提供
文化人类学的例子来表明,大多数的商业活动可以类比于愚昧社会的图腾和禁鼠行为,因此
他能够区分促进了他所谓的“生活过程”的事务性技术行为和阻碍“全能力生产”成就的行
为。艾尔斯也坚定地强调凡勃仑的下列观点,即经济学定位于均衡,它是一门非进化的科学,
新古典静态框架需要被一种演进的动态框架所取代。通过借鉴社会科学中的不同领域,他勇
敢地致力于将美国哲学家约翰·杜威所详细论证的概念“工具价值”融入经济学中。马克·图
尔(Marc Tool,1921--)是一位现代制度主义者,他继续遵循着艾尔斯的理论方法。

正如我们已经注意到的,非正统学派经常相互争执。例如,艾尔斯就对庞巴维克所提出的奥
地利学派资本理论进行批评。另一方面,艾尔斯之所以认同凯恩斯的观点,很大程度上是因
为消费不足概念这一元素贯穿于他自己的宏观经济理论。许多其他的制度主义者在凡勃
仑--艾尔斯世系中得以延续。
\clearpage
\section{准制度主义者}

现代制度主义者一直是相对紧密地连接在一起的群体,旨在维持他们之间所进行的对话,并
传播凡勃仑、康芒斯、米切尔,还有他们的追随者的见解。还有另一群经济学家,他们认同
制度主义者的很多见解,并强烈地受到制度主义者的影响。但是,他们过于个人主义和反传
统,因而不符合制度主义的模式。这些人中包括约瑟夫·熊彼特、贡纳·缪达尔
(Hunnar Myrdal,1898--1987)以及约翰·肯尼思·加尔布雷思。

\subsection{约瑟夫·熊彼特}

因为我们在第15章中已经介绍过熊彼特关于资本主义的极富煽动性的观点,所以,我们只需
就其贡献与非正统经济思想的联系简要地加以考察。熊彼特在20世纪30年代早期来到美国,
在哈佛大学任教,那里不大可能是非正统的温床。然而,他与年轻的保罗·斯威奇成为朋友;
尽管熊彼特明显地是一位保守派,但是他承认马克思关于历史变革观点的威力。熊彼特成为
非正统经济学家的一个因素是,他对新古典理论所聚焦的均衡缺乏兴趣。他自己更关心理论
的动态方面,这一点体现在《经济发展理论》(1912)与《经济周期》(1939)中,尤其体现
在他对企业家的描述中,企业家是其所有分析中的一个重要人物。像很多非正统经济学家一
样,熊彼特用粗线条进行描绘,他发现正统理论家的非常抽象的模型过于受到限制。他不断
地表现出处在新古典理论智力边界之外进行观望的非正统倾向,就像他在社会学、历史学以
及政治学的禁猎地偷猎一样。

尽管他坚定地宣称对正统范例感兴趣并给予支持,然而,他却在研究中忽视了他所提倡的惯
例。例如,他热心地支持数学与经济计量学在经济学中发挥更大的作用,但是,他自己的研
究却几乎完全缺乏这些正统工具。他说一事而做另一事这种古怪倾向的另一个例子,能够在
其百科全书式的《经济分析史》(1954)中看到。在导论部分,熊彼特承诺,将呈现经济分
析的历史,并且坚持绝对论者对经济理论发展的解释——现代理论包含一个分析上明确的核心,
它摆脱了价值判断,过去的理论贡献通过现代标准的运用得到解释,并且,因为它们提供了
对现代经济体的更好理解而得到重视。他计划表明怎样存在一种从错误到越来越多真理的稳
定行进。然而,这本书并不是一部经济分析(analysisi)的历史,而是一部经济思想
(thought)的历史。不过,熊彼特是复杂的、多面的,并有自己的主流方式;这一点反映在
下面的事实上,即他对那些开创了现代抽象一般均衡理论的经济学家怀有崇高的敬意。

熊彼特没有形成学派来坚持其经济学,但是,其经济制度与发展的动态方法,在理查德·纳尔
逊(Richard Nelson)与西德尼·温特(Sidney Winter)的著作,以及内森·罗森堡
(Nathan Rosenberg)与小L. E·伯泽尔(L. E. Birdzell)的著作中得到反映。他对企业家行
为的集中研究,也继续存在于I. M·柯兹纳(I. M. Kirzner)、哈维·莱宾斯坦(Harvey
Leibenstein)以及马克·卡森(Mark Casson)的著作中。

\subsection{贡纳·缪达尔}

第二位准制度主义者是贡纳·缪达尔,他是对经济学做出重要贡献的众多瑞典人之一。在这
一领域,纳特·威克塞尔是最著名的,但是,威克塞尔之后,很多人在经济理论发展中具有几
乎相同的地位。我们选择了获得诺贝尔经济学奖的贡纳·纪达尔,不是由于他是典型的瑞典
经济学家,而是由于他清楚地代表了非正统观点。缪达尔是一位国际人物,兴趣促使他研
究世界经济政策问题,虽然在其事业早期他对纯理论的技术问题更感兴趣。他对意识形态与
理论之间关系的经典研究《经济理论发展中的政治因素》(1930)一书,显示了他贯穿于社
会科学与人文学科的兴趣。20世纪40年代初期,他凭借一部关于人口问题的著作和对美国黑
人的重要研究而闯人社会学中。在南部各州,缪达尔因《美国的两难处境:黑人问题和现
代民主》(1944)的出版而扬名,引人注意的是,它包含了第二次世界大战后为黑人赢得更
多公民权利的法律斗争。在他事业的后期,他将注意力放在发达国家与不发达国家的计划上,
作为经济学教授、国会议员、内阁大臣、社会学家以及国际行政人员,他为这项任务带去了
丰富的经验。

缪达尔不满于正统经济理论,然而,他的批评不像凡勃仑、康芒斯或者霍布斯那样尖锐。
在气质上他更像韦斯利·米切尔,他平静地反对,然后使自己忙于手头的任务。他对正统经济
理论的主要批评集中于理论中价值判断的作用、理论的范围与方法以及理论固有的自由放任
偏好。

缪达尔认为,正统理论家创立一种摆脱规范判断的实证科学的尝试已经失败。根据他的观
点,不可能将规范与实证完全分离,从而在缺少应当(oughts)是什么的情况下获得一种分
析。他声称,正统经济学家的尝试仅仅形成了大量观点,其中暗示了规范性的判断,但从未
明确加以表达。缪达尔指出,经济学家感兴趣也应当感兴趣于政策问题,因此,他们对研
究主题与所用方法的选择将必然反映出价值判断。

在最初瑞典语版的《经济理论发展中的政治因素》中他断定,尽管在新理论形成的早期阶段,
意识形态经常与实证理论密切相连,然而,随着时间的变化,规范的或意识形态的因素将被
清除,一种纯粹的、实证的、科学的理论将保留下来。然后,经济学家就能运用这一实证的、
免于价值判断的知识体系,与暗含在任何既定目标中的规范性的价值相协同来制定政策。大
约十五年之后,这本书的英文版获得出版。其前言揭示出在这一重要问题上,缪达尔的观
点已经完全转变:
\begin{quotation}
  存在着不依赖于所有评价而获得的大量科学知识,就我现在所了解的来说,这一含蓄观点
  是天真的经验主义。事实不会仅仅由于被观察到就将自身组织成概念和理论;的确,除了
  概念与理论框架之外,不存在科学的事实,只有混乱。科学研究中存在着一种不可避免的
  演绎因素。在能给出答案之前必须提出问题。问题表达了我们对世界的关注,它们实际上
  就是评价。因此,在我们观察事实并展开理论分析的阶段,而不仅在从事实与评价中得出
  政治推论的阶段,就已然涉及评价。

  因此,我得出关于下列必要性的结论,即从一开始到最后,总下以明确的价值前提来进行
  研究。价值前提不能任意确定;它们对于我们生活的社会来说必须是相关的和有意义的。
\end{quotation}

缪达尔向正统理论提出的第二项批评涉及其范围与方法。和很多其他非正统经济学家一样,
他认为正统理论过于狭隘地界定经济学。缪达尔希望从全部社会科学中,尤其是心理学和社
会学中形成他的分析材料。他也批评经济学聚焦于短期问题,而无论它们是涉及资源配置还
是经济活动波动。缪达尔对涉及经济增长与发展的长期问题更感兴趣,并且认为,正统理论
的大多数分析框架与概念不适合这一任务。缪达尔认为,正统理论定位于均衡,这一点尤其
不适合解释发生在世界各地的经济、社会以及政治变革。他放弃了传统理论的静态均衡分析,
取而代之的是形成了一种累积因果概念。他的累积因果思想实质上是一种一般动态均衡框架,
其中,一般(general)这一术语暗示着除了纯粹的经济因素之外,其他因素也进入分析中。
本章后面将以缪达尔对不发达经济学的分析为例来说明这一思想。

最后,缪达尔就正统理论的如下假设表示了自己的不满,即制度中存在和谐,因此,自由放
任是所有国家应采取的最好政策,而无视它们的经济发展阶段。缪达尔将西方工业化国家的
长期发展过程看做是从重商主义政府控制阶段开始;接着到自由主义与自由放任阶段;再到
福利政治阶段,该阶段中政府在或多或少注重实效的基础上进行干预,以减轻紧迫的社会问
题,并最终向有计划的经济体阶段过渡,一些工业化国家,尤其是美国,尚未达到最后这一
阶段。自由放任终结的标志是日益增多的政府参与和干预,它们基于零碎的基础,没有总体
的协调。按照缪达尔的观点,现在的经验揭示了对经济体的宏观经济目标进行计划,并让市
场和私人企业在极大程度上在这一计划内配置资源的必要性。他说,如果没有总体计划使我
们超越福利国家,我们将会面对以通货膨胀、失业以及支付贸易差额困难为特征的经济体。
缪达尔的计划模型并不是前苏联经济学中的那种,也不像指令性计划那样完全。它假设宏观
经济变量的全国计划,其方式是官僚机构最小化与经济决策制定的最大分散化。纽达尔面向
未来,看到了将计划扩展到国际水平的必要,目的是随大我们进入全球福利社会,工业革命
的成果能够遍及每个人。

缪达尔对不发达国家和世界经济,以及富裕经济的特殊问题感兴趣。在美国,他的很多著作
也为除经济学家之外的人所阅读,这些著作包括《国际经济学》、《富裕国家和贫困国家》
(1957)、《超越福利国家》(1960)、《对富裕的挑战》(1962)、《亚洲的戏剧》
(1968),以及《世界贫困挑战》(1970)。在其对不发达国家的研究中,缪达尔发现,正
统经济理论并不是非常有帮助。在两个主要领域中它是失败的;一方面,当把正统国际贸易
理论用于发展中国家的对外贸易问题时,它给出了错误的答案;另一方面,正统理论似乎没
有能力阐明导致经济增长与发展的国内政策。

我们来考察缪达尔对正统理论进行批评的一个因素,以获得对其方法的认识。缪达尔认为,
富国与穷国之间在实际收入上存在日益扩大的差距。对这一日益扩大的差距,正统经济理论
没有给出令人满意的解释,它也未能提供适当的政策来扭转这些趋势。经济学家使用的定义
过于狭隘,且经济发展模型符合基本的静态均衡模型传统;它们未能抓住塑造经济发展的经
济、社会、政治以及心理因素之间复杂的相互关系。缪达尔认为,为了了解经济发展,对任
何人来说“历史与政治、理论与意识形态、经济结构与水平、社会阶层、农业与工业、人口
发展、健康与教育等,都不能孤立地而必须在它们的相互关系中加以研究。”

正统理论家认为增多的资本形成将导致经济增长,从而断定收入的不平等分配是合意的,原
因在于,它将导致较少的总消费与更多的储蓄和投资。然而,这是一种过于狭窄的关于投资
概念的观点。缪达尔认为,在不发达国家,劳动效率非常低下,部分原因在于与贫困相连的
所有不幸。他主张,劳动阶层所增加的消费,将导致更佳的健康、更高的生产力以及更好的
工作态度。因此,正统经济学家所谓的消费支出,在这种情况下就是一种人力资本投资。
(他们)未能就支出对生产力的影响来对支出进行界定,这成为他们“怀疑西方式的南亚经
济模型有用性的一个原因,这些模型强调产出、就业、储蓄以及投资之间的关系”。

缪达尔研究经济与社会问题大约30年,这使他确信有必要终结自由放任,同时,有必要在国
家与国际两个层面上开展全面的计划项目。他评论说,西欧国家已经完成了与自由不矛盾的
协调的国家计划,虽然计划过程中的一些难题有待解决。缪达尔认为,虽然说不制订计划已
经表现出明显的社会与经济成本,但是,美国尚未认识到为其经济体制订计划的必要性。不
发达国家无法按照西欧国家的方式享受渐进计划制度的奢侈——自由放任,零星的干预,然
后是全国性的计划。缪达尔推论,如果贫穷国家希望刺激其静止、受过去约束的、停滞的经
济来解决人口问题,并极大地提高人均收入,以便促进社会公平这一人们长期渴望的西方理
想,就必须从综合性的国家计划开始。

\subsection{约翰·肯尼恩·加尔布雷思}

约翰·肯尼恩·加尔布雷思是自凡勃仑以来,被一般公众中的知识分子广为研究的第一位美国
经济学家。他出生于加拿大,在伯克利从事研究生学习,其专业是农业经济学。加尔布雷思
经历广泛:他是第二次世界大战期间的政府官员,《财富》杂志的编辑,州与国家层面上民
主党政治家顾问,驻印度大使,哈佛大学经济学教授,美国经济学会会长。作为一位具有非
凡才能的作家,他不仅出版了有关经济学的诸多著作,还出版了经济学范围之外的很多著作。
他的经济学著作面向大众读者;事实上,其若干著作在非小说作品中销量一直是最好的。一
些学术上的同行,恼于他对正统经济理论的批评和他的声望,倾向于将他视为思想模糊的社
会批评家,而不是经济学家。然而,他凭借其一贯的智慧与魅力对这些指责进行了回击,他
认为自己的罪责是由于用清晰的英语进行创造,这使人们能够弄懂他的意思,而不是模仿经
济学家同行令人无法理解的创作方式。

像很多其他非正统作家一样,加尔布雷思只是对公认的经济理论提出批评,却并没有提出一
种定义明确且具有逻辑一致性的替代理论。他很久以前就已放弃对经济学专业进行变革的试
图;他似乎并不在意是否有新的理论结构出现,以符合其不确定的尝试性的表述。同样,他
对美国经济体的分析更关注于其现有的运转,而不是推测其未来的进程:“大体上来说,与
提供材料、思考工业制度已经到达哪里相比,我对断言工业制度将去向何处不太感兴趣。”

为了了解加尔布雷思的观点,我们先简要考察他的三部重要经济学者作:《美国资本主义》
(1952)、《丰裕社会》(1958)以及《新工业国》(1967),以期从中找出统一的主题。

\textbf{\sffamily 抗衡力量}

《美国资本主义》一书是从对正统经济理论的一大段批评开始的。加尔布雷思主张,传统理
论的主要不足在于:(1)其对经济学范围的界定过于狭窄——它没有从事经济与政治权力问题
的研究;(2)关于美国经济体的运转,它得出的结论是不正确的。传统理论的一个主要结论
是,市场上与竞争的任何背离都将导致弱于最佳资源配置的结果。然而,对美国经济体的考
察揭示出,垄断与寡头并不仅仅是正规市场结构或者通常市场结构的失常;更准确地说,它
们是经济体的实质。将正统经济理论应用于当前的经济体中,我们将不得不断定资源没有得
到有效的配置。但是,加尔布雷思声称,经济体运转得相当好,资源并没有经历无效率的配
置。因此,他暗示了一种荒刻的状况:“原则上,经济体没有使任何人感到满意;实际上,
在最近十年中它使大多数人感到满意。”

加尔布雷思接着提供了对美国资本主义的一种新分析,以此来解释为什么(根据正统理论)
当经济体严重失衡时它仍然持续运转。他认为,当竞争开始减弱、经济力量越来越集中在大
公司手中时,新的力量就会出现,以此来抑制或者“抵消”公司的力量:

\begin{quotation}
  实际上,对私人力量的新抑制看来是取代了竞争。它们因削弱或摧毁竞争的相同的集中过
  程而产生。但是,它们似乎并不与市场站在同一方,而是位于相反的一方,不是与竞争者
  一道,而是与消费者或供给者一道。给竞争的这一相似物起个名字将会更方便,我称其为
  抗衡力量(countervailing power)。
\end{quotation}

加尔布雷思说,作为经济体的调节机制,竞争已经被抗衡力量所代替。像竞争一样,搞衡力
量是一种自生的调节力量:在经济体中某一点上出现的力量会产生一种抗衡的力量。接着,
加尔布雷思继续举出这一假设的例子:大公司的成长引起同一行业中强大工会的成长;大制
造商的力量被大零售商的力量所抵消;持续的政府政策促进了抗衡力量的成长。尽管他确定
经济体中抗衡力量不能有效征服经济力量行使的领域,然而加尔布雷思主张,在大多数经济
体中,抗衡力量是一种异常重要的因素。《美国资本主义》使得认同加尔布雷思观点的人对
经济体的运转总体感觉乐观。

加尔布雷思说,正统理论将垄断力量等同为不幸是错误的。抗衡力量的自身特征导致了一种
充满垄断力量,但却为其社会产生福利的经济体。加尔布雷思的“看不见的手”取代了亚当·斯
密的“看不见的手”。他确实注意到了抗衡力量不能运转的一种重要情形:“当市场存在通
货膨胀或此压力时,作为对市场力量的抑制,它根本不起作用。”在这样的时期中,强大的
工会和公司发现“实现某种联合,在较高的价格上传递其协议成本,对双方都有利”。接
下来在评论加尔布雷思的另两部著作之后,我们再回到抗衡力量的概念上来。

\textbf{\sffamily 丰裕社会}

《美国资本主义》一书的基调是乐观的,而《丰裕社会》一书的基调则是混合的。在这本书
中,加尔布雷思扩展了在较早的著作中只做了简述的一些材料,推断经济体中正在发生基本
的资源配置不当。《美国资本主义》集中于私人部门的资源配置效率,而《丰裕社会》则关
注总产量在私人部门与公共部门之闻的分配。加尔布雷思开始对正统理论进行另一项抨击,
他是个擅长创造警句的人,他创造了一个术语用于他所否定的理论:传统智慧
(conventional wisdom)。因为正统价格理论的传统智慧是在当社会关注于提供基本必需品
的时候阐述的,所以,理论集中于稀缺性。然而,对美国经济体的观察表明,在很大程度上
我们已经解决了稀缺问题,市场中的私人部门正在提供不太紧缺的产品。加尔布雷思发现,
随着产品生产的增加,我们对生产更多产品的关注也在增加,这一点引人注意。我们之所以
关注生产或者说崇拜国内生产总值,部分原因在于,不平等的收入分配、个人的不安全感以
及萧条等问题,通过不断增加的产量得到减缓或解决。但是,主要的原因是,消费者的需要
被生产者所操纵,结果使消费者感到非常需要丰裕社会的产品。正统价格理论假设消费者个
人的需要是既定的,它们来自于个人内部;只有独立的消费者才能引导资源配置,以适合他
的需要。加尔布雷思主张,这一理论并不适用于生产者为其产品创造欲望的现代丰裕社会。
加尔布雷思将“需要不断地通过它们被满足的过程加以创造”的过程称为依赖效应
(dependence effect)。

消费者的需要在相当大的程度上是由生产者通过依赖效应创造的,这一主张严重损害了正统
价格理论。实际上,它要求改写整个消费者行为理论,完全推翻消费者主权观念。对生产与
经济增长的关注被视为是被误导的。“如果生产创造了需要,人们就不能为生产辩护,将其
说成是满足了需要。”福利经济学成为一堆废墟。但是,在加尔布雷思的体系中,依赖效应
这一概念的主要目的,是使有关经济体中公共与私人部门适当规模的问题更清楚一些。尽管
消费者经常被唤起对一辆新汽车、一把电动牙刷、一瓶除臭剂的直接需求,这些都能提高其
生活的各个方面,然而,对于公共产品却不存在可比较的依赖效应。这将导致社会不平衡,
因为我们生产并消费了大量高质量的消费者产品和少量劣等的公共产品。加尔布雷思用其最
佳的讽刺文笔使这一点大为增色:

\begin{quotation}
  差别在过去很明显,在现在仍然很明显,且并非只对那些喜欢阅读的人而言。一家人驾驶
  着配有空调、动力方向盘、制动系统的紫红樱桃色小汽车外出旅游,经过路面崎屿不平,
  被垃圾、破旧建筑物、广告牌、电线柱子这些早应埋入地下的东西弄得丑陋不堪的城市。
  他们穿越因商业艺术而在很大程度上无法辨认的乡村(商业艺术所宣传的产品,在我们的
  价值体系中占据绝对的优先地位,从而乡村景色的审美考虑就是第二位的了。对于这样的
  问题,我们意见一致)。他们从便携式的盒中取出包装精美的食物,在一条受到污染的河
  流旁野餐,并在危及公共健康与道德的停车场过夜。他们在尼龙帐篷下的充气床垫上入了
  睡之前,在腐烂废弃物的臭气中,可能还在迷迷糊糊地思考着他们的幸福之路令人惊奇地
  不平坦。难道这真的是美国人的天赋吗?
\end{quotation}

正统理论家们很快就认识到依赖效应概念对价格理论造成的损害,因而并不认同加尔布雷思
的这一论题。然而,加尔布雷思预见到了传统智慧持有者对其观点的否定,并能想象出来,
他说;“抛锚停下来说一些废话,是一件远比在杂乱无章的思想海洋里航行更好的事。”

\textbf{\sffamily 新工业国}

《丰裕社会》一书出版九年之后,加尔布雷思在《新工业国》一书中重新在不平静的思想海
洋上扬帆起航。以其惯有的特性,他又因尖锐的评论而著恼了正统理论家。“经济学的问
题……不是初始错误的问题,而是过时的问题。”加尔布雷思选择诺贝尔奖获得者保罗·萨缪
尔森的初级教科书,作为传统管慧的一个范例。该书从1947年到20世纪70年代起一直支配着
经济学初级教科书市场。那些提供了传统智慧的教科书说“有条理的错误胜于杂乱的真理”。
在《新工业国》中,加尔布雷思提出了新问题并且得出了关于美国资本主义的新结论。在他
以前的著作中,在论述依赖效应时他指出正统需求理论是不正确的。在《新工业国》中,他
通过批评厂商行为与供给理论,完善了对正统价格理论的批评。然后,他将这些结合在一起,
表明对市场过程的正统描述在很大程度上是错误的。

现代技术的应用需要大规模的企业。随着这些企业的发展壮大,出现了所有权与控制权的分
离,那些控制企业的人就是被支付工资的管理者,他们组成社会技术结构阶层的一部分。为
了规避风险和消除不确定性,企业怂恿政府稳定经济;它们与工会合作;它们尽可能多地用
未分配的利润进行投资,但最重要的是,它们控制着消费者的偏好。尽管这涉及制订计划,
然而,企业并不像正统价格理论所假定的那样,为了使利润最大化而制订计划;它们的主要
目标是企业运转或生存的连续性。企业一旦获得了这种安全感,它就会开始考虑规模的增长。
因此,加尔布雷思扩张并放大了其依赖效应这一概念——需要通过它们被满足的过程得以创
造——来表明:(1)技术与大规模企业的成长,提出了最低风险与不确定性这种经济秩序的必
要性;(2)包括消费者偏好管理在内的计划,如今是经济体的一个基本组成部分。正统理论
声称,市场通过主权消费者得以运转,这些消费者借助市场价格来指挥利润最大化的企业。
加尔布雷思将这一荒诞说法称为“公认的顺序”。他提出,在公司大而强的市场中“生产厂
商向前延伸以控制市场,并且进一步管理它表面上所服务的那些人的市场行为,塑造他们的
社会态度。”他将这种更为准确的描述称为“修正的顺序”。

下面我们来陈述加尔布雷思对正统价格理论的批评要点。他主张,正统价格理论的消费者与
厂商行为理论是错误的,当厂商大而强时,厂商对家庭的引导做出反应的观点是完全不准确
的。如果加尔布雷思是对的,那么由此可以得出结论,正统理论的政策建议并没有什么基础,
尤其是因为自由放任所以存在最佳资源配置的观点:
\begin{quotation}
  一旦赞同无论如何个人都面对规制——一旦允许存在修正的顺序——那么,使个人免受(假如
  说)政府干预的情形就会消失。受到保护的不是个人的购买权利,而是销售者管理个人的
  权利。
\end{quotation}

加尔布雷斯进一步主张,从新工业国所得出的不合意的后果之一是,我们的社会态度被技术
结构阶层所塑造。技术结构阶层生产产品并将产量视同为社会福利,因而使其在社会中的角
色合理化,并为自身设定了一种国家目的。国家支持技术结构阶层推进下列社会态度,即赞
美经济体所生产的产品数量,并与社会生活质量对立。我们的教育制度在一定程度上已经加
入对国内生产总值的崇拜,尽管仍然存在如下可能性,即就国家为生产所困扰而言,教育家
和一些人所感觉到的不安,将会在对经济与社会方向批判性的重新评价中体现出来。“自由
的危险在于将信仰置于工业制度需要之下。”

对加尔布雷思上述思想的总体观察也揭示出一些矛盾。《美国资本主义》反映出对资本主义
未来基本乐观的看法,原因在于,能够期望抗衡力量导致经济体的合理效率。大公司的力量
不一定是不合意的,如果它能够被抵消的话。尽管我们不能明确地了解示来,然而,加尔布
雷思在《丰裕社会》和《新工业国》中所勾画的景象是黑暗与阴沉的。他在这些著作中暗示,
尽管技术发展有可能解决生产与稀缺问题,但是,我们现在正处于成为工业制度奴隶而不是
其主人的极大危险中。如果知识分子反省这些问题,并且成为一种力量,使社会由关注更多
生产转向关注更高生活质量,则仍然存在一些希望。但是,具体将会发生什么尚有待观察:

\begin{quotation}
  如果教育制度通常服务于工业制度的信念,那么,后者的影响及其庞大而坚实的特性将会
  增强。出于同样的原因,如果它不服从于且独立于工业制度,那么,它就可能是怀疑论、
  解放论以及多元主义的必要力量。
\end{quotation}

\begin{mybox}{现代准制度主义者与社会经济学}
  \begin{multicols}{2}
    阅读熊彼特、缪达尔以及加尔布雷思著作的大多数人,其至那些不赞同他们政策建议的
    人,在这些经济学家所写的内容中,都会发现较强的判断力,正如早先的读者在传统制
    度主义中发现较强的判断力一样。然而,较强的判断力并不一定产生影响与变革,说他
    们的思想对经济学专业几乎没有什么影响也是公平的。尽管如此,较强的判断力是难以
    抑制的;今天,有很多准制度主义者,关于这些人,未来的思想史家极有可能问到:为
    什么他们所具有的影响这么小?这其中的一个群体松散地组织在“社会经济学”的旗帜
    下。它由阿米泰·埃奇奥尼所组织,并拥有自己的杂志《社会经济学杂志》。像制度主义
    者一样,社会经济学家认为,必须有力地将社会力量整合为经济模型。他们为效用函数
    打下了一个更加复杂的心理基础,其中,人们不只简单地被看做利己的利润最大化者。

    社会经济学家赞同价值上的共产主义方法。他们的理论认为,个人对集体的关心以及私
    利支配着其行为,因而政策需要用以建设集体。
  \end{multicols}

\end{mybox}

\section{新制度主义者}

新古典经济学不考虑制度,或者更准确地说,为了使可利用的数学方法起作用,假定其所需
的制度。它最初导致了静态分析的运用,然后是比较静态分析的运用,接着是微分,随后是
集合论、测量论以及最佳控制论。新古典经济学一个引人注意的方面是,在某种程度上,方
法推动了它所从事的问题以及它所找到的答案。

经济学极不可能明确地将制度包括进去,简单的原因在于,制度分析是杂乱的,而经济学中
的探索是为了获得适合现有方法的优雅的基本关系。然而,避免进行明确的制度分析并不能
使经济学脱离制度:新古典经济学在其基本的结构模型中包含着多种关于制度的隐式假设。
例如,考虑一下“厂商”这个新古典经济学的生产单位。它由很多个人组成并且相当复杂,
但是新古典经济学将其目标归结为单一目标——利润最大化——却没有解决这一目标如何能与厂
商内部个人的效用最大化相一致。例如,管理者与其他雇员会在损害利润的情况下从事对他
们有益的活动吗?这一点也适用于市场:新古典经济学假设存在具有明确数学特性的特殊类
型的市场,它并没有解释这样的市场是如何产生的,它们可能会怎样变化,它们的存在是否
会影响个人的行为与偏好,或者这些市场是否与我们在现实世界中看到的市场很接近。因此,
新古典经济学的关注点非常狭隘。

如此狭隘的关注点以及理论上的简单化排除了经济学批评家们所提出的很多问题,因此,与
新古典经济学家相比,作为一贯的社会批评家的非正统经济学家,更偏重于明确的制度分
析。

一些新古典经济学家认为,必须解决制度杂乱问题,他们建议在新古典框架内做这件事。与
通常的新古典经济学家相比,这些“新制度主义者”在他们的理论模型中包含了更多的制度
细节,但是,他们保留了传统新古典模型中的个人最大化过程。在他们的分析中,交易成本
发挥着重要作用。对于这些新制度主义者来说,罗纳德·科斯关于企业理论的文章(1937)是
一篇开创性的文章。他提出企业的发展是因为市场交易成本对于企业间的交易来说过高。

新制度主义有时也被称作寻租分析或者新古典政治经济学。其支持者主张,理性的个人不仅
在既定的制度结构内,而且通过改变制度来试图增进其福利。他们认为,经济分析必须包括
对制度结构决定力量的考虑。均衡制度结构是指对个人来说,不值得花费更多精力去改变制
度的一种结构。他们说,只有在均衡制度结构基础上才能进行相关的分析。这些新制度主义
者主张,竞争性的制度结构是不稳定的,原因在于,一些个人有强大的动机去改变制度结构,
使之有益于自己,而这种动机并不能被支持竞争性结构的动机所抵消。因此,说新古典经济
学是不相关的,并不是因为其最大化的假设,而是因为其假定的制度结构不是均衡的制度结
构。最大化假设并没有得到相当充分的支持。这些观点与原始制度主义者的少数残余追随者
的观点不同,只在经济学专业中引起了轻微的关注。奥利弗·威廉姆森对企业的研究符合我们
所谓的“新制度主义者”模式。

新制度主义者已经在经济学专业中获得了声望。20世纪90年代,他们开始建立“新兴制度主
义者”组织,其中包括自己的杂志以及一项积极的研究议程。这一群体包括很多有名的经济
学家,它在很多方面已经成为现代主流经济学的组成,而不是一个非正统群体。

\section{后凯恩斯主义者}

在对宏观经济学进行考察时我们看到,主流宏观经济学只遵循着凯思斯著作中很多思路中的
一条。这种状况部分原因在于,对于凯恩斯就宏观经济运转到底说的是什么,经济学家通常
无法取得一致意见。因此,20世纪70年代,最初由美国的西德尼·温特劳布(Sidney
Weintraub)与保罗·戴维森(PaulDavidson,1930--),以及英国的琼·罗宾逊与约翰·伊特
韦尔(John Eatwell)领导的一群人联合起来对主流新凯恩斯模型进行批评,这一批评非常
明确,足以使得人们,也使得这些经济学家将其自身视为一种经济学流派。他们将自己称作
后凯恩斯主义者(post--Keynesians),并在1974年召开了一次组织会议,会上创办了他们
的出版物一一《后凯恩斯经济学杂志》(Journal of Post--Kaynesian Economics,写
为JPKE)。在杂志的创刊号上,不同创始人和支持者试图表明后凯恩斯经济和学对他们来说
意味着什么。琼·罗宾逊将其称作“考虑到未来与过去之间差异的一种分析方法”;加尔布雷
思说,后凯恩斯经济学认为“工业社会处于一个连续的系统的变化过程中,公共政策必须适
应这种变化,实际上,通过这些公共行为,能够提高绩效”。其他经济学家则聚焦于不同的
问题,但是所有人都认为新古典与新凯恩斯经济学是不同的。因此,他们逐渐将其自身
视为凯恩斯信仰的真正维护者,并将主流宏观经济学称为“杂乱的凯恩斯主义”。

\subsection{英国的后凯恩斯主义者}

有关大多数后凯恩斯主义分析基本观念的一般性表述,并没有被证明存在特别问题,但是从
这些表述中得出的细节却存在问题。英国后凯恩斯主义者(有时被称作新李嘉图主义
者neo--Ricardians)认为,正确的方法是回归李嘉图的生产理论,并用一种卡莱斯基类型的
经济周期理论予以补充。他们按照皮埃罗·斯拉法在《用商品生产商品:经济理论批评绪论》
(1960)中的研究,主张工资收入与利润收入的区分是模糊的,且不受总产量的约束。因此,
收入分配并不是由边际生产力而是由其他力量决定的,这些力量本质上属于宏观经济。在这
一观点上,他们采用了一个与米哈尔·卡莱斯基1933年所提出的模型类似的模型,卡莱斯基用
下列表述来概括其模型,即工人支出他们所得到的,资本家得到他们所支出的。

卡莱斯基在其模型中做了三个重要假设。第一,他假设厂商采用成本加成定价方法。资本家
确定利润率与工资率,而不是总利润或工资总水平,因为它们是由产量总水平决定的。第二,
储蓄没有被转换成支出,所以,产量总水平是由凯恩斯乘数模式中总需求水平决定的。第三,
工人将其收人的百分之百都用来支出,因此,他们的边际消费倾向为百分之百。

资本家的投资支出倾向于具有任意性,与其利润水平(由储蓄组成)无关。如果他们将其全
部利润用来支出,那么,需求就足以多到购买全部产品,总产量与利润将会很高。如果资本
家变得悲观,其利润不是用于支出而是用于储蓄,那么,总需求与总产量将会较低,利润也
将较低(尽管利润率保持不变),失业就会跟着发生。因此,工资与利润之间的收入分配是
由宏观力量而不是由边际生产力决定的。这一模型中的大多数假设能够被修正,从而使得结
论更加不明确,但这并没有使下列总体看法失效,即宏观经济活动水平是收入分配的决定性
因素。
\clearpage
\subsection{美国的后凯恩斯主义者}

与英国后凯恩斯主义相比,美国的后凯恩斯主义理论分支更加不集中,但是,其所有成分都
是下列主题的变种,即经济体是“即时的"。阿尔弗雷德·艾克纳(Alfred Fichner)扩展了
对他所谓的巨型公司厂商的微观经济分析,认为这种厂商内在地决定净利润投资。因此,为
了了解投资——从而了解总产量——人们必须了解现代公司。

在《货币与真实世界》中保罗·戴维森主张,了解货币的作用是了解宏观经济如何运转的关键,
新古典经济学没有充分地考察货币的作用。在形成有关货币作用的后凯恩斯观点过程中,他
强调存在着不可遂时期(irreversible time)与实际不确定性(true uncertainty),它们
不能被还原为一种概率分布,从而不能被转变为风险继而被转变为确定性的对应物。经济体
的这两个相关特性,“使得人们去形成某些制度和游戏规则,例如,(1)货币,(2)货币
合约与一套法律执行制度,(3)黏性货币工资率,(4)即期外汇与远期外汇市场。”因此,
制度改变了宏观经济运行方式。戴维森的观点稍与海曼·明斯基(Hyman
Minsky,1919--1997)的观点类似,后者是另一位著名的后凯恩斯主义者,他认为金融制度
就像纸牌做的房子,处于即将倒塌的危险中。

后凯恩斯增长理论强调方法问题,因此它是一个观察角度的问题。它强调增长是经济过程的
一个重要方面,而直到最近主流经济学家还在强调静态问题。例如,罗伊·哈罗德与埃弗
塞·多马对增长的分析,在20世纪50年代是主流宏观经济学的基本组成部分,现在已很少出现
在主流中级宏观经济学教科书中。新教科书讨论稳定内生增长。这部分归因于后凯恩斯主义
对不稳定性的关注,因为哈罗德--多马模型提出,经济体中的均衡总是处于经济繁荣与萧条
交替边界的紧要关头。

从总体上上看,人们在后凯恩斯主义研究中能够看到概念的连贯性,而不是模型的一致性。
后凯恩斯主义一个持久的观念是经济体是不稳定的:市场这只“看不见的手”并不像新古典
理论所表明的那样充分起作用。则此得出的结论是,与正统理论相比,后凯恩斯主义认识到
了政府行为在纠正资本主义问题中更加突出的作用。后凯恩斯主义者最为知名的是他们对以
税收为基础的收入政策的支持。

\subsection{对后凯恩斯主义者的主流回应}

主流经济学家对美国后凯恩斯主义者的回应始终是完全置之不理,或都采取反问“还有什么
新东西”的态度。罗伯特·索洛总结如下:

\begin{quotation}
  我并不很同情自称为后凯恩斯主义者的流派。首先,我从未能够将其作为一种思想流派来
  加以了解。一方面是海曼·明斯基(他碰巧是我最老的朋友之一),另一方面是像阿尔弗雷
  德·艾克纳一类的人,我看不到他们之间存在什么学术联系,除了他们都反对同一件事(即
  主流而无论主流是什么)之外。

  我不同情他们的另一个理由是,我从未能够(我必须承认我从未非常努力地尝试过)拼凑
  出某种实证学说。通常似乎只有该团体才知道它在反对什么,但并不提供能够被看做一种
  实证理论的非常系统的东西。我读过保罗·戴维森的很多文章,这些文章对我来说通常没有
  什么意义。某些后凯斯主义的价格理论源自于下列信念,即普遍的竞争是一个有害的假设。
  我一生都了解普遍的竞争。所以,我发现这是一种不值得从事的方法,因此没有过多地予
  以注意。
\end{quotation}

主流经济学家也认为“不包含在正统理论所形成的一系列主张中的正确的新李嘉图主张,是
不存在的”。

\section{公共选择提倡者}

经济学家假定个人在经济事务上是理性的;为什么不假定他们在其他事务上也是理性的呢?
这就是詹姆斯·布坎南(James Buchanan)与戈登·塔洛克(Gordon Tullock)在20世纪50年
代早期所提出的问题,并因此开启了公共选择学派(public choice school)。塔洛克与布
坎南于20世纪60年代离开弗吉尼亚大学,部分原因是基于他们的非正统政策观点,他们在弗
吉尼亚工学院组建了公共选择研究中心,并于1983年将该中心迁至乔治梅森大学。

公共选择学派的中心思想是,个人与政府发生作用时,就像他们处理经济事务时一样,是富
有理性的。政府并不是一个或好或坏的机构,它只不过是个人通过政治实现其经济目标的中
介。公共选择理论家设计出了一种有关政治的经济理论。运用古典与新古典理论建立家庭与
厂商行为模型时所运用的框架,他们分析了政治选择或者说公共选择。

公共选择理论家的重要见解,有时会因许多公共选择拥护者的反统制观点而变得模糊,正如
像加尔布雷思一类的经济学家的见解,会因他们的亲统制观点而变得模糊一样。对双方来说
这都是不幸的。就我们对政策问题与经济理论的理解而言,公共选择理论家做出了重要贡献。
持各种政治信仰的经济学家都赞同政府失灵(govemment failures)——不能实现社会利益的
政府政策——是存在的,必须连同市场失灵(marketfailures)一起包含在我们对政策的分析
中。在我们提到的所有批评性思想流派中,公共选择学派是最成功的,他们对寻租行为的分
析已经扩散到主流中。很多具有公共选择风格的介绍性教科书广泛地被采用。詹姆斯·布坎南
于1986年获得诺贝尔经济学奖,这也反映了来自主流的认同。然而,主流经济学家对于完全
接受公共选择理论在很大程度上仍然犹豫不决。


\section{奥地利经济学派}

在第8章中我们考察了卡尔·门格尔在边际效用理论早期发展过程中所扮演的角色。后来,我
们着眼于门格尔的一些学生,例如庞巴维克与维色。他们的学生以及他们学生的学生创立了
一种连贯的组织良好的经济学方法,并逐渐以“新奥地利学派”或者简单地说以奥地利学派
而闻名。奥地利经济学家与主流产生分歧的大部分原因,与后凯恩斯主义者是一样的——他们
认为,经济学的形式化已经丧失或者放弃了早期作家的很多见解。到1960年为止,奥地利经
济学(Austrian economics)都被视为主流的组成部分;但是,随着新古典经济学的衰退,主
流经济学选择形式化的模型构建,奥地利学者便作为持不同意见者再度出现。

这并不是说他们没有分歧;即使在当时,他们与主流之间也有实质上的分歧。例如,奥地利
学者的生产分析,将资本视为只有在生产分析过程中才能被理解的中间产品。同样,奥地利
学者坚定地坚持将个人视为有目的的行动者,而不是对快乐与痛苦做出反应的功利主义机器。
这一点在一定程度上导致奥地利学者突出强调企业家精神。他们也提出一种不同的成本方法,
将成本看做主观个别决定,而不像在古典学派和新古典成本分析的某些阐释中那样,将成本
视为客观决定。

这些分歧尽管是实质上的,但在20世纪60年代之前,并没有因此将奥地利学者置于主流之外。
然而,20世纪60年代,随着(1)经济学越来越形式化;(2)主流几乎完全被一般均衡理论
所主导;(3)主流经济学将自身视为一门完全由模型构建和经济计量检验来决定真理的科学
这一倾向日益严重,奥地利经济学脱离了主流。目前被称为新奥地利经济学家的新近的几代
人,尤其是路德维希·冯·米塞斯与弗里德里希·冯·哈耶克的学生——莫瑞·罗斯巴德
(Murray Rothbard)、伊斯雷尔·柯兹纳(lsrael Kirzner)、路德维希·拉赫曼
(Ludwig Lachmann)——都认为门格尔的很多有价值的见解已经不为人知。

奥地利经济学的一个主要经济论题是,经济分析是一个过程而不是一种个人的静态相互作用,
时间是基本的考虑因素。它将竞争看做一个动态过程,在这一过程中高利润随着时间的变化
而消失。但是,这些高利润在推进制度方面起着非常重要的作用。在奥地利经济学中,个人
被假定为在变化的环境下进行操作,信息是有限的,未来是不确定的。按照他们的观点,最
引人注意的分析并不来自于研究均衡本身,而是研究个人探索均衡的过程,该过程强调企业
家的作用,新古典经济学将这一过程称做非均衡。

直到最近,在奥地利经济学中,仍然存在着强烈的政治暗示。要找到一位不是保守派的奥地
利学者始终比较困难;大多数人都简单地假设,对于获得个人自由来说,市场是合意的和必
要的。然而,很多奥地利学者自身并不将他们的政治观点描述为“保守的",而是说成是“激
进的自由主义”或者“反统制论"。他们认为,这样的看法是从对历史的研究中自然而然得出
的。

有时,奥地利经济学家反对从经验上证明经济原理的经济计量研究与尝试。遵照米塞斯
的“人类行为学”",他们觉得自己的任务就是从人类行为的逻辑中演绎性地得出结论。按照
他们的观点,这样得出的结论与理论不需要经过检验,因为真理已经符合逻辑地确定了。然
而近来,他们采取了一种稍微调和的观点,认为主流经济学所从事的经验研究类型——不包括
历史因素与启发式因素——是不适当的。

奥地利学者的文献中重要的开创性著作,是哈耶克1937年发表在《经济学》上的论文《经济
学与知识》,以及1945年发表在《美国经济评论》上的论文“知识在社会中的运用”。哈耶
克提出了下列法律问题,即所假定的市场参与者在均衡条件下所拥有的知识,是如何为参与
者所获得的。新古典理论假定知识是既定的。哈耶克认为,市场与竞争过程的一个重要作用,
就是发现先前不可利用的知识。哈耶克主张,均衡是所有代理人的计划都同步的一种情况;
因此,知识、预期以及信念是任何经济分析的重要成分。由于存在不确定性,个人计划的协
调较为困难,超出了单个人的理解范围。只有通过经由市场发展起来的自然秩序
(spontaneous order),制度才会运转。哈耶克的政策观点,是从他对待知识与不确定性的
态度中得出的,这种态度即我们并不了解自身行为的最终影响。因此,我们应当接受本能地
发展起来的制度,尤其是市场,与政治过程相比,市场更加有效地且有力地解决了我们的经
济问题。

尽管很多主流经济学家似乎愿意就现存制度与不确定性的重要性,对奥地利学者表示认同,
并且,不确定性的重要性问题使形式化的建模与经验研究变得困难,然而,主流经济学家仍
然认为奥地利学者:(1)过分强调了难度;(2)没有形成一种可接受的替代方案;(3)允
许价值判断蔓延进他们的启发式分析中。


奥地利经济学家因自身的非唯科学性而感到自豪。他们认为,经济学的适当方法不应当是应
用自然科学的原理来研究人类行为。他们对基本(内在)的洞察力比对令人印象深刻的外观
更感兴趣,后者即他们所说的唯科学的(与科学相对应)。哈耶克是这一术语最坚决的倡导
者,他在1974年获得了诺灵东经济学奖,他的作品由经济学演进到对社会基本法律与宪法结
构的研究。因此,他考察了制度主义者所考察的很多相同问题,并且使这一研究具有某种历
史意义和经济学意义。

\section{其它非正统经济学流派}

如果不提及现存的很多其它流派,例如,女权论经济学家、黑人经济学家以及自由论者经
济学家,我们对非正统经济学家的论述就是不完全的。我们没有将他们作为单独的流派列出
来,是因为他们通常会在某些方面符合其它流派,关注着很多相同的问题。此外,他们的关
注点不太具有一般性,而是更多地指向具体问题。自由论者经济学家关注自由与市场的道德
准则,黑人经济学家关注影响黑人的分配与平等问题,女权论者经济学家则关注影响妇女的
分配与平等问题。

黑人经济学家与女权论者经济学家都指出,经济学趋向于成为白人男性的专业,其组成影响
到其研究内容。我们来考察女权论者经济学,以此为例来看这些流派所提出的问题。女权论
者认为,女性与男性在处理问题时可能存在差异,因此,人们的分析应当考虑到这种可能性。
他们问道:为什么女性被分派充当家庭主妇而非养家糊口者的角色?正统答案是竞争优势,
但是,女权论者认为可能涉及制度歧视。这类分析也能被用来发展歧视模型。一些最主要的
女权论者经济学家致力于研究这些问题,包括朱莉·尼尔森(Julie Nelson)、芭芭拉·博格
有(Barbara Bergman)以及玛丽安·费伯(Marianne Ferber)。呈现这一领域研究的出版物
是《女权论者经济学杂志》。

当然,不是所有的女性经济学家都是女权论者经济学家,大多数都不过是与男性经济学家运用
相同方式处理问题的女性经济学家。研究生院充当着一种选择机制,表明情况确实如此。很
多女性和男性都认为,女性与男性之间的差异还没有大到足以有理由运用一种单独的方法的
程度。

\section{总结}

除了反正统之外,非正统经济学家几乎没有什么共同点。虽然他们以不同的方式表明对正统
的反对,但是,就正统理论的范围、方法以及内容而言,他们的意见也不统一。激进主义者、
制度主义者以及后凯恩斯主义者否定下列正统观点,即和谐支配着市场经济,因而自由放任
是一种适当的政府政策。公共选择提倡者与新奥地利学派倾向于成为主流经济学的政治右翼,
他们担忧市场中那些为正统理论所接受的政府干预程度。无论是主流的左翼还是右翼,非主
流经济学家的不同意见通常与伦理有关,也与科学性有关。

非正统流派之间存在着引人注意的差异与对比。第一,尽管就主流经济学的缺点而言,他们
之间经常意见不一致,然而,他们几乎总是赞同有必要扩展主流分析的范围。例如,尽管公
共选择理论家与激进主义者在政治系列上属于对立的两方,但是,他们都赞同政治与经济学
不能分离。第二,尽管非正统经济学家经常为主流所忽视,但是,他们仍然影响着主流。随
着他们对主流的影响,以及随着他们的观点有时被并入主流,他们作为非正统经济学家的角
色就被淡化。因此,长命不一定是非正统思想的一个绝对特征。第三,非正统经济学家具有
面朝自身内部,使自身脱离专业的倾向——在这种情况下,他们的分析成为一个单独研究领域,
要么完全取代主流经济学,要么独立于主流继续存在。第四,几乎所有的非正统流派都是偏
袒的;对于一个流派来说,要想对理论产生重要影响,就必须是无偏祖的,既不与左翼也不
与右翼联合。

公共选择理论似乎更接近于为主流所吸收。随着更多自由主义经济学家对寻租分析的发展,
以及新古典工具的运用,公共选择学派呈现出将其观点融入经济学专业的最大潜力。新奥地
利学派不大可能被吸收。然而,他们或许能够继续奋斗,主要原因是他们可以获得大量资助,
用以提供出版渠道和其他手段,借此影响经济思想。激进主义者发现自身处境较为困难:他
们获得的外部资助较少,因而只有很少的出版渠道,同时,他们的一些最佳思想则被并
人“宽泛解释”的主流理论中。如果没有外部政治力量使人们激进起来,他们就不大可能对
主流经济学产生重要影响。制度主义者遵循着不随俗的路线。他们与主流专业几乎没有什么
接触,要求也少,虽然通过新制度主义者的分析,他们的一些见解逐渐渗透到主流经济学中。
这种情况在很大程度上对后凯恩斯主义者也适用,虽然他们是一个变化非常多的群体,但是
他们中的一些人确实在主流专业中发挥着比较积极的作用。

一些评论者断言,由于非正统理论的独特看法未能取代正统理论,所以非正统理论已经失败
了。由于这一原因,非正统理论经常被经济理论中所遗漏。我们的观点有所不同。对非正统
思想的考察表明,尽管它没有取代经济思想公认的潮流,然而,它经常促使主流理论进入新
的路线,有时还提供开创性的思想,这些思想注定会成为公认的理论结构的组成部分。对思
想潮流方向与内容的这些贡献,是不能被忽视的。这些思想极有可能被稍后21世纪的历史学
家当做主流思想先驱予以追忆。

%%% Local Variables:
%%% mode: latex
%%% TeX-master: "../../main"
%%% End:
