\chapter{现代宏观经济思想的发展}

\begin{quotation}
  凯恩斯说,一些东西是新的,一些东西是对的;不幸的是,新的东西不对,对的东西不新。
  \raggedleft ——弗兰克·H·奈特
\end{quotation}

多少年来,人们对宏观经济问题的兴趣起起落落,大约在19世纪末期达到最低点。当时经济
学专业对宏观经济思想的态度,可以被描述为善意的忽视。此外,确实存在的宏观经济思想
也有些混乱。马歇尔系统编纂组织了宏观经济学,并一直打算也这样对待宏观经济学,但他
未能做到。马歇尔将其对宏观经济学的论述局限于价格总水平的确定,F.W·陶西格在其介绍
性的教科书中也是如此。

曾经为亚当·斯密所关注的增长,在古典时期与新古典时期的后期,只获得了轻微的重视。
经济学专业反倒是利用李嘉图提倡的静态推论,集中于开发形式化的配置与分配模型;斯密
模棱两可的解释,败给了李嘉图更加形式化的模型。经济周期也只是短暂地被提及;全部资
源都能被充分利用的标准假定,排除了对这些问题的更多考虑。充分就业的假定经常通过引
用萨伊定律即供给创造自身的需求而被证明是正确的。

利用充分就业假定,并集中解释价格总水平决定力量的分析,一直持续到20世纪30年代,当
时的大萧条引发了关于经济周期的新研究。从20世纪30年代到70年代后期,宏观经济学持续
持续关注经济周期,并成为以“凯恩斯经济学”而知名的一种方法。这种分类并不完全正确,
因为凯恩斯的思想很快就与新古典思想相融合。教科书中详尽阐述的实际的宏观经济学,被
称作新凯恩斯经济学可能更加合适。本章描述了这一演进及其历史基础。

20世纪70年代见证了对新凯恩斯经济学的反抗,其形式是新的古典革命,它将宏观经济学的
关注点由经济周期转向增长。从20世纪90年代开始,宏观经济学前沿的主要关注点一直在增
长上。

本章将首先考察关于宏观经济问题的早期研究,然后再论及凯恩斯宏
观经济学的发展,接着介绍新的古典革命,最后考察宏观经济学现在的
状况。

\section{现代宏观经济学的历史先驱者}

现代宏观经济学主要由货币理论、增长理论以及经济周期理论组成。多年来,对这些问题的
强调一直在变动,部分原因是经济体的历程发生了变化,部分原因是技术手段允许经济学家
去涉及先前认为难以操纵的问题。我们从对增长理论的论述开始。

\subsection{增长理论的早期研究}

经济增长分析是亚当·斯密的主要关注点,他强调自由市场、私人投资支出、自由放任与经济
增长之间的关系。李嘉图重新调整了经济学的关注点,将其从经济增长转向收入分配的决定
力量。斯密与李嘉图之间在经济学本质主题这一观点上的变化,从根本上说是对经济学的重
新定位,使之远离斯密增长的宏观经济学,转向李嘉图的微观经济学事务——什么决定了工资、
地租、利润和其他价格,以及因此而来的收入分配。对微观经济学及分配问题的重视,
在19世纪第一个二十五年中从李嘉图开始,持续支配着主流经济思想,直到20世纪30年代吞
没了工业化国家的大萧条时期为止。

约瑟夫·熊彼特在其著名的关于经济思想史的著作中,在对增长进行论述时,根据经济学家对
增长的看法,将他们做了两种类型的区分乐观主义者与悲观主义者。他认为,最为主流的经
济学家属于悲观派,最强硬的悲观主义者有马尔萨斯、李嘉图以及詹姆斯·穆勒。这些主流经
济学家坚定地强调收益递减、持续提高的地租以及经济体将近进的静止状态。即使他们身边
的经济体以远快于较早时期的速度在增长,他们也这样认为。正如熊彼特所指出的:“他们
确信技术进步与资本增加,最终也不能对抗致命的收益递减法则。”

在主要的经济学家中,在这点上有些例外的是约翰·斯图亚特·穆勒,与马尔萨斯或李嘉图比
起来,他更多地讨论了增长与技术问题并且他对持续增长的可能性更加乐观。但是,通过对
穆勒著作的仔细阅读可以发现,与其说他的思想建立在技术与资本的持续增长基础上,不如
说建立在下列信念基础上,即社会最终将自动地限制出生率,从而使不可避免的边际收益
递减放慢。

晚年时,穆勒更多地是一个悲观主义者。他似乎确信静止状态就在不远处。然而,他没有将
这一结果看做是坏事。他宁愿将静止状态看成是一种具有适度繁荣与合理平等的舒适状态。
这起因于他把收入分配视为由社会以及经济力量决定的。

采取乐观主义观点的是剩下的非正统经济学家,例如,亨利·凯里(Henry
Carey,1793--1879)和弗里德里希·李斯特。在第12章中已经论述过李斯特,他是德国历史学
派的一分子,该学派强调理论的经验观察与历史。因为他能够看到经济体正以比先前更快的
速度增长,所以对他来说,认为增长有可能无限期地持续下去是很自然的。凯里是一位美国
经济学家,他不强调理论,而强调历史与经验观察。这使他得出与李斯特同样的结论:经济
体的增长似乎看不到终结。考虑到当时美国所经历的边境扩张与农业用地日益增加,在美国
的环境中不太强调收益递减就很自然了。

值得注意的是,李斯特与凯里这样的乐观主义者拥护关税,而李嘉图一类的悲观主义者通常
支持自由贸易。这种差异可能源自于他们的理论观点以及对假设的运用。李嘉图理论上的比
较优势模型,引导人们思考自由贸易政策的利益。但是,模型的静态性质也导致了下列观点,
即一旦获得了贸易收益,增长将会停止。李斯特与凯里较少关注理论,更多关注观察和历史。
对经济体的直接观察表明了技术的重要性与持续增长的可能性。它也暗示通过关税来保护技
术是很重要的。斯密认为,贸易通过扩大劳动分工和实践学习扩展了技术,因此对各方都有
益,这是一个相当复杂的观点;它是根据经济体的动态观点得出的,这一观点在正规模型中
也难以把握。

当时的主流经济学家激烈地抨击李斯特与凯里的观点,并因为指出他们理论上的错误而高兴。
但是,主流经济学家这样做,并未领会李斯特与凯里研究中的广泛经验,即通过技术开发或
许能够永久地克服边际收益递减。

随着新古典经济学的发展,人们加速远离增长这一关注点。阿尔弗雷德·马歇尔关于增长的观
点与穆勒的观点相类似,除了马软尔之外,其他新古典学者更多地集中于静态均衡上。穆勒
与马歇尔都主张技术进步能够暂时创造出增长的条件,但是,农业与原材料的收益递减法则最
终还会获胜。

20世纪上半叶的经济学家在极大程度上没有涉及增长。一个重要的例外是约瑟夫·熊彼特,
他几乎不属于任何学派。

\subsection{熊彼特与增长}

在三十岁之前,熊彼特就已经为其在《经济发展理论》中的经济增长理论奠定了基础,该书
于1912年首次出版,并于1934年被译成英语。这是一个闪光的概念,它几乎一直潜伏着,原因
是它所包含的内容如此广泛,以至于无法参与经济模型的构建,而在大约五十年中,模型构建
都是主流经济学的时尚。在给爱德华·马兹最近对熊彼特的研究作序时,诺贝尔奖获得者和模
型构建者久姆士.托宾(James Tobin)表示,熊彼特的“发展理论和经济周期理论,在过去的五
十年里,难以并入到支配经济学尤其是美国经济学的风格与方法中",具有讽刺意义的是,熊
彼特坚定地支持在经济学中更多地运用数学和对假设进行经济计量检验。

熊彼特对经济增长过程的解释并不符合正统模式,原因在于他强调增长的非经济理由。尽管他
考察了一些严格的经济因素,然而他坚持认为,导致过去增长的首要因素和将导致末来增长
的因素都是非经济的。

我们首先着眼于他对经济因素的奇特分析。他本质上认同萨伊定律,虽然他认识到并且也分
析了资本主义下经济活动的波动。在他看来,萧条是自我纠正的,低于充分就业时并不存在
均衡。熊彼特认为,萧条有益于制度,它们是经济增长整个过程的一个完整部分。增长与周
期的繁荣阶段相联,原因是这一阶段代表了将新产品和技术引入经济体后的最终成果。但是,
随着信用过度膨胀和经营过分扩张,过剩形成了。过剩所引起的萧条之所以是有益的,是因为
它通过震动打开了经济体,开除了低效率的厂商,从而为一个生长中的、健康的、管理良好
的、富有效率的厂商经济体铺好了道路。

但是,根据熊彼特的观点,经济增长的首要要素是非经济的,在社会的制度结构中能够找到这
些要素。熊彼特将工业化国家所发生的巨大增长归因于他所谓的企业家的活动,在熊彼特看
来,企业家并不仅仅是一个经营者或管理者,此人是一个独特的个体,他生来就是风险承担
者,并将创新产品和新技术引入经济体。

熊彼特清楚地区分了发明过程与创新过程。只有少数有远见的创新的经营者,能够领会一种
新发明的潜能,并为了个人的收益而开发它。但是,他们的收益就是经济体的收益。企业家
采用一项成功的创新之后,其他经营者将跟着这么做,新产品与新技术就在整个经济体中扩
散开来。因此,经济增长的真正来源,存在于创新企业家的活动中,而不是存在于企业界大
多数人的活动中,他们只是规避风险的追随者。

因此,经济增长受到一种奖赏和鼓励企业家活动的制度环境的激励;早期资本主义实行私人
财产和自由放任的政府,这很理想地适宜于经济增长。就其强调激励的重要性与自由放任的
政府而言,熊彼特的这部分分析内容,在理论上和在意识形态上与古典增长理论一致;但是,
古典理论强调资本积累规模这一经济因素,而熊彼特在其对企业家作用的分析中,则强调一
种非经济的、文化的、社会的因素。熊彼特简洁地表明了他的增长观点与主流新古典经济学
观点的对比:

\begin{quotation}
  我们将要考虑的是制度内部引起的那种变化,它转移了均衡点,结果无法从原有的均衡点,
  借助极其微小的步骤到达新的均衡点。即使你愿意不断地增加邮车,你也永远不会因
  此得到一条铁路。
\end{quotation}

熊彼特对资本主义未来增长与发展的观察更加新奇。熊彼特推测,资本主义的终结是其成功
的结果。他是一个在意识形态上保守的经济学家,对经济体的增长持有一种稍显传奇的看法,
认为增长来自于侠客式的企业家的大胆行为。他希望看到这一过程能够持续,他也期待资本
主义由于它的成功而终止。导致这一情形的主要原因是随着社会变得更加富裕而出现的企业
家的消亡,以及知识分子作用的增强。成功的企业家会推动大企业的成长,大企业通过竞争
清除了行业中那些低效率且规避风险的企业。但是,大企业很快就成了风险规避者,变得谨
慎小心,并且由官僚委员会而不是创新的企业家来管理。然后,官僚化的巨型企业将排挤掉
企业和家,用“谨慎的”管理者取而代之。随着被雇用的管理着取代了企业家,大公司的所有
权将变成缺席者所有权。“社会主义的真正领跑者,”熊彼特说,“不是知识分子或社会主义
的鼓吹者,而是范德比尔特们、卡内基们以及洛克菲勒们。”

熊彼特认为,一旦巨型企业排挤掉很多小规模的所有者经营的企业,资本主义的一大部分政
治支持者就会被去除。此外,熊彼特主张,资本主义的成功将消除私人财产的旧有概念,以
及为之奋斗的愿望。一旦企业家消失了,拿薪水的管理者以及股东将不再捍卫私人财产的概
念。他们的态度也将在工人阶级和公众中普遍流行开来。“最终将没有人真正在意是否去支
持它——大企业范围内部和外部都没有人。”资本主义的成功将再次加速这一过程,因为资本
主义所引起的收入与财富的增加,将使社会上的一个智力群体得到成长,他们“运用口头的
力量”并且“对实际事务并不负直接责任”。资本主义的成功,允许这些知识分子依靠制度
的成果来生活,但他们同时却对制度加以批评。他们使劳工运动激进化;尽管他们通常并不
竞选公职,但却为政治家做事并提出建议。偶尔地,他们也会成为政府幕僚的一分子;但最
为重要的是,随着大众传播工具的不断增多,他们将有能力在整个社会散布对资本主义制度
的不满与怨恨。

熊彼特预想他热爱的制度正缓慢但无疑地接近其终结。他担心随着企业家的死亡和自由放任
的结束,政府将越来越多地干预经济体。一些人例如凯恩斯,欢迎这样的于预,认为它是拯
救资本主义的一种方式,但是,对熊彼特来说,它是资本主义即将终结的一种征兆。由于他所
谓的“财产实体的燕发”以及企业家的消亡,熊彼特预言,经济体中解释其以往增长的动态
因素将会消失。

熊彼特博得了一些人的称赞,但并不是来自于重要的主流经济学家。其非形式化的特征,并不
符合正在成为标准的形式化的建模方法。罗伊·哈罗德(Roy Harrod)、埃弗塞·多马(Evsey
Domar)、罗伯特·索洛以及特里沃·斯旺(Trevor Swan)于20世纪50年代的研究,创作了增长的
形式化模型,正是在这些研究中,增长理论才流行开来并成为宏观经济学的核心部分。然而,
随着经济学专业开始领悟20世纪30年代的萧条,并集中于经济周期理论时,大家对这些形式
化增长模型的兴趣完全点然失色了。

\subsection{消费不足主义者的主张}

对增长理论的兴趣,与对下列问题的关注并行,即市场经济是否导致充分就业,以及政府是
否应当干预经济体,以利于保持资源的充分利用。重商主义者特别希望了解决定经济体产品
与服务生产能力的力量,并确定实际产量水平是否达到了潜在水平。很多重商主义者认识到
私人利益与公共利益之间的根本冲突,从而认为经济体将不会实现其潜在的产量,除非政府
加以干预。他们的主张是双重的:根据吉恩·伯丁的看法,第一,他们认为私人利益导致垄断,
而垄断限制了产量;第二,他们认为当个人进行储蓄或者购买外国产品时,就会引发对国内产
品的需求不足,它削弱了经济体。重商主义者的观点是,政府应当规制国内和国外和贸易,
以便使经济体出现贸易顺差,并增加作为国家货币供给的黄金数量。

随着重商主义向古典经济学演进,对政府干预的态度也发生了急剧的
变化。与早期重商主义者不同,亚当,斯密认为,竞争性市场的力量如此
强大,以至于私人利益被引导着为公共利益服务,就像被一只“看不见的
手”引导着一样。只有当政府遵循自由放任的政策,经济体才能实现其洪
在产量。斯密赞同自由放任,他的分析是一种前后关联的主张,是在考虑
到可行的多种选择时做出的。他赞同重商主义者关于垄断减少了产量的观
点,但是他声称,旨在控制礁断的方法一一政府对垄断交易和分配的控
制一一只能使事情更糟,而不是更好。因此他认为,更可取的政策是依靠
自由放任与竞争,使资源尽可能充分地被利用。人允许市场繁荣的自由放任
政策鼓励劳动分工、专业化以及技术发展,从而鼓励增长。市场创造增长
这一倾向,也为马克思所强调。但是,马克思从分析中得出了完全不同的
合义,他集中研究了伴随增长的不平等的收入分配,及其对政治结构和社
会结构的可能影响。

第15章现代宏观
斯密与其他古典经济学家反击了重商主义者的消费不足主张(under-
consumptionarguments),他们认为,储蓄将自动被转换成投资支出,原因
在于储蓄决策就是投资决策。自由放任的经济体将自动引起资源的充分利
用,这一命题被称作萨伊定律,它成为古典与新古典思想的重要成分。古
典经济学家也择击了重商主义通过实行贸易顺差来增加黄金储量的主张,
认为一国的财富不是用贵金属而是用实际产量来衡量的,并且,人允许自由
贸易的国家将变得富裕,从而获得对外竞争优势。

古典经济学家尤其是斯密与约翰斯图亚特.穆勒承认,市场力量并
不是完美地发挥作用,但是他们主张,与可供选择的其他方案相比,市场
能更好地发挥作用。从1800年直到1930年,除了托马斯.马尔萨斯之外,
对经济周期的分析都留给了非正统和非主流经济学家,例如,米哈仇
尔:塔干-巴拉诺夫斯茜(MikhailTugan-Baranowsky,1865一1919),约
输.A霍布斯。能够依靠市场来控制经济体的古典信爷,将经济研究的关
注点由货币与财政力量转向了实际力量,对宏观经济问题的古典分析普遍
认同实际力量与名义力量之间的分离。
人般丰数量理论
古典与新古典经济学家至少在同一个宏观经济问题上有兴趣;什么决
定了价格总水平?他们通过运用在微观经济理论中形成的供求方法,致力
于研究这个经济问题。货币供给被假定为由货币当局决定,所以,一些正
统经济学家主张,所要分析的基本问题在需求方面。家庭与厂商被假定是
理性的,为了不同的使用目的而对货币有和需求。瓦尔拉斯、门格尔,还有
其他一些人创造了一种供求分析来解释货币的价值,但是,这些理论中最
为著名的可能是马菊尔发展的理论,它因货币数量论的剑桥现金余额学说
而知名。

大卫:休议在1752年就对货币数量理论(quantitytheoryofmoney)做
了最早的清晰陈述。正如通过文献流传下来的那样,该理论认为价格总水
平取决于流通中的货币数量。马吹尔的货币数量理论,试图为价格与货币
数量直接变动这一宏观理论提供微观基础。他通过详细阅述家庭与厂商行
为理论做到了这一点,从而解释了对货币的需求。马吹尔推论,家庭和厂
A2S



商愿意在其货币收入中持有一小部分现金余额。如采必代表货秆(现金加
二活期存款),PY代表货币收入,&为家庭与厂商愿意以货币形式持有的收
人所占的比例,那么,基本的现金余额方程十:
MokpYy
因为马歌尔承认萨伊定律,所以假定充分就业。假定保持不变,货币
数量的增加将引起货币收入PY的增加。因为假定了充分就业,所以货币
数量的增加将导致较高的价格,以及随之发生的货币收入的增加;然而,
实际收入并没有改变。货币数量的减少将导致货币收入随价格下降而减少,
实际收入也将保持不变。我们不去考察马吹尔这一表达的很多不同方面,
重要的一点是,马软尔的货币数量理论,在使最大化厂商与家庭的微观经
济行为与价格总水平的宏观经济问题相结合方面,进行了一次尝试。

一组经济学家,其中最突出的是美国人欧文.费雪,发展了另一种形
式的货币数量论,该理论因交易学说而知名。然而,他们对为价格总水平
的宏观经济分析寻找一种微观基础只表示出了很小的兴趣。在这一季说中,
MV=PT
这里履是货币的数量,V是货币的速度,了P是对价格水平的度量,了起
交易量。

尽管这两种方法具有重大的差异,但是,它们有一个共同因素:两者
都是为了解释价格总水平的决定力量。它们不是用来解释实际收入水平的,
实际收入被假定为在充分就业水平上,且因经济体中的非货币力量而固定。

并不是所有的经济学家都满意于这一分析。例如,纳特.威克塞尔认
为,货币数量理论未能解释“在既定条件下为什么对产品的货币或金钱需
求,超过了或者达不到产品的供给”。?威克塞尔试图发展一种所谓的收入
方法来解释价格总水平,也就是说,发展一种解释收入波动以及价格水平
波动的货币理论。
经济周期理论
尽管从商业资本主义开始,经济活动的、收入水平的以及就业的波动
Q@参见纳特:威克塞尔的《政治经济学讲义》一书第二卷第160页,该书由英国路特雷奇与
由萎.哥罗出版公司于1935年出版(该书最初用瑞典语于1901年和1906年出版)。
就一直发生着,并为正统理论家所承认,但是,19世纪90年代之前,经汪
学家并没有进行系统的尝试来分析萧条或者经济周期。非正统理论家则以
较多的精力从事着对这些问题的研究。因此,直到19世纪的最后十年,正
统经济理论都是由下列成分组成的,即发展相当充分的解释稀缺资源配置
与分配的微观经济理论结构、解释价格总水平决定力量的宏观经济理论,
以及一组关于经济增长的松散的观点。1890年之前,正统理论“对萧条与
周期的研究一直都是边缘的和离题的”。了

这种概括的一个主要例外是克莱门特-朱格拉《ClementJuglar,1819一
1905)的研究,他在1862年出版了《论法国、英国、美国的商业危机及其
周期》(DecrisecommercialesetdeleurrétourpériodiqueenFrance,enAngleterre
etauxFiats-Unis)。这部著作的第二版出版于1889年,运用历史和统计材
料进行了相当大的扩充。朱格拉是韦斯利.克莱尔*米切尔精神上的前莫,
原因是他并没有构建经济周期的演绎理论,而是宁愿收集历史和统计材料,
并用归纳的方法加以处理。他的主要贡献是,他声称周期不是经济体系外
部力量而是内部力量的结果。他认为周期包含三个阶段,这三个阶段按照
过续的顺序反复出现:
繁荣、危机、清算阶段,尽管受到人们生活中幸运和不六事件的影响,
然而,它们并不是偶然事件的结果,而是起因于全体居民的行为与活动,
最重要地起因于他们的储蓄习惯,以及他们使用可利用的资本与信用的
方式,包
尽管朱格拉的著作发起了对经济周期的研究,然而,有关波动的现代
正统宏观经济分析是以俄国人米哈伊尔:塔于-巴拉诺夫斯基的著作为基
础的。米蛤伊尔,塔干-巴拉诺夫斯基的《英国的产业危机》(Industrial
CrisesinEngland)一书于1894年首次用俄语出版,紧接着是德语版和法语
版。在对以往解释经济周期的尝试做出评论之后,米哈伊尔,塔干-巴拉
@阿尔文*汉森.商业循环与国民收入.美国:W.W:读顿出版公司,1951:225

名”参见克莱门特.朱格拉的《论商业危机》第X区页,该书第2版由法国吉约曼出版公司于
1889年出版。上述内容又被引用到工W.哈钦森的《1870一1929经济学说评论》一书的第372页,
该书由英国牛津大学出版部印刷所于1953年出版。

诺夫斯基断言它们都不能令人满意。对于理解经六周期来说,他的主要页
献是阐述了两条原则:(1)经济波动内在于资本主义制度,原因在于,它
们是制度内部力量作用的结果;(2)在投资支出的决定力量中将会找到经
济周期的主要原因。凯恩斯的收入决定分析,强调资本主义的内在不稳定
以及投资的作用,这一分析的源泉来自于塔干-巴拉诺夫斯基、朱格拉、
斯皮托夫(Spiethoff)、熊彼特、卡塞尔、罗宾、威克塞尔和费雪,以及
凡过仑、霍布斯、米切尔等人。

一些重商主义者、重农主义者以及随后的很多非正统经济学家,早就
提出资本主义具有引起萧条的内在力量。即使主流经济学家对周期的考察,
例如,杰文斯的日斑循环,通常也被忽视了。自1900年之后,正统理论家
就经济周期进行了比较认真的研究,但是,足以令人奇怪的是,这些研究
的存在伴随着一种持续的基本观念,即经济体的长期均衡位置提供了充分
就业。因此,我们看到像弗里德里希.冯,哈耶克一类的经济学家,将总
量波动问题作为一种协调失灵来探讨,同时对市场经济的自我平衡性质保
持着坚定的信念。无论是正统的还是非正统的经济学家,没有人能够挑战
这一信念,原因在于,没有人能构建起一种收入决定理论,来表明在低于
充分就业的水平上均衡也是有可能存在的。当约输,梅纳德-弛恩斯于
1936年形成了一种理论,认为在低于充分就业的水平上均衡可能存在时,
正统宏观经济理论的一个新阶段开始了。
也恩斯主义经济学是根据约输,梅纳德:叫恩短来从名的,碌礼斯的
父亲约输.内维尔:凯恩斯凭借自身的资历也成为一位重要的经济学家。
然而,儿子的成就很快就超越了父亲。在这一点上和其他一些方面,约
输.梅纳德.凯恩斯的生活颇似约翰.斯图亚特:穆勒。两个人的父亲都
是优秀经济学家的同辈与朋友:詹姆士.穆勒是大卫.李嘉图的朋友,约
输,内维尔-凯恩斯是阿尔弗雷德.马吹尔的朋友。小凯恩斯与小穆勒都
接受了特别为知识分子的孩子们提供的高质量的教育,这种教育训练他们
天生敏锐的头脑去开辟新天地,并借助他们的创作力量去说服其他人。穆


'
!
!
1
1
t
1
,
!
!
1
'
勒与饥恩斯都否定了父辈们的经济学的政策含义,着手于新的方品。但至
此为止,两个人的相似之处结束了,对约翰-斯图亚特-穆勒来说,他没
能完全与其父亲和李嘉图的理论结构绝交;最终在古典与新古典理论之间
采取了一种折衷。凯恩斯与过去的绝交一一与贯穿于斯密、李嘉图、约
翰:斯图亚特,穆勒以及马软尔的自由放任传统的绝交一一则要更加完全。
尽管他详熟基本的马葡尔局部均衡分析,然而,他构建了一种新的理论结
构,致力于研究对经济理论与经济政策都具有重要影响的总量经济。

20世纪那些思维狭窄的经济学家的一成不变并不符合凯因斯。事实上,
他由于将太少的时间用于经济理论,并且因为过于广泛地分散其兴趣而受
到批评。即使作为伊顿公学和剑桥大学的学生,他也显示出兴趣广泛的倾
向;因此,他以业余艺术爱好者而为人所知。完成教育后,他进入英国政
府的印度事务部做文职人员,在重返剑桥之前他在那里干了两年。他从未
专职做过教师。他对经济政策的持续兴趣,使他在一生中担任了很多政府
性职务。他积极参与经营性的事务,既是为了自己,也是因为他是国王学
院的会计,他的经营能力通过下列事实表现出来,即他的个人净资产从
1920年的几乎破产,到1946年他去世时,增加到超过两百万美元。凯恩斯
对戏剧、文学还有芭芝人舞颇感兴趣;他与一位巷萝舞女演员结婚,并且与
一群以布卢姆芯伯里派而知名的伦敦知识分子有联系,这个圈子包括诸如
克芋夫,贝尔(CliveBell)、E.M.福斯特《(E.M.Forster)、利顿.斯特雷
奇(LyttonStrachey)以及维吉尼亚:伍尔夫(VirginiaWoolf)等一类的名
人。作为一名大学本科生,他独特的多种才华使他能够成为一位学有成就
的数学家,创作出关于概率理论的著作,并且成为一位颇有影响的令人印
象深刻的散文家,他的《和约的经济后果》(EconimicConsequencesofthe
Peace)和收于《劝说集》(EssaysinPersuasion)与《自传文集》(Essaysin
Biography)两本书中的散文所体现出的对文字的彻底精通,清楚地显示了
这一点。

.作为经济学家的凯恩斯,一个最重要的方面是他对政策的倾向性。他
作为英国财政部的代表,参加了凡尔赛和平会议,但是1919年他又突然辞
去该职务。他对几尔赛条约的条款感到厌恶,该条约强加给德国大量的赔
毕节轧斯认为德国永远无法支付。由于在1919年出版的《和约的经济后




429;




WoryofGoonormueSoc
430
果》中对条约的条球进行批评,电恩斯获得了国际上的称赞。1940年他创
作了《如何为战争付账》(HowtoPayfortheWar),1943年他向国际货币当
局提出了一个被称为凯恩斯计划的建议,该建议在第二次世界大战后付诸
实施。作为前往布雷顿森林的英国代表团的首脑,他对国际货币基金组织
和世界银行的成立起了作用。但是,他对政策与理论的最重要的页献,包
含在他的《通论》(1936)一书中,该书创造了现代宏观经济学,并且仍然
是大学本科宏观经济学大部分所讲授内容的基础。保罗,萨缪尔森反思凯
恩斯时代时,捕提到了它的重要性,他写道:“有一种疾病最初袭击并杀死
了大批孤立的南海岛居民部落,《通论》以这种疾病意想不到的致命性,感
泥了大多数三十五岁以下的经济学家。”Q
在经济理论书籍中,可能没有哪一本书拥有比池恩斯的《通论》更日
行其是的第一章。的确,其他经济学家都在声明自身的原创性和才华,但
是,凯恩斯仅凭借一种气势这样做,并足以使他的声明令人心悦诚服。缺
少谦逊显然可以追溯到凯恩斯的年轻时代。当他刚从学院毕业参加文职考
试,在经济学上并没有得到最高分数时,他的反应是:“显然我比主考官了
解更多的经济学。”@在凯恩斯创作《通论》的过程中,他写信给乔治伯
纳德*肖说他正在创作一本新书,这本书将会使世界考虑经济问题的方式
发生革命。《通论》的第一章只有一个段落那么长。在此,凯恩斯简单地表
明他的新理论在下列意义上是一种通用的理论,即先前的理论是能够放入
其更加通用框架中的一种特殊情形。凯恩斯所说的“先前的理论”,既指古
典经济学也指新古典经济学,他将其界定为李嘉图的经济学(因为它符合
艾伊定律),以及遵循这一信念的那些人的经济学一一约翰,斯图亚特穆
勒、马软尔、埃奇沃斯还有庇古。

尽管作为经济学家,凯恩斯最重要的一个方面是他对政策的倾向性,
8保罗萨缪尔森的“通论:1946”一文,载于《饥恩斯的通论;三十年的报道》一书,该
引文出自第315页。该书由美国圣马丁出版社于1964年出版。

加”R.F.哈罗德.约翰梅纳德.凯恩斯的一生.美国哈考特,布瑞斯出版公司,1952;:
121


然而,他最重要的作同《通论》昌然具有政开含义,和在本质上却是一部理
论著作,其主要读者在职业经济学家中才能找到。凯恩斯写道“本书主要
是为我的同仁经济学家们创作的。我希望对其他人来说,它也是易于理解
的。但是,它的主要目的是处理理论中的难题,将这一理论应用于实际只
是第二位的。”D.:

通过了解凯恩斯运用理论的方式,我们能够调和这一表面上的矛盾。
很多经济理论可以被称为具有非关联性;即它们是在一种制度真空中发展
起来的。借助演绎逻辑能充分地理解这种理论;它们从最初的原理开始,
依据经过仔细表述的假设,由此推导出结论。进行这些假设时,不是去考
虑现实,而是试图理解假设之间相互作用的内在钠辑。这种理论可以被称
为解析理论(analutictheory)。正确地完成了推导的一般均衡分析就是一
种解析理论。因为假设不可避免地会远离现实,所以,从较宽范围的解析
理论中得出政策结论就相当复杂。

凯恩斯运用了一种不同的理论,它可以被称为“现实解析的",因为它
是现实方法与解析方法之间的一种折衷。现实解析理论(realytictheory)是
前后关联的,它将关于经济体的归纳性信息与演绎逻辑相混合。现实引导
着对假设的选择。现实解析理论很少天生地令人满意,但是因为它们密切
地与现实相符,所以,比较容易从中得出政策结论。在《通论》中凯恩斯
并不是从最初的原理开始,而是运用现实指导他对假设的选择。因此,尽
管他集中于理论,然而他从未忽略理论的政策含义。

举一个例子可能会使现实解析理论与解析理论之间的区别更加清楚。骸
恩斯假定价格与工资相对不变,而没有试图证明这些假设是正确的。尽管他
在《通论》中简短地讨论了弹性价格的含义,认为它们没有解决失业问题,
然而对他来说,很少关注对弹性价格含义的彻底考察,就所要解决的问题而
言一一针对失业该做些什么一一假定工资与价格不变就是合理的。通过运用
其现实解析方法,他能够做到这一点,而真正的解析模型则不允许这种假设。
凯恩斯把为其理论提供一种解析基础这项工作留给他人去完成。宏观经济思
相后来的大多数发展,都在试图为宏观经济学提供一种解析基础。
DD约彰,梅纳德,凯恩斯就业、利息与货币通论.身国:表现米宇出版公司,1936,3

MS
|DgPABecomeHoey
在完成其两卷本的《货币论》(TreatiseonMoney)之后,饥轧斯立刻开
始创作《通论》,前者运用货币数量论来讨论周期性波动。因同事丹尼斯
罗伯逊(DennisRobertson)的灰心失望,凯恩斯在《通论》中放弃了这种
方法,他先前与罗伯撑有过密切的合作。取而代之的是,他采用了一种集
中于储蓄与投资之间关系的简单的新方法。为了给自己提供大量的目标,
凯恩斯将新古典非均衡货币方法与较早的古典方法混杂在一起,奔大了它
们的看法,并将它们集体称作“古典理论"。他这样做,仿佛是在创作一幅
古典思想的漫画,强调了它与他的新方法之间的差异,但又隐藏了它的很
多微妙之处。

在教科书的多种模型中,凯恩斯经济学变得具体化了人,这些模型被称
做乘数模型(有时岂做AE/AP模型)、IS/LM模型以及AS/AD模型。整个
20世纪80年代,这些模型都是宏观经济学所讲授内容的核心,并且仍然出
现在很多最近出版的大学本科教科书中。但是,随着凯恩斯经济学的失宠,
前沿的宏观经济学在极大程度上按照不同的方向行进。
弛恩斯采雪柑型的兴起:.1940一1960
20捞纪40年代和50年代,经济学家探启来数multiplier)模型,天
以极度的细致来开发它。它被扩展到包括国际效应、不同类型的政府支出
以及不同类型的个人支出。诸如平衡预算乘数这样的术语,成为经济学术
语的标准构成,每个学经济学的学生都得学习凯恩斯模型。

值得注意的是,过去和现在通常被称作凯恩斯模型和饥恩斯货币与财政
政策的东西,并不能在凯恩斯的书中找到。《通论》中没有一个图表,也没有
任何关于货币与财政政策用途的论述。那么,乘数模型(通过代数和几何完
成的)是怎样成为20世纪50年代宏观经济争论的焦点的?部分原因是,与其
他选择相比,它似乎提供了对现状的一种更好描述。但是,其他因素也产生
了作用。关于凯恩斯经济学有效性的最初政策争论,集中在财政政策上〈战
争期间的政府赤字,显然将西方世界从大萧条中搜了出来)。因为乘数模型很
好地搬提到了财政政策的效应,所以它便倾向于成为弛轧斯模型。我们猜想,
在这一模型最初的采用和长期的认同中,社会原因也发挥了作用。对真理的
需要通常被专业的其他需要加以调和一一明确地说是教学需要以及在刊物上

发表文章的害要,来数模型完类地氨应了乾些二要。

乘数分析是在美国流行开的。保罗,萨缪尔森与阿尔文汉森将其发
展为主要的凯恩斯模型。萨缪尔森的教科书将它引信教学,其他书复制了
萨绢尔森的教科书,乘数模型很快就成了凯恩斯经济学。乘数分析具有很
多教学上的优势,便于讲授和学习。它通过为宏观经济学提供一种核心的
分析结构,而使其作为一个单独的领域得到发展,就像供求分析之于微观
经济学一样。

20世纪30年代的经济大萧条,改变了社会和经济学家观察市场的背
景。在此之前,赞同自由放任的新古典主张,不仅是经济理论的依据,而
且是关于政府的哲学判断与政治判断的依据。20世纪早期,除了激进分子
之外,几乎所有人的一般政治取向,都是反对经济体中大量的政府参与。
在当时的环境中,我们现在认为理所应当的很多政府项目,例如,社会安
全和失业保险,看上去都是很偏激的。

随着大萧条的开始,人们的态度也开始改变。很多人觉得,如果自由
市场能引起大萧条期间存在的那样的经济危难,那么,该是开始考虑其他
选择的时候了。随着经济学家开始更加详细地分析总量经济,很多人对他
们的政策处方变得不太自信了,而是更多地意识到新古典理论的缺点。结
果,经济学家开始提倡多种政策建议来解决与其主流新古典观点不一致的
失业问题。例如,20世纪30年代初期,英国的亚瑟.C:庇古和美国芝加
哥大学的一些经济学家,就主张公共建设工程项目以及赤字,以此作为对
杭失业的一种手段,
计乱斯经阐学包含政素论感,并形成了一种模型,该模型将对激进的
政府政策的需要植人其中。在这一模型中,总需求控制着经济体的收入水
平,政府通过货币政策与财政政策控制总需求。

20世纪50年代至60年代期间,饥恩斯的政策意味着通过货币政策与
财政政策进行调整。阿巴+P,勒纳〈(AbbaP.Lermmer,1903一1982)在引导
凯恩斯的分析朝着这种方向调整方面具有影响力。在其《控制经济学》(E-
conomicsofControl,1944)中,勒纳鼓吹政府不应当遵循合理财政(sound

.NS:经济轧想史
oryofGeororasoisgfy

finance)政策(预算总是平衡);而应当旭循功能财政functionalfinanee)
政策,只考虑政策的后果,不考虑政策本身。

功能财政允许政府“驾驭”经济体;用一个经常重复的比喻来说,货
币政策与财政政策被比喻为政府的方向盘。勒纳主张,财政政策与货而政
策是政府应当加以运用,以实现其宏观经济目标一一高就
及高增长的工具。赤字规模完全无关紧要:如果存在失业,政府应当增加
赤字和货币供给;如果存在通货膨胀,政府的做法则应当相反。

勒纳关于“凯恩斯主义者”观点的生硬表述,触动了很多凯恩斯主义
者的敏感性,并引起了相当多的讨论,其至引起凯恩斯自己否认凯恩斯主
义。D埃弗塞.多马是当时一位有名的凯恩斯主义者,他说:“其至连凯思
斯主义者,一听到勒纳关于赤字规模方面的无关紧要的观点,也都退缩了
并确定地说:不,是他(勒纳)弄错了。”®但是,凯恩斯很快就改变了主
意,并赞同勒纳的看法,像经济学专业中的大多数情况一样,没过多久,
凯恩斯经济政策就与功能财政同义了。

此外,货币政策与财政政策在政治上是合意的。很多经济学家和一
人认为,大攻条证实了政府在引导经济方面应当显示出更大的作用。货币
政策与财政政策的运用将这一作用保持在最低限度。市场能够像以前一样
自由地运转。政府并不直接决定投资水平,它只是通过管理赤字预算或到
余预算,闻接地影响总收入。在很多人看来,赤字的合法化具有另一个合
音的特征,它介许政府不征税就支由。


政策将理论与规范性的判断结合起来。因此,理解池恩斯于全,束需
要考察当时的经济学家尤其是凯恩斯的一般哲学观点。凯恩斯不是激进分
子,虽然在《通论》出版之后,他被指责为是激进分子。我们不能指望拥
有他那样的背景、教育、经历的人,赞成对其所处社会的制度结构进行激
列的变革就其有关改恋社会结构的观点来说,凯思斯革本上是保守的,
Q关于凯恩斯被美国经济学家接纳的历史,参见大卫,C:卡兰德和哈瑞,宇人瑞斯编答的
《凯恩斯主义来临美国》一书,该书由美国埃尔加出版公司于1996年出版。
吕”这段引语来自作者与埃弗寨.多马的一次未公开的面谈。
通常只是提倡那些能保持资本主义本质因素的变站。他的观点古,如末制
度最坏的缺点不消除,那么,个人将会放弃资本主义制度,并且失去的远
大于他们获得的。

凯恩斯乐于承认,社会组织的这些变化可能会解决一些经济问题,但
是,他觉得这种解决办法,只能以个人主义及其经济上和政治上的优势为
代价来换取。个人主义的经济优势,来自于利用私利以实现更大的效率与
创新,经济学家都充分了解这一点:
但是,首要的是,如果能够清除掉个人主义的缺点与苏端,那么,和在
下列意义上,它就是对个人自由的最佳保障,即与任何其他制度相比,它
极大地扩展了进行个人选择的领域。它也是对生活多样性的最佳保障,这
种多样性正好缘自于扩大了的个人选择领域,并且,多样性的损失是生活
单调或极权国家所有损失中最大的。人2
凯恩斯关于美好社会结构的宽泛斩学观点,引起了来目两个方面的皇
击。左咽认为他是资本主义及自身阶级的辩护者;右翼则将他看做是狂暴
的社会主义改革者,致力于瓦解资本主义制度。他对来自右咽批评的反应
至少比较温和。他写道:“因此,虽然政府职能的扩大……看上去将……是
对个人主义的令人和灵怖的侵犯,然而相反,我要为它辩护,它既是避免现
有经济形式完全被毁灭的唯一行得通的手段,也是个人主动性成功发挥作
用的条件。”®@凯恩斯发现,资本主义的主要好处之一是它给予个人主义自
由的发挥。他认为,确实来自个人主义的次端,不用摧毁资本主义就能得
到纠正。他说,资本主义的主要缺点或错误,“是它未能提供充分就业以及
其武断的不平等的财富与收入分配”。®

20世纪30年代的大痕条使很多经济学家确信,未能提供充分就业是资
本主义的一个主要缺陷。第二次世界大战后的经济学家所面对的一个主要
问题是.我们能用什么政策来保持资本主义最好的东西,同时防止大的萧
QD参见泌恩斯的《通论》第380页。
@®同上。
国同上,第372页。

1436
条?一开始,对很多美国人来说,凯恩斯的政策观点似乎过于上自由。电忆
斯主义者所提出的货币与财政政策,最终被美国经济学家所信奉,原因在
于,它们只需要少量的政府对经济体的直接干预。然而,这些政策受到将
凯恩斯主义者视为社会主义者的人的择击。劳瑞:塔西斯(LorieTarshis)
体会到了这一点,他写了第一本凯恩斯主义的介绍性教科书,但一个保守
派团体发起了一场运动,阻止男校友向使用塔西斯教科书的任何学校进行
捐助,并迫使塔西斯任教的斯坦福大学将其解雇。塔西斯介绍性的教科书
在商业上未获成功,但是,萨缪尔森的教科书紧随其后,并获得了极大的
成功,被广泛地加以模仿,部分原因是它用一种科学的形式庶荐了饥思斯
的经济学,从而避免了那种挫雏塔西斯的意识形态上的择击。
货息主义

整个20志纪50年代和60年代,货币主义者都是凯恩斯主义者的主要
陪衬。在米尔顿*弗里德曼的领导下,他们对凯恩斯的政策与理论提出了
有效的反对。凯恩斯主义者于20世纪50年代所使用的消费函数模型,没有
考虑货币的作用,也没有考虑价格或价格水平。初期缺乏对货币供给与价
格的关注,体现在以饥恩斯分析为基础的政策中。在第二次世界大战期间
与财政部签署的一项协议中,联邦储备银行同意购买任何必需的债券,以
将利率维持在固定水平上。联邦储备银行这样做,放弃了对货币供给的所
有控制。货币主义者认为,货币供给在经济体中起着重要的作用,不应当
被局限于保持利率不变一种作用上,因此,早期货币主义者的战斗口号就
是“货币至关重要”。

凯恩斯主义者很快就乐于赞同货币主义者关于货币至关重要的看法,
但是他们认为,在相信唯有货币至关重要这一点上货币主义者与他们不同。
助于IS-LM凯恩斯主义-新古典的综合,争论得以解决,在IS-LM中
货币主义者假定一条非常缺乏弹性的LM曲线,而凯恩斯主义者则假定一条
非常富有弹性的LM曲线。因此,至少就教科书所呈现的内容而言,货币主

义分析与凯恩斯主义分析,在通常的新弛恩斯IS-LM模型中走到了一起,
只是对一些参数的看法略有不同而已。

现代宏观经济学是经济学家完成新凯恩斯主义模型,并发现很多问题
的结果,这些问题中,一些纯粹是理论上的,一些则随着新凯恩斯主义政
策的失败而变得明显。
IS-LM模型(1S-LManalysis)一直是大多数宏观经济学家工具租中
的组成部分;它提供了大多数经济学家最初用来应对宏观经济分析的框架。
然而,到了20世纪60年代,它在文献中得到了充分探讨,并被认为在阁干
方面存在欠缺。

人
为,明
避拓主张直入训台机人(末数)的发生要人于价格或利率油整机机比较
静态分析失去了凯恩斯研究中的这一方面。

第二,在IS-LM模型中,实际部门与名义部门的相互关系通过利率而
发生,不能通过其他渠道发生。货币主义者对这一点不满意,因为他们认
为货币能够通过多种渠道影响经济体。很多凯恩斯主义者则对框架不满意
因为它几乎没有使通货膨胀问题更清楚,而通货交脱此同题在20世纪60年代
开始被看做是一个严重的经济问题。

第三,用来得出LM曲线的货币分析,并不是以一般均衡模型为基础
的,而是以一种相当特别的方式加以假定的。它并没有真正地将名义部门
与实际部门相结合,因为它没有捕捉到货币与财政部门的真正作用,所以,
它使这些部门的功能平凡化。当大多数经济学家实际上认为,,下降的价格
水平将使事情更糟而不是更好时,它使价格水平的下降看上去好像能够产
生均衡似的。不过,IS~-LM模型得到了采用。它整洁,能很好地服务于教
学功能,是一种粗燃但尚能使用的工具;它提供了关于经济体的普遍正确
的见解,并且是可以利用的最好模型。

然而,对现有模型的不满意,使很多宏观经济学家在其研究中转向其
他模型。这导致了一种分烈。虽然IS-LM模型一直是20世纪70年代和80

tly|






'
4
4
'
!
1
1
1
!
!
,
,
'
,
,
rt
!
!
!
!
!
!
1
,
年代主要的大学本科模型,然而,研究生阶段的研究则开始集中二元全个
同的问题。到了20世纪90年代早期,关注点的变化渗透到大学本科课程
中。宏观经济学的现代理论争论与1S-LM曲线形状没有太大关系。它们反
而是从一种微观经济角度接近宏观经济问题,并且涉及数量与价格的调整
速度。在某种意义上,20世纪70年代和80年代的很多宏观经济研究者认
为,我们应当跳过凯恩斯的1S-LM这一段,回归到20世纪30年代存在的
宏观经济争论中,当时,问题都是用微观经济术语来设计的。因此,从20
眉纪70年代开始,我们看到了对凯思斯经济学的反抗。
现代宏观经济学约活起
货币主义者对通货膨胀的关注,使之在20世纪70年代随厦通抽脱版的加
剧而处于显要地位。当这一切发生时,凯恩斯的政策与理论便失宠了。财政
政策在政治上被证明难以实现;关于支出和税收的决策,是基于其宏观经济
后果之外的理由而做出的。货币政策就成了佼佼者,但是,凯恩斯模型没有
将货币政策的潜在通货膨胀效应包含进去,因而并不完全适合于对货币政策
的讨论。因此,为了设计政策,出现了远离凯恩斯经济模型的运动。

同时,出于理论上的理由,也出现了远离凯恩斯模型的运动。当经济
学家试图为这些模型发展微观基础时,他们发现,在标准的一般均衡微观
经济方法的环境中,他们做不到这一点。发展微观基础的这种愿望,应当
被给予一些评论,原因在于,它对于理解下列运动来说是很重要的,这一
运动即远高新古典经济学,靠近现代形式主义折刻的模型构建经济学。

凯恩斯的宏观经济学并不符合新古典模式,因而可以被看成是远离新古
典,朝向表现现代经济学特征的折庄主义所迈出的一步。它开始于总量的相
互关系分析,而不是由最初的原理去发展这此关系。因此,它总是具有一种
无关紧要的理论上的存在性,它的主要作用是作为一种粗灼但尚能使用的政
策指导。整个20世纪50年代和60年代,松散的微观基础被添加到宏观经济
学中,它们看上去是适合的,但是,没有人试图由最初的原理去发展宏观经
济学模型。宏观经济学只不过还在那里一一一种与瓦尔拉斯理论几乎没有直
按联系的单独的分析,而瓦尔拉斯理论是理论微观经济学的核心。

第15章现代宏观经济思想的发展
宏观经济学的微观基础
20世纪70年代,经济学家为了修补这个问题,开始尝试看使凯恩斯柑
型适合新古典一般均衡模型,以此来奠定宏观经济学的微观基础。他们之
所以这样做,有两点原因:第一,为了理论上的完整性;第二,为了能够
扩展模型,以便在分析中包含通货膨胀。但他们这么做时发现,当把标准
的新古典原理应用于凯恩斯模型时,凯恩斯模型声塌了。凯恩斯宏观经济
学,即教科书中传统的宏观经济学与所讲授的微观经济学予盾。

有关微观经济基础的文献,确立了新的方式来考虑失业。凯恩斯主义
的分析将失业描绘成一种个人无法获得工作的均衡现象,而微观基础文献
则把失业描述为一种暂时的现象一一离开工作的工人和参加工作的新工人
的流动相互作用的结果。它认为部门之间的流动是失业的一个重要原因,
并且这些流动是动态经济过程的自然结果。对于宏观经济学的新微观基础
方法来说,失业是一个微观经济问题,而不是一个宏观经济问题。

微观基础经济学家认为,为了理解失业与通货膨胀,经济学家必须着
眼于个人与厂商的微观经济决策,并将那些决策与宏观经济现象相连。搜
索理论(searchtheory)研究不确定条件下个人的最佳选择,像很多新的动
态调整模型一样,它成了宏观经济学的一个中心论题。随着研究者开始越
来越多地集中于这些模型,他们越来越少地关注IS-LM模型。最初的微观
基础模型是局部均衡模型,但是,一旦打开了微观基础的盒子,经济学家
就需要从中得到一些将不同市场联合起来的方法。明显的选择是运用一般
均衡模型。因此,我们在第14章中看到的成为微观经济学重要模型的一般
均衡分析,连同微观基础文献一起,都被引导进宏观经济学中。

20世纪70年代早期,微观基础文献因准确地预测到了通货膨胀而被灌
输进经济学专业的意识中。微观基础方法的提倡者以理论为根据,认为菲
利普斯曲线(Phlillipscurve)一一表明在通货膨胀与失业之间权衡的一条曲
线一一只是一种短期现象,一旦通货膨胀被加入预期中,失业与通货膨胀
的权衡就会消失。长期菲利普斯曲线接近垂直,经济体倾向于一种自然失
业率(naturalrateofunemployment)。

新微观基础方法的政策含义相对突出。它消除了政府通过扩张性的货

币政策与财政政策影响长期自然失业率的潜力。这种尝试通过暂时欺骗了
人在短期中会见效,但是,扩张性的政策在长期中只会导致通货膨胀。根
据新的微观经济学观点,政府将失业降低到自然率之下的做法,是20世纪
70年代后期通货膨胀的原因。

然而,凯恩斯的货币政策与财政政策并没有完全被排除。至少在理论
上,它们仍然能够被用来暂时消除波动。因此,20世纪70年代早期,在凯
恩斯主义者的经济学和宏观经济学微观基础方法提佛者的经济学之间,出
现了一种折囊:在长期中,上古典模型是正确的;经济体将倾向于自然失业
率。然而,在短期中,因为假定个人缓慢地调整其预期,所以,凯恩斯的
政策能够具有一定的效果。
新的古典经济学的兴起
20世纪70年代中期,理性预期(rationalexpections)这一术语自先出
现在宏观经济学视野中。理性预期假设是查尔斯.C:霍尔特(Charles
C.Holt,1921一)、弗兰科,黄迪利安尼(FrancoModigliani,1918一)、约
翰,称斯(JohnMuth,1930一)以及赫伯特.西蒙(HerbertSimon,
1916一)所进行的微观经济分析的副产品,这些经济学家试图解释,为什
么很多人似乎并不按照新古典经济学假设他们将遵循的方式来使其行为最
大化。他们的研究有意要借助动态模型来解释西蒙所请的“满意度”行为,
即为什么厂商的行为与微观经济模型不相符。约坦:穆斯认为应优先进行
这项研究,他写道:
人们有时认为,经济学中的理性假设村致理论与所观紧的现从,尤其

是与不同时期的变化相了矛盾,或者导致理论没有得到充分的解释。……:
们的假设正好依据相反的观点.动态经济模型并不假设充分的理性,中
和斯主张建模时做那种假定是合理的,原因在于,韦期是对未来事件
有根据的预报,它们本质上符合相关的经济理论。正如西蒙所述:“(穆斯)
解开了难题。他不是通过详细阐述决策过程模型来小及不确定性,而是坚
ye-




第15章现代宏观经济轧想的发展
决地一一如果他的假设是正确的一一使过程不相关。”

运用其“动态理性”假设,穆斯将非均衡转化为均衡。正如新古典学
者运用理性来确保静态的个人最优性,或者确保个人向他或她的预算线与
无差异曲线的切点运动一样,穆斯运用理性来表示“动态的”个人最优
性一一将个人设定在他或她的跨期无差异曲线上。只要经济体中的私人参
与者,最为理想地适应了可利用的信息(不存在更好的理由来假定相反),
他们将总是在最佳的调整路径上。

尽管穆斯在1961年就创作了他的论文,但是,在罗伯特,卢卡斯
(RobertLucas,1937一)将理性预期假设用于宏观经济学,并且将它与宏
观经济学的微观基础研究相结合之前,理性预期假设并没有在经济学中发
挥重要作用。理性预期假设直击微观基础经济学家与凯恩斯主义者之间的
折衷,因为它认为,在发展进程中人们并不使其预期适应均衡。他们能发
现基本的经济模型并立刻加以调整,这样做,对他们是有益的。假定人们
具有理性预期,那么,长期中发生的任何事情短期中也将会发生。因为在
微观基础者-凯恩斯主义者折衷中,货币政策与财政政策的有效性取决于
错误的预期,所以,理性预期假设是破坏性的。根据新的观点,如果凯恩
斯的政策在长期中是无效的,那么,在短期中它也是无效的。

20所纪70年代中期,理性预期在宏观经济学中流行开来,关于政策的
无效和凯恩斯类型的货币政策与财政政策的不易操作,也存在一些重要的
讨论。理性预期这一尚在发展中的研究,很快就以新兴古典经济学(new
classicaleconomics)而知名,原因是其政策结论与较早的古典观点相似。到
了20志纪70年代后期,对很多人来说,宏观经济学的未来似乎在于新兴古
典思想,凯思斯经济学已经死亡。

新兴古典学派对宏观经济学的持久影响之一,是它们对宏观经济建模
理论的贡献。正如将在第16章中论述的那样,在诸如简.丁伯根(JanTin-
bergen,1903一1994)和劳伦斯克芋因(LawrenceKlein,1920一)一类经
济学家的研究中,凯恩斯主义者将宏观经济模型发展到一个相当高的复杂
程度。20霸纪60年代和70年代,很多这些经济计量模型并不是对经济体


1
,
!
HN
!
!

4
!

+

1
1
'
'
1
'
1
!
1
'
'
1
L
!
!
!
1
'
未来走势的很好预报器,很多经济学家开始不再信任它们。罗伯符,户卡
斯是新兴古典学派的领袖,在一场辩论中,他详细说明了为什么这些模型
不是很好的预报器的一个理由,他的说明以经济计量模型的卢卡斯批评
(Lucascritique》变得为人所知。他认为,个人的行为取决于所预期的政策;
因此,随着一项政策变得过时,模型的结构将发生改变。但是,如果模型
的基本结构改变了,适当的政策也将改变,模型就不再适合了。因此,运
用经济计量模型预测未来政策的效果是不适宜的。

多数人的反应是改变他们对模型的看法:模型是为特定政策问题提供
见解的实际工具;存在很多种能被加以利用的不同模型,只要它们看上去
适用;没有必要使所有的模型都具有广泛的一致性。因此,现代教科书是
作为一种分析工具,而不是作为从严格的微观基础中导出的某种东西来呈
现IS-LM模型的。这种模型化方法极大地不同于新古典方法,后者在原则
上将所有的模型都看做是从微观经济学的核心假设发展而来的。
一些现代经济学家从模拟、复杂性以及代理人模型中,继续为宏观经
济学寻求基础,在这些模型中,制度特征被植人代理人内部,然后通过模
拟,人们就能发现什么样的策略幸存了下来。这一研究引发了一个新的群
体,被称作新凯恩斯主义者(newKeynesians),他们认为能够为凯恩斯经
济学开发出一种新的基础。他们推论,微观经济学的宏观基础(macrofound-
ationsofmicroeconomics)与宏观经济学的微观基础存在同样多的需要。这些
现代经济学家非常乐于接受新兴古典学派对新凯恩斯主义模型的批评,但
是他们认为,凯恩斯经济学与理性预期之间并不存在内在的矛盾性。这使
他们确信对新兴古典学派进行适当的回应,并不会得出宏观经济学更加制
度化的现实微观基础。他们认为,理解凯恩斯宏观经济学的关键,是认识
到微观经济学宏观基础的必要性。人们不能脱离宏观经济学的背景去分析
典型代理人的选择,他的那些决策是在这一背景中做出的。总量生产函数
并不能从厂商生产函数中得出,产量会由于多种原因而发生实质性的变化,
所有这些都涉及协调失灵。他们主张,个人决策视其他人预期的决策而定,
并有日,经济体很有可能陷人预期难题中。
$b
'
',
'
'
‘
:
!
!
'
!
,
4
1
!
'
'
1
'
'
!
:
!


第15章”现代宏观经济思炉的发展


因此,由理性的个人所组成的社会发现自身处于一种预期难古中,
所有的个人都在进行理性的决策,但是,个人理性决策的最终结果是社会
的无理性。根据新凯恩斯主义者的观点,理性预期假设引导新兴古典学派
得出如下结论,即货币与财政政策是无效的,除非政策能够与在全体都合
意的产量水平上所有的市场都出清这一假设相结合。但是他们指出,这是
一个特别的假设,而不是从分析中符合逻辑地得出来的某种东西。

例如,个人可能共同地预期需求将降低,并因这一预期全都生产得较
少:供给降低是因为所预期的需求降低了。除非存在一种预期的协调制度,
使得当一个人降低他或她的需求预期时,存在一些机制来抵消预期的降低
对个人供给决策产生的影响,否则,供给将会太少,因为所预期的需求太
少。正是经济体将不可避免地在全体都合意的均衡水平上实现平衡的假设,
而不是理性预期的假设,才是这些新凯恩斯主义者所不能认同的。

大多数新凯恩斯主义者的研究高度抽象且理论化,都是从抽象的对策
性模型开始并证明多重均衡是可能的。D这些抽象模型在极大程度上还没有
渗透到介绍性的和中级的教科书中,但它们最终应当会渗透进去。

凯恩斯经济学理论兴趣的复活,并不意味着以凯恩斯经济政策而著称
的东西又重新获得了它们以前的地位。在20世纪70年代,对于货币政策与
财政政策在政治上是否是有效的工具,存在着日益增多的关注,虽然它们
在理论上是有效的。很多凯恩斯主义者认为,货币政策与财政政策在政治
上不可能被利用,政治而不是合理的经济原理决定着赤字的规模与货币供
给的增长。

凯恩斯主义者与新兴古典学派之间的争论很快就变得复杂化了。对于
一门有关思想史的课程来说,并不适合对此加以考察。需要重点指出的是,
大多数现代宏观经济研究和大多数宏观经济学的研究生培训,包括了就理
解现代争论而言所必需的技术背景。
回到增长与供给
新兴古典经济学显著地影响了宏观经济学,但是,与凯恩斯宏观经济
@在迈克尔:伍德福特的太阳黑子模型《1991)以及对策性宏观经济模型,例如约翰,布束
转特的研究(1983)中,能够找到这样的例子。



驰
Wy
从
\
学相比,它并没有大量地为其理论收集黑多的证据。对于提供任何党亲米
说,经验数据是根本不充分的。那个阶段,宏观经济学家们停止考虑经济
周期间题,开始将宏观经济学集中于增长问题。这符合当时的情况,因为
美国经济在整个20世纪90年代得到成长,而没有经历一个经济周期。

对增长的分析是因回归到索洛增长模型而开始的,作为对哈罗德-多
马模型的回应,索洛模型形成于20世纪50年代。哈罗德-多马模型认为增
长是锋利的刀刃,除非是经济体相当幸运,否则它将很有可能陷入萧条。
索洛模型通过取消不变的资本/劳动比率假设,向这一结论提出挑战;它表
明经济体将会回到一条平衡的增长路径上来。经济体是稳定的,不是不稳
定的。索洛增长模型也称作新古典增长模型,它完全集中于供给;需求在
产量决定中不起任何作用。当新兴古典学派试图解释为什么国家之间的增
长率不同时,他们发现索洛模型符合他们的偏好,并因此进一步对模型加
以发展。

宏观经济学转向强调增长,改变了宏观经济学的性质。增长模型是基于
供给的模型,其中没有需求的作用。因此,增长模型变得比较突出后,饥恩
斯模型就相形见纳了。随着这些模型一步步地首先进入中级教科书,接着又
进入介绍性的教科书中,宏观经济学与饥恩斯经济学之间的联系就淡化了,
货币数量论与增长理论因此成为现代宏观经济学的关注点。古典增长理论因
新的内生增长理论(endogenousgrowththeory)而得到补充。在内生增长理论
中,技术变革不再被认为是发生在经济模型之外的某种东西;它内生于模型,
它是研究与开发投资的自然结果。内生增长理论允许收益递增战胜边际收益
递减,其结果就可能是持续的增长,并且不会发生导致静止状态的运动。因
此,它使主流宏观经济学重新加入乐观派而不是悲观派组织。

对增长的关注置换了凯恩斯宏观经济学的大部分内容。饥恩斯类型的
模型仍然被使用着,但是,乘数不再被予以强调,并且,关于需求政策的
任何论述也不再被重视。货币政策被用来阻止通货膨胀,财政政策不切实
际,实际政策所关注的仅与供给方面的激励有关。

但是,关于索洛增长模型也出现了问题它并不是完美地符合经验事
件。两方面的修改有助于解决这个问题,调整模型,用关注技术的新的增
长理论来取代它,或者回归斯察。|
4





为了理解这些发展怎样符合我们关于一种新的经济学得到发展的断言
你首先得理解,明恩斯宏观经济学从未真正符合新古典经济学。它是被介
许存在的外来的某种东西,原因是与标准的古典模型相比,它似乎更好地
满足了政策需要并解释了经济事件。

新的古典革命由于这种矛盾而向凯恩斯宏观经济学提出挑战,并试图
将宏观经济学引入微观经济学组织。在某种意义上,它是新古典思想最后
的欢呼,它成功地动了宏观经济学的理论基础,但未能使宏观经济学重
新加入到新古典组织中。它只是分裂了宏观经济学,人允许多种相互矛盾的
模型得以发展,并被用在它们适合的特定用途上。在这一新的现实中,几
平不存在凯恩斯经济学与古典经济学的分离;两者只不过是现代经济
举一一让图运用模型来了解现实一一的不同方面而已。
不断变化的对于增长、经济周期以及通作膨胀与价格水平的关注,标
志着宏观经济学的发展历史。尽管亚当斯密起初对经济增长问题感兴趣,
后来的经济学家则将他们的分析集中在收入分配上,认为经济体将由于边
际收益递减规律而最终陷入一种静止状态。他们将价格视为主要由货币数
景论决定的,并认为这种决定必须与对实际经济体的分析分开来。他们将
经济体看做是本质上自我纠正的,基本不需要政府干预。

凯恩斯的《通论》标志着经济学关注点从资源配置的微观经济问题向经
济波动的宏观经济问题的重大变革。它强调短期胜过强调长期。饥恩斯提供
了一种新的分析框架来解释经济活动水平的决定力量。他不但发现了资本主
义的内在不稳定,而且得出结论认为,市场自动运行的通常结果是在低于充
分就业的水平上形成均衡。效仿马克思、塔干-巴拉诺夫斯基、威克塞尔还
有其他一些人,凯恩斯集中研究投资支出在决定经济活动水平中的作用。

随之而来的大量文献不仅扩展并改进了最初的凯恩斯理论,而且使弛
因斯模型与前凯恩斯模型之间约差异与相似成为比较尖锐的话题。凯轧斯

pSlrgofEomDorisgty
的观点所采取的形式,导致了数学模型构建与经验检验。理论上的单从很
快紧跟着政策上的革命,因为主要的工业化国家开始制定规划,并且组建
用以扶持充分就业的机构。

宏观经济学的凯恩斯化以一种相当奇怪的方式发展着:它采取由最主
要的凯恩斯主义者,例如阿尔文*汉森与保罗萨缪尔森,提出的乘数模
型的形式发展。凯恩斯宏观经济理论的发展,加上财政政策作为政府推动
充分就业的一种补偿行为被加以运用,这些或许解释了对乘数模型的这种
关注。为了回应纯遍恩斯理论的内在矛盾,以及货币主义者提出的有关货
币作用的问题,到1960年,IS-LM模型成了支配性的宏观经济模型。

然而,随着发生在1975年左右的争论日益正式化,人们发现这一模型
对经济研究来说并不令人满意。通货膨胀和失业一样,似乎都成为当时重
要的经济话题。认同这一点的一类新文献,试图揭示宏观经济学的微观基
础,并因此模糊了凯恩斯主义的一个方面一一把经济学划分为微观经济与
宏观经济领域。随着微观基础文献的增多,争论与理论发展都回到与20世
纪30年代早期的框架相接近的某种东西上。唯一的例外是,一般均衡分析
正在日益取代局部均衡分析。最初,宏观经济学密切地与经济计量学以及
经济体的大规模模型的发展相连。尽管存在大量这样的模型然而,它们
早先的承诺还没有兑现。因此,20世纪80年代,出现了远离这种模型和纯
理论问题的运动。现代宏观经济学具有高度的折衷性,没有哪一种方法能
为所有的经济学家所认同。

此外,当前仍处于一个过渡时期,学者们肩负着广泛的研究项目,致
力于解决很多不同的问题。宏观经济学今天的主要关注点在新增长理论上,
这一理论严重地背离了较早的古典增长理论,尤其是在强调内生的技术以
及认为静目状态可以避免方面。
关键术语

解析理论analytictheory发明invention
内生增长理论endogenousgrowththeory创新innovation
企业家entrepreneursIS-LM模型IS-LManalysis
功能财政functionalfinance户卡斯批评Lucascritique
