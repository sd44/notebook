\chapter{现代宏观经济思想的发展}

\begin{quotation}
  凯恩斯说,一些东西是新的,一些东西是对的;不幸的是,新的东西不对,对的东西不新。

  \raggedleft ——弗兰克·H·奈特
\end{quotation}

多少年来,人们对宏观经济问题的兴趣起起落落,大约在19世纪末期达到最低点。当时经济
学专业对宏观经济思想的态度,可以被描述为善意的忽视。此外,确实存在的宏观经济思想
也有些混乱。马歇尔系统编纂组织了宏观经济学,并一直打算也这样对待宏观经济学,但他
未能做到。马歇尔将其对宏观经济学的论述局限于价格总水平的确定,F.W·陶西格在其介绍
性的教科书中也是如此。

曾经为亚当·斯密所关注的增长,在古典时期与新古典时期的后期,只获得了轻微的重视。
经济学专业反倒是利用李嘉图提倡的静态推论,集中于开发形式化的配置与分配模型;斯密
模棱两可的解释,败给了李嘉图更加形式化的模型。经济周期也只是短暂地被提及;全部资
源都能被充分利用的标准假定,排除了对这些问题的更多考虑。充分就业的假定经常通过引
用萨伊定律即供给创造自身的需求而被证明是正确的。

利用充分就业假定,并集中解释价格总水平决定力量的分析,一直持续到20世纪30年代,当
时的大萧条引发了关于经济周期的新研究。从20世纪30年代到70年代后期,宏观经济学持续
持续关注经济周期,并成为以“凯恩斯经济学”而知名的一种方法。这种分类并不完全正确,
因为凯恩斯的思想很快就与新古典思想相融合。教科书中详尽阐述的实际的宏观经济学,被
称作新凯恩斯经济学可能更加合适。本章描述了这一演进及其历史基础。

20世纪70年代见证了对新凯恩斯经济学的反抗,其形式是新的古典革命,它将宏观经济学的
关注点由经济周期转向增长。从20世纪90年代开始,宏观经济学前沿的主要关注点一直在增
长上。

本章将首先考察关于宏观经济问题的早期研究,然后再论及凯恩斯宏
观经济学的发展,接着介绍新的古典革命,最后考察宏观经济学现在的
状况。

\section{现代宏观经济学的历史先驱者}

现代宏观经济学主要由货币理论、增长理论以及经济周期理论组成。多年来,对这些问题的
强调一直在变动,部分原因是经济体的历程发生了变化,部分原因是技术手段允许经济学家
去涉及先前认为难以操纵的问题。我们从对增长理论的论述开始。

\subsection{增长理论的早期研究}

经济增长分析是亚当·斯密的主要关注点,他强调自由市场、私人投资支出、自由放任与经济
增长之间的关系。李嘉图重新调整了经济学的关注点,将其从经济增长转向收入分配的决定
力量。斯密与李嘉图之间在经济学本质主题这一观点上的变化,从根本上说是对经济学的重
新定位,使之远离斯密增长的宏观经济学,转向李嘉图的微观经济学事务——什么决定了工资、
地租、利润和其他价格,以及因此而来的收入分配。对微观经济学及分配问题的重视,
在19世纪第一个二十五年中从李嘉图开始,持续支配着主流经济思想,直到20世纪30年代吞
没了工业化国家的大萧条时期为止。

约瑟夫·熊彼特在其著名的关于经济思想史的著作中,在对增长进行论述时,根据经济学家对
增长的看法,将他们做了两种类型的区分乐观主义者与悲观主义者。他认为,最为主流的经
济学家属于悲观派,最强硬的悲观主义者有马尔萨斯、李嘉图以及詹姆斯·穆勒。这些主流经
济学家坚定地强调收益递减、持续提高的地租以及经济体将近进的静止状态。即使他们身边
的经济体以远快于较早时期的速度在增长,他们也这样认为。正如熊彼特所指出的:“他们
确信技术进步与资本增加,最终也不能对抗致命的收益递减法则。”

在主要的经济学家中,在这点上有些例外的是约翰·斯图亚特·穆勒,与马尔萨斯或李嘉图比
起来,他更多地讨论了增长与技术问题并且他对持续增长的可能性更加乐观。但是,通过对
穆勒著作的仔细阅读可以发现,与其说他的思想建立在技术与资本的持续增长基础上,不如
说建立在下列信念基础上,即社会最终将自动地限制出生率,从而使不可避免的边际收益
递减放慢。

晚年时,穆勒更多地是一个悲观主义者。他似乎确信静止状态就在不远处。然而,他没有将
这一结果看做是坏事。他宁愿将静止状态看成是一种具有适度繁荣与合理平等的舒适状态。
这起因于他把收入分配视为由社会以及经济力量决定的。

采取乐观主义观点的是剩下的非正统经济学家,例如,亨利·凯里(Henry
Carey,1793--1879)和弗里德里希·李斯特。在第12章中已经论述过李斯特,他是德国历史学
派的一分子,该学派强调理论的经验观察与历史。因为他能够看到经济体正以比先前更快的
速度增长,所以对他来说,认为增长有可能无限期地持续下去是很自然的。凯里是一位美国
经济学家,他不强调理论,而强调历史与经验观察。这使他得出与李斯特同样的结论:经济
体的增长似乎看不到终结。考虑到当时美国所经历的边境扩张与农业用地日益增加,在美国
的环境中不太强调收益递减就很自然了。

值得注意的是,李斯特与凯里这样的乐观主义者拥护关税,而李嘉图一类的悲观主义者通常
支持自由贸易。这种差异可能源自于他们的理论观点以及对假设的运用。李嘉图理论上的比
较优势模型,引导人们思考自由贸易政策的利益。但是,模型的静态性质也导致了下列观点,
即一旦获得了贸易收益,增长将会停止。李斯特与凯里较少关注理论,更多关注观察和历史。
对经济体的直接观察表明了技术的重要性与持续增长的可能性。它也暗示通过关税来保护技
术是很重要的。斯密认为,贸易通过扩大劳动分工和实践学习扩展了技术,因此对各方都有
益,这是一个相当复杂的观点;它是根据经济体的动态观点得出的,这一观点在正规模型中
也难以把握。

当时的主流经济学家激烈地抨击李斯特与凯里的观点,并因为指出他们理论上的错误而高兴。
但是,主流经济学家这样做,并未领会李斯特与凯里研究中的广泛经验,即通过技术开发或
许能够永久地克服边际收益递减。

随着新古典经济学的发展,人们加速远离增长这一关注点。阿尔弗雷德·马歇尔关于增长的观
点与穆勒的观点相类似,除了马软尔之外,其他新古典学者更多地集中于静态均衡上。穆勒
与马歇尔都主张技术进步能够暂时创造出增长的条件,但是,农业与原材料的收益递减法则最
终还会获胜。

20世纪上半叶的经济学家在极大程度上没有涉及增长。一个重要的例外是约瑟夫·熊彼特,
他几乎不属于任何学派。

\subsection{熊彼特与增长}

在三十岁之前,熊彼特就已经为其在《经济发展理论》中的经济增长理论奠定了基础,该书
于1912年首次出版,并于1934年被译成英语。这是一个闪光的概念,它几乎一直潜伏着,原因
是它所包含的内容如此广泛,以至于无法参与经济模型的构建,而在大约五十年中,模型构建
都是主流经济学的时尚。在给爱德华·马兹最近对熊彼特的研究作序时,诺贝尔奖获得者和模
型构建者久姆士.托宾(James Tobin)表示,熊彼特的“发展理论和经济周期理论,在过去的五
十年里,难以并入到支配经济学尤其是美国经济学的风格与方法中",具有讽刺意义的是,熊
彼特坚定地支持在经济学中更多地运用数学和对假设进行经济计量检验。

熊彼特对经济增长过程的解释并不符合正统模式,原因在于他强调增长的非经济理由。尽管他
考察了一些严格的经济因素,然而他坚持认为,导致过去增长的首要因素和将导致末来增长
的因素都是非经济的。

我们首先着眼于他对经济因素的奇特分析。他本质上认同萨伊定律,虽然他认识到并且也分
析了资本主义下经济活动的波动。在他看来,萧条是自我纠正的,低于充分就业时并不存在
均衡。熊彼特认为,萧条有益于制度,它们是经济增长整个过程的一个完整部分。增长与周
期的繁荣阶段相联,原因是这一阶段代表了将新产品和技术引入经济体后的最终成果。但是,
随着信用过度膨胀和经营过分扩张,过剩形成了。过剩所引起的萧条之所以是有益的,是因为
它通过震动打开了经济体,开除了低效率的厂商,从而为一个生长中的、健康的、管理良好
的、富有效率的厂商经济体铺好了道路。

但是,根据熊彼特的观点,经济增长的首要要素是非经济的,在社会的制度结构中能够找到这
些要素。熊彼特将工业化国家所发生的巨大增长归因于他所谓的企业家的活动,在熊彼特看
来,企业家并不仅仅是一个经营者或管理者,此人是一个独特的个体,他生来就是风险承担
者,并将创新产品和新技术引入经济体。

熊彼特清楚地区分了发明过程与创新过程。只有少数有远见的创新的经营者,能够领会一种
新发明的潜能,并为了个人的收益而开发它。但是,他们的收益就是经济体的收益。企业家
采用一项成功的创新之后,其他经营者将跟着这么做,新产品与新技术就在整个经济体中扩
散开来。因此,经济增长的真正来源,存在于创新企业家的活动中,而不是存在于企业界大
多数人的活动中,他们只是规避风险的追随者。

因此,经济增长受到一种奖赏和鼓励企业家活动的制度环境的激励;早期资本主义实行私人
财产和自由放任的政府,这很理想地适宜于经济增长。就其强调激励的重要性与自由放任的
政府而言,熊彼特的这部分分析内容,在理论上和在意识形态上与古典增长理论一致;但是,
古典理论强调资本积累规模这一经济因素,而熊彼特在其对企业家作用的分析中,则强调一
种非经济的、文化的、社会的因素。熊彼特简洁地表明了他的增长观点与主流新古典经济学
观点的对比:

\begin{quotation}
  我们将要考虑的是制度内部引起的那种变化,它转移了均衡点,结果无法从原有的均衡点,
  借助极其微小的步骤到达新的均衡点。即使你愿意不断地增加邮车,你也永远不会因
  此得到一条铁路。
\end{quotation}

熊彼特对资本主义未来增长与发展的观察更加新奇。熊彼特推测,资本主义的终结是其成功
的结果。他是一个在意识形态上保守的经济学家,对经济体的增长持有一种稍显传奇的看法,
认为增长来自于侠客式的企业家的大胆行为。他希望看到这一过程能够持续,他也期待资本
主义由于它的成功而终止。导致这一情形的主要原因是随着社会变得更加富裕而出现的企业
家的消亡,以及知识分子作用的增强。成功的企业家会推动大企业的成长,大企业通过竞争
清除了行业中那些低效率且规避风险的企业。但是,大企业很快就成了风险规避者,变得谨
慎小心,并且由官僚委员会而不是创新的企业家来管理。然后,官僚化的巨型企业将排挤掉
企业和家,用“谨慎的”管理者取而代之。随着被雇用的管理着取代了企业家,大公司的所有
权将变成缺席者所有权。“社会主义的真正领跑者,”熊彼特说,“不是知识分子或社会主义
的鼓吹者,而是范德比尔特们、卡内基们以及洛克菲勒们。”

熊彼特认为,一旦巨型企业排挤掉很多小规模的所有者经营的企业,资本主义的一大部分政
治支持者就会被去除。此外,熊彼特主张,资本主义的成功将消除私人财产的旧有概念,以
及为之奋斗的愿望。一旦企业家消失了,拿薪水的管理者以及股东将不再捍卫私人财产的概
念。他们的态度也将在工人阶级和公众中普遍流行开来。“最终将没有人真正在意是否去支
持它——大企业范围内部和外部都没有人。”资本主义的成功将再次加速这一过程,因为资本
主义所引起的收入与财富的增加,将使社会上的一个智力群体得到成长,他们“运用口头的
力量”并且“对实际事务并不负直接责任”。资本主义的成功,允许这些知识分子依靠制度
的成果来生活,但他们同时却对制度加以批评。他们使劳工运动激进化;尽管他们通常并不
竞选公职,但却为政治家做事并提出建议。偶尔地,他们也会成为政府幕僚的一分子;但最
为重要的是,随着大众传播工具的不断增多,他们将有能力在整个社会散布对资本主义制度
的不满与怨恨。

熊彼特预想他热爱的制度正缓慢但无疑地接近其终结。他担心随着企业家的死亡和自由放任
的结束,政府将越来越多地干预经济体。一些人例如凯恩斯,欢迎这样的于预,认为它是拯
救资本主义的一种方式,但是,对熊彼特来说,它是资本主义即将终结的一种征兆。由于他所
谓的“财产实体的燕发”以及企业家的消亡,熊彼特预言,经济体中解释其以往增长的动态
因素将会消失。

熊彼特博得了一些人的称赞,但并不是来自于重要的主流经济学家。其非形式化的特征,并不
符合正在成为标准的形式化的建模方法。罗伊·哈罗德(Roy Harrod)、埃弗塞·多马(Evsey
Domar)、罗伯特·索洛以及特里沃·斯旺(Trevor Swan)于20世纪50年代的研究,创作了增长的
形式化模型,正是在这些研究中,增长理论才流行开来并成为宏观经济学的核心部分。然而,
随着经济学专业开始领悟20世纪30年代的萧条,并集中于经济周期理论时,大家对这些形式
化增长模型的兴趣完全点然失色了。

\subsection{消费不足主义者的主张}

对增长理论的兴趣,与对下列问题的关注并行,即市场经济是否导致充分就业,以及政府是
否应当干预经济体,以利于保持资源的充分利用。重商主义者特别希望了解决定经济体产品
与服务生产能力的力量,并确定实际产量水平是否达到了潜在水平。很多重商主义者认识到
私人利益与公共利益之间的根本冲突,从而认为经济体将不会实现其潜在的产量,除非政府
加以干预。他们的主张是双重的:根据吉恩·伯丁的看法,第一,他们认为私人利益导致垄断,
而垄断限制了产量;第二,他们认为当个人进行储蓄或者购买外国产品时,就会引发对国内产
品的需求不足,它削弱了经济体。重商主义者的观点是,政府应当规制国内和国外和贸易,
以便使经济体出现贸易顺差,并增加作为国家货币供给的黄金数量。

随着重商主义向古典经济学演进,对政府干预的态度也发生了急剧的变化。与早期重商主义
者不同,亚当·斯密认为,竞争性市场的力量如此强大,以至于私人利益被引导着为公共利益
服务,就像被一只“看不见的手”引导着一样。只有当政府遵循自由放任的政策,经济体才
能实现其潜在产量。斯密赞同自由放任,他的分析是一种前后关联的主张,是在考虑到可行的
多种选择时做出的。他赞同重商主义者关于垄断减少了产量的观点,但是他声称,旨在控制垄
断的方法——政府对垄断交易和分配的控制——只能使事情更糟,而不是更好。因此他认为,更
可取的政策是依靠自由放任与竞争,使资源尽可能充分地被利用。允许市场繁荣的自由放
任政策鼓励劳动分工、专业化以及技术发展,从而鼓励增长。市场创造增长这一倾向,也为
马克思所强调。但是,马克思从分析中得出了完全不同的合义,他集中研究了伴随增长的不
平等的收入分配,及其对政治结构和社会结构的可能影响。

斯密与其他古典经济学家反击了重商主义者的消费不足主张,他们认为,储蓄将自动被转换成
投资支出,原因在于储蓄决策就是投资决策。自由放任的经济体将自动引起资源的充分利用,
这一命题被称作萨伊定律,它成为古典与新古典思想的重要成分。古典经济学家也抨击了重
商主义通过实行贸易顺差来增加黄金储量的主张,认为一国的财富不是用贵金属而是用实际
产量来衡量的,并且,允许自由贸易的国家将变得富裕,从而获得对外竞争优势。

古典经济学家尤其是斯密与约翰·斯图亚特·穆勒承认,市场力量并不是完美地发挥作用,但是
他们主张,与可供选择的其他方案相比,市场能更好地发挥作用。从1800年直到1930年,除
了托马斯·马尔萨斯之外,对经济周期的分析都留给了非正统和非主流经济学家,例如,米哈
伊尔·杜冈--巴拉诺夫斯基(Mikhail Tugan-Baranowsky,1865一1919),约翰·A·霍布斯。能
够依靠市场来控制经济体的古典信仰,将经济研究的关注点由货币与财政力量转向了实际力
量,对宏观经济问题的古典分析普遍认同实际力量与名义力量之间的分离。


\subsection{货币数量理论}

古典与新古典经济学家至少在同一个宏观经济问题上有兴趣;什么决定了价格总水平?他们
通过运用在微观经济理论中形成的供求方法,致力于研究这个经济问题。货币供给被假定为
由货币当局决定,所以,一些正统经济学家主张,所要分析的基本问题在需求方面。家庭与
厂商被假定是理性的,为了不同的使用目的而对货币有和需求。瓦尔拉斯、门格尔,还有其
他一些人创造了一种供求分析来解释货币的价值,但是,这些理论中最为著名的可能是马歇
尔发展的理论,它因货币数量论的剑桥现金余额学说而知名。

大卫·休谟在1752年就对货币数量理论做了最早的清晰陈述。正如通过文献流传下来的那样,
该理论认为价格总水平取决于流通中的货币数量。马歇尔的货币数量理论,试图为价格与货
币数量直接变动这一宏观理论提供微观基础。他通过详细阐述家庭与厂商行为理论做到了这
一点,从而解释了对货币的需求。马歇尔推论,家庭和厂商愿意在其货币收入中持有一小部
分现金余额。如果 $M$代表货币(现金加上活期存款),$PY$代表货币收入,$k$为家庭与厂商
愿意以货币形式持有的收人所占的比例,那么,基本的现金余额方程是:
\[ M = k \cdot PY\]

因为马歇尔承认萨伊定律,所以假定充分就业。假定 $k$保持不变,货币数量的增加将引起
货币收入 $PY$的增加。因为假定了充分就业,所以货币数量的增加将导致较高的价格,以及
随之发生的货币收入的增加;然而,实际收入并没有改变。货币数量的减少将导致货币收入
随价格下降而减少,实际收入也将保持不变。我们不去考察马歇尔这一表达的很多不同方面,
重要的一点是,马软尔的货币数量理论,在使最大化厂商与家庭的微观经济行为与价格总水
平的宏观经济问题相结合方面,进行了一次尝试。

一组经济学家,其中最突出的是美国人欧文·费雪,发展了另一种形
式的货币数量论,该理论因交易学说而知名。然而,他们对为价格总水平
的宏观经济分析寻找一种微观基础只表示出了很小的兴趣。在这一学说中,
\[ MV = PT \]
这里 $M$是货币的数量, $V$是货币的速度,$P$是对价格水平的度量,$T$是交易量。

尽管这两种方法具有重大的差异,但是,它们有一个共同因素:两者都是为了解释价格总水平
的决定力量。它们不是用来解释实际收入水平的,实际收入被假定为在充分就业水平上,且
因经济体中的非货币力量而固定。

并不是所有的经济学家都满意于这一分析。例如,纳特·威克塞尔认为,货币数量理论未能解
释“在既定条件下为什么对产品的货币或金钱需求,超过了或者达不到产品的供给”。威克
塞尔试图发展一种所谓的收入方法来解释价格总水平,也就是说,发展一种解释收入波动以及
价格水平波动的货币理论。

\subsection{经济周期理论}

尽管从商业资本主义开始,经济活动的、收入水平的以及就业的波动就一直发生着,并为正
统理论家所承认,但是,19世纪90年代之前,经济学家并没有进行系统的尝试来分析萧条或者
经济周期。非正统理论家则以较多的精力从事着对这些问题的研究。因此,直到19世纪的最
后十年,正统经济理论都是由下列成分组成的,即发展相当充分的解释稀缺资源配置与分配的
微观经济理论结构、解释价格总水平决定力量的宏观经济理论,以及一组关于经济增长的松
散的观点。1890年之前,正统理论“对萧条与周期的研究一直都是边缘的和离题的”。

这种概括的一个主要例外是克莱门特·朱格拉(Clement Juglar,1819--1905)的研究,他
在1862年出版了《论法国、英国、美国的商业危机及其周期》。这部著作的第二版出版
于1889年,运用历史和统计材料进行了相当大的扩充。朱格拉是韦斯利·克莱尔·米切尔精神
上的前辈,原因是他并没有构建经济周期的演绎理论,而是宁愿收集历史和统计材料,并用
归纳的方法加以处理。他的主要贡献是,他声称周期不是经济体系外部力量而是内部力量的
结果。他认为周期包含三个阶段,这三个阶段按照过续的顺序反复出现:

\begin{quotation}
  繁荣、危机、清算阶段,尽管受到人们生活中幸运和不幸事件的影响,然而,它们并不是
  偶然事件的结果,而是起因于全体居民的行为与活动,最重要地起因于他们的储蓄习惯,以
  及他们使用可利用的资本与信用的方式,
\end{quotation}

尽管朱格拉的著作发起了对经济周期的研究,然而,有关波动的现代正统宏观经济分析是以
俄国人米哈伊尔·杜冈--巴拉诺夫斯基的著作为基础的。米哈伊尔·杜冈-巴拉诺夫斯基的《英
国的产业危机》一书于1894年首次用俄语出版,紧接着是德语版和法语版。在对以往解释经
济周期的尝试做出评论之后,米哈伊尔·杜冈--巴拉诺夫斯基断言它们都不能令人满意。对于
理解经济周期来说,他的主要贡献是阐述了两条原则:(1)经济波动内在于资本主义制度,
原因在于,它们是制度内部力量作用的结果;(2)在投资支出的决定力量中将会找到经济周
期的主要原因。凯恩斯的收入决定分析,强调资本主义的内在不稳定以及投资的作用,这一
分析的源泉来自于杜冈-巴拉诺夫斯基、朱格拉、斯皮托夫(Spiethoff)、熊彼特、卡塞尔、
罗宾逊、威克塞尔和费雪,以及凡勃仑、霍布斯、米切尔等人。

一些重商主义者、重农主义者以及随后的很多非正统经济学家,早就提出资本主义具有引起
萧条的内在力量。即使主流经济学家对周期的考察,例如,杰文斯的日斑循环,通常也被忽
视了。自1900年之后,正统理论家就经济周期进行了比较认真的研究,但是,足以令人奇怪的
是,这些研究的存在伴随着一种持续的基本观念,即经济体的长期均衡位置提供了充分就业。
因此,我们看到像弗里德里希·冯·哈耶克一类的经济学家,将总量波动问题作为一种协调失
灵来探讨,同时对市场经济的自我平衡性质保持着坚定的信念。无论是正统的还是非正统的
经济学家,没有人能够挑战这一信念,原因在于,没有人能构建起一种收入决定理论,来表明
在低于充分就业的水平上均衡也是有可能存在的。当约翰·梅纳德·凯恩斯于1936年形成了一
种理论,认为在低于充分就业的水平上均衡可能存在时,正统宏观经济理论的一个新阶段开
始了。

\section{凯恩斯主义宏观经济学}

凯恩斯的父亲约输·内维尔·凯恩斯凭借自身的资历也成为一位重要的经济学家。然而,儿子
的成就很快就超越了父亲。在这一点上和其他一些方面,约输.梅纳德.凯恩斯的生活颇似约
翰·斯图亚特·穆勒。两个人的父亲都是优秀经济学家的同辈与朋友:詹姆斯·穆勒是大卫·李
嘉图的朋友,约输·内维尔·凯恩斯是阿尔弗雷德·马歇尔的朋友。小凯恩斯与小穆勒都接受了
特别为知识分子的孩子们提供的高质量的教育,这种教育训练他们天生敏锐的头脑去开辟新
天地,并借助他们的创作力量去说服其他人。

穆勒与凯恩斯都否定了父辈们的经济学的政策含义,着手于新的方向。但至此为止,两个人
的相似之处结束了,对约翰·斯图亚特·穆勒来说,他没能完全与其父亲和李嘉图的理论结构
绝交;最终在古典与新古典理论之间采取了一种折衷。凯恩斯与过去的绝交——与贯穿于斯密、
李嘉图、约翰·斯图亚特·穆勒以及马歇尔的自由放任传统的绝交——则要更加完全。尽管他详
熟基本的马歇尔局部均衡分析,然而,他构建了一种新的理论结构,致力于研究对经济理论
与经济政策都具有重要影响的总量经济。

20世纪那些思维狭窄的经济学家的一成不变并不符合凯恩斯。事实上,他由于将太少的时间
用于经济理论,并且因为过于广泛地分散其兴趣而受到批评。即使作为伊顿公学和剑桥大学
的学生,他也显示出兴趣广泛的倾向;因此,他以业余艺术爱好者而为人所知。完成教育后,他
进入英国政府的印度事务部做文职人员,在重返剑桥之前他在那里干了两年。他从未专职做
过教师。他对经济政策的持续兴趣,使他在一生中担任了很多政府性职务。他积极参与经营性
的事务,既是为了自己,也是因为他是国王学院的会计,他的经营能力通过下列事实表现出来,
即他的个人净资产从1920年的几乎破产,到1946年他去世时,增加到超过两百万美元。凯恩斯
对戏剧、文学还有芭蕾舞颇感兴趣;他与一位芭蕾舞女演员结婚,并且与一群以布卢姆茨伯
里派而知名的伦敦知识分子有联系,这个圈子包括诸如克莱夫·贝尔(Clive Bell)、E.M·福斯
特、利顿.斯特雷奇以及维吉尼亚·伍尔夫(Virginia Woolf)等一类的名人。作为一名大学本
科生,他独特的多种才华使他能够成为一位学有成就的数学家,创作出关于概率理论的著作,
并且成为一位颇有影响的令人印象深刻的散文家,他的《和约的经济后果》和收于《劝说集》
与《自传文集》两本书中的散文所体现出的对文字的彻底精通,清楚地显示了这一点。

作为经济学家的凯恩斯,一个最重要的方面是他对政策的倾向性。他作为英国财政部的代表,
参加了凡尔赛和平会议,但是1919年他又突然辞去该职务。他对凡尔赛条约的条款感到厌恶,
该条约强加给德国大量的赔偿,凯恩斯认为德国永远无法支付。由于在1919年出版的《和约
的经济后果》中对条约的条款进行批评,凯恩斯获得了国际上的称赞。1940年他创作了《如
何为战争付账》,1943年他向国际货币当局提出了一个被称为凯恩斯计划的建议,该建议在
第二次世界大战后付诸实施。作为前往布雷顿森林的英国代表团的首脑,他对国际货币基金
组织和世界银行的成立起了作用。但是,他对政策与理论的最重要的贡献,包含在他的《通
论》(1936)一书中,该书创造了现代宏观经济学,并且仍然是大学本科宏观经济学大部分所
讲授内容的基础。保罗·萨缪尔森反思凯恩斯时代时,捕捉到了它的重要性,他写道:“有
一种疾病最初袭击并杀死了大批孤立的南海岛居民部落,《通论》以这种疾病意想不到的致命
性,感染了大多数三十五岁以下的经济学家。”

\subsection{通论的关联性}

在经济理论书籍中,可能没有哪一本书拥有比凯恩斯的《通论》更自行其是的第一章。的确,
其他经济学家都在声明自身的原创性和才华,但是,凯恩斯仅凭借一种气势这样做,并足以使
他的声明令人心悦诚服。缺少谦逊显然可以追溯到凯恩斯的年轻时代。当他刚从学院毕业参
加文职考试,在经济学上并没有得到最高分数时,他的反应是:“显然我比主考官了解更多
的经济学。”在凯恩斯创作《通论》的过程中,他写信给乔治·伯纳德·肖说他正在创作一本
新书,这本书将会使世界考虑经济问题的方式发生革命。《通论》的第一章只有一个段落那
么长。在此,凯恩斯简单地表明他的新理论在下列意义上是一种通用的理论,即先前的理论
是能够放入其更加通用框架中的一种特殊情形。凯恩斯所说的“先前的理论”,既指古典经
济学也指新古典经济学,他将其界定为李嘉图的经济学(因为它符合萨伊定律),以及遵循这
一信念的那些人的经济学——约翰·斯图亚特·穆勒、马歇尔、埃奇沃斯还有庇古。

尽管作为经济学家,凯恩斯最重要的一个方面是他对政策的倾向性,然而,他最重要的作品
《通论》虽然具有政策含义,在本质上却是一部理论著作,其主要读者在职业经济学家中才
能找到。凯恩斯写道“本书主要是为我的同仁经济学家们创作的。我希望对其他人来说,它
也是易于理解的。但是,它的主要目的是处理理论中的难题,将这一理论应用于实际只是第二
位的。”

通过了解凯恩斯运用理论的方式,我们能够调和这一表面上的矛盾。很多经济理论可以被称
为具有非关联性;即它们是在一种制度真空中发展起来的。借助演绎逻辑能充分地理解这种
理论;它们从最初的原理开始,依据经过仔细表述的假设,由此推导出结论。进行这些假设
时,不是去考虑现实,而是试图理解假设之间相互作用的内在逻辑。这种理论可以被称为解
析理论。正确地完成了推导的一般均衡分析就是一种解析理论。因为假设不可避免地会远离
现实,所以,从较宽范围的解析理论中得出政策结论就相当复杂。

凯恩斯运用了一种不同的理论,它可以被称为“现实解析的",因为它是现实方法与解析方法
之间的一种折衷。现实解析理论是前后关联的,它将关于经济体的归纳性信息与演绎逻辑相
混合。现实引导着对假设的选择。现实解析理论很少天生地令人满意,但是因为它们密切地与
现实相符,所以,比较容易从中得出政策结论。在《通论》中凯恩斯并不是从最初的原理开
始,而是运用现实指导他对假设的选择。因此,尽管他集中于理论,然而他从未忽略理论的
政策含义。

举一个例子可能会使现实解析理论与解析理论之间的区别更加清楚。凯恩斯假定价格与工资
相对不变,而没有试图证明这些假设是正确的。尽管他在《通论》中简短地讨论了弹性价格
的含义,认为它们没有解决失业问题,然而对他来说,很少关注对弹性价格含义的彻底考察,
就所要解决的问题而言——针对失业该做些什么——假定工资与价格不变就是合理的。通过
运用其现实解析方法,他能够做到这一点,而真正的解析模型则不允许这种假设。凯恩斯把
为其理论提供一种解析基础这项工作留给他人去完成。宏观经济思想后来的大多数发展,都
在试图为宏观经济学提供一种解析基础。

凯恩斯起初利用货币数量理论来讨论周期性波动,以此完成了他的两卷本《货币论》,之后
他立即开始写作《通论》,在《通论》中,凯恩斯放弃了这种方法,这让曾与他有过亲密合
作的同事丹尼斯·罗伯逊(Dennis Robertson)非常懊恼。 凯恩斯采用了一种简单的新方法,
专注于储蓄和投资之间的关系。 为给自己提供一个更大的目标,凯恩斯把新古典主义的非均
衡货币方法和早期的经典方法混为一谈,夸大了他们的信仰,并将它们统称为“经典理论”。
在这样做的过程中,他这样做,仿佛是在创造一幅古典思想的讽刺漫画,强调它与他的新方
法的不同,但隐藏了许多微妙之处。

在教科书的多种模型中,凯恩斯经济学变得具体化了,这些模型被称做乘数模型(有时叫
做AE/AP模型)、IS/LM模型以及AS/AD模型。整个20世纪80年代,这些模型都是宏观经济学所
讲授内容的核心,并且仍然出现在很多最近出版的大学本科教科书中。但是,随着凯恩斯经
济学的失宠,前沿的宏观经济学在极大程度上按照不同的方向行进。

\subsection{凯恩斯乘数模型的兴起:1940--1960}

20世纪40年代和50年代,经济学家探讨乘数(multiplier)模型,并以极度的细致来开发它。
它被扩展到包括国际效应、不同类型的政府支出以及不同类型的个人支出。诸如平衡预算乘
数这样的术语,成为经济学术语的标准构成,每个学经济学的学生都得学习凯恩斯模型。

值得注意的是,过去和现在通常被称作凯恩斯模型和凯恩斯货币与财政政策的东西,并不能
在凯恩斯的书中找到。《通论》中没有一个图表,也没有任何关于货币与财政政策用途的论
述。那么,乘数模型(通过代数和几何完成的)是怎样成为20世纪50年代宏观经济争论的焦
点的?部分原因是,与其他选择相比,它似乎提供了对现状的一种更好描述。但是,其他因
素也产生了作用。关于凯恩斯经济学有效性的最初政策争论,集中在财政政策上(战争期间
的政府赤字,显然将西方世界从大萧条中拽了出来)。因为乘数模型很好地捕捉到了财政政策
的效应,所以它便倾向于成为凯恩斯模型。我们猜想,在这一模型最初的采用和长期的认同
中,社会原因也发挥了作用。对真理的需要通常被专业的其他需要加以调和——明确地说是教
学需要以及在刊物上发表文章的需要,乘数模型完类地适应了那些需要。

乘数分析是在美国流行开的。保罗·萨缪尔森与阿尔文·汉森将其发展为主要的凯恩斯模型。
萨缪尔森的教科书将它引入教学,其他书复制了萨缪尔森的教科书,乘数模型很快就成了凯
恩斯经济学。乘数分析具有很多教学上的优势,便于讲授和学习。它通过为宏观经济学提供
一种核心的分析结构,而使其作为一个单独的领域得到发展,就像供求分析之于微观经济学
一样。

20世纪30年代的经济大萧条,改变了社会和经济学家观察市场的背景。在此之前,赞同自由
放任的新古典主张,不仅是经济理论的依据,而且是关于政府的哲学判断与政治判断的依
据。20世纪早期,除了激进分子之外,几乎所有人的一般政治取向,都是反对经济体中大量的
政府参与。在当时的环境中,我们现在认为理所应当的很多政府项目,例如,社会安全和失业
保险,看上去都是很偏激的。

随着大萧条的开始,人们的态度也开始改变。很多人觉得,如果自由市场能引起大萧条期间
那样的经济危难,那么,该是开始考虑其他选择的时候了。随着经济学家开始更加详细地分析
总量经济,很多人对他们的政策处方变得不太自信了,而且更多地意识到新古典理论的缺点。
结果,经济学家开始提倡多种政策建议来解决与其主流新古典观点不一致的失业问题。例
如,20世纪30年代初期,英国的亚瑟·C·庇古和美国芝加哥大学的一些经济学家,就主张公共
建设工程项目以及赤字,以此作为对抗失业的一种手段。

\subsection{凯恩斯的政策}

凯恩斯经济学包含政策观点,并形成了一种模型,该模型将其对激进政府政策的需要植人其
中。在这一模型中,总需求控制着经济体的收入水平,政府通过货币政策与财政政策控制总
需求。

20世纪50年代至60年代期间,凯恩斯的政策意味着通过货币政策与财政政策进行调整。阿
巴·P·勒纳(Abba P.Lerner,1903--1982)在引导凯恩斯的分析朝着这种方向调整方面具有
影响力。在其《控制经济学》(1944)中,勒纳鼓吹政府不应当遵循合理财政政策(预算总
是平衡);而应当遵循功能财政政策,只考虑政策的后果,不考虑政策本身。

功能财政允许政府“驾驭”经济体;用一个经常重复的比喻来说,货币政策与财政政策被比
喻为政府的方向盘。勒纳主张,财政政策与货币政策是政府应当加以运用,以实现其宏观经
济目标——高就业、价格稳定以及高增长的工具。赤字规模完全无关紧要:如果存在失业,政
府应当增加赤字和货币供给;如果存在通货膨胀,政府的做法则应当相反。

勒纳关于“凯恩斯主义者”观点的生硬表述,触动了很多凯恩斯主义者的敏感性,并引起了
相当多的讨论,其至引起凯恩斯自己否认凯恩斯主义。埃弗塞.多马是当时一位有名的凯恩斯
主义者,他说:“其至连凯恩斯主义者,一听到勒纳关于赤字规模方面的无关紧要的观点,
也都退缩了并确定地说:不,是他(勒纳)弄错了。”但是,凯恩斯很快就改变了主意,并赞同
勒纳的看法,像经济学专业中的大多数情况一样,没过多久,凯恩斯经济政策就与功能财政
同义了。

此外,货币政策与财政政策在政治上是合意的。很多经济学家和一些人认为,大萧条证实了政
府在引导经济方面应当显示出更大的作用。货币政策与财政政策的运用将这一作用保持在最
低限度。市场能够像以前一样自由地运转。政府并不直接决定投资水平,它只是通过管理赤
字预算或盈余预算,见解地影响总收入。在很多人看来,赤字的合法化具有另一个合意的特
征,它介许政府不征税就支出。

\subsection{凯恩斯政策的哲学路径}

政策将理论与规范性的判断结合起来。因此,理解凯恩斯革命,就需要考察当时的经济学家
尤其是凯恩斯的一般哲学观点。凯恩斯不是激进分子,虽然在《通论》出版之后,他被指责
为是激进分子。我们不能指望拥有他那样的背景、教育、经历的人,赞成对其所处社会的制度
结构进行激列的变革。就其有关改变社会结构的观点来说,凯恩斯基本上是保守的,他通常只
是提倡那些能保持资本主义本质因素的变革。他的观点是,如果制度最坏的缺点不消除,那
么,个人将会放弃资本主义制度,并且失去的远大于他们获得的。

凯恩斯乐于承认,社会组织的这些变化可能会解决一些经济问题,但是,他觉得这种解决办法,
只能以个人主义及其经济上和政治上的优势为代价来换取。个人主义的经济优势,来自于利
用私利以实现更大的效率与创新,经济学家都充分了解这一点:

\begin{quotation}
  但是,首要的是,如果能够清除掉个人主义的缺点与弊端,那么在下列意义上,它就是对
  个人自由的最佳保障,即与任何其他制度相比,它极大地扩展了进行个人选择的领域。它
  也是对生活多样性的最佳保障,这种多样性正好缘自于扩大了的个人选择领域,并且,多
  样性的损失是生活单调或极权国家所有损失中最大的。
\end{quotation}

凯恩斯关于美好社会结构的宽泛哲学观点,引起了来自两个方面的抨击。左翼认为他是资本
主义及自身阶级的辩护者;右翼则将他看做是狂暴的社会主义改革者,致力于瓦解资本主义
制度。他对来自右翼批评的反应至少比较温和。他写道:“因此,虽然政府职能的扩
大……恐怕要认为是对个人主义的令人恐怖的侵犯,然而我为之辩护。认为这是唯一切实可
行办法,它既可以避免现行经济状态之全部毁灭;又是必要条件,可以让私人策动力有适当
应用。”凯恩斯发现,资本主义的主要好处之一是它给予个人主义自由的发挥。他认为,确
实来自个人主义的弊端,不用摧毁资本主义就能得到纠正。他说,资本主义的主要缺点或错
误,“是它未能提供充分就业以及其武断的不平等的财富与收入分配”。

20世纪30年代的大痕条使很多经济学家确信,未能提供充分就业是资本主义的一个主要缺陷。
第二次世界大战后的经济学家所面对的一个主要问题是:我们能用什么政策来保持资本主义
最好的东西,同时防止大的萧条?一开始,对很多美国人来说,凯恩斯的政策观点似乎过于自
由。凯恩斯主义者所提出的货币与财政政策,最终被美国经济学家所信奉,原因在于,它们只
需要少量的政府对经济体的直接干预。然而,这些政策受到将凯恩斯主义者视为社会主义者
的人的抨击。劳瑞·塔西斯(Lorie Tarshis)体会到了这一点,他写了第一本凯恩斯主义的介
绍性教科书,但一个保守派团体发起了一场运动,阻止男校友向使用塔西斯教科书的任何学
校进行捐助,并迫使塔西斯任教的斯坦福大学将其解雇。塔西斯介绍性的教科书在商业上未
获成功,但是,萨缪尔森的教科书紧随其后,并获得了极大的成功,被广泛地加以模仿,部
分原因是它用一种科学的形式遮蔽了凯恩斯的经济学,从而避免了那种摧毁塔西斯的意识形
态上的择击。

\section{现代宏观经济学}

\subsection{货币主义}

整个20世纪50年代和60年代,货币主义者都是凯恩斯主义者的主要陪衬。在米尔顿·弗里德曼
的领导下,他们对凯恩斯的政策与理论提出了有效的反对。凯恩斯主义者于20世纪50年代所
使用的消费函数模型,没有考虑货币的作用,也没有考虑价格或价格水平。初期缺乏对货币
供给与价格的关注,体现在以凯恩斯分析为基础的政策中。在第二次世界大战期间与财政部
签署的一项协议中,联邦储备银行同意购买任何必需的债券,以将利率维持在固定水平上。联
邦储备银行这样做,放弃了对货币供给的所有控制。货币主义者认为,货币供给在经济体中
起着重要的作用,不应当被局限于保持利率不变这一种作用上,因此,早期货币主义者的战斗
口号就是“货币至关重要”。

凯恩斯主义者很快就乐于赞同货币主义者关于货币至关重要的看法,但是他们认为,在相信
唯有货币至关重要这一点上货币主义者与他们不同。借助于IS--LM凯恩斯主义--新古典的综
合,争论得以解决,在IS--LM中货币主义者假定一条非常缺乏弹性的LM曲线,而凯恩斯主义
者则假定一条非常富有弹性的LM曲线。因此,至少就教科书所呈现的内容而言,货币主义分
析与凯恩斯主义分析,在通常的新凯恩斯IS--LM模型中走到了一起,只是对一些参数的看法
略有不同而已。

现代宏观经济学是经济学家完成新凯恩斯主义模型,并发现很多问题的结果。这些问题中,
一些纯粹是理论上的,一些则随着新凯恩斯主义政策的失败而变得明显。

\subsection{IS-LM模型中的问题}

IS--LM模型(IS-LM analysis)一直是大多数宏观经济学家工具箱中的组成部分;它提供了大多
数经济学家最初用来应对宏观经济分析的框架。然而,到了20世纪60年代,它在文献中得到
了充分探讨,并被认为在若干方面存在欠缺。

第一,它迫使分析在比较静态均衡框架中进行。在很多经济学家看来,凯恩斯的分析关注于,
或者说本来关注于调整的速度。他们认为,凯恩斯主张收入调整机制(乘数)的发生速度要快于
价格或利率调整机制。比较静态分析失去了凯恩斯研究中这一方面的内容。

第二,在IS--LM模型中,实际部门与名义部门的相互关系通过利率而发生,不能通过其他渠道
发生。货币主义者对这一点不满意,因为他们认为货币能够通过多种渠道影响经济体。很多
凯恩斯主义者则对框架不满意因为它几乎没有使通货膨胀问题更清楚,而通货膨胀同题
在20世纪60年代开始被看做是一个严重的经济问题。

第三,用来得出LM曲线的货币分析,并不是以一般均衡模型为基础的,而是以一种相当特别
的方式加以假定的。它并没有真正地将名义部门与实际部门相结合,因为它没有捕捉到货币
与财政部门的真正作用,所以,它使这些部门的功能平凡化。当大多数经济学家实际上认为,
下降的价格水平将使事情更糟而不是更好时,它使价格水平的下降看上去好像能够产生均衡
似的。不过,IS--LM模型得到了采用。它整洁,能很好地服务于教学功能,是一种粗糙但尚
能使用的工具;它提供了关于经济体的普遍正确的见解,并且是可以利用的最好模型。

然而,对现有模型的不满意,使很多宏观经济学家在其研究中转向其它模型。这导致了一种
分裂。虽然IS--LM模型一直是20世纪70年代和80年代主要的大学本科模型,然而,研究生阶段
的研究则开始集中完全不同的问题。到了20世纪90年代早期,关注点的变化渗透到大学本科
课程中。宏观经济学的现代理论争论与IS--LM曲线形状没有太大关系。它们反而是从一种微
观经济角度接近宏观经济问题,并且涉及数量与价格的调整速度。在某种意义
上,20世纪70年代和80年代的很多宏观经济研究者认为,我们应当跳过凯恩斯的IS--LM这一
段,回归到20世纪30年代存在的宏观经济争论中,当时,问题都是用微观经济术语来设计的。
因此,从20世纪70年代开始,我们看到了对凯恩斯经济学的反抗。

\subsection{现代宏观经济学的兴起}

货币主义者对通货膨胀的关注,使之在20世纪70年代随着通货膨胀的加剧而处于显要地位。
当这一切发生时,凯恩斯的政策与理论便失宠了。财政政策在政治上被证明难以实现;关于支
出和税收的决策,是基于其宏观经济后果之外的理由而做出的。货币政策就成了佼佼者,但
是,凯恩斯模型没有将货币政策的潜在通货膨胀效应包含进去,因而并不完全适合于对货币政
策的讨论。因此,为了设计政策,出现了远离凯恩斯经济模型的运动。

同时,出于理论上的理由,也出现了远离凯恩斯模型的运动。当经济学家试图为这些模型发
展微观基础时,他们发现,在标准的一般均衡微观经济方法的环境中,他们做不到这一点。
发展微观基础的这种愿望,应当被给予一些评论,原因在于,它对于理解下列运动来说是很
重要的,这一运动即远离新古典经济学,靠近现代形式主义折衷的模型构建经济学。

凯恩斯的宏观经济学并不符合新古典模式,因而可以被看成是远离新古典,朝向表现现代经
济学特征的折衷主义所迈出的一步。它开始于总量的相互关系分析,而不是由最初的原理去
发展这此关系。因此,它总是具有一种无关紧要的理论上的存在性,它的主要作用是作为一
种粗糙但尚能使用的政策指导。整个20世纪50年代和60年代,松散的微观基础被添加到宏观
经济学中,它们看上去是适合的,但是,没有人试图由最初的原理去发展宏观经济学模型。宏
观经济学只不过还在那里——一种与瓦尔拉斯理论几乎没有直按联系的单独的分析,而瓦尔
拉斯理论是理论微观经济学的核心。

\subsection{宏观经济学的微观基础}

20世纪70年代,经济学家为了修补这个问题,开始尝试看使凯恩斯模型适合新古典一般均衡
模型,以此来奠定宏观经济学的微观基础。他们之所以这样做,有两点原因:第一,为了理论
上的完整性;第二,为了能够扩展模型,以便在分析中包含通货膨胀。但他们这么做时发现,
当把标准的新古典原理应用于凯恩斯模型时,凯恩斯模型坍塌了。凯恩斯宏观经济学,即教科
书中传统的宏观经济学与所讲授的微观经济学彼此予盾。

有关微观经济基础的文献,确立了新的方式来考虑失业。凯恩斯主义的分析将失业描绘成一
种个人无法获得工作的均衡现象,而微观基础文献则把失业描述为一种暂时的现象——离开
工作的工人和参加工作的新工人的流动相互作用的结果。它认为部门之间的流动是失业的一
个重要原因,并且这些流动是动态经济过程的自然结果。对于宏观经济学的新微观基础方法
来说,失业是一个微观经济问题,而不是一个宏观经济问题。

微观基础经济学家认为,为了理解失业与通货膨胀,经济学家必须着眼于个人与厂商的微观
经济决策,并将那些决策与宏观经济现象相连。搜索理论研究不确定条件下个人的最佳选择,
像很多新的动态调整模型一样,它成了宏观经济学的一个中心论题。随着研究者开始越来越
多地集中于这些模型,他们越来越少地关注IS--LM模型。最初的微观基础模型是局部均衡模
型,但是,一旦打开了微观基础的盒子,经济学家就需要从中得到一些将不同市场联合起来
的方法。明显的选择是运用一般均衡模型。因此,我们在第14章中看到的成为微观经济学重
要模型的一般均衡分析,连同微观基础文献一起,都被引导进宏观经济学中。

20世纪70年代早期,微观基础文献因准确地预测到了通货膨胀而被灌输进经济学专业的意识
中。微观基础方法的提倡者以理论为根据,认为菲利普斯曲线——表明在通货膨胀与失业之间
权衡的一条曲线——只是一种短期现象,一旦通货膨胀被加入预期中,失业与通货膨胀的权衡就
会消失。长期菲利普斯曲线接近垂直,经济体倾向于一种自然失业率。

新微观基础方法的政策含义相对突出。它消除了政府通过扩张性的货币政策与财政政策影响
长期自然失业率的潜力。这种尝试通过暂时欺骗工人而在短期中见效,但是,扩张性的政策
在长期中只会导致通货膨胀。根据新的微观经济学观点,政府将失业降低到自然率之下的做
法,是20世纪70年代后期通货膨胀的原因。

然而,凯恩斯的货币政策与财政政策并没有完全被排除。至少在理论上,它们仍然能够被用来
暂时消除波动。因此,20世纪70年代早期,在凯恩斯主义者的经济学和宏观经济学微观基础
方法提倡者的经济学之间,出现了一种折衷:在长期中,古典模型是正确的;经济体将倾向于
自然失业率。然而,在短期中,因为假定个人缓慢地调整其预期,所以,凯恩斯的政策能够具
有一定的效果。

\subsection{新的古典经济学的兴起}

20世纪70年代中期,理性预期这一术语自先出现在宏观经济学视野中。理性预期假设是查尔
斯·C·霍尔特(Charles C.Holt,1921--)、弗兰科·莫迪利安尼 (Franco
Modigliani,1918--)、约翰·穆斯(John Muth,1930--)以及赫伯特.西蒙(Herbert
Simon,1916--)所进行的微观经济分析的副产品,这些经济学家试图解释,为什么很多人似
乎并不按照新古典经济学假设他们将遵循的方式来使其行为最大化。他们的研究有意要借助
动态模型来解释西蒙所谓的“满意度”行为,即为什么厂商的行为与微观经济模型不相符。
约翰·穆斯认为应优先进行这项研究,他写道:

\begin{quotation}
  人们有时认为,经济学中的理性假设导致理论与所观紧的现象,尤其是与不同时期的变化
  相矛盾,或者导致理论没有得到充分的解释。……我们的假设正好依据相反的观点:动态经
  济模型并不假设充分的理性,
\end{quotation}

穆斯主张建模时做那种假定是合理的,原因在于,预期是对未来事件有根据的预报,它们本
质上符合相关的经济理论。正如西蒙所述:“(穆斯)解开了难题。他不是通过详细阐述决
策过程模型来涉及不确定性,而是坚决地——如果他的假设是正确的——使过程不相关。”

运用其“动态理性”假设,穆斯将非均衡转化为均衡。正如新古典学者运用理性来确保静态
的个人最优性,或者确保个人向他或她的预算线与无差异曲线的切点运动一样,穆斯运用理
性来表示“动态的”个人最优性一一将个人设定在他或她的跨期无差异曲线上。只要经济体
中的私人参与者,最为理想地适应了可利用的信息(不存在更好的理由来假定相反),他们将
总是在最佳的调整路径上。

尽管穆斯在1961年就创作了他的论文,但是,在罗伯特·卢卡斯(Robert Lucas,1937--)将理
性预期假设用于宏观经济学,并且将它与宏观经济学的微观基础研究相结合之前,理性预期
假设并没有在经济学中发挥重要作用。理性预期假设直击微观基础经济学家与凯恩斯主义者
之间的折衷,因为它认为,在发展进程中人们并不使其预期适应均衡。他们能发现基本的经
济模型并立刻加以调整,这样做,对他们是有益的。假定人们具有理性预期,那么,长期中发
生的任何事情短期中也将会发生。因为在微观基础者--凯恩斯主义者折衷中,货币政策与财政
政策的有效性取决于错误的预期,所以,理性预期假设是破坏性的。根据新的观点,如果凯恩
斯的政策在长期中是无效的,那么,在短期中它也是无效的。

20所纪70年代中期,理性预期在宏观经济学中流行开来,关于政策的无效和凯恩斯类型的货
币政策与财政政策的不易操作,也存在一些重要的讨论。理性预期这一尚在发展中的研究,
很快就以新兴古典经济学(new classical economics)而知名,原因是其政策结论与较早的
古典观点相似。到了20世纪70年代后期,对很多人来说,宏观经济学的未来似乎在于新兴古
典思想,凯恩斯经济学已经死亡。

新兴古典学派对宏观经济学的持久影响之一,是它们对宏观经济建模理论的贡献。正如将在
第16章中论述的那样,在诸如简·丁伯根(Jan Tinbergen,1903--1994)和劳伦斯·克莱
因(Lawrence Klein,1920--)一类经济学家的研究中,凯恩斯主义者将宏观经济模型发展到一
个相当复杂的程度。20世纪60年代和70年代,很多这些经济计量模型并不是对经济体未来走
势的很好预报器,很多经济学家开始不再信任它们。罗伯特·卢卡斯是新兴古典学派的领袖,
在一场辩论中,他详细说明了为什么这些模型不是很好预报器的一个理由,他的说明以经济计
量模型的卢卡斯批评(Lucas critique)变得为人所知。他认为,个人的行为取决于所预期的
政策;因此,随着一项政策变得过时,模型的结构将发生改变。但是,如果模型的基本结构
改变了,适当的政策也将改变,模型就不再适合了。因此,运用经济计量模型预测未来政策的
效果是不适宜的。

多数人的反应是改变他们对模型的看法:模型是为特定政策问题提供见解的实际工具;存在
很多种能被加以利用的不同模型,只要它们看上去适用;没有必要使所有的模型都具有广泛
的一致性。因此,现代教科书是作为一种分析工具,而不是作为从严格的微观基础中导出的
某种东西来呈现IS--LM模型的。这种模型化方法极大地不同于新古典方法,后者在原则上将
所有的模型都看做是从微观经济学的核心假设发展而来的。

\subsection{新凯恩斯经济学与协调失灵}

一些现代经济学家从模拟、复杂性以及代理人模型中,继续为宏观经济学寻求基础,在这些
模型中,制度特征被植人代理人内部,然后通过模拟,人们就能发现什么样的策略幸存了下
来。这一研究引发了一个新的群体,被称作新凯恩斯主义者,他们认为能够为凯恩斯经济学
开发出一种新的基础。他们推论,微观经济学的宏观基础与宏观经济学的微观基础存在同样
多的需要。这些现代经济学家非常乐于接受新兴古典学派对新凯恩斯主义模型的批评,但是他
们认为,凯恩斯经济学与理性预期之间并不存在内在的矛盾性。这使他们确信对新兴古典学
派进行适当的回应,并不会得出宏观经济学更加制度化的现实微观基础。他们认为,理解凯
恩斯宏观经济学的关键,是认识到微观经济学宏观基础的必要性。人们不能脱离宏观经济学
的背景去分析典型代理人的选择,他的那些决策是在这一背景中做出的。总量生产函数并不
能从厂商生产函数中得出,产量会由于多种原因而发生实质性的变化,所有这些都涉及协调
失灵。他们主张,个人决策视其他人预期的决策而定,并且,经济体很有可能陷人预期难
题中。

因此,由理性的个人所组成的社会发现自身处于一种预期难题中,所有的个人都在进行理性
的决策,但是,个人理性决策的最终结果是社会的无理性。根据新凯恩斯主义者的观点,理性
预期假设引导新兴古典学派得出如下结论,即货币与财政政策是无效的,除非政策能够与在
全体都合意的产量水平上所有的市场都出清这一假设相结合。但是他们指出,这是一个特别
的假设,而不是从分析中符合逻辑地得出来的某种东西。

例如,个人可能共同地预期需求将降低,并因这一预期全都生产得较少:供给降低是因为所
预期的需求降低了。除非存在一种预期的协调制度,使得当一个人降低他的需求预期时,存
在一些机制来抵消预期的降低对个人供给决策产生的影响,否则,供给将会太少,因为所预
期的需求太少。正是经济体将不可避免地在全体都合意的均衡水平上实现平衡的假设,而不
是理性预期的假设,才是这些新凯恩斯主义者所不能认同的。

大多数新凯恩斯主义者的研究高度抽象且理论化,都是从抽象的对策性模型开始并证明多重
均衡是可能的。这些抽象模型在极大程度上还没有渗透到介绍性的和中级的教科书中,但它
们最终应当会渗透进去。

凯恩斯经济学理论兴趣的复活,并不意味着以凯恩斯经济政策而著称的东西又重新获得了它
们以前的地位。在20世纪70年代,对于货币政策与财政政策在政治上是否是有效的工具,存
在着日益增多的关注,虽然它们在理论上是有效的。很多凯恩斯主义者认为,货币政策与财
政政策在政治上不可能被利用,政治而不是合理的经济原理决定着赤字的规模与货币供给的
增长。

凯恩斯主义者与新兴古典学派之间的争论很快就变得复杂化了。对于一门有关思想史的课程
来说,并不适合对此加以考察。需要重点指出的是,大多数现代宏观经济研究和大多数宏观
经济学的研究生培训,包括了就理解现代争论而言所必需的技术背景。

\subsection{回到增长与供给}

新兴古典经济学显著地影响了宏观经济学,但是,与凯恩斯宏观经济学相比,它并没有大量
地为其理论收集更多的证据。对于提供任何答案来说,经验数据是根本不充分的。那个阶段,
宏观经济学家们停止考虑经济周期间题,开始将宏观经济学集中于增长问题。这符合当时的
情况,因为美国经济在整个20世纪90年代得到成长,而没有经历一个经济周期。

对增长的分析是因回归到索洛增长模型而开始的,作为对哈罗德--多马模型的回应,索洛模
型形成于20世纪50年代。哈罗德--多马模型认为增长是锋利的刀刃,除非是经济体相当幸运,
否则它将很有可能陷入萧条。索洛模型通过取消不变的资本/劳动比率假设,向这一结论提出
挑战;它表明经济体将会回到一条平衡的增长路径上来。经济体是稳定的,不是不稳定的。
索洛增长模型也称作新古典增长模型,它完全集中于供给;需求在产量决定中不起任何作用。
当新兴古典学派试图解释为什么国家之间的增长率不同时,他们发现索洛模型符合他们的偏
好,并因此进一步对模型加以发展。

宏观经济学转向强调增长,改变了宏观经济学的性质。增长模型是基于供给的模型,其中没
有需求的作用。因此,增长模型变得比较突出后,凯恩斯模型就相形见纳了。随着这些模型
一步步地首先进入中级教科书,接着又进入介绍性的教科书中,宏观经济学与凯恩斯经济学
之间的联系就淡化了,货币数量论与增长理论因此成为现代宏观经济学的关注点。古典增长
理论因新的内生增长理论而得到补充。在内生增长理论中,技术变革不再被认为是发生在经
济模型之外的某种东西;它内生于模型,它是研究与开发投资的自然结果。内生增长理论允
许收益递增战胜边际收益递减,其结果就可能是持续的增长,并且不会发生导致静止状态的
运动。因此,它使主流宏观经济学重新加入乐观派而不是悲观派组织。

对增长的关注置换了凯恩斯宏观经济学的大部分内容。凯恩斯类型的模型仍然被使用着,但
是,乘数不再被予以强调,并且,关于需求政策的任何论述也不再被重视。货币政策被用来
阻止通货膨胀,财政政策不切实际,实际政策所关注的仅与供给方面的激励有关。

但是,关于索洛增长模型也出现了问题它并不是完美地符合经验事件。两方面的修改有助于
解决这个问题:调整模型,用关注技术的新的增长理论来取代它,或者回归斯密。

\subsection{现代宏观经济学展望}

为了理解这些发展怎样符合我们关于一种新的经济学得到发展的断言,你首先得理解,凯恩
斯宏观经济学从未真正符合新古典经济学。它是被允许存在的外来的某种东西,原因是与标
准的古典模型相比,它似乎更好地满足了政策需要并解释了经济事件。

新的古典革命由于这种矛盾而向凯恩斯宏观经济学提出挑战,并试图将宏观经济学引入微观
经济学组织。在某种意义上,它是新古典思想最后的欢呼,它成功地动了宏观经济学的理论
基础,但未能使宏观经济学重新加入到新古典组织中。它只是分裂了宏观经济学,允许多种
相互矛盾的模型得以发展,并被用在它们适合的特定用途上。在这一新的现实中,几平不存
在凯恩斯经济学与古典经济学的分离;两者只不过是现代经济学——试图运用模型来了解现
实——的不同方面而已。

\section{总结}

不断变化的对于增长、经济周期以及通货膨胀与价格水平的关注,标志着宏观经济学的发展
历史。尽管亚当·斯密起初对经济增长问题感兴趣,后来的经济学家则将他们的分析集中在收
入分配上,认为经济体将由于边际收益递减规律而最终陷入一种静止状态。他们将价格视为
主要由货币数景论决定的,并认为这种决定必须与对实际经济体的分析分开来。他们将经济
体看做是本质上自我纠正的,基本不需要政府干预。

凯恩斯的《通论》标志着经济学关注点从资源配置的微观经济问题向经济波动的宏观经济问
题的重大变革。它强调短期胜过强调长期。凯恩斯提供了一种新的分析框架来解释经济活动
水平的决定力量。他不但发现了资本主义的内在不稳定,而且得出结论认为,市场自动运行
的通常结果是在低于充分就业的水平上形成均衡。效仿马克思、杜冈--巴拉诺夫斯基、威克
塞尔还有其他一些人,凯恩斯集中研究投资支出在决定经济活动水平中的作用。

随之而来的大量文献不仅扩展并改进了最初的凯恩斯理论,而且使凯恩斯模型与前凯恩斯模
型之间的差异与相似成为比较尖锐的话题。凯恩斯的观点所采取的形式,导致了数学模型构
建与经验检验。理论上的革命很快紧跟着政策上的革命,因为主要的工业化国家开始制定规
划,并且组建用以扶持充分就业的机构。

宏观经济学的凯恩斯化以一种相当奇怪的方式发展着:它采取由最主要的凯恩斯主义者,例
如阿尔文·汉森与保罗·萨缪尔森,提出的乘数模型的形式发展。凯恩斯宏观经济理论的发展,
加上财政政策作为政府推动充分就业的一种补偿行为被加以运用,这些或许解释了对乘数模
型的这种关注。为了回应纯凯恩斯理论的内在矛盾,以及货币主义者提出的有关货币作用的
问题,到1960年,IS--LM模型成了支配性的宏观经济模型。

然而,随着发生在1975年左右的争论日益正式化,人们发现这一模型对经济研究来说并不令
人满意。通货膨胀和失业一样,似乎都成为当时重要的经济话题。认同这一点的一类新文献,
试图揭示宏观经济学的微观基础,并因此模糊了凯恩斯主义的一个方面——把经济学划分为微
观经济与宏观经济领域。随着微观基础文献的增多,争论与理论发展都回到与20世纪30年代
早期的框架相接近的某种东西上。唯一的例外是,一般均衡分析正在日益取代局部均衡分析。
最初,宏观经济学密切地与经济计量学以及经济体的大规模模型的发展相连。尽管存在大量
这样的模型,然而它们早先的承诺还没有兑现。因此,20世纪80年代,出现了远离这种模型
和纯理论问题的运动。现代宏观经济学具有高度的折衷性,没有哪一种方法能为所有的经济
学家所认同。

此外,当前仍处于一个过渡时期,学者们肩负着广泛的研究项目,致力于解决很多不同的问
题。宏观经济学今天的主要关注点在新增长理论上,这一理论严重地背离了较早的古典增长
理论,尤其是在强调内生的技术以及认为静止状态可以避免方面。


%%% Local Variables:
%%% mode: latex
%%% TeX-master: "../../main"
%%% End:
