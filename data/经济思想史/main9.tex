\chapter{新古典经济学的制度性与历史性批判}

新古典经济学并不是毫无争议地就产生了。随着新古典经济学的出现,德国历史学派向其方
法论基础发出挑战,整个19世纪80年代后期,在奥地利学者(特别是门格尔)与德国历史学派
的一些成员之间,就经济学的适当方法问题存在着激烈的争论。新古典经济学横扫了英国与
法国,却不包括德国,在美国也遇到了抵制。因此,在19世纪末20世纪初前后,美国经济学
专业的研究生为获得博士学位而在德国学习,这种情形仍然是一件平常的事。很多这样的学
者,满怀渊博的学识和对人德国历史学派观点的赞成态度返回美国。在美国,除了对古典理
论这种“输入性的”批评外,还有一些明显的美国因素,它们根植于美国中西部的平民主义
渐进运动。

本章首先总结主要发生在讲德语的经济学家之间有关方法的争论,然后考察20世纪一-些美国
非正统经济学家的贡献,聚焦于经常作为制度主义者被提及的一组美国经济学家。

即使是有限的聚焦,也不容易决定将哪些经济学家包括进来。我们在历史学派中强调古斯塔
夫·冯·施称勒(Hustav von Schmoller,1838--1917),是由于他在争论中的重要性。我们之
所以从20世纪早期进行创作的美国学者中挑选凡勃仑,是因为他对后来非正统思想产生的公
认影响;挑选韦斯利·克菜尔·米切尔(WesleyClairMitchell,1874--1948),是因为他在收集
并分析与经济波动相关的数据方面的开拓性工作;挑选约翰·R·康芒
斯(John R.Commons,1862一1945),是因为他对现在的社会理论与立法所产生的影响。最后,
我们挑选了英国人霍布斯,作为非美国的非正统经济学家的代表,因为他对当代英国社会政
策的态度具有深远影响。

非正统理论与正统理论早期的意见不同主要体现在两个方面:第一,与正统理论的范围和方
法意见不同,以及与其理论核心中的其他因素意见不同;第二,与正统理论最重要的下列观
点意见不同,即市场系统普遍导致经济力量的和谐结果,因此自由放任是政府应遵循的最住
政策。

\section{方法上的争论}

即使在门格尔、杰文斯、瓦尔拉斯以及马歇尔开始将边际分析应用于价值与分配理论之前,
正统古典理论就受到了某些非社会主义德国经济学家的批评。虽然这些经济学家的观点之间
有一些显著的差别,但是,它们有充分的共同点来集体被称作德国历史学派。这一学派的影
响在德国开始于19世纪40年代期间,并延伸到20世纪。很多历史学家注意到了早期的经济学
家与稍后的经济学家在看法上的差异——很大程度上来自于德国的不断变化,以及对正
统理论的回应——而将其划分为旧历史学派与新历史学派。

19世纪70年代,英国也出现了独立于德国历史学派的对正统经典理论的批评,以及对所谓
的历史方法的提倡。然而,历史方法的这些英国提倡者,没有形成聚合的群体,因此,英国
历史学派这样的说法是不适宜的。因为德国与英国的这些强调历史的经济学家对某些新古典
经济学家,特别是阿尔弗雷德·马歇尔产生了巨大影响,所以值得我们去关注。鉴于在德国接
受研究生教育的美国经济学家为数不少,因此德国学者也影响了美国的经济理论与政策。

\subsection{旧历史学派}

旧历史学派中的重要经济学家有弗里德里希·李斯特(FriedrichList,1789--1846)、威廉.罗
雪尔(William Roscher,1817--1894)、布鲁诺·希尔德布兰德(Bruno
Hildebrand,1812--1878)以及卡尔·克尼斯(Kanl Knies,1821--1898)。他们主张,不能将
古典经济理论应用于所有的时期与文化中,尽管斯密、李嘉图以及约翰.斯图亚特·穆勒的结
论对于像英国这样工业化中的经济体来说是正确的,然而,并不能应用于农业化的德国。这
些经济学家的经济分析中包含大量的民族主义情感。此外他们断言,经济学与社会科学必须
使用一种以历史为依据的方法,在李嘉图及其追随者的控制下,古典理论在试图模仿自然科
学方法上是错误的。学派中一些比较中立的成员承认,理论演绎法与历史演绎法是一致的;
但是,一些人特别是克尼斯反对抽象理论的任何一种应用。

李斯特尤其表达了强烈的民族主义观点,他拒绝接受如下看法,即证典理论关于自由放任的
结论适用于那些不如英国发达的国家。古典理论主张,国家的福利来自于个人在自由放任的
环境中对私利的追求,而李斯特则认为,国家的指导是必要的,尤其是对于德国和美国来说。
他主张,考虑到其工业的先进状况,自由贸易有益于英国,而对德国与美国而言,关税与保
护则是必要的。从1825年至1830年李斯特在美国花了五年时间,以及后来在德国花了大约十
年时间出版了《政治经济学的国民体系》一书,该书汲取了他在美国的经验。其保护主义
观点在美国得到热烈的认同,以至于他通常被称作美国保护主义之父。

这些经济学家所提倡的历史方法是什么呢?他们的工作反映了如下信仰,即经济学的首要
任务是发现支配经济增长与发展阶段的规律。例如,李斯特声称,处于温带的经济体将经历
五个阶段:游牧生活;畜牧生活;农业;农业与工业;农工商业。希尔德布兰德断言,理解
经济增长阶段的要记是在交换条件中找到的;因此,他设置了基于物物交换、货币以及信用
的三个经济阶段。对增长阶段的这些描述,显然包含了一定量的理论并且高度抽象。然而,
这些经济学家的确收集了大量历史与统计资料来广持他们对于经济发展的分析。在更近的时
期,沃尔特,惠特曼,.罗斯托(Walt WhitmanRostow,1916一)提出了一种遵循旧历史学派传
统的经济发展阶段理论6正如可以预料到的那样,与经济学家自身的认同相比,他的著作获得
了其他社会科学中学者们更好的认同。新历史学派第二代德国历史学派有一位杰出的领导痢
记斯培夫:冯,施稳币。像旧历史学派的成员一样,新历史学派的经济学家们择击古典经济
理论,万其是古典理论适用于所有时间与地点的观点。在历史方法的应用上,他们一般不如
旧学派那样雄心过过,他们愿意创作关于经济与社会不同方面的专论,而不是阐述重大的经
济发展阶段理论。在努力创作的过程中,他们更育欢运用归纳方法,似乎认为收集到足够的
经验证据之后,理论就可能出现。他们也对借助国家行为的社会改革非常感兴趣。因为这一
点,他们被称为“讲坛社会主义者”",这是他们乐于接受的一个称号,他们认为,不接受诸
如所得税一类建议的批评家是反对进步的人。

门格尔、杰文斯以及瓦尔拉斯在19世纪70年代早期应用边际分析,并
构建抽象演绎模型,这在德国只有很少的影响或者说没有影响。尽管门格
尔这位奥地利人用德语创作了他的《经济学原理》,但是,在德国的大学中
并没有被加以研究,因为这些学校排外性地只赞同历史方法。在其早期著
作中,施穆勒乐于承认两种方法在经济研究中都占有一席之地,虽然他并
不推荐构建抽象的理论模型、1883年,门格尔出版了一本关于方法论的著
ODWW.WW.罗斯醋.经济增长的阶段.喘国:全术大学出版社,1960
341
0
2
24
作《社会科学尤其是政治经济学方法的研究》(InquiriesintotheMethodofthe
SocialScienceandParticularlyPoliticalEconomy),它开启了一场一直持续到
20世纪的长久的、沉闷的、无果而终的争论。关于方法的这场争论,是经
济理论发展中曾经发生的最为激烈的方法上的争论之一;只有美国制度主
义者与正统理论家之间最近的争论能与之相比。门格尔的著作中包含了对
经济学与社会科学中方法问题的一般性概述,然而,他也皇击了历史方法
的错误。施穆勒回应了这一择击,战争于是开始了。门格尔发表了对施称
勒回应的驶斥,其他人也参与到争论中来。双方都摆出攻击的架势,都认
为自己的方法几乎是唯一可以使用的。正像施穆勒指出的那样,双方都用
表示敬意的术语将自己的方法表述成经验的、现实的、现代的、精确的方
法,同时将对方的方法称作是投机的、无用的、次要的。

从某种观点看,这场争论可以看做是经济文献的纯粹死胡同,是经济
学作为一门学科加以发展的有害物,因为有才能和心智的人把他们的时间
都耗费在没有意义的争论上了。另一方面也有可能是,这场争论帮助经济
学家认识到,在他们的学科中理论与历史、演绎与归纳、抽象模型构建与
统计数据收集并不是相互排斥的。

尽管个别经济学家可能倾向于将其主要精力专门投入到这些方法中的
一种上,然而,一个健康发展的学科要求方法的多样性。因为没有哪一种
方法能够完全排斥另一种方法而被加以认同,所以,实际问题在于给予每
种方法应有的重视。我们的观点是,学科的内在发展将决定这个问题,所
以,为此进行争论是没有意义的。

从这场争论中能够得到另外一个教训。如果某种特定方法的创立者,
对方法的正确性变得如此确信,以至于不允许其他观点在从事研究和研究
生培养的大学中表现出来那么,经济学的发展将会受到损害。这一点发
生在德国,自以为是且刚惰自用的智力领袖施穆勒,在德国极具影响力,
以至于遵循门格尔、杰文斯、瓦尔拉斯以及马和软尔所创立路线的抽象理论
家们,无法在德国找到大学职位。结果,经济思想的主流忽视了德国经济
堂家,经济学作为一门知识学科在德国遭受了几十年的损害。
Ma
i
i
i
a
第12瘟新古典经济学的制度性与所出性批判
美国的历史方法
19志纪最后二十五年期间,许多天国经济学穴批评正统户典理论,并
提倡运用历史方法进行经济学研究。这些经济学家与德国经济学家不同,
他们未能形成一个聚合的群体,也没有受到德国经济学家的直接影响。在
经济思想中,对于英国的传统来说历史归纳方法并不陌生。亚当,斯密的
《国富论》是历史材料与描述性材料的混合,再配合一个松散的理论结构。
李嘉图代表了经济学方法向抽象演绎模型构建的一次主要变动,模型几乎
完全没有历史的或制度的内容。西尼尔拥护并扩展了李嘉图对演绎推理的
运用。然而,约翰"斯图亚特.穆勒和阿尔弗雷德*马歇尔回归斯密方法,
利用他们对历史材料和制度材料的渊博学问与知识,赋予其理论结构以实
质内容。

历史方法在英国的首要倡导者是T,E.克利夫-莱斯利,他对古典经济
学的方法(主要是李嘉图及其追随者的方法)进行了批评。莱斯利主张,
斯密的经济理论不适用于现代英国的情况,但是总的来说,斯密的方法还
是相当合理的,因为斯密广泛运用历史材料来得出结论。尽管阿诺德汤
因比(AmoldToynbee,1852一1883)英年早逝,使其成为一名经济史学家
的伟大心愿未能得到完全实现,然而,他的《18世纪英国工业革命讲稿》
(LecturesontheIndustrialRevolutionoftheEighteenCenturyinEngland,1884)
则是运用历史方法来了解发生在英国的根本变革以及因此出现的工业化经
济体问题的一个壮观例子。正是阿诺德.汤因比创造了工业革命(Industrial
Revolution)这一术语。威廉.阿什雷(WilliamAshley,1860一1927)与威
廉:坎宁安(WilliamGunningham,1849一1919)关于英国经济史的著作,
仍旧受到高度的尊重。其他经济学家运用历史方法来分析具体的主题,沃
尔特,贝格豪特创作了《朗伯德街》(LombardStreet,1837),它是一部研
究英国银行业的经典;约翰.K:英格拉姆(JohnK.Ingram,1823一1907)
出版了《政治经济学史》(HistoryofPoliticalEconomy,1888),它是用英语
创作的关于经济理论史的第一部系统性著作。

尽管历史学派没有对理论的新发展产生重要影响,然而,它的经验一
直都是有用的,并且影响到了很多经济理论批评家,我们将在本章的以下
343
0
3
MSEE
部分以及第17章中对他们进行夯凤。
一、托尔斯坦,凡描仑
托尔斯坦:邦德*几惑仑是通常饭称作人制度主义(institutionalism)的
美国非正统分支的学术创始人。他与正统理论在科学上和道德上的不同意
见;极大地影响了美国非正统思想的发展。凡皂仑的观点,部分地可以通
过其背景得到解释。他是挪威移民的儿子,在美国威斯康星州和明尼苏达
州的农村长大。当他进入卡尔顿学院时,他对英语的掌握就像他对美国社
会的了解一样不充分,他从来没有完全融入到美国主流社会。他就像一个
从火星上来的人一样,以其讽刺智慧评述着经济与社会秩序的荡廖。在卡
尔顿,他的才华得到了约翰.贝艾.克拉克的认同,后者当时对边际分析
做出了开创性的贡献。在克拉克的鼓励下,凡勃仑去了东部的研究生院。
他在耶鲁获得了哲学博士学位,但是未能得到一份从教的工作,这显然是
由于他的无神论观点。所以,凡蔓仑回到农场,并与他的大学情人结婚,
花了七年的时间继续读书思考。

三十五岁时,他获得了康奈尔的博士后奖学金。在仍然末能找到一份
学术工作后,他又接受了芝加哥大学的奖学金,在那里他终于得到了一份
经济学讲师的工作,并被给予《政治经济学》杂志编辑的职位。他从未受
到大学管理者的欢迎,从来没有获得正教授的级别,他将其生命的剩余时
间,花在从一所大学迁往另一所大学上。无法确定的是,他未能获得其学
识足以担保的专业认可,是因为他对美国资本主义进行敏锐批评,并且除
了最好的学生之外几乎完全漠视其他所有人的结果,还是由于他的个人生
活造成的,他的个人生活因风流更事和婚姻上的难题而复杂难解。然而,
在20世纪20年代中期,经过几年政治上的上暗斗之后,美国经济学会向几孝
仑提供协会的任期条件是他要加入协会并发表一篇演说。凡勃仑拒绝了
这一提议,声称当他需要的时候这一荣誉并没有出现。

凡勒仑在偏僻土地上接受的教养、他的哲学训练、他在社会科学方面
的广泛阅读,以及他对达尔文进化论重要性的深入评价,都反映在他对美
国资本主义的分析中。他的文体和对交字的选择,赋予其作品一种品质,
;344
np-
i
第12章”新古典经济学的制度性与历史性执汶
一些经济学家发现这种品质非常令人愉快,另一些经济学家则对这种策庄
予以让责。他是一个擅长创造警句的人,喜欢通过使用像炫次性消费(con-
spicuousconsumption)一类的术语来描述新兴富足社会的购买方式,从而使
其读者不舒服。他认为,我们要么是受控制的阶级成员,要么是基本人口
成员;大学校长是学识首领,生意人的主要工作是实行破坏;工业无度地
多产,要获利就要切实地消除效率。凡勃仑将教会描述为“一种公认的发
泄,以使蝴落的事物从文化机体中流出”。米切尔提出,人们需要有一种项
睦感来欣赏凡勃仑,也许这就是他为什么很少得到经济学家赏识的原因。
凡支全对新古典理论的批判
凡皂仑创造了新古典这一术语,用以强调这种类型经济理论的主典世
系。他认为,上古典与新古典方法两者都是不科学的。他对新古典理论的批
评包含在他的全部著作中,虽然他的一部论文集《科学在现代文明中的地
位》(ThePlaceofScienceinMordernCivilization)包括了其大部分明确的方法
论著作。他受到的哲学训练,部分地解释了他对其所处时代公认的经济学
所进行的皇击的性质。凡过仑对理论结构的微小变动——例如,纠正体系
中次要的逻辑缺陷不感兴趣。他攻击新古典理论的核心,声称其学说的基
本假设是不科学的。对一种理论结构基本原理的这种攻击,使得接受这一
结构培养的人面临两种选择:他们也许接受批评,并依据改变了的假设构
建一种新理论,或者了台回批评。对一种理论结构的批评,如果接受了该理
论结构的基本假设,但提供了新的更加符合逻辑的或者更加符合经验的正
确结论,那么,这种批评可能被接受或者被驳回。但是,在这种情形下,
对于那些已经进行了学科训练的人来说,接受这种批评痛苦就会稍小些,
因为这不要求对其训练和定位作深度的重新调整。凡勃仑明白,李嘉图、
马歇尔还有凯恩斯,虽然接受由斯密构建的理论结构假设,却没有试图改
善古典经济学的理论结构;他希望拆印整个结构,重建一个由经济学、人
类学、社会学、心理学、历史学组成的统一的社会科学。值得注意的是,
凡勃仑依据他批评正统理论所依据的相同的理由,批评了以前的非正统思
想,认为历史学派是不完善的,因为它的基本假设与预见是不科学的。

丹动仓的观点是,尽管正统经济理论的术语从斯密时代起一直发生着
345.
'
~
el
Co
i
t
变化,,但是,其基本假设与预见保持不变。在斯密之前,对经访社会的
大多数分析基于下列预见,即社会受到超自然力量的支配以获得合意的结
果。后来,对超自然力量或者上帝的求助,被下列观点所取代,即自然法
则存在于经济和社会中,就像存在于自然科学中一样,适当的调查与研究
将揭示这些自然法则的运转。

凡勃仑说,从斯密到马吹尔,正统经济理论的全部都基于相同的假设:
系统中存在和谐,或者说凡过仑所谓的“改良性趋势"。这一点出现在斯密
自然价格的概念中,以及将私人恶行转变为公共利益的“看不见的手”的
运转中。马歇尔理论中正常价格与均衡的概念,以及长期均衡下的完全竞
争市场产生有益结果的预期,都反映了这一信仰。约坦,贝获.克拉克的
有关长期竞争性均衡产生一种公平的收入分配的结论,是关于经济体和谐
假定一个特别显著的例子。对凡勃仑来说,正统理论家所运用的均衡概念
是标准化的:他们毫无证据地暗示均衡是优秀的,均衡状态下的市场所产
生的结果在全社会中是有益的。

我们从不同的视角来考察这一点,并使之与几孝仑对正统理论的男一
个批评相结合。凡过仑借鉴了哲学与生物学中的概念,他断定正统理论是
目的论的《teleological),因而是前达尔文的。正统理论之所以是目的论
的,是因为它将经济体描述为朝向一个目标——长期均衡——运动,而这
一目标在经验上是达不到的,却在分析开始之前就设定了。正统理论之所
以是前达尔文的,是因为正如几勃仑对达尔文的解释一样,进化是一个纯
机械过程,借助该过程生物体适应环境情况,随着时间的变化而得到发展。
进化中没有目的或者设计。

古典思想拒绝承认经济体是不断变革与演进的,而是集中研究经济体
的静态方面,因此,古典思想也是前达尔文的。几厚仑主张,这种静态的
前达尔文的经济理论,应当被一种动态的关于经济与社会进化的达尔文分
析所取代。凡撮仑用生物学的术语指出了同一点,并谴责古典理论是分类
学的(taxonomic),因此也是不科学的前达尔文的。之所以说它是分类学
的,是因为它对经济体及其组成部分进行了分类,但是,没有就它们作为
一套演进的变革的制度进行解释并形成概念。正统经济学聚焦价格理论时,
假定很多东西是既定的或者不变的(例如,品位或者消费者偏好、技术、
'
4
1346

Wi
oo
ba
人
“第12章新古典经济学前制度性与历史和性拒汶
社会与经济的组织安排,等等)。几孝仑指出,经济学家不仅应当人研究价格
的形成与资源的配置,而且应当研究他们使之保持不变的那些因素。他对
马歇尔试图与静态分析决裂表达了一些溢美之词,却断言马歇尔在这方面
的努力是不成功的。

凡孝仑指出,经济学是不科学的,这一提法的一个原因是,亚当…斯
密“看不见的手”的概念从来没有被证明。因此,经济学建立在一个从未
予以考察的假设之上:赚钱就等同于生产产品。按照正统理论,后意人在
追求利润的过程中,将以最低可能的成本生产那些消费者需要的产品。葛
争性市场促使生意人的私利符合社会利益。追求他或她自身私利的每个个
别生意人,推进了社会利益。凡勃仑认为,除了经济学家之外,对所有人
来说,下列结论都是显然的,即生产产品与获利是两种不同的事情,企业
界为利润而奋斗,这对经济与社会经常会产生有害的效应,追求他或她自
身私利的每个个别生意人,将只推进他或她自身的私利。有人提出,凡撮
仑对经济与社会的这种观念,在他年轻的时候,当他离开其位于明尼苏达
州路德教会家庭的边境农场,前往卡尔顿学院时就已形成了。在卡尔顿学
院上学的,主要是来自新英格兰、具有公理会背景、很会赚钱的人家的孩
子。@19志纪最后二十五年中,大公司规模与实力的增强,以及托拉斯的形
成,也影响了凡勃仑。此外,耕种土地的平民主义者对工商业——谷物升
隆机、铁路、农用设备制造商,以及银行——的敌对,在他的家庭中也一
定很深。

凡过仑主张,亚当,斯密时代,赚钱与生产对社会有用的产品两者之
间,存在一种相当紧密的联系。但是,随着经济的发展,这一点发生了变
化。凡过仑严格区分了涉及生产产品的人——生产经理、监督者以及工
人——和涉及企业管理的人。工商业的目标是金钱收益,几厚仑直指一般
利益受到逐利侵害的例子,并为自己的做法感到高兴。他的观点是,利渔
增加是产量减少的结果,这对社会显然是有害的。几拖仑时代正在形成的
大公司的目标,不是提高效率,而是获得垄断实力并限制生产。他指向厂
GB志斯利.克莱尔.米切尔.经济理论的类型.美国:奥贞斯塔斯*M,哎六出版公司,
1967.619
347
To
Wh,
0
沁(YYY/轻济思想史
FOonyofErororeShug
|348
商的广告行为,质疑它们对整个社会的有用性。厂商之间的国际市场苋争
导致冲突,最终引发战争。大企业首脑们的金钱活动,将不可避免地导致
经济萧条与大量失业。本质上说,凡过仑否定关于完全竞争市场的正统假
设,否定生意人控制下的市场将会产生对社会合意的结果的观点。正统理
论凭异有效的资源配置与充分就业,发现了资本主义下的和谐发展;而凡
过仑凭借生意人为了获利而破坏系统,并最终导致经济萧条,发现了资本
主义下的不和谐发展。

在凡勃仑看来,正统理论的预见,反映出经济学未能与自然科学和生
物科学的发展保持并列。正统经济理论无视心理学、社会学以及人类学的
发展,基于不科学的人类本性与行为概念来构建模型,在这些方面应受到
责备。按照凡过仑的观点,正统理论基于下列假设,即人类在享乐主义心
理的基础上,受到快乐最大、痛昔最小愿望的驱使。给定这一假设,经济
学家正确地推论出了其逻辑结果。届辑是没有缺点的,但是,假设是错误
的。凡孝仑主张,正统经济学是对人的研究,却将人从分析中抽象掉了。
在他最尖锐的一些散文中,他嘲笑公认的消费者行为理论:
经济学家关于心理学和人类学的预见,几代人之前就得到了心理字和
社会科学的公认。享乐主义关于人的观点,是把人当做一个快乐与痛苦的
闪电计算器,他像一个均匀的追求幸福的小球一样,在刺激的推动下摇摆,
这种推动使他改变方向,却完整无缺。他上既无先行者也无后继者。除了冲
击力使他在一个方向或另一方向上摆动之外,他是稳定状态下孤立地确定
的人类已知数。他在自然空间中自我加强,围绕着他自己的精神数轴对称
旋转,直至力的平行四边形向他施加压力,使他按照力的合成路线摆动。
当冲击力耗完时,他停止移动,又成了像以前一样的一个孤立的欠望
小球_@
凡寺仑对正统理论的最后一项批评,与其他批评相比阐述得不是很明
Q@参见托尔斯坦,凡过仑的“为什么经济学还不是一门进化科学”一文,该文载于《科学在
现代文明中的地位》一书。该书由美国B.W.休伯斯科出版公司于1919年出版。访引文引自该书的
第73~74页。
第12和章”产古典经济学的制度性与历史和狂批淹
确,即正统理论未能使经济体的理论与经济体的实际相一臻。因此,几动
他的作品包含着对更多经验研究的含蓄要求和对归纳研究的更多强调。
岂寺从于资本主义的分析
凡孝仑强调,经济学的主划应当完全不同于辟行的经济理论的主有。
凡勃仑时代的正统理论,主要对社会如何在多种可蔡代用途之间分配其稀
缺资源感兴趣。凡过仑主张,经济学应当是对演进的制度结构的一项研究,
并将制度界定为在任何特定时期都被接受的思想习惯。在对经济学主引的
这一界定中,几勃仑试图解释塑造社会与经济的力量。一种文化的特定制
度,被正统经济理论假定为既定,凡过仑却试图加以解释。他主张,对盛
行文化的解释,要求一种演进的方法,因为任何文化只有通过其先辈才能
被理解.
文化的成长龙习惯的一种系积顺序,其方式与手段是人类本性对紧总
事件的习惯反应,这些紧急事件无节制地且累积地变化,却在累积变化中
带有几分一致的顺序,这些累积变化是这样发生的——无节制地,这是因
为每个新的运动都产生一种新的状况,引起习惯的反应方式更新的变化;
累积地,这是因为每种新的状况,都是它前面状况的一种变化,它体现的
因果关系因素已经到了为前面的状况所预料的程度;一致地,这是因为反
应连于人类本性(倾向,自然倾向,诸如此类】的根本特征而发生……人
类本'性充分保持不变,人0
为1理解工业社会的发展以及现在的作用,我们必须了解存在于人类
本性特征与文化之间复杂的相互关系系统。
不仅个人的行为受到他与群体中同伴习惯性关系的限制和引叶,而且
这些关系具有一种制度特征,随着制度安排的变化而变化。个人行为的需
要和窝望、目标与目的、方式与手段、幅度与动向,是具有非常复杂又完
DD托尔斯坦,几勒仑的“边际浆用的局限”一文载于《科学在现代文明中的地位》一书,本
引文出自该书第241~242页。
349,
'
-mmm~
i
350
VUGE
全不稳定特征的制度变量的水数,负
当个体出现在文化中时,他们发现自身依照已经确立的行为模式来行
动,这种行为模式是个体与文化之间过去相互作用的产物,并且具有制度
的特征与力量。凡皂仓将这些相对不变的人类行为的根本特征称作本能
(instinctis)。他深受心理学当代发展的影响,即强调本能在指导人类行为中
的作用。凡孝仑认为,塑造人类经济活动最重要的本能有亲体本能、技艺、
闲散的好奇心以及获得。亲体本能最初就是对家庭、部族、阶层、国家以
及人类的关注。技艺的本能使我们希望生产高质量的产品,为技艺而自罕
并赞美它,并关心工作中的效率和经济性。闲散的好奇心引导我们提问,
并寻求对周围世界的解释,它在解释科学知识的发展中是一项重要因素。
获得本能与亲体本能相对,因为它使个体关心他或她自身的福利其于关心
其他人的福利。
王分法
人类本能的驱使产生了某些紧张状态。亲体、技艺、闲散好奇心的本
能,将导致人类以极大的效率生产高质量的、有益于同类人的产品。然而,
由于获得本能是利己主义的,所以它将导致有益于个体的行为,尽管它
可能对社会上的其他人产生有害的结果。凡撮仑说,对经济体的分析,揭
示了这种根本的紧张与对抗,它在人类本性中是基本的。每种文化都能通
过观察人类行为的两个方面而得到分析:一个是推动经济生活过程的方面,
另一个是抑制社会生产能力充分发展,对人类福利具有负面效应的方面。

凡擂仑将主要从亲体、技艺、闲散好奇心本能中产生的活动称为生产
性的(industrial)(或技术性的technological)职业。它们包括事务性的、
因果性的关系。他在推测的历史中进行研究——尽管他激烈地批评了正统
理论的这一行为——并解释说在瑰远的过去,人类通过使用符叶祈求超自
然力量发生作用,通过绕着茎秆跳舞,诉诸超自然的力量种植谷物,并试
图以此来解释未知的东西。凡过仑将这种非制度的、非技术的、近代科学
D同上,第242~243幢。
WSPemarene

we"oe

Wp
oe,
第12章新古典经济学的制度性与历史性批羯
之前的接近未知并寻求解释或效果的方式,称为礼仪性行为《ceremonialbe-
havior)。礼仪性行为是静态的,并与过去相共合。它用图腾与禁忌、用对
权威与情感的诉求来表明自身,对人类福利来说,它产生了不受欢迎的结
果。然而,生产性或技术性职业是动态的,并且,我们越是运用科学的、
事务性的观点接近问题的解决,我们的工具、技术以及解决问题的能力就
越强。技术并不倒退,但是,礼仪性行为则根植于过去。

凡亏仑对他所处时代文化与经济体的分析,建立在这种二分法的基础
上。他的几乎全部论文和书籍,都在反复地阐明这一主题。他认为,这一
框架及其运用不包括规范性的判断,而是构成了对发展以及文化与社会结
构的事务性实证分析。在他的散文“生产性职业和金钱性职业”(Industrial
andPecuniaryEmployments),以及可能是他的一部最好的经济分析著作《企
业论》(TheTheoryofBusinessEnterprise,1904)中,能够最为清楚地看到二
分法的纯经济运用。现代文明中的礼仪性行为,在凡厚仑所谓的金钱性
(pecuniary)或生意性(business)职业中最能体现出来。在工业经济出现
之前的手工业时期,工匠拥有自己的工具和材料,用他自己的劳动,生产
能够表达其技艺本能和亲体本能的商品。从这些活动中获得的收入,是对
所付出努力的合理度量。随着经济体的发展,很多东西发生了变化。工人
不再拥有生产工具或者材料,企业所有者现在对赚钱比对生产产品更感兴
趣;获得本能比技艺本能和亲体本能更重要。放款发展起来了,缺席所有
权(absenteeownership)变得更平常了,个人现在拥有“依时效而取得的权
利,以免费得到某种东西"。大企业的首领出现了,跟着是一段激烈竞争的
时期。大企业的首领很快意识到竞争是不合意的,所以,借助投资银行家
的手段,形成了控股公司、托拉斯以及连锁董事会,也形成了既定利益的
全世界大工会和缺席的所有者。对于工人和工程师以及大企业的首领和缺
席所有者来说,所有这些发展都导致了不同的思想习惯。基于生产性职业
的日常基础——产品生产,工人和工程师被包括进来。这促使他们在因果
方面进行思考,表达出他们的技艺本能和亲体本能。但是,大企业的首领
和缺席所有者只关注利润,凡按仑的观点是,获利与生产产品经常发生
冲突。

推动凡勃仑对他所处时代工业社会进行分析的主要力量是,他认为正
3251|
WD和
obi
a
NG
erotGoorioriacorigth
统理论在下列主张上是错误的,即生意人引导下的经济体将促进社会利僵。
他无情地指向工商业导致的“道境"。拥有芍断实力的厂商,为了获得更大
的利润而实行“经过考虑的赋闲”。产量的这种减少,提高了利润,导致了
一种“无效率的股本”。“生产为了生意上的缘故而继续着,而不是相
反。”2大量活动被误导,生产对人类没有用的产品,并加以营销和广告。
生意人并不是社会的恩人,而是社会的破坏者。
有站阶级
礼仪性-生产性二分法也适用于几玛仑所训的有困阶级(leisure
class)。1899年,凡孝仑出版了被证明是其最被广泛阅读的书籍《有闲阶
级论》(TheTheoryoftheLeisureClass);这是在他所处时代很多知识分子特
别喜欢的一本书。他在书中运用其基本的二分法来讨论炫次性消费、炫浪
性闲暇、炫次性浪费、金钱竞赛以及表达金钱文化的着装。凡勃仑推论,
在不发达的文化中,一个人或一个部族的捕食能力受到高度的尊重,具有
这种能力的人,被给予受人竟敬的地位。在现代工业经济体中,这些捕食
能力通过为社会少数成员带来高收入就业表现出来。然而,如果高收入不
能得到认可,就将毫无价值,所以,我们的文化提供了很多允许它们得以
展示的机会。因为攀比是一种有力的动机,所以,这些财富展示活动会很
快遍及社会。

对我们所购商品的炫炉性消费,是展示我们捕食能力的一种最有效手
段。我们的汽车、住房,尤其是服装,清楚地表明了我们在捕食次序中的
地位。如果家庭中的男性过分忙于从事其捕食活动,那么,他的妻子就被
指望承担展示家庭财富的重任。她通过穿着和展示其他商品,以及小心避
免任何类别的工作来完成这一重任——所麻用的仆人数量是经济能力的一
项可靠指标。此外,因为有闲阶级是高收入阶级,所以,所从事的工作应
当在严格的金钱性职业中;缺席所有权受欢迎但是,如果必须做一些实
际工作,那么,高级管理、人金融以及银行都是可以接受的礼仪性行为。法
GD托尔斯坦,凡过仓.企业论.美国,查尔斯,斯元莱布语之子出版公司,1904.26
1
‘352
be
i
律是一项很好的职业,因为“律师的工作充满了有关捕食诡计的细节。”
凡擂仑说,闲暇活动也反映了在文化中获得受人尊敬的地位这一愿望。高
等教育使一个人不适合从事正当工作,却具有重要价值。有闲阶级也培养
对体育活动的极大兴趣,并以它们促进了身体健康和男子汉气概为由,使
之合理化。凡过仑评论说:“有种说法,即足球和体育的关系与斗牛和农业
的关系极为相同,并非不恰当。”®®

凡抛仑认为,与技术性职业相关联的个人,例如,发明家与工程师,
都是胆大而足智多谋的,美国的生意人展示了一种寂静主义精神——“折
囊、谨慎、共谋以及诈骗”".®但是,生意人以非劳动所得收获了技术社会
的利益。他曾提到:“有一种家常的但被充分认同的美国俗话——“寂静的
猪吃猪食。”@凡过仑的观点是,学术与科学训练使一个人不适合工商业,
工商业的经历与研究学问是不相容的。

从董事会变动到学术行政,几孝仑将大学校长称作是“学识首领”。他
说,尽管通常他们都是以前的学者,但是,他们被卷人社会的金钱价值中,
误导了大学的努力;像工商业的首领一样,他们在手段与目的之间变得困
惑了。大学之间相互竞争,资源的浪费比得上工商业竞争造成的浪费;相
对于教育项目和政策,校长与董事会对建筑物、场所、房地产更感兴趣;
资源浪费在既对大学没有价值,也对社会毫无用处的体育运动、法律与商
学院、典礼以及盛会上。凡勃仑没有宽恕“教授会",那些教授认为“他们
的薪水不具有工资的性质”"”,他们没有集体谈判的权利,他们立志成为“乡
强"。为了控制全体教员,校长任命院长和其他具有“现成的信仰多样性,
并坚定忠诚于其生计”的人。凡皂仑所推荐的使大学回到研究学问上来的
主要行动计划是取消校长与董事会。很难断定凡勃仑对于这一讽刺性的建
议有多勾认真,但是,至少他认识到这是很不可能发生的。
托尔斯坦、几勒仑.有闲阶级论.美国:肾殉顿*米弗林出版公司,1973:156
同上,第173~174页。

参见托尔斯坦*凡皂仑的《美国的高级学术研究》第70页。

同上,第71页。

同上上,和第94页。
WW
®
®)
@
S)
353;
3
:peorpLEom
1918年,几过仑在一部题为《美国事会的掌控之下,他们都是生意人或者
的高级学术研究》(TheHigherLearning”受生意人控制的政治家和牧师。凡过仑
iAmerica)的著作中,将其分析应用到”发现下列现象离奇而有趣,即通过追求
大学中。书的副标题“实业家大学行为。利润来证明其捕食能力的个人,预期要
备忘录”反映了论文的主题,即“礼仪。去了解做学问的事,
性-生产性”二分法。凡孝仓推论,知的确预计到了一种固执的相反的偏
识是当闲散好奇心和技艺的本能任意发见,应当容易看到,事实是,在任何一
挥时,通过大学的制度获得并提出的。点上董事会都没有实质性用处;它们唯
但是,大学已经被文化的某些评价所污一有效的功能是,妨碍与工商业性质无
米,它们给予追求礼仪性行为和金钱性。关的,并在其能力和其惯常兴趣范围之
职业很高的地位。大学政策在托管人董。外的学术管理。"
*参见1954年版的托尔斯坦,凡孝仓的《美国的高级学术研究》一书第66页。该书由
美国斯坦福大学出版社出版。
资本主义的稳定性与长期趋东
凡孝仑将他对金钱性职业与生产性职业的区别用于发展经济同期理论,
思考资本主义在相当长时期中的趋势。在周期的繁荣阶段,生意人的金钱
活动导致信用扩张,公司获取利润的无形能力被赋予了较高的价值。增加
的资本价值用于附加的新增信用。这一过程暂时得到自我增强,因为信用
的数量和资本产品的附加价值,继续随着资本产品价格的上升而扩张。但
是,在资本产品的获利能力和其以证券行市体现的价值之间,存在着较大
的差距,这一点很快就变得明显了,清偿与紧缩阶段开始了。

下降的价格、产量以及就业,还有减少的信用,导致厂商在更加现实
的基础上改变资本结构。在周期的萧条阶段期间,较弱的厂商被排挤出去,
或者被大而强的厂商兼并,使美国行业的所有权与控制权和集中在较少人手
中。周期的萧条阶段包含了自我纠正的力量,因为实际工资下降,利润边
际提高。最后,超额信用被从经济中挤出来,资产负债表中表示的工商业
的人金钱价值,反映了对行业产量更加合理的评估。
和
,354
于
Ss
i
第12章”新古典经济学的制度性与历史性批判
3
r
尽管凡孝仑所有的著作都推测了制度的长期趋势,但是,他在《有困
阶级论》、《企业论》以及名为“社会主义理论中一些被忽视的观点”
(SomeNeglectedPointsintheTheoryofSocialism)的短文中,最为明确地涉及
了这些问题。

凡勃仑根据金钱性职业与生产性职业碰撞所产生的冲突与紧张,对未
来进行推测。他在《有闲阶级论》中指出,竞赛、奉承以及产品消费中招
人反感的比较,将导致一个专心于炫漆性消费、炫次性浪费以及广告和营
销成本增加的经济体。只要生产受到追求利润的生意人的控制,我们就能
预期到,阻止人类进步的产品流将增加。但是,生产的职业造成了事务性
的因果关系,如果工人与工程师凭借他们与这些关系的日常联系,掌握对
制度的控制,那么,工业化的经济体就可能实现其承诺。

凡孝仑认为,资本主义条件下所形成的消费模式竞赛,力量是如此强
大,以至于它可能在制度中产生紧张和引起工人阶级的不满,并导致私人
财产的终结。任何数量的个人实际绝对收入增加,都不能减缓这些紧张,
因为个人都希望比其他人得到更多,而不是多一点点:
作为人类的本性,每个人都在奋斗,以拥有上比其邻居更多的东西,这
与私人财产制度是不可分的。-……推论似乎是……不可能有和平——这一
点必须得到承认——从这种不光彩的竞赛形式来说,或者从与之相伴的不
满即废除私人财产的一方来说。@
资本主义可能因为个人对其自身相对福利的关注而终皆,这一主张泵
凡孝仑的分析中藻雇特性的又一个例子。几邯仑提出,资本主义不是由于
其失败而是由于其成功才终止的。

然而,凡勃仑拒绝彻底表明自己的意见,并提出这一切实际上不可能
发生。资本主义的未来和私人财产是不确定的。一种可能性是,工人阶级
和工程师中间产生的科技态度日益增强,导致对生意人的替代,因此,对
经济体的控制将传到技术统治论者手中。凡过仑说,如果有了这些发展,
它将意味着缺席所有权、人金融操纵以及利润追求的终结,行业将被加以引
QD同上,些397~398页。
355
A想史
DryofSonomePsp
导而生产对人类有用的庆吴。

也有可能是发生一场真正的社会主义革命,结束阶级差异、王朝政治
以及国际仇恨。还有另一种可能是,当工人阶级和工程师遵从民族抱负、
好战目标以及民主主义时,随之而起的是一场右倾的经济与政治运动,平
息下来后进入集权国家。由于深深地根植于达尔文的进化论,几艺仑没有
确定地预测未来。凡勃仑关于未来的观点中,唯一的必然性是变化。脆纶
的制度是否将击败事务性的技术,还有待观察:
两种敌对的因素中,哪一种能证明在长期中更加强大,这有几分盲目
精测;但是,可预测的未来,似乎属于其中的一个或另一个。看上去很有
可能这样说,对企业的完全支配必然是一种短时间的支配。中
岂亏公约页献
一般而言,非正统经济理论,尤其是凡邢仑的理论,经第从关于经济
思想史的著作中被忽略掉,这或许是因为,它们对于现代正统经济理论只
有非常小的直接作用。凡过仑对正统经济理论非常不满。正统经济理论在
阿尔弗雷德.马鞭尔的经济学中获得了最成熟的表述。凡勃仑希望抛弃这
一体系,因为他认为,这一体系坚持着错误的方法。他声称正统理论在方
法上是原子论的,试图从对经济体组成部分即家庭与厂商的初始分析开妈,
继续下去从而在整体上了解经济体。但是,整体与部分的总和并不相同;
凡勃仑认为,一种适当的方法应当始于文化、社会以及经济层面。

有人说凡邯仑根本不是一个经济学家,而是一个社会学家,并且在一
些经济学家看来,是一个思维不清晰的社会科学家。不把几孝仑当成经济
学家的看法,至少与他的方法和贡献是一致的。凡过仑的一种理论恰好是
我们不能通过使人类的经济行为与其他活动相孤立的方法,来了解所谓的
经济体。因此,凡勃仑实际上提出了社会科学的融合。

凡过仑对正统理论家们感兴趣的一系列问题并不感兴趣。他希望了解
制度结构的发展,这种制度结构是通过指导经济活动的思想习惯形成的。
QD参见托尔斯坦,凡支仑的《企业论》第400页。
‘
t
4
'
}
'356
EE
于
hs
人
六
dn
第12童新古典经济学的制度性与历史特批羯
从这个角度来看,几皂仑的贡献能够锌看做是对正统理论的名元。然而,
凡过仑认为,一旦了解了变化的制度结构,解决正统理论所研究的比较有
限而狭窄的问题,就要求一套不同于经济学家现在所使用的假设和工具。
他坚持认为,经济学必须使用一种演进的方法,抛弃有关自然均衡或正常
均衡的目的论概念;必须与其他社会科学相融合;必须抛弃关于竞争市场
和享乐主义家庭等不现实的假设,必须认识到和谐体制的含蓄假设使其大
部分分析无效;必须用更多的调查和统计工作补充其贫竣的演绎方法。

凡孝仑发现了经济学中的很多缺陷,但是,他所提出的可供选择的方
法,并不是非常宣有成效。他没有运用容易确定的假设和符合逻辑的上层
建筑来构建庞大的模型,从而无可变更地得出明确的结论。他用本能心理
替代正统理论的享乐主义,而本能心理后来也受到了心理学家的否定。

正统理论家通过替换一个较少引起反对的术语,回应了凡扫仑对他们
运用享乐主义概念的批评但是,他们的基本模型仍然假设理性的且精于
算计的家庭与厂商。凡过仑握击的完全竞争市场假设,并没有因人垄断竞
争市场和窒头市场理论而得到明显修改,尽管这种理论的开发者之一张伯
伦向凡勃仑致以谢意。这些市场理论至今仍然不能令人满意。作为福利经
济学发展的结果,以及凯恩斯关于均衡可以与大量失业同时发生这一结论,
均衡本身不再被认为是合意的。凡勃仑对消费者主权概念的择击,以及他
对竞赛与广告在经济体中作用的分析,在不完全市场理论和约翰肯尼
思.加尔布雷思(JohnKennethGalbraith,1908一)的著作中得到了进一步
的延伸。第二次世界大战后,随着关注点转向世界不发达国家的增长与发
展问题,凡过仑关于进化变革的观点,引起了一些注意。

然而,凡孝仑的另一个贡献来自于他偶尔宣扬但从未实践的某种东
西——用科学的方法收集实际材料来检验假设。他以正统理论是一个完全
演绎的系统,未能从经验上检验其假设或结论为由来批评正统理论。然而,
凡勃仑自己的理论,也不是以适合于检验的形式提出来的,他也未能运用
统计资料来证明其主张。凡勃仑对正统理论的批评,的确在一定程度上人迫
使经济学家更加关注实际;过去六十年期间,经济学中经验研究的奇特增
长,可以部分地解释为是对凡过仑遗愿的反应。我们马上将考察韦斯利-
克莱尔,米切尔的一些贡献,他是几过仑的学生,在收集数据用于分析经
357,
[1
eeHSEE
:NYofeom
济周期方面,他是一位开振。

最后,我们必须承认凡勃仑对经济学的规范性贡献。其著作从头到尾
不仅是对正统的一种科学背离,而且是一种道德背离。正统理论家,例如
凡勃仑的老师约翰*贝获-克拉克,吃惊于现代工业经济体所生产的物质
福利,而凡邯仑则运用他对客观现实的讽刺和看法来描述充满不幸的经济
体。对于很多认为政府行为可以矫正金钱文化最明显缺点的人来说,凡蔓
仓成了一种号召力。
二、韦斯利.,殉末尔:米切尔
1896年,韦斯利.克菜尔.米切尔进入世加妈大和学学习石典文学。在
修完约翰-杜威(JohnDewey)和托尔斯坦.凡勃仑的课程之后,他对哲学
与经济学变得更感兴趣,并最终决定研究经济学。米切尔后来成为20世纪
美国最主要的一位经济学家:经济周期的权威,建立研究机构研究经济体
的先驱,经济理论发展的敏锐观察家。尽管米切尔不完全接受凡孝仓的很
多思想,但他的经济学不是正统经济学,所以,他通常被确定为所谓的制
度学派。他认同并补充了凡勃仑对正统经济理论的一些批评,但是,他没
有试图构建一个完整的理论结构来解释工业经济体的演进。米切尔试图遵
循凡勃仑在其关于方法的短文中所推荐的方法,运用经验材料仔细调查,
并为其全部理论研究打基础。他作为一名学者和研究者的风范,以及他为
建立全国经济研究局来分析和收集宏观经济数据所做的工作,比他对纯粹
理论的贡献更为重要。

米切尔的许多短文,以及他的《经济理论类型演讲笔记》(LectureNotes
onTypesofEconomicTheory),都表达了他对正统经济理论的看法。2在一封
号给J.M.克拉克的非同寻常的快信中,他透露了使他偏离主流经济理论的
Q@@”这些笔记是由一个叫约翰,梅耶斯的学生在1926一1927学年记录,并以油印形式复制的。
增加部分是由梅耶斯在后来年份完成的,直至1935年;那一年的版本比1926一1927年的版本多了
大约百分之三十。米切尔1948年去世,1949年的油印版由奥古斯塔斯*M.凯莱出版公司出版。最
佳的原始资料是在由约瑟夫道夫曼编辑的书籍中,他也写了导言。同时请参见韦斯利.克莱尔
米切尔的《经济理论类型》一书,该书由美国奥二斯塔斯*M:凯莱出版公司于1967年出版。
+

1

0

:358
、
0
oe
有
第12意新古虎经济学的制度忻与历史和性批判
思想转变,2米切尔说,年轻时他就开始谨欢具体的而不是抽象的问古和万
法。他回忆起他的寻祖母,“她是最好的浸信会教友,确切地知道上帝是如
何规划世界的。”@米切尔记得他如何开发了“一种顽皮的乐趣,方法是显
摆我的婕祖母无法应对的好辑难题。她总是汶回到钦辑安排中而无视实际,
我却逐渐对实际产生了个人兴趣”。

米切尔通过引证下列事实来解释他向经济学的靠拢,即当他进入芝加
可大学时,他既学习哲学又学习经济学。他认为经济学比哲学容易,尽管
从岁奈到马歇尔的经济理论“与形而上学者的精致相比,属于相当粗炊的
事情……理论中的技术部分容易。给我假设前提,我能纺出按码论的推测
来。我也知道我的“推论,是无用的"*@。米切尔对凡勃仑印象深刻,并认
为“在使理论尽量延长方面,很少有人能比得上他”。然而,米切尔意识
到,凡孝仑的体系像正统理论一样,具有相同的方法上的弱点,两者都未
令人满意地检验其假设或者结论。“但是,如果能有什么令我确信正统经济
学的标准步骤不适应科学检验的话,那就是,凡勃仑在另一套假定体系下
的精湛表演,只获得了非常少的肯定。”®

这一独特态度体现在米切尔的两项终身成就中。在对经济思想史的研
究中,他并不对个别理论家说了什么感兴趣,而是对下列事情感兴趣,即
为什么他们择击某些问题而不是另一些问题,为什么他们毫无疑义地接受
某些假定,为什么他们同时代的人接受他们的结论,并认为这些结论重要。
米切尔在经济理论史方面的作品,可能代表了最佳的相对论者观点。他断
定,经济理论在很大程度上能被解释为对当时间题的智力反应。这一看法
也体现在他对经济周期的研究中。他并未留下一个建立在抽象假定基础上
的周期理论,从假定中演绎出结论来。他的方法是仔细构建并解释时间序
Q@在很多地方能找到这封令人愉快的信件,包括露西。斯普拉格,米切泉的“个人系描,
它们收录在《韦斯利:区莱尔:米切尔:经济科学家》一书第93~99页中,该书由美国国家经济
研究局于1952出版,同时它们也收录在J.M.克拉克的《社会经济学前言》一书的第410~416页,
该书由美国法勒与瑞耐哈特出版公司于1936年出版,我们引用了克拉克的。

回同上,第410页。

@同上。

@同上,第411页。

加同上,第412页。
i399
iBap-
ed
0
HSEE
SOoyrrRp
列,使之作为初始步骤来证明他所提出的暂定理论。有时,他对经济周期
的研究看上去几乎与理论无关,但是,在整个分析下面存在着一种理论
结构。

米切尔批评了正统理论的抽象模型。“投机类型的经济理论像高等数学
或诗歌那样,被廉价而便利地生产出来——倘若一个人有这种天赋的话。
正如那些想象的产物一样,这种经济理论与现实之间是一种同样有问题的
关系。”@他也反对正统理论的享乐主义心理假定,但没有接受凡勃仑的本
能理论。他声称,依据以经验为基础的行为主义心理,社会科学能够对人
类活动做出一种更好的解释,与让不同分支独自行动所实现的方法相比,
他提倡运用一种更加一般化的方法来研究人类行为。正统理论错误地聚焦
系统中的常态和均衡,而不是考察系统的动态相关性。

在他对经济周期的研究中,米切尔特别强调演进累积因果方法。米切
尔著作中所暗含的是一种对正统理论的道德背离以及科学背离。米切尔希
望运用经济知识来改善福利,他认为对经济体的研究揭示出,为了更好地
综合厂商的活动,并更好地控制经济活动中的波动,需要国家的计划。

米切尔将凡勃仑对金钱性职业与生产性职业的划分,作为对其经济周
期研究方法的一种广泛指导。经济活动中的波动,在很大程度上能够通过
工商业对利润率变化的反应得到解释。因为经济决策是在预期与不确定环
境中做出的,所以,生意人的投资决策总是反映出对未来乐观或者悲观的
看法。在具有发达货币制度的经济体中,能预期到经济活动中的波动;因
此,具有常态、静态以及均衡这些概念框架的正统理论是不适当的。米切
尔并未试图构建经济周期的另一种抽象模型。取而代之的是,他努力解释
经济周期期间发生了什么,并提出他所谓的对周期的描述性分析。因为每
个周期都是独特的,所以,发展一种一般性理论的可能性就受到限制;然
而,所有的周期都具有某些相似点,原因是所有的周期都揭示出了在萧条、
复苏、繁荣、和危机不同阶段经济力量的相互作用。

尽管米切尔之前的一些人将周期看做是一个自生的过程,但是,他第
@@参见韦斯利*克莱尔.米切尔的“社会科学研究院”一文,该文载于类国大等协会学报琳
起1929年期,该引文旱白第63页。
1

'360

aman
iy
ee
和
让
ty
人
第12章新古暴经济学的制度性与历史怕执淹
一个赋予这一概念明确的形式,并用广沁的经济数据予以支持。他对周期
的解释基于工商业对利淘水平变化的反应。萧条携带着其后复苏的种子,
因为利息率下降,没有效率的厂商被排除,不变成本与可变成本两者都下
降,存货减少,等等。繁荣也携带着危机与萧条的种子,因为成本上升,
利润随之受到挤压。

米切尔的描述性分析,实际上体现出一个学者对理论、描述以及历史
的明智混合,并且全无数学障碍,在这点上有些像马葡尔的分析。然而,
支撑马歇尔微观经济分析的坚硬理论内核不见了,程度之严重以至于一些
人将米切尔的研究称作是没有理论的度量。另一些具有后凯恩斯主义后见
之明的人,在米切尔的研究中发现了乘数过程、加速原理,以及凯恩斯资
本边际效率和流动性偏好的对应内容。米切尔认为,不能离开经济体的其
余部分来考察经济周期,它们是系统的重要部分,并且实际上是由系统产
生的。当周期的每个阶段向下一阶段演进时,社会制度结构发生改变,所
以“每一代的经济学家将看到重铸他们在年轻时所学的经济周期理论的理
由”。@

1920年,在四十五岁时,米切尔创建了全国经济研究局。这一私人非
赢利组织对于资助美国的经济研究极为重要。尽管它最重要的成就包括国
民收入的度量和经济周期研究,但是,它资助了对经济体几乎全部领域的
研究。如果我们要考察美国经济研究的发展,那么,米切尔的作用至少需
要一个长篇章节来阐述。在第16章中,我们将在他的一些学生——例如,
西蒙-库兹涅区(SimonKuznets,1901一1985)一的著作中,了解到米切
尔的若干直接影响,并且,在对度量经济活动比对构建抽象演绎模型更感
久趣的经济学家的著作中,了人解到他的间接影响。
中、约间.RR,康亡斯
约输,R…康芒斯比几勒仑小五岁,但比米切尔大十二岁,是太一位来日
美国中西部的非正统经济学家。他出生于俄玄俄州,在印地安那州长大,就
计于奥伯林学院,接受了当时一流的古典教育,包括一门繁重的神学课程,
(DD韦斯利:克莱尔:米切尔.经济周期.美国,但特:'官兰元林出版公司,1913:.583
1
1
361};
和
ee
这门课由通常也是神职人员的教授开设。他在约翰斯.上霍普金斯大学进行经
济学研究生学习,并深受理查德*T:伊利(RichardT.Ely)的影响。

因为伊利留学德国,受到德国历史学派的影响,所以,约翰斯.霍普
金斯大学的政治经济学包括经济学、政治科学、社会学,还有历史。伊利
在劳动经济学方面的兴趣——他在康芒斯进入约翰斯霍普金斯大学的两
年前即1886年出版了《美国劳工运动》(LabourMovementinAmerica)——
传给了他的学生,康芒斯的全部生涯都献给了这一经济学领域。两年后,
康芒斯离开约翰斯*管普金斯大学,在1904年最终随同伊利前往威斯康星
大学之前,曾经在一些地方教书。

被一些人称作威斯康星学派(Wisconsinschool)的经济学方法,主要是
在康芒斯的影响下在威斯康星大学得到发展。这种方法支撑了美国的非正
统经济理论,发动了改变美国经济结构与功能的改革,从这些方面来看它
是重要的。在到威斯康星大学之前,康芒斯并没有在任何一所大学竺很久,
这或许是因为他的政治和经济观点,或许是因为作为大学本科教师,他没
有得到充分的认可。然而在威斯康星大学,他为其具有真知灼见的异议找
到了肥沃的土壤,其至得到了激进的政治家的鼓励,这些政治家热衷于寻
找愿意支持社会改革的学术专家。

在1932年退休之前,在威斯康星大学的二十八年期间,康芒斯在三个
主要领域中为经济学做出了重要贡献:社会改革、研究生教育以及劳动经
济学。也许他最重要的贡献是对于社会立法的制定发挥了作用。这一立法
改变了美国经济体的结构。康芒斯的第一本著作《财富的分配》(Disiribu-
tionofWealth,1893)并没有获得充分认同。批评家认为,这是康芒斯为其
社会主义思想确立科学基础的一次令人不满尝试。然而,康芒斯并不是一
个试图改变私人财产和自由企业社会结构的革命家。他认为,资本主义的
本质可以并且应当保持完整无缺,但是,经济秩序的运转规则需要变革,
以消除自由放任经济体的明显缺陷。在威斯康星大学,他的观点获得了州
长拉,。于利特(LaFollette)的支持。

康芒斯在威斯康星大学的几年期间(1904一1932),如今已成为平常之
事的专业学者与政治家之间的关系得到发展,这种关系在富兰克林,罗斯
福新政(NewDeal)时期在全国范围内再次出现。威斯康星州政府广泛利用
;1362
第12章新古典经济学的制度性与历史性抠判
在麦迪偿的大学教职员工充当新思想的智守团、法律的起草者以及指定委
员会的成员。康芒斯在威斯康星大学的经历显示,他花了大量的时间帮助
起草、审查通过并执行社会立法。

在这些成就中,一种依稀可辨的模式发展起来了。康芒斯经常在其研
究生的帮助下透彻地研究某个问题。他与经济界中受到新立法影响的人讨
论问题,获得更激进的工商界人士或工人领袖的支持。法律通过之后,他
四处游说,使用其他手段推动新立法向其他州传播。体现在罗斯福新政社
会立法中的很多思想来自于威斯康星州,这一点很少有人怀疑。毫无疑问,
1932年,很多在麦迪逊接受培养的经济学家和其他人都搬到了华盛顿特区。
康时斯的演闫
康芒斯被描述为“福利国家运动的智力来源”。,1904年,他a到达麦迪
逊,第二年就为州长拉,弗利特起草了一项行政事务法;在后来的几年中,
他影响着下列领域中的社会立法:公共事业规制,产业安全法,工人的赔
偿,童工法,妇女最低工资法,失业赔偿法。失业赔偿立法可能是康芒斯
在社会立法方面最大的成就。他对1920年经济萧条的反应,以及对欧洲失
业赔偿计划的研究,促使他为威斯康星州的立法机构起草了一项法案。这
一法案一次又一次地被提出,直至1932年康芒斯以前的一个学生哈罗德。
格罗弗斯(HaroldGroves,当时既是一位参议员,又是一位在大学专门研究
公共财政的经济学教授)又提出这一法案,并最终获得通过。1934年,当
罗斯福强烈要求国会通过一项失业赔偿法时,他组建了一个经济安全委员
会来担议立法,委员会的主管是康芒斯的学生E.E.威特(E.E.Witte),当
时他是威斯康星大学的经济学教授。

康芒斯在这些社会立法领域的成就,来自于他深信现代工业经济体要
想正常运转并实现社会公平,就需要政府干也。起源于威斯康星州的大多
数立法并不会震撼现代读者,尤其是激进的、好幻想的读者。然而,在康
站斯时代,这些社会改革思想在美国并不能得到普遍认同。在这一点上,
Q@@参见肯尼斯.伯丁的“制度主义新观察”一文,该文载于《美国经济评论》1957年5月
48期,该引文出自第7页。
'
1

'

'

'

t
363:
I/WeroAEaoptwraceorsgtA?
康芒斯代表了一种非同导常的非正统经济学家类型。他所做的不只是有反对
正统理论所主张的在极大程度上不干预市场配置资源;他对通过社会立法
改变现状感兴趣,并积极参与,努力去实践。并不是他的所有努力都获得
了成功,例如,在实现全国健康保险计划上他并未成功。

康芒斯的第二个贡献与他在社会改革领域的努力相连。对于全世界的
经济学家来说,威斯康星大学经济系以其作为重要的研究生培训中心而著
称。威斯康星大学一次所授予的经济学博士学位,比其他任何大学的都多。
更重要的是,康芒斯特殊的经济学方法,深埋在经济系的架构中;因此,20
世纪80年代之前,“威斯康星学派”方法一直持续着。这一点与几勃仑或
米切尔形成了鲜明对比,他们俩人没有对任何研究生计划产生持久的影响。

在C.E.艾瑞斯(C.E.Ayres)领导下,位于奥斯丁的德克萨斯大学经
济系,以及在艾伦:格仑奇(AllanGruchy)领导下的马里兰大学经济系,
也在短时期内保持着特定的非正统方法但是,这些机构所授予的博士学
位数量以及它们的影响力,与威斯康星大学相比要小。要了人解威斯康星学
派方法的死亡,更一般地,要了人解集中于特定院系的非正统经济学研究生
教育的终结,需要更多的历史视角。除了少数教职员工外,威斯康星大学、
德克萨斯大学以及马里兰大学,似乎都安全地回到了正统信仰中来。

无论如何,康芒斯的方法似乎并不能通过研究生们持续下去,或者传
播到其他大学,原因在于,在威斯康星大学受到训练的经济学家,在极大
程度上是以经济学的应用领域而不是以经济理论为导向的。他们中的大批
人毕业后,供职于政府部门、研究机构以及大学。但是,因为对诸如劳动、
公共财政以及公共事业一类的问题感兴趣,所以他们很少对当时几乎专
指微观经济学的正统理论产生兴趣。正如我们将看到的那样,康芒斯批评
正统理论,但是却将他的大部分时间花在了应用领域和社会改革中。

如今,麦迪偿的博士项目与美国其他大学的博士项目具有相同的惯例,
这表明蒙恩于康芒斯的威斯康星学派方法已经死去。然而在康芒斯阶段,
甚至第二次世界大战之后的一段时期,与今天一般的大学本科经济学专业
在标准中级理论课程方面所受的训练相比,在威斯康星大学受到更少的正
统经济理论训练,则更有可能获得博士学位。但这已不再是事实了。不过,
康芒斯的影响导致威斯康星大学在将近五十年的时间中培养出大量的经济
;364
E人
第12草“新石与经济学的制度性与历史性投判
学家,他们把在应用经济学和社会改亩方面的偏好,传送到人研究机构、政
府部门以及其他大学。

康芒斯的第三个重要贡献是在劳动经济学领域。当康芒斯的老师理查
德.T.伊利从约翰斯-霍普金斯大学转到威斯康星大学时,他把康芒斯一
起带着。因为伊利对劳工运动史感兴趣,所以,他开始收集劳工史方面的
资料。他希望康芒斯根据这些资料研究出权威性的美国劳工史,这些工作
占据了康芒斯在威斯康星大学的大部分学术时间。在其研究生的大力帮助
下,1910年康芒斯出版了《美国工业社会的文献史》(ADocumentaryHistory
ofAmericanIndustirialSociety),这是一部与劳工史有关的十卷重要资料集。
紧接着是四卷的《美国劳工中》(HistoryofLaborintheUnitedStates):1918
年出版了两卷,1935年又出版了两卷。康芒斯成了关于美国劳工方面公认
的权威,威斯康星大学也变成最主要的培养劳动经济学家的大学。最著名
的毕业生可能是塞利格:珀尔曼(SeligPeriman),他的《劳工运动理论》
(TheoryoftheLaborMovement,1928)至今依旧是一部经典。
康基斯的经济思想
尽管康亡斯独立地得出他对正统经济理论的批评:但是,这一批评与
凡勃仑和米切尔的批评相似。他研究社会问题的整个方法,否决了新古典
理论狭窄的、静态的、演绎的方法。康芒斯试图将全部社会科学与法律带
入分析中。他将社会与经济看做是演进与变化的,强烈地反对正统理论几
乎唯一的演绎方法,以及享乐主义代理人和竞争性市场的假设。最后,康
芒斯认为,自由放任政策所依据的含蕾假设,即经济体是和谐的,与他的
经验观察相反。

康芒斯对美国资本主义进行分析的起点,与正统价格理论的分析起点
相同,但分析本身完全不同。他声称,有关价格形成与交换的正统理论是
不现实的。它假定理性的个人在竞争性市场中几乎机械地行动。康芒斯说,
并不是在竞争性市场中行动的、原子式的、具有享乐主义的个人,形成了
将经济体单独部分连接起来的交换关系。正统价格理论可能令人满意地解
释了一些非常特殊情形下的交换与价格例如,高度组织的商品市场或者
安全市场,因为在这些市场中,存在交换但不存在交换关系(exchangerela-
365
-2
ee
iryofEenormeDrsgrt
366
tionship)。在这些市场中,买者与卖者之间完全是匿名的,影响通第的市
场交易的习惯、风俗以及所有的文化、社会、心理力量都缺失了。在康芒
斯的理论结构中,交易成了一种主要因素:
实际上,交易变成了经济学、物理学、心理学、伦理学、法字以及政
治学的聚会点。单个交易是明确包含所有这些内容的一种观察单位,原因
是,人类的一些意愿,诸如选择可供选择的事物、克服阻力、协调自然资
源与人力资源,受到有关效用、同情、责任或者它们对立面承诺或警告的
驴请,这些意愿被解释和执行公民权利、责任、自由的政府官员或者工商
业企业或工会官员加以放大、抑制或者展露。正是人类的这些愿望,在有
限资源和机械力的社会中,使得个人行为适合或者不适合国家、政治、工
商业、劳工、家庭以及其他集体运动的集合行为。_
康芒斯发现了经济体中的三种交易类型。“天卖区易通过法律对手之间
的自愿协议转移(transfer)财富所有权(ownership》。”®权利的法律平等
并不意味着平等的经济力量。确定最终市场和要素市场价格的买卖交易
(bargainingtransactions)是正统价格理论的主题,但是,这一理论的确只适
用于竞争性市场的不正常情形,在竞争性市场中,讨价还价的力量、强迫、
说服、习惯、风俗以及法律都被假设忽略了。第二种类型的交易是管理交
易(managerialtransaction),涉及法律上和经济上上级对下级的命令。“它
是工头与工人人、州长与市民、管理者与被管理者、主人与仆人、所有者与
妈隶之间的关系。”®管理交易涉及财富的创造。康芒斯确定的第三种类型
的交易是限额交易(rationingtransactions)。它们涉及“在若干参与者之间
达成一种协议的谈判,这些参与者有权力将收益与负担分配给合办企业的
成员”,@然后,康芒斯继续进行曾述,界定他所谓的制度:
议三种类弄的交易今在一起,上成为经济研究上一个较大的单位,在英
约输,R.…康芒斯.资本主义的法律基础.美国麦克米三出彼公丁,1924:3
约翰R'康芒斯.制度经济学.美国麦克米兰出版公司,1934:68

同上,第64页。

同上,第67~68页。
VD
®@
3
(@)
0oe
i
i
第12意新古次经济学的制流性与历史性氟羯
国和美国的实践中被称作运行中的机构。运转规则使其不断运行,从守巍、
公司、工会、同业协会直到国家本身,正是这些运行中的机构,我们称之
为制度。消极的概念是“集团”.积极的概念是“运行中的机构。”包
制度(institution)被界定为控制、解放、扩张个体行动的集体行动。
经济交易涉及冲突——我得到的越多,你得到的就越少。这些冲突并未在
大部分交易中体现出来,原因是随着时间的变化,通过风俗、习惯、法律
等开创了先例,这些先例从冲突中产生了秩序。康芒斯将这些先例称作运
行中的机构运转规则。

借助康芒斯方法的这一基本轮廓,有可能略述他对美国资本主义的分析。
新古典理论主张,由稀缺资源问题引起的冲突,在非个人的竞争性市场中能
够得到解决,这种竞争性市场通过假设,从分析中消除了所有的文化、社会、
心理以及法律因素。新古典理论认为,这些冲突在竞争性市场中的解决所引
起的结果,在极大程度上优于通过政府干预可能取得的任何结果。

康芒斯方法的基本要点是将社会科学、历史以及法律包含到他的分析
中,并认识到为了产生合意的社会结果,政府干预经常是必要的。我们的
大部分经济活动并不是个体活动;我们作为集团成员来行动,集团受到运
行中的机构运转规则的指导与影响。尽管这些运转规则的功能是从冲突中
产生秩序,然而有时历史导致的变革又引起新的冲突。随后,这些冲突与
争执得到解决,并且,旧的运转规则得到修正。这是一个无止境的正在进
行的过程。康芒斯认为,经济学的适当主题是通过集体行动塑造我们的生
活与社会制度。这种集体行动不仅控制个体行动,而且通过对其他个体加
以抑制来解放个体行动,“使其免受强迫、威胁、歧视或者不公平的竞争。
并且,集体行动还不仅是对个体行动的抑制和解放——它是个体意志的扩
张(expansion),扩张到远远超过他靠自己的微弱行为所能做到的范畴”。@

因为非规制的经济体产生不合意的社会结果,所以,资本主义需要通
过政府干预予以修正。防止经济萧条的货币政策、意识到劳工组织权利的
立潜、援助失业者的工人赔企、关怀不幸者的健康与意外保险、阻止芍断
D同上,第69页。
中同上,第73页。
367,

2\
appp

Tp

We
|strofEeveeisphy
和
.
:368
行为的公共事业规制以及其他社会改革,者是由康六斯所倡村的。因此,
尽管他基本上没有对正统理论产生影响,但是,他所倡导并帮助完成的改
革,极大地影响了美国资本主义的制度结构。
五、约间,A:霍市斯
尽管英国是从斯密到马歇尔的正统经济理论大本受,信奉的主要原则
是非规制的市场将导致社会福利最大化,然而,也存在大批异端。其中最
有影响力的可能就是约翰*A.霍布斯,他的非正统思想成为当今英国福利
国家的智力源泉。在其第一部经济学著作出版后不入,霍布斯的教师生涯
就结束了。他和于了工作,原因是“读了我的书的一位经济学教授的干涉,
他认为有理由将我的书等同于试图证明地球是扁平的”2?(那种异端学说
——编者注)。然而,足以自给的收入使他能够继续对正统理论进行邱击,
他出版了将近四十部书,并发表了大量文章。在凯恩斯于《通论》中对他
加以赞赏之前,他的著作在学术圈中从未得到充分认同;尽管填布斯对纯
理论的影响几乎被忽略,然而,在塑造英国经济政策方面,他是有一定地
位的。霍布斯像很多非正统经济学家一样,对正统理论的不充分性有先见
之明,并且有能力将它们描述出来,但是,却从来没有能力曾明一种能够
推翻公认学说的理论结构。

从宽泛的角度看,才布斯的非正统思想是对下列公认学说的一种择击,
即自由放任是最佳的政策,因为市场将导致社会福利最大化。正统理论主
张,竞争性市场将在极大程度上,在最低可能的社会成本上,生产至高无
上的消费者所要求的产品。这些市场所产生的收入分配,是根据参与者的
生产力而给予其酬劳。此外,这些经济力量的运转,将导致社会资源的充
分利用。因为通常来说,价格很好地度量了经济体中所发生的成本和所生
产的效用,所以,它们是一个社会所获得的福利的指标。

尽管替布斯接受了正统理论的一些主要假设,然而,就自由放任市场
经济的适当性而言,他得出了完全不同的结论。他发现他所处时代英国经
济运行中的三个主要缺陷。第一,,它未能提供充分就业,原因是存在慢性
DD约翰.AA,乱布斯.经济异疾的志明.美国,帮兄杀兰出版公司,1938:30
消费不足或者过度储蓄。第二,收入分配不公平地偏问那些高收入群体,
主要原因是他们出众的讨价还价能力。第三,市场并不是对社会成本与社
会效用的一种好的度量,因为整个价格系统是以货币利润为导向的。正统
思想家在经济体中发现和谐,然后构建了一种理论来证明那种和谐,霍布
斯则假设自由放任经济体的负面影响,然后试图构建一种理论结构来弥补
现有工业社会的缺陷。霍布斯主张,如果一个社会的目标被明确地加以界
定,那么,经济理论将允许社会实现“好的生活”。

他反对约翰:内维尔凯恩斯关于我们能够区分是什么和应当是什么
的观点,并且反对将活动仅限于是什么的正统分析人和倾向。在霍布斯看来,
就经济理论帮助社会实现“所应当的”而言,它正好是有用的。正统理论
所尝试的规范-实证二分法是不可能的,因为同样的事实既是道德的也是
经济的。霍布斯对正统理论的择击,始于他在与别人合著的第一本书中对
茸伊定律的否定:
因此,我们得出下列结论,即从亚当。斯宅以米,全部经济教义所依据
的基础即每年生产的数量是由可利用的正常要素、资本以及劳动决定的,这
一基础是错误的,相反,所生产的数量永远不会超过这些总量所施加的限度,
它们可能并且实际上由于生产的减慢而被减少到远低于这一最大量,这种减
上瘟是不适当的桩蓝以及随之发生的殿给过剩的累积施加在生产上的,0
在支持过度储蓄导致经济萧条这一观点时,乱布斯及其合关者A.F.玛
麦瑞(A.F.Mummery)的论点是不完善的,主要是因为他们认同下列正统
观点,即全部储蓄随着投资广出回到收入流中。

在后来的著作中,霍布斯从未动摇由于充分就业下的过度储蓄,资本
主义趋向于导致经济萧条这一结论。1902年,他出版了《帝国主义》(lm-
perialism)一书,断言资本主义国家的殖民扩张,在很大程度上是充分就业
下所产生的过度储蓄以及产品供给过剩的一条出路。列宁大量借鉴了雹布
斯的帝国主义理论。霍布斯断定,通过帝国主义实践,通过战争支出,,通
过用以改善工人阶级条件的政府支出,通过增加国内奢侈上品消费以及通过
(DDA.FF玛霉瑞和约翰'A,埠布斯.工业生理学.美国:饥菜与米兰出版公司,1956,VI
3,

!
369;
ey
更加平等的收和分配,能够实现充分就业。道德上正确的备选方案是很明
确的:可以通过课税对收入进行重新分配,并与政府支出相结合,来改善
穷人的状况。

霍布斯广泛地就收入分配进行创作。他否定分配的边际生产力理论,
理由是将边际产品归因于分开的要素是不可能的。他认为,在现代复杂经
济体中,生产是一种社会企业或者合作企业;如果我们借助微分学,试图
确定不同生产要素的边际贡献,我们就是在回避围绕收入分配的道德问题。
此外,在对要素价格决定的分析中,正统理论含蓄地假设,不同生产要素
具有相同的讨价还价能力;但是他主张,对经济体的观察揭示出,劳动的
讨价还价状况相对较弱,这一点导致了低工资。支付给不同生产要素的报
酬,能被解析成以下三个部分:〈(1)仅仅允许要素维持其自身的报酬;
(2)允许要素提高数量和生产力的报酬;(3)超过了用于维持和提高所必
需数量的报酮,霍布斯称之为“非生产性剩余”。现代工业经济体所生产的
产量,超过了足以支付不同要素的维持部分,正是要素市场定价的讨价还
价过程,决定了哪种要素得到非生产性剩余。霍布斯宣称,土地因其天然
的稀缺性,得到一种非生产性剩余,资本因其一流的讨价还价能力和由于
芍断实力产生的人为稀缺性,得到一种非生产性剩余。给予劳动较高工资
的更加平等的收入分配,不仅会更加公平,而且也提高了劳动的生产力。
此外,更多的平等将增加消费,减少储黄,从而使经济体避人免经济萧条。

霍布斯并不满足于将他对正统理论的反对停留在这一点上。他继续对
有关价格系统含义的正统分析进行基本的、彻底的择击。根据霍布斯的观
点,正统理论错误地认为价格是生产产品的社会成本以及从产品消费中获
得的社会收益的反映。霍布斯认为,无论从成本方面,还是从收益方面,
价格都是对福利的一种不充分度量。“从这一情形上说,仍然将货币视为其
价值标准并将人类看做是赚钱手段的学科,没有能力面对深刻而复杂的构
成社会问题的人类问题。”%

霍布斯的方案是,我们应当考虑人类(human)成本以及人类效用,前
者不同于用价格表示的成本,后者不同于市场价格。在这一分析中,霍布
QD约翰,A,者布斯.社会问题.英国;J.尼斯波特出版公司,1901:;38
1

!

!

'

:370
ee
第12章新古典经济学的制度性与历史性拒判
斯在供给成本方面以及在需求收益方面,集中注意力于现代福利理论及所
及的外部性。他对需求方面的分析,反映出凡过仑的影响;他指向炫耀性
消费所导致的浪费,以及现代经济体中所实行的精巧的推销艺术。霍布其
的方案是消除政府的自由放任方法,以及现代经济体以利润为导向的性质。
“直接的社会控制对我们行业正常过程中私人逐利动机的兰代,对于社会重
建的任何合理方案来说都是必要的。”

这样简短考察霍布斯与总结他对正统理论所进行的全面反击的特色,
并没有什么区别。他否定萨伊定律,反对正统分配理论,认为价格系统是
对社会福利的一种不充分的度量,否定正统理论的规范-实证二分法,明
确提倡将道德考虑注和人经济分析中,认为利润动机对社会具有负面影响,
最重要的是,提倡终结自由放任。他经历了很多具有开创性的非正统思想
家的命运:他未能在正统学说控制下的学术界找到工作。他的思想通常未
经仔细考察就被否定。1913年,约翰.梅纳德凯恩斯评论说:“人们怀着
矛盾的感情阅读稚布斯先生的新书,希望看到使人兴奋的观点和一些从独
立的、个人的立场对正统学说所进行的富有成效的批评但也期待更加诡
辩的、令人误解的、不合常情的思想。”@

后来,随着凯恩斯对萨伊定律的否定及其从正统观点中的退出,他对
霍布斯的评价也因此改变。1936年,他赞扬霍布斯的《工业生理学》
(PhysiologyofIndustry)是“霍布斯先生在将近五十年的时间里,以饱满的
热情和不届不挠的勇气攻击正统学派的地位而创作的许多著作中的第一本,
也是最重要的一本,但它未能撼动正统学派。尽管这本书今天已经被完全
淡忘了,然而从某种意义上说,它的出版标志着经济思想史的新纪元”。®
凯恩斯接着认为,霍布斯属于一群重要的消费不足主义非正统经济学家,
“他们宁可赁着直觉,腾胱地不完全地探究真理,也不愿坚持错误,他们借
助简单的逻辑,条理清晰且前后一致地得出结论,但都建立在与事实不符
的前提上”。@
约翰"A霍布斯.工作和财富.美国麦死米宇出版公司,1914:;293

参见约翰梅纳德.凯恩斯的《经济杂志》1913年9月第23期第393页。

约翰*梅纳德.凯恩斯.就业、利息与货币通论.英国:麦克米兰出版公司,1936:364~365
同上,第371页。
LD
®
®
@
371.
21
-erannrl

nagap!
Ws
1
'372
yy
MS
;tory2GeorormeteorspAh
像大多数非正统经济学家一样,霍布斯的直觉见解并没有使他得出一个
一臻而有序的理论结构。因此,在现在的正统理论中,并不存在可以确认的
乱布斯成分。他揭示出正统经济学家满足于隐藏问题的特点。但当这些问题
最终被予以考虑时,解决方案则是由经济学家而不是霍布斯提出的。不过,
霍布斯对英国的经济政策具有重要作用,因为他的思想成为工党的主要智力
影响。在第二次世界大战之后的时期中,英国实行的对产业进行社会控制以
及包括充分就业政策的劳工计划就根植于约翰.A,堆布斯的经济学中。
冶,革
除了对正统理论提出异议外,新古典经济学的早期批评几乎没有共同
点。不同的经济学家以不同的方式表达了异议,但一般而言,它是与正统
理论在范围、方法和内容上的背离,以及对正统经济学家下列观点的否定,
即和谐万行于市场经济中,因此自由放任是适当的政府政策。所以,非正
统的背离是科学的背离也是道德的背离。很多非正统经济学家明确地指责
正统理论包含规范或道德判断,并试图通过假装发展一种实证科学来加以
掩藏。

德国历史学派反对奥地利学者尤其是门格尔抽象的理论化,讲德语的
经济学家之间发生了一场关于经济学适当方法的著名争论。历史学派也反
对下列古典观点,即古典经济理论与政策适用于欠发达国家如德国,也适
用于工业化国家如英国。他们希望保护其“幼稚产业”。与自由放任古典观
点相比,他们提倡政府发挥更大的作用。

凡近仑鼓吹科学的方法,但只创作了令人印象深刻的作品;米切尔受
教于凡过仑,他实践着科学,但难以根据所收集的数据得出理论上的结论。
没有哪位经济学家提出了一种理论结构来取代其所批评的模式。康芒斯的
确提供了一种可供选择的结构,但并不被后来的经济学家——正统的或者
非正统的——认真考虑。像康芒斯一样,埠布斯引人注目地影响了经济政
策,但是,他的理论贡献在很大程度上被忽视了将近三分之一世纪,直到
一些人回顾过去并认识到其见解的价值。

与大部分正统经济学家的结论相比,所有这些经济学家在不同程度上
第12章”新古典经济学的制度性与田史和性魏判
都得出要求市场中有更多政府十预的结论。一些评论者断定,因为非正统
理论的特定说法未能取代正统理论,所以,非正统理论是一种失败。我们
的观点有所不同。对非正统思想的考察揭示出,尽管它没有取代经济思想
主流,然而,它经常迫使正统理论进入新的路线,有了时还会提供开创性的
思想,这些思想成为公认理论结构的一部分。对思想潮流方向及其内容的
这些贡献不能被忽视。

对新古典经济学的制度主义批判和其他非正统批判,并没有在早期批
评家那里结束。对正统理论的抒击一直持续着(在某些场合下变得更强
烈)。尽管对政策的反击不一定都是正确的,然而非正统经济学家正在提出
的很多政策变草,实际上在20世纪已经实现了。夫布斯与其他英国改革家
影响着英国的社会政策,并且,美国制度主义者的很多思想在新政中得到
贯御。因此,非正统经济学家对于资本主义制度结构具有重大影响,他们
的很多批判由于对批判的反应而显得直率。

然而,在理论领域他们只有和较小的影响。随着西方经济制度结构的变
革,建立在与纯粹市场经济最为相关的制度结构基础上的新古典理论,并
没有发生变革;相反,它只是更深地退回到与政策有很少或者没有相关性
的纯粹抽象理论上来。正如我们在17章中考察最近的非正统经济思想时看
到的那样,对主流思想的挑战是制度主义性质的,因为它们是对凡勃仑、
康芒斯以及米切尔的智力继承,这些挑战越来越多地集中于正统理论与现
实的分高。
