\chapter{新古典经济学的\\制度性与历史性批判}

新古典经济学并不是毫无争议地就产生了。随着新古典经济学的出现,德国历史学派向其方
法论基础发出挑战,\textbf{整个19世纪80年代后期,在奥地利学者(特别是门格尔)与德国
  历史学派的一些成员之间,就经济学的适当方法问题存在着激烈的争论。新古典经济学横
  扫了英国与法国,却不包括德国,在美国也遇到了抵制。}因此,在19世纪末20世纪初前后,
美国经济学专业的研究生为获得博士学位而\textbf{在德国学习},这种情形仍然是一件平常
的事。很多这样的学者,满怀渊博的学识和对人德国历史学派观点的赞成态度返回美国。在
美国,除了对古典理论这种“输入性的”批评外,还有一些明显的美国因素,它们根植
于\textbf{美国中西部的平民主义渐进运动}。

本章首先总结主要发生在讲德语的经济学家之间有关方法的争论,然后考察20世纪一些美国
非正统经济学家的贡献,聚焦于经常作为\textbf{制度主义者}被提及的一组美国经济学家。

即使是有限的聚焦,也不容易决定将哪些经济学家包括进来。我们在历史学派中强调古斯塔
夫·冯·施穆勒(Hustav von Schmoller,1838--1917),是由于他在争论中的重要性。我们之
所以从20世纪早期进行创作的美国学者中挑选凡勃仑,是因为他对后来非正统思想产生的公
认影响;挑选韦斯利·克菜尔·米切尔(Wesley Clair Mitchell,1874--1948),是因为他在收
集并分析与经济波动相关的数据方面的开拓性工作;挑选约翰·R·康芒斯(John
R. Commons,1862--1945),是因为他对现在的社会理论与立法所产生的影响。最后,我们挑
选了英国人霍布斯,作为非美国的非正统经济学家的代表,因为他对当代英国社会政策的态
度具有深远影响。

非正统理论与正统理论早期的意见\textbf{不同}主要体现在两个方面:第一,与正统理论
的\textbf{范围和方法}意见不同,以及与其理论核心中的\textbf{其他因素}意见不同;第
二,与正统理论最重要的下列观点意见不同,即\textbf{市场制度普遍导致经济力量的和谐
  结果,因此自由放任是政府应遵循的最住政策}。

\section{方法上的争论}

即使在门格尔、杰文斯、瓦尔拉斯以及马歇尔开始将边际分析应用于价值与分配理论之前,
正统古典理论就受到了某些非社会主义德国经济学家的批评。虽然这些经济学家的观点之间
有一些显著的差别,但是,它们有充分的共同点来集体被称作德国历史学派。这一学派的影
响在德国开始于19世纪40年代期间,并延伸到20世纪。很多历史学家注意到了早期的经济学
家与稍后的经济学家在看法上的差异——很大程度上来自于\textbf{德国的不断变化,以及对
  正统理论的回应——而将其划分为旧历史学派与新历史学派。}

19世纪70年代,英国也出现了独立于德国历史学派的对正统经典理论的批评,以及对所谓
的历史方法的提倡。然而,历史方法的这些英国提倡者,没有形成聚合的群体,因此,英国
历史学派这样的说法是不适宜的。因为德国与英国的这些强调历史的经济学家对某些新古典
经济学家,特别是阿尔弗雷德·马歇尔产生了巨大影响,所以值得我们去关注。鉴于在德国接
受研究生教育的美国经济学家为数不少,因此德国学者也影响了美国的经济理论与政策。

\subsection{旧历史学派}

\textbf{旧历史学派}中的重要经济学家有弗里德里希·李斯特
(FriedrichList,1789--1846)、威廉·罗雪尔(William Roscher,1817--1894)、布鲁诺·希
尔德布兰德(Bruno Hildebrand,1812--1878)以及卡尔·克尼斯(Kanl Knies,1821--1898)。
他们主张,不能将古典经济理论应用于所有的时期与文化中,尽管斯密、李嘉图以及约翰·斯
图亚特·穆勒的结论对于像英国这样工业化中的经济体来说是正确的,然而,并\textbf{不能
  应用于农业化的德国}。这些经济学家的经济分析中包含大量的民族主义情感。此外他们断
言,经济学与社会科学必须使用一种以\textbf{历史为依据的方法},在李嘉图及其追随者的
控制下,古典理论在试图模仿自然科学方法上是错误的。学派中一些比较中立的成员承认,
理论演绎法与历史演绎法是一致的;但是一些人,特别是克尼斯反对抽象理论的任何一种应
用。

\textbf{李斯特}尤其表达了强烈的民族主义观点,他拒绝接受如下看法,即古典理论关于自
由放任的结论适用于那些不如英国发达的国家。古典理论主张,国家的福利来自于个人在自
由放任的环境中对私利的追求,而李斯特则认为,\textbf{国家的指导是必要的,尤其是对
  于德国和美国来说。他主张,考虑到其工业的先进状况,自由贸易有益于英国,而对德国
  与美国而言,关税与保护则是必要的。}从1825年至1830年李斯特在美国花了五年时间,以
及后来在德国花了大约十年时间出版了《政治经济学的国民体系》一书,该书汲取了他在美
国的经验。其保护主义观点在美国得到热烈的认同,以至于他通常被称作\textbf{美国保护
  主义之父}。

这些经济学家所提倡的历史方法是什么呢?他们的工作反映了如下信仰,即经济学的首要任
务是发现支配经济增长与发展阶段的规律。例如,李斯特声称,\textbf{处于温带的经济体
  将经历五个阶段:游牧生活;畜牧生活;农业;农业与工业;农工商业。}希尔德布兰德断
言,理解经济增长阶段的要记是在\textbf{交换条件}中找到的;因此,他设置了基于物物交
换、货币以及信用的三个经济阶段。对增长阶段的这些描述,显然包含了一定量的理论并且
高度抽象。然而,这些经济学家的确收集了大量历史与统计资料来广持他们对于经济发展的
分析。在更近的时期,\textbf{沃尔特·惠特曼·罗斯托(Walt Whitman Rostow,1916--)提出
  了一种遵循旧历史学派传统的经济发展阶段理论},正如可以预料到的那样,与经济学家们
的认同相比,他的著作获得了其他社会科学中学者们更好的认同。

\subsection{新历史学派}

\textbf{第二代德国历史学派有一位杰出的领导者古斯塔夫·冯·施穆勒。}像旧历史学派的成
员一样,新历史学派的经济学家们抨击古典经济理论,尤其是古典理论适用于所有时间与地
点的观点。在历史方法的应用上,他们一般不如旧学派那样雄心勃勃,\textbf{他们愿意创
  作关于经济与社会不同方面的专论,而不是阐述重大的经济发展阶段理论。在努力创作的
  过程中,他们更喜欢运用归纳方法,似乎认为收集到足够的经验证据之后,理论就可能出
  现。他们也对借助国家行为的社会改革非常感兴趣。因为这一点,他们被称为“讲坛社会
  主义者”,}这是他们乐于接受的一个称号,他们认为,不接受诸如所得税一类建议的批评
家是反对进步的人。

门格尔、杰文斯以及瓦尔拉斯在19世纪70年代早期应用边际分析,并构建抽象演绎模型,这
在德国只有很少的影响或者说没有影响。尽管门格尔这位奥地利人用德语创作了他的《经济
学原理》,但是,在\textbf{德国的大学}中并没有被加以研究,因为这些学校\textbf{排外
  性地只赞同历史方法。}在其早期著作中,施穆勒乐于承认两种方法在经济研究中都占有一
席之地,虽然他并不推荐构建抽象的理论模型、1883年,门格尔出版了一本关于方法论的著
作《社会科学尤其是政治经济学方法的研究》,它开启了一场一直持续到20世纪的长久的、
沉闷的、无果而终的争论。关于方法的这场争论,是经济理论发展中曾经发生的最为激烈的
方法上的争论之一;只有美国制度主义者与正统理论家之间最近的争论能与之相比。门格尔
的著作中包含了对经济学与社会科学中方法问题的一般性概述,然而,他也抨击了历史方法
的错误。施穆勒回应了这一抨击,于是战争开始了。门格尔发表了对施称勒回应的驳斥,其
他人也参与到争论中来。双方都摆出攻击的架势,都认为自己的方法几乎是唯一可以使用
的。\textbf{正像施穆勒指出的那样,双方都用表示敬意的术语将自己的方法表述成经验的、
  现实的、现代的、精确的方法,同时将对方的方法称作是投机的、无用的、次要的。}

从某种观点看,这场争论可以看做是\textbf{经济文献的纯粹死胡同},是经济学作为一门学
科加以发展的有害物,因为有才能和心智的人把他们的时间都耗费在没有意义的争论上了。
另一方面也有可能是,这场争论帮助经济学家认识到,\textbf{在他们的学科中理论与历史、
  演绎与归纳、抽象模型构建与统计数据收集并不是相互排斥的。}

尽管个别经济学家可能倾向于将其主要精力专门投入到这些方法中的一种上,然而,一个健
康发展的学科要求方法的多样性。因为没有哪一种方法能够完全排斥另一种方法而被加以认
同,所以,实际问题在于给予每种方法应有的重视。我们的观点是,学科的内在发展将决定
这个问题,所以,为此进行争论是没有意义的。

从这场争论中能够得到另外一个教训。如果某种特定方法的创立者,对方法的正确性变得如
此确信,以至于不允许其他观点在从事研究和研究生培养的大学中表现出来那么,经济学的
发展将会受到损害。这一点发生在德国,自以为是且刚愎自用的智力领袖施穆勒,在德国极
具影响力,以至于遵循门格尔、杰文斯、瓦尔拉斯以及马歇尔所创立路线的抽象理论家们,
无法在德国找到大学职位。结果,经济思想的主流忽视了德国经济学家,经济学作为一门知
识学科在德国遭受了几十年的损害。

\subsection{美国的历史方法}

19世纪最后二十五年期间,许多英国经济学家批评正统古典理论,并提倡运用历史方法进行
经济学研究。这些经济学家与德国经济学家不同,他们未能形成一个聚合的群体,也没有受
到德国经济学家的直接影响。在经济思想中,对于英国的传统来说\textbf{历史归纳方法}并
不陌生。亚当·斯密的《国富论》是历史材料与描述性材料的混合,再配合一个松散的理论结
构。李嘉图代表了经济学方法向抽象演绎模型构建的一次主要转变,模型几乎完全没有历史
或制度的内容。西尼尔拥护并扩展了李嘉图对演绎推理的运用。然而,约翰·斯图亚特·穆勒
和阿尔弗雷德·马歇尔回归斯密方法,利用他们对历史材料和制度材料的渊博学问与知识,赋
予其理论结构以实质内容。

\textbf{历史方法在英国的首要倡导者是T. E·克利夫·莱斯利,他对古典经济学的方法(主要
  是李嘉图及其追随者的方法)进行了批评。}莱斯利主张,斯密的经济理论不适用于现代英
国的情况,但是总的来说,斯密的方法还是相当合理的,因为斯密广泛运用历史材料来得出
结论。尽管\textbf{阿诺德·汤因比}(Amold Toynbee,1852--1883)英年早逝,使其成为一名
经济史学家的伟大心愿未能得到完全实现,然而,他的《18世纪英国工业革命讲稿》则是运
用\textbf{历史方法}来了解发生在英国的根本变革以及因此出现的工业化经济体问题的一个
壮观例子。正是阿诺德·汤因比\textbf{创造了工业革命(Industrial Revolution)这一术
  语。}威廉·阿什雷(William Ashley,1860--1927)与威廉·坎宁安(William
Gunningham,1849--1919)关于英国经济史的著作,仍旧受到高度的尊重。其他经济学家运用
历史方法来分析具体的主题,沃尔特·贝格豪特创作了《朗伯德街》,它是一部研究英国银行
业的经典;约翰·K·英格拉姆(John K. Ingram,1823--1907)出版了《政治经济学史》,它是
用英语创作的关于经济理论史的第一部系统性著作。

尽管历史学派没有对理论的新发展产生重要影响,然而,它的经验一直都是有用的,并且影
响到了很多经济理论批评家,我们将在本章的以下部分以及第17章中对他们进行夯凤。

\section{托尔斯坦·凡勃仑}

\textbf{托尔斯坦·邦德·凡勃仑是通常被称作制度主义(institutionalism)的美国非正统分
  支的学术创始人。}他与正统理论在科学上和道德上的不同意见;极大地影响了美国非正统
思想的发展。凡勃仑的观点,部分地可以通过其背景得到解释。他是挪威移民的儿子,在美
国威斯康星州和明尼苏达州的农村长大。当他进入卡尔顿学院时,他对英语的掌握就像他对
美国社会的了解一样不充分,他从来没有完全融入到美国主流社会。他就像一个从火星上来
的人一样,以其讽刺智慧评述着经济与社会秩序的荒谬。在卡尔顿,他的才华得到
了\textbf{约翰·贝艾·克拉克}的认同,后者当时对边际分析做出了开创性的贡献。在克拉克
的鼓励下,凡勃仑去了东部的研究生院。他在耶鲁获得了哲学博士学位,但是未能得到一份
从教的工作,这显然是由于他的无神论观点。所以,凡勃仑回到农场,并与他的大学情人结
婚,花了七年的时间继续读书思考。

三十五岁时,他获得了\textbf{康奈尔}的博士后奖学金。在仍然末能找到一份学术工作后,
他又接受了\textbf{芝加哥大学}的奖学金,在那里他终于得到了一份经济学讲师的工作,并
被给予《政治经济学》杂志编辑的职位。他从未受到大学管理者的欢迎,从来没有获得正教
授的级别,他将其生命的剩余时间,花在从一所大学迁往另一所大学上。无法确定的是,他
未能获得与其学识相称的专业认可,是因为他对美国资本主义进行敏锐批评,并且除了最好
的学生之外几乎完全漠视其他所有人的结果,还是由于他的个人生活造成的,他的个人生活
因风流韵事和婚姻上的难题而复杂难解。然而,在20世纪20年代中期,经过几年政治上的上
暗斗之后,美国经济学会向凡勃仑提供会长一职,任职条件是他要加入协会并发表一篇演说。
凡勃仑拒绝了这一提议,声称当他需要的时候这一荣誉并没有出现。

凡勃仑在偏僻土地上接受的教养、他的哲学训练、他在社会科学方面的广泛阅读,以及他对
达尔文进化论重要性的深入评价,都反映在他对美国资本主义的分析中。他的文体和对文字
的选择,赋予其作品一种品质,

一些经济学家发现这种品质非常令人愉快,另一些经济学家则谴责这种品质。他是一个擅长
创造警句的人,喜欢通过使用像\textbf{炫耀性消费}(conspicuous consumption)一类的术
语来描述新兴富足社会的购买方式,从而使其读者不舒服。他认为,我们要么是\textbf{受
  控制的阶级成员},要么是\textbf{基本人口成员};\textbf{大学校长是学识首领,生意
  人的主要工作是实行破坏;工业无度地多产,要获利就要切实地消除效率。凡勃仑将教会
  描述为“一种公认的发泄,以使堕落的事物从文化机体中流出”。}米切尔提出,人们需要
有一种幽默感来欣赏凡勃仑,也许这就是他为什么很少得到经济学家赏识的原因。

\subsection{凡勃仑对新古典理论的批判}

\textbf{凡勃仑创造了新古典这一术语},用以强调这种类型经济理论的古典世系。他认为,
古典与新古典方法两者都是不科学的。他对新古典理论的批评包含在他的全部著作中,虽然
他的一部论文集《科学在现代文明中的地位》包括了他大部分明确的方法论著作。他受到的
哲学训练,部分地解释了他对其所处时代公认的经济学所进行的抨击的性质。凡勃仑对理论
结构的微小变动——例如,纠正体系中次要的逻辑缺陷不感兴趣。他攻击新古典理论的核心,
声称其学说的基本假设是不科学的。对一种理论结构基本原理的攻击,使得接受这一结构培
养的人面临两种选择:他们也许接受批评,并依据改变了的假设构建一种新理论,或者驳回
批评。而对理论的批评如果接受了该理论结构的基本假设,但提供了新的更加符合逻辑的或
者更加符合经验的正确结论,那么,这种批评同样可能被接受或者被驳回,但是,在这种情
形下,对于那些已经进行了学科训练的人来说,接受这种批评痛苦就会稍小些,因为这不要
求对其训练和定位作深度的重新调整。凡勃仑明白,李嘉图、马歇尔还有凯恩斯,虽然接受
由斯密构建的理论结构假设,却没有试图改善古典经济学的理论结构;他希望拆卸整个结构,
重建一个由\textbf{经济学、人类学、社会学、心理学、历史学组成的统一的社会科学。}值
得注意的是,凡勃仑依据他批评正统理论所依据的相同的理由,批评了以前的非正统思想,
认为历史学派是不完善的,因为它的基本假设与预见是不科学的。

凡勃仑的观点是,尽管正统经济理论的术语从斯密时代起一直发生着变化,但是,其基本
假设与预见保持不变。在斯密之前,对经济与社会的大多数分析基于下列预见,即社会受到
超自然力量的支配以获得合意的结果。后来,对超自然力量或者上帝的求助,被下列观点所
取代,即自然法则存在于经济和社会中,就像存在于自然科学中一样,适当的调查与研究将
揭示这些自然法则的运转。

凡勃仑说,从斯密到马歇尔,正统经济理论的全部都基于相同的假设:制度中存在和谐,或
者说凡勃仑所谓的“改良性趋势"。这一点出现在斯密自然价格的概念中,以及将私人恶行转
变为公共利益的“看不见的手”的运转中。马歇尔理论中正常价格与均衡的概念,以及长期
均衡下的完全竞争市场产生有益结果的预期,都反映了这一信仰。约翰·贝茨·克拉克的有关
长期竞争性均衡产生一种公平的收入分配的结论,是关于经济体和谐假定一个特别显著的例
子。对凡勃仑来说,\textbf{正统理论家所运用的均衡概念是标准化的:他们毫无证据地暗
  示均衡是优秀的,均衡状态下的市场所产生的结果在全社会中是有益的。}

我们从不同的视角来考察这一点,并使之与凡勃仑对正统理论的另一个批评相结合。凡勃仑
借鉴了哲学与生物学中的概念,他断定\textbf{正统理论是目的论的(teleological),因
  而是前达尔文的。}正统理论之所以是目的论的,是因为它将经济体描述为朝向一个目
标——\textbf{长期均衡}——运动,而这一目标在经验上是达不到的,却在分析开始之前就设定
了。\textbf{正统理论之所以是前达尔文的,是因为正如凡勃仑对达尔文的解释一样,进化
  是一个纯机械过程,借助该过程生物体适应环境情况,随着时间的变化而得到发展。进化
  中没有目的或者设计。}

古典思想拒绝承认经济体是不断变革与演进的,而是集中研究经济体的静态方面,因此,古
典思想也是前达尔文的。\textbf{凡勃仑主张,这种静态的前达尔文的经济理论,应当被一
  种动态的关于经济与社会进化的达尔文分析所取代。}凡勃仑用生物学的术语指出了同一点,
并谴责古典理论是\textbf{分类学}的(taxonomic),因此也是不科学的前达尔文的。之所
以说它是分类学的,是因为它对经济体及其组成部分进行了分类,但是,\textbf{没有就它
  们作为一套演进的变革的制度进行解释并形成概念。}正统经济学聚焦价格理论时,假定很
多东西是既定的或者不变的(例如,品位或者消费者偏好、技术、社会与经济的组织安排,
等等)。凡勃仑指出,经济学家不仅应当研究价格的形成与资源的配置,而且应当研究他们人
为使之保持不变的那些因素。他对马歇尔试图与静态分析决裂表达了一些溢美之词,却断言
马歇尔在这方面的努力是不成功的。

凡勃仑指出,经济学是不科学的,这一提法的一个原因是,\textbf{亚当·斯密“看不见的
  手”的概念从来没有被证明。因此,经济学建立在一个从未予以考察的假设之上:赚钱就
  等同于生产产品。}按照正统理论,生意人在追求利润的过程中,将以最低可能的成本生产
那些消费者需要的产品。竞争性市场促使生意人的私利符合社会利益。追求他或她自身私利
的每个个别生意人,推进了社会利益。凡勃仑认为,除了经济学家之外,对所有人来说,下
列结论都是显然的,即\textbf{生产产品与获利是两种不同的事情,企业界为利润而奋斗,
  这对经济与社会经常会产生有害的效应,追求他或她自身私利的每个个别生意人,将只推
  进他或她自身的私利。}有人提出,凡勃仑对经济与社会的这种观念,在他年轻的时候,当
他离开其位于明尼苏达州路德教会家庭的边境农场,前往卡尔顿学院时就已形成了。在卡尔
顿学院上学的,主要是来自新英格兰、具有公理会背景、很会赚钱的人家的孩子。19世纪最
后二十五年中,大公司规模与实力的增强,以及托拉斯的形成,也影响了凡勃仑。此
外,\textbf{耕种土地的平民主义者对工商业——谷物升隆机、铁路、农用设备制造商,以及
  银行——的敌对,必定使他的家庭深受影响。}

凡勃仑主张,亚当·斯密时代,赚钱与生产对社会有用的产品两者之间,存在一种相当紧密的
联系。但是,随着经济的发展,这一点发生了变化。凡勃仑严格区分了涉及生产产品的
人——生产经理、监督者以及工人——和涉及企业管理的人。工商业的目标是金钱收益,凡勃仑
直指一般利益受到逐利侵害的例子,并为自己的做法感到高兴。他的观点是,\textbf{利润
  增加是产量减少的结果,这对社会显然是有害的。凡勃仑时代正在形成的大公司的目标,
  不是提高效率,而是获得垄断实力并限制生产。他指向厂商的广告行为,质疑它们对整个
  社会的有用性。厂商之间的国际市场竞争导致冲突,最终引发战争。大企业首脑们的金钱
  活动,将不可避免地导致经济萧条与大量失业。}本质上说,凡勃仑否定关于完全竞争市场
的正统假设,否定生意人控制下的市场将会产生对社会合意的结果的观点。正统理论凭借有
效的资源配置与充分就业,发现了资本主义下的和谐发展;而凡勃仑凭借生意人为了获利而
破坏制度,并最终导致经济萧条,发现了资本主义下的不和谐发展。

在凡勃仑看来,正统理论的预见,反映出经济学未能与自然科学和生物科学的发展保持并列。
正统经济理论无视心理学、社会学以及人类学的发展,基于不科学的人类本性与行为概念来
构建模型,在这些方面应受到责备。按照凡勃仑的观点,正统理论基于下列假设,即人类在
享乐主义心理的基础上,受到快乐最大、痛苦最小愿望的驱使。给定这一假设,经济学家正
确地推论出了其逻辑结果。逻辑是没有缺点的,但是,假设是错误的。凡勃仑主张,正统经
济学是对人的研究,却\textbf{将人从分析中抽象掉了}。在他最尖锐的一些散文中,他嘲笑
公认的消费者行为理论:

\begin{quotation}
  经济学家们所持有的心理学和人类学偏见是早在几代人之前就被心理学和社会科学所接受
  的观念。享乐主义关于个体的概念是\textbf{将人视作闪电般地计算快乐与痛苦的计算器},
  他象一个追求快乐的摇摆的同质的小球,外界的刺激使他移动,但不会使他有所改变。
  他\textbf{既没有前因又无后果}。他是一个孤立的、确定的人类已知数,除了冲击力使其
  在不同方向移动外,\textbf{他始终处于稳定均衡态。}他在空间上自我驱动,绕着自己的
  灵魂轴心对称地旋转,直到外界力量强加于他,使他不得不屈从。当这些作用消失之后,
  他又成了一个和以前一样\textbf{静止的、不易冲动的欲望小球。}本质上,享乐主义的个
  人不是一种原动力,除非在受制于外在的和异已的环境力量强加给他的一系列变化这种意
  义上,他并不位于生活过程之中。(《经济学为什么不是一门演化(进化)科学》,本书
  中文版此段翻译极烂,以上内容我转
  自\url{http://www.aisixiang.com/data/32005.html}。
\end{quotation}

凡勃仑对正统理论的最后一项批评,与其他批评相比阐述得不是很明确,即正统理论未能使
经济体的理论与经济体的实际相一致。

\subsection{凡勃仑对资本主义的分析}

凡勃仑强调,经济学的主旨应当完全不同于盛行的经济理论的主旨。凡勃仑时代的正统理论,
主要对社会如何在多种可替代用途之间分配其稀缺资源感兴趣。凡过仑主张,经济学应当是
对演进的制度结构的一项研究,并将制度界定为在任何特定时期都被接受的思想习惯。在对
经济学主旨的这一界定中,凡勃仑试图解释塑造社会与经济的力量。一种文化的特定制度,
被正统经济理论假定为既定,凡勃仑却试图加以解释。他主张,对盛行文化的解释,要求一
种\textbf{演进的方法,因为任何文化只有通过其先辈才能被理解。}

\textbf{文化的成长是一种累积顺序的习惯,其方式与手段是人类本性对紧急事件的习惯反
  应,这些紧急事件无节制地且累积地变化,却在累积变化中带有几分一致性,}这些累积变
化是这样发生的——\textbf{无节制地},这是因为每个新的运动都产生一种新的状况,引起习
惯的反应方式更新的变化;\textbf{累积地},这是因为每种新的状况,都是它前面状况的一
种变化,它体现的\textbf{因果关系}因素已经到了为前面的状况所预料的程度;\textbf{一
  致地},这是因为\textbf{反应迫于人类本性(倾向,自然倾向,诸如此类)的根本特征而
  发生——人类本性充分保持不变。}

为了理解工业社会的发展以及现在的作用,我们必须了解存在于人类本性特征与文化之间复
杂的相互关联制度。
\begin{quotation}
  不仅个人的行为受到他与群体中同伴习惯性关系的限制和引导,而且这些关系具有一
  种\textbf{制度特征},随着制度安排的变化而变化。个人行为的需要和欲望、目标与目的、
  方式与手段、幅度与动向,具有\textbf{非常复杂又完全不稳定特征的制度变量的函数}。
\end{quotation}

当个体出现在文化中时,他们发现自身依照已经确立的行为模式来行动,这种行为模式是个
体与文化之间过去相互作用的产物,并且具有制度的特征与力量。凡勃仑将这些相对不变的
人类行为的根本特征称作\textbf{本能}(instinctis)。他深受心理学当代发展的影响,即
强调本能在指导人类行为中的作用。凡勃仑认为,塑造人类经济活动最重要的本能
有\textbf{亲体本能、技艺、闲散的好奇心以及获得}。亲体本能最初就是对家庭、部族、阶
层、国家以及人类的关注。技艺的本能使我们希望生产高质量的产品,为技艺而自豪并赞美
它,并关心工作中的效率和经济性。闲散的好奇心引导我们提问,并寻求对周围世界的解释,
它在解释科学知识的发展中是一项重要因素。获得本能与亲体本能相对,因为它使个体关心
自身的福利甚于关心其他人的福利。

\subsection{二分法}

人类本能的驱使产生了某些紧张状态。亲体、技艺、闲散好奇心的本能,将导致人类以极大
的效率生产\textbf{高质量的、有益于同类人的产品}。然而,由于获得本能是\textbf{利己
  主义}的,所以它将导致有益于个体的行为,尽管它可能对社会上的其他人产生\textbf{有
  害}的结果。凡勃仑说,对经济体的分析,揭示了这种根本的紧张与对抗,它在人类本性中
是基本的。每种文化都能通过观察人类行为的两个方面而得到分析:一个是推动经济生活过
程的方面,另一个是抑制社会生产能力充分发展,对人类福利具有负面效应的方面。

凡勃仑将主要从亲体、技艺、闲散好奇心本能中产生的活动称为\textbf{生产性
  的}(industrial)或\textbf{技术性的}(technological)职业。它们包括事务性的、因
果性的关系。他在推测的历史中进行研究——尽管他激烈地批评了正统理论的这一行为——并解
释说在遥远的过去,人类通过\textbf{使用符咒祈求超自然力量}发生作用,通过绕着茎秆跳
舞,诉诸超自然的力量种植谷物,并试图以此来解释未知的东西。凡勃仑将这种非制度的、
非技术的、近代科学之前的接近未知并寻求解释或效果的方式,称为\textbf{礼仪性行
  为}(ceremonial behavior)。\textbf{礼仪性行为是静态的,并与过去相黏合。它用图
  腾与禁忌、用对权威与情感的诉求来表明自身,对人类福利来说,它产生了不受欢迎的结
  果。}然而,生产性或技术性职业是\textbf{动态}的,并且,我们越是运用科学的、事务
性的观点接近问题的解决,我们的工具、技术以及解决问题的能力就越强。技术并不倒退,
但是,礼仪性行为则根植于\textbf{过去}。

凡勃仑对他所处时代文化与经济体的分析,建立在这种二分法的基础上。他的几乎全部论文
和书籍,都在反复地阐明这一主题。他认为,这一框架及其运用不包括规范性的判断,而是
构成了对发展以及文化与社会结构的事务性实证分析。在他的散文《生产性职业和金钱性职
业》,以及可能是他的一部最好的经济分析著作《企业论》(1904)中,能够最为清楚地看
到二分法的纯经济运用。\textbf{现代文明中的礼仪性行为},在凡勃仑所谓的\textbf{金钱
  性}(pecuniary)或\textbf{生意性}(business)职业中最能体现出来。在工业经济出现
之前的手工业时期,工匠拥有自己的工具和材料,用他自己的劳动,生产能够表达其技艺本
能和亲体本能的商品。从这些活动中获得的收入,是对所付出努力的合理度量。随着经济体
的发展,很多东西发生了变化。工人\textbf{不再拥有生产工具或者材料},企业所有者现在
对\textbf{赚钱}比对\textbf{生产产品}更感兴趣;\textbf{获得本能}比\textbf{技艺本
  能}和\textbf{亲体本能}更重要。\textbf{放款}发展起来了,\textbf{所有权缺
  席}(absente eownership)变得更平常了,个人现在拥有“依时效而取得的权利,
以\textbf{免费}得到某种东西"。大企业的首领出现了,跟着是一段激烈竞争的时期。大企
业的首领很快意识到竞争是不合意的,所以,借助投资银行家的手段,形成了\textbf{控股
  公司、托拉斯以及连锁董事会},也形成了既定利益的\textbf{全世界大工会和缺席的所有
  者}。对于工人和工程师以及大企业的首领和缺席的所有者来说,所有这些发展都导致了不
同的思想习惯。基于生产性职业的日常基础——产品生产,工人和工程师被包括进来。这促使
他们在因果方面进行思考,表达出他们的技艺本能和亲体本能。但是,大企业的首领和缺席
所有者只关注利润,凡勃仑的观点是,\textbf{获利与生产产品经常发生冲突}。

推动凡勃仑对他所处时代工业社会进行分析的主要力量是,他认为正统理论在下列主张上是
错误的,即生意人引导下的经济体将促进社会利僵。他无情地指向工商业导致
的“\textbf{道境}"。拥有垄断实力的厂商,为了获得更大的利润而实行“\textbf{经过考
  虑的赋闲}”。产量的这种减少,提高了利润,导致了一种“无效率的股本”。“生产为了
生意上的缘故而继续着,而不是相反。”大量活动被误导,生产对人类没有用的产品,并加
以营销和广告。生意人并不是社会的恩人,而是\textbf{社会的破坏者}。

\subsection{有闲阶级}

礼仪性--生产性二分法也适用于凡勃仑所训的\textbf{有闲阶级}(leisure
class)。1899年,凡勃仑出版了被证明是其最被广泛阅读的书籍《有闲阶级论》;这是在他
所处时代很多知识分子特别喜欢的一本书。他在书中运用其基本的二分法来讨论炫耀性消费、
炫耀性闲暇、炫耀性浪费、金钱竞赛以及表达金钱文化的着装。凡勃仑推论,在不发达的文
化中,一个人或一个部族的捕食能力受到高度的尊重,具有这种能力的人,被给予受人尊敬
的地位。在现代工业经济体中,这些捕食能力通过为社会少数成员带来高收入就业表现出来。
然而,如果高收入不能得到认可,就将毫无价值,所以,我们的文化提供了很多允许它们得
以展示的机会。因为攀比是一种有力的动机,所以,这些财富展示活动会很快遍及社会。

对我们所购商品的炫耀性消费,是\textbf{展示我们捕食能力}的一种最有效手段。我们的汽
车、住房,尤其是服装,清楚地表明了我们在捕食次序中的地位。如果家庭中的男性过分忙
于从事其捕食活动,那么,他的妻子就被指望承担\textbf{展示家庭财富}的重任。她通过穿
着和展示其他商品,以及小心避免任何类别的工作来完成这一重任——所雇佣的仆人数量是经
济能力的一项可靠指标。此外,因为有闲阶级是高收入阶级,所以,所从事的工作应当在严
格的金钱性职业中;缺席所有权受欢迎,但是如果必须做一些实际工作,那么,\textbf{高
  级管理、金融以及银行}都是可以接受的礼仪性行为。\textbf{法律}是一项很好的职业,
因为“律师的工作充满了有关\textbf{捕食诡计}的细节。”凡勃仑说,闲暇活动也反映了在
文化中获得受人尊敬的地位这一愿望。高等教育使一个人不适合从事正当工作,却具有重要
价值。有闲阶级也培养对\textbf{体育活动}的极大兴趣,并以它们促进了身体健康和男子汉
气概为由,使之合理化。凡勃仑评论说:“有种说法,即\textbf{足球和体育}的关系
与\textbf{斗牛和农业}的关系极为相同,并非不恰当。”

凡勃仑认为,与技术性职业相关联的个人,例如,发明家与工程师,都是胆大而足智多谋的,
美国的生意人展示了一种寂静主义精神——“折衷、谨慎、共谋以及诈骗”。但是,生意人以
非劳动所得收获了技术社会的利益。他曾提到:“有一种家常的但被充分认同的美国俗
话——“寂静的猪吃猪食。”凡勃仑的观点是,学术与科学训练使一个人不适合工商业,工商
业的经历与研究学问是不相容的。

从董事会变动到学术行政,凡勃仑将大学校长称作是“\textbf{学识首领}”。他说,尽管通
常他们都是以前的学者,但是,他们被卷入社会的金钱价值中,误导了大学的努力;像工商
业的首领一样,他们在\textbf{手段与目的}之间变得困惑了。大学之间相互竞
争,\textbf{资源的浪费}比得上工商业竞争造成的浪费;相对于教育项目和政策,校长与董
事会对建筑物、场所、房地产更感兴趣;资源浪费在既对大学没有价值,也对社会毫无用处
的体育运动、法律与商学院、典礼以及盛会上。凡勃仑没有宽恕“\textbf{教授会}",那些
教授认为“他们的薪水不具有工资的性质”,他们没有集体谈判的权利,他们立志成
为“\textbf{乡绅}"。为了控制全体教员,校长任命院长和其他具有“现成的信仰多样性,
并坚定忠诚于其生计”的人。凡勃仑所推荐的使大学回到研究学问上来的主要行动计划是取
消校长与董事会。很难断定凡勃仑对于这一讽刺性的建议有多勾认真,但是,至少他认识到
这是很不可能发生的。

\clearpage

\subsection{资本主义的稳定性与长期趋势}

凡勃仑将他对金钱性职业与生产性职业的区别用于发展经济同期理论,思考资本主义在相当
长时期中的趋势。在周期的繁荣阶段,生意人的金钱活动导致信用扩张,公司获取利润的无
形能力被赋予了较高的价值。增加的资本价值用于附加的新增信用。这一过程暂时得到自我
增强,因为信用的数量和资本产品的附加价值,继续随着资本产品价格的上升而扩张。但是,
在资本产品的获利能力和其以证券行市体现的价值之间,存在着较大的差距,这一点很快就
变得明显了,清偿与紧缩阶段开始了。

下降的价格、产量以及就业,还有减少的信用,导致厂商在更加现实的基础上改变资本结构。
在周期的萧条阶段期间,较弱的厂商被排挤出去,或者被大而强的厂商兼并,使美国行业的
所有权与控制权和集中在较少人手中。周期的萧条阶段包含了自我纠正的力量,因为实际工
资下降,利润边际提高。最后,超额信用被从经济中挤出来,资产负债表中表示的工商业的
金钱价值,反映了对行业产量更加合理的评估。

尽管凡勃仑所有的著作都推测了制度的长期趋势,但是,他在《有闲阶级论》、《企业论》
以及名为“社会主义理论中一些被忽视的观点”的短文中,最为明确地涉及了这些问题。

凡勃仑根据金钱性职业与生产性职业碰撞所产生的冲突与紧张,对未来进行推测。他在《有
闲阶级论》中指出,竞赛、奉承以及产品消费中招人反感的攀比,将导致一个专心于炫耀性
消费、炫耀性浪费以及广告和营销成本增加的经济体。只要生产受到追求利润的生意人的控
制,我们就能预期到,阻止人类进步的产品流将增加。但是,生产的职业造成了事务性的因
果关系,如果工人与工程师凭借他们与这些关系的日常联系,掌握对制度的控制,那么,工
业化的经济体就可能实现其承诺。

凡勃仑认为,资本主义条件下所形成的\textbf{消费竞赛模式},力量是如此强大,以至于它
可能在制度中产生紧张和引起工人阶级的不满,并导致私人财产的终结。任何数量的个人实
际绝对收入增加,都不能减缓这些紧张,因为个人都希望比其他人得到更多,而不是多一点
点:
\begin{quotation}
  作为人类的本性,\textbf{每个人都在奋斗,以拥有比其邻居更多的东西,}这与私人财产
  制度是不可分的。……推论似乎是……不可能有和平——这一点必须得到承认——从这种不光
  彩的竞赛形式来说,或者从与之相伴的不满即废除私人财产的一方来说。
\end{quotation}

\textbf{资本主义可能因为个人对其自身相对福利的关注而终结},这一主张是凡勃仑的分析
中荒谬特性的又一个例子。凡勃仑提出,\textbf{资本主义不是由于其失败而是由于其成功
  才终止的}。

然而,凡勃仑拒绝彻底表明自己的意见,并提出这一切实际上不可能发生。资本主义的未来
和私人财产是不确定的。一种可能性是,工人阶级和工程师中间产生的科技态度日益增强,
导致对生意人的替代,因此,对经济体的控制将传到技术统治论者手中。凡勃仑说,如果有
了这些发展,它将意味着缺席所有权、金融操纵以及利润追求的终结,行业将被加以引导
而生产对人类有用的产品。

也有可能发生一场真正的社会主义革命,结束阶级差异、王朝政治以及国际仇恨。还有另一
种可能是,当工人阶级和工程师遵从民族抱负、好战目标以及民主主义时,随之而起的是一
场右倾的经济与政治运动,平息下来后进入集权国家。由于深深地根植于达尔文的进化论,
凡勃仑没有确定地预测未来。凡勃仑关于未来的观点中,唯一的必然性是变化。脆弱的制度
是否将击败事务性的技术,还有待观察:

\begin{quotation}
  两种敌对的因素中,哪一种能证明在长期中更加强大,这有几分盲目猜测;但是,可预测
  的未来,似乎属于其中的一个或另一个。看上去很有可能这样说,对企业的完全支配必然
  是一种短时间的支配。
\end{quotation}

\subsection{凡勃仑的贡献}

一般而言,非正统经济理论,尤其是凡勃仑的理论,经常从关于经济思想史的著作中被忽略
掉,这或许是因为,它们对于现代正统经济理论只有非常小的直接作用。凡勃仑对正统经济
理论非常不满。正统经济理论在阿尔弗雷德·马歇尔的经济学中获得了最成熟的表述。凡勃仑
希望抛弃这一体系,因为他认为,这一体系坚持着错误的方法。他声称\textbf{正统理论在
  方法上是原子论的},试图从对经济体组成部分即家庭与厂商的初始分析开始,随之在整体
上了解经济体。但是,\textbf{整体与部分的总和并不相同};凡勃仑认为,一种适当的方法
应当始于文化、社会以及经济层面。

有人说凡勃仑根本不是一个经济学家,而是一个社会学家,并且在一些经济学家看来,是一
个思维不清晰的社会科学家。不把凡勃仑当成经济学家的看法,至少与他的方法和贡献是一
致的。凡勃仑的理论之一恰好是我们不能通过使人类的经济行为与其他活动相孤立的方法,
来了解所谓的经济体。因此,凡勃仑实际上提出了\textbf{社会科学的融合}。

凡过仑对正统理论家们感兴趣的一系列问题并不感兴趣。他希望了解制度结构的发展,这种
制度结构是通过指导经济活动的思想习惯形成的。从这个角度来看,凡勃仑的贡献能够被看
做是对正统理论的补充。然而,凡勃仑认为,一旦了解了变化的制度结构,解决正统理论所
研究的比较有限而狭窄的问题,就要求一套不同于经济学家现在所使用的假设和工具。他坚
持认为,经济学必须使用一种演进的方法,抛弃有关自然均衡或正常均衡的目的论概念;必
须与其他社会科学相融合;必须抛弃关于竞争市场和享乐主义家庭等不现实的假设,必须认
识到和谐体制的含蓄假设使其大部分分析无效;必须用更多的调查和统计工作补充其贫瘠的
演绎方法。

凡勃仑发现了经济学中的很多缺陷,但是,他所提出的可供选择的方法,并不非常富有成效。
他没有运用容易确定的假设和符合逻辑的上层建筑来构建庞大的模型,从而无可变更地得出
明确的结论。\textbf{他用本能心理替代正统理论的享乐主义,而本能心理后来也受到了心
  理学家的否定。}

正统理论家通过替换一个较少引起反对的术语,回应了凡勃仑对他们运用享乐主义概念的批
评。但是,他们的基本模型仍然假设理性的且精于算计的家庭与厂商。凡勃仑抨击的完全竞
争市场假设,并没有因为\textbf{垄断竞争市场和寡头市场理论}而得到明显修改,尽管这种
理论的开发者之一张伯伦向凡勃仑致以谢意。这些市场理论至今仍然不能令人满意。作为福
利经济学发展的结果,以及凯恩斯关于均衡可以与大量失业同时发生这一结论,均衡本身不
再被认为是合意的。凡勃仑对消费者主权概念的抨击,以及他对竞赛与广告在经济体中作用
的分析,在\textbf{不完全市场理论}和约翰·肯尼思·加尔布雷
思(JohnKennethGalbraith,1908--)的著作中得到了进一步的延伸。第二次世界大战后,随
着关注点转向世界不发达国家的增长与发展问题,凡勃仑关于进化变革的观点,引起了一些
注意。

然而,凡勃仑的另一个贡献来自于他偶尔宣扬但从未实践的某种东西——用科学的方法收集实
际材料来检验假设。他以正统理论是一个完全演绎的系统,未能从经验上检验其假设或结论
为由来批评正统理论。然而,凡勃仑自己的理论,也不是以适合于检验的形式提出来的,他
也未能运用统计资料来证明其主张。凡勃仑对正统理论的批评,的确在一定程度上人迫使经
济学家更加关注实际;过去六十年期间,经济学中经验研究的奇特增长,可以部分地解释为
是对凡勃仑遗愿的反应。我们马上将考察韦斯利·克莱尔·米切尔的一些贡献,他是凡勃仑的
学生,在将收集数据用于分析经济周期方面,他是一位开拓者。

最后,我们必须承认凡勃仑对经济学的规范性贡献。其著作从头到尾不仅是对正统的一种科
学背离,而且是一种道德背离。正统理论家,例如凡勃仑的老师约翰·贝获·克拉克,吃惊于
现代工业经济体所生产的物质福利,而凡勃仑则运用他对客观现实的讽刺和看法来描述充满
不幸的经济体。对于很多认为政府行为可以矫正金钱文化最明显缺点的人来说,凡勃仑成了
一种号召力。

\section{韦斯利·克莱尔·米切尔}

1896年,\textbf{韦斯利·克菜尔·米切尔}进入芝加哥大学学习古典文学。在修完约翰·杜威
和托尔斯坦·凡勃仑的课程之后,他对哲学与经济学变得更感兴趣,并最终决定研究经济学。
米切尔后来成为20世纪美国最主要的一位经济学家:\textbf{经济周期的权威,建立研究机
  构研究经济体的先驱,经济理论发展的敏锐观察家。}尽管米切尔不完全接受凡勃仑的很多
思想,但他的经济学也不是正统经济学,所以,他通常被确定为所谓的\textbf{制度学派}。
他认同并补充了凡勃仑对正统经济理论的一些批评,但是,他没有试图构建一个完整的理论
结构来解释工业经济体的演进。米切尔试图遵循凡勃仑关于方法的短文中所推荐的方法,运
用经验材料仔细调查,并为其全部理论研究打基础。他作为一名学者和研究者的风范,以及
他为\textbf{建立全国经济研究局来分析和收集宏观经济数据}所做的工作,比他对纯粹理论的贡献更
为重要。

米切尔的许多短文,以及他的《经济理论类型演讲笔记》(Lecture Notes on Types of
Economic Theory),都表达了他对正统经济理论的看法。在一封写给J. M·克拉克的非同寻常的
快信中,他透露了使他偏离主流经济理论的思想转变。米切尔说,年轻时他就开始喜欢具体
的而不是抽象的问题和万法。他回忆起他的寻祖母,“她是最好的浸信会教友,确切地知道
上帝是如何规划世界的。”米切尔记得他如何开发了“一种顽皮的乐趣,方法是显摆我的婶
祖母无法应对的好辑难题。她总是溜回到逻辑安排中而无视实际,我却逐渐对实际产生了个
人兴趣”。

米切尔对凡勃仑印象深刻,并认为“在使理论尽量延长方面,很少有人能比得上他”。然而,
米切尔意识到,凡勃仑的体系像正统理论一样,具有相同的方法上的弱点,两者都未令人满
意地检验其假设或者结论。“但是,如果能有什么令我确信正统经济学的标准步骤不适应科
学检验的话,那就是,凡勃仑在另一套假定体系下的精湛表演,只获得了非常少的肯定。”®

这一独特态度体现在米切尔的两项终身成就中。在对经济思想史的研究中,他并不对个别理
论家说了什么感兴趣,而是对下列事情感兴趣,即为什么他们抨击某些问题而不是另一些问
题,为什么他们毫无疑义地接受某些假定,为什么他们同时代的人接受他们的结论,并认为
这些结论重要。米切尔在经济理论史方面的作品,可能代表了最佳的相对论者观点。他断
定,\textbf{经济理论在很大程度上能被解释为对当时问题的智力反应}。这一看法也体现在
他对经济周期的研究中。他并未留下一个建立在抽象假定基础上的周期理论,从假定中演绎
出结论来。他的方法是仔细构建并解释时间序列,使之作为初始步骤来证明他所提出的暂定
理论。有时,他对经济周期的研究看上去几乎与理论无关,但是,在整个分析下面存在着一
种理论结构。

米切尔批评了正统理论的抽象模型。“\textbf{投机类型的经济理论}像高等数学或诗歌那样,
被廉价而便利地生产出来——倘若一个人有这种天赋的话。正如那些想象的产物一样,这种经
济理论与现实之间是一种同样有问题的关系。”他也反对正统理论的享乐主义心理假定,但
没有接受凡勃仑的本能理论。他声称,依据\textbf{以经验为基础的行为主义心理的社会科
  学}能够对人类活动做出一种更好的解释。与让不同分支独自行动所实现的方法相比,他提
倡运用一种更加一般化的方法来研究人类行为。正统理论错误地聚焦制度中的常态和均衡,
而不是考察\textbf{制度的动态相关性}。

在他对经济周期的研究中,米切尔特别强调演进累积因果方法。米切尔著作中所暗含的是一
种对正统理论的道德背离以及科学背离。米切尔希望运用经济知识来改善福利,他认为对经
济体的研究揭示出,为了更好地综合厂商的活动,并更好地控制经济活动中的波动,需要国
家的计划。

米切尔将凡勃仑对金钱性职业与生产性职业的划分,作为对其经济周期研究方法的一种广泛
指导。经济活动中的波动,在很大程度上能够通过工商业对利润率变化的反应得到解释。因
为经济决策是在\textbf{预期与不确定环境}中做出的,所以,生意人的投资决策总是反映出
对未来乐观或者悲观的看法。在具有发达货币制度的经济体中,能预期到经济活动中的波动;
因此,具有常态、静态以及均衡这些概念框架的正统理论是不适当的。米切尔并未试图构建
经济周期的另一种抽象模型。取而代之的是,他努力解释经济周期期间发生了什么,并提出
他所谓的对周期的描述性分析。因为每个周期都是独特的,所以,发展一种一般性理论的可
能性就受到限制;然而,所有的周期都具有某些相似点,原因是所有的周期都揭示出了在萧
条、复苏、繁荣、和危机不同阶段经济力量的相互作用。

尽管米切尔之前的一些人将周期看做是一个自生的过程,但是,他第一个赋予这一概念明确
的形式,并用\textbf{广泛的经济数据}予以支持。他对周期的解释基于工商业对利润水平变
化的反应。\textbf{萧条携带着其后复苏的种子,}因为利息率下降,没有效率的厂商被排除,
不变成本与可变成本两者都下降,存货减少,等等。\textbf{繁荣也携带着危机与萧条的种
  子,}因为成本上升,利润随之受到挤压。

米切尔的描述性分析,实际上体现出一个学者对理论、描述以及历史的明智混合,并且全无
数学障碍,在这点上有些像马歇尔的分析。然而,支撑马歇尔微观经济分析的\textbf{坚硬
  理论内核}不见了,程度之严重以至于一些人将米切尔的研究称作是\textbf{没有理论的度
  量}。另一些具有后凯恩斯主义后见之明的人,在米切尔的研究中发现了乘数过程、加速原
理,以及凯恩斯资本边际效率和流动性偏好的对应内容。米切尔认为,不能离开经济体的其
余部分来考察经济周期,它们是制度的重要部分,并且实际上是由制度产生的。当周期的每
个阶段向下一阶段演进时,社会制度结构发生改变,所以“每一代的经济学家将看到重铸他
们在年轻时所学的经济周期理论的理由”。

1920年,在四十五岁时,米切尔创建了全国经济研究局。这一私人非赢利组织对于资助美国
的经济研究极为重要。尽管它最重要的成就包括国民收入的度量和经济周期研究,但是,它
资助了对经济体几乎全部领域的研究。如果我们要考察美国经济研究的发展,那么,米切尔
的作用至少需要一个长篇章节来阐述。在第16章中,我们将在他的一些学生——例如,西蒙·库
兹涅茨(Simon Kuznets,1901--1985)一的著作中,了解到米切尔的若干直接影响,并且,在
对度量经济活动比对构建抽象演绎模型更感兴趣的经济学家的著作中,了解到他的间接影
响。

\section{约翰·R·康芒斯}

康芒斯比凡勃仑小五岁,但比米切尔大十二岁,是另一位来自美国中西部的非正统经济学家。
他出生于俄玄俄州,在印地安那州长大,就读于奥伯林学院,接受了当时一流的古典教育,
包括一门繁重的神学课程,他在约翰斯·霍普金斯大学进行经济学研究生学习,并深受理查
德·T·伊利(Richard T. Ely)的影响。

因为伊利留学德国,受到德国历史学派的影响,所以,霍普金斯大学的政治经济学包括经济
学、政治科学、社会学,还有历史。伊利在劳动经济学方面的兴趣——他在康芒斯进入约翰斯
霍普金斯大学的两年前即1886年出版了《美国劳工运动》——传给了他的学生,康芒斯的全部
生涯都献给了这一经济学领域。两年后,康芒斯离开霍普金斯大学,在1904年最终随同伊利
前往威斯康星大学之前,曾经在一些地方教书。

被一些人称作\textbf{威斯康星学派}(Wisconsin school)的经济学方法,主要是在康芒斯的
影响下在威斯康星大学得到发展。\textbf{这种方法支撑了美国的非正统经济理论,发动了
  改变美国经济结构与功能的改革},从这些方面来看它是重要的。在到威斯康星大学之前,
康芒斯并没有在任何一所大学待很久,这或许是因为他的政治和经济观点,或许是因为作为
大学本科教师,他没有得到充分的认可。然而在威斯康星大学,他为其真知灼见的异议找到
了肥沃的土壤,其至得到了激进的政治家的鼓励,这些政治家热衷于寻找愿意支持社会改革
的学术专家。

在1932年退休之前,在威斯康星大学的二十八年期间,康芒斯在三个主要领域中为经济学做
出了重要贡献:社会改革、研究生教育以及劳动经济学。也许他最重要的贡献是对于社会立
法的制定发挥了作用。这一立法改变了美国经济体的结构。康芒斯的第一本著作《财富的分
配》(1893)并没有获得充分认同。批评家认为,这是康芒斯为其\textbf{社会主义思想}确立
科学基础的一次令人不满尝试。然而,康芒斯并不是一个试图改变私人财产和自由企业社会
结构的革命家。他认为,\textbf{资本主义的本质可以并且应当保持完整无缺,但是,经济
  秩序的运转规则需要变革,以消除自由放任经济体的明显缺陷。}在威斯康星大学,他的观
点获得了州长拉·弗利特(La Follette)的支持。

康芒斯在威斯康星大学的几年期间(1904--1932),如今已成为平常之事的专业学者与政治家
之间的关系得到发展,这种关系在富兰克林,罗斯福新政(New Deal)时期在全国范围内再次出
现。\textbf{威斯康星州政府广泛利用在麦迪逊的大学教职员工充当新思想的智囊团、法律的起草者
以及指定委员会的成员。}康芒斯在威斯康星大学的经历显示,他花了大量的时间帮助起草、
审查通过并执行社会立法。

在这些成就中,一种依稀可辨的模式发展起来了。康芒斯经常在其研究生的帮助下透彻地研
究某个问题。他与经济界中受到新立法影响的人讨论问题,获得更激进的工商界人士或工人
领袖的支持。法律通过之后,他四处游说,使用其他手段推动新立法向其他州传播。体现
在\textbf{罗斯福新政社会立法中的很多思想来自于威斯康星州},这一点很少有人怀疑。毫
无疑问,1932年,很多在麦迪逊接受培养的经济学家和其他人都搬到了\textbf{华盛顿特区}。

\subsection{康芒斯的遗产}

\textbf{康芒斯被描述为“福利国家运动的智力来源”。}1904年,他到达麦迪逊,第二年就
为州长拉·弗利特起草了一项行政事务法;在后来的几年中,他影响着下列领域中的社会立
法:\textbf{公共事业规制,产业安全法,工人的赔偿,童工法,妇女最低工资法,失业赔
  偿法。}失业赔偿立法可能是康芒斯在社会立法方面最大的成就。他对1920年经济萧条的反
应,以及对欧洲失业赔偿计划的研究,促使他为威斯康星州的立法机构起草了一项法案。这
一法案一次又一次地被提出,直至1932年康芒斯以前的一个学生\textbf{哈罗德·格罗弗
  斯}(Harold Groves,当时既是一位参议员,又是一位在大学专门研究公共财政的经济学教
授)又提出这一法案,并最终获得通过。1934年,当罗斯福强烈要求国会通过一项失业赔偿法
时,他组建了一个\textbf{经济安全委员}会来担议立法,委员会的主管是康芒斯的学
生E. E·威特(E. E. Witte),当时他是威斯康星大学的经济学教授。

康芒斯在这些社会立法领域的成就,来自于他深信\textbf{现代工业经济体要想正常运转并
  实现社会公平,就需要政府干预。}起源于威斯康星州的大多数立法\textbf{并不会震
  撼}现代读者,尤其是激进的、好幻想的读者。然而,在康芒斯时代,这些社会改革思想在
美国并不能得到普遍认同。在这一点上康芒斯代表了一种\textbf{非同寻常的非正统经济学
  家类型}。他所做的不只是有反对正统理论所主张的在极大程度上不干预市场配置资源;他
对\textbf{通过社会立法改变现状}感兴趣,并积极参与,努力去实践。并不是他的所有努力
都获得了成功,例如,在实现全国健康保险计划上他并未成功。

康芒斯的第二个贡献与他在社会改革领域的努力相连。对于全世界的经济学家来说,威斯康
星大学经济系以其作为重要的研究生培训中心而著称。威斯康星大学\textbf{一次所授予的
  经济学博士学位,比其他任何大学的都多。}更重要的是,康芒斯特殊的经济学方法,深埋
在经济系的架构中;因此,\textbf{20世纪80年代之前},“威斯康星学派”方法一直持续着。
这一点与凡勃仑或米切尔形成了鲜明对比,他们俩人没有对任何研究生计划产生持久的影
响。

在C. E·艾瑞斯(C. E. Ayres)领导下,位于奥斯丁的德克萨斯大学经济系,以及在艾伦·格仑
奇(Allan Gruchy)领导下的马里兰大学经济系,也在短时期内保持着特定的非正统方法。但
是,这些机构所授予的博士学位数量以及它们的影响力,与威斯康星大学相比要小。要了人
解威斯康星学派方法的死亡,更一般地,要了解集中于这些特定院系的非正统经济学研究生
教育的终结,需要更多的历史视角。除了少数教职员工外,威斯康星大学、德克萨斯大学以
及马里兰大学,似乎都安全地\textbf{回到了正统信仰中来}。

无论如何,康芒斯的方法似乎并不能通过研究生们持续下去,或者传播到其他大学,原因在
于,在威斯康星大学受到训练的经济学家,在极大程度上是以\textbf{经济学的应用领域}而
不是以经济理论为导向的。他们中的大批人毕业后,供职于政府部门、研究机构以及大学。
但是,因为对诸如劳动、公共财政以及公共事业一类的问题感兴趣,所以他们很少对当时几
乎专指微观经济学的正统理论产生兴趣。正如我们将看到的那样,康芒斯批评正统理论,但
是却将他的大部分时间花在了应用领域和社会改革中。

如今,麦迪逊的博士项目与美国其他大学的博士项目具有相同的惯例,这表明蒙恩于康芒斯
的威斯康星学派方法已经死去。

康芒斯的第三个重要贡献是在劳动经济学领域。当康芒斯的老师\textbf{理查德·T·伊利}从
约翰斯·霍普金斯大学转到威斯康星大学时,他把康芒斯一起带着。因为伊利对\textbf{劳工
  运动史}感兴趣,所以,他开始收集劳工史方面的资料。他希望康芒斯根据这些资料研究出
权威性的美国劳工史,这些工作占据了康芒斯在威斯康星大学的大部分学术时间。在其研究
生的大力帮助下,1910年康芒斯出版了《美国工业社会的文献史》,这是一部与劳工史有关
的十卷重要资料集。紧接着是四卷的《美国劳工史》:1918年出版了两卷,1935年又出版了
两卷。康芒斯成了关于美国劳工方面公认的权威,威斯康星大学也变成最主要的培养\textbf{劳动经
济学家}的大学。最著名的毕业生可能是塞利格·珀尔曼(SeligPeriman),他的《劳工运动理
论》(1928)至今依旧是一部经典。

\subsection{康芒斯的经济思想}

尽管康亡斯独立地得出他对正统经济理论的批评:但是,这一批评与凡勃仑和米切尔的批评
相似。他研究社会问题的整个方法,\textbf{否决了新古典理论狭窄的、静态的、演绎的方
  法}。康芒斯试图将\textbf{全部社会科学与法律}带入分析中。他将社会与经济看做
是\textbf{演进与变化的},强烈地反对正统理论几乎唯一的演绎方法,以及享乐主义代理人
和竞争性市场的假设。最后,康芒斯认为,自由放任政策所依据的含蕾假设,即经济体是和
谐的,与他的经验观察相反。

康芒斯声称,有关价格形成与交换的正统理论是不现实的。它假定理性的个人在竞争性市场
中几乎机械地行动。康芒斯说,并不是在竞争性市场中行动的、原子式的、具有享乐主义的
个人,形成了将经济体单独部分连接起来的交换关系。正统价格理论可能令人满意地解释了
一些\textbf{非常特殊情形下的交换与价格,例如高度组织的商品市场或者安全市场,因为
  在这些市场中,存在交换但不存在交换关系。}在这些市场中,买者与卖者之间完全
是\textbf{匿名}的,影响通常的\textbf{市场交易的习惯、风俗以及所有的文化、社会、心
  理力量}都缺失了。在康芒斯的理论结构中,交易成了一种主要因素:

\begin{quotation}
  实际上,交易变成了经济学、物理学、心理学、伦理学、法字以及政治学的聚合点。单个
  交易是明确包含所有这些内容的一种观察单位,原因是,人类的一些意愿,诸如选择可供
  选择的事物、克服阻力、协调自然资源与人力资源,受到有关效用、同情、责任或者它们
  对立面承诺或警告的劝诱,这些意愿被解释和执行公民权利、责任、自由的政府官员或者
  工商业企业或工会官员加以放大、抑制或者展露。正是人类的这些愿望,在有限资源和机
  械力的社会中,使得个人行为适合或者不适合国家、政治、工商业、劳工、家庭以及其它
  集体运动的集合行为。
\end{quotation}

康芒斯发现了经济体中的三种交易类型。“\textbf{买卖交易通过法律对手之间的自愿协议
  转移(transfer)财富所有权(ownership》。}”\textbf{权利的法律平等并不意味着平等的
  经济力量。}确定最终市场和要素市场价格的\textbf{买卖交易}(bargaining
transactions)是正统价格理论的主题,但是,这一理论的确只适用于\textbf{竞争性市
  场}的不正常情形,在竞争性市场中,讨价还价的力量、强迫、说服、习惯、风俗以及法律
都被假设忽略了。第二种类型的交易是\textbf{管理交易}(managerial transaction),涉及
法律上和经济上\textbf{上级对下级的命令}。“它是工头与工人人、州长与市民、管理者与
被管理者、主人与仆人、所有者与奴隶之间的关系。”管理交易涉及财富的创造。康芒斯确
定的第三种类型的交易是\textbf{限额交易}(rationing transactions)。它们涉及“在若干
参与者之间达成一种\textbf{协议的谈判},这些参与者有权力将收益与负担分配给合办企业
的成员”,然后,康芒斯继续进行阐述,界定他所谓的制度:

\begin{quotation}
  这三种类型的交易合在一起,成为经济研究上一个较大的单位,在英国和美国的实践中被
  称作运行中的机构。运转规则使其不断运行,从家庭、公司、工会、同业协会直到国家本
  身,正是这些运行中的机构,我们称之为制度。消极的概念是“集团”,积极的概念是“运
  行中的机构。”
\end{quotation}

\textbf{制度(institution)被界定为控制、解放、扩张个体行动的集体行动。}经济交易涉
及冲突——我得到的越多,你得到的就越少。这些冲突并未在大部分交易中体现出来,原因是
随着时间的变化,通过风俗、习惯、法律等开创了先例,这些先例从冲突中产生了秩序。康
芒斯将这些先例称作运行中的机构运转规则。

借助康芒斯方法的这一基本轮廓,有可能略述他对美国资本主义的分析。新古典理论主张,
由稀缺资源问题引起的冲突,在非个人的竞争性市场中能够得到解决,这种竞争性市场通过
假设,从分析中消除了所有的文化、社会、心理以及法律因素。新古典理论认为,这些冲突
在竞争性市场中的解决所引起的结果,在极大程度上优于通过政府干预可能取得的任何结
果。

康芒斯方法的基本要点是将社会科学、历史以及法律包含到他的分析中,并认识到为了产生
合意的社会结果,政府干预经常是必要的。我们的大部分经济活动并不是个体活动;我们作
为集团成员来行动,集团受到运行中的机构运转规则的指导与影响。尽管这些运转规则的功
能是从冲突中产生秩序,然而有时历史导致的变革又引起新的冲突。随后,这些冲突与争执
得到解决,并且,旧的运转规则得到修正。这是一个无止境的正在进行的过程。

康芒斯认为,经济学的适当主题是\textbf{通过集体行动塑造我们的生活与社会制度。这种
  集体行动不仅控制个体行动,而且通过对其他个体加以抑制来解放个体行动},“使其免受
强迫、威胁、歧视或者不公平的竞争。并且,集体行动还不仅是对个体行动的抑制和解
放——它是\textbf{个体意志的扩张(expansion),扩张到远远超过他靠自己的微弱行为所能做
  到的范畴}”。

因为非规制的经济体产生不合意的社会结果,所以,资本主义需要通过政府干预予以修正。
防止经济萧条的货币政策、意识到劳工组织权利的立法、援助失业工人的赔偿、关怀不幸
者的健康与意外保险、阻止垄断行为的公共事业规制以及其他社会改革,都是由康芒斯所倡
导的。因此,尽管他基本上没有对正统理论产生影响,但是,他所倡导并帮助完成的改革,
极大地影响了美国资本主义的制度结构。

\section{约翰·A·霍布斯}

尽管英国是从斯密到马歇尔的正统经济理论大本营,信奉的主要原则是非规制的市场将导致
社会福利最大化,然而,也存在大批异端。其中最有影响力的可能就是\textbf{约翰·A·霍布
  斯,他的非正统思想成为当今英国福利国家的智力源泉。}在其第一部经济学著作出版后不
入,霍布斯的教师生涯就结束了。他丢了工作,原因是“读了我书的一位经济学教授的干涉,
他认为有理由将我的书等同于试图证明地球是扁平的”?。然而,足以自给的收入使他能够
继续对正统理论进行抨击,他出版了将近四十部书,并发表了大量文章。在凯恩斯于《通论》
中对他加以赞赏之前,他的著作在学术圈中从未得到充分认同;尽管霍布斯对纯理论的影响
几乎被忽略,然而,在塑造英国经济政策方面,他是有一定地位的。霍布斯像很多非正统经
济学家一样,对正统理论的不充分性有先见之明,并且有能力将它们描述出来,但是,却从
来没有能力曾明一种能够推翻公认学说的理论结构。

从宽泛的角度看,霍布斯的非正统思想是对下列公认学说的一种择击,即自由放任是最佳的
政策,因为市场将导致社会福利最大化。正统理论主张,竞争性市场将在极大程度上,在最
低可能的社会成本上,生产至高无上的消费者所要求的产品。这些市场所产生的收入分配,
是根据参与者的生产力而给予其酬劳。此外,这些经济力量的运转,将导致社会资源的充分
利用。因为通常来说,价格很好地度量了经济体中所发生的成本和所生产的效用,所以,它
们是一个社会所获得的福利的指标。

尽管霍布斯接受了正统理论的一些主要假设,然而,就自由放任市场经济的适当性而言,他
得出了完全不同的结论。他发现他所处时代英国经济运行中的三个主要缺陷。\textbf{第一,
  它未能提供充分就业,原因是存在慢性消费不足或者过度储蓄。第二,收入分配不公平地
  偏向那些高收入群体,主要原因是他们出众的讨价还价能力。第三,市场并不是对社会成
  本与社会效用的一种好的度量,因为整个价格系统是以货币利润为导向的。}正统思想家在
经济体中发现和谐,然后构建了一种理论来证明那种和谐,霍布斯则假设自由放任经济体的
负面影响,然后试图构建一种理论结构来弥补现有工业社会的缺陷。霍布斯主张,如果一个
社会的目标被明确地加以界定,那么,经济理论将允许社会实现“好的生活”。

\textbf{他反对约翰·内维尔·凯恩斯关于我们能够区分是什么和应当是什么的观点,并且反
  对将活动仅限于是什么的正统分析人和倾向。}在霍布斯看来,就经济理论帮助社会实
现“所应当的”而言,它正好是有用的。正统理论所尝试的\textbf{规范--实证二分法是不
  可能的,因为同样的事实既是道德的也是经济的。}霍布斯对正统理论的抨击,始于他在与
别人合著的第一本书中对萨伊定律的否定:

\begin{quotation}
  因此,我们得出下列结论,即从亚当·斯密以来,全部经济教义所依据的基础即每年生产的
  数量是由可利用的正常要素、资本以及劳动决定的,这一基础是错误的。相反,所生产的
  数量永远不会超过这些总量所施加的限度,它们可能并且实际上由于生产的减慢而被减少
  到远低于这一最大量,这种减慢是不适当的储蓄以及随之发生的供给过剩的累积施加在
  生产上的。
\end{quotation}

在支持过度储蓄导致经济萧条这一观点时,霍布斯及其合著者A. F·玛麦瑞(A. F. Mummery)的论
点是\textbf{不完善}的,主要是因为他们认同下列正统观点,即\textbf{全部储蓄随着投资
  支出回到收入流中}。

在后来的著作中,霍布斯从未动摇由于\textbf{充分就业下的过度储蓄,资本主义趋向于导
  致经济萧条这一结论。}1902年,他出版了《帝国主义》一书,断言资本主义国家的殖民扩
张,在很大程度上是充分就业下所产生的过度储蓄以及产品供给过剩的一条出路。列宁大量
借鉴了霍布斯的帝国主义理论。\textbf{霍布斯断定,通过帝国主义实践,通过战争支出,
  通过用以改善工人阶级条件的政府支出,通过增加国内奢侈上品消费以及通过更加平等的
  收和分配,能够实现充分就业。道德上正确的备选方案}是很明确的:可以通过\textbf{课
  税}对收入进行重新分配,并与\textbf{政府支出}相结合,来\textbf{改善穷人的状况}。

霍布斯广泛地就收入分配进行创作。他否定分配的边际生产力理论,理由是将边际产品归因
于分开的要素是不可能的。他认为,在现代复杂经济体中,生产是一种社会企业或者合作企
业;如果我们借助微分学,试图确定不同生产要素的边际贡献,我们就是在回避围绕收入分
配的道德问题。此外,在对要素价格决定的分析中,正统理论含蓄地假设,不同生产要素具
有相同的讨价还价能力;但是他主张,对经济体的观察揭示出,劳动的讨价还价状况相对较
弱,这一点导致了低工资。支付给不同生产要素的报酬,能被解析成以下三个部分:(1)仅
仅允许要素维持其自身的报酬;(2)允许要素提高数量和生产力的报酬;(3)超过了用于
维持和提高所必需数量的报酬,霍布斯称之为“非生产性剩余”。现代工业经济体所生产的
产量,超过了足以支付不同要素的维持部分,\textbf{正是要素市场定价的讨价还价过程,
  决定了哪种要素得到非生产性剩余。霍布斯宣称,土地因其天然的稀缺性,得到一种非生
  产性剩余,资本因其一流的讨价还价能力和由于垄断实力产生的人为稀缺性,得到一种非
  生产性剩余。给予劳动较高工资的更加平等的收入分配,不仅会更加公平,而且也提高了
  劳动的生产力。此外,更多的平等将增加消费,减少储蓄,从而使经济体避免经济萧条。}

霍布斯并不满足于将他对正统理论的反对停留在这一点上。他继续对有关价格系统含义的正
统分析进行基本的、彻底的择击。根据霍布斯的观点,正统理论错误地认为价格是生产产品
的社会成本以及从产品消费中获得的社会收益的反映。霍布斯认为,无论从成本方面,还是
从收益方面,\textbf{价格都是对福利的一种不充分度量}。“从这一情形上说,仍然将货币
视为其价值标准并将人类看做是赚钱手段的学科,没有能力面对深刻而复杂的构成社会问题
的人类问题。”

霍布斯的方案是,我们应当考虑人类(human)成本以及人类效用,前者不同于用价格表示的成
本,后者不同于市场价格。在这一分析中,霍布斯在供给成本方面以及在需求收益方面,集
中注意力于现代福利理论及所及的外部性。他对需求方面的分析,反映出凡勃仑的影响;他
指向炫耀性消费所导致的浪费,以及现代经济体中所实行的精巧的推销艺术。霍布斯的方案
是消除政府的自由放任方法,以及现代经济体以利润为导向的性质。“\textbf{直接的社会
  控制}对我们行业正常过程中私人逐利动机的替代,对于社会重建的任何合理方案来说都是
必要的。”

这样简短考察霍布斯与总结他对正统理论所进行的全面反击的特色,并没有什么区别。他否
定萨伊定律,反对正统分配理论,认为价格系统是对社会福利的一种不充分的度量,否定正
统理论的规范--实证二分法,明确提倡将道德考虑注入经济分析中,认为利润动机对社会具
有负面影响,最重要的是,提倡终结自由放任。他经历了很多具有开创性的非正统思想家的
命运:他未能在正统学说控制下的学术界找到工作。他的思想通常未经仔细考察就被否
定。\textbf{1913年},约翰·梅纳德·凯恩斯评论说:“人们怀着矛盾的感情阅读稚布斯先生的新书,希
望看到使人兴奋的观点和一些从独立的、个人的立场对正统学说所进行的富有成效的批评但
也期待更加诡辩的、令人误解的、不合常情的思想。”

后来,随着凯恩斯对萨伊定律的否定及其从正统观点中的退出,他对霍布斯的\textbf{评价
  也因此改变}。\textbf{1936年},他赞扬霍布斯的《工业生理学》是“霍布斯先生在将近
五十年的时间里,以饱满的热情和不屈不挠的勇气攻击正统学派的地位而创作的许多著作中
的第一本,也是最重要的一本,但它未能撼动正统学派。尽管这本书今天已经被完全淡忘了,
然而从某种意义上说,它的出版标志着经济思想史的新纪元”。凯恩斯接着认为,霍布斯属
于一群重要的消费不足主义非正统经济学家,“他们宁可赁着直觉,朦胧地不完全地探究真
理,也不愿坚持错误,他们借助简单的逻辑,条理清晰且前后一致地得出结论,但都建立在
与事实不符的前提上”。

像大多数非正统经济学家一样,霍布斯的直觉见解并没有使他得出一个一致有序的理论结构。
因此,在现在的正统理论中,并不存在可以确认的霍布斯成分。他揭示出正统经济学家满足
于隐藏问题的特点。但当这些问题最终被予以考虑时,解决方案则是由经济学家而不是霍布
斯提出的。不过,霍布斯对英国的经济政策具有重要作用,因为他的思想成为\textbf{工党
  的主要智力影响}。在第二次世界大战之后的时期中,英国实行的\textbf{对产业进行社会
  控制以及包括充分就业政策的劳工计划}就根植于约翰·A·霍布斯的经济学中。

\section{总结}

除了对正统理论提出异议外,新古典经济学的早期批评几乎没有共同点。不同的经济学家以
不同的方式表达了异议,但一般而言,它是与正统理论在范围、方法和内容上的背离,以及
对正统经济学家下列观点的否定,即和谐盛行于市场经济中,因此自由放任是适当的政府政
策。所以,非正统的背离也是科学和道德的背离。\textbf{很多非正统经济学家明确地指责
  正统理论包含规范或道德判断,并试图通过假装发展一种实证科学来加以掩藏。}

德国历史学派反对奥地利学者尤其是门格尔抽象的理论化,讲德语的经济学家之间发生了一
场关于经济学适当方法的著名争论。历史学派也反对下列古典观点,即古典经济理论与政策
适用于欠发达国家如德国,也适用于工业化国家如英国。他们希望\textbf{保护其“幼稚产
  业”}。与自由放任古典观点相比,他们提倡\textbf{政府发挥更大的作用}。

凡勃仑鼓吹科学的方法,但只创作了令人印象深刻的作品;米切尔受教于凡勃仑,他实践着
科学,但难以根据所收集的数据得出理论上的结论。没有哪位经济学家提出了一种理论结构
来取代其所批评的模式。康芒斯的确提供了一种可供选择的结构,但并不被后来的经济学
家——正统的或者非正统的——认真考虑。像康芒斯一样,霍布斯引人注目地影响了经济政策,
但是,他的理论贡献在很大程度上被忽视了将近三分之一世纪,直到一些人回顾过去并认识
到其见解的价值。

与大部分正统经济学家的结论相比,所有这些经济学家在不同程度上都得出要求市场中有更
多政府干预的结论。一些评论者断定,因为非正统理论的特定说法未能取代正统理论,所以,
非正统理论是一种失败。我们的观点有所不同。对非正统思想的考察揭示出,尽管它没有取
代经济思想主流,然而,它经常迫使正统理论进入新的路线,有时还会提供开创性的思想,
这些思想成为公认理论结构的一部分。对思想潮流方向及其内容的这些贡献不能被忽视。

对新古典经济学的制度主义批判和其他非正统批判,并没有在早期批评家那里结束。对正统
理论的抨击一直持续着(在某些场合下变得更强烈)。尽管对政策的反击不一定都是正确的,
然而非正统经济学家正在提出的很多政策变革,实际上在20世纪已经实现了。霍布斯与其他
英国改革家影响着英国的社会政策,并且,美国制度主义者的很多思想在新政中得到贯彻。
因此,非正统经济学家对于资本主义制度结构具有重大影响,他们的很多批判由于对批判的
反应而显得直率。

然而,在理论领域他们只有较小的影响。随着西方经济制度结构的变革,建立在与纯粹市场
经济最为相关的制度结构基础上的新古典理论,并没有发生变革;相反,它只是更深地退回
到与政策有很少或者没有相关性的纯粹抽象理论上来。正如我们在17章中考察最近的非正统
经济思想时看到的那样,对主流思想的挑战是制度主义性质的,因为它们是对凡勃仑、康芒
斯以及米切尔的智力继承,这些挑战越来越多地集中于正统理论与现实的分离。

\chapter{奥地利学者对新古典经济学的批判\\以及关于社会主义和资本主义的争论}


正统主流经济学家经常不考察下列较为宽泛的问题,即哪种经济制度更优越。取而代之的是,
他们集中研究关于市场的经济学。他们对马克思的反应大多是沉默,好像这个话题较为低级。
例如埃奇沃思曾说“我们对那些坚持关注马克思理论的科学作家抱有很大同情”。当一些主
流新古典经济学家确实参与到争论时,是因为争论略微涉及一个技术上的要点——市场与社会
主义是否相容。\textbf{新古典经济学家认为自由市场与社会主义是相容的,所以引发了奥
  地利经济学家对这一问题的质疑。}

\section{定义资本主义和社会主义}

“资本主义”和“社会主义”这两个词有着\textbf{普遍而不确切的含义}。它们将经济特点
与意识形态结合在一起;对一些人来说,它们意味着好与坏,对另一些人来说,它们意味着
坏与好。从理论上讲,我们可以\textbf{精确地定义}这些词,但是如果这样做的
话,\textbf{可能没有任何一个社会(例如英国)适合被用作任一主义的标准。}一方面,我
们对什么是资本主义和社会主义有了理论上的认识;另一方面,我们有既包含理论资本主义
又包含理论社会主义要素的现有制度。当每一个制度的倡导者在理论上为他们选择的制度构
造他们的论点,但又在一个现存的社会中构造反对他们的对手的证据时,这最后一点就变得
相关了。我们将在本章中以理论术语为主要框架进行讨论。

在资本主义中,经济决策主要由个人在其角色中完成。作为消费者、生产要素所有者和企业
管理者,大多数经济资源是私有的。在社会主义中,经济决策是由个人来做,他主要是作为
选民、政治家和公司管理者;经济资源可能私有或公有,但资源配置由政府控制,不由资
源所有者控制。

这些定义以经济标准为中心,但不可避免地与政治和社会问题相关。在任何一种制度下,自
由(经济和政治)和民主都可以得到高度发展或弱化。资本主义的拥护者经常断言,自由只
有在资本主义(读作“理论资本主义”)下才可能存在,在社会主义(读作“实际社会主
义”)下并不存在。社会主义的捍卫者经常认为,真正的自由在资本主义下是不可能的(
读“现有资本主义”),只有在社会主义下才能真正实现(读“理论社会主义”)。我们将
回到资本主义、社会主义和自由这一问题上来,因为最近人们对它进行了广泛的讨论。

这两种社会制度的起源完全不同。\textbf{资本主义是一种制度,这是历史性的发展,}当经
济学家试图解释这个制度地运作时,它就变成了一种\textbf{智力或理论结构}。\textbf{社
  会主义则相反,}它首先是在\textbf{智力上}作为现有制度的替代理论结构,然后开始作
为\textbf{现有制度}进行尝试。

这两个制度在理论上和(特别是)实际形式上都在不断发展。这种发展,一部分是因为我们
对理想类型的两个制度的理论理解有了进步。另一部分是因为现有制度会随着时间的推移而
变化。由于变化,今天的资本主义和社会主义与50年前大不相同,这些变化使分析复杂化。

\textbf{从20世纪30年代到60年代,}资本主义在理论和实践上均发生了变化。\textbf{就资本主义的定义来说,
它越来越倾向于政府对资本主义的管控地位,以及在企业管理控制方面和政府监管方面的管
理权与所有权分离。}

\textbf{从20世纪80年代到21世纪初期,社会主义一直在变化;}在这段时期的理论和实践
中,\textbf{市场和私有制}被视为与社会主义一致。

因此,在理论和实践中,既有社会主义更多利用资本主义制度,也有资本主义更多利用社会
主义制度的转变。这些观察结果使一些人猜测这两个制度正在\textbf{收敛},每个制度都会
脱离其纯粹形式的缺点并向共同的分母发展。

\section{资本主义思想的出现}

没有人知道资本主义是何时形成和发展起来的。没有预见或计划,一种社会组织就出现在西
欧与英国,并发展成为马克思所谓的资本主义。之前的社会被过去牢牢束缚:宗教与政治力
量的传统和权威阻止变革。\textbf{出现资本主义的一个本质因素是个人从教会、行会以及
  国家中解放出来。新的经济产品种类随着资本主义而出现——人们能够自由购买与销售劳动、
  土地以及资本。}

土地由获得地租的地主所有;劳动力由获得工资的工人控制;资本被获得利润的资本家控制。
这些群体构成了截然不同的社会和经济群体,被用来作为古典分析的基础。决定这些群体之
间收入分配的力量是什么?制度增长的动力是什么?产品的资本所有者被视为提供了增长的
动力——因此这样的社会得名资本主义。

在封建制度下,劳动、土地以及资本的使用不由市场活动所决定,而是由传统与权威所决定。
随着社会与经济组织新形式的出现,出现了企业家这个新的参与者,它们成为资本主义中变
革的要素。至关重要的是,\textbf{作为与封建主义相对的资本主义,在其制度中深埋下了
  进一步变革的机制。这是人们通过研究资本主义的那些伟大学生——亚当斯密、卡尔·马克思、
  约瑟夫·熊彼特所得到的最重要的见解之一。}

尽管企业家是资本主义动态变革中的因果要素,然而,还有另外一个要素使进化的改革成为
可能,尽管它没有引发改革。在封建主义与重商主义下,国家的一项功能是抑制引起变革的
力量。\textbf{在重商主义下,国家广泛地被特殊利益群体加以利用来保护特权阶级,尤其
  是商业群体。}随着市场的成长,发生了政治生活的重大调整,更多的民主政治安排与不都
变化的经济结合成为\textbf{民主资本主义}。\textbf{民主之所以重要是因为它允许变革,
  但却得以保留根本的政治和体制结构。}最近发生在社会主义社会中的革命变化,部分原因
是缺乏一种制度结构,这种制度结构可以容忍微小的变化,同时保护制度的基本完整性。

\textbf{资本主义社会取代封建主义有两个引人瞩目的因素:一个是赋予制度动力的企业家;
  另一个是使新的安排更加容易,没有分裂当时社会基本结构的民主政治。}

在既定的财产权结构下,市场进行调整。市场允许人们交易,从而提高了他们最初的天赋权
利的价值。但是,市场不能解决最初不被认可的或不公平的财产权问题,不能解决当财产权
尚未得到发展的分配问题。民主政治是一种政府制度,它允许人们投票来决定政府政策,
并且更改现有的财产权,以使制度充分公正,人们能够接受它。在资本主义下,我们已经看
到通过课税、规制以及授权对财产权进行的巨大更改,而基本的市场框架维持不变。

古典政治经济学先驱、古典学派以及新古典学派考察了这一新兴的变革的制度,从特定的意
识形态角度给予我们对资本主义理论上的理解。随着市场制度开始出现,作为个体经济活动
的协调者,价格起了较大的作用。对市场功能的这种看法,前古典学者朦胧地注意到,亚当·斯
密和古典学者非常清楚地看到,新古典学者不仅将其作为一种看法,而且用正式的模型加以
表述,并详细叙述了导致资源有效配置的条件。新古典经济理论发展成为一种解释在既定的
私人财产制度下市场如何运作的理论。从这个意义上说,因此而形成的新古典经济理论是资
本主义经济思想。\textbf{新古典经济理论将制度作为既定;它并不致力于研究粗线条的问
  题,诸如私人财产是怎样产生的或者怎样的财产权结构是最佳的。奥地利经济学家重点考
  虑了这些粗线条的问题。}

实际上,流派之间的区别并不是这么清晰。在思想流派之间整齐地划出分界线来是创作者们
开发的教学上的辅助手段,用来阐明方法与观点的差别。正如我们将要在下面所讨论的那样,
奥地利经济理论由新古典经济学演变而来。因此,他们的观点能被看做新古典思想的一部分。
对奥地利学者而言,这样说是正确的。但是,随着时间的发展,主流经济学与奥地利经济学
已经分离。关注使两个群体开始走向不同道路的早期区别,就容易理解它们现在的差异。

\section{奥地利思想的演变}

奥地利经济思想流派的早期主要成员是门格尔、维色以及庞巴维克。门格尔被认为是新古典
思想的奠基者之一,\textbf{新古典思想聚焦于功利主义和由个人的主观观点决定而不是由
  成本决定的价值。}维色和庞巴维克是门格尔的追随者,始终拥护主流新古典经济学。但
是,\textbf{主流经济学很快就倾向于形式主义的数学思想,聚焦于完全竞争和一种狭窄的
  分析,假定市场存在并避开粗线条的问题。正是在这些问题上,奥地利经济学开始与新古
  典经济学产生分歧。}

奥地利学派成员有时考虑形式问题,同时也考虑粗线条的问题,并认为这些问题比技术上的
问题对经济思想更加重要。因此,正是这个群体带头回应关于什么制度更优越的社会主义挑
战,并带头捍卫资本主义。尤其是,庞巴维克就众所周知的改革问题挑战马克思主义者;稍
后的一位奥地利学者路德维希·冯·米塞斯挑战社会主义经济学的真实基础,认为在社会主义
经济体中不存在合理资源配置的基础。

尽管从同一起点开始,但是,在方法与关注点上,奥地利经济学变得日益与新古典经济学相
分离,在方法上,主流新古典经济学越来越数学化,而奥地利经济学非数学化地发展着,并
将法律与制度融入其分析中;在关注点上,新古典经济学关注于均衡,而奥地利学派关注对
制度、过程以及非均衡的研究。此外,主流新古典经济学主要是由英国和法国的新古典学派
组成的,集中研究作为参照点的完全竞争,奥地利经济学则并非如此。对于正确的制度结构
而言,奥地利经济学具有一定的意义,但是,对于正确的价格而言则没有什么意义。对于奥
地利学者来说,正确的价格是正确的制度结构所形成的任何价格。奥地利经济学与主流新古
典经济学的差别体现在,门格尔缺乏对数学形式的关注,以及维色将一种权力理论与他的市场
理论相结合来得出关于经济体的完整理论。

随着新古典经济学的发展,门格尔的追随者维色、庞巴维克、米塞斯以及哈耶克离新古典主
流越来越远。但是,\textbf{只有在20世纪下半叶,奥地利经济学才逐渐被看做一种独立的
  非正统方法,而不是新古典经济学的一个小分支。}一旦奥地利经济学被看做是一种独立的
方法,奥地利学者们的早期作品就被重新加以考察,它们与新古典作品的差别就成为焦点。
例如,\textbf{维色强调经济学演进的制度方面,认为个体所创造的制度,引发对自由
  的“自然控制”,而自由影响着个体的行为。这些自然控制包括财产权、契约以及法律制
  度。}因此,根据他的观点,经济学家考虑经济学与政策时,就得远远地超越市场和市场价
格,并考虑市场力量运行所经历的整个过程。维色在其经济分析中也包含了一种\textbf{权
  力理论},并在《社会经济学》一书中发展了一种\textbf{规范化的经济政策方案},该方
案\textbf{远胜于}新古典主流中的任何政策方案。

新古典经济学成为了一种价格理论;奥地利经济学成为了一种经济过程与制度理论。正是由
于这个原因,奥地利学者回应了马克思对资本主义的抨击,而主流新古典经济学却在极大程
度上忽视了马克思的抨击。

\section{社会主义经济思想的发展}

在主流经济思想家为资本主义谱写颁歌的同时,其他人则得出了不同的结论。即使在耶稣基
督诞生之前,一些人就非常担忧地观察到,生活中的经济方面被给予了更多的注意力。在资本
主义于工业革命期间充分形成之前,一些经济学家就已经非常了解初生的资本主义,并发现
它对个人和社会所产生的有害影响。

这些早期的哲学家和道德家是早期和中期社会主义思想的先驱者,即马克思所谓
的\textbf{乌托邦社会主义者}(utopian socialists)。前马克思社会主义者将他们的著作
定位在对资本主义社会的批判上,而对于说明他们所倡导的社会(社会主义)的本质可能是
什么样子,则投入了很少的注意力。他们\textbf{尤其忽视了社会主义的经济组织}。

\subsection{关于社会主义的早期作品}

资本主义的一些早期批评家的共同点非常少,确实能将这一广泛群体绑在一起的一条共同线
索是,他们都将19世纪西欧资本主义的运行看做是不和谐的。几乎所有的前马克思早期资本
主义批评家,都提倡消除社会冲突的非暴力手段,尽管他们开出的治疗方法随每位经济学家
的不同而变化。

\textbf{社会主义这一术语的最早使用出现在路易斯·博朗(Louis Blanc,1811--1882)的
  作品中。}他认为,经济制度应当为每个人提供一份工作,他将社会主义界定为一种制度,
在这种制度中所有人都拥有支付了合理工资的工作。这一术语很快就包括了政府,政府对生
产手段的控制使之成为这些工作的提供者。\textbf{博朗创造了“各尽所能,按需分配”这
  一说法。}

\textbf{罗伯特·欧文(Robert Owen,1771--1858)}是英国一位重要的早期社会主义者,也
是一位将其注意力转向资本主义罪恶的成功实业家。\textbf{他遵循戈徳温传统,断言人类
  是日臻完美的,社会罪恶是制度因素的结果。因此,他提倡教育改革,以及用合作的市场
  过程替代竞争的市场过程。}值得注意的是,他否定任何有关他所处时代社会阶级冲突的观
点。

另一群英国经济学家得出了与欧文的观点相似的结论,但是,因为他们对社会缺陷的批评性
分析是从劳动价值理论开始的,所以,他们以\textbf{李嘉图社会主义者}(Ricardian
socialists)而为人所知。在李嘉图体系中,地主是获得部分社会股利的寄生虫,并不履行
实质性的经济职责。这些经济学家运用李嘉图的劳动价值理论得出结论认为,因为劳动是全
部价值的源泉,所以,资本家通过剥夺劳动的一部分成果而剥削劳动。这些经济学家中最重
要的是约翰·布雷(John Bray,1809--1897)、约翰·格雷(John Gray,1799--1883)、查尔
斯·霍尔(Charles Hall,约1740--约1820)、托马斯·霍奇斯金(Thomas
Hodgskin,1787--1869)以及威廉·汤普森(William Thompson,1775—1833)。

早期法国社会主义者中最突出的是昂利·德·圣西门、查尔斯·傅立叶以及皮埃尔-约瑟夫·蒲鲁
东(Pierre-Joseph Proudhon,1809--1865)。\textbf{圣西门提出通过国家计划扩大经济
  产量的可能性,其中,科学家与工程师起着主要作用,他因此给人们留下了深刻印象;傅
  立叶的美好社会的观点,设想了保证支付所有人最低收入的合作企业;薄鲁东不信任国家
  行为,推崇一种无政府状态,其中信用被给予所有的人而无需向借用人索要任何利息。}

尽管德国早期社会主义者对经济理论的发展具有很少的直接或间接影响,但是,一位瑞士经
济学家\textbf{吉恩·查尔斯·西斯蒙第},却值得仔细关注,将西斯蒙第归类为\textbf{社会
  改革家}比社会主义者更为合适。西斯蒙第是一位在历史方面多产的经济学家,他创作了十
六卷的意大利史以及三十一卷的法国史。他对经济思想的主要贡献包含在他的《政治经济学
新原理》中。在他的早期作品中,西斯蒙第遵循了亚当斯密的思想,认识到经济体基本上是
和谐的,并认为自由放任的政府政策最有益于社会。但是,在其《新原理》中,他断言斯密、
李嘉图以及萨伊都过高估计了自由放任的好处。\textbf{他抨击萨伊定律,认为对大多数人
  来说,自由放任政策将导致失业和穷困。尽管他确信自由放任市场所实现的收入分配是不
  合理、不公正、不公平的,但是,他赞同李嘉图收入分配是经济学中最重要问题的看法。}西
斯蒙第注意到,小农场所有者和小商店所有者在缓慢却无疑地消失。他设想了一
个\textbf{阶级冲突}的社会,而不是和谐的社会,因为在无产阶级与资产阶级之间,社会变
得越来越\textbf{两极分化}。他认为,由日益工业化引起的总产量的大量增加,并没有随着
福利的提高而传递给普通公民。因此,西斯蒙第对正统学说的批评要点是,否定古典自由主
义提出的和谐,相反却发现了不和谐,这种不和谐体现为制度\textbf{未能提供充分就业},
从而体现为日益增多的阶级冲突。西斯蒙第显然是马克思的前辈。

西斯蒙第对资本主义失灵的评价,更多的是直觉上的而不是分析上的;他所提出的补救对策
含糊不清,一部分内容内在不一致。在西斯蒙第看来,经济活动水平周期性波动的主要原因
是竞争性市场的不确定性,以及小农场主和工匠的消失。他的补救措施是放慢由资本主义所
导致的生产增长,回归到下列这样一种经济体中,即劳动与资本的分离是最小的,生产更加
密切地与经济体的消费能力相吻合。

\textbf{他提倡小规模独立的工业和农业经济单位,这使得他捍卫私人财产。}西斯蒙第对萨
伊定律的否定,以及他用资本家与劳动者之间的阶级冲突这一不和谐取代古典体系中的和谐,
都使得他的观点与他曾经认同的斯密传统存在尖锐对立。

\subsection{马克思与社会主义思想}

正是马克思将不同的社会主义思想集合,转化为改变社会的一种理论结构与社会运动。马克
思将劳动价值理论作为对资本主义的批判而不是资本主义的支柱。与以人类的贪婪为基础的
资本主义制度相对,他赞成一种社会主义制度,在这种制度中人类的善良普遍盛行。马克思
的论点是双重的:\textbf{(1)资本主义具有内在的不稳定性,将会自我灭亡,(2)作为
  一种社会结构,资本主义在道德上是错误的。}

马克思宣称资本主义由于其内在矛盾而必将灭亡,并将被社会主义最终被共产主义所取代。
他赞成一种可供选择的经济制度,这种经济制度考虑到了在经济制度内部造成内在矛盾的紧
张状态。他认为,当这些紧张状态被予以考虑时,就可以看到资本主义是不稳定的,并且将
会失败。

马克思在《资本论》第二、三卷出版之前就于1883年去世了,所以《资本论》第二、三卷是
由他的合作者恩格斯整理出版的。但是,马克思的去世并没有结束对社会主义的讨论。
在19世纪90年代后期以及20世纪早期,对马克思理论的重大讨论在知识分子中间一直持续着,
这使得一位观察家将这一时期称作马克思主义的黄金时期。情况可能正是如此。然而,主流
经济学并没有参与任何一次对马克思现点的重大讨论,它已经开始从事一种新的新古典方
法。

\section{关于经济制度的争论}

关于经济制度的争论包括三个次争论:\textbf{转型问题争论、过渡问题争论、资源合理配
  置争论。}每一个都在更广泛的社会主义、资本主义争论中发挥了一定的作用,第三个是最
重要的。

\subsection{价值转移问题争论}

新古典经济学已经放弃了古典经济学的劳动价值论,取而代之的是价值理论,其中需求和供
给决定了价格。劳动理论被拒绝的原因之一是\textbf{转型问题——当行业之间的资本密集度
  不同时,如果一个人在整个经济体中也采取统一的利润率,就无法从劳动理论中获得市场
  价格。从而,人们无法将劳动价值转化为市场价值。}

在《资本论》的前两卷中,马克思通过以下方式避免了转型问题假设所有行业的资本密度都
相同。有了这个假设,它才有可能从劳动价值转向市场价格;没有它,这是不可能的。马克思
承诺在第三卷中删除这一假设并结束他的理论体系。但是,当第三卷出现在1894年,经济学
家仔细观察马克思是否有解决了转型问题,显然他没有。奥地利经济学家庞巴维克迅速指出
了这一失败。在卡尔·马克思及其制度的封闭中,\textbf{庞巴维克认为,马克思简单地承认,
  当各行业的资本密集度不同时,市场价格与劳动力价值不成正比。因此他未能说明劳动价
  值论如何解释市场价格。}在庞巴维克这破坏了马克思整个体系的逻辑基础。

然而,这不是大多数马克思主义经济学家的观点。他们认为庞巴维克的观点太狭隘了。他们
断言,\textbf{经济理论的目的,不是解释市场价格,而是解释社会现象。}此外,马克思主
义者不关心相对价格,而是关注资本主义内部矛盾导致的崩溃。他们拒绝了资本主义的消亡
取决于劳动价值论能否解释相对价格的观点。

\subsection{过渡问题争论}

资本主义并没有像马克思预言的那样,因为生产力和生产关系之间的矛盾而在革命中崩溃。
它也没有像西德尼和比阿特丽斯·韦伯预言的那样,通过和平的政治手段,顺利地演变为社会
主义。马克思预言的资本主义崩溃的失败引起了马克思主义者在19世纪末和20世纪初的许多
讨论。\textbf{关于无产阶级革命和资本主义衰落为什么没有发生的一系列解释出现了:列
  宁认为,资本主义社会是帝国主义的,是通过对世界欠发达地区的剥削;其他人认为,资
  本主义正在自我调整,从而减缓内部矛盾破坏制度的进程;还有人认为,资本主义的死亡
  必须借助于革命行动,由于社会主义制度是一个比较好的制度,他们坚持实行武力是合适
  的。}

列宁于1917年在俄罗斯领导了一场共产主义革命,并通过法令建立了一个新的社
会——\textbf{一场不是无产阶级的革命,而是一场自称无产阶级先锋的精英的革命。}俄罗斯
似乎不太可能成为社会主义的起点,因为它没有高度工业化,而且在很多方面都是封建
的。\textbf{(今天许多人认为,当时在俄罗斯建立的新社会与社会主义关系不大。)}奥地
利人和其他人认为,这场俄罗斯革命是马克思分析法预测未来的又一次失败。\textbf{奥地
  利人}比其他人更有力地争辩说,\textbf{强制实行社会主义表明社会主义与个人自由不相
  容,因此是一种不可取的制度。他们指出,社会主义最初是在俄罗斯实施的,这一事实表
  明,它不是对资本主义社会紧张局势的回应,而是人为地强加给一个小而强大的精英阶层。}

\subsection{资源分配争论}

\textbf{第三个争论是社会主义社会如何在没有自由市场的情况下分配资源和进行经济活动,
  作为分配的主导机构。}这场关于社会主义下资源配置的辩论分为两部分:一部分是粗线条
辩论,另一部分是技术辩论。因为这场波澜壮阔的辩论已经渗透到了公众,触及了经济和政
治制度的各个方面。这场技术性的辩论虽然焦点更窄,但却加深了我们对微观经济理论及其
局限性的理解。

由于马克思没有对社会主义社会如何分配稀缺资源提供指导,那些对这个问题感兴趣的人转
向了马克思以外的作家和(或)对这个问题提出了新的看法。

\subsubsection{资源配置问题的早期工作}

\textbf{马克思写的是资本主义,而不是社会主义下如何分配资源。}在很大程度上,马克思
之后的社会主义者作家直到20世纪20年代才解决这一问题。尽管关于这一问题的一些有趣的
讨论早就出现了,但直到20世纪20年代和30年代对资源分配问题进行了更全面的研究之后,
他们才感受到这一问题的影响。尽管这些著作几乎没有直接影响到后来的左派。他们为20世
纪20年代开始的关于社会主义社会资源配置的大辩论奠定了基本框架。

1874年,德国人\textbf{阿尔伯特·沙弗尔}(Albert Schaffle,1831--1904)出版了一本书,
二十年后被翻译成《社会主义的精髓》。沙弗尔是一个非社会主义者,他对德国历史学派产
生了浓厚的兴趣,对社会主义提出的问题产生了兴趣。他提出了两个问题,这是后来的社会
主义经济理论文献中的主要问题。首先,利用什么机制来分配稀缺资
源?\textbf{Schaffle认为,如果一个社会主义经济体的价格建立在一个不考虑使用价值的
  价值理论基础上,而只关注成本方面,即劳动力成本,那么它就不能有效地分配资
  源。Schafle提出的第二个问题是关于社会主义和自由之间可能发生冲突的问题。他的立场
  是,社会主义的好处可能会被个人自由的丧失所抵消。}Schafle的两位同时代人Lujo
Brentano(1844--1931)和Erwin Nasse(1829--1890)扩展了Schafle关于社会主义和自由
的思想,认为社会主义和计划与自由是不相容的。

瑞典经济学家古斯塔夫·卡塞尔(1866--1945)在本世纪初对奥地利人的边际效用思想产生了
兴趣。他在《价格基本理论纲要》(1899年)中研究的一个问题是,一个不以私有财产为基
础的经济体是否能够有效地分配资源。他认为,社会主义的一个根本缺陷是不能正确地确定
生产要素的价格,因而不能正确地指导生产。

维尔弗雷多·帕累托把他的注意力转向了《社会主义经济学体系》,1902--1903年出版了两卷。
他把帕累托最优福利理论应用到社会主义经济中,没有找到在社会主义条件下实现最大福利
的理由。帕累托的追随者意大利恩里科·巴龙(1859--1924)进一步探讨了这些问
题。\textbf{1908年,巴龙成为第一个系统地研究社会主义制度下实现资源优化配置所需条
  件的经济学家。}巴龙首先展示了在完全竞争的市场条件下资本主义制度下实现最大福利的
条件,然后建立了除劳动力以外的所有资源都是集体所有制和生产部类控制着经济。他得出
的结论是,\textbf{如果该部类制定的价格等于生产成本,并且生产成本处于最低水平,则
  存在资源的最佳配置,并实现最大福利。}保罗·萨缪尔森在1947年写的文章中说:“自巴
龙的作品问世三分之一世纪后,英语语言中没有更好的表述,这是对他作品的一种敬意。”

\subsubsection{奥地利的贡献}

技术辩论的关键问题是如何在社会主义下分配资源。路德维希·冯·米塞斯(1881--1973)成
为作家,他对这个问题的后续发展影响最大,可能部分是因为他对社会主义的攻击,部分是
因为他一生中大部分时间都在关注这些问题。后来,他的学生弗里德里希·冯·哈耶克
(1889-1992)加入了他对这个问题的研究。

\textbf{1920年,米塞斯发表了一篇文章,他在文章中声称,在社会主义条件下不可能实现
  资源的合理配置。}米塞斯观察了资本主义条件下的市场运作,特别指出了要素市场发挥的
关键作用。在这些市场中,土地、劳动力和资本的所有者向要求他们的公司提供生产要素。
价格应运而生,基于这些价格和现有技术,公司决定以最经济的方式结合各种因素生产最终
产品。\textbf{在社会主义条件下,生产要素大部分不是个人所有,而是社会所有。}米塞斯
认为,由于没有生产要素的独立所有人,因此\textbf{没有要素市场},也没有一套价格从这
些市场中出现。\textbf{没有要素价格,资源配置的理性决策是不可能的}:“一个人一旦放
弃了高阶商品(生产要素)自由确定的货币价格的概念,理性生产就完全不可能。把我们从
生产资料私有制中解放出来的每一步,都会把我们从理性经济学中解放出来。”

尽管帕雷托和巴龙的早期工作已经证明米塞斯的论点是错误的,但他的立场受到
了\textbf{F. M. 泰勒}1928年在美国经济协会的总统演讲中的挑战。泰勒声称,在社会主义下,
分配资源的问题是\textbf{可以合理解决}的。他建议国家按照接受的任何目标分配收入,并
允许家庭将收入用于自由市场。\textbf{国有企业将计划生产以满足消费者需求,从而使价
  格等于生产成本。生产要素的价格将由插补过程决定。试错法会向规划者揭示各因素的均
  衡价格。因此,社会主义下不存在根本的资源配置问题。}

随后,诺贝尔奖获得者F. A·哈耶克和莱昂内尔·罗宾斯(1898-1984)开始了一场新的争论,
争论不断扩大。他们认为,虽然解决社会主义分配问题在理论上是可行的,但实际上是不可
能的。为了理解他们的论点,把经济想象成一台巨大的计算机。对于一个家庭需要的每一种
商品,都有一个等式;对于一家公司提供的每一种商品,都有一个等式;等等。\textbf{哈
  耶克和罗宾斯坚持认为,社会党计划者不可能收集到合理分配所需的大量数据,更不用说
  同时求解这些方程了。}

这一阶段的辩论被\textbf{奥斯卡·兰格}(1904--1965)在1936-1937年发表的两篇论文
中\textbf{有效地结束}了,这两篇论文以《社会主义经济理论》为标题进行了修订和出版。
兰格和其他许多关于这些问题的作者一样,也对福利经济学做出了重要贡献。他是一个社会
主义者,在美国\textbf{芝加哥大学}任教,二战后回到波兰。在回应米塞斯、哈耶克和罗宾
斯的论点时,\textbf{兰格声称,一旦认识到要素价格可以用于合理分配,无论要素价格是
  来自竞争市场还是由国家计划者制定,他们的论点都失败了。市场价格实际上只是为买卖
  双方提供的替代品的指数。}在资本主义的竞争市场中,家庭销售因素和公司购买因素对决
定这些价格的力量\textbf{并没有真正的了解}。但缺乏知识并不影响他们的行动。他们以价
格为参数并采取相应的行动。通过\textbf{反复试验},规划者将找到使供应量等于需求量的
价格,从而清理市场。

兰格接着指出,\textbf{在竞争资本主义下,新古典主义理论发现三个条件是均衡的。(1)
  消费者和生产者都处于最大化地位:(a)消费者支出其有限收入以最大化满意度;(b)
  生产者最大化利润。(2)每一个价格都是这样的:供应的数量等于需求的数量,这样所有
  的市场都被清理干净。(3)向消费者收取的收入,与其销售要素收入加利润相
  等。}在\textbf{计划社会主义均衡下(1a)是不变的。}因此,兰格认为,消费者将能够
花他们的收入,以最大限度地满足。\textbf{条件(1b)不再适用于社会主义,因为国有企
  业对利润最大化不感兴趣。}兰格将通过要求生产商遵循两条规则来\textbf{取代条
  件}(1b):\textbf{第一,他们以尽可能低的成本生产每一种产品;第二,他们选择生产
  规模,使价格等于边际成本。条件(2)是由自由市场力量在资本主义中产生的。兰格主张,
  社会主义下的市场出清将由国家计划者在试错基础上调整价格来实现。过高的价格会带来
  盈余,并向规划者表明降低价格的必要性。价格太低会导致短缺。条件(3)在社会主义下
  是成立的,除非没有利润。}

兰格意识到他的文章只是对泰勒论点的延伸和阐释。在社会主义下,没有比在资本主义下更
需要一台庞大的计算机来解决供求关系。帕雷托--巴龙--泰勒--兰格--勒纳的论点实质上是
说,如果一个社会主义经济通过计划和指导,能够产生与完全竞争市场状态下相同的结果,
那么它将最有效地分配资源。因此,国有企业将通过在其长期平均成本曲线的最小点上运营
来满足消费者的需求,即边际成本等于价格。

\textbf{到1940年,业内一致认为米塞斯和哈耶克错了,社会主义可以合理分配资
  源。}1948年,在美国经济协会的赞助下,Abram Bergson(1914-)在两卷调查文章集中撰
写了一篇文章,证明了这一接受,该文章旨在供经济学家们在各个领域用公认的思想来更新
这些文章。

伯格森在对社会主义经济理论的调查中指出,“到目前为止,人们普遍认为米塞斯提出的那
些问题的论据……他接着指出,哈耶克在有关国有企业监管和规划者获得提高效率所必需的
知识的辩论中所作的一些贡献“夸大了问题的困难”。

米塞斯、哈耶克和其他人不能有效地表达他们对理论和实践社会主义的批评,其中一个原因
是1930年代和1940年代的经济模式的状态,它本质上是一个平衡模式。很少有人讨论如何进
行不平衡调整,个人创业行为也不是正式模式的一部分。由于不平衡调整不是正式模式的一
部分,对基于需要采取此类行动的论点的批评是不可接受的。

接受奥地利的论点会破坏社会主义的理论论据,也会破坏资本主义的正式理论论据,因为这
也是基于静态的一般均衡模型,对个体创业行为没有任何明确的作用。

\textbf{从20世纪30年代到70年代,经济学界、非社会主义者和社会主义者都接受了社会主
  义理论上和实践上能够合理配置资源的核心论点,很少有例外。然而,到了20世纪的最
  后25年,这种观点开始改变。}一些人认为,认为社会主义制度下的企业管理者应该遵守旨
在有效运营工厂、将边际成本等同于边际收益的规则,这是相当幼稚的。这些批评者认
为,\textbf{如果没有竞争性私人财产社会中存在的惩罚,企业的管理者的行为将导致效率
  低下。}经济学界的其他人追随哈耶克的领导,哈耶克的观点出现在1937年和1945年出版的
两篇开创性文章中。\textbf{哈耶克指出,新古典主义模式假定消费者和生产者拥有完全的
  知识是错误的,事实上,这是市场和过程的功能之一。竞争的关键在于发现这种知识。规
  划者无法获得所需的知识,除非通过市场运作揭示出来的。}

20世纪80年代,Don Lavoie和Israel Kirzner再次提出了这些问题,他们认为哈耶克提出的
知识问题比新古典主义经济学先前所认为的更为严重。这一次,经济学家们对这一论点更为
开放,原因有两个:\textbf{第一,共产主义经济正在崩溃,部分原因是资源配置不足;第二,新古
典主义经济学已经演变成一种更加折衷的现代经济学},在这种经济学中,不存在早期新古典
主义经济学的必然性。

\subsection{社会主义和自由}

我们已经看到,到19世纪90年代,沙夫勒、布伦塔诺和纳赛尔已经公开质疑社会主义与自由
的相容性。20世纪,德国、意大利和苏联在两次世界大战期间的发展,迫使社会科学家注意
到这个问题。在讲英语的经济学家中,\textbf{哈耶克在《通往奴役之路》(1944年)}中再次提出了
这个问题。

在这本书和其他著作中,哈耶克坚持\textbf{社会主义与自由不相容}。经济计划不能简单地
存在,它需要一个具体的行动方案。因为规划者不能了解社会中每个人的偏好,从而他们必
须“将他们的偏好范围强加给他们计划的社区”。因此,\textbf{社会主义蓝图表明市场社
  会主义将允许计划经济中的消费者和职业选择自由,这是错误的。}哈耶克认为,计划和自
由选择是不相容的。

这可能反映了1948年大多数主流经济学家的态度,今天事后看来,伯格森以一种有趣的方式
回应了这一新策略:

\begin{quotation}
  不幸的是,似乎也没有可能提到最近社会主义计划和自由的争论中对其他基本问题讨论的
  贡献。鉴于俄国革命所展现的特殊情况,该国在所谓的规划和自由问题上的贡献可能不像
  人们所想的那么具有决定性。必须承认的是,社会主义批评者最近对这个问题的强调有时
  会出现一种战术策略,借此支持否定米塞斯理论。但是,必须首先考虑围绕计划和自由问
  题的争论;不提及它们,显然无法在社会主义上取得平衡。
\end{quotation}

经济与政治自由、社会主义与资本主义的关系问题不属于主流经济学的正常范畴,但也受到
了许多作家的追捧。许多经济学家支持计划与自由不相容,而资本主义与自由相容的论点。
最著名的是弗兰克·奈特、亨利·C·西蒙斯,最近的一位是米尔顿·弗里德曼和亨利·沃利奇。

即使那些同情社会主义思想的作家也对失败表示了担忧。即使是同情社会主义思想的作家也
对马克思主义社会主义政府不允许政治自由表示关切。\textbf{罗伯特·海尔布隆纳}认为
\begin{quotation}
  \textbf{除了稍纵即逝的国家之外,任何宣称自己基本上是反资本主义的国家,也就是说
    在自封的“马克思主义”社会主义范围内,民主自由尚未出现。所有这些国家的趋势都
    是走向限制性的,通常是压制性的政府,这些政府系统地压缩或消除了政治和公民自由。}
\end{quotation}

\raggedbottom
\clearpage
\subsection{实践中的社会主义资源配置}

任何持续的经济必须分配其资源。奥地利经济学家的观点是,在实践中,一个社会主义经济
体\textbf{不可能确定一套能够有效分配资源的价格体系}。这有助于考虑俄罗斯在制定价格
时所遵循的道路,首先是看到社会主义定价的困难,其次是在什么情况下计划经济可以有效
地分配资源。

\textbf{俄国革命后,实用主义压倒了理论上的考虑。制定了五年计划,注重实现实物产出
  和实物产出指标。对价格和效率问题几乎不关心。这些计划是通过试错的物料平衡而达到
  平衡的},在物料平衡中,所需的产品被放在分类帐的一边,生产这些产品的资源被放在另
一边,使之遇到了可预测的问题。例如,当农民没有得到足够高的价格时,他们拒绝将他们
的产出交给政府,以便政府向城市部门供应食品和材料。作为回应,俄罗斯政府决定将农场
集体化,从而“保证”农产品的供应。集团化进一步加剧了激励和效率问题。该计划的重点
是投资商品,而不是消费品,这意味着许多零售公司的货架上空无一人,工人们几乎没有工
作的动力。商店里的货物很少是人们想要的货物;工业通过生产货物而不是满足需求来满足
他们的配额。此外,技术变革的激励因素很少,因此俄罗斯的制造技术远远落后于西方技
术。

一些苏联经济学家认识到了苏联物质平衡计划体系的效率低下。他们是走向改革派和开放派
运动的知识分子之父。为了理解马克思主义对现代经济思想的贡献,有必要了解马克思主义
理论为什么抑制了经济规划。

\begin{figure}[ht]
\begin{mybox}{致命的自负}
  \begin{multicols}{2}
    当西方经济学家在20世纪90年代初进入曾经的社会主义国家时时,一个前社会主义经济
    学家最想研究的经济学家之一是F. A·哈耶克,特别是他1944年《通往奴役之路》
    和1988年《致命的自负》中的思想,副标题为“社会主义的错误”,他们认为这些书抓
    住了这个问题。他们的国家经历过。在这些书中,哈耶克将市场与政治自由联系起来。

    前苏联最近的变化似乎承认,要达到可容忍的经济效率,就需要比以前更大的政治自由。

    It is clear that recognizing the issue of the relationship of various
    economic systems to freedom is important in assessing the performance and
    acceptability of competing institutional arrangements. There are broad
    philosophical issues of interest and importance as well as more technical
    questions of interest largely to economists. One specific issue concerns
    freedom, not merely as an end, in itself, but as a means to an end—in
    particular, how much freedom is necessary to achieve the economic goals of
    efficiency and reasonable growth. Hayek’s early recognition of this
    connection is a tribute to his understanding of economic systems.
  \end{multicols}
\end{mybox}
\end{figure}

\begin{description}
\item[劳动理论与计划] 我们已经看到用劳动价值论来确定市场价格的困难。起初,计划者
  试图使他们的计划与劳动价值论保持一致。随着时间的推移,情况发生了变化。对劳动价
  值论的攻击并不存在一种广义上的推动力,而是为解决计划中日常问题的副产品。获得过
  诺贝尔奖的坎托罗维奇,诺维茨洛夫在1939年发表的论文,赫鲁晓夫在60年代初批准对这
  些问题的更充分讨论,时间差可以证明\textbf{意识形态的力量和苏联体制的威权性质}。
  这些人是第一批含蓄地质疑劳动价值论的人。

\item[影子价格] \textbf{坎托罗维奇},一个受过训练的数学家,被要求帮助解决胶合板行
  业的一个调度问题。早在很久以前,苏联的数学家就已经开发出一些可以应用于工业的技
  术。然而,由于坎托罗维奇面临的特殊问题与现有技术不相适应,他提出了一种新的解决
  方法。因此,他成为\textbf{线性规划的创始人},这是1947年在美国独立发现的一种技
  术。

  在线性规划问题的求解中,导出了一些所谓的乘数。虽然坎托罗维奇没有立即意识到它们
  的重要性和含义,但他对线性规划的应用和经济意义的进一步研究表明,它们在经济规划
  中的作用是显而易见的。这些乘数被经济学家称为\textbf{影子价格,它们反映了商品的
    稀缺价值。}

  许多苏联经济学家很快就清楚地认识到,把影子价格作为价值指标的规划者,将比使用由
  计划委员会制定的、源自某种意识形态和权利性的价格的规划者,实现更有效的资源配置。
  其他人同样迅速地发现,线性规划产生的影子价格意味着\textbf{相对价格不仅是劳动力
    时间的函数,而且还取决于资本和土地的稀缺性价值。因此,影子价格的使用是对劳动
    价值论的一个明显而根本的攻击。}

\item[机会成本] 另一个反对马克思正统运动的拔钉钳也开始试图解决有限的实际规划问题。
  假设一个计划委员会必须在几个投资备选方案中进行选择。它应该拨出资金(资本)来建
  设水力发电厂、炼钢厂还是机床厂?即使是在像苏联这样的经济体中,一个不考虑利息的
  劳动价值论也无助于解决这一日常问题。这是一系列问题的一个例子,只有承认资本的生
  产力和稀缺价值才能解决这些问题。

  这些问题在20世纪30年代末引起了经济学家\textbf{诺沃齐洛夫}的注意,并导致他撰写了
  一系列论文。他对合理计算问题的解决,虽然细节复杂,但轮廓清晰,主旨明确。他建议
  用经济学家所说的\textbf{机会成本来衡量价值或价格,从而不仅考虑劳动力成本,也考
    虑资本和土地成本。}他用劳动单位表达了机会成本的概念,给人留下了留在马克思正统
  派帐篷里的印象。

  在后斯大林主义时期,对坎托罗维奇和诺沃齐洛夫的提议进行了相对自由和公开的讨论,
  并且开始沸腾起来。其他人研究了这些关于资源分配的早期讨论,
  并\textbf{在20世纪60年代对苏联的经济规划进行了批判性的审验。}经济学家Evsei
  Gregorevich Liberman(1897-1983)建议给国有企业更大的决策自由度,减少国家规划者
  分配给国有企业的生产目标数量;这是一个更大的权力下放的强烈建议。自由人还主张停
  止根据产出向企业发放奖金,因为更多不需要或劣质产品的生产是浪费的。相反,他建议
  奖金应以公司的盈利能力为基础。

  慢慢地,潮流开始转向,见证了后斯大林时期、赫鲁晓夫时代和戈尔巴乔夫统治下的苏联
  解体。随着这些事件的发生,东欧许多国家发生了深刻的变化。
\end{description}

\begin{figure}[ht]
  \centering
  \begin{mybox}{经济体制趋同}
    \begin{multicols}{2}
      前苏联市场机制的进一步使用,以及政府对西方经济体宏观和微观部分的更大控制的
      运动导致了对这些进程可能产生的结果的猜测。有人提出,\textbf{所有的社会都是
        务实的,他们将抛弃那些不受欢迎的制度部分。因此,前苏联将变得更像我们,因
        为他们利用市场和激励手段来实现更高的效率,我们将使用更多的计划来消除我们
        经济的主要缺陷——不能充分利用资源。}

      一些代表广泛政治意识形态和专业培训的作家得出结论,\textbf{资本主义和社会主
        义的融合}将发生。仅举几个例子,埃里希·弗洛姆、阿诺德·J·汤因比、罗伯特·海
      尔布隆纳和简·廷伯根。

      在某些方面,他们肯定是对的。直到20世纪70年代,大多数西方经济体都在向其经济
      体引入更多的计划。然而,在20世纪70年代,情况发生了变化,西方经济体似乎有意
      限制政府的参与。20世纪80年代和90年代,随着许多以前的社会主义经济体的垮台,
      以及它们试图引入市场经济的诱惑,这种趋同似乎更多地发生在\textbf{市场}方面。
      所以最近出现了\textbf{不对称收敛}。这种趋势是否会继续,或者是否会发展出某种
      新的经济组织形式,仍然是一个悬而未决的问题。
    \end{multicols}
  \end{mybox}
\end{figure}

\subsection{计划与经济理论:评价}

随着20世纪90年代许多社会主义经济体的垮台,以及它们将市场引入他们的经济体的尝试,从
这个历史的角度回顾和思考关于社会主义经济计划的辩论是有帮助的。实际上,主流经济学
家似乎是错的,米塞斯和哈耶克是对的。社会主义国家的规划并没有导致像有效分配资源这
样的结果。这并没有导致大多数人所说的个人自由的增加或分配平等。社会主义国家的大多
数人都认为共产党只是一个压迫者。

目前尚不清楚的是,这种失败是社会主义特有的,还是出于其它原因。根据历史作出判断是
困难的。然而,我们认为,包括知识获取、自由和企业家在内的非均衡调整过程是理解经济
的重要组成部分,而前社会主义国家发生的事件应使主流更仔细地考虑这种非均衡过程的
必要性。


\begin{figure}[ht]
\begin{mybox}{熊彼特对资本主义、社会主义以及民主的看法}
  \begin{multicols}{2}
    约瑟夫·阿洛伊斯·熊彼特在其1942年出版的最著名的著作《资本主义、社会主义与民主》
    中涉及了本章中我们所关注的很多主题。

    熊彼特出生在奥地利,在门格尔的学生维色和庞巴维克的教导下学习。他于1932年到美
    国,在1950年去世之前在哈佛大学任教。从他学习经济学第一年起,熊彼特就对经济学
    的更大问题怀有天生的倾向,对于增补已经认同的理论只表现出很小的兴趣。他也承认
    受到了马克思的深刻影响。他钦佩马克思的学术造诣,试图弄清楚马克思对资本主义的
    发展上说了些什么。不过,在他自己的著作中,熊彼特坚决否定了马克思的分析中他看
    做意识形态的因素。在政治上,熊彼特是一位\textbf{保守派};所以,马克思是以鄙视
    的态度看待演进的资本主义,而熊彼特则歌颂它,并对其最终的灭亡感到悲哀,在这一
    点上与他想像的马克思一样。

    在熊彼特对资本主义的分析中,他运用了一些有关经济增长的早期观点(参见本书
    第15章中的“熊彼特与增长")。尽管他在意识形态上是保守的,但是,\textbf{熊彼特
      预言了由于资本主义的成功,资本主义最终向社会主义过渡。}他认为,由于成功的大
    企业中谨慎管理者的出现,\textbf{企业家精神消失了},并且,得到高度生产性资本主
    义制度支持的知识分子,开始与使其非生产性生活变得可能的制度相敌对,资本主义的
    动力将减弱,\textbf{不断增加的政府干预与政府所有将变成规范}。具有讽刺意味的是,
    社会主义之所以代替资本主义,不是像马克思所设想的那样因为资本主义的溃败,而是
    因为它的\textbf{成功}。随着资本主义的灭亡,导致高经济增长率的动力也将消失。
  \end{multicols}
\end{mybox}
\end{figure}

\section{总结}

\textbf{除非在特定时间的背景下,否则很难界定社会主义和资本主义。}社会主义和资本主
义的理论定义随着时间的推移而发生变化,现有的制度也是如此。评价将社会组织起来的可
供选择的方式优点时,其中一个难点是\textbf{纯粹理论上的制度与现有实际制度之间的分
  离}。某一特定制度的倡导者,倾向于将他们所喜欢的完美的纯粹理论制度,与他们所否定
的有缺陷的实际制度相比较。

随着民主的发展,资本主义包含两个元素,使其具有动态性和稳定性:企业家赋予其变革和
成长;民主促进了资本主义制度结构的变化,同时又没有破坏市场的基本制度。新古典经济思
想解释了市场如何在私有财产制度中运作,因此也就是资本主义经济思想。

社会主义经济思想部分地是对资本主义社会“失败”的一种回应。大多数社会主义者——乌托
邦者、马克思主义者等——集中于对资本主义的缺陷进行分析,很少谈论他们怎样预期社会主
义社会将如何组织经济。十九世纪初的一些作家提出了两个关于今天仍然重要的社会主义问
题:社会主义能否合理地分配资源?社会主义和自由是否兼容?

直到20世纪20年代初,当路德维希·冯·米塞斯断言社会主义无法有效地分配资源时,这些问
题仍一直处于休眠状态。他的指控引发了今天仍在继续的辩论。他的学生弗里德里希·冯·哈
耶克支持米塞斯关于资源分配的主张,并指责社会主义与经济和政治自由不相容。持续的辩
论使我们更好地理解了理论体系的优点和缺点,并且也展示了新古典微观经济模型的一些优
点和缺点。关于社会主义的辩论主要是在新古典理论框架内的理论层面进行的,并且在广泛
的层面上进行,其中有来自整个社会科学和历史的论据。

基础广泛的辩论远没有那么具有决定性,而且至今仍在继续。有人认为,只有社会主义才有
可能实现自由,而其他人则认为资本主义和自由之间存在历史和理论关系。

%%% Local Variables:
%%% mode: latex
%%% TeX-master: "../../main"
%%% End:
