\chapter{向新古典经济学的过渡:扩展的边际分析}

第一代边际理论家杰文斯、门格尔以及瓦尔拉斯,通过提出边际分析改变了经济学方
法。……更为显著的是,它结束了约翰·斯图亚特·穆勒的古典经济学。但是第一代理论家只
能部分地想象出这种工具的有用性。他们不能理解其发现的全部分量:相比古典学派,他
们强调理论内容方面的区别,远甚于强调方法论的区别。

因为早期边际主义者如此着力强调他们与李嘉图劳动价值理论结论的区别,所以他们未能认
识到自己与李嘉图构建的抽象模型之间的密切关系。李嘉图在解释地租的决定力量时,也运
用了边际分析。通过运用数学工具,边际分析家们将边际主义逐渐应用于微观经济学理论的
各个部分。杰文斯和瓦尔拉斯两人都受过数学训练,但门格尔没有。第二代边际理论家,出
了门格尔的奥地利弟子之外,都运用微积分对经济理论的前沿问题进行研究。

由第一代边际理论家引起的潮流一直持续到现在。使用令人难忘的数学方法,发展高度抽象
的模型,这是今天的常态。这些发展受到一些人的地址,尤其是阿尔弗雷德·马歇尔、德国
与英国历史学派、美国制度主义者、新奥地利经济学家、激进经济学家,以及很多不被划分
为主流的经济学家。

\section{扩展的边际分析:第二代}

杰文斯、门格尔、瓦尔拉斯的理论内容在很多方面还有不完善之处。他们将边际分析几乎专
门用于需求理论,差不多完全忽视了供给理论。杰文斯和门格尔很少注意供给,原因是他们
为价值几乎唯一地取决于边际效用的观念所困扰。瓦尔拉斯也没有明显地专注于供给方面,
因为他在一般均衡模型集中于经济变量的相关性。

他们的模型绝大部分都假定供给是既定的,并且资源配置问题仅仅是在可替代的用途之间分
配固定供给的问题。更具体地说,他们没有解释当生产要素供给不固定时这些要素价格的决
定力量;没有解释收入分配的决定力量;没有对厂商经济学进行有意义的分析;没有洞察到
一些独特的问题,这些问题是在发展理论的过程中为了解释工资、地租、利润以及利息而必
须加以解决的。

实际上,边际分析已经被两位较早期的经济学家应用于要素定价和收入分配中,然而像戈森
一样,他们的成就在很大程度上被同时代的人忽视了。蒙迪福特·兰格菲尔德在《政治经济学
讲义》(1834)中批评了劳动价值理论,提出了一种边际生产力分配理论。但他并不为杰文
斯、门格尔、瓦尔拉斯以及马歇尔所知,他的著作1903年才通过E.R.A·塞利格曼的介绍而引
起经济学界人士的注意。尽管汉.杜能对微观经济学问题具有比较重要的见解,但是,阿尔弗
雷德.马歇尔似乎是早期唯一一个发现冯·杜能边际生产力巨大影响力的人。

实际上,冯.杜能看来是将微积分应用于经济理论中的第一人。他不仅能够形成不同要素边际
产品的概念,而且提出了一个基于这些原理的相当正确的分配理论。在冯·杜能用了将近二十
年的努力来应对“所有的经济力量决定要素价格”这一简单陈述所表达的问题后,他非
常满意自己的最终结果,所以,他要求在他的墓碑上刻上他关于劳动工资的公式。但不幸的是,
他的成就对后来的经济思想几乎没有直接的影响,虽然马歇尔大方地向冯·杜能致谢。

第二代边际主义者凭借一种广泛应用于需求和供给理论的新工具,使经济理论得以复苏。然
而,这一工具几乎专门用来分析需求方面,尤其是家庭理论,很少用来分析供给理论或者厂
商理论。

来自奥地利、英国、瑞典以及美国的经济学家们,都对这一理论主体有显著的贡献,这不仅
表明这些成就代表了众多学者的共同努力,而且表明经济学作为一种学术上的努力,正变得
日益专业化。

\section{边际生产力理论}

收益递减原理在现代经济理论中发挥着根本性的作用。在微观经济理论中,它解释了厂商短
期供给曲线的形状,以及厂商的生产要素需求曲线形状。

李嘉图研究了今天所谓的农业生产函数,即土地的物质投入与物质产出之间的关系。他假定
生产过程中资本对劳动的比率因可利用的技术而固定,并且资本与劳动的组合以这种技术上
不变的比例添加到数量固定的土地上。在这些假设的基础上,他断定,对于连续的资本与劳
动组合来说,所形成的产量将呈现边际产品递减的特征。

经济分析史中的异常之一是,从李嘉图将边际生产力分析应用于地租的决定,到边际生产力
分析普遍应用于所有生产要素,花了将近七十五年的时间。一个类似的异常是,李嘉图所形
成的用于供给方面的边际分析,在19世纪70年代经历了首次重大扩展,当时,它不仅被用于
分析边际生产力,而且被用于分析边际效用。第二代边际主义者,最终挖掘出了为人所知的
收入的边际生产力理论的基础。这些经济学家中最重要的是奥地利学者弗里德里希:冯·维色
和欧根·冯·庞巴维克;美国学者约翰·贝茨。克拉克(John Bates Clark,1847一1938);瑞
典学者纳特·维克塞尔(Knut Wieksell,1851一1926);以及英国经济学家菲利普·亨利·维
克斯蒂德(Philip Henry Wicksteed,1844—1927)和弗朗西斯·Y·埃奇沃斯。这些经济学家,
连同杰文斯、门格尔、瓦尔拉斯以及马软尔,都是这一时期正统经济理论的智力伟人。他们
的首部重要著作都出现在1871年至1893年期间。

\subsection{收益递减原理}

如果保持一种生产要素不变,增加一种可变要素到这种不变要素上,因此形成的产量经常是
一开始增加速度递增,然后增加速度递减,最终递减。表9.1的例子显示了物质投入
与物质产出之闻的这种关系。

% Please add the following required packages to your document preamble:
% \usepackage{booktabs}
% \usepackage{graphicx}
\begin{table}[htbp]
  \centering
  \caption{生产函数}
  \label{tab:productfunc}
  \footnotesize
  \begin{tabularx}{\linewidth}{@{}XXXX@{}}
    \toprule
    劳动 & 劳动的总产品\newline(谷物吨数) & 劳动的平均产品\newline(谷物吨数) & 劳动的边际产品\newline(谷物吨数) \\ \midrule
    0 & 0 & 0 & 0 \\
    1 & 10 & 10.0 & 10 \\
    2 & 21 & 10.5 & 11 \\
    3 & 33 & 11.0 & 12 \\
    4 & 46 & 11.5 & 13 \\
    5 & 58 & 11.6 & 12 \\
    6 & 68 & 11.3 & 10 \\
    7 & 75 & 10.7 &  7\\
    8 & 80 & 10.0 &  5\\
    9 & 83 &  9.2 &  3\\
    10 & 83 &  8.3 &  0\\
    11 & 80 &  7.3 &  -3\\ \bottomrule
  \end{tabularx}%
\end{table}

如果土地数量保持不变,例如为100英亩,每年使用1个人,那么将会发现总产品为10吨谷物。
然后用每年使用2个人来重复试验,记录一个21吨的谷物产量,等等。注
意,\cref{tab:productfunc}中劳动总产品一列的数据,被假定为在给定的不变投入和可变
投入数量下,能够被生产出的最大量的产品。简言之,它假定实现了最大化的技术效率。此
外,当记录这些“投入--产出”关系时,假定技术水平保持不变。
\begin{figure}[ht]
  \centering
  \begin{tikzpicture}
    \datavisualization [school book axes,
    visualize as smooth line/.list={tpl, ap, mpl},
    style sheet=strong colors,
    % style sheet=vary dashing,
    % legend=north west inside,
    tpl={pin in data={text=$TP_L$, pos = 0.8}},
    ap={pin in data={text=$AP_L$, pos = 0.7}},
    mpl={pin in data={text=$MP_L$,  pos =0.55, }},
    y axis={length=7cm,ticks={step=10},label=总产品}, x axis={label=劳动的数量,length = 10cm},
    ]
    data [set=tpl] {
      x, y
      0, 0
      1, 10
      2, 21
      3, 33
      4, 46
      5, 58
      6, 68
      7,75
      8,80
      9,83
      10,83
      11,80
    }

    data [set=ap] {
      x, y
      0, 0
      1, 10
      2, 10.5
      3, 11
      4, 11.5
      5, 11.6
      6, 11.3
      7,10.7
      8,10.0
      9,9.2
      10,8.3
      11,7.3
    }

    data [set=mpl] {
      x, y
      0, 0
      1, 10
      2, 11
      3, 12
      4, 13
      5, 12
      6, 10
      7,7
      8,5
      9,3
      10,0
      11,-3
    }

    info {
      \draw [blue,dashed,very thick] (visualization cs: x=4.1, y=47)  -- (visualization cs: x=4.1,y=0) node [below = 12pt] {$L_1$};
      \draw [red,dashed,very thick] (visualization cs: x=5.3, y=61)  -- (visualization cs: x=5.3,y=0) node [below = 12pt] {$L_2$};
      \draw [dashed,very thick] (visualization cs: x=10, y=83)  -- (visualization cs: x=10,y=0) node [below = 12pt] {$L_3$};
    }

    ;
  \end{tikzpicture}
  \caption{\label{fig:productfunc}劳动的总产品、平均产品、边际产品}
\end{figure}

\cref{tab:productfunc}从数字上显示了可变投入劳动的平均产品或边际产
品,\cref{fig:productfunc}从图形上将它们呈现出来。劳动的平均产品是用总产品除以劳
动数量计算出的。劳动的边际产品通常更为准确地被称作劳动的边际物质产品,它被定义
为:
\[MPP_L=\frac{\Delta TP}{\Delta L}\]

在几何上,它是总产品曲线的斜率,或者说总产品对劳动的一阶导数。当劳动的数量
为$L_1$时,劳动的边际产品达到最大值,当劳动的数量为$L_2$ 时,劳动的平均产品达到最
大值,并且边际产品与平均产品相等;当劳动的数量为$L_3$时,总产品达到最大值,并且劳
动的边际产品为零。劳动的数量超过$q_3$,将引起总产品下降,边际产品为负数。

19世纪的最后几年间,生产函数的准确性质及其含义才慢慢挖掘出来,描绘并计算任何生产
要素的边际产品成为可能。例如,我们可以保持劳动数量固定,从而得出土地的边际产品曲
线。

\subsection{新的与旧的}

随着对生产中各种关系的更多理解,人们逐渐认识到,假定完全竞争行业中的某个厂商只使
用劳动这种可变的生产要素,生产要素的需求曲线能够从边际产品曲线中得出。厂商在完全
竞争的市场上销售其最终产品,因而最终产品的价格不随着厂商销售量而变化。换句话说,
厂商面对的是对其最终产品完全有弹性的需求曲线。厂商在完全竞争的市场上购买可变投入;
所以对厂商来说,投入的价格不随着购买的数量而改变。也就是说,厂商面对的是对可变投
入完全有弹性的供给曲线。最理想的状态下,厂商将把可变投入固定到某一点上,在这一点
上,最后一单位所购买的投入为厂商的总收益所增加的数量,等于厂商的总成本所增加的数
量。这个条件能被表述如下:
\[劳动的价格=劳动的边际物质产品 \cdot 产出的价格\]

等式左侧度量的是由于另雇用了一单位劳动而增加的总成本。等式右侧度量的是由于销售劳
动所生产的产品而增加的总收益,它一般被称为边际产品价值。

考虑到\cref{tab:productfunc}中给出的数据,假定劳动的价格为每人每年10000美元,最终
产品的价格为每吨1000美元。如果本例中的厂商雇用5单位的劳动,那么,最佳的劳动雇用等
式将会给出下列数值:
\begin{gather*}
  P_L=MPP_L \cdot P_o\\
  \$ 10000 < 12 \cdot \$1000\\
  \$ 10000 < \$ 12000
\end{gather*}

所雇用的最后一单位劳动增加了10000美元的总成本和12000美元的总收益;因此,利润增加
了2000美元。对利润最大化感兴趣的厂商,将增加对可变投入劳动的使用。随着这样做,劳
动的边际物质产品下降。所雇用的第六单位劳动增加了10000美元的总成本和10000美元的总
收益。第七单位劳动增加了10000美元的总成本,但只增加了7000美元的总收益。最佳的劳动
量是6单位,原因在于,劳动的价格等于劳动的边际产品价值。

然而,由于大多数生产过程涉及多种投入,所以,需要更一般化的投入品最佳使用原则。假
设我们有若干投入,$A, B, C , \cdots N$……当下列条件成立时,这些投入就以一种最佳
的方式被使用着:

\[\frac{MPP_A}{P_A} = \frac{MPP_B}{P_B} = \frac{MPP_C}{P_C} = \cdots = \frac{MPP_N}{P_N}\]

等式表明,当花在购买每种要素上的最后一美元产生了相等的边际物
质产品时,投入就得到了最佳使用。如果这一条件不能满足,就有可能改
变要素购买,用相同的总成本生产更多的最终产品,或者以较低的总成本
生产既定的最终产品,这是同样的事情。

现在能够很容易得出对一种投入的需求。对一种投入的需求被界定为厂商在不同价格下所雇
用的数量。假定我们从一家正最佳使用投入的厂商开始;也就是说边际物质产品与投入价格
的比率相等。如果我们降低一种投入的价格那么,厂商将会使用更多的这种投入,直至花在
这种投入上的最后一美元产生的边际物质产品,与花在所有其他投入上最后一美元产生的边
际物质产品相等。边际生产力理论也表明,当竞争性市场的厂商最佳地使用其投入时,所有
投入的价格都将等于其边际产品价值。

就像一些创始者所认同的那样,边际生产力的这些新观点与李嘉图的地租理论密切相联。分
析地租理论时,李嘉图假设资本与劳动就像是一种可变投入,按照由技术决定的固定比例,
运用到土地这种不变投入中,李嘉图借此把一个三种投入的模型变为两种投入的模型。为了
说明最新发展的边际生产力理论与李嘉图地租理论之间的密切关系,我们考虑只有劳动与土
地两种投入的模型。在这一模型中,李嘉图按照\cref{fig:richardoMP}所表示的方式度量了地租。

\begin{figure}[ht]
  \centering
  \subcaptionbox{\label{fig:richardoMPa}土地数量不变,劳动数量可变}{%
    \begin{tikzpicture}
      \draw (0,0) node[below left] {$O$};
      \draw[very thick] (0,5) -- node[left=12pt, text width =1em] {劳动的边际产品} ++(0,-5)  -- node[below=10pt] {劳动} ++ (6,0);

      \coordinate (a) at (0,4);
      \coordinate (m) at (4,1);
      \coordinate (b) at ($(a)!0.6!(m)$);
      \coordinate (d) at ($(a)!(b)!(0,0)$);
      \coordinate (c) at ($(0,0)!(b)!(6,0)$);

      \draw (a) node [left] {$A$} -- (m) node [below right] {$M$};
      \draw[dashed] (b) node [right] {$B$} -- (d) node [left] {$D$} node[pos=0.6, above = 0.1cm] {地租};
      \draw[dashed] (b) -- (c) node [below] {$C$} node [pos=0.5, left=18pt] {工资};

    \end{tikzpicture}%
  }\hfill
  \subcaptionbox{\label{fig:richardoMPb}劳动数量不变,土地数量可变}{%
    \begin{tikzpicture}
      \draw (0,0) node[below left] {$O$};
      \draw[very thick] (0,5) -- node[left=12pt, text width =1em] {土地的边际产品} ++(0,-5)  -- node[below=10pt] {土地} ++ (6,0);

      \coordinate (a) at (0,4);
      \coordinate (m) at (4,1);
      \coordinate (b) at ($(a)!0.6!(m)$);
      \coordinate (d) at ($(a)!(b)!(0,0)$);
      \coordinate (c) at ($(0,0)!(b)!(6,0)$);

      \draw (a) node [left] {$F$} -- (m) node [below right] {$N$};
      \draw[dashed] (b) node [right] {$G$} -- (d) node [left] {$I$} node[pos=0.6, above = 0.1cm] {工资};
      \draw[dashed] (b) -- (c) node [below] {$H$} node [pos=0.5, left=18pt] {地租};

    \end{tikzpicture}%
  }
  \caption{\label{fig:richardoMP}李嘉图地租理论的劳动与土地边际生产力模型}
\end{figure}


在\cref{fig:richardoMPa}中,曲线 $ABM$ 表示劳动的边际物质产品。如果使用等
于 $OC$的劳动数量,那么,总产品就是面积$OABC$即边际产品的总和。然而,李嘉图并不集中
于边际产品,虽然他假设边际产品递减。他集中研究地租的决定。他推断地租等于面积$ABD$。
每个劳动者得到的工资为$OD=BC$,总工资支出为面积$ODBC$。从总产品中减去总工资就得到剩余
的$ABD$,它就成了不变生产要素土地的地租。

在\cref{fig:richardoMPb}中,曲线$FGN$度量了土地的边际物质产品。总产品等于面
积$OFGH$,每一单位土地获得的地租为$OI=HG$,总地租为$OIGH$。现在,工资就用不变要素
劳动的剩余增加量来度量,它等于$FGI$。

因此,新的边际生产力理论的一个结果,就是适应了李嘉图的地租理论并使之一般化。李嘉
图所强调的不是可变投入的边际产品,而是不变要素的剩余增加量。然而,新的理论集中于
可变投入的边际产品。李嘉图内将边际生产力分析应用于地租的决定,而新的理论家认识到
任何一种投入都能够改变,并且它们的边际产品能够被计算出来。他们也看到,厂商将继续
投入,直至投入的价格等于可变投入的边际产品价值。这些新的观点引发了这一时期广为争
论的很多问题。

\subsection{产品用尽}

李嘉图的分配理论在下列意义上来说是一种剩余理论,即地租是将工资和利润从总产品中扣
除之后的剩余;利润是将工资(马尔萨斯人口学说所定义的工资)从工资与利润(李嘉图分
配理论中的工资与利润)之和中扣除之后的剩余(见第5章7.1小节“分配理论”)。根据剩
余分配理论,支付给不同生产要素的报酬等于总产品,这一点不存在问题,因为决定要素报
酬的方法保证了总产品被予以分配。

我们假定一个只有劳动与土地两种投人的简单经济体。利用李嘉图剩余理论来解释收入的分
配,我们的推理如下:\cref{fig:richardoMPa}表明经济体的总产品等于 $OABC$,劳动获得
的份额等于 $ODBC$,剩余的就是地租或者说总产品与总工资支付的差额。因为地租是作为一
种剩余计算出来的,所以,工资加上地租一定等于总产品。然而,边际生产力分配理论并不
能这么明显地得出这一结论。如果在竞争性市场上,每种要素获得其边际产品价值,那么,
有理由假定所有这些边际产品的总和将恰好等于总产品吗?

新近发展的边际生产力理论主张,每种要素将获得其边际产品。通
过\cref{fig:richardoMPa},我们断定劳动的边际物质产品为$BC$,总工资支出等于所使用
的劳动数量 $OC$乘以劳动的边际产品,从而得到面
积 $ODBC$。在\cref{fig:richardoMPb}中,土地的边际物质产品为 $GH$,总地租为土地的
边际产品 $GH$ 乘以土地的数量 $OH$即面积 $OIGH$。如果两者都用边际产品方法来计算,
那么工资与地租的总和等于总产品吗?面积 $ODBC$(工资)加上面积 $OIGH$(地租)等于
面积 $OABC$(总产品)吗?换句话说,用边际产品方法计算的工资 $ODBC$ 等于用剩余方法
计算的工资 $IFG$ 吗?同样的问题也可以用于地租:$OIGH$等于 $DAB$ 吗?支付给生产要
素的报酬等于总产品的主张,可以用下面的等式形式来表达:
\[Q=MPP_L·L + MPP_T · T\]

这里的 $Q$为产出的物质数量(总产品),$MPP_L与MPP_T$为劳动与土地的边际物质产
品,$L与T$是劳动与土地的数量。

约翰·贝茨·克拉克声称,支付每种生产要素其边际产品,将正好用尽总产品,但是,他没有
为这一主张提供证据。19世纪90年代展开了对这个问题的争论,并一直持续到20世纪。其中
所涉及的最重要的经济学家有威克斯蒂德、维克塞尔、巴罗内(Barone)、埃奇沃斯、维尔
弗雷多·帕累托(Vilfredo Pareto,1848一1923)以及瓦尔拉斯。\footnote{对这个问题公
  认的最好总结,包含在乔治·斯蒂格勒的《生产与分配理论》第12章“欧拉定理与边际生产
  力理论”中。该书由美国麦克米兰出版公司于1941年出版。}我们将集中评论威克斯蒂德和
维克塞尔,他们的贡献极大地影响了边际生产力理论的发展。

1894年P.H·威克斯蒂德出版了一本题为《论分配法则之协调》的小册子,他在其中提出,古
典理论要求分别解释支付给土地、劳动、资本的报酬,在这点此是不完善的,而边际生产力
理论用一个统一的原理来解释任何生产要素的收益,从这点上说它是一种更好的理论。维克
斯蒂德断定在竞争性市场上,每种生产要素将拥有等于其边际产品价值的一个价格——他承认,
这就提出了如果所有的要素获得其边际产品、总产品是否被用尽的问题。他试图证明这一结
果,即所说的产品用尽将会发生。尽管他在这一尝试上未能成功,然而威克斯蒂德的确指出,
发生产品用尽必须存在竞争,并且,厂商的生产函数必须具有某些性质。在对威克斯蒂德
《论分配法则之协调》的评论中,A.W·弗拉克斯(A.W.Flux)也对这些发展有所贡献。他证
明只有当生产函数具有某些数学性质时,才会导致产品用尽。瑞士一位数学家莱昂哈德·欧拉
早先考察过这些性质,他的名字因此开始与涉及产品用尽的问题相关。

对于因支付给每种要素的报酬等于其边际产品,而使总产品恰好用尽来说,生产函数必须具
有如下性质,即所有投入增加既定的比例将使产量或总产品按相同的比例增加。在我们的例
子中,如果劳动与土地的数量增加一倍,那么,总产量增加一倍;如果两种投入增加两倍,
那么,总产量增加两倍,等等。适用于这些函数的数学习语是,它们是一次齐次
(homogeneous to the degree one)的。这些函数也被描述为“线性齐次的”,尽管这样
一种描述可能会误导不是数学家的人,因为它们不一定是线性的。小于一次齐次的生产函数
会引起一种状况,即所有投入增加一定比例,引起产量以较低的比例增加。如果生产函数是
大于一次齐次的生产函数,那么所有投入增加一定的比例,将引起产量以较高的比例增加。

经济学家使用规模收益这一习语来描述产量或成本对所有投入成比例增加的反应方式。如果
所有投入都成比例增加,并且总产量也增加了相同的比例,那么,平均成本不变,这种结果
被称为规模收益不变。规模收益不变由一次齐次生产函数得出。如果所有投入都成比例增加,
并且总产量增加了一个较小的比例,就存在规模收益递减,平均成本增加。规模收益递减由
小于一次齐次的生产函数得出。

在完全竞争市场上销售其产出并购买其投入的厂商,如果拥有能够产生规模收益不变的生产
函数,将会发现,如果所有的投入按照其边际产品价值来支付,那么,厂商的总收益将完全
被这些支付用尽。要素市场上的竞争将导致每种投入获得其边际产品价值,最终产品市场上
的竞争,其结果是厂商获得零利润。如果获得零利润,那么,厂商的总收益必定等于总成本;
因为总成本是给各种投入的支付,所以,就发生了产品用尽。

一个简单的代数表达式能阐明这个问题。所陈述并在这一时期被讨论的问题是,为每种投入
支付其边际产品是否用尽了厂商的总产品。对于一个简单的劳动与土地生产函数,我们先前
用等式的形式来表述这个问题:

\begin{gather*}
  Q=MPP_L \cdot L + MPP_T \cdot T
  \shortintertext{乘以最终产品的价格后}
  PQ = P \cdot MPP_L \cdot L +P \cdot MPP_T \cdot T\\
  现在,P \cdot MPP_L \cdot L = 劳动的边际产品价值(VMP_L),\\并且P \cdot MPP_T=
  土地的边际产品价值(VMP_T)。因此\\
  PQ =VMP_L \cdot L + VMP_T \cdot T
\end{gather*}

等式右侧表示支付给劳动的总报酬与支付给土地的总报酬。因此,它代表了厂商的总成本。
等式左侧表示厂商的总收益。在完全竞争下,所有投入获得其边际产品价值,并且利润为零,
这意味着厂商的总收益将等于总成本。于是,支付给生产要素的报酬用尽了厂商的总收益。

大于一次的齐次生产函数引起规模收益递增和平均成本下降。这意味着边际成本一定低于平
均成本,且一种投入的边际物质产品将超过那种投入的平均产品。如果投入是在竞争性市场
上购买的,那么,厂商必定要为每种投入支付其边际产品价值。但是,如果所有的投入获得
其边际产品价值,那么,厂商的总收益将会低于支付给所有投入的报酬。从成本或者从产出
的角度处理这个问题,都能证明这一结果。如果经历平均成本递减的厂商,其行为是竞争性
的,即在等于边际成本的价格上销售其产品,那么,这种操作是亏本的;也就是说,总成本
将会超过总收益。类似地,如果投入的边际物质产品超过了其平均产品,并且投入获得等于
其边际产品的报酬,那么,支付给投入的报酬将超过总产量,厂商将亏损。

小于一次的齐次生产函数引起规模收益递减或者平均成本递增,在此,边际成本高于平均成
本,一种投入的边际物质产品将低于那种投入的平均产品。具有竞争性行为的厂商,将使边
际成本等于价格,在那一产量上将会获得利润。这就暗示了,当所有要素获得其边际产品价
值时,支付给投入的报酬将少于总产量。在这些情况下,总收益超过总成本,厂商将获得利
润。

\subsection{维克赛尔论产品用尽}

纳特:维克塞尔是一位瑞典经济学家,为宏观与微观经济理论做出了很多重要的贡献,他也
是边际生产力理论的早期独立发现者。他对于欧拉定理和产品用尽相关的问题感兴趣,与他
所处时代的任何其他经济学家相比,维克塞尔为解决这些问题做出了更大的贡献。在他关于
这一主题的较早作品中,像大多数其他经济学家一样,他认为一个既定厂商或行业将呈现规
模收益递增、不变或者递减。这些类型似乎是相互排斥的。然而,1902年,维克塞尔得出了
一个完全不同的结论,即既定厂商可能会经历规模收益的所有三个阶段。

正在扩大产量的厂商将首先经历规模收益递增,但迟早会遭遇规模收益递减。在收益从递增
变为递减的产量水平上,一定会发生规模收益不变。维克塞尔明确地形成了厂商长期U形平均
成本曲线的概念,这一概念表明平均成本递减,然后达到一个最低点,最后递增。维克塞尔
认为,厂商的生产函数不一定是一次齐次的,从而发生产品用尽。如果厂商在长期平均成本
曲线的最低点所出现的产量水平上生产,并且利润为零,那么,会发生产品用尽。维克塞尔
推论完全竞争的市场将会导致这些结果,原因在于竞争将引起每个厂商在最小成本下生产,
并使利润为零。这样,尽管厂商的生产函数会形成收益递增、不变以及递减,然而,竞争将
会保证在长期均衡状态下,厂商在其生产函数的某一点上运转,在这一点上,存在收益不变;
在这一点上,函数是一次齐次的;在这一点上,平均成本是最小的。

维克塞尔在关于产品用尽问题的解决方案中,提出了新的引起注意的理论问题,经济学家们
始终追踪着这些问题,直至20世纪。维克塞尔提出了对长期平均成本曲线形状的一些解释,
但是,在20世纪30年代之前,这些问题一直没有被彻底弄明白。

\subsection{边际生产力理论的道德含义}

约翰·贝茨·克拉克独立发现并发展了边际效用与边际生产力观点。他对边际效用理论的发展
不如杰文斯、瓦尔拉斯或门格尔那样敏锐,但是,他对收入的边际生产力理论的贡献,等同
于第二代英国和欧洲经济学家。克拉克承认,他对边际生产力理论的发展,是对美国社会评
论家亨利·乔治所提出问题的回应。我们在第5章中了解到,享利·乔治断定,土地的收益是一
种不劳而获的收入,因此他质疑地租的合理性。乔治的主张促使克拉克试图确定由个别生产
要素产生的产品,以及边际生产力理论。约翰·贝茨.克拉克的儿子J.M·克拉克也是一位重要
的经济学家。

他的《财富的分配》包含了其边际生产力分配理论的实质,也包含了对竞争性市场所产生的
合意的道德后果的广泛发展。没有必要详细地展开克拉克对边际生产力理论的贡献。有关的
要点是他的以下结论,即在完全竞争市场中,每种生产要素将获得等于其边际产品价值的收
益。这一收益度量了一种要素对正在生产的特定产品的贡献,也度量了它对社会的贡献。资
本的收益通过资本是生产性的这一事实被证明是正当的;收益不是抢夺,而是诚实、公平以
及公正的。同样,土地的收益也不是一种不劳而获的收入,而是土地生产力的收益。同样的
结论也适用于劳动的收益。克拉克的结论是,完全竞争市场所产生的收入分配,是一种道德
上正确的分配,因为它根据生产要素对社会产品的经济贡献,对它们进行了奖赏。他主张,
剥削理论和不劳而获收入理论是天真的,原因在于,它们未能了解经济体中市场力量的运
作。

约翰·贝茨·克拉克对边际分析,尤其是对边际生产力理论的贡献,为他获得了全世界的认可。
他是第一位对经济理论做出重要贡献的美国经济学家,这样认为是公平的。然而,他从边际
生产力理论中得出的道德结论,与他对实证理论的贡献相比,吸引了更多的批判性注意。这
么认为或许更公平些,因为克拉克把他的道德结论视为他最重要的贡献。

他的主张即竞争性市场导致道德上合意的收入分配到底具有多少价值呢?这一主张最重要的
问题是它违背了休谟格言:它从非道德的分析中得出道德含义。一个人“应当”获得的,可
能与他或她真正获得的没有什么联系。很多其他的问题也被提了出来。例如,即使给定完全
竞争市场的假设,也没有理由断定因为每种要素(factor)获得其边际产品价值,所以每个
个体(individual)就获得了一种收益,这种收益度量了他或她对经济体和社会的贡献。个
体的收入将取决于他或她在市场上销售的要素的价格,以及所销售的要素的数量。拥有资本
和土地的个体将获得这些来源的收入,但是,这些报酬代表了要素的贡献,而不是个体的贡
献。

克拉克道德结论的另一个难题是他对完全竞争市场的依赖。克拉克明白厂商与工会两者的垄
断力量,并试图将垄断力量的影响置于收入分配和他的道德结论中。他特别乐观的观点使他
将这些与竞争性市场的背离视为在数量上无关紧要。有趣的是,他最有才华的学生之一托尔
斯坦·凡勃仑也观察了的同样的经济体与社会,并且就它们的道德结果来说,得出了完全不同
的结论。

\subsection{作为一种就业理论的边际生产力理论}

尽管边际生产力分析在最初形成时,是为了解释生产要素的价格决定力量与收入分配,但是,
人们很快就认为,它也能被用来解释就业水平的决定力量。在一个局部均衡分析的框架中,
如果劳动的价格上升,厂商将雇用较少的劳动,直至劳动的边际产品价值等于较高的劳动价
格。雇用较少的劳动会引起劳动的边际物质产品增加,因此提高劳动的边际产品价值。在行
业水平上,劳动的价格取决于对劳动的需求和劳动的供给,对劳动的需求又得自于劳动的边
际产品价值。如果一个行业中,劳动的价格在均衡水平之上,那么,所供给的劳动数量将超
过所需求的劳动数量——存在劳动的过剩或者说失业。

当边际生产力分析家将这一理论扩展到整个经济体时,他们断定,超过摩擦性失业的失业是
由于盛行的工资高于均衡水平而导致的。劳动的超额供给,就像任何其他商品一样,是通过
供求分析得到解释的。考虑到这种分析和有弹性的工资率,市场系统会自动纠正失业,因为
工资会下降。失业是劳动市场非均衡的体现;当劳动市场恢复均衡时,失业就会消除。以边
际生产力的这一应用为基础,人们在不同时期得出了很多政策结论:工资应当保持弹性,任
何对弹性工资的妨碍,例如,工会合约或最低工资法,都是不合意的;工会与最低工资法会
导致失业;如果经济萧条产生了失业,那么,使工资表失弹性的制度因素,将阻止市场通过
降低工资而自动消除失业。

从边际生产力理论中得出的宏观经济政策结论是,通过允许工资下降,能够消除经济萧条与
失业。尽管一些经济学家基于社会理由,很不情愿地提倡降低工资来消除失业和治理经济萧
条,然而,正统理论得出这些结论是没有什么疑问的。在讨论这个问题时,阿尔文.汉森
(AlvinHansen,1887一1975)引用了亚瑟·C·庇古20世纪20年代的著作,这些著作描述了就
业与工资之间的这样一种关系。汉森说,他提名“庇古是20世纪经济学家普遍接受的思想中
最杰出的(此外,也是最有社会头脑的)代表;很多经济学家无数次地提到他(也包括我自
己较早著作中的段落),任何人都能容易地增加关于他的内容,只要不怕麻烦这样做”。这
些观点都是从边际生产力理论中产生的,在20世纪30年代中期,在受到约翰·梅纳德·凯恩斯
严厉批评之前,这些观点一直为正统理论家所主张。诺贝尔奖获得者经济学家约翰·R·希克
斯(John R.Hicks,1904一1989)在他的《工资理论》(Theory of Wages,1932)中,拿出
两章来论述工资规制与失业。希克斯断定,由于工会的压力或者由于立法而人为地使工资高
于竞争性均衡的工资,将导致失业,并且“失业必定会持续,直到人为确定的工资被放松,
或者直到竞争性工资上升到人为确定的水平”。

\begin{figure}[ht]
  \centering
  \begin{tikzpicture}
    \draw (0,0) node[below left] {$O$};
    \draw[very thick] (0,5) node[above] {价格} -- (0,0)  --  (6,0) node[right] {数量};

    \coordinate (d) at (0.75,4);
    \coordinate (d1) at (4,0.8);
    \draw (d) node [above left] { $D$} -- (d1) node[below right] { $D'$};

    \coordinate (s) at (4,4);
    \coordinate (s1) at (0.75,0.8);
    \draw (s) node [above right] { $S$} -- (s1) node[below left] { $S'$};

    \coordinate (X) at (intersection cs:first line={(d)--(d1)}, second line={(s)--(s1)});
    \draw [dashed] (X) -- (X -| 0,3) node [left] (we) {$W_e$};

    \coordinate (b) at ($(s)!0.2!(s1)$);
    \draw [dashed] (b) -- (b -| 0,3) node [left] (w1) {$W_1$};
    \draw [dashed] (b) -- (b |- 3,0) node [below] (q2) {$Q_2$};

    \coordinate (a) at (intersection cs:first line={(b)--(w1)}, second line={(d)--(d1)});
    \draw [dashed] (a) -- (a |- 3,0) node [below] (q1) {$Q_1$};

    \draw[ultra thick,decorate,decoration={brace,mirror},red] (q1.south)-- node[below] {失业} (q2.south) ;

  \end{tikzpicture}%
  \caption{\label{fig:mshiye}非均衡的劳动市场}
\end{figure}

我们能用\cref{fig:mshiye}中的简单供求分析来阐明这一推理。劳动的需求曲线 $DD'$ 为
劳动的边际产品价值。在工资上,被雇用的劳动数量为 $OQ_1$,。厂商将雇用劳动,一直到
所支付的工资等于劳动的边际产品价值的那一点上。在工资 $W_1$上,所提供的劳动数量
为 $OQ_2$ 。在工资 $W_1$上,劳动的超额供给等于 $Q_1Q_2$,它被称作失业。如果市场自
由运行,工资将下降到均衡水平 $W_e$上,失业将消失。因此,失业是(1)劳动市场暂时非均衡
的结果,或者(2)阻止工资下降的某些障碍的结果。

边际生产力理论加上其强烈的自由放任市场导向,使得美国的经济学家们在20世纪30年代早
期提出,减轻经济萧条的最佳政策是使政府置身于经济体之外,让市场运行,从而降低工资。
我们来简要地考察一下凯恩斯对把边际生产力理论作为一种就业理论的主要批评。边际生产
力理论声称,在竞争性市场上,工资将等于劳动的边际产品价值。劳动的边际产品价值是劳
动的边际物质产品乘以最终产品的价格。凯恩斯指出,从厂商的观点来看,工资是一种成本,
而从工人的角度来看,工资是收入。因此,工资率的削减,降低了厂商的成本,也降低了劳
动者的收入。当劳动收入开始下降时,最终产品的需求及其价格也将下降。最终产品价格的
下降将引起劳动的边际产品价值下降。边际生产力理论作为一种就业理论的难点在于,它假
定降低工资不会减少对最终产品的需求一一换句话说,总供给与总需求并不是互相连接的。
理论只集中于工资削减的成本方面,忽视了凯恩斯所说的总需求。

\subsection{受到批判的边际生产力理论}

对边际生产力的批评一直持续到现在。早期的批评包括对边际生产力一般理论的广泛抨击,
无论理论是应用于劳动、资本还是土地;批评也包括对特殊问题的具体论述,这些特殊问题
起因于将理论应用于利润与利息的决定时。下一部分我们将讨论这些特殊问题,现在,着眼
于对边际生产力理论最重要的早期批评,即度量一种生产要素的边际产品是不可能的。

评论家表示,厂商、行业或经济体的最终产量是劳动、土地、资本联合努力的结果,不可能
区分起作用的要素的边际产品。F.W·陶西格(F.W.Taussig,1859一1940)在哈佛大学经济
学系的早期发展中是一位发号施令式的人物,他在其有影响力的《经济学原理》中声称,在
使用资本与劳动的过程中,“工具在一方,使用工具的劳动者在另一方,不存在这样的分
离。……我们无法明确分离劳动的产品与资本的产品。”这一批评更大众化的说法包含在乔
治·伯纳德.肖(George Bernard Shaw)令人愉快的《聪明女人的社会主义和资本主义指南》
中。肖认为,尽管通过给予每个人所生产的产品来奖赏劳动,可能是合意的,然而,这是做
不到的:“当一个农场主与他的劳动者播种并收割了一块麦地时,世上没有人能说清他们中
的每个人种了多少小麦。”又如,假定木匠(劳动)使用铁锤(资本)在盖一座房子。如果增
加另一单位的木匠,其边际物质产品是什么?在任何生产过程中,增加劳动通常都要求同时
增加资本,这样就产生了区分新增劳动的边际产品与新增资本的边际产品的困难。马歇尔就
这个问题的解决方案是,通过从新增劳动与资本的边际产品价值中扣除资本成本来度量劳动
的净产品。约翰·贝茨·克拉克提供了另一种方案,他提出资本的数量保持不变,但是其形
式可以改变。然而,因为资本的形式只能随着时间而改变,所以,克拉克的方案就功际产品
的计算问题提出了一种长期观点。

\section{利润与利息}

边际生产力理论的一些早期发展者,尤其是庞巴维克觉察到,尽管边际生产力分析对于劳动
和土地收益来说,是一种令人满意的解释,但是,它未能解释为人所知的利润与利息收益问
题。回顾以往,我们能看到,与解释利润与利息性质和数量相关的问题,在边际生产力分析
形成之前,其至没有出现过。

古典经济理论通常将生产要素划分为劳动、土地、资本。劳动的收益是工资;土地的收益是
地租;资本的收益是利润。利润这一术语,就像古典经济学家所使用的那样,包括了今天所
谓的利润与利息。未能区分利润收益与利息收益是可以理解的,原因是当时的典型厂商结合
了资本家与企业家的角色。投资基金的供给者与管理者是同样的一个人,所以,没有对利润
与利息进行区分。在我们所研究的这一时期,一项成就是认识到需要对这两者进行区分。

我们利用边际生产力理论,不仅能够解释劳动的工资、土地的地租、资本的利息,而且能够
解释流向企业家的利润吗?当时的经济学家断定,边际生产力理论能够令人满意地解释工资
与地租,然而利润与利息所特有的问题则要求更加成熟的理论予以解释。

\subsection{利润理论}

尽管古典经济学家不加区别地将利润这一术语用于指资本家——企业家的全部收入,然而,他
们的确也意识到这种收入至少包含三种不同因素的报酬:对使用资本的报酬,给予企业家所
贡献的管理服务的报酬,以及补偿经济活动风险的报酬。支付给厂商的对使用资本的报酬
(假定这一报酬不涉及风险)被归入利息(interest)的现代分类中,我们将在紧接着的那
部分进行讨论。我们能将企业家才能确定为第四种生产要素,从而将企业家的边际产品界定
为他对厂商的管理服务贡献与风险承担贡献的度量吗?

约翰·贝茨·克拉克是边际生产力理论最重要的早期发展者,他认识到这种解决方案并不令人
满意。企业家作为管理者的收益并不是利润,而是工资。利润——或者更加准确地说纯利
润——一定被界定为,在厂商使用的全部投人被支付了等于其机会成本的价格之后残留的剩余。
长期均衡下的完全竞争市场,使所有要素获得其边际产品价值,亦即其机会成本。假定在一
个齐次生产函数中,这些支付就是厂商的成本,当从总收益中减去时,会产生一个零利润率。
因此,利润的存在一定被解释为,或者是竞争性市场没有达到长期均衡,或者实际市场不
是完全竞争市场的结果。

当然,长期竞争性均衡是一种理论上的构造,没有市场符合这一构造。然而,当我们分析出
现在非长期均衡的市场或经济体中的利润时,还是保留竞争性的假设。当厂商购买投入、生
产某种产出时,他们承担着风险。必须估计产量的最终价格,并且,投入的价格以及支付给
投入的报酬成了契约债务。如果厂商的总收入超过支付给投入的报酬就产生了利润;如果收
入小于支付的报酬就产生了亏损。于是,完全竞争市场中的利润可以被解释为当经济体向长
期均衡的新位置移动时所出现的非均衡结果。

克拉克、马歇尔、熊彼特提出,应将利润解释为由经济体动态变化所产生的暂时性收入。假
定经济体处于长期均衡中,所有要素获得等于其机会成本的收益,并且,一个典型厂商的收
入等于其成本。消费者偏好的变化或者某种技术变革将为一些行业带来利润。然而,随着资
本向那些超过正常收益率的市场转移,通过竞争性力量,这些利润将会被消除。因此,利润
不是生产要素的收益,而是与经济体中的动态因素相连的意外好运。

弗兰克.H·奈特(Frank H.Knight,1885一1972)通过将风险要素、管理能力、经济变革结合
成一种理论,颇有意义地使先前的利润理论成为一体并加以扩展。在《风险、不确定性与利
润》中,奈特区分了厂商所采取的能够投保的风险和没有有效保险的风险。例如,一个厂商
可能会在火灾中失去其设备,但精算知识允许运用保险来弥补这一风险。保险费用成了厂商
成本的一部分。因此,这种风险并不是利润的来源。利润之所以存在,是因为市场中存在不
适合保险的不确定性。这些都起因于市场的动态变化。然而,如果我们放弃完全竞争的候设,
利润则可能由于很多原因而出现,最重要的是垄断与买主垄断。资本与利息理沦随着边际生
产力理论的发展,经济学家开始更加细致地区分利润与利息。这使得普遍公认的利润理论得
到发展,然而,资本与利息理论直到今天还存在争议。罗伯特·M·索洛(Robert M.Solow)写
道:“当一个理论问题80年后还保留着争议时,就存在一种推测,即这一问题是以不适当的
方式引起的——或者的确非常深奥。”C.E·费格森(C.E.Ferguson)就资本理论尚未被解
决这一特征,提出了若干理由:
\begin{quotation}
  每个人都知道,或者明显地觉察到资本理论是不容易的。对于这一点,存在一个肤浅的理
  由,即大量资本理论的文献陷入了辩论术和语义学中。但是,也有一个更为本质的原因。
  资本理论必然要涉及时间;而时间又涉及预期和不确定性,虽然我们一般通过假定一种静
  止状态,或者一条黄金时代的增长路径将它们去除。
\end{quotation}

我们将首先通盘考虑自1890年以来资本与利息理论的发展。包括熊彼特、费雪还有奈特在内
的一组经济学家,对资本的性质和利息存在的原因进行了广泛的哲学研究。另一组经济学家
只是在表面上略微谈到利息存在的原因,而集中解释决定利息率的经济力量。利息率决定力
量的理论可以被划分为非货币的、货币的以及新凯恩斯的,最后一种是最早由希克斯提出的
模型中另外两种方法的综合。非货币利息理论集中研究决定利息率的长期实际力量,因此属
于古典传统,这些理论从重商主义后期一直持续到20世纪30年代。货币利息率理论包括可贷
资金理论与流动性偏好理论。从1890年至20世纪30年代,利息理论方面三位最重要的经济学
家是庞巴维克、奈特和费雪,我们在本章考察他们的理论。

重商主义者强调经济体中货币的作用,以及因此形成的货币利息理论。他们主张,货币数量
的增加不仅提高了价格总水平,降低了货币价值,而且降低了利息率的一般水平。在重商主
义阶段的稍后时期研究利息理论的一些经济学家进行了更加敏锐的分析。尽管理查德·坎蒂隆
提出了一种非货币利息理论,然而,他也指出货币数量的增加能够引起利息率的上升或者下
降。如果货币供给的增加首先到了节俭的人手中,利息率将下降。但是,如果货币首先到了
挥金如土的人手中,利息率将会上升,原因在于,支出的增加将引起厂商投资增加,从而使
可借贷资金的需求增加。

古典理论聚焦于国民财富的长期实际决定力量,发展出非货币利息或者说实际利息理论。古
典经济学家认为,利息率取决于投资支出的收益率。货币力量在短期内能够改变利息率,但
是在长期中,决定利息率的是资本生产力这种实际力量。李嘉图最为简洁地表达了这一观点,
他说:
\begin{quotation}
  利息率取决于由于使用资本所能获得的利润率,并且完全与货币的数量或者货币的价值无
  关。无论银行是贷出一百万、一千万还是一亿,都不会永久地改变市场利息率,只能改变
  它们所发行的货币的价值。
\end{quotation}

我们能引用李嘉图的其他段落来表明,他的确意识到利息率并不是与货币数量“完全无
关”的。核心问题是,古典经济学家对经济体中长期力量的关注导致他们不强调货币力量,
其原因在于,货币力量只对利息率有短期影响,并不能改变资本的生产力,资本的生产力才
是长期中决定利息率的实际力量。宽泛地来看,从1500年至1750年大约250年的时间里,时兴
货币利息理论;从1750年至1930年大约180年的时间里,正统理论家提出了非货币利息理
论。20世纪30年代期间,出现了两种新的货币利息理论,即流动性偏好理论与可借贷资金理
论,随着这两种理论的出现,人们认识到一般均衡框架中所发展的利自理论,必须既包括货
币力量,也包括实际力量。

\subsection{利息难题}

经济理论的发展表明,回答老问题的新理经常会引起新的问题。我们已经了解到,边际生产
力分析的发展,粉碎了陈旧的古典分配理论。古典理论将人口划分为工人、资本家以及地主,
并将支付给这些要素的报酬解释为工资、利润以及地租。因为古典分配理论是一种剩余理论,
所以产品用尽问题——确定支付给要素的报酬是否等于总产品的数量——就不是理论上的问
题。恰恰是边际生产力理论首先提出了这个问题。边际主义者断定,在完全竞争市场的长期
均衡下边际产品价值的总和等于总产品。他们并不担心这一结论要求线性齐次生产函数。如
果需要这个假设,他们完全能做出这一假设。然而,产品用尽的观点提出了涉及利息与资本
的新的复杂问题。我们现在转向对这些问题的解释,在考察后来的理论家所提供的答案之前,
我们将它们共同称为利息问题。

在完全竞争市场的长期均衡下,生产要素将获得销售最终产品的全部收入。边际生产力理论
的这个结论提出了下面的问题我们应当怎样解释被称作利息的资本的收益?资本是一种先前
使用的劳动与土地所形成的产品,劳动与土地才是原始的生产要素。边际生产力理论主张,
资本的收益一定恰好等于用来形成资本的劳动与土地的价值。如果这是正确的,为什么资本
以利息的形式获得更多的收益?换句话说,为什么支付给资本的报酬超过了支付给用来生产
资本的劳动与土地所必需的报酬?在生产要素中,资本看上去很独特,因为它产生了一个永
久地流向其所有者的剩余价值。

一种显而易见的回答是资本是生产性的,它解释了利息的存在。然而,这种回答并不令人满
意。资本是生产性的,因为与资本一起使用的劳动与土地生产了更多的产量。但是,边际生
产力理论认为,资本的边际生产力导致用来生产资本的劳动与土地具有较高的收益,这意味
着资本不存在净收益。长期均衡下资本的收益必定恰好等于生产资本的成本,但是,在实际
中我们观察到,利息收入不变地流向资本所有者。这个问题还由于如下事实而进一步复杂化,
即现有的资本是过去劳动、土地以及资本的产品。边际生产力理论认为,市场将现有资本的
生产力价值归因于用来生产现有资本的生产要素。如果我们利用这一程序,反过来经历一遍
生产过程,那么,我们将只剩下原始的生产要素——劳动与土地。为了阐明利息问题,我们来
考察另一种生产要素即劳动。劳动是生产性的,但是流向劳动的收入或者说工资可以度量并
且等于劳动的生产力。劳动没有净收益,就像资本看上去没有净收益一样。利息问题得到了
庞巴维克的承认,但是熊彼特在其《经济发展理论》第5章中对这个间题给予了最清晰的说明,
《经济发展理论》于1912年最先用德文出版。

我们怎样解释利息的来源、基础以及持久性?在考察一些有关利润问题的过程中,我们发现
在长期均衡下,利润消失变为零。然而,即使在长期均衡下,也能观察到利息的持续存在。
熊彼特不仅简洁地提出了利息问题,而且提出了一个框架,在其中考察可能的管案。对于利
息问题,存在三种可能的解决方案。第一种方案是存在三个而不是两个原始生产要素,利息
是第三种要素的收益。第二种方案认为边际生产力理论的如下主张是错误的,即在长期竞争
性均衡下,销售最终产品的收入将恰好等于流向生产要素的报酬。第三种方案认为边际生产
力理论是一种竞争性静态市场理论;因为实际经济体既不是竞争性的,也不是静态的,所以,
经济体中非竞争性的动态因素能产生一个正的利息率。利息的问题就说到这里,现在我们来
考察1890年至1930年期间所提供的一些解决方案。

\subsection{庞巴维克的利息理论}

1888年,庞巴维克在《资本实证论》中提出了他自己关于资本与利息理论的观点:“现在的
产品通常比同一种类和同一数量的未来的产品更值钱。这个命题是我要提出的利息理论的要
点和中心”。考虑到正的利息率的存在,这一陈述显然是正确的。在这些情况下,一个人宁
愿要今天的1美元,也不愿意要从现在起一年后的1美元,原因是今天得到的1美元能够被贷出,
因此比未来的1美元更值钱。然而,庞巴维克的陈述并没有马上解释利息存在的原因,虽然他
指出,利息存在的根本原因是现在的产品比等量的未来的产品更值钱。

庞巴维克在《资本与利息》中对以前的利息理论进行考察,并得出结论认为,没有人曾经解
释过利息的原因。他认为这些原因不是在社会的制度结构中找到的,而是在独立于社会形态
的技术与经济因素中找到的。

庞巴维克提出了三种理由来解释现在产品的较高价值。“现在的产品与末来的产品在价值上
的差别,第一个重要的原因在于需求的环境与供应的环境在现在和未来的差别。”给予现在的
产品较高价值的第二个原因是“我们蓄意地低估未来的需要,以及将要满足这些需要的产
品”。然而,庞巴维克在对利息存在的第三个解释中提出了生产者贷款市场。这一解释声称,
利息之所以存在是因为现在的产品比未来的产品更具有技术上的优越性。

庞巴维克关于现在的产品比未来的产品更具有技术上的优越性的观点,引起了很多问题,这
些问题在当时的文献中,尤其是在他与约翰·贝茨·克拉克以及欧文·费雪的争论中被广泛地加
以讨论。20世纪30年代末,在涉及弗兰克·H·奈特与尼古拉斯·柯达尔(Nicholas Kaldor)的
争论中,这些问题又被重新考察。

庞巴维克认为,利息存在的第三个理由独立于前两个理由。但是,欧文·费雪正确地提出,当
缺少庞巴维克的前两个理由时,迂回方法下的较高生产力并不会导致正的利息率。前两个理
由本质上表明,出于心理上的原因个人宁愿选择现在的产品,而不愿选择未来的产品。我们
假定个人并不是宁愿选择现在的产品而不愿选择未来的产品,并且单独来考察第三个理由。
考虑到资本是生产性的,以及延长生产过程将增加最终产品的持续供应量这一假设,当时间
偏好不存在时,社会希望使生产过程中形成的最终产品数量最大化,而不管它们形成的时期。
如果社会在消费最终产品的时间上没有差别,那么,现在的产品在技术上的优越性,将不会
导致个人愿意支付利息,以便消费今天的产品而不是未来的产品。庞巴维元阐明了一个一致
性的利息理论所有的必要因素,但是却错误地渐定,与时间偏好独立且分离的资本生产力将
导致正的利息率。欧文·费雪接受了庞巴维更具有开创性但令人困惑的观念,放弃了一些非本
质的因素,清晰明白地说明了现在公认的利息理论的本质要点。

\subsection{费雪论利息}

尽管欧文费雪采用了庞巴维克利已理论中的很多基本概念,然而,他的方法代表了对庞巴维
克方法的明显改变,是一种新古典经济学家所采用的方法。古典理论是以下列判断为基础继
续其研究的,即人们能够对生产要素进行相当明显的区分,并且这些要素的收益能够被划分
为工资、地租、利息还有利润。庞巴维克延续了这一传统,因此,他对利息理论的论述是以
下列观点为基础的,即资本的收益是利息,对照工资与地租来解释利息时,需要一种特殊的
理论。费雪最早在其《利息率》中,稍后在进行了相当大修订和推荐后改名为《利息理论》
的版本中提出了他的观点。

费雪反对工资、地租、利润以及利息这种通行的收入划分方式。他不仅将利息看做是资本所
获得的收入的一部分,而且看做是考察各种收入流量的一种方式。随着时间的变化,所有的
生产要素都产生收入流量。如果这些收入流量以现有的利息率进行贴现,就得到其资本化的
价值。某种生产要素的所有者通过计算资本化的价值与收入流量,来计算那种要素的利息收
益。举个例子来阐明费雪的观点。例如,说土地获得一种称作地租的收益;然而,如果我们
拿被称为地租的收入流量,与土地的资本化价值进行比较,收益就是利息。正如费雪所说
的:“地租与利息仅仅是度量相同收入的两种方式。”弗兰克·H·奈特赞同关于利息理论的这
一看法,并在其关于这一主题的著作中充分地进行了解释。奈特1949年声称“只有历史上的
意外事件或者‘心理状态’,能够解释‘利息’与‘地租’被视为来自于不同来源,尤其是
来自自然要素和资本产品这一事实”。历史上称作工资的劳动收益,也能被视为利息。职业
培训上的一笔投资,将增加工人未来的收入流量。因此,由于收入流量必须按照利息率进行
贴现,使之等于培训成本,所以,通常称作劳动的生产要素就能够被视为资本。根据这一观
点,费雪断定“利息不是收入的一部分,而是收入的全部”。费雪放弃了庞巴维克对要素的
分类及其有关生产阶段的全部观点,主张利息是通过个人在市场上调整其收入流量而形成的。
利息率度量了个人为了现在而不是未来获得收入所支付的价格。任何生产要素的所有者总是
拥有改变收入流量的选择权。为了购买或者建造能够增加未来收入流量的机器,或者,为了
投资于未来高薪工作所必须的培训,就必须削减现在的消费支出。

在市场经济中,两种力量决定着利息率:主观力量反映了个人对现在的产品或收入,而不是
对未来的产品或收入的偏好;客观力量取决于可利用的投资机会以及用来生产最终产品的要
素的生产力。个人能够通过借入或贷出、投资或减少投资来改变他们的收入流量。他们的行
为将取决于他们的时间偏好、不同投资可得到的收益率以及市场上的利息率。庞巴维克认为,
资本的生产力单独地——他所称的现在的产品在技术上的优越性——能够解释利息的存在。
费雪则宣称,对于解释利息的存在来说,资本的生产力以及个人的时间偏好两者都是必要的。
换句话说,资本的生产力将导致人们对收入的消费由现在延期到未来;但是,除非个人宁愿
选择现在的产品而不愿选择将来的产品,否则将不会普遍存在正的利息率。

尽管在费雪对其利息理论的说明中,当把理论应用于简单情形时,使用了无差异曲线分析;
当把理论应用于多个个体多个时期的情形时,使用了数学方法,然而,我们通过运用比较传
统的供求分析,也能理解其方法的实质。个人通过储蓄(投资)或者通过动用储蓄(借入),
能够改变他们的收入流量。储蓄的供给是利息率的函数:在较高的利息率上,储蓄的数量将
会增加。个人在现在的收入与将来的收入之间存在一种偏好,他们将进行储蓄或者减少投资,
直至他或她未来收入与现在收入之间的边际时间偏好率等于利息率。投资的需求曲线也是利
息率的函数,在较低的利息率上,投资的数量将会增加。费雪将投资的预期利润率称为弥补
成本后的边际收益率,它与凯恩斯的资本的边际效率概念类似。通过储蓄和动用储蓄,个人
能够改变他们的收入流量;在个人的均衡位置或最佳位置上,要求弥补成本后的边际收益率
等于利息率。当借入者希望借入的资金数量等于贷出者希望贷出的资金数量时,就实现了市
场均衡。在这一点出现之前,利息率将会变动。例如,如果在现有的利息率上,贷出的愿望
超过了借入的愿望,利息率将下降。在长期均衡下,个人改变其收入流量的行为,将导致利
息率等于边际时间偏好率以及弥补成本后的边际收益率。费雪的观点实际上比我们在此总结
的要更加复杂,它代表了先前存在的有关利息性质与利息率决定力量观点的重大进展。

\subsection{利息难题一个总结}

大约19世纪末20世纪初,正统经济条家开始将边际分析应用于生产要素的定价以及分配理论。
边际生产力理论提出了产品用尽的问题,并推断在完全竞争市场下,要素边际产品的总和将
用尽总产品。这提出了一个严肃的关于资本收益的理论问题。资本似乎永久地以利息的形式
获得收益;但是,如果最终产品的价值完全被生产要素所吸收,那么将不会有任何东西剩下
来为资本提供利息收入。一件资本产品的价值,将向后流入支付给用来生产资本产品的生产
要素的较高价值中。

庇巴维克的利息理论与费雪的利息理论,借助个人宁愿选择现在的产品而不愿选择等量的未
来产品这一事实,解释了长期竞争均衡下利息的存在,从而解决了这一明显的矛盾。由于这
种时间偏好,今天给予生产要素的报酬,将低于明天所生产的最终产品的价值。生产要素将
获得其边际产品的贴现值;当生产最终产品时,这些贴现值与边际产品价值之间的差额就是
利息。

\vspace*{3cm}

\section{总结}

19世纪90年代见证了微观经济理论重要的新发展。尽管早期边际主义者强调其观点与古典正
统观点在内容上的区别,但是,经济学家逐渐认识到他们之间重要的区别是方法上的区别,
即边际主义的应用与抽象模型的构建之间的区别。第一代边际经济学家将他们的方法几乎专
门应用于需求方面和家庭,很少构建理论来解释供给、生产要素的价格、收入的分配以及与
利息和利润相关的特殊问题。但是,考察边际量上运转的经济力量的新方法,逐渐被用来得
出了生产要素的需求曲线,并表明厂商使用若干要素的最佳方式。边际生产力分配理论得到
发展,并提出了新的引起注意的理论问题。因为古典经济学家使用分配的剩余理论,所以,支
付给生产要素的报酬总和必定等于总产品。新的理论主张每种要素得到其边际产品,从而提
出了产品用尽的问题。能够导致产品用尽的生产函数必须具有的数学性质已经被发现,完全
竞争的市场在长期均衡下满足这些先决条件的看法也被提了出来。但是,这一解决方案引起
了另外的问题,例如,决定厂商长期平均成本曲线的经济力量,以及规模收益不变与竞争的
相容性问题。

克拉克试图从边际生产力理论中得出道德结论,其他人则利用该理论来解释经济萧条。基于
很多理由它遭到了批评,最重要的是,它不可能确定合作要素的边际产品。经济学家很快就
认识到,利润与利息是需要进行特殊研究的收益,并提出了很多利润理论,所有的理论都能
从本质上推导出利润或者起因于垄断力量,或者起因于完全竞争市场的暂时非均衡。将利息解
释为非货币现象的古典传统持续着,但是,除了资本的生产力这一古典客观原因之外,个人的时
间偏好也被承认是利息的一种主观原因。结果,利息理论适合于这一时期出现的基本的供求
框架。在第10章考察完马歇尔的经济学之后,我们就能够总结和评价边际效用学派强调需求
和古典学派强调供给时的优缺点,以及马歇尔处理这些问题的尝试。



%%% Local Variables:
%%% mode: latex
%%% TeX-master: "../../main"
%%% End:
