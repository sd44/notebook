\chapter{卡尔·马克思及其对古典经济学的批判}

卡尔·马克思是一位经济学家,同时也是一位哲学家、社会学家、预言家、革命家,他的生
涯证明了经济理论的重要性。他的作品鼓舞着一代又一代经济思想家,因为他的缘故整个社
会被改变了。因此,考察像卡尔·马克思这样一位具有不同寻常影响力的人的观点,对我们
来说始终都是重要的。

\section{对马克思的总体看法}

马克思首先并且最重要的是一位哲学家,他觉得他的工作不但是解释和分析社会,而且是推动
他所期待的社会变革。作为一个有派别的变革倡导者,他与斯密、李嘉图或者约翰·斯图亚特,
穆勒并没有区别。然而,与古典经济学家相反,马克思提倡社会与经济的基本革命,而不是
小的边际变化。马克思科社理论方面很少涉及怎样组织社会主义或者共产主义经济体。

马克思的经济理论是其历史理论在资本主义经济体中的应用。他希望揭开资本主义运行的规
律。当其他古典经济学家将注意力集中在经济体的静态均衡上时,马克思则将注意力集中于
变革的动态过程。马克思的经济学帮助人们了解构成市场基础的力量,而标准的古典分析在
组织和运转社会主义经济体时又是有益的。

已故的奥斯卡·兰格(Oskar Lange,1904一1965)反复强调这一观点。他主张,马克思的经济
分析和正统的经济分析应当被视为互相补充的,而不是相互排斥的。兰格说,通过运用正统
的新古典理论,能够获得对市场日常运行的理解;而要获得对资本主义进化发展的理解,只能
在马克思的框架中完成。

讨论增长时,马克思强调了技术与收益递增的决定作用。他认为,由于技术的原因,厂商将
会变得越来越大。在强调这点时,马克思预见到了现代内生增长理论家所做的研究,他们使
现代经济学回到对增长与收益递增的关注上。与现代研究相比,马克思的讨论更加广泛和深
远,并将注意力集中于同样的问题上——技术在决定经济体运转中的重要性,以及收益递增
的念义。
\begin{tcolorbox}[title = {《共产党宣言》中的共产主义是什么样的},
  fonttitle = \sffamily\bfseries, fontupper = \small, fontlower =
  \small\itshape, left=20pt]
  \begin{multicols}{2}
    \begin{enumerate}
    \item 剥夺地产,把地租用于国家支出。
    \item 征收高额累进税。
    \item 废除继承权。
    \item 没收一切流亡分子和叛乱分子的财产。
    \item 通过拥有国家资本和独享垄断权的国家\\银行,把信贷集中在国家手里。
    \item 把全部运输业集中在国家的手里。
    \item 按照总的计划增加国家工厂和生产工具,开垦荒地和政良土壤。
    \item 实行普遍劳动义务制,成立产业军,特别是在农业方面。
    \item 把农业和工业结合起来,促使城乡对立逐步消灭。
    \item 对所有儿童实行公共的和免费的教育;取消现在这种形式的儿童的工厂劳动;把教
      育同物质生产结合起来,等等,
    \end{enumerate}
  \end{multicols}
  % \tcblower
  % This is the lower part.
\end{tcolorbox}


\subsection{马克思观点的智力来源}
马克思出生于一个犹太教家庭,他的家庭后来转为基督教,青年马克思开始时研究法律,但
是很快就变得对哲学感兴趣。在他的研究早期,他被另一位德国经济学家G.W.F·黑格
尔(Georg Wilhelm Friedrich Hegel,1770一1831)的知识思想所吸引。就像我们将要看到的,
这一思想成为马克思体系中的重要成分。在获得哲学博士学位之后,由于观点激进,马克思
未能找到一份学术职位,因此他转到报社工作。他的政治观点对于他所处时代的德国来说,
是激进的但还不是社会主义的,这使得他被驱逐出德国。在巴黎和和布鲁塞尔,他开始研究
法国社会主义思想与古典政治经济学。马克思拥有非凡的智力,同时有着强烈的阅读与研究
动力。被从巴黎和布鲁塞尔驱逐之后,他移居到伦敦,花费了其生命最后33年的时间,在世
界上最大的图书馆之一——大英博物馆中阅读与写作。

\subsection{马克思的历史理论}

马克思的思想结合了黑格尔的哲学、法国乌托邦思想以及古典政治经济学——尤其是李嘉图主
义。马克思对资本主义的分析,是他的历史理论在其所处时代的一种应用,他的历史理论来
自于黑格尔。黑格尔认为,历史不会像很多人所认为的那样,通过一系列重复情形循环地往前
推进,而是通过一组三种力量的相互作用,沿着一条直线日益增进地向前移动,他将这三种
力量称为论题(thesis正题)、对立面(antithesis反题)、综合(synthesis合题)。因为这些力
量是意识形态的,所以它只存在于观点的研究中,不存在于能够找到历史规律的过去事件的
研究中。按照黑格尔的观点,在任意既定时间都存在一个公认的观点或者说论题,但是很快就
会与它的相反种物或者说对立面相抵触。在这些观点的冲突中形成了一种综合,它代表了真
理的更高形式,并又成了一个新的论题。这个新的论题同样受到它的对立面的对抗,并被转
化成一个新的综合,等等。这样,在一个永无休止的观念链条中,每个观念都更加接近真理,通
过一个无休止的过程,历史得到了发展。在这个过程中,借助诱导冲突的变革,所有的事情都
逐渐变得更加完美。黑格尔将这一过程以及研究过程的方法称为辩证法。

马克思在历史中——总的来说是在现实中——察觉到了一种类似的过程,并使用了一种类似
的方法来研究它,他也称其为辩证法。但是,黑格尔哲学与马克思哲学的重大区别是,黑格
尔的哲学是唯心主义的,而马克思的哲学是唯物主义的。对于黑格尔来说,发生变革的本体
是观念,对马克思来说则是物质,马克思声称物质本身就包含着持续冲突的萌芽。因此,马
克思的哲学被称为辨证唯物主义(dialectical materialism)。

引起马克思注意的重大问题是:能够形成一种解释不同时期不同社会组织方式的理论吗?这一
理论能被用来预测社会未来可能的组织方式吗? 我们称作封建主义和资本主义的社会结构,
是能被加以分析的进化发展中的一部分,抑或仅仅是随机历史事件的结果?

马克思指责资本主义的中产阶级经济学家,说他们创作时就好像只有过去而没有未来一
样——从早先制度演化来的资本主义制度,仿佛是一种永远存在的理想的社会结构。因此,马克
思体系中的一个重要成分就是变革。马克思认为,尽管我们不可能准确地知道未来将会带来
什么,但是,我们的确知道它与过去和现在不同。

通过将历史变革的主要决定因素(尽管不是唯一因素)集中于唯物主义力量或者经济力量上,
马克思彻底改革了社会科学的思想。

\begin{quotation}
  我所得出的以及曾经得出的,并始终作为我研究主线的一般结论,可以主要概括如下:在
  人类从事生产的社会中,人们是一些明确关系的组成部分,这些关系是他们的意志所必需
  的并独立于他们的意志;这些生产关系对应于他们物质生产能力的明确发展阶段。这些生产
  关系的总和构成了社会的经济结构——实际基础,法律的与政治的上层建筑建立在这一基础
  之上,明确的社会意识形态与这一基础相适应。物质生活中的生产方式决定了社会生活、
  政治生活以及精神生活过程的一般特征。不是人类的意识决定了他们的存在,而是相反,他
  们的社会存在决定了他们的意识。在社会生产力发展的某一阶段上,它与现有的生产关系
  发生冲突,或者——使用同一事物的法律表述——与以前在其中运转的财产关系发生冲突。就
  生产力的发展形态而言,这些关系变成了它们的束缚。接看,社会革命的时期就到来了。
  随着经济基础的变革,整个广大的上层建筑或多或少地被迅速转变。
\end{quotation}

马克思认为,除了没有阶级的社会之外,有所有的社会都能解析成两个部分:生产力与生产关
系。生产力是社会生产物质产品时所使用的技术;表现为劳动技能、科学知识、工具以及资本
产品,它们天然是动态的。生产关系是比赛规则;存在着一个人与另一个人之间的关系即社会
关系,以及人与物之间的关系即财产关系。为了进行生产,必须解决经济秩序问题;历史上
决定的生产关系提供了一定的制度框架,人们在该框架内做出经济决策。与动态的和变化的
生产力相反,生产关系是静态的并与过去相黏合。生产关系的静态性质被马克思所说的社会
上层建筑所加强,它的作用是维持历史上既定的生产关系。社会上层建筑由艺术、文学、音
乐、哲学、法律、宗教以及其他社会公认的文化形态所构成,它的主要目的是保持生产关系
的完整无缺——维持现状。

静态的生产关系是马克思辩证法的论题,动态的变化的生产力是对立面。在任何历史阶段的
初始,生产力与生产关系之间都是和谐的,但是,随着时间的发展,变化的生产力引发了系统
中的矛盾,因为现有的生产关系(制度)不再适应生产力(技术)。马克思说,这些矛盾将会
在阶级斗争中显露出自身来。最后,矛盾会变得如此激烈,以至于会有一个社会革命阶段,
并会导致一套新的生产关系产生。新的生产关系就是综合,它是旧论题(生产关系)与对立面
(生产力)冲突的结果,这些生产关系就变成了新的论题。在这一点上,历史再次出现和谐,
然而,动态的变化的生产力确保了新的了矛很快就会形成。

\subsection{近看辩证法}

考察马克思的社会上层建筑概念,将有助于分清马克思的历史理论与马克思主义对待社会的
态度。马克思对个体自身的实现感兴趣。这一兴趣最清楚地展现在他的《1844年经济学与哲
学手稿》中,这一著作消失了80年,直到1932年才出版。在这些早期手稿中,马克思解释了他
反对资本主义的哲学理由,以及他如何认为资本主义使人类异化(alienation)了自身。根据
马克思的观点,私人财产与市场使人类所接触到的一切东西减值并被贬低,因此,使人类疏远
了真正的自我。从而,市场的真实存在——尤其是劳动市场——破坏了人类获得真正幸福的能力。

马克思认为古典经济学简单地接受了市场,并没有考虑私人财产的性质,以及市场的存在给
人们造成的影响。他主张有必要研究如下事物之间的联系,即“私人财产、贪婪与劳动、资
本、土地财产分离之间的关系;交换与竞争之间、人的价值与贬值之间、垄断与竞争之间等的
关系;所有这些疏远与货币体系之间的联系”。因此,他对古典经济学的主要批评是,古典经
济学没有考虑生产力如何破坏了生产关系。

马克思主张,一旦市场最后产生了能够满足人类物质需要的生产力,财产权和市场中固有的异
化将使个体脱离市场,并形成一个消除了私人财产及与之相连的异化的社会。

马克思的历史理论探寻了从封建主义到资本主义的社会发展,以及社会末来的发展。按照马
克思所预测的,社会将进入社会主义社会,并最终进入共产主义社会。马克思主张,在封建
阶段早期,生产关系适合当时存在的生产力,并且这些生产关系借助社会上层建筑得到支撑
与加强。然而,变化的生产力很快摧毁了这一和谐,因为封建主义制度结构与发展中的农业
技术、增加的贸易以及制造业的出现相矛盾。和生产力与生产关系之间的这些冲突,在阶级
斗争中显现出来,最终形成了一套新的生产关系——资本主义。

像封建主义一样,资本主义包含了其自身毁灭的种子,因为随着生产力的变革,不可避免地会
形成冲突。随着资本主义的衰落,将会出现一套新的生产关系,马克思称之为社会主义;社会
主义最终将依次让位给共产主义。在我们转向马克思对资本主义的详细考察之前,马克思的
历史理论所提出的几个其他问题也值得我们关注。

\subsection{社会主义与共产主义}

根据马克思的观点,社会主义是一套跟在资本主义之后的生产关系,它包含着资本主义的一
些残余。他说,资本主义的一个主要特征是资本这种生产手段,不为无产阶级拥有或支配。
发生在资本主义向社会主义转变过程中的主要变革是剥夺者被剥夺——无产阶级拥有生产手段。
然而,社会主义中的一种资本主义残余表现为仍然主要通过使用激励机制来组织经济活动:为
了引导人们劳动,必须给予报酬。

按照马克思所使用的共产主义概念,它将从社会主义经济体中诞生。共产主义经济体完全不
同于社会主义经济体。人们不再受货币或物质激励而去工作,资本主义下存在的社会阶级,
在社会主义下将在较低程度上存在,到共产主义则将会消失。共产主义是一个废除了国家的
没有阶级的社会。在社会主义下,每个人按照他或她的能力为经济过程做出贡献,并且按照他
或她的贡献获得收入;在共产主义下,每个人按照他或她的能力做出贡献,但按照他或她的需
要进行消费。”可以看到,马克思认为人类会变得完美,人类的善良被视为受到了现有社会
的抑制与扭曲。这一思路延续了从威廉.戈德温(Wiliam Godwin,1756一1836)开始的哲学无
政府主义的智力血统。

\section{马克思的经济理论}

马克思的体系是哲学分析、社会学分析以及经济分析的混合,因此将其纯经济理论与其他理
论分开有些不公平。在探寻生产力与生产关系之间的矛盾时,马克思将其历史理论应用到他
所处时代的社会中,确信资本主义不可避免地要瓦解。他认为,这些矛盾在阶级斗争中显露出
来,其原因在于,正如他在《共产党宜言》中所阐述的,所有社会的历史都是阶级半争的历
史。一种社会的生产关系以至制度结构的基本决定因素将是生产力。马克思断言,与手推磨
相适应的制度结构是封建主义,与蒸汽机相适应的制度结构是资本主义。技术过程的逻辑产
生了使蒸汽磨能够从手推磨进化而来的条件与力量,随着生产力的变革,旧的生产关系必须
让位给更加适合的制度形态。因此,马克思将现在视为辩证法历史演变的一部分。

\subsection{马克思的方法论}

马克思研究经济体的方法是非传统的。现代经济理论尤其是微观经济理论,试图了解经济体
的全部,使用的方法是考察经济体的组成部分,例如,家庭、厂商以及市场价格。而马克思
从整个社会和经济体的层面开始,通过考察它们对其组成部分的影响来分析它们。从而,在
现代方法论中,主要的因果关系是从部分到整体,而在马克思的方案中,则是整体决定了部
分。对马克思理论与现代经济理论不同方法的这种描述只是一种简化描述,原因在于,两者
都考虑到了部分与整体之间的相互作用,但是又的确在方向上体现出了基本区别。

\subsection{商品与阶级}

马克思通过考察拥有生产手段的资本家与只在市场上出卖劳动的无产阶级之间的交换关系开
始他的分析。他认为,资本主义的主要特征之一是劳动与生产手段所有权的分离。资本主义下,
劳动者不再拥有和车间、工具或者生产过程的生产资料。因此,资本主义本质上是两个阶级
的社会,资本主义社会最重要的一个方面是交换,即发生在资本家与无产阶级之间的工资议
价。基于这一原因,马克思形成了一种解释商品价格或者说交换价值的理论。因为对解释财
产收入的来源特别感兴趣,所以,他考察了下列两种价格的决定力量,即劳动所生产的商品的
价格(价值),以及劳动所获得的价格(劳动力价值),后者是对劳动生产性努力的支付。

李嘉图的经济理论,以及随后的正统微观经济理论,都以商品价格开始他们对经济体的分析。
所以,人们通常假设马克思对同样的基本问题,即解释商品价格的决定力量感兴趣。然而,
马克思并不主要对形成一种相对价格理论感兴趣。他的兴趣所在是工资,他认为这是资本主
义制度中最为关键的因素,因为工资揭示了一种矛盾,该矛盾有助于解释资本主义制度的运动
规律。对他来说,劳动价值理论是更广泛目的的一种手段——目的就是了解社会的演变。

按照马克思的观点,在前资本主义经济体中,人们主要是为了使用价值而生产产品;也就是说
商品被生产出来,是为了生产者消费用的。资本主义的一个主要特征是资本家生产商品不是
为了它们的使用价值,而是为了它们的交换价值。因此,要了和解资本主义,就需要了解发
生在商品所有者之间的交换关系,最重要的是资本家与无产阶级之间的关系。

可以用另一种方式表达这一点。按照马克思的观点,资本主义制度中的商品价格代表了两种不
同的关系:(1)商品之间的定量关系(两头海狸交换一头野鹿);(2)经济体中个人之间的社会关
系或定性关系。作为经济体中价格的工资,代表了资本家与无产阶级之间的定量关系,也代
表了两者之间的社会关系或定性关系。马克思对价格感兴趣,主要是就它们揭示了这些社会
关系而言的;他对价格感兴趣的第二个原因是它们反映了商品之间的定量关系。

\subsection{马克思的劳动价值理论}

在形成一种相对价格理论,或者说商品之间定量关系的过程中,马克思实质上使用了李嘉图的
劳动价值理论。商品在它们的价格中时现出某种定量关系,按照马克思的观点,这意味着所
有的商品必须包含一种共同的成分,这种成分一定存在于某种可度量的数量中。马克思考虑
用使用价值或者说效用这一概念作为这种共同的成分,但是最后否定了这种可能性。之后,他
转向以劳动作为共同的成分,并断定正是生产商品所需要的劳动时间的数量制约着商品的相
对价格。作为劳动价值理论的倡导者,马克思解决了内在于劳动价值理论阐述中的各种问题,
就像李嘉图在他之前所做的那样,并且实质上遵循了李嘉图的解决办法,马克思能够就劳动
价位理论中的难题给予比较清楚的呈现,得是与李嘉图相比,他没有更多的能力来解决这一
问题。

对马克思而言,生产商品的唯一社会成本就是劳动。在最高的抽象层面上,马克思忽略了劳
动的不同技能,确信社会可利用的生产商品的总劳动是一个同质的量,他称其为抽象劳
动(abstractlabor)。商品的相对价格反映了生产产品所必需的这种抽象劳动供给的数量,它
用时钟的小时数来度量。社会必要劳动时间被界定为当时拥有平均劳动熟练程度的人所花费
的时间。对于高于平均技能水平的劳动来说,通过度量其较高的生产力,并进行适当的调整,
可被还原为平均水平。我们看到,斯密用支付给劳动的工资来度量劳动技能的差异,为此陷
入循环推论。马克思通过假设劳动技能的差异不仅通过工资,而且通过物质生产力差异来度
量,从而避开了一些问题。

劳动价值理论提出的另一个问题是如何解释资本产品对相对价格的影响。马克思使用李嘉图
的办法来回答这个问题,认为资本是贮藏起来的劳动。于是,生产一件商品所要求的劳动时
间就是立刻投入的劳动的小时数加上在生产过程中和毁坏的资本所要求的小时数。马克思的
方案和李嘉图的方案一样,未能考虑到如下事实,即在资本被使用的情形中,可能为基金支
付了利息,这些基金用来支付贮藏在资本中的间接劳动,时间是从间接劳动的支付时间开始
到产品销售时间为止。

劳动价值理论还必须解决因土地不同肥力所引起的问题。马克思通过采用李嘉图的级差地租
理论来应对这个问题,优等肥力土地上较高的劳动生产力,被地主作为级差地租吸收了。竞争
将导致优等级别土地的地租上升,直到所有级别土地的利润这者相等。因此,地租是价格被
决定的因素,而不是决定价格的因系。

内在于劳动价值理论中的最后一个难题来自利润对价格的影响。这个问题的一个关键方面涉
及不同行业的“劳动--资本比率”。与较低资本密集度的行业相比,高度资本密集的行业所
生产的产品,其利润占其最终价格的比例较大。基于对李嘉图的详细研究,马克思完全知道
这个问题,但是在《资本论》头两卷中,他自始至终都通过假设所有的行业和厂商拥有相同的
资本密集度来回避这个问题。(蛋蛋注:资本论第一卷是资本的一般规律,全部--一般--特
殊--个别是哲学术语,本书作者首先没有意识到这点。即使对马克思抱有一定肯定态度的西
经学者往往弱视资本论二、三卷。第三卷中梦幻的、颠倒的,资本主义的发展反过来反对它
赖以生存的生产力没有被重视。)然而,他在第三卷中结束了这个假定,并
且试图设计一个具有内在一致性的劳动价值理论。但是,他未能做到这一点,就像李嘉图未
能做到一样。在更加近距离地考察这个问题之前,我人需要更多地熟悉马克思的其它一些概
念。

\subsection{剩余价值与剥削}

马克思主要将劳动价值理论用做发展剩余价值全与剥削概念的工具。在此,我们不去关注劳
动价值理论的数学方面以及专业细节。我们关注的是,马克思将生产概念化,将它分成两个
部分:生产成本与剩余价值(surplus value),前者指生产商品所花费的劳动时间,后者指产
品价格与其生产成本之间的差额。

马克思对价值的论述包含着客观成分,它使经济体的某些方面能够被加以观察,但是,它也明
确地包含着意识形态的成分。剥去意识形态的色彩后,马克思的预言是简单的,即与所需要
的相比,任何经济体将生产出更多的产品与服务,以支付全部实际社会生产成本。因此,从
一国的总年产量中减去生产这些产量必须支付的全部实际成本,将会形成一个剩余,它可以
被称为剩余价值。这些实际成本包括劳动成本与资本成本。因而,马克思的剩余价值与重农
主义者的净产品概念相似。剩余是如何被分割的是一个复杂的问题,涉及哲学与法律方面的
问题。在马克思进行创作的时代,这些问题非常令人担心。工业革命导致了世界上生产出的
年剩余价值大量增加。马克思提出了一个合理的问题:在社会参与者之间分配这种社会生产的
剩余的一种公平办法是什么?

但是,马克思并不满足于仅仅提出这个问题,他也不满足于指出:在
他所处的时代,社会馅饼的切割是不公平、不公正、不合理的。马克思超越了这一点,宣称
劳动所生产的剩余被拿走了,原因在于,它不拥有生产手段的所有权。

\section{马克思对资本主义的分析}

为了揭示资本主义运动规律,辨别生产力与生产关系之间的矛盾,马克思将其历史理论应用
到他所处时代的社会与经济中。他关注经济体的长期趋势,当他考察现在时,总是将它放在
历史之中。在对资本主义的分析中,他阐明了某些原理,它们业已以马克思主义原理而著称,
一些马克思主义者对待它们,就像一些正统经济学家对待供求原理一样,怀着差不多同样的
敬意。马克思的资本主义规律包括如下内容:失业的后备军、下降的利润率、经济危机、产业
日益集中于少数厂商、无产阶级越来越贫困。

在对资本主义经济学的分析中,除了少数例外,马克思运用了基本的古典经济学工具,尤其
是李嘉图的理论。因此,他假设了:(1)解释相对价格的劳动成本理论,(2)货币中
立,(3)制造业收益不变,(4)农业收益递减,(5)完全竞争,(6)理性的精于计算的经济
人,,(7)修订过的工资基金学说。在他的大多数分析中,他否定了李嘉图的生产不变系数、
充分就业假设以及马尔萨斯的人口学说。

马克思与李嘉图在资本主义经济学分析中表现出的部分差别,并不是来自他们基本分析框架
的任何差别,而是来自他们各自意识形态的差别,认识到这一点很重要。因为马克思对资本
主义不满,他着眼于寻找制度中的缺陷或矛盾来考察它;李嘉图基本上接受了资本主义,把它
视为经济过程的一个和谐结果。马克思模型中的主要参与者,与李嘉图模型一样,都是资本
家。资本家对利润的寻求和对利润率变动的反应,在很大程度上解释了资本主义制度的动态
性。但是,在马克思的体系中,资本家理性地且精于计算地追求他们的经济利益,播下了他
们自身毁灭的种子,而在李嘉图的体系中,这些同样理性且精于计算的资本家,在追求他们
自身私利的过程中,促进了社会利益。尽管古典经济学家对静止状态的长期预测的确是悲观的,
但这样一种状态并不是资本主义的缺陷;相反,在他们看来,这是根据马尔萨斯的人口学说和
历史上的农业收益递减得出的。然而,对马克思来说,资本主义制度产生了不合意的社会后
果;他认为随着时间的变化,当资本主义中的矛盾变得更加明显时,作为历史的一个阶段的资
本主义将会逝去。(蛋蛋注:罗莎·卢森堡认为资本主义将崩溃,中后期考茨基认为在马恩
文献中,没有一处提及资本主义必将崩溃,只是不断面临危机)

失北后各年马克思否定马尔萨斯的人口理论。在古典分析中,这一理论对于解释利润的存在
来说是必要的。古典经济学家认为,资本积累导致了劳动需求增加和劳动的实际工资上升。
如果工资随着资本积累而持续上升,那么,利润水平将会下降。然而,马尔萨斯的人口学说,解
释了为什么工资不上升到使利润停止存在的那个水平:,因为工资的任何上升将导致更多的人
口和劳动力,因此,工资将下降到维持生活的最低水平。所以马尔萨斯的人口理论不仅解释
了古典体系中利润的存在,而且部分地解释了工资率的决定力量。

马克思反对马尔萨斯的理论,这意味着他必须寻找一些其他的工具来解释剩余价值与利润的
存在。在马克思的模型中,资本积累的增加将增加劳动的需求。随着工资率的上升,什么阻
止了剩余价值与利润降为零?马克思对这一问题的回答,在于他的失业后备军这一概念,它在
马克思体系中所起的理论作用,与马尔萨斯的人口理论在古典模型中所起的作用是相同的。
按照马克思的观点,市场上总是存在劳动的超额供给,它具有压低工资、使剩余价值和利润为
正数的作用。他注意到失业后备军可以从几种来源中得到补充。当机器在生产过程中取代了
人时,就发生了直接的补充。资本家对利润的寻求促使他们使用新机器,因而提高了经济体的
资本密集程度。被新机器取代的工人人,没有被吸收到经济体的其他领域中。间接补充来自
于新成员进入劳动力大军。完成学业的孩子们,以及随着家庭责任变化而希望进入劳动市场
的家庭主妇们找不到工作,从而进入失业的行列。这些失业后备军压低了竞争性劳动市场上
的工资(蛋蛋注:有必要加入国家政府资本主义意识形态人为制造失业后备军的政策影响,
这点将会厘清后几段内容)。

在马克思的体系中,后备军的规模以及利润与工资的水平,随着经济周期而变化。在经济活
动与资本积累扩张时期,工资提高,后备军的规模减少。工资的提高最终导致利润下降,资
本家通过用机器取代劳动对此做出反应。资本取代劳动所形成的失业,推动工资向下并恢复
了利润。

失业的后备军概念与正统分析中的一些方面相反。李嘉图在其《原理》第三版新的“论机
器”一章中提出了短期技术性失业的可能性。在古典体系中,技术性失业或者除了摩擦性失业
之外的任何失业在长期中都不可能存在。马克思关于长期的持续的技术性失业的假设,相当
于是对预测资源充分利用的萨伊定律的一个否定。大多数正统经济理论,由于以下原因从来
不愿意接受马克思的失业后备军概念:后备军的概念暗示了劳动超额供给的存在,即存在没有
出清的劳动市场,但是,如果供给的数量超过了需求的数量,并且存在竞争性的市场,那么,
经济力量将会推动价格向下,直到供给的数量等于需求的数量,并且市场出清。因为马克思
假定了完全竞争的市场,所以,正统经济学家认为马克思自身体系的逻辑,使其持续的技术
性失业的概念无效。

马克思主义者反对这一看法,他们指出,正统的框架是一种相对静止一一它假定当供给与需
求的力量发生作用,降低了工资并减少了失业时,若其他条件保持不变,尤其是当劳动市场
出清时,不会发生机器对人的替代。马克思主义者承认,考虑到正统理论的静态框架,正统分
析在理论上是正确的,但是他们认为,对劳动市场更加动态的分析,将会考虑到持久的非均
衡。将注意力集中在动态搜索理论上的现代正统宏观经济学家,赞同如下观点,即相对静态
框架中可能存在看上去像长期非均衡的某种东西,尽管他们认为,超额劳动供给暗示着经济
体中存在一个平均的高于竞争性均衡水平的工资。

一种可能的可以用来探究马克思的失业后备军概念有效性的方法,是考察随时间变化的失业
水平。然而,这一过程并不能给出明确的答案,因为用于统计度量的失业定义包含一些偏差。
在大多数国家,失业的人被认为是正在寻找工作但又未能找到工作的劳动力队伍的一部分。人
口中的一些成员并没有正在寻找工作,原因在于,过去他们没有能力找到工作,因而就从劳
动力队伍中退出来了。如果就业机会能够改善,那个工人就有可能重新返回劳动力队伍中。
劳动力队伍中的积极者与总人口的比率,通常称为参加率,它直接随着经济活动水平而变
化。“一个希望全职就业但正在从事兼职工作的人,通常被认为属于就业者队伍。马克思主
义者主张,退出劳动力队伍的人和从事兼职工作的人,促进了工资率的下降,他们应当包含
在失业的后备军中。因此,对美国经济来说,例如6\%的统计失业率,并不是失业后备军规模
的充分体现,因为它没有将劳动力中愿意但是未能获得全职就业的那部分考虑进去。(蛋蛋
注:关于失业人数统计,可参考美国失业率统计分档U1-U6,最普遍宣传的是U3失业率。即使
是U6失业率也未覆盖U1、U2 \url{https://www.haojingui.com/shiyelv/2048.html})


\subsection{下降的利润率}

在马克思所说的将最终导致资本主义灭亡的生产力与生产关系矛盾中,其中一个重要的矛盾
是利润率的下降。在此,他遵循了斯密、李嘉图、穆勒的古典传统,他们全都预测随着时间
的变化,利润率将会下降。。

根据马克思的观点,资本家具有资本积累的强大动力。马克思认为,商品与劳动市场上的竞争,
将会导致利润按照下列方式下降。资本积累意味着将要支付更多的资本给劳动,这促使工资
上升,失业后备军规模减小,利润率下降。资本家通过用机器取代劳动,对工资上升与利润
下降做出肥应一一增加经济体中的资本数量,而这将促使利润率更低。马克思意在表明每个
个别资本家,在对工资上升与利润下降做出反应时,所采取的行动将会有力地进一步降低经
济体中的利润率。

商品市场的竞争也将导致利润率的连续下降,原因在于,为了以较低的价格出上售最终产品,
资本家不断地试图降低生产成本。这些竞争性的力量,引导资本家寻找新的低成本的生产方法,
以减少生产既定商品所必需的劳动时间。这些新的更有效率的生产技术,几乎总是要求增加资
本,这将导致利润率下降。因此,马克思断定,劳动与资本市场上的竞争必然引起资本的增
加,接着会引起利润率的下降。

然而,因为经济体中资本数量的增加,产生了影响利润率的两种相反的力量,所以问题比这
更复杂。若其他条件保持不变,资本数量的增加将引起利润率的下降,原因在于收益递减原
理一一增加的资本降低了生产力。然而,资本数量的增加通常与新技术相混合,这会降低成
本,从而提高利润率。简而言之,当其他条件变化时,随着时间的发展,利润率是否下降取
决于资本积累变动速度与技术改进变动速度的比较。理论上不能确定这些相反力量的结果一
一这是一个经验问题。(蛋蛋注:马克思对利润率下降是个经验论述,这点是毋庸置疑的,
后来人也尝试以数学实证等方法来解释或否定利润率下降,目前尚无定论。本书在此方面阐
述的最大缺陷在于马克思并非是假设其他条件保持不变,而是长期内技术提高的普遍化,不
变资本对可变资本(劳动力价值)的比例提高等一系列因素将使利润率归回原点并保持下降
态势。)

因此必须断定,即使限制在马克思的模型结构中,随着时间的变化,利润率采取什么进程,
将取决于收益递减与技术改进这样两种力量的相对增长速度。马克思设置了一个不断下降的
利润率,尽管他的模型并没有为此提供理论基础。马克思、斯密、李嘉图,还有约翰·斯图亚
特·穆勒都因为实质上相同的原因,得出了利润率将会下降的结论:收益递减抵消了技术改
进。

然而,预测利润率变化时一个关键的因素是难以预测的,即预测技术发展的速度。技术发展
会在未来足以抵消资本积累引起的收益递减的速度吗?这个问题很难回答,很大程度上是因为,
经济学家还无法拥有能够满意地解释技术发展速度的理论。缺乏这样一种理论时,经济学家
倾向于低估技术发展的未来预期速度。这也是为什么斯密、李嘉图、穆勒都断定长期中利润
率将下降的原因。它也是为什么马尔萨斯断定,与食物供给相比,人口趋向于以更快的速度
增加的原因。借助图中的简单图形,能够更精确地吝焦这个问题。

\begin{figure}[ht]
  \centering
\begin{tikzpicture}
  \draw[very thick] (0,5) -- node[left=12pt, text width =1em] {利润率} ++(0,-5) -- node[below=10pt] {资本} ++ (10,0);
  \draw[dashed] (1.5,0) node[below]{$C_1$} -- (1.5,2) -- (0,2) node[left]{$P_1$};
  \draw[dashed] (2.5,0) node[below]{$C_2$} -- (2.5,2.4) -- (0,2.4) node[left]{$P_3$};
  \draw[dashed] (2.5,1.5) -- (0,1.5) node[left]{$P_2$};
  \path (1.5,2) -- (2.5,1.5) coordinate[pos=-1.2](dd) coordinate[pos=3](ff);
  \draw (dd) node [above]{$M$} -- (ff);
  \draw[shorten <= -3.8cm, shorten >= 3cm](2.5,2.4) -- +($(dd)-(ff)$);
  \draw (2.5,2.4) node [above, xshift=-1.5cm, yshift=0.8cm] {$M'$};
\end{tikzpicture}%
  \caption{\label{fig:marxlirun}利润率的下降 }
\end{figure}

我们用向下倾斜的曲线M表示资本积累增加引起的收益递减,或者今天所知的投资支出。其他
条件保持不变,由于收益递减,增加的资本积累$\Delta C=(C_2 - C_1)$引起利润率
从$P_1$下降到$P_2$,。并且,其他条件保持不变,技术发展意味着利润率提高,这能通过曲
线村向上移动到M'在图形上表示出来。因此,通过沿着水平轴的运动来表示资本积累增加,
通过曲线用向上移动来表示技术发展。在\cref{fig:marxlirun}所代表的例子中,技术发展
较多地抵消了与资本积累增加相连的收益递减,所以利润率由$P_1$提高到$P_3$,。容易看出,
存在其他两种可能;M'的移动正好使利润率保持不变,或者其移动使利润率随着时间的变化而
下降。再者,随着时间的变化,利润率将会怎样变动,只能通过参考经验信息来决定,而不
能通过纯理论来决定。不幸的是,度量某一经济体不同时期利润率变动的统计问题是非常有
难度的。


无论如何,马克思主张随看时间的变化,利润率将会下取,并且这是制度中生产力与生产关
系之间矛盾的一种表现。他声称利润率的下降是由资本家的活动引起的,因此,他们是制度
最终瓦解机制的一个组成部分。尽管在古典模型中,长期利润率的下降引起了一种静止状态,但
是,在马克思的模型中,它是引起资本主义瓦解的一个因素。此外,马克思有关利润率下降的
观点,构成了他的经济危机和产业集中程度提高理论的一部分,并且构成了“马克思--列
宁”关于帝国主义概念的一部分。

\subsection{经济危机的起因}

马克思对经济活动总水平波动原因的全部分析,包含在对资本主义制度内在矛盾更一般的描
述中。因此,说马克思自己的经济周期理论与他的追随者的理论相对立是不正确的。他提出了
很多经济波动的原因,但是,这些提议从来没有被清楚地描绘在他的作品中。然而马克思主
张,资本主义下生产力与生产关系之间的一个主要矛盾是内在于资本主义经济体中的周期性
萧条,这一点是没有问题的。尽管马克思本人没有清楚地区分他关于经济波动根源和性质的
不同见解,但是,为了清楚的缘故,我们在此将进行区分。

作为主要前提之一,古典经济学认同萨伊定律:除了总产量的较小波动之外,资本主义经济体
倾向于在充分就业的水平上运转。马克思抨击了这一古典观点,断言它呈现了一种对资本主
义扭曲和非历史的看法。马克思认为,在一个简单的物物交换的经济体中,人们生产产品,
或者是为通过直接消费这些商品而获得使用价值,或者是为了通过交换所生产的产品而获得
使用价值。在这些情况下,生产与消费是完全同步的。家庭生产鞋子是为了它自己使用,或者
用来交换要消费的食物。因而,经济活动或生产背后的全部动机就是获得使用价值。将货币引
进到这样一种经济体中,不一定改变生产的方向而使之远离使用价值。在一个货币经济体中,
人们生产他们用来交换货币的商品;货币接着被用来交换能给消费者提供使用价值的商品。在
这样的经济体中,货币仅仅是一种交换的媒介,它促进了劳动分工与贸易。这样两种经济体能
被示意性地表示如下:
% Please add the following required packages to your document preamble:
% \usepackage{booktabs}
% \usepackage{graphicx}
\begin{table}[htbp]
  \centering
    \begin{tabular}{@{}llr@{}}
     简单经济体 &  $C \rightarrow C$ & $C=商品$ \\
     货币经济体 &  $C \rightarrow M \rightarrow C$ & $C=商品$ \\
     资本主义经济体 &  $ M \rightarrow C \rightarrow M'$ &
    \end{tabular}%
\end{table}


简单经济体C一CC=商品货币经济体C一M一CM=货币但是,按照马克思的观
点,资本主义并不只是一个简单的或者说是物物交换的经济体,其中货币只是作为交换媒介
被添加进去。资本主义代表了经济活动方向的改变,从使用价值的生产转向了交换价值的生
产。指挥生产过程的资本家希望获得利润。他带着货币进入市场,购买不同的生产要素,并
且把它们的活动引导到商品生产上。然后,他在市场上用这些商品交换货币。他的成功用他
获得的剩余价值来度量,即他最初的货币数量与最终的货币数量之间的差额。资本主义的经
济体可表示为M—CM M'与M之间的差额AM就是资本家所实现的剩余价值。马克思不断地强调,资
本主义下经济活动的方向是交换价值与利润。他批评李嘉图对萨伊定律的认同,理由是萨伊
定律暗示了物物交换的经济体与资本主义经济体之间没有根本的区别,货币仅仅是促进了劳
动分工与贸易的交换媒介。在一个物物交换的经济体中,或者在一个货币只是一种交换媒介,
并且经济活动的方向是生产使用价值的经济体中,不会有生产过剩的问题。人们只在他们想要
消费这些产品,或者用它们去交换并消费其他商品的时候,才去生产产品。在资本主义下,经
济活动的方向是交换价值与利润,生产过剩就成为一种可能。马克思研究经济波动的基本方
法,是考察资本家对利润率变动的反应,即对“AM/M的比率”或者P变动的反应。马克思断定,
利润率的变动将引起投资支出的变动,并且他引证投资支出的这es i第7了7章卡秒,蕊更办
及其对古典经济学的批判人种波动,来表明它是经济活动总水平流动的主要原因。马殉思对
投资文出的兴趣,为很多现代宏观经济学家所共有。周期符环的波动循环周期是马殉思提出
的有关经济波动的一个模型。严国委织业引人注目的增长给马克思留下了深刻印象,他猜测
技术进步的爆发能够产生一个经济周期。技术进发将引起资本积累和劳动需求增加。后备军
的规模将减少,工资将上升,剩余价值将减少,剩余价值率将下降,利润率也将下降。随着经
济体螺旋向下进入萧条,利润率的下降将导致资本积累减少。但是,根据马克思的观点,萧
条中包含了迟早会形成经济活动新的扩张的因素。随着总产量下降,失业后备军的规模扩大。
这种闲置劳动的竞争压力降低了工资,并因此提供了较大的利润机会。这些较大的利润会激励
更多的资本积累,随着周期的上升阶段开始,经济活动将会增加。马克思提出,萧条自我纠正
的另一个方面是它们对资本价值的破坏。因为利渔是用货币计算的,所以,由于从周期的繁
荣阶段结转来的固定资产价值膨胀,无利可图的业务随着经济萧条期间资产价值降低变得有
利可图了。因技术进发而开始的周期,随着资本设备的消耗,未来可能会产生更多的周期。
如果随着时间的变化,所有的工厂和设备都被均匀地替换了,那么,将会存在一个不变的投资
水平来蔡代消耗掉的资本产品。然而,当技术进发期间投入使用的资本产品突然要求立刻置
换上时,就会形成一个置换周期。比例央调司机一旦经济体从物物交换阶段运行到高度的劳
动专业化以及利用作币和和市场的阶段,协调经济体不同部门的产量水平,就可能存在着困
难。在资本主义下,市场机制执行这一功能,,但是,马克思质疑市场平稳地再分配资源的能
力。假定行业A的产品需求增加了,行业B生产的商品需求减少了。在一个平稳作用的资本主义
经济体中,行业A的价格与利润将上升,行业B的价格与利润将下降。为了应对变化的利润,资
本家从收缩的行业向扩张的行业转移资源。行业B的超额供给,或者说生产过剩将因此持续一
个较短时期,对经济活动的总体水平不会有明显的有影响。被李嘉图称作pn, |)272 >WE想史
部分供过于求的一个行业中的生产过剩,不会扩散到经济体的其余行业并导致经济活动的总
体下降,或者说萧条。

马克思主张,在经济体的不同分市场上,供给与需求并不总是完全吻
合,因而资源再分配的全部过程,不会像在古典模型中那样平稳地发挥作
用。他的理论是,随着需求减少,B行业中产生的失业,将扩散到经济体其
余行业,并引起经济活动总体下降,这一观点与正统古典经济学家的倾向
直接相反。古典理论指望市场解决资源配置的问题。它强调均衡,认为非
均衡的状况只持续较短时间,均衡与非均衡之间会发生平稳的过渡。马克
叫假定了制度的不和谐,并寻找市场力量发挥作用时的基本矛盾。正统理
论并没有太多注意比例失调危机(disproportionalitycrises)理论,认为相对
于整个经济体来说,个别行业太小了,以至于一个行业生产过剩的扩散,
导致经济活动总体下降是不太可能的。他们还认为,资源的流动性比马克
思认可的要大。然而,如今像汽车这样的主要行业的生产过剩,可能会令
人信服地扩散到经济体的其余行业。
利润率下降与经济危机
到目前为止已经考察的马克思的两个经济危机理论,即周期循环的流
动与比例失调危机,清楚地否定了萨伊定律。马克思将他的利润率下降规
律结合到这两个理论中。因此,他的如下理论:当技术不能平稳发展时就
会导致萧条的理论,因为一个行业的生产过剩能对经济体的其他行业产生
不利影响,所以将会发生比例失调危机的理论,以及利润率将稳步下降的
理论,全部都是某个综合观点的不同方面,这个观点即资本主义不能在资
源充分利用的状态下,提供稳定的经济活动水平。:

马克思对萧条一一或者说危机,就像他的说法那样一一还有另外一种
不寻常的解释,这种解释认同了萨伊定律。他说,即使我们做了所有必要
的假设,以便能够让萨伊定律成立,资本主义仍然会由于内在的导致经济
危机的矛盾而失败。在马克思的模型中,资本主义经济体无疑取决于资本
家的行为,资本家对利润率变化和预期利润变化的反应,是解释经济危机
的重要部分。马克思利用他的利润率长期连续下降规律来解释经济活动的
短期波动,断言资本家在寻求较高利润的过程中,增加了资本支出从而导
heeh
ny
Wo
Bo
第7章卡尔.马克轧及其对古典经济学的揪关
致利润率下降。资本家将通过减少投资支出,周期性地对利润率的这种下
降做出反应,这将引起经济活动的波动并造成危机。因此,即使在一个认
同萨伊定律的模型中,马克思也推论出了和危机。
经济庆机一个总结
马克思对经济周期根源和性质的解释,与他对资本主义更广泛的分析
缠绕在一起。他没有完全接受任何一个理论并开发其全部的内涵和含义。
关于马克思对经济周期理论所作贡献的性质与重要性,引起了马克思主义
者自身之间以及经济思想史学家之间的很多争论。尽管马克思的各种危机
理论的相对重要性为经济思想史学家们所争论,然而也存在一般的共识,
即认为马克思的确就经济活动的波动提出了三种独特的解释利润率的下
降、新技术的不均衡使用、经济体一个部门发生的比例失调经扩散引起经
济活动总体水平的下降。马克思的作品也包含了一些更加含糊的提示,这
些提示涉及经济波动的消费不足主义解释,但是这些研究从未继续下去。

尽管马克思没有充分地发展其经济危机理论,但他明确指出,经济活
动的周期性波动是资本主义经济体的基本组成部分,也是导致资本主义最
终灭亡的基本矛盾的更多表现。他将这些周期性的波动视为制度所固有的,
原因在于它们是以资本家的活动为基础的,而资本家寻求利润并对利润率
的变动做出反应。认识到这一点也是很重要的。无论马克思的经济危机理
论如何表述,毫无疑问的是,马克思的如下观点代表了将资本主义作为一
种经济制度的重要见解,这个观点就是,由于存在内在矛盾,资本主义本
质上是不稳定的,面临着经济活动的周期性波动。然而马克思对资本主
义内在不稳定的看法,在20世纪30年代之前,被正统经济理论极大地和忽
视了。
资本积罕与资本集中
尽管基本的马克思模型假设了完全竞争的市场,市场上每个行业中和丰
在大量的小厂商,然而,马克思也意识到了厂商规模的逐渐增大,其后果
是竞争前弱、芍断力量增大。马克思断定,这一现象来自于资本积聚和资
本和集中的增加。当个别资本家积累越来越多的资本时,就发生了资本积聚
923
~mm
el
Ca
ds
1
eh
MWES
的增加,从而增加了他们所控制的资本的绝对量。厂商或生产的经济香位
的规模相应增加,市场竞争度趋向于变弱。

竞争减弱更重要的原因是资本的集中。资本集中是通过将已经存在的
资本进行重新分配而发生的,方式是使资本的所有权和控制处于越来越少
的人手中。马克思认为,较大的厂商能够获得规模经济,因此,与较小的
厂商相比,能以较低的平均成本进行生产。大的低成本厂商与小厂商之间
的音争,将会导致小厂商被排挤掉,从而产生芍断。
竞争的战役是通过商品降价打响的。如果其他条件相同,商品的廉价
取决于劳动生产力,而劳动生产力又取决于生产的规模。因此,大资本战
胜小资本,了D
资本集中的增加,通过信用制度和经济组织的公司形式得到加强。从
管在马克思所处的时代,公司才刚刚开始显示其重要性,然而,马克思对
公司经济的成长所带来的一些长期后果表现出了非凡的洞察力。公司资本
主义(corporatecapitalism)的特征在于如下事实,即它的
企业呈现出有别于个人企业的社会企业形式。它是在资本主浆生产上自
身限度内,对作为私人财产的资本的废除。实际执行职能的资本家向纯粹
的管理者、他人资本的管理人转变,资本所有者向纯粹的所有者、纯粹货
币资本家转变,@
马克思的观点是:资本积累、规模经济、信用市场的成长以及公司在
经济组织中所占的优势,将引起资本积队,同时集中到越来越少的人手中。
竞争将会通过自身的灭亡而结束,大公司将呈现出垄断实力。随着大公司
字现所有权与控制权的分离,将会出现很多不合意的社会后果:
@参见卡尔马克思的《资本论:政治经济学批判》一书第3卷第一部分的第686矶。该书
原版由弗里德里奇*恩格斯编辑,本文所引版本译自塞缪尔.摩尔和爱德华埃夫林编辑的德语第
3版,校正并补充自欧内斯特*安特曼编辑的德语第4版。该版本申美国查尔斯H*克尔出版公司
于1926年出版。




%%% Local Variables:
%%% mode: latex
%%% TeX-master: "../../main"
%%% End:
