\chapter{经济计量学与经济学经验方法的发展}

经济学研究的是真实世界的事件。因此,下列事实并不令人感到奇怪,即大多数关于是否
应当接受这种而不是那种经济理论的争论都涉及经验方法,这些方法将关于经济过程的理论
观点与对真实世界的观测相连接。问题大量存在。存在将理论与现实相连接的方式吗?如果
有这样的方式,是否不止一种?对真实世界的观测,提供了对理论的有效检验吗?与非正式
的启发式的识别力相对应,对经济现象直接而有目的的观测,能在多大程度上推进我们对经
济事件的理解?考虑到数据的不明确,形式上的理论化只是在玩花招吗?经济学应当更多地
关注直接观测与常识吗?在本章中我们将简要地考察经济学家在这些问题上所做出的努力。
他们的努力开始于简单观测,然后转向统计学,再转向经济计量学,最近转到了校准、模拟
以及实验研究上。

对于经济学经验方法的争论,有微观经济上的看法,也有宏观经济上的看法。微观经济看法
在极大程度上涉及生产函数与供求曲线的经验检验;宏观经济看法一般涉及宏观经济关系及
其与个人行为相联系的经验检验。宏观经济检验问题包括所有的微观经济问题,外加其他更
多的问题,所以,与微观经济学的经验研究相比,对宏观经济学的经验研究争论更多就不足
为奇了。

我们首先对不同经济学家所使用的四种经验方法进行一般性的陈述,以此开始我们的考察,
然后着眼于经济学家将统计研究与非形式化的观测相结合的早期尝试。接着我们来了解,针
对数据统计处理这一问题,合理但特别的决策是怎样做出的,并且怎样导致了经济学分支一
经济计量学的发展。最后,我们考察这些较早的特别决策,怎样使得一些经济学家对经济计
量研究以及今天经验经济学的未决状态进行冷嘲热讽。

\section{经验经济学}

几乎所有的经济学家都认为,经济学最终必定是一门经验学科,他们大于经济体如何运转的
理论,必须与真实世界的事件及数据相连(如有可能的话,再进行检验)。但是,关于如何
去做,以及能得出什么含义,经济学家之间则有很大的分歧。我们将区分把理论与真实世界
相连的四种不同方法:通常经验论、统计分析、古典经济计量方法以及贝叶斯经济计量分析。

\textbf{通常经验论}是借助最少的统计手段,通过对真实世界中事件的直接观测,将理论与
现实相连的一种方法。你着眼于周围的世界,并确定它是否与你的理论主张相符,这是19世纪
后期之前,大多数经济学家处理经济问题的方式。在那之前,大多数经济学家并没有接受过
多少统计方法的训练,运用统计方法所必要的数据也不存在,我们现在认为理所当然的很多
标准的统计方法尚未得到发展,计算能力也有限。

通常经验论有时被轻莽地称作闭门造车的经验论。这一带有贬义色彩的术语表达了下列意义,
即某人坐在书桌旁开发了一种理论,然后有选择性地挑选数据和事件来支持这一理论。支持
通常经验论的人反对这种描述,原因在于这种方法包含了仔细的观测、多领域的研究、案例
分析,以及与经济事件和所研究的制度的直接接触。通常经验论的支持者认为,能够对个人
加以训练,使之面对现实世界中的广泛事件;个人能够客观地评价他们的理论是否与那些事
件相符。通常经验论要求经济学家以经过专门训练的眼光,不断地观察经济现象,从而留意
其他人错过的东西。并不存在一个精确的划分界限来最终确定是应当还是不应当接受一个理
论,但确实存在一个不精确的界限。如果你预期一种结果,但出现的却是另一种结果,你就
应当向理论提出质疑了。研究者为他自己的诚实度设置了分界线。

\textbf{统计分析}方法也要求着眼于现实,但是,它强调能够被量化从而能面对统计度量和
分析的事件。其关注的焦点经常放在统计分类、度量以及描述经济现象上。这种方法有时被
戏称为没有理论的度量(measurement without theory)。该方法的支持者反对这种描述,
认为它只不过是考虑到很多理论中的可能性,并允许研究者选择最相关理论的一种方法。他
们声称,这种方法能够防止预先考虑的理论引导对数据的解释。

统计分析方法与通常经验论非常相似,但是,与通常经验论不同,统计方法运用任何可以利
用的统计工具和技术,从每一组数据中得出全部的认知。它并不试图将数据与理论相连接;
取而代之的是,它让数据(或者分析数据的计算机)来说话。随着计算机技术提高了研究者
从统计上分析数据的能力,通常经验论方法与统计分析方法就分离了。

\textbf{古典经济计量方法}(classical econometric)是一种将理论与数据直接相连的经
验分析方法。研究者通常的敏感性,或者他对现象的理解,在实验分析中只起着非常小的作
用;古典经济计量学家只不过是能使数据对理论进行检验的技术员。这种方法是利用了古典
统计方法来从形式上检验一种理论的有效性。经济计量方法形成于20世纪30年代,现在它是
现代经济学系所讲授的最具代表性的方法。其历史是本章主要关注的内容。

\textbf{贝叶斯方法}直接将理论与数据相连,但是,在任何统计检验的解释中,它所采取的
观点都是:检验并不是确定的。它以统计学的贝叶斯方法为基础,该方法并不是将或然性法
则作为一种客观法则,而是作为主观信仰程度来加以探讨。在贝叶斯分析中,统计分析不能
被用来确定客观事实;它只能是得出主观判断的一种辅助手段。因此,研究者只不过是运用
统计检验来修正他们的主观看法。贝叶斯经济计量学是通常经验论的一种技术扩展。在贝叶
斯经济计量学中,数据与数据分析并不回答问题,它们只是辅助研究者进行判断的工具。

这些方法并不完全相互排斥。例如,在最初开发一种理论时,人们可以利用通常经验论,然
后运用经济计量学来检验理论。类似地,贝叶斯分析要求研究者借助一些可供选择的方法,
例如通常经验论,得出他们自己偏好的结论。然而,统计学的贝叶斯方法与古典解释是相互
排斥的,每个研究者最终必须选择一种或者另一种。

技术不仅影响经济体本身,而且影响经济学家用来分析经济体的方法。因此,计算机技术在
经济学家接近经济体并进行经验研究方面具有重要的影响就不足为奇了。正如一位观察家所
指出的:如果汽车经历了与计算机同样的技术进步,法拉利将只卖到50美分。这还不能改变
你的驾驶习惯吗?计算机无疑改变了经济学家的经验研究,未来它会更加如此。

在一些情况下,技术仅仅使得我们已经在做的事情变得更容易做了。例如,统计检验现在能
够借助计算机在形式上得以完成。具有更复杂力学的递归系统正在寻找更多的观众。贝叶斯
度量开始在标准计算机软件统计程序中露面。另一群经济学家正在运用一种向量自动回归
(VAR)万法。他们只是照看着计算机,以期找到独立于任何理论的数据模式。

另一组变化比演进更加革命。最近,一群实验经济学家一直在更多地关注代理人建模。这些
都是模拟,其中,不同种类代理人的地方化的个人最大化目标被具体化和模型化。但是结果
被加以模拟,以确定继续存在的策略,而不是通过演绎来确定。在这些模拟中,允许个人逐
步确立制度并进行联合,并提供了一个与真实世界现象更加接近的相似物。

我们了解到的另一种变化是发展与利用一种称作宏观经济模型校准的方法。模型不再进行经
验检验;取而代之的是,它们被加以校准,来看经验证据是否与模型预测的结果相符。在校
准中,除简单的动态时间序列平均数之外,也强调简单的一般均衡模型的作用,其参数是通
过内省确定的。作为经验研究的一个主要目标,统计上的“适合”明确地被否定。就校准所
准确表明的内容而言,存在着争论,但是,如果一个模型不能被校准,它就不应该被保留。

最后一个变化是经验研究的“自然实验”方法的发展。这种方法利用来自直觉的经济理论,
而不是结构模型,并且运用自然实验作为数据点。

\subsection{数理经济学、统计学以及经济计量学}

在我们考察经济计量学的发展之前先人简要地考虑一下数理经济学、统计学以及经济计量学
之间的区别是值得的。它们经常被集合在一起,虽然不应当这样。

数理经济学(mathematical economics)这一术语仅指将数学方法用于阐述假设。它是用来
形成假设并阐明其含义的形式化且抽象的分析。统计学(statistics)这一术语指用数字表
示的观测资料的集合,统计分析指的是运用得自概率论的统计检验,来获得对那些用数字表
示的观测资料的见解。经济计量学(econometrics)结合了数理经济学与统计分析,前者用
来阐述假设,后者用来在形式上检验假设。这种结合并不是对称的;人们不用从事经济计量
学就能从事数理经济学,但是,不首先从事数理经济学,就不能从事经济计量学。只有数理
经济学,才能给予人们一种足以明确地在形式上被检验的理论。

我们能够看到数理经济学与统计学在历史上的分离。19世纪后期,最强烈反对经济思想数学
阐述的经济学家,是德国历史学派以及美国制度学派的先驱者。这群人中包括数据收集和统
计分析的一些积极倡导者——他们认为,在对真实世界的某种现象进行有意义的理论概括之
前,必须了解这种现象表明了什么。另外,那一时期很多正规的理论家对于运用统计分析犹
移不决。例如,马歇尔与埃奇沃斯都对运用统计度量需求曲线的能力犹疑不决,他们认为,
用来解析性地导出曲线的其他条件保持不变的假设,使得曲线难以量化。埃奇沃斯在帕尔格
雷夫经济学大辞典(1910年版〉中论述需求曲线时写道:“杰文斯运用统计学构建需求曲线
的希望是否能够实现,可能令人怀疑。”

经济学家希望从数理经济学中获得的,是假设检验的精确性,这种精确性使得减少检验的含
糊性成为可能。例如,他们希望不依赖于常识以及对需求曲线向下倾斜的通常启发式的理解,
就能够在经验上证明需求曲线向下倾斜。在对经济理论进行数学阐述之前,经济学家运用文
字来陈述经济理论和假设,一般通过与现有情况或者历史事件相比来进行一般性假设的检验。
但是,在两种情形下,统计学的运用都是最小限度的。这种本质上是诱导式的方法,并没有
使得假设以一种为正规的经济学家所认同的方式被加以检验。

20世纪60年代与70年代,形式上的统计检验以及对经济计量方法的了和解都有了巨大进展。
计算机技术的进步使得进行极为复杂的经验研究成为可能。较早要花几天时间完成的统计检
验,现在可能只需要几秒钟。在那一时期,对经济计量学的期望很高。一些人认为,经济计
量学将使经济学成为一门科学,其中,所有的理论都能被加以检验。在此期间,逻辑实证主
义与波普尔证伪主义是占统治地位的方法。人们认为,过去的错误——以一种导致理论无法
检验的方式来阐述理论——能够被加以避免。时至今日,最初的这些愿望大多数尚未得到实
现。

\subsection{早期的经验研究}

为需求关系增加经验基础的早期尝试是格利高里·金(Gregory King,1648--1712)的研究,
他重申了查尔斯·戴维南特(Charles Davenant,1656--1714)的一些研究。戴维南特在《论
如何使一个民族在贸易平衡中成为获利者的可能方法》中,描述了价格与数量之间大致的相
反关系。在书中,他提出了下列法则。我们认为,歉收可能会以下列比例提而谷物的价格:

\begin{table}[htbp]
  \centering
    \begin{tabularx}{.9\linewidth}{@{}|l|X|X|X|X|X|@{}}
      \hline
    歉收    & $1/10$ & $2/10$ & $3/10$ & $4/10$ & $5/10$ \\ \hline
    价格上涨 & $3/10$ & $8/10$ & $1.6$ & $2.8$ & $4.5$ \\ \hline
    \end{tabularx}%

    \bigskip \raggedright\small 所以,当谷物价格上升到通常价格的三倍时,可以假定
    歉收了通常产量的1/3以上;如果歉收了通常产量的5/10或者说一半,价格将上升为通常
    价格的近五倍。
\end{table}

\begin{mybox}{可检验与不可检验的理论:马尔萨斯的人口理论}
  马尔萨斯对人口理论的陈述是不可检验理论的一个很好例证。在其《人口原理》第一版中,
  马尔萨斯提出了下列假设,即长期中人口趋于以比食物供给更快的速度增加。这一假设极
  有可能在统计上被予以反驳。

  然而,在其《人口原理》第二版以及后来的版本中,马尔萨斯又说,人口主题不能在经验
  上进行检验;他补充了一种无法度量的人口控制,即“道德控制的增长”,指的是婚姻的
  推迟以及婚前性行为的节制。当把道德控制作为对出生率的控制增加到理论中时,观测到
  的人口增加可能与上升的、下降的以及不变的人均收入相结合,并且仍然符合理论。因此,
  马尔萨斯人口理论的这一版本,变得无法进行经验检验。
\end{mybox}

\section{新古典经济学与经验分析}

经验研究的早期尝试是例外而不是惯例。17世纪后期,大多数古典经济学家采取通常经验论
方法。他们就经济体如何运转的规律进行假设,并用实例来支持这些规律。因为不存在公认
的对一种理论的检验,所以,有关什么理论是正确的争论就一直进行着。

19世纪后期,随着新古典经济学的开始,这种方法出现了问题。正如我们在第三部分看到的
那样,新古典理论变得比较形式化,存在着关于经济学成为一门精密科学的论述。这意味着
经济学家经验研究的方法形式化,讨论得最多的是自身正在经历一场革命的统计分析方法。

新古典经济学家采用很多不同的统计分析方法。例如,威廉,斯坦利.杰文斯把统计学看做是
将经济学变成一门拥有精密法则的精密科学的方法。另外,莱昂·瓦尔拉斯则很少从事经验研
究;他不断地发展其与经验检验可能性无关的理论。阿尔弗雷德.马歇尔相信经验研究,但并
不进行形式上的统计分析,他把直接观测与通常经验论看做是收集经验信息的最有效方式。

\subsection{享利·L·穆尔}

19世纪末期20世纪初期,统计方法与概率理论的重大研究,使得它们被运用到经济学中。将
形式化的统计方法运用于经济学的最早提倡者之一是亨利·L·穆尔(Henry L.
Moore,1869--1958)。20世纪初期,穆尔最先使用了很多统计方法,这些方法后来都成了标
准。穆尔应用了弗朗西斯·高尔顿盟士、卡尔·皮尔森以及其他人的统计研究。这些统计学家
证明,有可能在一个可控制的环境中,运用复相关与相依表,根据统计数据,在形式上确定
推论。这项研究给称穆尔留下了深刻印象,他断定有可能将这些统计方法应用于经济理论的检
验中。

不像杰文斯那样仅仅“目测”同一坐标方格中的两张图表,穆尔在形式上比较了两个系列的
数据,并发展了统计学,使之能够为他提供有关这两组数据之间关系的信息。然而重要的是,
要注意到他这样做,是对皮尔森研究的勇敢突破,后者仅仅分析在其他实物影响能够被加以
控制的环境中所进行的研究。穆尔没有这种奢求,因为在经济学中受控制的实验通常是不可
行的;因此他假定在受控制的实验中使用的统计方法,在不受控制的环境中也有效。

他特别有兴趣进行检验的理论是约翰·贝茨·克拉克的工资的边际生产力理论,该理论预言个
人将被支付其边际产品。为了达到检验上目的,穆尔研究了工资与边际生产力、个人能力、
罢工以及行业集中之间的关系。克拉克的理论意味着:(1)与低能力的个人相比,具有较高
能力的个人将被支付更多工资;(2)能力相当的个人在垄断性行业和竞争性行业工作时,在
垄断性行业工作的人将被支付更多工资;(3)与非集中性行业相比,集中性行业为了较高工
资而进行的罢工更有可能获得成功。

穆尔发现了能力与工资之间的关系,但是,他的分析中也存在着重大问题。在其检验中,穆
尔并没有非常严格地详细说明他的理论结构。例如,在某个检验中,他运用平均产品而不是
边际产品进行检验,结果并没有检验实际工资,而是检验了货币工资。穆尔也发现了罢工与
行业集中之间的关系,但是,这种关系只基于有限的数据。

穆尔的统计研究也是有问题的,原因在于,他不仅对简单地从科学上检验克拉克的理论感兴
趣,对与政策相关的问题也有着浓厚的兴趣。他希望利用他的统计分析来反对要求更多收入
平等的社会主义政策建议。有“私心”并不一定使理论或经验研究的结果无效,但是,人们
确实质疑含糊不清的结果是否将被合理地加以阐释。发觉一个理论家或经济计量学家的动机,
并不是对一种假设或理论有效性的检验。研究有时发生在受到资助的智囊团中,这些智囊团
反映了特定的意识形态,只要研究结果成了能被所有人检验的公共财产,那么,重大的偏差
就有可能被指出来。

穆尔的早期研究使他被确立为将统计方法与经济学相结合的领导者。他后来的贡献——一个是
对需求曲线的经验度量,另一个是对经济周期的度量——也都很重要:前者奠定了现代微观经
济计量学的基础,后者构成了现代宏观经济计量学的基础。

\subsection{穆尔的需求曲线与确定问题}

穆尔最为著名的研究也许是他对农产品和生铁需求曲线的估计。对其贡献进行仔细分析是有
正当理由的,原因是它指出了经验估计中的很多问题,这些问题在后来的争论中起到一定的
作用。

考虑一下从经验上度量一条需求曲线的困难。市场观测值是交易发生时价格与数量的结合。
如果市场处于均衡状态,观测到的价格与数量就是既在供给曲线上也在需求曲线上的点;如
果市场不处于均衡状态,观测到的价格与数量可能在供给曲线上,也可能在需求曲线上,或
者两条曲线上都没有。研究者怎么知道是哪种情形?如果研究者能够进行可控制的实验,并
保持所有其他条件不变,就像在下列方程中一样:

\begin{gather*}
Q_D=f(P_Q,所有其他产品的价格,偏好,收入,…)\\
Q_s=g(P_Q,生产要素的价格,技术,…)\\
Q_e=h(f,g)
\end{gather*}

方程中,除了价格与数量之外,其他任何条件都保持不变,然后,人们就能度量价格与数量
之间的实际关系。但是,在做不到这一点的地方(借助经济统计学无法做到的地方),研究
者不知何故一定要将观测到的价格与数量的数据和理论相连。这里就展现出了经济计量问题
的核心:将从不可控制的实验中观测到的数据与理论相连。

在他对农业市场的分析中,穆尔乐于接受下列假设,即市场趋向于均衡,所以观测到的价格
与数量能被假定为是均衡价格与数量即 $P_e$和 $Q_e$,它们是既在供给曲线上又在需求曲
线上的点。在\cref{fig:equiliass}中能够看到这一假设。它使我们假定,观测到的点是诸
如$(P_e,Q_e)$而不是 $(P_1,Q_1)$的点,在点 $(P_1,Q_1)$上,市场处于向均衡调整过程中
的非均衡。

\begin{figure}[ht]
  \centering
  \begin{tikzpicture}[thick, scale=0.8]
    \draw (0,0) node[below left] {$O$};
    \draw[very thick] (0,6) node[above] {价格} -- (0,0)  --  (6,0) node[right] {数量};

    \draw [name path = D, shift={(8,7.09)}] plot[domain=3.44:4.24,variable=\t]({1*6.6*cos(\t r)+0*6.6*sin(\t r)},{0*6.6*cos(\t r)+1*6.6*sin(\t r)}) node[below right]{$D$};
    \draw [name path = S, shift={(-4.83,11.9)}] plot[domain=5.22:5.69,variable=\t]({1*12.26*cos(\t r)+0*12.26*sin(\t r)},{0*12.26*cos(\t r)+1*12.26*sin(\t r)}) node [above] {$S$};

    \path [name intersections={of=D and S,by={X}}];
    \draw [dashed] (X) -| (X -| 0,0) node [left] {$P_e$};
    \draw [dashed] (X) |- (X |- 0,0) node [below] {$Q_e$};

    \coordinate (X1) at ([xshift=0.6cm, yshift=-0.9cm]X);
    \draw [dashed] (X1) -| (X1 -| 0,0) node [left] {$P_1$};
    \draw [dashed] (X1) |- (X1 |- 0,0) node [below] {$Q_1$};
  \end{tikzpicture}%
  \caption{\label{fig:equiliass}均衡假设}
\end{figure}

穆尔也乐于假设,对于农业商品来说,供给外生地由夏季降雨量来决定,所以,不受当前收
获时期价格的影响。他进一步含蓄地假设,过去的事件对于供给和和需求没有影响,并且,
变化着的预期对实际数据的确定不起作用。这些假设改变了模型的图形,
由\cref{fig:moreassu}呈现出来。因为供给量被假定为外生决定,所以,所估计的点
$(P_1,Q_1)$和$(P_2,Q_2)$一定是需求曲线上的点。

\begin{figure}[ht]
  \centering
  \begin{tikzpicture}[thick, scale=0.8]
    \draw (0,0) node[below left] {$O$};
    \draw[very thick] (0,6) node[above] {价格} -- (0,0)  --  (6,0) node[right] {数量};

    \draw [name path = D, shift={(8,7.09)}] plot[domain=3.44:4.24,variable=\t]({1*6.6*cos(\t r)+0*6.6*sin(\t r)},{0*6.6*cos(\t r)+1*6.6*sin(\t r)}) node[below right]{$D$};
    \draw [name path = S1] (3,0) node [below] {$Q_1$} -- (3,5) node [above] {$S_1$};
    \draw [name path = S2] (4,0) node [below] {$Q_2$}-- (4,5) node [above] {$S_2$};

    \path [name intersections={of=D and S1,by={XS1}}];
    \draw [dashed] (XS1) -| (XS1 -| 0,0) node [left] {$P_1$};

    \path [name intersections={of=D and S2,by={XS2}}];
    \coordinate (X1) at ([xshift=0.6cm, yshift=-0.9cm]X);
    \draw [dashed] (XS2) -| (XS2 -| 0,0) node [left] {$P_2$};
  \end{tikzpicture}%
  \caption{\label{fig:moreassu}穆尔关于外生供给的假定}
\end{figure}

为了完成其分析,穆尔将数据表示为围绕某一趋势的百分比变化,并按照百分比的变化得出
需求关系。他提出了一条线性方程的需求曲线和一条三次方程的需求曲线。线性需求曲线具
有一般形式:$P=a-bQ$,式中 $P$为价格,$a$是需求曲线的价格截距,$b$是需求曲线的斜
率,$Q$为数量。负号的 $b$系数表明向下倾斜的需求曲线。穆尔运用下列系数,估计了两种
不同的曲线:

\begin{gather*}
  \Delta P/P_{t-1}= 7.8 - 0.89 \Delta Q/Q_{t-1}\\
  R^2=0.61, s=16
  \shortintertext{以及}
  \Delta P/P_{t-1}=1.6-1.1 \Delta Q/Q_{t-1}+0.02(\Delta Q/Q_{t-1})^2 -
  0.0002(\Delta Q/Q_{t-1})^3\\
  R^2=0.71, s=14
\end{gather*}
注意到在两种情形中,需求曲线都具有理论所预言的负号(它是向下倾斜的),并且具有
相当高的可决系数。

穆尔所估计的需求曲线并没有立刻给他带来赞誉;很多人不明白他的成就,确实明白的一些
人(例如埃奇沃斯)断言,考虑到基础理论的复杂性,经验需求分析显得过于简单。埃奇沃
斯主张,构成结论基础的未经检验的假设非常多,拘泥于形式并没有什么好处。这些批评尽
管是实质性的,且仍然在不同程度上被视同反对经济计量研究,但是这些批评不应当被看成
是贬低穆尔的贡献。他是从统计上度量需求曲线的最早的经济学家之一,虽然像南希·伍维
克指出的那样,尚不清楚穆尔是否打算估计一条传统需求曲线。

对穆尔农业需求曲线估计的淡漠认同,与对其生铁需求估计的淡漠接受相比,前者相当于一
种积极的恭维。穆尔声称,生铁的需求曲线斜率为正数,所以,当价格上升时,需求量也上
升。他提出了下列需求方程:
\[\Delta P/P_{t-1}=4.48+0.5211 \Delta Q/Q_{t-1}\]

穆尔声称发现了一条正斜率的需求曲线,从而直接反对微观经济理论,这引起了强烈的批
评性回应。

考虑到穆尔作为一位经济理论家所具有的复杂头脑,伍维克指出,穆尔的正斜率需求曲线并
不是错误造成的结果,也不是未能理解确定问题(为了估计另一条曲线,需要保持供给或需
求不变)的结果。根据伍维克的观点,它代表了解决数据局限性的一种尝试,并允许这些
局限引导其分析,而不是让理论分析引导其经验研究。这一观点得到了下列事实的支持,即
在穆尔的著作中,他很清楚其需求曲线并不是根据马歇尔的理论得出的一条典型的需求曲线,
而是一条与经验规律有关的动态需求曲线,涉及很多交互式变化。

很多相互影响能够使穆尔的动态需求曲线符合静态需求理论。例如,当生铁价格上升时,总
收入与经济活动有可能增加,这就与需求的增加相连。因为不可能外生地说明生铁的供给
(估计静态需求曲线时,这是必要的),所以穆尔认为,他的动态需求曲线抓住了经验规律,
从而在对经济体进行预测时是一种有用的工具。

穆尔认为,尽管不能外生地说明供给,但人们可以估计一条曲线,它包含了对可度量的供给
相关变动的正常反应。这些正常反应包括以一种前后一贯的方式使静态需求曲线变动;并使
得被度量的且包含相互依赖关系的动态需求曲线向上倾斜。如果后来的这些相互依赖关系是
正确的,那么,无论何时我们看到主要行业产品供给外生地增加,我们都将会预期到这些行
业的产品价格上升,而不是下降。这就是穆尔的结论。穆尔认为,基本上不需要将这一动态
需求曲线与基础性的静态理论相连,原因在于,这样做只是一种训练,并不能令人信服。他
写道:

\begin{quotation}
  根据统计方法即其他条件保持不变的方法,解释现象时所遵循的适当过程是:在理论上依
  次调查每种要素对价格的影响,假定其他条件保持不变,然后最终进行综合!但是,如果发
  生每种要素与价格的联系,那么,其他条件保持不变的假设,就涉及大量的且至少是有问
  题的假设,当人们谈到多种影响的最终综合时,难道不会完全使自己迷失在含蓄假设的路
  途中吗?我们不采用这种令人困惑的方法,而是遵循着相反的过程,完全具体地着手处理
  价格与供给关系的问题。

  统计相关性理论的富有成效,与上述方法的总无成果,形成了强烈对比,两种方法在处理
  多重影响问题上遵循着相反的过程。例如,考虑一下天气对农作物的影响问题。在假定
  方式下,假定其他未被列举的天气因素保持不变,试图解决关于降雨量对农作物的影响问
  题,这是怎样一种毫无价值的思考?有关温度影响的问题,也能保持其他条件不变吗?最
  后,一种综合怎么能由多种个别影响组成呢?多重相关性的统计方法则没有带来这种无益
  的问题。它直接质询农作物与降雨量之间的关系是什么,不用保持其他条件不变,其他事
  物根据它们的自然状况而变化,
\end{quotation}

穆尔研究中的合理因素涉及经济计量学中一些尚未解决的相关问题。它们提供了关于制度主
义学派经验方法的观点,即数据应当引导理论分析,而不是理论引导经验研究。经济学专业
越来越意识到静态分析的局限性,并开始将这种分析与经验观测相结合,近来为穆尔研究所
做的辩解就是在这样一种氛围中进行的。在穆尔所处的时代或20世纪中期这种辩解并没有出
现。穆尔受到了来自两个方面的抨击——那些反对形式化的理论与经验研究的人,这些人认为
他的统计方法太复杂,以及那些赞同形式化的理论与经验研究的人,这些人认为穆尔未能将
足够的注意力放在理论上。

尽管在专业中留下了不可磨灭的痕迹,但穆尔忍受着对其向上倾斜需求曲线的奚落,并最终
放弃了经济计量研究。这使得他的学生来完成经验革命。在他的学生中,最著名的是享利.舒
尔茨(Henry Schultz,1893一1938),他的《需求与供给的统计法则》(1928)和《需求理
论及计算》(1938)在现代微观经济计量学的发展中发挥了重要作用。

\subsection{亨利·舒尔茨与自变量和因变量}

亨利·舒尔茨的贡献是从他对关税的分析中派生出来的,对关税的分析要求估计一条需求曲线。
当试图估计需求曲线时,舒尔茨获得了一个引人注意的发现:正如穆尔所进行的研究那样,
通过价格对数量进行回归,而不是数量对价格进行回归,人们能够得到完全不同的弹性。在
论述这些问题时,舒尔茨认为,如果对于哪种变量是所要回归的正确变量(哪种是自变量,
哪种是因变量),人们有一种居先的观点,那么,这一观点将决定选择的正确性。然而,如
果人们并没有一种居先的观点,就没有办法在两者之间选择。舒尔茨认为,在这种情况下,
选择能够较好地符合由皮尔森卡方测验所确定的回归是最好的。

舒尔茨的见解是重要的;它意味着不能独立于理论来考察统计度量。你所看到的,部分地取
决于你所相信的。这一见解促成了当前经济计量学要求研究者仔细区分自变量与因变量的习
惯。

当然,说到统计度量的变化与理论有关,并不是说度量完全取决于理论。它不表明理论是决
定性的;理论只是提供了一种人们能够从统计学中得出的有限解释。

\section{宏观经济学与经验分析}

微观经济学的经验研究比较困难,但是,在宏观经济学中它更加困难,原因在于万事万物都
倾向于相互关联。最早的贡献之一是威廉·斯坦利·杰文斯做出的。

\subsection{杰文斯的日斑理论}

威廉·斯坦利·杰文斯是数理方法与效用理论的先驱者之一。他因这些研究而获得了高度的称
赞。尽管杰文斯现在最为人所知的是他对新古典理论的微观经济贡献,然而,他在度量宏观
经济关系方面的经验尝试,在经济计量学史中最为著名。他的研究是早期形式化宏观经济经
验研究的尝试之一。虽然他在微观经济学中的研究得到了赞誉,然而,他关于经济周期的宏
观经济统计研究并没有得到经济学界的充分认同。事实上,他经常受到奚落。

杰文斯对挖掘导致价格波动的贸易或经济周期原因感兴趣。因为循环行为看上去并不与个人
效用最大化行为相关,所以,他认为自然界中一定存在某种原因——一些引起波动的自然现象。
初步的研究使他认为,经济活动波动的原因很有可能是与天气有关的某种东西。他把注意力
集中在太阳黑子(太阳活动的周期性波动)上,将其视为可能的原因。\footnote{经济波动
  由太阳黑子引起的思想具有某种幽默光环。较早由穆勒提出的另外一种解释与“商业情
  绪”的改变有关,记住这一点很重要。为什么情绪会改变,或者为什么这种改变似乎存在
  某种规律,关于这些,并没有任何解释。考虑到这一点,对内生周期感兴趣的经济学家,
  就自然地转向诸如太阳黑子一类的自然现象。关于杰文斯太阳黑子研究的一篇优秀论文,
  参见桑德拉·J·皮尔特的《太阳黑子与预期》:威廉·斯坦利·杰文斯的经济波动理论”一文,该
  文载于《经济思想史杂志》1991年秋第2期第24--265页。}

杰文斯的具体假设是,太阳黑子循环以 $11.1$年为一个周期而发生,这些循环导致了天气的
循环,从而导致经济周期。为了验证他的理论,杰文斯着眼于13世纪和14世纪以来可供使用
的有关收成波动的农业数据。其后,他试图将这些收成波动与19世纪对太阳黑子活动的估计,
即$11.1$年一个周期的估计相连。他假设日斑循环的长度不变,通过在代表 $11$年的一个网
格上展示数据,并目测数据,对两者进行比较。他注意到了一种相对来说较好的“适合”,
循环看上去匹配。然后,他考察19世纪期间商业信用的周期,并发现平均周期是 $10.8$年。
他断定,经济周期的可能原因是太阳黑子。

杰文斯的日斑理论没有被19世纪的经济学家所采纳,大多数人认为它相当怪异。它之所以值
得提及,主要是因为它是运用统计学来形成并检验一种宏观经济理论的尝试,并因此为杰文
斯确立了经济计量方法先驱者的地位。

\subsection{穆尔对宏观经济计量学的贡献}

商业循环是一种持久的经济现象,因而,穆尔对估计农作物需求所做的贡献,因其宏观经济
贡献而得到加强,这就不足为奇了。在宏观经济计量学中,穆尔的主要贡献是,既提供了商
业循环理论,也试图从统计上上度量它。在分析商业循环时,穆尔的动态需求曲线获得了更
多的正当理由。尽管存在一种静态需求理论,然而,并不存在类似的商业循环理论。穆尔认
为,演绎的和其他条件保持不变的推论,对于解释这种波动是没有帮助的。

像杰文斯一样,称尔选择天气的循环作为经济波动的外生原因。他将这一观点与他向上倾斜
的生铁需求曲线相结合,作为对于商业循环的解释。他的五个论点如下所示:

(1)下雨增多且农作物产量增加;(2)贸易差额上升;(3)生产者产品的需求、价格以及数量上升;
(4)就业增加,因此农作物的需求增加;(5)物价上升。

当雨量减少时,过程相反。穆尔运用统计分析来支持这一主张。

穆尔的原创分析受到了菲利普赖特的批评,后者在1915年的一篇文章中,将雨量的度量调整为
与生长有关的雨量,而不是全年的总雨量。赖特的研究显示,统计关系中断了。赖特的观点
有力地动摇了穆尔的统计分析,这促使穆尔将其覆盖面扩展到更多的国家。在其扩展分析中,
穆尔发现持续存在着一个八年周期。1923年,他创作了关于这一主题的第二部更加精细的著
作,其中,他将天气设置为诸多经济和社会原因之一。

理论的扩展使之将多重原因包含进来,这使理论获得了合意的支持,但是他对天气周期的深
入分析,强调金星在八年间隔中位于地球与太阳之间,这导致他的理论被冠以“商业循环的
金星理论”的称号。穆尔研究商业循环的特定方法,没有被其他人所继续,但是,他为后来的
经济计量研究提供了一定的基础,并且为制度经济学家的分析奠定了基础。

\subsection{韦斯利·克莱尔·米切尔:非正统经验论者}

早期制度主义者之一韦斯利·克莱尔·米切尔在宏观经济学的经验研究问题上,与正统新古典
经济学家有重大不同。穆尔的研究提供了一种有益的焦点,我们能够从中考察米切尔的经验
研究方法,这种方法在20世纪上半叶得到发展,并且是全国经济研究局(NBER)最初采用的
方法。穆尔比较形式化的统计方法存在着问题,这是米切尔的方法获得偏爱的原因之一。

米切尔关于理论与实际分析之间适当关系的观点,在他关于经济周期
的星期研究中就表达出来了:
致力于了解当今经济活动周期性盛衰特征的人,会发现对于经济周期
的众多解释既是启发性的,也是令人困惑的。所有的解释都似是而非,但
是,哪一种是有效的?没有任何一种解释必定排斥所有其他解释,但是,


A466
哪一种是最重要的?每种解释都可以说明菜种现架;任何一种解释性能说
明所有的现象吗?这些相互竞争的解释,能够以某种方式结合起来,产生
一种一致的且完全充分的理论吗?

通过证明理论并批评理论这种逻辑过程来得到这些问题的答案,似乎
仅存渺蒜的希望。因为无论这些理论可能拥有怎样的独创性与一致性优点,
要不是由于它们就经济周期现象提出了敏锐的见解,它们都只有微小的价
值。正是通过研究它们声称要加以解释的事实来使理论得到检验。

但是,如果我们开始通过收集证据,依次检验每种理论来确认或者否
认它,就会扰乱研究视角。问题的要点不在于任何经济学家观点的有效性,
而在于对事实的清楚理解。对繁荣、危机以及匡条现象的观察、分析、系
统化是首要的任务。如果我们直接着手于这一任务,与我们考察理论现象
这一了迁回方式相比,会有更好的贡献。

直接着手于事实的这一计划,决不排除对其他人所得出结论的自由利
用。相反,他们的结论表明应当寻找菜些事实,应当进行某些分析,应当
尝试某些安排。的确,如果我们不从各个方面寻求帮助,整个研究将是粗
炮的和肤浅的。但是,希望得到的帮助是将一种新的考察变成事实。0
米切尔的方法注重实效;它并不把对理论进行实际检验这一重要作用
看做是一种有用背景,而是把理论看做是解释经验观察的一种有用背景。
与这一观点相符,米切尔并不将经济学看做是一门科学,而是看做有助于
政策形成的一门艺术。最终,对于米切尔来说,并不存在能在一个优雅的
模型中被详细说明的不变理论;经济体过于复杂并且正在经历不断的结构
变化。考虚到这种复杂的变化,创造一般理论就是在数往塞责;唯一可接
受的理论是所传授的常识,只有通过将常识与统计分析有机地综合,才能
理解经济体。

尽管就形式上的科学性而言,数据并不适合于检验理论,然而,它们适
合检验有关周期行为的不同假设。在后来的《测算经济周期》(MeasuringBus-
inessCycles,伯恩斯与米切尔,1946)中,米切尔检验了熊彼特关于不同周期
Q@参见韦斯利.克莱尔-米切尔的《经济周期及其原因》第3卷第19~20页,
在1913年的《加州大学论文集》中。





第16章经济计量学与经济学经验方法的发展


之间关系的假设,并否定了这种关系。他也检验了他目己的假设,即周期性
行为中存在一种长期的世俗变化,他和合作者也否定了这一假设。他们确实
发现了变化,但是,这些变化是不规则和任意的。因此,通过将形式上的统
计检验——例如相关性检验与显著性的检验——与基于制度知识与数据知
识的判断相结合,它们能够非形式化地“检验”假设。尽管在科学方法中,
形式化的检验决定了一种理论的有效性或虚假性,但对于米切尔来说,这样
的检验仅仅是对常识与主观判断的一种辅助。20世纪30年代期间,米切尔的
数据与经济经验分析方法,为美国的主流实观经济学家所运用。
一些数据例如煤炭价格,能被简单地加以收集和运用。迁合理论第构
的数据通常必须被加以构建。必须确定可以计量的概念,然后收集数据。
这项工作通常是困难且苟求的。我们来看一些例子。

经济学家运用价格总水平的概念,但是经济体中并不存在对所有价格
的度量。20世纪40年代以来,价格总水平的提高(通货膨胀)获得了大量
关注。在通货膨胀能被度量之前,关于价格总水平度量的构建已经进行了
相当多的研究。在《指数的编制》(TheMakingofIndexNumbers,1922)
中,欧文*费雪考察了打算用来度量价格与经济活动的指数在构建过程中
遇到的一些问题。寄希望于通过取消中间产品,并适当地赋权重于最终产
品,人们就能构建一种度量价格总水平变化的指数。借助于这种度量,有
可能为通货膨胀概念增加更多的准确性,并检验有关货币供给变化与价格
变化之间关系的假设。应当注意到,货币供给并不是作为简单的数据存在
并被加以收集和分析的,必须构建对货币供给的度量。很多经济学家将其
生命中的大部分时间花在度量与数据收集领域的研究上。

一些经济学家在国民收入核算领域做出了重要贡献。凯恩斯理论迫切
需要对国民收入、消费、支出、储蓄以及投资支出进行度量。在进行数据
收集的量化研究之前,这些宏观理论上的概念要求解决极其困难的概念性
问题。理查德.斯通鲜士(SirRichardStone,1913一1991)与詹姆士*米
德(JamesMeade,1907一1995)都是诺贝尔经济学奖获得者,他们为英国
发展了适合凯恩斯理论模式的国民收入核算体系。


经济上电

UY™

y
A68
在美国,国民收入核算的研究者是诺贝尔奖获得痢西过*库兹涅次,
他在米切尔的指导下创作了博士论文,并继续在米切尔的指导下在全国经
济研究局进行研究。库北湿芯的主要贡献是为美国构建了国民收入度量,
并且运用统计学来度量和比较不同国家的增长模式。库兹涅芯玫助创立的
国民收入核算,是后凯恩斯宏观经济计量模型的一个重要组成部分。

华西里:列昂惕夫也是诺贝尔经济学奖获得者,他在组织数据收集过
程中发挥了作用;他设计了投入产出分析,这是一种用来处理经济体相互
关系的实用计划工具。列昂惕夫强烈地不满于毫无经验内容的现代主流经
济模型构建。他提倡关注经济学的实际应用,运用数据进行研究而不是构
建复杂的数学模型。他的论文“理论假设与不可测的现实”(TheoreticalAs-
sumptionsandNonobservedFacts,1971),是对人们能够感受到的脱离现实的
模型构建最好的批评之一。

另外两位对数据收集做出贡献的经济学家是区布拉姆.伯格森(Abram
Bergson,1914一)与亚历山大:格申克龙(AlexanderGerschenkron,1904一
1978)。伯格森是一位有才能的理论家,他在24岁还是一位研究生时,便创
作了一篇福利经济学方面的经典论文。他成为美国苏联问题方面的主要专
家,就前苏联经济活动的度量进行了开创性的研究。在苏联解体之前,人
们经常听说苏联的计划者采用产生于美国的对苏联经济活动的度量,原因
在于,这些度量比他们自己的统计更可靠。在将哈佛大学俄罗斯中心建成
研究苏联社会的重要研究中心的过程中,伯格森起了主要作用。

像库兹涅芯一样,格申克龙出生在俄国,但20世纪20年代期间在维也
纳接受经济训练。他是库兹涅艾与伯格森在哈佛的同事。尽管他的书面作
品数量不多,但是,他是教授中的教授,他掌握多种语言,针对帕斯捷尔
纳克(Pasternak),以及针对纳巴科夫(Nabokov)对普希金(Pushkin)
《尤金.奥涅金》(EugeneOnegin)的翻译,发表了批评性文章。格申克龙
对增长的度量,尤其是对苏联增长的度量进行了重要的研究,并且表明了
工业生产指数中所使用的基年的选择是如何影响指数所表示的增长率的。
他的研究揭示出,苏联的增长并不像苏联计划者所显示的那样快,原因是
他和们的度量有偏差。

20世纪60年代,米切尔的宏观经济经验分析方法成为一种少数人使用
的方法,受到微观经济学与宏观经济学中的经济计量方法的排挤。很多理
由解释了为什么主流不再依赖米切尔的方法而是转向经济计量学:(1)统
计方法与经济计量方法的进一步发展,使它们避免了穆尔研究中的一些问
题;(2)经济学专业和整个社会在完成并检验理论过程中,对准确性有强
烈要求;(3)数理经济学迅速发展,(4)经济计量学使经济学变成一门精
密科学有了希望;(5)有才气且意志坚强的经济计量方法倡导者对这种方
法进行着宣传。
在微观经济学中推进经济计量方法的一项发展是上】次殉因(EE.
Working,1900一1968)的确定问题方法。价格与数量之间的简单相关性,
即使提供了数据的“良好适合”也几乎没有什么意义,原因在于,经济理
论规定价格与数量是由供给与需求的相互作用所决定的。人们见到过供给
曲线与需求曲线的存在吗?

沃克因表示,如果人们能够独立地说明供给,从而准确地了解供给关
系以及它将如何变动,那么,所得到的点将估计出一条需求曲线。也可以
说,如果人们独立地说明和需求关系,就能估计出一条供给曲线。如果人们
不能独立地说明任何一方,那么,没有额外的信息,就不能估计出供给曲
线或者需求曲线。

确定问题的“解决”至少大体上使得从经验上说明静态关系成为可能,
即使其他条件保持不变的条件不成立。人们相信,随着计算技术的改进
(通过计算机来实现),将会出现静态理论与经验理论和经验度量之间的更
好关系。
包恩斯理论与宏观经济计量党
20嵌纪30年代,并不是微观经济学的发展从根本上向前推进了经济计


470
量学;而是宏观经济学的发展起了推动作用,因为这一时期是宏观经济计
量学的发展时期。大萧条将经济学家的思想转向宏观经济学。到了20世纪
30年代后期,凯恩斯理论席卷经济学领域,人们都在奋力为大萧条提供令
人满意的解释,政策也致力于解决大萧条问题。因此,20世纪30年代直至
60年代,经济计量学的历史集中于宏观经济计量学。

20世纪30年代,对宏观经济模型化的关注是符合逻辑的。这一时期,
宏观经济学极大地受到凯恩斯宏观经济学的影响,人们尝试着寻找凯恩斯
理论经验上的对应物,得出了很多对乘数的估计。科林,克拉克(Colin
Clark)估计乘数介于1.5到2.1之间;米哈尔.卡莱斯基(MichalKalecki)
估计乘数约为2.25。

当然,只有当凯恩斯理论有意义时,乘数才有意义,所以,存在一种
强烈的推动力,以从经验上确定凯恩斯的理论是否正确。因此,出现了很
多度量消费与收入之间的关系,即凯恩斯所谓的“消费函数”的尝试。这
一时期也出现了下列信念的缺失,即经济力量具有推动经济体朝向充分就
业的自动倾向,与此相应,对中央计划的关注也增强了。这种中央计划要
求对经济体的关系做出估计。因此,重要的研究发生在像荷兰中央计划局
这一类的学会中,就不足为奇了。
拉格纳。弗里希、简,丁伯根以及大型宏观经济计重模型的发展
挪威经济学家拉格纳,弗里希(RagnarFrisch,1895一1973)是20基
纪20年代末期和30年代早期最有影响力的经济计量学家之一。弗里希是受
过高度训练的数学家,他既对宏观经济计量学也对微观经济计量学做出了
贡献,并且在使实验经济学改变方向、远离制度方法、朝向经济计量方法
方面发挥了重要的作用。实际上,正是他创造了经济计量学(econometrics)
这一术语。尽管弗里希在微观经济计量学中取得了一些重大发现(他完成
了一项具有决定性的对沃克因确定问题的数学处理,并且表明普通最小二
乘估计是有偏差的),然而,正是他对宏观经济计量学的贡献才表明了他的
重要性。他与简:丁伯根一道,在通过发展经济体的一种宏观经济计量模
型来创建宏观经济计量学中扮演了重要的角色。在弗里希的著作《运用完
全回归系统的统计合流分析》(StatisticalConfluenceAnalysisbyMeansofCom-


mpleteRegressionSystems,1934)中能够看到他的主要研究。他在书中主张,
大多数的经济变量同时在“汇合的系统”中相互连接,在该系统中,没有
任何一个变量能够单独变化;他设计出了多种方法来处理这些问题。

简:丁伯根是弗里希的朋友,1936年受国际联盟招募,承担对商业周
期理论的统计检验工作。1939年他的报告《商业循环理论的统计检验》
(StatisticalTestingofBusinessCycleTheories)得以出版。这一著作和集中于根据
数据发展动态宏观经济理论,并对它们进行检验。丁伯根发展了一种展示
周期性趋势的商业循环理论或宏观经济模型。

诸如弗里希与丁伯根一类的经济计量学家认识到,宏观经济学中的经
济计量研究,与微观经济学中的经济计量研究相比,在概念上要难得多。
在微观经济学中,人们困扰于供给方程与需求方程这两个单独的结构方程
的确定问题;在宏观经济学中,理论表明存在一个大的相关方程体系,该
体系构成了宏观经济力量的基础。不知什么缘故,研究者几乎无限地扩展
对大量方程的微观经济分析,详细地说明结构方程体系,并检验那些方程。

弗里希与丁伯根正是将他们的分析转向这一任务,两人因他们的贡献
而获得了诺贝尔经济学奖。像穆尔一样,他们的目的不仅仅是简单地检验
一种理论的有效性:他们对政策感兴趣。他们认为,如果他们能够说明对
经济体进行描述的结构性方程组,那么,就能确定一系列的政策来改变那
些方程的结构,并通过那些政策实现经济体合意的目标。

丁伯根的研究引起了严肃的批评,批评来自约翰梅纳德.凯恩斯与
米尔顿-弗里德最,两人都反对研究的整个过程以及从研究中得出的含义。
他们认为,丁伯根的估计程序是,运用相同的数据得出用来检验可能相互
冲突的理论的模型,这使得统计显著性的正常检验不相关。他们的观点代
表了经济计量学不能代蔡所传授的常识这一坚定看法。即使在其最初阶段,
宏观经济计量学也受到了重大质疑。
将里夫,。哈维默与经济计量学中的或然说蛙命
特里夫-蛤维默(TrygveHaavelmo,1911一)是一位挪威经济竺家,他
上拉格纳.弗里杀一同从事研究,他因为将概率方法(probabilisticap-
A471|

proach)引入经济计量学与经济理论中而获得赞誉。在引入概率方法之前,
经济学家假定他们正在试图度量的基础经济理论是精确的。如果人们能够
在事实上保持其他任何事物不变,那么,就能获得一种精确的关系。险维
默反对这一假设,他主张应当将经济理论视为概率理论,不是描述精确的

在哈维默的论文“经济计量学概率方法”(TheProbabilityApproachtoE-
conomics,1944)发表之前(但该论文1941年之前就以手稿的形式广为传
播),经济计量学家使用统计方法,但或含鞭或明确地认为,概率理论基本
上没有提供什么东西,他们正在试图探寻的基础性法则是精确的法则。哈
维黑认为,因为概率理论是统计方法背后的理论实体,所以,如果不承认
人们正在寻求或然性法则,却运用统计方法,这是不合适的。对经济法则
或然性特征的接受,使得很多统计方法与检验被正式加以利用,在此之前,
这些方法与检验在被使用时并没有正式基础,这也使得它处于经济计量学
现代方法的中心。哈维默于1989年获得诺贝尔经济学奖。
蛤维默的概率方法得到了考利斯经济学人研究委员会研究痢的愉辣,该
委员会由阿尔弗雷德.考利斯三世(AlfredCowelsIII)于1932年组建,他
是一位富有的投资顾问。考利斯集合了一群非常有智慧的经济学家,包括
欧文费雪、哈罗德.埠特林(HaroldHotelling,1895一1973)以及拉格
纳.弗里希,并要求他们研究如何将数学和统计方法应用于经济问题研究。
考利斯委员会最初位于美国科罗拉多州的斯普林斯;1937年迁到芝加哥并
一直在那里,直到20世纪50年代搬到现在的地点耶鲁大学。

大多数现在被视为标准经济计量研究的东西,都是由考利斯委员会从
事的。这一研究包括评估普通最小二乘估计是否向下偏倚(它被发现向下
偏倚百分之二十五):在小数据组中发展丝特卡罗方法。研究渐近收鳅和评
QD经济计量学太年轻了,其历史下到最近才馈加以考虑。就哪辟经济学家最为午要来说仔在
不同意见,并不是不正常。例如,玛丽*摩根断定是哈维默将概率方法引入经济计量学与经济学中
的,但是,菲利普-米罗维斯基则认为是其他人,例如诺贝尔奖获得者TJ库普曼斯(1910一
1984),

第16瘟经济计量学与经济学经验方法的发展
倍痢的无作从问赴。

我们应当还记得,这一时期计算上的难度是极大的,因为我们今天所
了解的计算机并不存在。人们不能简单地向计算机输入“得出OLS估计”
或者“得出最大似然估计”命令来获得结果。人们手工从事研究。考利斯
委员会遵循哈维默的思想,假设最好的经济计量学方法是概率方法,在这
一方法中,结构方程有一个假定的误差项分布。这种概率方法以考利斯委
员会方法而变得为人所知。来自于考利斯委员会的最著名的经济计量模型
之一是克莱因-哥德伯格宏观经济计量模型(Klein-Goldbergermacroecono-
mometricmodel,早期克莱因模型的一种改进),它是宽泛的凯恩斯体系最早
的经验研究代表,和包括63个恋量,其中很多是内生的,有43个是前定变
《>计量历史学与罗伯特W-
定量方法发展中一个引人注意且有
所争议的分支是将经济计量学应用于历
史分析。这一新的领域被称为新定量历
史或者计量历史学(cliometrics,在希腊
神话中clio是历史女神缘斯)。最为突出
的新历史学家是罗伯特,入,福格尔
(RobertW.Fogel,1926一),他在1964
年出版了《铁路和美国经济增长计量
经济史学文集》。在这一研究中,福格泵
结会了新古典经济学与统计推论,并对
很多结论产生怀疑这些结论是从事文
字创作的历史学家,根据他们对铁路与
美国经济增长之间的关系的研究得出的。

一些经济学家开始实践计量历史学,
并产生了大量文献。同时也有大量关于
新方法正确性的论述。A.H.康拉德
(A.H.Conrad)与JR.运耶(J.RMeyer)
1958年发表了一篇引起争议的文章——
“南北战争前的南方奴隶制经济学”
(TheEconomicsofSlaveryintheAnte—Bel-
JumSouth),在文章中,他们否定了如隶
制不是一种有利可图的制度的结论。福
格尔与S.L思格曼(S.L.Engerman,
1936一)出版了《十字架上的岁月;美
国黑人奴妹制经济学》(Tomeonthe
Cross:TheEconomicsofAmericanNegro
Slavery,1974),他们在文中运用大量数
据和研究,接受并扩展了康拉德-迈耶
的论题。新古典经济学与经济计量学在
历史中的这一新应用,在历史学领域内
部产生了相当多的争议,这是经济学侵
蚀其他学科的又一个例子。

以

济电想风
五、宏观经济计量学科学优雅性的受深
20世纪60年代期间,很多凯恩斯式的宏观经济计量模型得到发展,全
都呈现出某种科学面目,这其中包括数据研究学会《DRI)模型、沃顿模型
以及各种联邦储备模型。作为经济体的预言者,这些宏观模型一直流行至
20世纪70年代早期,但是,到了70年代中期这项研究失去了支持。在罗
代.爱淡斯坦(RoyEspstein)对这些模型的论述中,他写道;
进入20世纪70年代,实用经济计量学家的自信并没有持续很久。十年
的经济震荡开始使得根据大量结构性宏观模型做出的预测无效,这促使研
究者对其体系进行不断的重新说明和重新估计。伴随这项工作的是日益增
多的研究,这些研究将大模型的预测质量与新一代单变量时间序列简单模
型进行比较。这些比较也经党表明,结构模型做出的预测并不比简单模型
好,弟里德曼1949年做出的预测是一个明显的证实。中
对宏观经济计量模型的批评,其原因与经济学家反对早期研究的原因
类似。第一,古典统计检验的有效性取决于独立(independently)发展的数
据理论。然而在现实中,大多数以经验为根据的经济研究者“采集数据”,
寻找“最佳的适合”——实现了最佳*、t以及下统计量度量理论正确性
的统计量)的理论阐述。数据采集(datamining)侵蚀了统计检验的有效
性。第二,即使适当地实行了统计检验,数据的有限可得性也使得指派代
理人成为必然,这种指派可能合适也可能不合适。因此,检验的有效性取
决于代理人的适当性,但是,并不存在对代理人适当性的统计上度量。第三,
几乎所有的经济理论都包括一些不可度量的变量,这些变量能够被指望着,
而且经常被指望着解释不符合理论的统计结果。第四,经济计量检验的复
制通常是不可行的,原因是经济学家很少〈(即使曾经)能够进行可控制的
字验。这使得企何结果的可靠性都是未知的,它依赖于主观判断。
QD罗颁.J,爱泌斯坦.经济计量学史.美国芝加哥伊利诺仇大学出版社,1987;205

罗伯特索洛是一位宏观经济学家,获得过诺贝尔经济学奖,他捕近
下了经济学专业对形式上的宏观经济计量模型的大量关注,他写道:
我认为,不可能通过经济计量来解决这些争议。我认为,经济计量和
并不是一种相当有力和有用的宏观经济时间序列工具。因此,人们被迫就
经济休的结构做出判断。虽然你总是能够提供模型,运用经济计量来支持
你的观点,但是,对于双方来说这术容易了。人们从来没有能力找到共同
的经验依据.中
对经济计量检验的冷嘲热讽,使得很多研究者对其统计研究杀取了一
种随便的态度。@结果是很多研究不能被重复,极少的研究能被加以复制,
在已发表的经验研究文章中出现错误是平常事。爱德华.里莫(Edward
Leamer)是加州大学洛杉矶分校的一位经济计量学家,他总结了这一现象。
他写道,
经济计量建模是在建筑物的地下室完成的,而经济计量理论刘程则大
在顶楼(第三层)讲授的。我为同一种语言在两个地方使用这一事实所困
贡。更加令人惊异的是个别人的变形,他们在地下室中次意地犯错,随着
他们上到第三层,就变成了最高主教,多
他指出,摆脱进退两难局面的一种方式是运用贝叶斯经济计量学,基
中,研究者的信人黎程度被考虑进统计检验中;但是,这样做,过程太复杂,
所以大多数研究者只是不断地在做他们常做的事情。正是经验检验的这一
难点,使得我们在第1章附录中描述的修辞方法和社会学方法大量出现。
@参见罗伯特.索洛的《与经济学家的对话》一书的第137页,该书由美国罗曼与艾伦薪估
出版公司于1984年出版。

多”能够从托马斯.近耶一篇未发表的论文所进行的调查中,看到经济计量学当前的一些状况。
他集中于结果的选择性报告问题,询问它如何影响经济学家对杂志上的经验研究结果的信任。26%%
的人说,它使他们相当怀疑;54%的人说,它使他们稍有怀疑;9%的人说,他们不相信所有的经
济计量结果,所以选择性报告不成其为问题,8%的人说,它只是一个小问题。这些结果看来代表
了专业的情况。

四参见EE里莫的《详述探寻》一书的前言,该书由美国威利出版公司于1978年出版。

HMEeee
:/saoryofEcornomueDiorfy
|对宏观经济计量模型的现代批评
一项获得大量支持的对宏观经济计量
模型的批评,被称作卢卡斯批评,因为它
是由罗伯特,卢卡斯提出的,卢卡斯是一
位宏观经济学家,他是新兴古典宏观经济
革命的领导者。"上户卡斯认为,个人的行
为取决于预期的政策;因此,随着一项政
策变得过时,模型的结构将发生变化。但
是,如果模型的基本结构发生疏变,那
么,适当的政策也将改变,模型也就不再
合适了。所以,运用经济计量模型来预测
未来政策的效应是不合适的。

其他现代宏观经济计量模型的现代
批评家包括大卫,.享德里(DaivdHen-
dry),他认为,宏观经济计量学家应当
运用最新的方法和大量的检验来使数据
适合;也就是说,所揭示的统计关系应
当优先于理论。第三位批评家是克里斯
多佛,西门斯(ChristopherSims),他的
观点与训德里的观点略微相似。他声称,
当今的方法施加给数据过多的理论结构,
不施加结构会更好,应当在本质上将所
有的变量都视为内生的,并运用统计方
法来揭示关系。他上赞成运用向量自回归,
或者自回归移动和平均模型一类的方法。
这些方法简单地采用人们提交的所有数
字,没有结构,如果时间结构关系持续,
再找到这些数字在未来的最住估计。只
有计算机了解这一基本结构。

向量自回归方法是米切尔方法的现
代化身,它用最少的理论关注于数据。
传统的宏观经济计量学家指出,这些新
方法没有运用任何对经济体的理论见解。
和较早的经济计量理论批评家一样,向|
量自回归提介者回应说,传统的宏观经
济计量模型与结构模型,是以非常有限
的理论为基础的,以至于没有理论会更
好些。


*户卡斯之前的经济学家拥有很多
见识,他们清楚这些问题。20世纪40年
代后期,考利斯委员会也讨论过,在雅
各布.蕊尔沙克(JacobMarschak,
1898一1977)、库普曼斯(Koopmans)以
及威廉.菲利普斯(AlbanWilliam
H.Phillips,1914一1975)20批纪50年
代至60年代的研究中能够找到这一讨
论。然而,这一批评的影响来自于卢卡
斯的研究。
do
贝叶斯方法是对统计学含义男一种根本不同的解释。与客观解释相对,
它提出了一种对统计学的主观解释。贝叶斯们提出放弃古典解释,从而放

弃传统的古典经济计量学。不必说,就贝叶斯方法与古典方法而阁,统订
学家们之间存在着重大争议。要理解围绕经济计量检验的很多混清,理解
两种方法之间的差异是关键性的。

为了了解这一差别,假定我们希望估计某一参数的值。在古典统计学
中,人们得到参数的点估计,它满足某些特征,例如BLUE标准(BLUEcri-
teria),其中,B为best(最住),,L为linear(线性),U为unbiased(无
偏),E为estimator(估计)。此外,它必须具有合意的渐进性,使得当可以
利用大量数据时,估计也会聚合到参数的真值上。上古典分析的整个率焦点
在估计量和表明其特征的统计学上。

在贝叶斯方法中,对估计量的解释完全不同。贝叶斯分析不是产生数
据的点估计,而是形成数据的密度函数,它被称作后验密度函数。密度函
数并不是一种样本分布。它只能借助于人们先前所确信的东西加以解释。
当就数据的真值进行预言时,它通常是作为研究者给出的几率被予以论述
的。它是关于可能性的主观概念,而不像在古典方法中那样,是关于可能
性的客观概念或频率论者概念。

因此在贝叶斯方法中,人们必须说明其初始的信仰程度,并运用经验
证据作为改变信仰程度的手段。人们既有一种先验密度函数,也有一种后
验密度葡数。在贝叶斯分析中,人们只是简单地运用经验数据来修正其先
前的信仰,而在古典方法中,人们则不断地试图确立模型的真正性质。

经济学家在很大程度上没有使用贝叶斯方法。其原因并不在于他们反
对主观主义可能性的基本哲学特性;取而代之的是很实际的原因:(1)难
以将先前的信你形式化为一种形式上的分布,(2)难以发现后验分布的过
程;(3)难以使其他人确信贝叶斯结果的有效性,原因是它们明确地受到
干扰,或者只能通过个人信仰得到解释。尽管存在这些实际问题,很多经
济计量学家仍然真诚地热康于贝叶斯经济计量学。

贝叶斯方法没有引人注意地流行开来,但是,关于经济计量课程中所
讲授的内容如何很少真实地反映了经济计量学家所做的事情,一直存在着
众多的抱怨。例如,必瑞里盖特(Intriligator)、博多金(Bodkin)以及程肖
(Hsiao)写道


现有的大多数经济计量学教科书中,至少80%的材料单纯集中在经济
方法上。与之相反,应用型的经济计量学家,典型地只花20%其至更少的
时间和精力在经济计量方法本身上;剩余的时间和精力花在研究的其他方
面,尤其是花在相关经济计量模型的构建、估计进行之前适当数据的开发,
以及估计进行之后结果的解释上,2
造成这一差异的原因是,讲授经济计量学的教授通闸并不是实际进行
经济计量学研究的人。正如马格纳斯(Magnus)与摩根(Morgan)在1999
年所强调的,实际经济计量研究是在干中学,而不是教会的。这些抱怨是
下会导致未来更好的经验研究,还有待观察。
七、实验经济学家与模
最近,一群经济学家开始在经济学中采取一种不同的经验人研究方法。
利用动物或者人担当一种不知名商品的购买者和销售者,在了解基本的供
给与需求条件的前提下,来确定理论是否正确地预测了实验中出现的结果。
这些实验经济学家(experimentaleconomists)宣称,通过他们的实验,已经
证明了不同的经济命题。

我们来考察他们运用一种称作“双重口头拍卖市场”的程序所进行的
检验,其中,购买者与销售者公开宣布浣价并出价。弗农.史密斯〈Vernon
Smith)是这一研究的领导者和发展者,他于1956年进行了实验室实验来检
验双重口头拍卖市场是否能够实现均衡。学生们承担了供给者与需求者的
角色,并喊出他们的价格。各方都有由十四个学生组成的市场,在十五分
钟内价格非常接近于均衡价格;一旦价格到达均衡价格位置时就倾向于停
留在那里。当需求变动时当给学生几张纸,告诉他们不同的需求条件
时),价格相对迅速地调整到新的均衡价格上。这一实验被很多其他经济学
家所复制。

这种方法具有多种可能的用处。通过运用实验方法,经济学家就能了
Q@”迈克尔-入瑞里盖特,罗纳德.博多金,程肖.经济计量模型、方法及其应用.美国:普
雷惕斯-霍尔出版公司,1996:迎

第16意经济计量学与经济学经验方法触发展

解市场是如何在不同的制度条件下发挥作用的。在最近的一次实驼中,斌
究者检验了有牌价市场,并将它与双重口头拍卖市场进行比较。在牌价市场
中,厂商与购买者在一定时期内展示价格并坚持这一价格。人研究者发现,
与双重口头拍卖市场相比,在牌价市场上价格倾向于较高,这一发现使得
美国交通部向实验经济学家寻求帮助,以解决涉及铁路与驳船定价的问题。
铁路方面要求交通部将私人磋商的运费率转变为公开展示的费率,认为公
开展示将保护它们以及小驳船所有者免受大驶船所有者未公开的前价侵害。
然而,当实验者模拟这两种类型的市场时,他们发现情况恰好相反:价格
展示比私人磋商更倾向于产生较高的价格,并伤害小驶船经营者。铁路因
此放弃了它们的要求。

实验经济学家所进行的另一项检验是有关科斯定理的,科斯定理表明
能够相互损害但又进行谈判的各方将进行讨价还价,从而得到一个有效率
的结果,无论哪一方有造成损害的法律权利。实验的结果证明了这一预言。
然而实验发现,当通过撕硬币赋予个人法律上的权利时,他们几乎都不可
避免地未获得博穿论所预测的议价剩余中全部的个人合理份额。取而代之
的是,讨价还价者们几乎不可避免地平等地分享了剩余。这表明支配分配
的是一种公平的道德规范,而不是纯理性的个人最大化。实验进而表明,
如果任意赋予不对称的财产权,那么,个人并不认为它们是合法的。然而,
实验者注意到,当财产权被赋予给实验前在技能游戏中获胜的个人时,三
分之二拥有财产权的个人获得了大部分的共同剩余,而在任意分配的情况
下,没有人获得。

考虑到经验检验理论中存在的问题,这一研究获得重视并不令人感到
奇怪。它的专业认同具有宽泛的含义,不仅要求经济学家的训练发生重大
的变化,而且要求他们在社会中的作用,以及他们处理经济问题的整个方
法发生重大的变化。

一项相关的发展是通过模拟进行分析的。在这项研究中,模型被设计
为具有多个代理人,他们遵循着简单的地方化规则,然后进行模拟,并确
定哪些规则能够存在,哪些不能。这使得建模者借助假设的继续存在来选
择假设,而不是借助内省。
479


很难评价经济计量竺最近的发展。它的历史十末实现的希盘和预期的
历史。这些无法实现的希望和预期,在宏观经济计量学中比在微观经济计
量学中要多得多。然而,应当提醒读者,我们对于宏观经济计量模型构建
现代历史的看法,可能过于悲观。构建了这些模型的人认为已经取得了进
步,例如劳伦斯,克莱因及其同事断定:
我们发现,在半个多世纪的宏观经济建模过程中,很多引人注意的教
训已经被吸取。当然,这并不意味着宏观经济建模的进步是直线的,或者
其至是单调的或连续的。出现了一些人退步,如果有后见之明,很多事情本
可以按照不同的方式去做。但是,已经取得了进步,并且我们认为,我们
对现实世界宏观经济的了解,与半个世纪前相比要多得多,也
这一观点有可能是主流观点,但不是我们的观点。它也不是传统宏观
经济计量学批评家们的观点,他们认为:(1)主流的结构模型并没有和弄清
楚正在估计和正在检验的是什么;(2)与合理的数量相比,模型提倡者所
要求的数量过多了。批评家们所质疑的并不是经验研究;而是经验研究的
适当种类。皇击来自于两个方面。制度主义者希望将更多的注意力放在定
性数据上。一些人希望通过计算机来研究未受到理论阻碍的数据。还有一
些人希望看到比当前所使用的检验还要复杂的检验。

因为微观经济计量学集中于局部均衡问题,所以,它所呈现的是不太
严重的方法问题。不过,它也存在问题,也有批评者。具体来说,批评者
指责说,即使是微观经济计量学,也需要经济学中难以获得的信息,并且,
没有可控制的实验,古典统计检验就不能表明它们自认为已表明的东西。
至于宏观经济计量学,批评者采取两种完全不同的主张制度主义者声称,
经济学家应当更多地关注被传授的通常经验论,它较好地结合了制度知识
与历史知识,贝时斯的批评者则认为,我们需要更多的技术上的检验,,记
Q@罗纳人笃"博多人金,劳伦斯*殉莱因,志培,马永瓦.宏观经济计量建模史.美国:埃尔加,
1991).553~554


们抓住了统订竺的主观生性。
爱泌斯坦在他对经济计量学史的研究中,掌握了很多批评者对经济计
景学近来发展的看法,我们用他的这段话来作为本音的结束语,。
在没有可控制的实验或者得自于统一而稳定和的人口大样本的帮助下,
立志于更大的定量准确性,就这一点而言,经济计量学在所有学科中可能
是独一无二的。迁今为止经验表明,即使是最大的模型也具有精确而单纯
的结构,这些结构所代表的实际现象的有效性非常之低。美国制度主义者
的研究项目……与之相反的是在定量上不精确,但强调经济结构的一种复
杂的、分解的、历史的方法。当所要处理的问题相当特殊时,他们的政策
最为成功,并且允许以“干中学”的形式进行实验,例如,失业保险计划
的设计,劳动仲裁委员会的实行,或者配给项目的管理。一个更加关键的
因素,是对受到这些措施影响的不同经济群体的不同情况与动机的详细了
解。与之相反,温宁(Vining,1949)的观测具有一定的合理性,即经济计
量学家似乎完全关注于“整个文明的病理学”。这一研究表明,很多经济计
量模型的不精确,是对经济现象极为单纯化解释不可避免的结果。经济计
量分析在一种制度约束与个人行为能够更加清楚地加以辨别的新的层面上
进行,可能是最富有成效的,中
贝叶斯经济计量学Bayesianeconometrics”二文斯的日斑理论Jevons’ssunspot
BLUE标准BLUEcriteriatheory
古典经济计量学classicaleconometrics数理经济学mathematicaleconomics
通常经验论common一senceempiricism没有理论的度量measurementwithout

计量历史学cliometricsoon
数据采集datamining称尔的动态需求曲线Moore’sdynamic

经济计量学econometricsdemandcurve
实验经济学家experimentaleconomists”概率方法probabilisticapproach

确定问题identificationproblem统计分析statisticalanalysis
QDR.J]安泌斯击.

经济计量

学

芝加哥盆利语贷大学出版社,1987.:217~218
A81



%%% Local Variables:
%%% mode: latex
%%% TeX-master: "../../main"
%%% End:
