\part{现代经济学及其批判}

现代经济学不能被有效地称作新古典经济学,因为新古典这一术语所表示的观点、方法以及
假设,包括边际主义、全面理性的假设、边际生产力理论观点、瓦尔拉斯一般均衡理论、马
歇尔供求分析以及自由放任主张,并不是当代经济分析所特有的。在过去130年间的不同时期,
所有这些概念在经济学中都起着重要作用,当你听到“新古典经济学”这一术语时,这些概
念则是应当浮现在脑海中的东西。

然而,这些都变了。现代经济学包含着更宽的世界观,远比“新古典”这一术语所暗示的内
容更加折衷。什么是被允许的,什么是不被允许的,现代经济学同样有它的惯例,但它们不
是新古典经济学的惯例。因此我们看到,获得诺贝尔奖的经济学家\textbf{阿玛蒂
  亚·森(Amartya Sen)质疑效用最大化是否是解决贫困问题的一种有意义的方法;}多罗伯
特·索洛也是一位诺贝尔奖获得者,他探索劳动市场上的社会问题,两人毫无疑问都是主流经
济学家,但是,他们在这些场合下所使用的方法并不是新古典的。

在发展模型的过程中,现代经济学依赖于一套不同的技术工具,这些工具远远超越了边际主
义者的微积分学。现代经济分析要求作品用一种\textbf{数学模型}来加以表述,得出引人注
意的见解,并且\textbf{原则上能够进行经验检验}。于尔格·尼汉斯(J\"urg Niehans)将经
济学的现时代称为“模型化的时代”鲍、罗伯特·索洛也有类似评论:

\begin{quotation}
  今天,如果你问一位主流经济学家关于经济生活中几乎任何方面的一个问题,回答将是:
  假定我们将这一情况模型化,来看发生了什么…… 有上千个例子:要点是除了这一过程的
  例子之外,现代主流经济学很少由其它内容组成。
\end{quotation}

显然,大部分现代经济学是高度经验化和定量的,正如其所做的那样,包含着解决问题的特
定的模型化方法——着眼于一个问题,将其还原为一个原则上能实行经验检验的简单模型,
然后分析该模型。在完成这一点之后,必须加上形成模型时被抽象掉的因素,并将因此获得
的信息应用于要解决的问题中。

经济学家总是在做模型,但是将现代方法区分开来的,是\textbf{它的严密性和几乎专门集
  中于通过正式的数学模型,而不是通过例如启发式的模型来处理问题。}阿尔费雷德·马歇
尔提倡烧掉作为模型基础的数学,用文字表达观点,而现代经济学则似乎经常采取下列做法,
即如果一种观点不能被转换成一种数学模型,那么就必须放弃它。在对数学模型的要求方面,
与刻有马歇尔烙印的新古典经济学相比,现代经济学可能似乎有\textbf{更多的限制性},但
是,在假设和所使用的数学方法种类方面,现代经济学则\textbf{更加折衷},可以被看成是
程度没那么深。例如,新古典经济学将其分析限于唯一的均衡模型。熊彼特写道:

\begin{quotation}
  多重均衡不一定没有用,但是,从任何精密科学的观点看,\textbf{一种唯一确定的均衡
    的存在当然是极其重要的,即使证据不得不以非常严格的假设为代价而获得;在无论多
    高的抽象程度上,证明一种唯一确定的均衡(或者无论如何,少量可能的均衡的存在),
    是没有任何可能性的,现象领域确实是一种失去分析控制的混沌。}
\end{quotation}

与新古典经济学比起来,现代经济学更安心于多重均衡模型,乃至没有均衡的动态模型。两
者之间的另一种分歧涉及模型中所允许的假设:新古典模型要求特定的假设,例如存在全面
理性的和效用最大化的个体。虽然在现代经济学中仍然能够看到这些要求的很多痕迹,但是
也能看到不同类型的模型。例如,\textbf{一个导致集团而非个人最大化的进化博弈理论模
  型,或者一个基于认知不一致的模型,或者一个代理人模拟模型,都不会为新古典模型所
  接受},但是,这样的模型如果具有深刻的见解并符合形式上的要求,在今天则是可以接受
的。因而我们看到,有限理性模型被应用于宏观经济学,利他主义模型被应用于微观经济
学。

从20世纪40年代以来一直在远离新古典经济学。无法查明这一变革的具体日期,因为演化是
渐进的——它是一种缓慢的过渡而不是突然出现的变化。也难以确定新古典时期是什么时候终
结的,因为“新古典”这一术语本身就从未得到清楚的界定。托尔斯坦·凡勃仑1900年在他的
《经济学的先入之见》一文中最先创造了这一术语。凡勃仑运用这一术语时,是对马歇尔经
济学的一种负面描述,他认为马歇尔经济学本身是对门格尔和杰文斯的边际主义与斯密、李
嘉图以及穆勒的较宽泛的古典论题的一种综合。从一开始,这一术语就被一位局外人,而不
是被一位富有同情心的观察者用来表现另一集团的思想特征。

凡勃仑所创造的这一术语,并不是对美国主流经济学的一种描述。20世纪早期,经济学被加
以分割,至少在美国,制度主义远比新古典思想更加深入人心。但是,到了20世纪30年代,
马歇尔经济学在学术界获胜,并确定了专业的研究议程。个体所依据的边际主义概念、强制
性的最大化又被详细地予以探讨。随着这一探讨的发生,方法上发生了重要变化;经济学专
业由马歇尔的方法向瓦尔拉斯的方法转变,但是,研究议程仍然围绕着边际主义、供求均衡
模型以及理性来设置。对研究议程提出质疑的经济学家都是非正统的,而非主流经济学家。

对边际概念的这种前沿探讨结束于20世纪40年代。有两本书在很多方面将零碎材料集合在一
起,并捕捉到了新古典经济学的本质,这两本书是\textbf{希克斯的《价值与资本》以及保
  罗·萨缪尔森(Paul Samuelson,1915--2009)的《经济分析基础》(1947),它们是新古典经
  济学的终极著作——它们使所有的边际主义片段成为一个整体。}有了这两本书,前沿理论就
建立在逻辑与集合论的基础之上。这些新的手段引出新古典理论的下一步——\textbf{阿罗与
  德布鲁的一般均衡研究},这一研究正式探讨了竞争性经济体一般均衡的存在及其稳定性的
证据。20世纪50年代后期这一研究结束,对新古典理论广泛的理论探讨完成了。虽然仍有很
多零碎材料有待整理,但是,新古典经济学自身引起的主要理论问题已经得到回答,剩下要
做的就是应用理论。

正是在这个时候,\textbf{专业中的前沿由新古典经济学转向形式主义的和折衷的模型构建
  经济学。}然而,专业中的大多数人在“边界”上滞后于此数十年,所以,新古典研究
在20世纪一直持续着。大学本科教科书滞后更多。人们从大学本科教科书中学到的大多数内
容,仍然反映了新古典的版式,原因是模型方法较难讲授。这也是为什么大学本科教科书完
全不同于研究生教科书的原因。

在许多领域,模型应用相当成功。由新古典稀缺模型发展而来的运算法则,例如线性规划,
被应用于多种商业决策中。在金融方面,新古典模型导致了选择定价模型和新的金融工具,
政变了金融市场的性质。就具体问题的应用来说,新古典经济学是成功的。但是,就其更为
广泛的探求以获得对经济体令人满意的理解方面,新古典模型则不太成功。

与这些关注点以及新古典研究议程的完成相适应,经济研究扩大到包含更多的问题;正如其
所做的那样,经济学从新古典经济学演进到形式主义的和折衷的模型构建经济学。

\section*{第四部分的结构}
\addcontentsline{toc}{section}{第四部分的结构}

这一部分由四章组成。头两章描述了微观经济学与宏观经济学由新古典向形式的和折衷的模
型构建经济学转变。第三章着眼于经验研究的演进,原因是它在模型方法中的重要性。最后
一章则考察了当今的现代经济学非正统批评家。

第14章追溯了微观经济学的演进。这一演进的方式并不平稳:它是一条曲折的道路,开始于
马歌尔经济学优于制度主义的20世纪30年代。20世纪50年代,随着微观经济学的关注点远离
政策事务,它开始转向使新古典理论形式化和一般化,马歇尔经济学失去宠爱。\textbf{随
  着阿罗和德布鲁一般均衡研究变得知名,这种纯粹形式主义的研究在20世纪70年代达到巅
  峰,}于是,关注点再一次转向将微观经济学应用于政策问题。然而,这\textbf{并不是对
  马歇尔经济学的回归}。它是形式主义与应用研究的结合,并且运用数学上与经验上可检验
的模型来完成这种应用。

一开始,经济学家做了很多努力,试图使这些模型与一般均衡相一致,但是,由于认识到一
般均衡形式并不能回答大多数政策问题,或提供一种政策所依据的稳定立足点,现代经济学
开始自由地发展与一般均衡基础不一致的模型。随着这一现象的发生,起源于20世纪30年代、
由于不符合一般均衡因而大部分都没有被予以探讨的不完全竞争模型又恢复流行,多个强调
价格中信息内容的模型也是如此。

现代微观经济学的主要议程注重实效;它的目标并不是开发一种完全可以检验的总体经济模
型,而是提供一套专业化的手段一一能被用来解决特定问题的模型。现代经济学家受到训练,
以便将模型应用于更多种类的问题上。因此,学习现代微观经济学意味着学会应用大量当前
流行的模型。

第15章中呈现的宏观经济学的演进,则遵循了一条完全不同的路径。20世纪30年代
的\textbf{凯恩斯革命}(之所以这样称呼是因为它是根据约翰·梅纳德·凯恩斯的研究得出的),
引进了一种从根本上不同的宏观经济方法来解决问题。不是逐步创建方法,而是从总量开始,
然后在它认为合适的地方增加微观经济基础。在凯恩斯的宏观经济学中\textbf{总量关系是
  中心}。这一点违反了新古典方法,并导致对凯恩斯模型的较大抵制,但是,到
了20世纪60年代早期凯恩斯宏观经济学得以确立,政治家们都承认自己是凯恩斯主义者。所
以,尽管存在宏观经济学不符合新古典模型这一事实,凯恩斯宏观经济方法仍然成为经济学
的一种方法。

整个20世纪60年代,就宏观经济学如何符合微观经济学而言,存在认知上的不一致。正如保
罗·萨缪尔森所言;:“我们总是假设凯恩斯的不充分就业均衡是在限价与不完全竞争的基础
上提出的。……我担心微观基础。”\textbf{宏观经济学从不拒绝接受没有合理的微观经济
  基础的指责,它只是不涉及这一问题。}

20世纪60年代后期和70年代早期,经济学家开始发展微观基础,重新衔接微观经济学与宏观
经济学。这样做,从根本上政变了宏观经济学。到了\textbf{20世纪80年代},将现代经济学
的一部分称作“凯恩斯经济学”就不再适合了。微观基础经济学家表示,如果认同全面理性、
完全竞争以及均衡,那么,就不可能从标准的微观经济模型中得出凯恩斯的结论。这些微观
基础研究,外加凯恩斯经济学未能预言20世纪70年代的通货膨胀,从而导致了宏观经济学中
的第二次革命,即新兴古典革命。

\textbf{新兴古典学派脱离凯恩斯经济学,面向一种与微观经济学假设更加一致的宏观经济
  学。}所以,在这一时期现代宏观经济学更加向新古典假设靠拢。但是,它从未浪子回头。
随着这场革命持续贯穿于20世纪80年代,它也遇到了预测以及为宏观经济学提供适当基础的
问题。回应新兴古典学派的多种新凯恩斯主义得到发展。20世纪90年代后期和21世纪初期,
不存在普遍公认的宏观经济学方法。它成了一个混乱的领域。

新兴古典学派与凯恩斯主义者最初的大部分争论,是关于\textbf{经济周期}的,争论的最终
结果并不明确。然而,\textbf{20世纪90年代,宏观经济学家不再解决经济周期争论问题,
  而是停止就经济周期进行论战,将其注意力转向增长,认为增长远比经济周期重要。他们
  最初使用一种新古典增长模型但是,很快就用强调技术和新经济的新增长理论模型来加以
  补充。}

现代经济学高度经验化,大部分数学模型的建立使经验检验可行。因此,为了了解现代经济
学及其问题,必须了解模型的经验检验方法。第16章考察了经验检验及其在经济学中的历史。
它通盘考察了经济学中从经验主义常识到现代经济计量学统计分析的运动。这是一段引人瞩
目的相关历史,现在才开始获得应有的注意。在这一章中我们将指出,经济学家所运用的很
多统计方法,是从一些研究领域借鉴而米的,在这些领域中,可控实验是构建知识的通常办
法。不幸的是,因为可控实验在经济学中很难实施,所以,就出现了经济计量学中的主要问
题,并将现代经济学留给一些严肃的批评家来评判。

本书的最后一章第17章,涉及现代经济学的非正统批评家。正如我们在全书中所主张的那样,
我们认为,非正统经济学家在经济学中扮演着重要角色。尽管他们在将其特定观点确立为新
的正统理论方面可能并不成功,然而他们经常是所发生的变革的催化剂。在这一章中我们认
为,他们仍然扮演着重要角色,只过由于现代经济学的折衷性质,这是一个困难的角色 。

\chapter{现代微观经济理论的发展}

新古典经济学不是单一实体:它是随着时间的发展而不断演进的多维的思想流派。它聚焦于
边际主义、理性假设以及市场奏效这一强烈的政策假定,尽管这种政策假定受到很多附带条
件的支配。新古典学派非常不固定:其至有可能在新古典经济学家演变成正统的经济学家之
前,该学派就开始变化。经济学渐渐地远离其新古典的立足点。边际主义的微积分学被集合
论所取代;理性假设被心理学见解加以修正;经济分析用来解决的假定,尽管这种政策假定
受到很多附带条件的支配。新古典学派非常不固定:新古典经济学家一变成正统的经济学家,
甚至有可能在此之前,该学派就开始变化。经济学渐渐地远离其新古典立足点。边际主义的
微积分学被集合论所取代;理性假设被心理学见解加以修正;一系列分析用社会学见解来扩
展;社会学上的解释用来补充劳动市场分析。随着(类似)这些变化的发生,曾经被看做是
新古典思想必要组成的内容,不再作为现代思想的组成。我们的观点是,相当多的组成已经
改变,有正当理由使用一个新的术语来描述现代经济学。

新古典经济学的很多因素,仍然存留在现代微观经济学中,但是,区分现代微观经济学的,
并不是这些因素,而是解决问题的模型方法。与模型是否在经验上符合现实相比,模型的假
设与结论并不重要。

\section{远离马歇尔经济学的运动}

马歇尔的分析的发动机将供求曲线与常识相结合,这能够回答某些问题,但有些问题则超越
了其范围。供求分析是被应用于相对价格问题的局部均衡分析。但是,经济学家试图回答的
很多问题,诸如什么决定了收入分配或者什么引起了某些规则和税制,要么超越了局部均衡
分析的适用性,要么违反了局部均衡分析的假设。不过,经济学家仍然将局部均衡主张应用
于这些问题,并假定总体市场一定是由一些至今未知的局部均衡市场的联合构成的。

大多数经济学家在很长一段时间里满足于这一状况。毕况,马歇尔的经济学确实提供了一种
能够回答很多现实问题的理论,即使在形式上不很严密。它是一种中间立场。马歇尔式的经
济学家是工程师而不是科学家,工程师对考虑基础性的力量不感兴趣,而感兴趣于构建发挥
作用的事物。马歇尔式的经济学家对经济学艺术感兴趣,对实证经济学或规范经济学不感兴
趣。开如琼·罗宾逊所指出的,马歇尔有认识困难问题的能力,但明显地可以看出他将它们隐
藏起来了。

马歇尔的经济学试图在形式主义方法与历史制度方法之间,走一条完美的路线。由此引来了
两边的批评家,这并不令人感到奇怪。在美国,一个被称为制度主义者的群体简单地希望消
除理论,认为应当强调历史和制度,结束不适当的理论。我们称作形式主义者的一些批评家
则走向相反的方向:他们认为经济学应当是一门科学,而不属于工程学领域,如果经济学得
出市场运行良好的结论,我们就需要一种理论来表明它是如何并且为什么运转良好。形式主
义者也赞同制度主义者关于马歇尔的经济理论是不适当的看法,但是,他们的回答不是消除
理论:他们希望提供一种更好的、更严密的、能够充分地回答更复杂问题的一般均衡基础。


\section{微观经济学的形式主义革命}

20世纪30年代后期,形式主义研究方法获胜,马歇尔方法开始衰落。到了20世纪50年代,
形式主义者再次将微观经济学表达为一种依赖于瓦尔拉斯而不是马歇尔的数学结构。与逻辑
一致性相比,应用变得不重要了。

1959年,随着阿罗--德布鲁模型的发表,形式主义革命达到巅峰。一般均衡的这项研究完成
之后,经济学家再次转向应用研究。但是,他们并没有回归到马歇尔分析的发动机方法上,
这种方法不重视使用数学,而是强调判断。与该方法不同,经济学家将政策规定融合进数学
模型中。随着这一现象的发生,新古典时代演进到了现代模型化时代。在模型化方法中,数
学被用来建立能够很好地抓住问题本质的简单模型,然后运用经济计量方法来检验模型、模
型的形成及其经验检验成为现代经济学方法。

\subsection{形式主义方法的奋斗}

数学方法来源于19世纪和20世纪早期一些人物的思想,我们在前面关于新古典经济学的章节
中讨论过这些人物。在用数学形式陈述假设的伟大先驱者中,最早的一位是安东尼·奥古斯
丁·古诺,他于1838年出版了《财富理论之数学原理的研究》。古诺预期到,他将数学引入经
济学的尝试,将会遭到大多数经济学家的否定,但是,他仍然坚持他的方法,因为他认为,
能够用更为准确的数字予以表达的理论,其文字表达是一种浪费,也令人烦恼。

瓦尔拉斯与继瓦尔拉斯之后洛桑大学的经济学教授维尔弗雷多·帕累托,是数理经济学的另外
两位早期拥护者。马歇尔集中研究局部均衡,瓦尔拉斯则运用代数方法,集中研究一般均衡。
他的一般均衡理论已经基本取代马歇尔的局部均衡理论,作为经济研究的基本框架。杰文斯
在其有影响力的《政治经济学理论》(1871)中,也提倡在经济学中更加广泛地运用数学。

紧接杰文斯的是另一位数理经济学先驱弗朗西斯·Y·埃奇沃斯,他在1881年指出,微观经济理
论的基本结构不过是对最大化原理的重复使用。这一发现提出了下列问题,即为什么一再地
重新考察相同的原理?对特定的制度背景加以抽象,并将某一问题还原为数学核心,人们很
快就能抓住问题的实质,并将该实质应用到所有这样的微观经济问题中。根据这一推理,埃
奇沃思宣称,无论是对经济体的理解还是对适当的政策的表达,都被认为与数学的使用相一
致。他指责马歇尔式的经济学家受到“纵横交错的绚丽的文献路径”的诱惑。

在这一扩展发生的同时,也有人尝试着不仅将数学扩展到实证经济学,而且扩展到经济政策
问题中。对于很多学习经济学的学生来说,熟悉维尔弗雷多·帕累托的名字,是由于它在帕累
托最优标准这一术语中的使用,帕累托在20世纪初期,将瓦尔拉斯的一般均衡分析扩展到经
济政策问题中。因此,在推进形式化的过程中,实证经济学与经济学艺术之间基本上没有区
别,约翰·内维尔·凯恩斯对两者所做的区别不存在了,两者使用相同的形式上的方法。

欧文·费雪在19世纪的最后十年进行创作,他是美国形式主义的一位早期先驱者,他支持并扩
展了西蒙·纽康(Simon Newcomb,1835一1909)关于增加数学在经济学中的使用的倡议。然而
在美国,几乎在20世纪中期之前,数学方法并没有得到完全接受。所有这些先驱者,都是被
忽视的未来预言家。对他们成就的疏忽,部分地归因于马歇尔分析的力量,马歇尔的分析是
对理论、历史以及制度因素的一种明智的混合。由于未能竞争过马歇尔方法,在20世纪30年
代之前,经济学中早期的数学研究在实践上被主流经济学家忽视了。

20世纪30年代早期,这种状况开始变化。如今为大学本科微观经济学提供基础的很多几何方
法开始充斥杂志。边际收益曲线、短期边际成本曲线、不完全竞争模型以及收入替代效应模
型,都在这一阶段得到“发现”和探讨。这些新的工具尽管来源于马歇尔,却使马歇尔的分
析形式化,从而,它们也就离其所代表的实际制度越来越远。将理论与制度相连结的马歇尔
方法,就像一个跷跷板:只要两边平衡,它就工作。但是,一旦理论一边略微偏重,平衡就
被打破,经济学就严重地导向理论一边,将历史与制度悬在空中。

\textbf{历史与制度方法之所以被放弃,是因为新的数学方法要求准确地说明什么是被假定
  的、什么是变化的,这种说明方式使新的方法能够处理全部分析。}历史与特定的制度方法
不再适合。人们不再像在较早的马歇尔经济学中那样认为,“一个合理的经营者”将以某种
方式来行动,并且,读者的敏感性受到吸引,想知道“合理”意味着什么。现在,“合理
的”含义被转换成一个准确的概念——“理性的”,它被界定为依照某些确定的公理进行决策。
类似地,竞争性经济体被界定成其中的所有个体都是“价格接受者”的经济体。建立数学模
型需要非关联性的观点,该观点可以从任何实际背景中抽象出来,只要在其中清楚地说明假
设。

尽管在马歇尔的分析中,几何作为一种工具被加以使用只是一个小小的开端,但是,它却是
马歇尔经济学终结的开始。当几何方法揭示出马歇尔经济学中众多的逻辑问题时,新的马歇
尔主义者就用更加形式化的东西予以回应。因此,到了1935年,经济学变革的时机成熟了。
保罗·萨缪尔森总结了这一状况:“对于一个有分析能力的人来说,能够敏锐地认识到,数学
工具在经济学中是一把有力的剑,在1935年,经济学的前途无限美好。地上洒满了迷人的法
则,等待被拾起并按统一的方式来排列。”

因为很多经济学家到这一时期已经掌握了必备的分析工具,所以,20世纪30年代后期和40年
代早期,见证了形式主义获胜这一微观经济理论的演进。古诺、瓦尔拉斯、帕累托以及埃奇
沃思赢得了更多的敬意,马歇尔经济学则主朗在大学本科教育中发挥作用。

微观经济理论数学化的第一步是扩展对家庭、厂商以及市场的边际分析,并使其更加具有内
在的一致性。当经济学家转向较高水平的数学方法时,他们就能够超越局部均衡而进入一般
均衡,原因在于,数学提供了一种方法,借助这种方法,能够比较准确地明了他们脑海中先
前略微松散地持有的一些内容。第二步是再次表达问题,且表达方式要与处理问题所使用的
工具和方法相一致。第三步是添加新的方法,以阐明未解决的问题。今天这一过程是连续性
的。

这些步骤并不遵循单一的路径。其中一条路径具有明显的欧洲根源,它包括使一般均衡理论
一般化和形式化。这条路径的一位早期先驱者是古斯塔夫·卡塞尔(Gustav
Cassel,1866一1945),他在其《社会经济理论》中,简化了瓦尔拉斯一般均衡理论的表达,
使其更容易理解。

20世纪30年代,两位数学家亚伯拉罕韦尔多和约翰·冯·诺依曼(John von
Neumann,1903--1957)将他们的注意力转向对静态和动态模型中均衡条件的研究。他们很快
提出经济分析在技术上的复杂性,揭示了先前的经济学家政策分析和理论分析中的大量不足。
他们的研究被诸如肯尼斯·阿罗和杰勒德·德布鲁一类的经济学家作为注释,这些经济学家将
他们的研究加以扩展,并应用于瓦尔拉斯理论,以形成对瓦尔拉斯一般均衡理论更加准确的
阐述。在韦尔多的引导下,阿罗和德布鲁接着又重新发现了埃奇沃斯的早期著作。这些经济
学家给他们留下了极为深刻的印象,以至于他们宣称现代微观经济学的真正鼻祖是埃奇沃斯,
而不是马歇尔。这些理论家的研究,依次延续了一般均衡理论家的高度形式化的传
统。

\textbf{对市场的无约束利用将导致公共利益吗?如果是,是在什么意义上?市场这只“看
  不见的手”推动了社会利益吗?如果是,什么类型的市场是必需的?因为它们涉及整个体
  系,所以,从本质上说这些都是一般均衡问题,而不是局部均衡问题。}因此,它们并不能
在马歇尔的框架中得到回答,尽管它们可以用相对宽泛的术语来加以论述,就像在正式的一
般均衡分析形成之前那样。

一般均衡理论家已经回答了“\textbf{看不见的手}”是否起作用这一问题——只要某些条件成
立,管案是肯定的。他们的证据也是阿罗和德布鲁获得诺贝尔奖的原因,它是经济学中的里
程碑,原因在于它回答了亚当·斯密所作的猜想,这一猜想开启了古典经济学传统。后来的大
多数研究都是在一般均衡理论中进行的,以期更加优美地说明“看不见的手”的原理,并修
正其假设。阿罗和德布鲁因为最先证明了它,从而为他们在经济思想史中赢得了一席之地。

\subsection{保罗·萨缪尔森}

在涉及这种形式化的众多经济学家中,保罗·萨缪尔森可能是最为知名的。萨缪尔森出生
于1915年,在1935年拥有雄厚的大学本科数学背景之后,开始在哈佛大学进行经济学研究生
学习。在哈佛,他继续发表一些重要文章,将数学运用于微观和宏观经济理论。在1941年二
十六岁时,他获得了博士学位,三十二岁时便已经成为麻省理工学院的正教授,并成为美国
经济学会约坦·贝蒂·克拉克奖的首位获得者,该奖授予做出重大专业贡献的四十岁以下的经
济学家。萨缪尔森后来成为\textbf{第一位获得诺贝尔经济学奖的美国人}。

萨缪尔森的学术灵感来源于古诺、杰文斯、瓦尔拉斯、帕累托、埃奇沃斯以及费雪,所有这
些人都将数学部分地应用于经济理论中。借助其数学背景,萨缪尔森扩展了他们的研究,并
促进了\textbf{现代经济理论数学基础的奠定}。像埃奇沃思一样,他对马歇尔也有微词,针
对马歇尔的含糊,他说:“\textbf{(这种含糊)麻痹了我们专业里盎格鲁萨克逊分支中最
  优秀的头脑三十年。}”他继续说:
\begin{quotation}
  我渐渐觉得马歇尔的权威性意见——“是否任何人都应花费足够的时间,阅读冗长的由经济
  学转换而来的数学。这似乎令人怀疑,我自己从来就不这么做”,马歇尔的意见应当恰好
  被\textbf{颠倒}过来。就像大多数现代经济理论所显示出的特征那样,将本来简单的数学
  概念变成辛苦的文字工作。这不仅从学科进步的观点来看是不值得的,而且也涉及一种特
  有的堕落型的智力训练。
\end{quotation}

\subsection{均衡与稳定}

根据萨缪尔森的观点,构筑并统一了微观与宏观理论特有成分的理论结构,依束于两个非常
普遍的假设:首先是关于\textbf{均衡}的,第二是关于\textbf{均衡的稳定性}的。对于比
较静态问题,均衡的条件可以置于熟知的最大化框架中,先前关于微观经济理论的大多数研
究都是在这一框架中进行的。\textbf{萨缪尔森通过彻底研究厂商成本最小和利润最大、消
  费者满意最大以及福利理论来阐明这一方法的统一。}先前的经济学家较少注意动态分析,
萨缪尔森则表明,一旦某一体系的动态特征得到详细说明,该体系的稳定性能得到评定。因
此,均衡条件与稳定条件就成为构成经济理论基础的两部分结构。

尽管萨缪尔森的《经济分析基础》及其后来的研究几乎专门涉及数理经济理论,然而,他对
数理经济学与经济研究过程之间的关系也很敏感。他一贯试图阐明那些具有操作意义的定理,
而不仅仅是那些优雅的定理一一换句话说,提供经济研究有用的可检验的假设。“所谓有意
义的定理,”他说,“我的意思只不过是关于经验数据的一个假设,如果只在理想条件下,
它会令人信服地被驳回。”

\subsection{形式主义者、数学以及教学方法}

数理经济学使得简明而准确地陈述经济理论变得可行,通过数学处理,也使得对既定假设下
的理论含义进行推论变得可行。形式主义者在数学上揭示了文字论证中的矛盾,纠正了其中
的逻辑错误,这些文字论证被用来扩展局部均衡分析。此外,他们表明马歇尔模型的不同方
面,诸如需求更论与生产理论,只不过是一个一般化的受约束的最大化模型的具体应用。正
如他们所做的那样,他们的数学方法粉碎了运用局部分析的理由。认识到这一点,萨缪尔森
回归到瓦尔拉斯,观察他是怎样来处理相互联系的市场的。从瓦尔拉斯的分析开始,并运用
代数与微积分,萨缪尔森就能确定出均衡所必需的稳定条件。这为经济论证提供了一种更加
稳固的理论基础,以及一种多市场均衡的分析核心,后者可以作为现代微观经济学的基础。

然而形式主义的采用,也提出了教学上的问题:瓦尔拉斯一般均衡方法非常有难度。为了掌
握该方法,人们必须学会一种新的语言(数学),并且能够掌握非常抽象的、缺少关联的结论。
但是,经济学的大多数大学本科生,并不打算成为经济学家,因而也就很少有动机来掌握相
当数学化的技能,这种技能对于理解一般均衡相互作用的复杂性是必需的。教学上的这个问
题引起了微观经济学当前的分层,原因是研究生所喜欢的经济学理论,对于典型的大学本科
生来说太难了。萨缪尔森通过编写一本初级经济学教科书,满足了大学本科教育的特殊需要,
这本教科书已经销售了几百万册,历经了很多版本。从1947年的第一版开始,萨缪尔森的教
科书支配着经济学领域大约三十年,大多数其他介绍性的教科书复制了他的格式。这本初级
教科书塑造了现代大学本科经济学,正如他的《经济分析基础》塑造了研究生经济学一样。

在其\textbf{大学本科教科书}中,萨缪尔森用图形呈现了微观经济学,以此作为对理性的个
人在竞争性市场结构中相互作用的一种逻辑扩展。在保留了马歇尔的方法但去掉了较早经济
学教科书中大多数的陈词滥调和随意的类推后,\textbf{萨缪尔森构筑了一个更加符合一般
  均衡分析的相当程度上非关联的理论}。现在对研究生经济学和大学本科经济学的划分,正
是按照这种方式发展而来的。\textbf{大学本科导论性的教科书保留了马歇尔的方法,强调
  二维的图形方法,而不是多元的微积分;而研究生微观经济学则朝着完全的数学方法继续
  前进,这更加符合瓦尔拉斯和古诺的方法,而不是马歇尔的方法。}

\subsection{演变中的方法}

微观经济学的演进,牵涉到从一种数学语言到另一种数学语言的改进,每一种数学语言都能
够解决一些含糊之处,这些含糊之处损害了原有的方法。最初,像保罗·萨缪尔森和约
翰·R·希克斯这样的经济学家,将20世纪30年代的几何转换成20世纪60年代的\textbf{多元微
  积分。偏微分代表了部门之间的相互关系;二阶偏导数的符号表明了稳定条件;一阶导数
  的符号说明了相互影响。需求的交叉弹性、线性齐次生产函数、位似需求以及固定替代弹
  性(CES)生产函数,都出现在微观经济学的术语中。}再次用数学形式来表示微观经济学,
其结果给人们留下了深刻的印象。当经济学家将问题研究了一遍之后,他们开始认识
到\textbf{价格与拉格郎日乘数}(Lagrangian multipliers)之间的关系。价格是否内在于经
济体系中的问题,时先已经引起了争论,但是现在,\textbf{数理经济学家能够表明,价格
  通过一种最大化过程自然发生,即使缺乏市场,受到约束的最大化过程也将仍然具有一
  种“价格”(称作影子价格)。}如果价格不存在,那么,必将有另一种计量上的设计取代价
格。

他们也说明了如何能够容易地\textbf{将一个受约束的最大化问题复述为受约束的最小化问
  题}:通过\textbf{转换约束和目标函数},\textbf{“服从技术生产约束的产量最大
  化”问题,就等同于“服从生产某一产量的成本最小化”问题。}这样的一种复述,
以“\textbf{二元分析}”而知名,通过表明产量或约束的微小变动如何改变了情况,为研究
最大化问题的性质提供了洞察力。

这些发展具有实践意义和理论意义。在实践方面,对影子价格和二元分析的理解,导致了现
代管理方法的重大发展。在理论方面,二元分析将稀缺性融入经济学家的分析中,这种对称
性加深了他们对问题的理解。先前要用几卷篇幅(经常是不正确地)呈现的问题,现在能够用
一页或两页篇幅来涵盖(对于那些懂这种语言的人来说)。考虑到较早的非正式模型的误用,
以及模型含义造成的混乱,大多数经济学家将这些发展视为重大的收获。

但是,用多元微积分来复述微观经济理论也存在问题。\textbf{多元微积分要求连续性},并
且是以一种非常少见的方式提出最大化问题的。为了应对微积分的这些不足,经济学家用很
多方式修正了最大化问题。一些修正使微观经济学在商业中更加实用,另一些则提供了对经
济体的更深刻理解。

\textbf{20世纪70年代,比较静态微积分的可行性开始枯竭,理论研究的刀刃放在了时间被
  明确予以考虑的动态微积分上。}为了了解为什么动态微积分是相关的,来考察一下生产问
题。中级微观经济方法说,厂商面对着生产问题,在既定的一套投入和相对价格下,厂商选
择一个最佳产量。但是,模型中的时间在哪里呢?它受到了抑制;因此,模型实际上怎样运
转并不确定。比较静态解释(comparative static interpretation)的采用,提供了一个略微
具有暂时性的维度。问题要被考虑两次:某种单一变化之前和之后。因此,它就成为时期中
的两点分析。然而,比较静态方法并没有考虑人们怎样从一点变到另一点上,或者那一段时
间周期有多长。

为了更好地分析从一点到另一点的变化过程,对这一问题的数学表述必须明确地包含时间路
径,人们沿着这条路径从初始状态达至最终状态。做到了这一点的微积分是\textbf{最佳控
  制理论}(optimal control theory)。学生们一般是在微积分课程中,在微分方程之后学
习最佳控制理论的,而微分方程紧随着多元微积分。解决方法都相似,但不是用拉格朗日乘
数来表示,而是用汉密尔顿函数和汉米尔顿行列式来表示。

增加了所使用的微积分的复杂性之后,微观经济分析在此基础上继续发展,原因有实践方面
的,也有理论方面的。在实践上,它向\textbf{线性模型}转变,原因是存在线性运算法则,
借助于这些法则,人们能够很容易地计算出数解。因此,简单的线性描述与现实世界的问题
更加相关,线性、网络以及动态规划添加到经济学家的工具箱中。在理论研究方面,对一般
均衡问题的阐述很快超出了微积分的范围,转而运用\textbf{集合论与博弈论}。经济学家更
喜欢这些方法,原因是它们更加精确,并且不像微积分那样要求连续性的假设。随着方法发
生变革,术语也改变了;在研究生的微观经济理论课程中,诸如上半连续和纳什均衡一类的
术语已经成为平常话。

微观经济学的另一项重要变革,明显地体现在它对不确定性的处理上。面对不确定的未来,
人们必须做出经济决策。马歇尔并没有试图直接处理不确定性问题。然而,现代微观经济学
正式面对了不确定性问题,虽然经常是以随机的而不是静态的方法来面对。为了分析这样的
模型,微观经济学运用了应用统计决策理论,它是统计学、概率理论以及逻辑学的综合。

像通常的情形那样,人们能够从正面和负面来观察某一领域的发展。以博弈论为例,我们简
要地将其提出来,仅仅作为完成一般均衡分析时比较优美而精确的可供选择的方式。确实如
此,但它还有更多的内涵。它是对所存在的人类相互作用最普遍的分析,并且,它为经济学
家提供了分析相互依赖行为所采用的方式,否则经济学家将只能进行假设。因此,博弈论为
理解垄断行为提供了实用模型,而\textbf{垄断行为占据了大多数西方经济体组成中的大部
  分}。类似地,它也对社会问题提出了重要见解,正如在托马斯·谢林(1921一)的著作《冲
突的战略》(The Strategy of Conflict,1960)中显示的那样。作为一种选择,它提供了一
种综合所有社会科学的方法,正如在马丁·舒比克〈(MartinShubik,1926一)的《社会科学中
的博弈论》(Game Theory in the Social Science,1982)中显示的那样。

因此,博弈论的逻辑在今天就像当年约翰·冯·诺依曼和奥斯卡·摩根斯坦最早于1944出版《博
弈论与经济行为》时那样引人注目。现代的研究生教育突出地强调博弈论方法,博弈成为现
代经济学家工具箱中的组成部分。

\subsection{将模型应用于政策制定}

到了20世纪80年代,理论研究基本上完成,随着经济学家开始应用模型来解决政策问题,关
注点迅速转到了将微观经济学应用于政策问题之上。只要这些模型(通常都是局部均衡模型),
有深刻的见解且在经验上得到支持,就\textbf{无需符合一般均衡理论},除了个别可计算的
一般均衡模型之外,一般均衡理论处于不起眼的位置。这种政策应用的一个例子
是\textbf{柠檬模型},它是由伯克利大学的经济学家乔治·阿克洛夫〈George Akerlof)设计
的。该模型解释了在特定情况下为什么市场会失灵。

柠檬模型的实质如下:例如,你希望购买一辆二手车,某人走近你,以非常低的价格向你提
供了一辆车。你应当购买这辆车么?不一定,因为你没有关于车的完全信息。某人以如此低
的价格卖车,这一事实暗示这车可能有问题。\textbf{模型的要点是价格有可能携带着有关
  产品质量的信息,如果的确如此,那么市场将不一定形成适当的价格,它可能会失灵。}由
于信息问题市场可能失灵,据此建立模型,是现代微观经济学所做的典型理论研究。柠檬模
型具有很多潜在的应用性。现代经济学家将它的变种和类似的模型应用于多种情形中,
来“解释”为什么我们观察到的现象会发生。

大多数现代微观经济学家并不建立诸如柠檬模型一类的模型。他们只将这些模型应用于特定
的问题。享利·瓦里安是一位顶级的现代微观经济学家,他就如何构建模型给出了下列建议。
他建议浏览报纸,找到某个问题,然后看你是否能对此有一种经济解释。例如,为什么商场
要实行折扣?选择好你的问题之后,就应当试着使之模型化。他写道:
\begin{quotation}
  你很幸运,所有的经济学模型看上去几乎一样。有一些经济代理人,他们为了实现他们的
  目标而做出决策。选择不得不满足各种约束,所以,就得调整某种东西,以使所有的选择
  都相容。这一基本结构使人想到了一种进攻计划……它能帮助你识别一个模型的片段,
\end{quotation}

他继续说,然后你应当构建模型,所遵循的标准是KISS(Keep It Simple,Stupid)。在你建
立了一个简单的模型之后,再使模型一般化。


因为经济学家运用的大多数模型,都是局部均衡模型,而不是一般均衡模型,所以随着应用
建模方法成为规范,现代应用研究与一般均衡理论之间的关系也就变得无关紧要
了。

\section{米尔顿·弗里德曼与微观经济学的芝加哥方法}

逐渐支配了经济学专业的现代建模方法,在经济学的芝加哥方法中也有一定的基础,芝加哥
方法从20世纪50年代直至20世纪70年代,都与形式主义的方法背道而驰。芝加哥方法的特征
在于:第一,它认为作为组织社会的一种手段,市场比其他选择能更好地发挥作用;第二,
它将非形式化的方法与模型化相联系。

在整个现代经济学时期,米尔顿·弗里德曼能够与保罗·萨缪尔森相妨美。弗里德曼将其芝加
哥方法总结如下:

\begin{quotation}
  \textbf{在阐述经济政策时,“芝加哥”象征着对作为组织资源手段的自由市场效率的信
    任,象征着对政府事务的怀疑,象征着对货币数量作为形成通货膨胀主要因素的重视。}

  在阐述经济科学时,“芝加哥”象征着一种方法,该方法严肃地将\textbf{经济理论的运
    用}作为分析相当广泛具体问题的一种工具,而不是当做虽然非常完美但力量很小的一种
  抽象的数学结构;该方法坚持对理论概括进行经验检验,以同样的方式\textbf{拒绝没有
    理论的事实和没有事实的理论。}
\end{quotation}

弗里德曼的经济学方法是马歇尔式的而非瓦尔拉斯式的。他将经济学视为用来解决实际问题
的分析的发动机,并认为,不应当把经济学变成一种抽象的数学上的考虑,从而全无制度关
联以及与现实世界问题的直接联系。

在弗里德曼考察政策问题时,他将对\textbf{个人权利与自由}的强烈信念,与对市场保护这
些权利时的有效性的强烈信念相结合(参见《资本主义与自由》,1962)。他的政治导向基本
上是\textbf{亲市场和反政府}的。他提倡的很多政策建议,诸如\textbf{使用凭证来资助教
  育,取消职业许可,使毒品合法化}等,最初被看做是激进的,但后来就变得可以接受了。

大约在1950年,弗里德曼创作了大量关于方法论的具有煽动性的论文,以及一篇关于马歇尔
需求曲线和货币边际效用的论文。在20世纪50年代后期,他在其《货币数量论研究》(1956)
中,为宏观经济学做出了贡献。很多人阅读他在《新闻周刊》上的专栏,一部题为“自由选
择”的电视系列剧,给予了他比大多数理论家更大的名声。他于1976年获得诺贝尔经济学
奖。

正当弗里德曼越来越知名的时候,其马歇尔式的方法却面临衰亡,这部分地是因为其马歇尔
式的方法,被很多人视为\textbf{受到意识形态或者规范化的影响,从而导致研究者回归形
  式主义},以避免意识形态上的偏见。在以罗纳德·科斯(Ronald Coase,1910--)命名的科
斯定理中,能够看到被一些经济学家认为是经济学\textbf{芝加哥方法中规范化偏见}的一个
例子。科斯是另一位有影响的芝加哥经济学家,他的研究导致了最近的法经济学领域的发
展。\textbf{科斯定理是对庇古方法的回应,庇古方法将外部性的存在看做是政府干预的一
  种理由。在“社会成本问题”中,科斯认为从理论上看,外部性并不是政府干预的理由,
  原因在于,因一项行为而得到帮助或受到伤害的任何一方,都能自由地与其他人谈判来消
  除外部性。因此,如果工厂释放了大量的烟尘,那么,受到烟尘伤害的邻近居民可以支付
  工厂以减少烟尘排放。}

科斯定理在文献中被大量予以讨论。一般的结论是,它在本质上不是意识形态的,只不过是
事先影响人们对政府干预态度的外部性理论。涉及政府干预的问题很复杂,根据理论是得不
出结论的;在现代经济学中,一个政府失灵理论总是与一个市场失灵理论并存。哪一个更适
当,取决于相对成本与收益,这些都是个人无法取得一致意见的事情。

不过,芝加哥方法激励了很多新思想,正是芝加哥方法而不是更加形式主义的方法,为微观
经济学在未来的重要发展播下了种子。在受到激屿的这些新思想中,有阿门·阿尔钦和哈罗关
于作为市场基础的财产权的研究。因为芝加哥方法认为,市场有效地发挥作用的假设是最好
的,所以,大量对市场无效率的论述(例如,有可能来自于垄断竞争)都是行事不当。但是,
市场依赖于财产权;因此,对财产权的研究对经济学来说具有极为重要的意义。什么是根本
的财产权?它们是如何形成的?它们是怎样变化的?

弗里德曼最重要的后继者是\textbf{加里·贝克尔}(GaryBecker,1930--),他于1992年获
得诺贝尔经济学奖。他利用微观经济模型来研究有关\textbf{犯罪、求偶、婚姻以及生育}方
面的决策。贝克尔表示,基于理性个人假设的简单的最大化微观经济模型,具有潜力无穷的
应用性,最近几年已经见证了它被用于广泛的领域。\textbf{经济理论对其他学科的这些侵
  入,有时被那些宣称经济学方法过于简单的人以开玩笑的方式加以对待。在某种意义上来
  说,他们是正确的。“万事万物的经济学”的观点与政策结论经常显得简单化。但是,仅
  仅是简单化并不表明它们是错误的。}市场动机对人们的行为来说是重要的,非经济专家在
他们的分析中,经常并不把对这些动机的充分考虑包括进去。但是,\textbf{当只考虑经济
  动机,不充分重视制度与社会动机时,分析又会误入歧途。不幸的是,考虑到现代经济学
  家所受到的非关联的建模训练,这一点是经常发生的事情。}

随着米尔顿·弗里德曼及其同事乔治·斯蒂格勒的退休,以及随着加里·贝克尔临近退休,艺加
哥经济学在变化,变得更加数学化而较少直觉化。它没有停留在简单的模型上,而是沿着瓦
里安所建议的路线将模型一般化。显然,芝加哥经济学已经成为现代经济学流派,并且,现
代经济学流派已经变得相当同质化了。一个人在哈佛大学、芝加哥大学、麻省理工学院、斯
坦福大学或者任何顶尖研究生学院的研究生项目中学到的,本质上都是相同的东西。

\section{现代应用经济学问题}

现代应用性建模运动在很多方面是值得称赞的。它通过建模从经验上对证据进行分析,并试
图避免较早阶段所特有的武断地发表意见的情况。但是,它也存在问题。因为它与一般均衡
理论的关联被消除了,所以就\textbf{不存在理论核心来限制假设}。简单地说,人们会认为
现代应用经济学主要是\textbf{凭借“科学经验检验”的外表进行数据采集},这些外表是为
了使之看上去不是太特别而添加的。请不要误会:采集数据并没有任何问题。通过考察数据,
人们能够发现关于经济体的很多东西。但是,使用这样的一种方法削弱了一个人在形式上和
在统计上检验结果的能力。\textbf{如果模型的选择是特别的,那么,结果就是特别的。}这
并不意味着模型不能进行非形式上的检验,不能与现实进行比较。然而,现代经济学主要强
调的是模型形式上的经验检验,虽然存在非常多的表面上的形式检验,由于检验所必需的条
件得不到满足,很多检验并不令人满意。(对于这些问题的讨论,参见第16章。)

问题因专业中存在的\textbf{出版动机}而突出。这些动机通常促使经济学家选择\textbf{特
  别的实用模型}——因为它们具有被出版的可能性——这些模型要求统计意义的实证检验以及经
验统计的适应性,而\textbf{不要求合理的结论}。这些问题是严重的,但它们并不是新古典
经济学的问题。实际上,它们是逐渐出现的问题,原因是现代经济学已经远离新古典假设,
变得更加折衷。

现代经济学处理这一问题的一种方式是借助复杂性理论家的研究。他们的研究提供了一种替
代一般均衡基础的选择。根据复杂性方法,人们认为像\textbf{总量经济}这样非常复杂的事
物,不具有形式上的分析基础,因此,我们对它的理解就必须通过替代手段来完成。在复杂
系统中,秩序本能地随着模式的出现而形成。在对动力学和迭代过程的研究中,而不是在结
构简化中,能够找到复杂系统的简化。根据复杂性方法,任何事情都是数据采集,但是,它
是在才刚刚形成的具体法则下完成的非常精细的数据采集。它仍然是一种建模方法,但研究
是利用计算机模拟完成的。回顾最近的这些发展时,有一个人位居现代经济学的中心,他便
是约翰·冯·诺依曼。他和奥斯卡·摩根斯坦1944年关于博弈论的著作,指出了通过博弃论扩展
一般均衡的方法,他对人工智能和计算机的研究,是经济学复杂方法的基础。不断下降的计
算成本将在21世纪向前推进这种方法,在将来的版本中,我们将有可能见到对冯·诺依曼的更
多论述。


\section{新古典微观经济学与现代微观经济学的比较}

让我们通过论述能够区分新古典经济学与现代经济学的六个方面的特征来结束这一章。

\begin{enumerate}
\item 新古典经济学集中于\textbf{既定时期的资源配置}。这一特征体现在罗宾斯的定义
  中——\textbf{稀缺资源在可供选择的用途之间的配置}——这已经成为对新古典经济学的标准
  定义。

  很久以前,对既定时点资源配置的关注就结束了。在其位谋其事。现代经济学研究的关注
  点转向了随\textbf{时间变化}的配置,这是一个更难的问题。例如,在20世纪90年代期间,
  增长是一个主要话题,\textbf{新增长理论}断然成为主流,并且是非新古典的。实际上,
  它通常与新古典增长理论形成对比。

\item 新古典经济学认为功利主义的一些变种在理解经济体时发挥重要作用。接近需求与主
  观选择理论,远离对供给的考察,是早期新古典思想的特点。昌然最初几平完全集中于功
  利主义与需求,但是很快就发展为下列观点,即需求只是剪刀的一个刀片。

  现代经济学家很少有人认同\textbf{功利主义}一一大多数人将其视为仅仅存在于历史上一
  一今天也很少运用\textbf{效用理论}。阿玛蒂亚·森在其获诺贝尔奖的演讲中阐述了功利
  主义的问题。虽然事实上在原理和中级教科书中,学生们仍然被灌输着功利主义的观点,
  然而,这些都仅仅是为了教学上的原因而呈现出来的,并不是因为功利主义是现代经济学
  的盛行方法。

\item 新古典经济学集中于\textbf{边际权衡}。随着微积分延伸到经济学中,这种情形就开
  始存在,其最初的研究是以边际权衡为中心的,这也是微积分所关注的。

  尽管很多大学本科教科书仍然在一个边际框架中呈现经济学,但这并不是研究生学院呈现
  经济学的方式,也不是顶尖经济学家考虑问题的方式。实际上,20世纪30年代在前沿理论
  中,为了获得真知灼见而被加以挖掘的\textbf{微积分已经被丢弃},所使用的数学正在转
  向\textbf{集合论与拓扑学},原因是经济学家试图扩展经济学的范围,以包括更宽泛种类
  的议题。在现代研究生微观经济学中,博弈论几乎完全取代了微积分成为主要的建模工
  具。

\item 新古典经济学假定有远见的理性。为了在受约束的最大化框架中组织一个经济问题,
  人们就得以一种符合受约束的最优化的方式来详细说明理性。明确的理性假设很快成为新
  古典方法的核心。

  对功利主义关注的减少,伴随着有远见的理性假设的减少。在现代经济学中,有限的理性、
  基于规范的理性(或许是通过进化博弈论而确立的),以及经验确定的理性,都是被充分
  认同的解决问题的方法。

\item 新古典经济学认同方法上的个人主义。这一假设和它前面的两个假设一样,与受约束
  的最大化方法密切相连。一些人一定在进行最大化决策,在新古典经济学中,这是个人的
  行为。一个人是从个人理性开始的,市场将个人理性转化为社会理性。

  虽然个人主义依旧盛行,但是,它受到现代经济学某些分支的抨击。复杂性理论家向完全
  的个人主义方法提出挑战,至少是当这种方法被用来理解总量经济时。进化博弈理论家也
  正在试图表明,博弈论的规范是如何促进和抑制行为的。新制度主义者一贯是在与方法上
  的个人主义不一致的框架中进行操作的。

\item 新古典经济学是围绕经济体的\textbf{一般均衡}概念构建的。与其他的特征相比,最
  后这个特征引起了更多的争议。熊彼特使经济体的一般均衡概念成为其界定新古典经济学
  的中心。这一点无可否认地很重要,但是,如果它是绝对的中心,那么,就会将马歇尔从
  新古典学派中清除出去。然而,\textbf{熊彼特}在下列方面是正确的:为了使新古典经济
  学不只是一种解决问题的应用政策方法(熊彼特希望做的一些事情),\textbf{人们需要一
    个唯一经济体的一般均衡概念。正式的福利经济学就是建立在这种一般均衡概念基础之
    上。}

  存在一种唯一的一般均衡,这仍然是被主流所持有的观点,但是,这主要是因为一般均衡
  模型很少被使用。理论上对多重均衡的研究正在进行中,均衡选择机制是研究的一个重要
  组成部分。新古典经济学从未认真地考虑过多重均衡的问题。在现代经济学中,理论经济
  学家非常愿意考察多重均衡,就像在诸如卡尔-谢尔(Karl Shell)和和迈克尔:伍德福
  特(Michael Woodford)一类的经济学家的研究中能够看到的那样。毫无疑问,\textbf{对
    政策的现代研究通常避免了讨论多重均衡,这也是现代经济学的矛盾之一,}但是,多重
  均衡这一主题不再是被禁止的了。
\end{enumerate}

\section{总结}

现代微观经济学从新古典而来,并得到了重要的进化,现代微观经济学通过折衷的形式化建
模方法得到了更好的界定。它的根源是在古诺和埃奇沃思身上,而不是在马歇尔身上找到的。
随着约翰·R·希克斯的《价值与资本》和保罗·萨缪尔森的《经济分析基础》的出版,在20世
纪30年代后期,人们开始远离马歇尔经济学,因为马歇尔避免使经济理论形式化,这使一些
经济学家受到阻碍,希克斯与萨缪尔森的研究是对很多年来这种阻碍的终结。紧跟他们研究
的是阿罗与德布鲁在其研究中给予新古典思想更多的形式化。在这项研究结束后,现代微观
经济学转向折衷的应用政策研究,其中的假设有别于核心的一般均衡假设。

在贯穿于20世纪70年代的芝加哥学派中,仍然能够找到马歇尔的方法,尽管马歇尔经济学的
芝加哥分支具有一种强烈的亲市场偏见。随着米尔顿·弗里德曼的去世和加里贝克尔临近退休,
以及乔治·斯蒂格勒的去世,芝加哥经济学逐渐合并到现代经济学中,变得更多数学化、更少
直觉化。

\textbf{现代经济学也存在问题,但它们并不是新古典经济学的问题。}现在的一些经济学家
将他们对经济学的批判对准新古典经济学,他们这样做,是未能弄清表现现代经济
学\textbf{特色的流行范式}。


%%% Local Variables:
%%% mode: latex
%%% TeX-master: "../../main"
%%% End:
