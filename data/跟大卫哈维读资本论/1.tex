\part{《跟大卫哈维读资本论》VOL1和《资本论》VOL1笔记}
\label{HarveyVol1}

本文借助于《跟大卫哈维读《资本论》第一卷》一书,对《资本论》第一卷的一些概念本身
和它们彼此之间的联系进行了简单概括和整理。

\textbf{Note标记,不是原文自带,而是对笔者个人来说,Note段落有助于理解中国当前形势的内容。}

\chapter{天才资本家海大富的《资本论》第一卷实践记}
\label{cha:Capital}


我们首先假设在18世纪末,一个名为海大富的英国人作为“可敬的东印度公司”的一位员工,
被派往印度,收取印度人民田赋。靠着高的离谱的田赋,肥了东印度公司,也肥了海大富,
还有就是饿死了很多印度人。

海大富比较恋家不喜漂泊异国,带着他的第一桶金回到英国曼彻斯特市。当时英国正在实行
“公有地圈围法”\pagescite[][832]{capital},地主借以把人民的土地当做私有财产赠送
给自己,许多农民失地,并且失去了生产资料,向城市游离寻求工作。之前的殖民掠夺和这
里使农民与生产资料剥离的活动都是资本的原始积累。海大富适时地开办了一家棉纺织工厂,
并招入多名廉价失地农民。\bigskip


产品生产过程中,转变为生产资料即原料、辅助材料、劳动资料的那部分资本,在生产过程
中并不改变自己的价值量,叫作不变资本。转变为劳动力的那部分资本,在生产过程中改变
自己的价值。他再生产自身的等价物——体现必要劳动时间的、等同于可变资本的劳动力价值,
和一个超过这个等价物而形成的余额——剩余价值。 \bigskip


资本主义经济的运行规律导致了资本家们无限制的竞争,竞争的强制规律又导致资本家对于
增加剩余价值的不懈追求,海大富向往自由轻松但也不能免俗,只能坠入这无休止的循环。

他引进了先进机器。因机器的引入,使生产力得到极大提高,另外可简单通过提高机器速度,
扩大工人劳动范围——增加工人可以控制的机器数目等手段增加工人劳动强度。缩短生产商品
的社会必要劳动时间,增加工作日中必要劳动时间和剩余劳动时间的量的比例,由此而生产
的剩余价值叫作相对剩余价值。

工人们的主粮是面包,面包是他们基本和占比较多的生活资料。恰逢英国面包业生产力大幅
提高,导致面包价格下降较多,导致整个资本家阶级受益,所有的工人劳动力价值降低,这
也使相对剩余价值得到提高。 \bigskip


海大富以前就通过直接延长工作日来获取更多利益。引入机器后,因为机器不使用所带来的
没有剩余价值作为补偿的折旧费用(虽然与因为使用而带来的机器损耗相对立),和机器面
临的无形损耗——现有机型贬值或者出现更好机器——所带来的现有机器交换价值贬值,更加促
使海大富延长了工人劳动时间。通过延长工人工作日所生产的剩余价值,叫作绝对剩余价值。
\bigskip


随着工作强度、时长日益增大,工人们也越来越不满,常常嚷嚷着工厂要遵守劳动基本法,
还要求一天工作最多十小时,还说不然就闹罢工,砸机器,上法院。海大富心想要及时制止
住这股歪风邪气,就从牛津请了个砖家西尼尔来给工人们上课。砖家一张口,就知有没有,
西尼尔给工人们算了一笔账,疾呼“你们工厂最后一小时才产生利润啊,海老板是活菩萨
啊”,感动得工人们鼻涕一把泪一把,还没听完演讲就回去继续十二小时工作了。海大富都
要被自己感动了,仔细一算,才明白西尼尔的把戏简单来说是将被消费的生产资料与转化生
产资料为产品价值的劳动力,这两个本来同时进行的部份进行了分离压缩,然后只对工人付
出的劳动进行了大减法,却没减去分毫生产资料的价值。\bigskip


生产完商品后,就是卖商品了,我们睿智的、从不失败的资本家海大富总能顺利卖出他的商
并得到货币\footnote{马克思在《资本论》第一卷中一般假定资本按正常方式完成自己流通
  过程,要到第二卷才论述资本流通过程。另外,马克思站在宏观一般分析角度上,还假设资本
  主义生产者独得全部剩余价值,没有将其分为不同类人占据的不同部分。如利润、利息、
  租金、税收等。}。商品出售后所得的货币额,包含海大富最初以货币形式预付的资本价值
(不变资本+可变资本)和剥削工人劳动所得的剩余价值。海大富将剩余价值再转化为资本这
叫做资本积累。具体地说,积累就是资本以不断扩大的规模进行的再生产。 \bigskip

海大富是个脸盲,他没有办法从相貌看出任何两个人之间的不同之处,但在他的内心中,他、
是以价值来衡量和标记一切他物他人的。每件物品都有他的价值,每个人都有他的价值,每
个工人都生产他们的价值。海大富心中只有商品,没有人名,没有人和人之间的社会关系。

有评论家指出\[\mbox{剩余价值率} = \mbox{剥削率} = \frac{\mbox{剩余价
值}}{\mbox{可变资本}} = \frac{\mbox{剩余价值}}{\mbox{劳动力价值}}\],并狂言资本
家追求  的就是剥削率。乖乖,我们海大富是个成熟的、睿智的资本家,但绝不是一个虐待狂
和剥削狂啊。他对这一无耻言论表示非常遗憾。他说一个成熟的资本家所关心的其实是利润
率,也就是\[\mbox{利润率} = \frac{\mbox{剩余价值}}{\mbox{不变资本}+\mbox{可变资
本}}\],虽然剥削率和利润率公式的分子是相同的,但这只是一个意外。并且作为一个成熟
资本家来说,看见利润率这个公式其实是非常敬畏的,因为随着劳动率的提升,不变资本在
预付资本中所占比重必将越来越大,这有可能是驱使资本主义进入利润率下降的周期性危机
之一。据说一个著名的马姓学者在一本名为《资本论》第三卷的书中对此有所论述
\pagescite[][144]{davidcapital1}。

脸盲与只重利润率不重剥削率均为商品的拜物教性质。

商品价值是靠创造价值的工人劳动得以体现,但消费者购买商品时,只能看到死的物,而不
知工人,这也是拜物教性质。

为了价值增殖,资本家海大富也不拥有真正的自由,他必须将利润(剩余价值)进行再投资,
以实现资本的再生产,这也是商品的拜物教性质,\textbf{它将人与人之间的社会关系,替
  换为人与人之间的物质关系和物与物之间的社会关系}。人们扮演的经济角色不过是经济关
系的人格化,不止工人,资本家也不例外,海大富只有在行使资本职能时才是资本家,资本
家是资本的人格化。 \bigskip

海大富的生意越做越大,劳动生产率明显比其他同行要高,工人们贡献的剩余价值也大。偶
有工人要求涨工资时,他就说“外面劳动市场那么多人没活干,你还嫌工作时间长,工资
低?”工人们立刻就不敢作声了。

海大富凭借着自己商品剩余价值比其他同行高一截的优势,略微降价就把其他同行们打的七
零八落,有些同行变成了工人,有些同行把自己厂子卖给了他。人们都喜欢资本英雄,纷纷
带着自己的钱来投资海大富厂子,经过一系列兼并整合,海大富成立了棉纺织托拉斯股份公
司,全国棉布供应均由这个垄断的托拉斯说了算。

虽然海大富聪明智慧,生意场上屡屡得胜,仍感觉竞争压力巨大,对资本增殖的欲望也日益
增强。通过借助货币资本的信用手段,他一步步纵向垄断产业链,横向价格战或收购或合作
垄断了本行业竞争对手。在这一过程中,海大富公司从卡特尔、辛迪加扩张到托拉斯垄断。

垄断之路实在是太艰辛。现如今不止工人,就连资本家群体、国家也怨声载道,埋怨海大富
剥削了所有人。号召联合起来抵制、打压、拆分海大富。海大富发现自己风头太盛,惹人嫉
妒。

但他已经发现了一个道理——钱可以生钱。这也是资本拜物教。一个罗马时代的万元户,靠每
年放贷10\%的收益,那么到了18世纪,所拥有的财富将比全世界有货币以来的所有财富加起
来还要多的多。于是,他顺势响应号召,主动将一些垄断资产较为低廉的出售,并投资慈善
事业。成为举世闻名慈善家的同时,海大富藏身幕后,将巨额金钱投入到了银行、金融,甚
至在一定程度上操控了货币,海大富成为了超级信用金融资本家。知道内情的权贵富豪们背
地里称他为康采恩·海。

海大富后半生致力于办大学,办基金会,宣扬资本自由化、市场看不见的手大于国家、组织
管制的“自由价值观”,并名垂青史,令人仰望。

王权是什么鬼?总统、首相、世界政治不过是海大富的手中玩物,他要谁当谁就当。海大富
一生未出任政府要职,每当别人说他统治世界,他都以此否认,我连个科长都不是。这叫啥?
这叫统治阶级不统治。


\chapter{大卫·哈维对资本论的解读笔记}

大卫·哈维书中对《资本论》第一卷进行了辩证的导读和再评估,除对马克思的敬仰外,也
结合《资本论》问世150余年以来世界历史和思想的变动状况,以及个人知识背景和社会、
政治阅历对《资本论》一些内容做了修正或批评。

以下几点是哈维对马克思的几点解读和批评,这些解读和批评在笔者看来是务实且具进步意
义的。


\begin{enumerate}

\item 只有读至《资本论》的结尾,我们才能充分理解这些概念的作用。……马克思就是从
洋葱表皮开始,透过外部显示的层层包裹,一直到达它的核心,即概念的内核(剩余价值)。
随后,他再次将论点向外展开,通过不同的理论层次而逐渐返回表面(货币资本等)。当回
到实践王国时,论点的真实力量才变得更为明确。\pagescite[][9]{davidcapital1}


\item 原始积累。哈维继承和发扬了罗莎·卢森堡的“资本主义的长期历史是以这种动态关
系为中心的,即持续的原始积累和贯穿于《资本论》所描述的扩大再生产体系的积累动力之
间的动态关系”,哈维倾向于将原始积累“称为通过剥削手段而进行的积累……特别是通过
采用殖民主义和帝国主义策略并对自然资源的掠夺性抢
劫”\pagescite[][328-336]{davidcapital1}。也就是说,原始积累并非是马克思认为的发
生在资本主义生产方式确立以前,而是贯穿资本主义生产方式始终的。

\item 经济基础决定上层建筑……上层建筑又反作用于经济基础。马克思在一些著作中论述
过经济基础和上层建筑的概念,但是从没有下过决定论。哈维认为这观点带有错误的宿命论
和因果关系,失却了马克思的辩证哲学,是被马克思的“朋友和他的敌人们”误解了
\pagescite[][211-218]{davidcapital1}。哈维认为,根据《机器和大工业》这一章节的第
4个脚注,与自然的关系、技术、生产方式、社会关系、日常生活的再生产、对世界的精神
观念这六者是一个生态的整体,每个时刻都紧密地内化了其他所有时刻。虽然哈维在此说六
者互相内化的这一框架有助于通过一个基本的方法,将历史唯物主义作为基础。笔者却感觉
这恰恰是否定了历史唯物主义,希望诸位大神指导下我,答疑解惑。

根据以上说法,哈维认为马克思所说“要学会把机器和机器的资本主义应用区别开来”是错
误的,因为哈维据以上六个方面互相内化的理解,认为“技术和社会关系是相互整合的”,
“技术在社会整体中,不是中立的”\pagescite[][237-238]{davidcapital1}。


\item 哈维针对“拜物教”的了解敏感而深刻。他常指出《资本论》中贯穿全文的显性或者
隐性提及的“拜物教”现象,并对这一现象做了相当精彩的阐述。笔者在上文笔记中有所提
及,在此不再赘述。

\item 建立在劳动价值论之上的剩余价值论。“价格是否可以在它们的价值之外附加其他的
东西?它们是否在任何情况下都会出现与价值无关,在所有范围内的量化波动,并且马克思
为什么如此关注劳动价值论?……马克思在这里没有维护他自己的判断……我认为马克思可
能会求助于物质基础的概念:如果每个人都试图以瀑布的壮丽场景为生,或进行良心和荣誉
的交易,那么将没有人能够生存。实际生产,通过劳动过程的实际的自然变换对我们的存在
至关重要……”\pagescite[][64]{davidcapital1}。哈维认为马克思所做的论述,相比起
对现实存在的资本主义的演化做一个历史性的论述来说,更多的是在进行一种逻辑的陈述
(以对古典自由政治经济学的乌托邦主张的批判为基础)
\pagescite[][94]{davidcapital1}。像哈维所说,“如果你想按照马克思的原意理解《资
本论》,你就必须准备好在某种程度上,按照这些原则去接受某个观点,至少直到你读到书
的结尾时都应该是这样”。笔者认为,马克思的兴趣,相较于对政治经济原理的细微解析来
说,其实更多将重心放在人类的异化、被剥削、被损害上,所以他不喜欢承认自己是经济学
家,也不喜欢承认自己是社会学家,这两方面在他看来都无视或轻视了人类被剥削的更深层
的原因和关系。也如哈维所说,马克思是有浪漫色彩的。

\item 通过阶级策略提高相对剩余价值。因为政府干预劳动价值,如纽约市不对食品征税,
还有在英国历史上因资产阶级压倒封建地主阶级导致《谷物法》的取消,这些有意识的阶级
策略和国家干预使得工人能够获得的生活资料的数量增加了,从而造成工人劳动力价值降低,
进而提高了相对剩余价值。(马克思理论框架里没有这些,他坚持对自由市场的乌托邦主义
的限制性假设\pagescite[][183]{davidcapital1}。

\item 供需关系。哈维说,“马克思认为,供需关系在一种特定商品形成价格的变动中起到
  了至关重要的表面作用,但是当供求达到均衡时,他认为,供给和需求就不能解释任何事
  情了……这必须由完全不同的原因来解释,即凝结的社会必要劳动时间或价
  值”\pagescite[][184]{davidcapital1}。借鉴自劳动的“必要价格”(重农学派)
  或“自然价格”(亚当·斯密)。

\item 工厂模式。哈维认为马克思受限于曼彻斯特大工厂模式,没有考察伯明翰之类地方的
手工合作模式等。资本主义倾向于保留一种对劳动体系的选择。如果它们不能通过工厂体系
获得足够的利润,他们就愿意回到对家庭体系的选择。哈维认为,资本主义生产方式的一个
长期特点(选择)是,不同劳动体系之间的竞争,将成为追求剩余价值的斗争中资本用来对
抗劳动的工具。笔者认为这一方面两人都很难说错,中国当代的互联网公司就像当时的曼彻
斯特大工厂,甚至远远有过之而无不及,巨额资本量、涉及多种产业、产业链和优势产业的
垄断模式等,马克思大工业的论述在这里常常高度契合。而哈维说的也无错,真正的资本市
场,是活跃的,多样的,像马克思所说是革命性的生产方式。这方面现实的经验已经足以脱
离理论直接证明这点\pagescite[][245]{davidcapital1}。

\item 一般与个别。

  \begin{enumerate}
    \item 马克思在《大纲》中提出,他写作《资本论》的目的为阐明资本运动的一般规律。
这就意味着《资本论》集中于阐明剩余价值的生产和实现,抽象和排除掉他称之为“个别
的”分配(利息、租金、税收、甚至实际工资和利润,哈维认为如果对这些个别分配细致分
析的话可能产生不一样的结论),因为它们都是偶然的、相关的和具体时空瞬间的。对供求
关系也是做得平衡的假设。(详情请看读资本论第二卷)

    \item 《大纲》中,马克思认为,消费是最难分析的,如同对使用价值的研究
\footnote{在大卫·哈维另一本书《叛逆的城市》中文版中第37页译者注中,译者叶齐茂、
倪晓辉认为哈维将“使用价值”和“消费”做了替换,两位译者不认同这种做法。},消费
“实际上属于经济学之外的领域。”。
    \end{enumerate}哈维认为这导致我们需要超出马克思已经提出的见解,创造出新的理
论,如信用、利率、城市化、经济危机等。

\end{enumerate}

\chapter{经典原文}

\begin{quotation}
  工业较发达的国家向工业较不发达的国家所显示的,只是后者未来的景象。\pagescite[][8]{capital} 

  
劳动生产力是由多种情况决定的,其中包括: 工人的平均熟练程度,科学的发展水平和它在
工艺上应用的程度,生产过程的社会结合,生产资料的规模和效能,以及自然条
件。\pagescite[][53]{capital} (在 \pagescite[][26]{davidcapital1},哈维将这里的
劳动生产力与价值画了等号?)

  劳动作为使用价值的创造者(Note: 通过劳动加工自然物质),作为有用劳动,是不以一
切社会形式为转移的\textbf{人类生存条件},是人和自然之间的物质变换即人类生活得以实
现的永恒的自然必然性。\pagescite[][56]{capital}

  同商品体的可感觉的粗糙的对象性正好相反,在商品体的价值对象性中连一个\textbf{自
然物质原子}也没有。……商品只有作为\textbf{同一的社会单位}即人类劳动的表现才具有
价值对象性,因而它们的价值对象性纯粹是社会的,那么不用说,\textbf{价值对象性只能在
商品同商品的社会关系中表现出来}。\pagescite[][61]{capital} (Note: 哈维说,价值
是一种社会关系,实际上你无法直接看到、摸到或感觉到社会关系;他们是非物质但却客观
存在的。那么马克思是严格唯物主义吗?)

  希腊社会是建立在奴隶劳动的基础上的,因而是以人们之间以及他们的劳动力之间的不平
等为自然基础的。(Note: 在此不平等的生产方式基础上,亚里士多德认为“不存在没有平
等的交换……也不存在不可通约的平等。”)

  (哈维对价使用价值、交换价值和价值的看法)这三个不同的概念可以基本被内化于不同的
时空中。使用价值存在于事物的物理物质世界,交换价值包含在运动的相对时空和商品交换
中,而价值只能以世界市场的相关时空概念来理解(社会必要劳动时间的非物质的关联性价
值存在于正在演变的、资本主义全球发展的时空中)。

  对人类生活形式的思索,而对它的科学分析,总是采取同实际发展相反的道路。这种思索
是从事后开始的,就是说,是从发展过程的完成的结果开始的。给劳动产品打上商品烙印、因
而成为商品流通的前提的那些形式,\textbf{在人们试图了解它们的内容而不是了解它们的
历史性质 (人们已经把这些形式看成是不变的了) 以前,就已经取得了社会生活的自然形式
的固定性。}因此,\textbf{只有商品价格的分析才导致价值量的决定,只有商品共同的货
币表现才导致商品的价值性质的确定。} 但是, 正是商品世界的这个完成的形式—— 货币形
式,用物的形式掩盖了私人劳动的社会性质以及私人劳动者的社会关系, 而不是把它们揭示
出来。\pagescite[][93]{capital}

这种种形式恰好形成资产阶级经济学的各种范畴。对于这个历史上一定的社会生产方式即商
品生产的生产关系来说,这些范畴是有社会效力的、因而是客观的思维形式。因此,一旦我们
逃到其他的生产形式中去,商品世界的全部神秘性,在商品生产的基础上笼罩着劳动产品的一
切魔法妖术,就立刻消失了。\pagescite[][93]{capital}

设想有一个自由人联合体,他们用公共的生产资料进行劳动,并且自觉地把他们许多个人劳动
力当作一个社会劳动力来使用。……在那里,鲁滨逊的劳动的一切规定又重演了,不过不是在
个人身上,而是在社会范围内重演。鲁滨逊的一切产品只是他个人的产品,因而直接是他的使
用物品。这个联合体的总产品是社会的产品。这些产品的一部分重新用作生产资料。这一部
分依旧是社会的。而另一部分则作为生活资料由联合体成员消费。因此, 这一部分要在他们
之间进行分配。 这种分配的方式会随着社会生产机体本身的特殊方式和随着生产者的相应的
历史发展程度而改变。(Note: 资本论中,为数不多的马克思对于社会主义未来的描述
。)\pagescite[][96]{capital}

价格和价值量之间的量的\textbf{不一致的可能性},或者价格偏离价值量的可能性,已经包含
在价格形式本身中。 但这并不是这种形式的缺点,\textbf{相反地,却使这种形式成为这样一
  种生产方式的适当形式,}在这种生产方式下,规则只能作为没有规则性的盲目起作用的平均
数规律来为自己开辟道路。\pagescite[][123]{capital}(Big Note)

价格形式不仅可能引起价值量和价格之间即价值量和它的货币表现之间的量的不一致,而且能
够包藏一个质的矛盾,以致货币虽然只是商品的价值形式, 但价格可以完全不是价值的表
现。 有些东西本身并不是商品,例如良心、名誉等等,但是也可以被它们的所有者出卖以换取
金钱, 并通过它们的价格, 取得商品形式。因此,没有价值的东西在形式上可以具有价格。在
这里,价格表现是虚幻的,就象数学中的某些数量一样。另一方面,虚幻的价格形式—— 如未开
垦的土地的价格,这种土地没有价值,因为没有人类劳动物化在里面—— 又能掩盖实在的价值关
系或由此派生的关系。(哈维Note: 价格是否可以在它们的价值之外附加其他的东西?它们
是否在任何情况下都会出现与价值无关、在所有范围内的量化波动,并且马克思为什么如此
关注劳动价值论?……我认为马克思可能会求助于物质基础的概念:如果每个人都试图以瀑
布的壮丽场景为生,或进行良心和荣誉的交易,那么将没有人能够生存。实际生产,通过劳
动过程的实际自然变换,对我们的存在至关重要;而且正是这种物质劳动形成了人类生活的
生产和再生产的基础。\pagescite[][65]{davidcapital1})(Big Note)

有些人喜欢将辩证法严格地理解为论点、对比和综合。但是马克思在这里所阐述的是没有经
过综合的论点。这里只存在矛盾的内化和更大程度上的适应。矛盾永远不能得到最终解决;
它们只能在一个永恒运动的体系中(像椭圆),或者在更大范围内被复制。……这种矛盾存
在一种永久扩张的趋势。

由于这一原因,我对将马克思的辩证法当成一种封闭的分析方法的人们有些不耐烦。他不是
确定的,相反,他不断地扩大,而且在此它正在准确地解释它是如何扩大的。……这个论点
的变化是一个持续的重塑过程,包括对矛盾陈述的重新措辞和扩展。这也解释了为什么在
《资本论》中会出现这么多重复的地方。马克思的思想每前进一步都需要追溯到前面所论述
的矛盾,以便解释下一个矛盾将会在哪里出现。(Big Note)\pagescite[][67]{davidcapital1} 

凯恩斯花费了大量精力研究他所称的“流动性陷阱”,在其中一些市场上发生,使那些持有
货币的人感到紧张,所以他们更愿意持有货币而不是用于投资或花掉它,从而造成了对商品
需求的下降。突然间人民不能卖掉他们的商品。不确定性越来越多地困扰着市场,而且更多
的人也都紧紧抓住他们手中的货币,这是他们安全的保证。其后,整体经济就呈螺旋状下降。
凯恩斯的观点是,政府必须接入经济,并通过创造不同的财政刺激政策来扭转这一进程。随
后私人储藏的货币将被引诱回市场。\pagescite[][72]{davidcapital1} (Big Note)

货币和社会必要劳动时间的关系,甚至对黄金来讲都有问题,而且问题已经变得更加深入和
令人难以捉摸了。……并不是说这个问题不存在。国际货币市场上的混乱与不同国家经济体
中的物质生产率的不同有关。马克思所提出的货币形式和商品价值之间存有问题的关系,目
前仍伴随着我们,而且他率先提出的分析提纲是非常开放的,虽然当代的表现形式与当时完
全不同。\pagescite[][75]{davidcapital1} 

在质的方面,或按形式来说,货币是无限的,也就是说,是物质财富的一般代表,因为它能直接转
化成任何商品。但是在量的方面,每一个现实的货币额又是有限的,因而只是作用有限的购买
手段。货币的这种量的有限性和质的无限性之间的矛盾,迫使货币贮藏者不断地从事息息法斯
式的积累劳动。他们同世界征服者一样,这种征服者把征服每一个新的国家只看作是取得了新
的国界。\pagescite[][156]{capital} (Note)

我们可以说,商品交换的扩大必然导致货币形式的出现,而且,在货币形式之间存在内部矛
盾,反过来,也必然会导致资本主义流通形式的出现,在其中货币被用于得到更多的货币这
就是到目前为止可以粗略总结出的《资本论》中所阐述的观点。(Note: 大卫哈维认为这是
马克思\textbf{错误的历史决定论}:资本主义的出现是人类历史上不可避免的阶段,它的出
现源于商品交换的逐渐扩大。哈维认为这是一个\textbf{历史合理的逻辑},需用“可能”、
“好像”、“也许”来替代“必须”)。

这种矛盾在生产危机和商业危机中称为货币危机\footnote{本文所谈的货币危机是任何普
  遍的生产危机和商业危机的一个特殊阶段,应同那种也称为货币危机的特种危机区分开来。
  后者可以单独产生,只是对工业和商业发生反作用。这种危机的运动中心是货币资本,因此
  它的直接范围是银行、交易所和财政。 ( 马克思在第3版上加的注)}的那一时刻暴露得特
别明显。这种货币危机只有在一个接一个的\textbf{支付的锁链}和\textbf{抵销支付的人为
  制度}获得充分发展的地方,才会发生。当这一机构整个被打乱的时候, 不问其原因如何,
\textbf{货币就会突然直接地从计算货币的纯粹观念形态变成坚硬的货币}。 这时, 它是不
能由平凡的商品来代替的。 商品的使用价值变得毫无价值, 而商品的价值在它自己的价值形
式面前消失了。(Big Note)\pagescite[][162]{capital} 

让我们回到2005年,那是存在一种共识,即在世界市场上游荡着\textbf{大量的流动性剩余}。
银行家持有剩余的基金,几乎向所有人出借,甚至包括那些没有信誉没有足够钱的人……因
为像房子这样的商品是一种安全的赌博。但是,后来房子的价格不再上升了,而且当债务到
期时,有越来越多的人无力偿还。在那个时候,流动性突然枯竭。钱在哪里?美联储突然间
不得不向银行系统注入大量的基金,因为现在“\textbf{货币是唯一的商品}”。\pagescite[][85]{davidcapital1} 

在一定的历史时刻,流动媒介的突然短缺,同样也会引起危机。从市场上收回短期信用,可
能引起商品生产的萎缩。……但是银行收回了流动性,造成了经济衰退,能够存活的公司也
倒闭了,它们由于缺乏获得支付手段的渠道而不得不出售公司。西方资本和银行随后进入并
以几乎免费的成本全部收购了这些公司。\pagescite[][86]{davidcapital1}(Big Big Note) 

国家的货币政策不会免除国家对跨越世界市场的商品交换的流动所起的规制性的影响。\pagescite[][88]{capital} 

(第二篇,货币转化为资本)我已经对下述问题提出过多次质疑,即,马克思是在进行
一种逻辑的论述(以对古典自由政治经济学的乌托邦主张的批判为基础),还是对现实存在
的资本主义的演化做一个历史性的论述。总的说来,虽然在为资本主义生产方式的增长提供
条件所必须的环境的考虑中(例如,国家的作用与不同货币形式之间的关系),可能会获得
重要的历史洞察,但我更倾向于采取哪种历史性的逻辑阅读的方式。(Big Note: 可以参看
罗莎卢森堡 《资本积累论》序言和其他部分 )\pagescite[][94]{davidcapital1} 

封建秩序、土地所有权和封建土地控制权的阶梯,在很大程度上是通过商人资本和高利贷资
本的力量完成的。你也可以在《共产党宣言》中找到对这一主题非常明确的论述。

(Big Note: 下面马尔萨斯、卢森堡、马克思的不同论述非常重要,我单独列出)
\begin{quotation}
马尔萨斯证明了一种非生产的消费阶级会永久存在(资产阶级寄生虫)。马尔萨斯还通过提
出这一消费者阶级也可能存在于国家之外,从而在某种程度上修正了他的观点,即对外贸
易,甚至外国的供奉,也有助于解决这一问题。这在后来成为罗莎·卢森堡研究的重要问
题之一,即在资本主义体系中必须存在的有效需求,最终只有靠与外部建立某种联系来保
证——简而言之,通过施加帝国主义者强加给她的对供奉的索取。

\bigskip 马克思的一些论点:

我们拿表现为单纯的商品交换的流通过程来说。……就使用价值来看,交换双方显然都能得到
好处。双方都是让渡对自己没有使用价值的商品, 而得到自己需要使用的商品。……因此,就
使用价值来看,可以说, “ 交换是双方都得到好处的交易 ” 。就交换价值来看,情况
就不同了。\pagescite[][183]{capital} 

因此,商品流通就它只引起商品价值的形式变换来说,在现象纯粹地进行的情况下, 就只引起
等价物的交换。 连根本不懂什么是价值的庸俗经济学, 每当它想依照自己的方式来纯粹地观
察现象的时候,也都假定供求是一致的, 就是说, 假定供求的影响是完全不存在的。 因此,
就使用价值来看,交换双方都能得到利益, 但在交换价值上, 双方都不能得到利益。不如
说,在这里是: “ 在平等的地方, 没有利益可言。” 诚然,商品可以按照和自己的价值相偏
离的价格出售, 但这种偏离是 一 种 违 反 商 品 交 换 规 律 的 现 象。商品交换就其纯
粹形态来说是等价物的交换,因此,不是增大价值的手段。

因此,那些试图把商品流通说成是剩余价值的源泉的人,其实大多是弄混了,是把使用价值和交
换价值混淆了。\pagescite[][184-185]{capital} 

商品之间就只有一种区别,即商品的自然形式和它的转化形式之间的区别,商品和货币之间
的区别。

假定卖者享有某种无法说明的特权,可以高于商品价值出卖商品,……但是,他当了卖者以
后,又成为买者。现在第三个商品所有者作为卖者和他相遇,并且也享有把商品贵卖10\%的特
权……商品的这种名义上的普遍加价,其结果就象例如用银代替金来计量商品价值一样。 商
品的货币名称即价格上涨了,但商品间的价值比例仍然不变。

我们再反过来,假定买者享有某种特权,可以低于商品价值购买商品。在这里,不用说,买者还
要成为卖者。他在成为买者以前,就曾经是卖者。他在作为买者赚得10\%以前,就已经作为卖
者失去了10\%。结果一切照旧。因此,剩余价值的形成,从而货币的转化为资本,既不能用卖者
高于商品价值出卖商品来说明,也不能用买者低于商品价值购买商品来说明。\pagescite[][187]{capital} 

在流通中,生产者和消费者只是作为卖者和买者相对立。说生产者得到剩余价值是由于消费者
付的钱超过了商品的价值,那不过是把商品所有者作为卖者享有贵卖的特权这个简单的命题加
以伪装罢了。卖者自己生产了某种商品,或代表它的生产者,同样,买者也是自己生产了某种已
体现为货币的商品,或代表它的生产者。因此,是生产者和生产者相对立。他们的区别在于,一
个是买,一个是卖。商品所有者在生产者的名义下高于商品价值出卖商品,在消费者的名义下
对商品付出高价,这并不能使我们前进一步。\footnote{“利润由消费者支付这种想法显然
  是十分荒谬的。消费者又是谁呢?”( 乔·拉姆赛《论财富的分配》1836年爱丁堡版第183页)}

因此,坚持剩余价值来源于名义上的加价或卖者享有贵卖商品的特权这一错觉的代表者,是假
定有一个只买不卖,从而只消费不生产的阶级。从我们上面达到的观点来看,即从简单流通的
观点来看, 还不能说明存在着这样一个阶级。但是, 我们先假定有这样一个阶级。这个阶级
不断用来购买的货币, 必然是不断地、不经过交换、白白地、依靠任何一种权利或暴力, 从
那些商品所有者手里流到这个阶级手里的。 把商品高于价值卖给这个阶级, 不过是骗回一
部分白白交出去的货币罢了。(Note: 这不正是帝国主义吗?)


可见,无论怎样颠来倒去,结果都是一样。如果是等价物交换,不产生剩余价值;如果是非等价
物交换,也不产生剩余价值。流通或商品交换不创造价值。\footnote{“ 两个相等的价值相
交换,既不增大也不减少社会上现有价值的量。两个不相等的价值相交换......同样也改变不
了社会上的价值总额,因为它给这一个人增添的财富,是它从另一个人手中取走的财富。” (
让·巴·萨伊《论政治经济学》 1817年巴黎第3版第2卷第443、444页)这个论点是萨伊几乎逐
字逐句地从重农学派那里抄袭来的。}\pagescite[][190]{capital}

\bigskip

从某些方面讲,上述关于有效需求的论述存在问题,而且罗莎·卢森堡在这一问题上,也向马
克思提出了一个强烈的挑战,她认为,\textbf{以帝国主义为主导的、与非资本主义相对立
  的社会形成,为解决有效需求问题提供了部分答案。从那以后一直存在对这些问题的争论。
  但是,在这些段落中,马克思只是关注剩余价值是如何生产的,而没有关注它可能通过消
  费,如果被支付和实现的。剩余价值必须在它被消费之前被生产出来,我们不能通过诉诸
  消费过程来了解他的生产过程。}

所以,这些关于有效需求的观点不能解释剩余价值是如何被生产出来的,特别是,如
果\textbf{“我们还是留在卖者也是买者、买者也是卖者的商品交换范围内吧。我们陷入困
  境,也许是因为我们只把人理解为人格化的范畴,而不是理解为个
  人。}\pagescite[][188-189]{capital}

我想,正是在这里,我们遇到了马克思文本中的一个真正的对抗,即他对古典政治经济学中
对于乌托邦主义趋势批判的信任,和他为我们理解并描述现实存在的资本主义性质的愿望之
间的对抗。\textbf{实际上,马克思正在谈论的是,我们必须在一个地理意义上封闭的和完
  全的资本主义生产方式内,去寻找剩余价值来源问题的答案;在这种理想的状态下,将剩
余价值的来源归因于食利阶级、消费主义,或者对外贸易的观点,都必须被排除掉。}……在
分析中,他没有将\textbf{有效需求}在一般意义上作为非相关的因素,因为在这里,在第一
卷,他只关心生产问题。只有在第二卷,他才将着手研究价值在市场和消费领域的实现问
题。\pagescite[][104]{davidcapital1}

\textbf{必须要找到另一个方法去解决,在等价交换中,一个非等价物是如何产生的这一矛盾(例
如,剩余价值)。}(Big Note)

\end{quotation}

G—W—G 的形式, 为贵卖而买,在真正的商业资本中表现得最纯粹。因而,商业资本只能这样来
解释:\textbf{寄生在购买的商品生产者和售卖的商品生产者之间的商人对他们双方进行欺骗。
  富兰克林就是在这个意义上说: “ 战争是掠夺,商业是欺骗。”}如果不是单纯用对商品生
产者的欺骗来说明商业资本的增殖,那就必须举出一长串的中间环节,但是在这里,商品流通及
其简单要素是我们唯一的前提,因此这些环节还完全不存在。\pagescite[][191]{capital} 

资本不能从流通中产生,又不能不从流通中产生。它必须既在流通中又不在流通中产生。这
样,就得到一个双重的结果。货币转化为资本, 必须根据商品交换的内在规律来加以说明,因
此等价物的交换应该是起点。我们那位还只是资本家幼虫的货币所有者,必须按商品的价值购
买商品,按商品的价值出卖商品, 但他在过程终了时必须取出比他投入的价值更大的价
值。 他变为蝴蝶,必须在流通领域中,又必须不在流通领域中。这就是问题的条件。这里是罗
陀斯,就在这里跳跃罢!\pagescite[][193-194]{capital} 

要从商品的使用上取得价值,我们的货币所有者就必须幸运地在流通领域内即在市场上发现这
样一种商品,\textbf{它的使用价值本身具有成为价值源泉的特殊属性,因此,它的实际使用本身就是劳
动的物化,从而是价值的创造}。货币所有者在市场上找到了这种特殊商品,这就是劳动能力或
劳动力。\pagescite[][194]{capital}(Note: )

\bigskip
货币所有者要在市场上找到作为商品的劳动力,必须存在各种条件。……劳动力只有而且只是
因为被它自己的所有者即有劳动力的人当作商品出售或出卖,才能作为商品出现在市场上。劳
动力所有者要把劳动力当作商品出卖, 他就必须能够支配它, 从而必须是自己的劳动能
力、 自己人身的自由的所有者。

第二个基本条件就是:劳动力所有者没有可能出卖有自己的劳动物化在内的商品,而不得不把
只存在于他的活的身体中的劳动力本身当作商品出卖。

可见,货币所有者要把货币转化为资本,就必须在商品市场上找到自由的工人。这里所说的自
由, 具有双重意义:一方面, 工人是自由人,能够把自己的劳动力当作自己的商品来支配,另一
方面,他没有别的商品可以出卖,自由得一无所有,没有任何实现自己的劳动力所必需的东西。\pagescite[][194-195]{capital} 

\bigskip

有了商品流通和货币流通,决不是就具备了资本存在的历史条件。\textbf{只有当生产资料和
  生活资料的所有者在市场上找到出卖自己劳动力的自由工人的时候,资本才产生}; 而单是
这一历史条件就包含着一部世界史。因此,资本一出现,就标志着社会生产过程的一个新时
代。

生活资料的总和应当足以使劳动者个体能够在正常生活状况下维持自己。由于一个国家的气
候和其他自然特点不同,食物、衣服、取暖、居住等等自然需要也就不同。另一方
面,\textbf{所谓必不可少的需要的范围,和满足这些需要的方式一样,本身是历史的产物,因
  此多半取决于一个国家的文化水平},其中主要取决于自由工人阶级是在什么条件下形成的,从
而它有哪些习惯和生活要求。因此,和其他商品不同,劳动力的价值规定包含着一
个\textbf{历史的和道德的要素}。\pagescite[][199]{capital}

蜘蛛的活动与织工的活动相似,蜜蜂建筑蜂房的本领使人间的许多建筑师感到惭愧。但是,最
蹩脚的建筑师从一开始就比最灵巧的蜜蜂高明的地方, 是他在用蜂蜡建筑蜂房以前, 已经在
自己的头脑中把它建成了。劳动过程结束时得到的结果,在这个过程开始时就已经在劳动者的
表象中存在着,即已经观念地存在着。他不仅使自然物发生形式变化,同时他还在自然物中实
现自己的目的,这个目的是他所知道的,是作为规律决定着他的活动的方式和方法的,他必须使
他的意志服从这个目的。(Big Note)

但是这种服从不是孤立的行为。除了从事劳动的那些器官紧张之外,在整个劳动时间内还需要
有作为注意力表现出来的有目的的意志,而且,劳动的内容及其方式和方法越是不能吸引劳动
者,劳动者越是不能把劳动当作他自己体力和智力的活动来享受,就越需要这种意志。
\pagescite[][208]{capital} (哈维Note: 有些分析家认为,马克思在这里精神分裂了。因
为存在两种马克思主义:一种是这一段的马克思,精神概念对于有意识的和有目的的活动起
了重要作用。另一种是决定论的马克思,他实际上认为我们的意识和同时我们所想的和所做
的,由我们的物质环境决定。哈维不认同这两种观点。) 

马克思对劳动过程的辩证理解,直接说明了人类的观念不可能没有任何出处。……认为观念
来自物质与自然的物质变换的关系,而且一直带有那种起源的烙印的说法是不奇怪的。……
当马克思提出精神的概念、意识、目的性和使命这些观点时,它们与社会演变和自然转变的
动力之间是通过劳动发生联系的,所以这并不是超越常规的。相反,这是基本规律。



劳动资料的使用和创造, 虽然就其萌芽状态来说已为某几种动物所固有, 但是这毕竟是人类
劳动过程独有的特征, 所以富兰克林给人下的定义是“a toolmaking animal” , 制造工具
的动物。\pagescite[][210]{capital} 

关于奴隶和工薪劳动之间关系的长篇幅脚注\pagescite[][229]{capital}……当这两种劳动
体系相互碰撞并变得具有竞争性时,其产生的影响是特别有害的。在市场被整合进资本主义
的竞争性鞭挞下,奴隶制会变得更加残忍,同时,相反地,奴隶制也对工资和工作条件施加
了强大的负面压力。……在一个纯粹的奴隶体系中,不存在抽象劳动,……这一理论只在自
由劳动的情况下才起作用。记住,对于马克思来讲,\textbf{价值不是普遍的概念,而是在
  资本主义生产方式中的工薪劳动的一个特殊概念}。(Big Big Note)


这个领域确实是天赋人权的真正伊甸园。那里占统治地位的只是自由、平等、所有权和边沁
\footnote{杰里米·边沁(Jeremy Bentham,公元1748年2月15日—公元1832年6月6日)是英
国的法理学家、功利主义哲学家、经济学家和社会改革者。有用哲学即功利主义的代表人物
之一。对他来说,个人的利益是一切行动的动力。然而,一切利益,如果正确加以理解,又
处于内在的和谐状态中。各个人的正确理解的利益也就是社会的利益。}。自由!因为商品
例如劳动力的买者和卖者,只取决于自己的自由意志。他们是作为自由的、在法律上平等的
人缔结契约的。契约是他们的意志借以得到共同的法律表现的最后结果。平等!因为他们彼
此只是作为商品占有者发生关系,用等价物交换等价物。所有权!因为每一个人都只支配自
己的东西。边沁!因为双方都只顾自己。使他们连在一起并发生关系的唯一力量,是他们的
利己心,是他们的特殊利益,是他们的私人利益。正因为人人只顾自己,谁也不管别人,所
以大家都是在事物的前定和谐下,或者说,在全能的神的保佑下,完成着互惠互利、共同有
益、全体有利的事业\pagescite[][204]{capital}。。

不变资本是已经凝结在商品中的过去劳动,它在目前的劳动过程中被用作生产手段。生产手
段的价值是既定的。

可变资本,即出让的、用于雇佣的劳动者的价值。

死劳动被活劳动重新唤醒,并被转移到新商品的价值里。


劳动力一天的维持费和劳动力一天的耗费,是两个不同的量。前者决定它的交换价值,后者
构成它的使用价值。……劳动力的使用价值即劳动本身不归他的卖者所
有。\pagescite[][225-226]{capital}

劳动价值论是关于社会必要劳动时间如何通过劳动者被凝结在商品中的理论。这是由货币商
品和一般货币代表的价值的标准。在另一方面,劳动力的价值,只是在市场上作为劳动力被
出卖的商品的价值。(Big Note)\pagescite[][149]{davidcapital1} 

除了这种纯粹身体的界限之外, 工作日的延长还碰到道德界限。工人必须有时间满足精神的
和社会的需要, 这种需要的范围和数量由一般的文化状况决定。 因此, \textbf{工作日是在
  身体界限和社会界限之内变动的。} 但是这两个界限都有极大的伸缩性, 有极大的变动余
地。\pagescite[][269]{capital}



作为资本家,他只是人格化的资本(拜物教)。他的灵魂就是资本的灵魂。而资本只有一种
生活本能,这就是增值自身,创造剩余价值,用自己的不变部分即生产资料吸吮尽可能多的
剩余劳动。资本是死劳动,它像吸血鬼一样,只有吸吮活劳动才有声明,吸吮的活劳动越多,
他的生命就越旺盛\pagescite[][269]{capital}。。

资本家要坚持他作为买者的权利,他尽量延长工作日……可是另一方面,这个已经卖出的商品
的特殊性质给它的买者规定了一个消费的界限,并且工人也要坚持他作为卖者的权利,他要求
把工作日限制在一定的正常量内。于是这里出现了二律背反,权利同权利相对抗,而这两种权
利都同样是商品交换规律所承认的。 \textbf{在平等的权利之间,力量就起决定作用。}所
以,在资本主义生产的历史上,工作日的正常化过程表现为规定工作日界限的斗
争,这是\textbf{全体资本家即资本家阶级和全体工人即工人阶级之间的斗
  争。}\pagescite[][271]{capital}(Big Big Note)

马克思指出,工作日长度的问题,不能通过诉诸权利和交换的规则和合法性来解决。这种问
题只有通过阶级斗争才能解决,即“力量”能够决定“公平权利”之间存在的问题。……许
多政治力量已经投入这样一种观点,即,诉诸个体的人权是塑造一个更人道的资本主义体系
的一种途径。马克思在这里发出的信号是,如果不能重新塑造阶级斗争的含义,那么,在权
力领域产生的许多重要问题就不能得到解决。……在平等的权力之间,不能进行公正的判决。(Big Big Note)\pagescite[][152]{davidcapital1} 

如果在一个社会经济形态中占优势的不是产品的交换价值, 而是产品的使用价值, 剩余劳动
就受到或大或小的需求范围的限制, 而生产本身的性质就不会造成对剩余劳动的无限制的需
求。……不过,那些还在奴隶劳动或徭役劳动等较低级形式上从事生产的民族,一旦卷入资本
主义生产方式所统治的世界市场,而这个市场又使它们的产品的外销成为首要利益,那就会在
奴隶制、农奴制等等野蛮灾祸之上,再加上一层过度劳动的文明灾祸。\pagescite[][272-273]{capital} 

资产阶级,由于一切生产工具的迅速改进,由于交通的极其便利,把一切民族甚至最野蛮的
民族都卷到文明中来了。它的商品的低廉价格,是它用来摧毁一切万里长城、征服野蛮人最
顽强的仇外心理的重炮。它迫使一切民族——如果它们不想灭亡的话——采用资产阶级的生产方
式;它迫使它们在自己那里推行所谓的文明,即变成资产者。一句话,它按照自己的面貌为
自己创造出一个世界。\pagescite[][276]{capital} 


“分钟”和“秒”只是在17世纪后期才开始成为通用的计时单位,直到最近一段时期,像纳
秒这样的概念才被发明出来。时间单位的发明,不仅有自然的,而且有社会的决定因素。\pagescite[][163]{davidcapital1} 

我们看到, 这些按照军队方式一律用钟声来指挥劳动的期间、界限和休息的详尽的规定,决不
是议会设想出来的。它们是作为现代生产方式的自然规律从现存的关系中逐渐发展起来的。
它们的制定、被正式承认以及由国家予以公布,是长期阶级斗争的结果。\pagescite[][326]{capital} 

工厂视察员的出现是有土地贵族促成的,是为了限制无情的产业资本家的权力。

令人遗憾的是,目前的劳动条件变得更加恶劣了。……这是新自由主义的反革命倾向以及部分劳工运动造成的权力的损失所带给我们的结果。悲观地说,马克思的分析与我们当代的情况也是戚戚相关的。

自由竞争使资本主义生产的内在规律作为外在的强制规律对每个资本家起作
用。\pagescite[][312]{capital}


孤立的工人,“自由”出卖劳动力的工人,在资本主义生产的一定成熟阶段上,是无抵抗地
屈服的。因此,正常工作日的确立是资本家阶级和工人阶级之间长期的多少隐蔽的内战的产
物\pagescite[][346]{capital}。

  价格是由平均价格即归根到底是由商品的价值来调节的,我说‘归根到底’,是因为平均
价格并不像亚当·斯密、李嘉图等人认为的那样,直接与商品的价值量相一致。
\pagescite[][194]{capital}

  这一规律同一切以表面现象为根据的经验显然是矛盾的。每个人都知道,就所使用的总
资本两个部分各占的百分比来说,纺纱厂主使用的不变资本较多,可变资本较少,面包房老
板使用的可变资本较多,不变资本较少,但前者获得的利润或剩余价值并不因此就比后者少。
要解决这个表面上的矛盾,还需要许多中项,就像从初等代数的角度来看,要了解0/0可以代
表一个真实的量需要很多中项一样。尽管古典经济学从来没有表述过这一规律,但是它却本
能地坚持这一规律,因为这个规律是价值规律本身的必然结果。古典经济学企图用强制的抽
象法把这个规律从现象的矛盾中拯救出来。以后我们会看到,李嘉图学派是怎样被这块拦路
石绊倒的。“确实什么也没有学到”的庸俗经济学,在这里也像在其他各处一样,抓住了现
象的外表来反对现象的规律。它与斯宾诺莎相反,认为“无知就是充足的论
据”。\pagescite[][355-356]{capital}

相对剩余价值……在历史上,还产生了由国家组织的干预劳动价值的策略。例如,为什么纽
约州不对食品征收消费税?因为那是基本的劳动力价值的决定因素。一种情况是,产业资产
阶级曾经支持控制租金、低廉的(社会的)住房、对租金和农产品的补贴,也是因为要将劳
动力的价值降低。所以我们能够指出很多都是曾经并且现在仍是由阶级策略提出的,通过国
家机器来降低劳动力价值的情况。\pagescite[][183]{davidcapital1} (Big Big Note)

无限制的竞争不是经济规律的真实性的前提,而是结果—— 是经济规律的必然性得到实现的表
现形式。对于象李嘉图那样假定存在着无限制的竞争的那些经济学家们来说,这就是假定资产
阶级生产关系特征的充分现实性和充分实现。因此,竞争不能说明这些规律,它使人们看到这
些规律,但是它并不产生这些规律。\pagescite[][47]{karlvol46b} (Big Note)

工人物质生活水平提高的同时,剥削率也可以提高。……不幸的是,马克思没有强调这一点,
因为它会很容易地预先设置一个错误的、虚假的理论和历史批判的界限。……作为历史和阶级
斗争的一个重要方面,由生产率获得的利益是如何被分享的。……正是新自由主义思想将工人
状况不能得到改进的阶段与凯恩斯主义的福利国家的阶段区别开来,在凯恩斯主义的福利国
家阶段,来自生产率提高的收益倾向于在资本和劳动之间更加均衡地分享。\pagescite[][188-189]{davidcapital1} 

资产阶级意识一方面称颂工场手工业分工,工人终生固定从事某种局部操作,局部工人绝对
服从资本,把这些说成是为提高劳动生产力的劳动组织,同时又同样高声责骂对社会生产过程
的任何有意识的社会监督和调节,把这说成是侵犯资本家个人的不可侵犯的财产权、自由和自
决的“独创性”。\pagescite[][412-413]{capital} 

资本家喜欢在他们的工厂内部进行的生产的有计划的组织,而憎恶任何社会性的关于社会生
产计划的观念。资本家在意识形态上的抱怨正在表现出来,即计划是一件糟糕的事情。特别
是资本家要从根本上攻击它,因为这种社会生产计划将以对他们自己的可怕工厂的印象,来
重塑世界。\pagescite[][203]{davidcapital1} (Big Big Note)

当然,集中计划的主张是不可行的,或是由于其复杂程度,或是因为它对私有产权关系的侵
害没有被清除。不可置信的市场体系的无效性(特别是在环境方面),竞争的强制规律阶段
出现的残酷性,同时还包括典型地产生于工作场所的这种日益增长的专制主义,都不能说是
对市场机制协调功能优越性的良好宣传。\pagescite[][203]{davidcapital1} (Note)

马克思经常被他的朋友,也包括他的敌人描述成一个技术决定论者,即,他认为是生产力的
变化主导了人类历史进程,包括社会关系,精神观念、与自然的关系及其其它方面的演化。
……我并不完全赞同追随解释。马克思在总体上回避了带有因果关系的语言,在这一脚注中,
他没有说技术是“引起”或“决定”,但是他提出,技术“揭示”,或者以另外一种解释,
“揭开”了人与自然的关系。……技术和组织形式在内化了精神观念、社会关系、日常生活
和劳动过程的同时,也内化了人与自然之间一定的关系。凭借这种内化的优势,对技术和组
织形式的研究,就一定会“揭示或解开”关于所有这些要素的许多问题。相反,所有这些其
他的要素也内化了与技术相关的问题。\pagescite[][211-212]{davidcapital1}(Big Big Note)

所有这些要素共同演化,而且都服从于作为整体中的动力时刻的永久更新和转变。但是,它
不是一个黑格尔主义的整体,即,在其中每个时刻都紧密地内化了其他所有时刻。它更像一
个生态的整体,即,在其中每个时刻都紧密地内化了其他所有时刻。它更像一个生态的整体,
即列斐伏尔所说的“整体”或德勒兹所说的“集聚”,在其中时刻是以一种开放、辩证的方
式共同演化的。……这种共同的演化在空间和时间上的不均衡发展,会带来所有情况的地方
的偶发性,虽然这些偶发性受卷入演化过程中的集聚中的要素之间的相互作用,和世界市场
中经济发展过程的日益增长的空间(而且有时是竞争性的)整合的限制。\pagescite[][214-215]{davidcapital1} (Note)


在一种商品上只应耗费生产该商品的社会必要劳动时间,这在商品生产的条件下表现为竞争
的外部强制\pagescite[][400]{capital}。

一切发达的、以商品交换为中介的分工的基础,都是城乡的分离。可以说,社会的全部经济
史,都概括为这种对立的运动。

一定量同时使用的工人,是工场手工业内部分工的物质前提,同样,人口数量和人口密度是
社会内部分工的物质前提,在这里,人口密度代替了工人在同一个工厂内的密集。但是人口
密度是一种相对的东西(蛋蛋注:一句话是说明交通工具的水平可以改变相对密度)。
\pagescite[][408]{capital}。

像其他一切发展劳动生产力的方法一样,机器是要使商品便宜,是要缩短工人为自己花费的
工作日部分,以便延长他无偿地给予资本家的工作日部分。\textbf{机器是生产剩余价值的手段}
\pagescite[][427]{capital}。(Big Note)

大工业必须掌握它特有的生产资料,即机器本身,\textbf{必须用机器来生产机器}。这样,大工业才
建立起与自己相适应的技术基础,才得以自立\pagescite[][441]{capital}。(大卫·哈维:
在机器帮助下的生产机器的能力,是羽翼丰满、动态的资本主义生产方式的技术基础)(Big Note)


只有在机器的价值和它所代替的劳动力的价值(不是有用劳动)之间存在差额的情况下,机
器才会被使用\pagescite[][451]{capital}。

但因为资本是天生的平等派,就是说,它要求把一切生产领域内剥削劳动的条件的平等当作
自己的天赋人权\pagescite[][457]{capital}。

在自动工厂里,代替工场手工业所特有的专业化工人的等级制度的,是机器的助手所要完成
的各种劳动的平等化或均等化的趋势,代替局部工人之间的人为差别的,主要是年龄和性别
的自然差别\pagescite[][457]{capital}。

不是工人使用劳动条件,而是劳动条件使用工人,不过这种颠倒只是随着机器的采用才取得
了在技术上很明显的现实性\pagescite[][487]{capital}。

一旦与大工业相适应的一般生产条件形成起来,这种生产方式就获得一种弹性,一种突然地
跳跃式地扩展的能力,只有原料和销售市场才是它的限制\pagescite[][519]{capital}。

大工业从技术上消灭了那种使一个完整的人终生固定从事某种局部操作的工场手工业分工,
而同时,大工业的资本主义形式又更可怕地再生产了这种分工\pagescite[][557]{capital}。

现代工业从不把某一生产过程的现存形式看成和当作最后的形式。因此,现代工业的技术基
础是革命的,而所有以往的生产方式的技术基础本质上是保守的
\pagescite[][560]{capital}。

资本主义生产方式同时为一种新的更高级的综合,即农业和工业在它们对立发展的形态的基
础上的联合,创造了物质前提\pagescite[][579]{capital}。

然而实际上正好相反,劳动生产率和劳动强度的变化,或者是在工作日缩短以前,或者是紧
接着在工作日缩短以后发生的\pagescite[][601]{capital}。

资本主义生产方式迫使每一个企业实行节约,但是它的无政府状态的竞争制度却造成社会生
产资料和劳动力的最大的浪费,而且也产生了无数现在是必不可少的、但就其本身来说是多
余的智能\pagescite[][605]{capital}。(Note)

在商品市场上同货币所有者直接对立的不是劳动,而是工人。工人出卖的是他的劳动力。当
工人的劳动实际上开始了的时候,它就不再属于工人了,因而也就不再能被工人出卖了。劳动
是价值的实体和内在尺度,但是它本身没有价值。

在 “劳动的价值” 这个用语中,价值概念不但完全消失,而且转化为它的反面。 这是一个
虚幻的用语,就象说土地的价值一样。\pagescite[][615-616]{capital} 


可变资本不过是工人为维持和再生产自己所必须的生活资料基金或劳动基金的一种特殊的历
史的表现形式;这种基金在一切社会生产制度下都始终必须由劳动者本身来生产和再生产
\pagescite[][655]{capital}。

一旦劳动力由工人自己作为商品自由出卖,这种结果就是不可避免的。但只有从这时起,商
品生产才普遍化,才成为典型的生产形势;只有从这时起,商品生产才普遍化,才成为典型
的生产形式;只有从这时起,每一个产品才一开始就是为卖而生产,而生产出来的一切财富
都要经过流通。只有当雇佣劳动成为商品生产的基础时,商品生产才强加于整个社会;但也
只有这时,它才能发挥自己的全部潜力\pagescite[][677]{capital}。

美国……经济发展中百分之七十的驱动力是依靠\textbf{负债刺激}的消费主义。……表明
了分配条件(金融、信用、利息)实际上可能在资本主义的动态变化中起到了核心作用,而
不是辅助性的作用。……新自由主义的特点是去工业化、持续的结构性失业、螺旋性的工作
不稳定和高度的社会不平等。\pagescite[][267]{davidcapital1} 




在生产的巨流中,全部原预付资本,与直接积累的资本即重新转换为资本的剩余价值或剩余
产品比较起来,总是一个近于消失的量。……全部现存的资本都是积累起来的或资本化的
(全部原预付资本的)利息,因为利息不过是剩余价值的一部分
\pagescite[][678]{capital}。

它们越是整个地被使用而只是部分地被消费,那么,它们就越是像我们在上面说过的自然力
如水、蒸汽、空气、电力等等那样,提供无偿的服务。被活劳动抓住并赋予生命的过去劳动
的这种无偿服务,会随着积累规模的扩大而积累起来\pagescite[][702]{capital}。

资本的积累就是无产阶级的增加……

应当使工人免于挨饿,但不应当使他们拥有任何可供储蓄的东西。……因为对一切富裕民族
有利的是:绝大部分穷人永远不要无事可做,但要经常花光它们所收入的一切……要使社会
(当然是非劳动者的社会)幸福,使人民自己满足于可怜的处境,就必须使大多数人既无知
又贫困。知识会使我们产生更大和更多的愿望,而人的愿望越少,他的需要也就越容易满足
\pagescite[][709]{capital}。

如果工人阶级提供的并由资本家阶级所积累的无酬劳动量增长的十分迅速,以致只有大大追
加有酬劳动才能转化为资本,那么,工资就会提高。而在一切情况不变时,无酬劳动就会相
应地减少。但是,一旦这种减少达到这样一点,即滋养资本的剩余劳动不再有正常数量的供
应时,反作用就会发生:收入中资本化的部分减少,积累削弱,工资的上升运动受到反击。
可见,劳动价格的提高被限制在这样的界限内,这个界限不仅使资本主义制度的基础不受侵
犯,而且还保证资本主义制度的规模扩大的再生产。可见,被神秘化为一种自然规律的资本
主义积累规律,实际上不过表示:资本主义积累的本性,决不允许劳动剥削程度的任何降低
或劳动价格的任何提高有可能严重地危及资本关系的不断再生产和它的规模不断扩大的再生
产\pagescite[][716]{capital}。

不管是条件还是结果,只要生产资料的量比并入生产资料的劳动力相对增长,这就表示劳动
生产率的增长。因而,劳动生产率的增长,表现为劳动的量比它所推动的生产资料的量相对
减少,或者说,表现为劳动过程的主观因素的量比它的客观因素的量相对减少
\pagescite[][718]{capital}。

积累一方面表现为生产资料和对劳动的支配权的不断增长的积聚,另一方面,表现为许多单
个资本的互相排斥。

社会总资本这样分散为许多单个资本,或它的各部分间的互相排斥,又遇到各部分间的互相
吸引的反作用。这已不再是生产资料和对劳动的支配权的简单的、和积累等同的积聚(单个
资本的表现)。这是已经形成的各资本的积聚,是它们的个体独立性的消灭,是资本家剥夺
资本家,是许多小资本转化为少数大资本。

竞争斗争是通过使商品便宜来进行的。在其他条件不变时,商品的便宜取决于劳动生产率,
而劳动生产率又取决于生产规模。因此,较大的资本战胜较小的资本。其次,我们记得,随
着资本主义生产方式的发展,在正常条件下经营某种行业所需要的单个资本的最低限量提高
了。因此,较小的资本挤到那些大工业还只是零散地或不完全地占领的生产领域中去。在那
里,竞争的激烈程度同互相竞争的资本的多少成正比,同互相竞争的资本的大小成反
比。……信用事业,随同资本主义的生产而形成起来……很快它就成了竞争斗争中的一个新
的可怕的武器,最后,它转化为一个实现资本集中的庞大的社会机构。

随着资本主义生产和积累的发展,竞争和信用——集中的两个最强有力的杠杆,也以同样的速
度发展起来\pagescite[][722,723]{capital} 。

资本积累最初只是表现为资本的量的扩大,但是以上我们看到,它是通过资本构成不断发生
质的变化,通过减少资本的可变组成部分来不断增加资本的不变组成部分而实现的
\pagescite[][722,723]{capital} 。

资本主义积累不断地并且同它的能力和规模成比例地生产出相对的,即超过资本增殖的平均
需要的,因而是过剩的或追加的工人人口\pagescite[][726]{capital} 。

工人人口本身在生产出资本积累的同时,也以日益扩大的规模生产出使他们自身成为相对过
剩人口的手段。……事实上,每一种特殊的、历史的生产方式都有其特殊的、历史地发生作
用的人口规律。抽象的人口规律只存在于历史上还没有受过人干涉的动植物界。

过剩的工人人口是积累或资本主义基础上的财富发展的必然产物,但是这种过剩人口反过来
又成为资本主义积累的杠杆,甚至成为资本主义生产方式存在的一个条件。过剩的工人人口
形成一支可供支配的产业后备军,他绝对地从属于资本,就好像它是由资本出钱养大的一样
\pagescite[][728]{capital} 。(Big Note)



每一个资本家的绝对利益在于,从较少的工人身上而不是用同样低廉或甚至更为低廉的花费
从较多的工人身上榨取一定量的劳动。在后一种情况下,不变资本的支出会随着所推动的劳
动量成比例地增大,在前一种情况下,不变资本的增长则要慢得多。生产规模越大,这种动
机就越具有决定意义。他的力量随资本积累一同增长\pagescite[][732]{capital}。
\bigskip

法国1850年9月5日的十二小时工作日法令是临时政府1848年3月2日法令的资产阶级化的翻版;
这个法令适用于一切作坊\pagescite[][319]{capital}。

“至于个人受教育的时间,发展智力的时间,履行社会职能的时间,进行社交活动的时间,
自由运用体力和智力的时间,以至于星期日的休息时间,——这全都是废话!但是,资本由于
无限度地盲目追逐剩余劳动,像狼一般地贪求剩余劳动,不仅突破了工作日的道德极限,而
且突破了工作日的纯粹身体的极限。”\pagescite[][306]{capital}

《1861年爱尔兰面包业委员会的报告》中提到,“委员会认为,把工作日延长到12小时以上,
是横暴地侵犯工人的家庭生活和私人生活,这就侵犯一个男人的家庭,使他不能履行他作为
一个儿子、兄弟、丈夫和父亲所应尽的家庭义务,以致造成道德上的非常不幸的后果。12小
时以上的劳动会损害工人的健康,使他们早衰早死,因而造成工人家庭的不幸,恰好在最必
要的时候,失去家长的照料和扶持。”\pagescite[][292]{capital}

工人阶级中就业部分的过度劳动,扩大了它的后备军的队伍,而后者通过竞争加在就业工人
身上的增大的压力,又反过来迫使就业工人不得不从事过度劳动和听从资本的摆布。工人阶
级的一部分从事过度劳动迫使它的另一部分无事可做,反过来,它的一部分无事可做迫使他
的另一部分从事过度劳动,这成了各个资本家致富的手段,同时又按照与社会积累的增进相
适应的规模加速了产业后备军的生产。

决定工资的一般变动的,不是工人人口绝对数量的变动,而是工人阶级分为现役军和后备军
的比例的变动,是过剩人口相对量的增减,是过剩人口时而被吸收、时而又被游离的程度
\pagescite[][733]{capital}。

(马克思驳斥教条的经济学)可是,在真正有劳动能力的人口因工资提高而可能出现某种实
际增长以前,已经一再经过了这样一个时期,在这个时期必然发生工业战,展开厮杀,并且
决出胜负(马克思驳斥教条的经济学)\pagescite[][733]{capital}。

产业后备军在停滞和中等繁荣时期加压力于现役劳动军,在生产过剩和亢进时期又抑制现役
劳动军的要求。所以,相对过剩人口是劳动供求规律借以运动的背景。它把这个规律的作用
范围限制在绝对符合资本的剥削欲和统治欲的界限之内\pagescite[][736]{capital}。
\bigskip


工人数量的自然增长不能满足资本积累的需要,但同时又超过这种需要,这是资本运动本身
的一种矛盾(少年、成年、中年、老年工人)。

这种社会需要,是通过早婚这一大工业工人生活条件的必然后果,并通过剥削工人子女以奖
励工人生育子女的办法来满足的。

资本主义生产一旦占领农业,或者依照它占领农业的程度,对农业工人人口的需求就随着在
农业中执行职能的资本的积累而绝对地减少,而且对人口的这种排斥不像在非农业的产业中
那样,会由于更大规模的吸引而得到补偿。因此,一部分农村人口经常准备着转入城市无产
阶级或制造业无产阶级的队伍\pagescite[][739]{capital}。

社会的财富即执行职能的资本越大,它的增长的规模和能力越大,从而无产阶级的绝对数量
和他们的劳动生产力越大,产业后备军也就越大。可供支配的劳动力同资本的膨胀力一样,
是由同一些原因发展起来的。……这种后备军越大,常备的过剩人口也就越多,他们的贫困
同他们所受的劳动折磨成反比。最后,工人阶级中贫苦阶层和产业后咱平台大,官方认为需
要救济的贫民也就越多。这就是\textbf{资本主义积累的绝对的、一般的规律}。像其他一
切规律一样,这个规律的实现也会由于各种各样的情况而有所变化,不过对这些情况的分析
不属于这里的范围\pagescite[][742]{capital}。

那时人们被迫从事劳动,因为他们是别人的奴隶;而现在,人们被迫从事劳动,因为他们是
自己需求的奴隶\pagescite[][745]{capital}。

英国经济学家艾伦·伯德(Alan Budd),记录了他作为玛格丽特·撒切尔政府首席经济顾问
期间的经历,他承认在后期与邻里相处时感到非常羞愧,因为“20世纪80年代,通过压缩经
济规模和公共指出来治理通货膨胀的政策,只是对打击工人力量的一种掩盖。提高失业率是
政府非常希望的削弱工人阶级力量的措施。当时所构建的,按照马克思的说法,是一种资本
主义的危机,这一危机再次造就了劳动后背大军,并保证资本家获得前所未有的利润”。与
李根一样……其目的还是为了加强对工人的管制,以保证资本家获得利润和无边无尽的资本
积累。\pagescite[][308]{capital} 
(Big Big Note)

新自由主义的规划,是以日益增长的财富积累和资本家阶级中的上流集团对剩余价值越来越
多的占有为方向的……降低工资水平,并通过技术变化来取代工人,从而造成失业,集中资
本力量,打击工人组织,同时干扰市场中的供求关系,通过外包和进出口将全世界所有潜在
形式的过剩人口都调动起来,并尽可能地压缩福利水平。这就是新自由主义“全球化”的真
正含义。社会必要条件已经具备,这在很大程度上与马克思在第一卷中的分析是一致的,即,
以其他所有人的利益为代价的、在一端的、巨额财富的积累。当然,问题是,这种新自由主
义的资本主义,只有“它同时破坏了一切财富的源泉——土地和工人”才能生存。……在自由
市场的乌托邦条件下,日益增长的资本积累和集中将是不可避免的。(能源、制药、媒体行
业,特别是金融行业日益增长的集中趋势)……朝着更严重的寡头垄断、甚至是垄断的方向
发展。\pagescite[][309]{davidcapital1} 




所谓原始积累只不过是生产者和生产资料分离的历史过程。这个过程所以表现为“原始的”,因
为它形成资本及与之相适应的生产方式的前史\pagescite[][745]{capital}。

掠夺教会地产,欺骗性地出让国有土地,盗窃公有地,用剥夺方法、用残暴的恐怖手段把封
建财产和克兰财产转化为现代私有财产——这就是原始积累的各种田园诗式的方法。这些方法
为资本主义农业夺得了底盘,使土地与资本合并,为城市工业造成了不受法律保护的无产阶
级的必要供给\pagescite[][842]{capital}。

只有大工业才用机器为资本主义农业提供了牢固的基础,彻底地剥夺了极大多数农村居民,
使农业和农村家庭手工业完全分离,铲除了农村家庭手工业的根基——纺纱和织布
\pagescite[][858]{capital}。

公债成了原始积累的最强有力的手段之一。它像挥动魔杖一样,使不生产的货币具有了生殖
力,这样就使它转化为资本。

用国家的名义装饰起来的大银行,从一产生起就只不过是私人投机家的公司,它们支持政府,
依靠取得的特权能够把货币贷给政府。因此,国债积累的最准确的尺度就是这些银行的股票
的不断涨价。

随着国债的产生,国际信用制度出现了。国际信用制度常常隐藏着这个或那个国家原始积累
的源泉之一。(劫掠!)

因为国债是依靠国家收入来支付年利息等等开支,所以现代税收制度就成为国债制度的必要
补充。借债使政府可以应付额外的开支,而纳税人又不会立即有所感觉,但借债最终还是要
求提高税收。另一方面,由于债务一笔接着一笔的积累而引起的增税,又迫使政府在遇到新
的额外开支时,总是要借新债。因此,以对最必要的生活资料的课税(因而也是以它们的昂
贵)为轴心的现代财政制度,本身就包含着税收自行增加的萌芽
\pagescite[][865]{capital}。

一定要看罗莎·卢森堡的《资本积累论》!!!!特别是国际借款等章节!!!!!
资本主义的长期历史是以这种动态关系为中心的,即持续的原始积累和贯穿于《资本论》所
描述的扩大再生产体系的积累动力之间的动态关系。她提出,马克思将原始积累限制在某个
上古的时点、某段资本主义的史前时期是错误的。如果它不参与新一轮的、主要是通过帝国
主义的暴力手段的原始积累,那么,资本主义应该在很早以前就消失了。

通过将农业生产者从土地上驱赶出来建设“经济特区”,是资本主义发展可持续性的一个必
要的先期行为,这就像对所谓的城市居民贫民窟的清理对开发商的资本扩张及其在城市的经
营行为是必须的一样。这种通过国家的主导力量或一些相应地合法措施侵占土地的做法,在
近期已经成为一种被广泛传播的现象。\pagescite[][334]{davidcapital1} (Big Big Note)

信用体系可以保证将巨大的货币力量迅速集中到一起。在股份公司和其他公司组织形式中,
巨额货币的力量就被聚集在少数指挥者和管理者的控制之下。兼容和并购一直是一种大型商
业行为,这种商业行为能够引发新一轮的通过剥夺而进行的积累。同时还存在各种各样的大
企业吞并小企业的伎俩。为大企业的发展让路而进行的对小经营者的剥夺通常会得到信用机
制的帮助,这已经是一个长期惯例了所以,关于可以用于投资的货币资本的组织,整合和集
中的问题从来都没有消失。\pagescite[][339]{davidcapital1} 

马克思所展示的是,均衡远不能自动达到,如果资本总是流向利润率最高的地方,那么,螺
旋式出现的比例失调会严重地干扰资本主义的再生产。

有越多的人持有货币,经济体系的情况就越糟糕。这就是凯恩斯所称的流动性陷阱。人们必
须找出将处于隐藏状态的货币引诱出来的方法,答案之一是通过负债的政府支出以重振资本
循环(另一个解决方法是发动战争)。另一方面,我认为,Andrew Glyn和其他人看到了发
达资本主义国家在20世纪60年代后期的重重困难中存在一个强大的利润挤压因素是基本正确
的,在哪里,劳动的稀缺和强大的劳动组织显然对积累的进程起到了阻碍作用。过度的垄断
同时有助于减缓生产率的增长,而且,这与国家的财政危机一起酿成了一个长期的滞胀阶段,
而且这种滞涨局面只能通过加强对劳动的管制和放松竞争的强制规律加以解决。在这种情况
下,危机就会从一个障碍节点传递到另一个障碍节点,然后再重新返回。
\pagescite[][361]{davidcapital1} (Big Big Note)

\end{quotation}


%%% Local Variables:
%%% mode: latex
%%% TeX-master: "../main"
%%% End:
