\chapter{商人资本(第三卷 第16—20章)}

从马克思的许多论证中重现总体思路,即便不是不可能的,也是很困难的。有些地方可以做
到,但在其他地方,我觉得最好还是从冗长的文本中摘录出似乎是核心观点和想法的内容,
来看看是否能提炼出更一般的分析框架——例如,除了投机性地阅读关于投机的材料外别无选
择。

“商人资本或商业资本分为两个形式或亚种,即商品经营资本和货币经营资
本。”\pagescite[][297]{capital3}

马克思也清晰地表达了不会偏离他集中于资本运动一般规律的任务(正如他在《〈政治经济
学批判大纲〉导言》里描述的)的决心,甚至当他处理分配的特殊性的时候也是如此。……
问题是,马克思并不总是肯定什么和一般规律有关,什么和一般规律无关。我们阅读的时候,
要对此保持一种批判的眼光。

\begin{quotation}既然像读者已经感到遗憾地认识到的那样,对资本主义生产过程的现实
的内部联系的分析,是一件极其复杂的事情,是一项极其细致的工作;既然把看得见的、只
是表面的运动归结为内部的现实的运动是一种科学工作。那么,不言而喻,在资本主义生产
当事人和流通当事人的头脑中,关于生产规律形成的观念,必然会完全偏离这些规律,必然
只是表面运动在意识中的表现。商人、交易所投机者、银行家的观念,必然是完全颠倒的。
工厂主的观念由于他们的资本所经历的流通行为,由于一般利润率的平均化而被歪曲了。在
这些人的头脑中,竞争也必然起一种完全颠倒的作用。\pagescite[][348]{capital3}
\end{quotation}

在某种意义上,银行家和金融家是最不应该相信的人,不是因为他们都是诈骗犯、说谎者
(尽管他们中的一些人很明显是),而是因为他们可能已经沦为自己的故弄玄虚和拜物教观
念的牺牲品。劳埃德·布兰克费恩在国会面前宣称,他的银行——高盛——只不过是做了上帝该
做的工作而已。如果有机会对他加以评论的话,不难想象马克思会怎样讽刺他。

这几章正是要尝试纯粹从“资本主义生产方式的角度,并且在资本主义生产方式的界限
内”\pagescite[][362]{capital3}解决这个问题。

“在资本主义生产充分发展时,即在产品只是作为商品,而不是作为直接的生存资料来生
产”时,商业的规模扩大了,达到“自己的最大限度”。商人资本“只是对商品交换起中介
作用”,但是“商人为许多人而进行买卖。买和卖都集中在他手中;因此,买和卖就不再与
购买者(作为商人)的直接需要联系在一起了”

\begin{quotation}
  不论以商人为中介进行商品交换的各生产部门的社会组织如何,商人的财产总是作为货币
  财产而存在,他的货币也总是作为资本执行职能。这个资本的形式总是G-W-G`;货币,交
  换价值的独立形式,是出发点,而增加交换价值是独立的目
  的。\pagescite[][363]{capital3}

  要理解商人资本为什么在资本支配生产本身以前很久就表现为资本的历史形式,这丝毫也
  不困难。商人资本的存在和发展到一定的水平,本身就是资本主义生产方式发展的历史前
  提。1. 因为这种存在和发展是货币财产集中的先决条件;2. 因为资本主义生产方式的前
  提是为贸易而生产,是大规模的销售……另一方面,商人资本的一切发展都会促使生产越
  来越具有面向交换价值的性质,促使产品越来越转化为商品……商人资本的发展就它本身
  来说,还不足以促成和说明一个生产方式到另一个生产方式的过
  渡。\pagescite[][364]{capital3}
\end{quotation}

“资本作为商人资本而实现的独立的、优先的发展,意味着生产还没有从属于资本”,所以
“商人资本的独立发展,是与社会的一般经济发展成反比例的”。[16]换句话说,处于霸权
地位的商人阶级将试图压制产业资本的发展,因为他们剥削弱小的、易受伤害的生产者榨取
超额利润的能力将受到这种发展的抑制。

“纯粹商业民族的优势的衰落和这些民族的以这种转运贸易为基础的商业财富的衰落”,反
映了“商业资本在资本主义生产的发展进程中从属于产业资本”。(笔者注:资本主义之前
社会)

马克思评价了威尼斯人、热那亚人、荷兰人经营的转运贸易的实质,以此阐述了这种规律的
力量。这些地方都严重依赖“纯粹的商人资本”,而且通过作为汇兑的中介和积累货币资本,
以及通过贱买贵卖,有效地积累起了他们的财富。尽管所交换的商品是人类劳动的表现,因
而具有价值,但“它们不是相等的价值量”。但是正如他之前所说的,商人越是把商品交换
的世界转变成一个交换成为“正常的社会行为”的世界,那么价值度量标准就越来越处于支
配地位。(Big Note)

\begin{quotation}
  在资本主义社会以前的各阶段中,商业支配着产业;在现代社会里,情况正好相反。当然,
  商业对于那些互相进行贸易的共同体来说,会或多或少地发生反作用;它会使生产越来越
  从属于交换价值……它由此使旧的关系解体。它增进了货币流通。它已经不再是仅仅掌握
  生产的余额,而且逐渐地侵蚀生产本身,使整个整个的生产部门依附于它。[20]
\end{quotation}

一开始,商人资本从“侵占和欺诈”中获得它的大部分财富。当它占“主要统治地位”时,
它“到处都代表着一种掠夺制度”。[21]我们应该注意,这种掠夺制度完全违背了《资本论》
中通常假定的自由公平的市场交换规则,并且让我们回到了第一卷中描述的原始积累的世界。
但是它越来越规范,所以越来越受到规则的限制。而且至少在理论上,这些规则是根据资本
主义生产方式的需要设定的,商业的发展满足了这些需要。

\begin{quotation}商业和商业资本的发展,到处都使生产朝着交换价值的方向发展,使生
产的规模扩大,使它多样化和世界化,使货币发展成为世界货币。因此商业对各种已有的、
以不同形式主要生产使用价值的生产组织,到处都或多或少地起着解体的作用。但是它对旧
生产方式究竟在多大程度上起着解体作用,这首先取决于这些生产方式的坚固性和内部结构。
并且,这个解体过程会导向何处,换句话说,\textbf{什么样的新生产方式会代替旧生产方
式,这不取决于商业,而是取决于旧生产方式本身的性质}。
\pagescite[][370]{capital3}(Note)
\end{quotation}

因此,并不一定会走向资本主义生产方式。在《政治经济学批判大纲》中,马克思以极大的
热情,用相当长的篇幅详细地阐述了货币如何“瓦解”了古代的共同体并产生了一个新的社
会。然而,

\begin{quotation}毫无疑问……在16世纪和17世纪,由于地理上的发现而在商业上发生的
并迅速促进了商人\textbf{资本}发展的大革命,是促使封建生产方式向资本主义生产方式
过渡的一个主要因素。世界市场的突然扩大,流通商品种类的倍增,欧洲各国竭力想占有亚
洲产品和美洲宝藏的竞争热,殖民制度——所有这一切对打破生产的封建束缚起了重大的作用。
[23]
\end{quotation}为资本主义生产提供动力的不再是商业和世界市场的扩张,驱动力转换为
资本主义生产,使得工业化国家(英国)在资本主义发展中取代商业强权(荷兰)夺得霸主
地位。……正是资本主义生产驱使着商人成为殖民主义和帝国主义的先锋,摧毁了印度的手
工业,为英国生产的商品创造了市场。

“从封建生产方式开始的过渡有两条途径。生产者变成商人和资本家”,这是“真正革命化
的道路。或者是商人直接支配生产”。[25]马克思后来增加了第三种途径,“商人把小老板
变成自己的中介人,或者也直接向独立生产者购买;他在名义上使这种生产者独立,并且使
他的生产方式保持不变”。

这里有两点富有洞见。第一,商人资本压倒性的力量和它的组织形式,经常既促进又抑制了
成熟的产业资本主义的发展。有相当多的历史证据支持这个观点。但是第二个更现代的观点
是,当商人保持着控制权时,他们经常保存和保留按照传统方式组织的古老的、落后的生产
形式。这种做法
\begin{quotation}到处都成了真正的资本主义生产方式的障碍,它随着资本主义生产方式
的发展而消灭。它不变革生产方式,只是使直接生产者的状况恶化,把他们变成单纯的雇佣
工人和无产者,使他们所处的条件比那些直接受资本统治的人所处的条件还要坏,并且在旧
生产方式的基础上占有他们的剩余劳动。同样的情况在伦敦一部分手工家具制造业中也可以
看到,不过略有变化。这种制造业,特别在陶尔哈姆莱茨区,经营的规模非常大。
\pagescite[][373]{capital3}

\end{quotation}

这种生产体制在资本主义历史中长期存在,在过去的四十年里急剧扩散了(尽管拥有了现代
的外观)。贝纳通、沃尔玛、宜家、耐克等商业资本家组织,几乎必定把来自于他们转包的
生产者的“剩余价值的最大部分装进了自己的腰包”。那么,在何种意义上我们仍然可以说
“生产占主导地位”呢?(Note)

我发现陶尔哈姆莱茨区的组织类型在增加而不是缩小(笔者注:陶尔哈姆莱茨区的生产,按
马克思说法来说,是向合并组织成大工厂方向发展)。不过,马克思正确的地方在于他评论
了这种劳动组织形式下的残酷剥削。左拉在他的小说《小酒店》中,对一对夫妻的悲惨生活
有着入木三分的描述,他们在公寓里制作金链,商人按月提供金子并收回产品。在当代世界,
有大量的证据表明,过度剥削是许多由商业资本动员和组织起来的分包网络的特点(因此丑
闻不断出现在主流媒体上,包括丽诗加邦的服装、耐克的鞋、童工制作的毯子和足球——踢这
些球的球员可以赚到数百万——以及可可豆的收割)。(Note)

不同劳动体制间的竞争仍然是当代全球资本主义的一个至关重要的方面,这反过来意味着生
产者和商人相对作用的差别。全球经济中存在特定的部门和特定的场所,在那里生产者似乎
的确支配着商人;然而在其他地方和部门,情况则恰好相反。例如,在汽车工业,生产者倾
向于支配着分销商;但是现在纺织品行业几乎总是完全相反。然而,在通用汽车的案例中,
一种叫通用汽车金融服务公司的混合形式出现了,这个公司成为通用汽车组织汽车信贷的一
个独立、自主的分支(最终在2008—2009年经济危机中取得了银行资格)。

\begin{quotation}既然它有助于流通时间的缩短,它就能间接地有助于产业资本家所生产
的剩余价值的增加。既然它有助于市场的扩大,并对资本之间的分工起中介作用,因而使资
本能够按更大的规模来经营,它的职能也就提高产业资本的生产效率和促进产业资本的积累。
既然它缩短流通时间,它也就提高剩余价值对预付资本的比率,也就是提高利润率。既然它
把资本的一个较小部分作为货币资本束缚在流通领域中,它就增大了直接用于生产的那部分
资本。\pagescite[][312]{capital3}(Note: 需批判接受。)

商品经营资本——撇开可以和它结合在一起的一切异质的职能,如保管、发送、运输、分类、
散装等,只说它的真正的为卖而买的职能——既不创造价值,也不创造剩余价值,它只是对它
们的实现起中介作用,因而同时也对商品的实际交换,对商品从一个人手里到另一个人手里
的转让,对社会的物质变换起中介作用。\pagescite[][314]{capital3}

\end{quotation}

正如我们一直以来看到的,资本必须保持运动的\textbf{连续性、平稳性和流动性}。商品
经营资本在这方面发挥了至关重要的作用。

马克思在这儿借助了第三卷第二篇中详细考察的利润率平均化原理。既然我们还没有考虑这
个原理,我在这里简要说明一下它的含义。马克思说,资本倾向于流向利润率最高的地方
(尤其是在竞争条件下)。从直觉上看,这是讲得通的。结果所有部门的利润率都趋于均
等——从纺织业到农业再到石油开采。问题是这种趋势不会引导资本流向生产剩余价值最多的
地区。资本密集型部门(资本的价值构成或者有机构成高的部门)从劳动密集型(资本的价
值构成或者有机构成低的部门)部门获取剩余价值。这种与价值和剩余价值相关的投资错配,
产生了各种各样的复杂后果(包括利润率下降的趋势,因为在既定的市场力量下,资本家对
利润率而不是剩余价值生产做出反应)。后面的章节偶尔提到了这个趋势的效果。然而在这
儿,马克思仅仅断言商人资本的利润率与产业资本的利润率趋于相等。他后来在别的地方说,
如果一般利润率趋向于下降,那么商人资本的收益率必定也会下降。货币资本的利息率和产
业资本的利润率是否趋于相等呢?我们将在随后的章节进行讨论。(Note)

“商人的出售价格之所以高于购买价格,并不是因为出售价格高于总价值,而是因为购买价
格低于总价值”。\pagescite[][319]{capital3}换句话说,“在平均利润率中,总利润中
归商业资本所有的部分已经计算在内了”。\pagescite[][318]{capital3}

“商人资本虽然不参加剩余价值的生产,但参加剩余价值到平均利润的平均化。因此,一般
利润率已经意味着从剩余价值中扣除了属于商人资本的部分,也就是说,对产业资本的利润
作了一种扣除。”[48]结果,“同产业资本相比,商人资本越大,产业利润率就越小。反过
来,情况也就相反”。\pagescite[][319]{capital3}(Note)

一旦承认产业资本利润和商人资本利润之间的关系在某种意义上是因情况而异的,那么不均
衡的力量关系便存在各种各样的可能性,从而扭曲和干扰马克思假定的通过利润率平均化而
实现的均衡。这也意味着对产业资本和商人资本之和的投资得到的利润率,低于只对产业资
本投资时得到的利润率(后一种计算方法在第三卷前面的章节中用过)。

\begin{quotation}因为商人资本决不是别的东西,而只是一部分在流通过程中执行职能的
产业资本的独立化的形式……问题首先要在这样的形式上提出,即商人资本所特有的各种现
象还没有独立地表现出来,而是还和产业资本直接联系在一起,作为产业资本的一个分支表
现出来。\pagescite[][333]{capital3}

  对单个商人来说,他的利润量取决于他能够用在这个过程中的资本量,而他的店员的无酬
劳动越大,他能够用在买卖上的资本量就越多……这些店员的无酬劳动,虽然不创造剩余价
值,但能使他占有剩余价值;这对这个资本来说,就结果而言是完全一样的。因此,这种劳
动对商业资本来说是利润的源泉。否则,商业就不可能大规模地经营,就不可能按资本主义
的方式经营了。正如工人的无酬劳动为生产资本直接创造剩余价值一样,商业雇佣工人的无
酬劳动,也为商业资本在那个剩余价值中创造出一个份额。\pagescite[][327]{capital3}
\end{quotation}

但是有一个遗留的难点:如何解释商人资本为购买劳动力而花费的可变资本呢?即使它不生
产剩余价值,也应该把它包括在总资本所使用的总可变资本中吗?它是生产性劳动还是非生
产性劳动呢?马克思承认对于这个主题还有许多东西需要研究,并且以他一贯的一丝不苟的
态度试着加以解释,在这儿我不打算重复了。他尝试性的结论是

\begin{quotation}商人用[他的可变资本]购买的,按照假定,只是商业劳动,即只是对
资本的流通职能即对G—W和W—G起中介作用所必要的劳动。但商业劳动是使一个资本作为商人
资本执行职能、对从商品到货币和从货币到商品的转化起中介作用所必要的劳动。这种劳动
实现价值,但不创造价值。并且,只是由于一个资本执行了这些职能……这个资本才作为商
人资本执行职能,才参加一般利润率的规定,也就是说,才从总利润中取得它的份额。
\pagescite[][332]{capital3}
\end{quotation}

(哈维注:结果,商业工人的工资)并不与他帮助资本家实现的利润量保持任何必要的比例。
资本家为他支出的费用,和他带给资本家的利益,是不同的量。他给资本家带来利益,不是
因为他直接创造了剩余价值,而是因为他在完成劳动——部分是无酬劳动——的时候,帮助资本
家减少了实现剩余价值的费用。\pagescite[][335]{capital3} (Big Note: 机器、商业资
本在马克思看来都是不创造剩余价值的,但是却对实现剩余价值、减少剩余价值费用起中介
作用。这种说法是否具有矛盾?)

对于商业工人阶级后来的遭遇和它现在的处境,显然需要更细致地研究。从马克思的时代以
来,现实条件明显改变了。

\begin{quotation}当然,商人资本对生产资本的周转起中介作用,但这只是就它缩短生产
资本的流通时间来说的。它不会直接影响生产时间,而生产时间也是对产业资本周转时间的
一个限制。这对商人资本的周转来说是第一个界限,第二……商人资本的周转\textbf{最终
要受全部个人消费的速度和规模的限制}。\pagescite[][338]{capital3}
\end{quotation}

最后提示的这一点在后面基本被忽略了,大概是因为消费对马克思来说属于政治经济学范围
外的“\textbf{个别性}”,正如他在《政治经济学批判大纲》里论证的那样(我想不到其
他原因了)。但是,历史上,商人资本对消费的作用恰恰是:\textbf{刺激消费者的欲望,
用产业资本家提供的商品挑逗公众,尽可能地确保潜在的消费者拥有足够的可支配货币(通
常是信用)来迅速吸收产品,并使消费动力的增长和产业资本寻求的无止境的积累同步进行。
但是马克思将消费的周转时间所造成的限制描述为“决定性的”,我很惊讶他没有进一步讨
论这个观点}。(Big Big Note)

利润率平均化,然而利润率平均化对产业资本和商业资本的不同周转时间很敏感。商业资本
的周转能够同时或者连续地“对极不相同的生产资本的周转起中介作用”。
\pagescite[][340]{capital3} 另一方面,产业资本的周转,是由生产和再生产的周期性设
置的,这里流通时间也“形成一个界限……它对价值和剩余价值的形成或多或少起着限制的
作用,因为它对生产过程的规模发生着影响……从而参加决定一般利润率的形成”。
\pagescite[][344]{capital3}通过减少流通时间来减少产业资本的周转时间可以提高利润
率。不管周转时间是多久,商业资本(理论上)能获得一般利润率。那么,尽管商业资本不
能通过加速它的周转时间来提高自己的利润率,却可以影响一般利润率,因为价值和剩余价
值的实现需要的商业资本减少了。“商人资本的绝对量和它的周转速度成反比。”而且,
“各种会缩短商人资本平均周转的情况,例如,运输工具的发展,都会相应地减少商人资本
的绝对量,从而会提高一般利润率”。\pagescite[][345]{capital3}

作为这一章的结束,马克思尖锐地指出了拜物教的观念和信条是怎样轻易地从商业和生产活
动的复杂交织中建构起来的:“一切关于再生产总过程的表面的和颠倒的见解,都来自对商
人资本的考察,来自商人资本特有的运动在流通当事人头脑中引起的观
念。”\pagescite[][348]{capital3} 他进一步说,“在资本主义生产当事人和流通当事人
的头脑中,关于生产规律形成的观念,必然会完全偏离这些规律,必然只是表面运动在意识
中的表现。商人、交易所投机者、银行家的观念,必然是完全颠倒的”。他甚至断言,“在
这些人的头脑中,竞争也必然起一种完全颠倒的作用”。

“因此,从商人资本的观点来看,周转本身好像决定价格。另一方面,虽然产业资本的周转
速度,由于它会影响一定量资本所剥削的劳动的多少,所以会对利润量、从而会对一般利润
率起决定和限制的作用,但对商业资本来说,利润率是外部既定的,利润率和剩余价值的形
成之间的内在联系已经完全消失。”\pagescite[][349]{capital3} 当我们进入分配领域时,
这是一个普遍的问题。我们在处理生息资本流通的时候会再次遇到这个现象。与剩余价值生
产相关的一切联系都在社会的表面上抹去了,这是各色拜物教信条的来源。

不过,个别的商业资本家的确能通过加快周转速度,使其超过社会平均水平,而在竞争中获
得额外利润,这强化了表象世界的力量。“在这种情况下,他会赚到超额利润,正像在比平
均条件更有利的条件下进行生产的产业资本家会赚到超额利润一样。”(\textbf{这不就是
在第一卷提出的相对剩余价值理论吗}?)而且,“如果那些使他能加速资本周转的条件本
身是可以买卖的,例如店铺的位置,那么,他就要为此付出额外的租金,也就是说,把他的
一部分超额利润转化为地租”。这将我们带到了商人资本和地租的关系领域,以及这种关系
在城市环境下建构的方式(看看麦迪逊大道或者牛津街,你就明白马克思的意思了)

\begin{quotation}从总资本中有一定的部分在货币资本的形式上分离出来并独立起来,这
种货币资本的资本职能,是专门替整个产业资本家和商业资本家阶级完成这些活动……所以,
这种货币资本的运动,仍然不过是处在自己的再生产过程中的产业资本的一个独立部分的运
动。
\end{quotation}

这种资本形式的\textbf{“自主性”和“独立性”}的术语是至关重要的,对后面的分析有
许多提示。

\chapter{利息、信用和金融(第三卷 第21—26章)}

但是,首先我必须提醒你们,接下来所要讨论的这部分文本是恩格斯根据马克思的手稿付出
巨大努力后重新整理撰写的。虽然大多数学者认为恩格斯在做这项工作时已经尽力遵照马克
思的原意,但后来对马克思原始手稿的研究却表明,恩格斯的取舍可能并不完全正确。比如
说,每一个章节都由恩格斯从连续的手稿中分离出来并拟定标题——正因为如此,并不奇怪,
这些章节之间的联系才会如此紧密。你也会注意到恩格斯在原文中插入了一些自己的长篇论
述,为的是完善、改正和更新马克思的工作。在这里,我不会详细阐述这些问题。在接下来
的分析中,我会把原文看做是对马克思的观点的准确概述,尽管不是很完整。

货币资本(定义为用于生产剩余价值的货币)可以采取商品的形式。它既有交换价值(价格)
也有使用价值。它的使用价值即它促进了剩余价值的生产;它的交换价值(价格)就是利息。
这和马克思在第二卷中的阐述差异很大。在第二卷中,马克思说货币作为资本只能执行货币
的职能,也就是被用于买和卖。这种概念的转换非常重要。我并不认为这是马克思改变了他
的想法或是前后不一致。这更不是关系性概念的含义随着研究内容的展开发生变化的例证。
那么,这到底是怎么一回事呢?

我认为,当遇到这类问题时,从整体上考察马克思论述的变化是非常明智的。这些章节之间
的重要线索是马克思再一次明确提出了拜物教的概念。这一概念在第一卷的第1章中占非常
重要的地位。在那里,他提出资本的真正基础(即剩余价值的生产)埋藏在了真实存在却有
误导作用的表面现象之下。的确,我们会去市场用货币购买商品(包括劳动力)。但问题在
于,这些市场关系掩盖了凝结在商品生产和将商品带入到市场的整个过程中的劳动的
\textbf{社会性和官能性}。马克思所做的工作就是深入到表面现象背后。

马克在在《资本(论》第三卷这一位置回头讨论表面现象的拜物教性质。这里他宣称:
“\textbf{资本的物神形态和资本物神的观念已经完成}。”听起来,他为这一结论感到兴
奋并对其抱有必胜的信念。他宣称,\textbf{生息资本为“资本神秘化的最显眼的形式”}。
(Note)

我认为这里有一个马克思分析的重点。他在第二卷中不愿意(甚至达到强迫性拒绝的程度)
讨论特殊性,和在这里他为理解生息资本流通而研究这种情况的必要性形成了对比。

这就提出了这些特殊性和资本运动的一般规律之间的关系的问题。只有在这个语境下,从第
二卷中对根本事实的考察转向第三卷中拜物教性质的表面现象才是有意义的。……马克思对
此的可能回答是,表面运动所展示出来的强烈矛盾只有通过研究内在动力才能被预见和理解;
这一内在动力不仅生产出拜物教的形式,而且巩固了拜物教对资本运动规律的干预。因此,
我们阅读这些章节的目的就是去揭露内在规律和表面形式之间的这些关系实际上是如何发挥
作用的。

马克思认为利息同时具有“\textbf{自主的和独立的}”(马克思的原话)的性质,但是又
把它包括在价值和剩余价值生产的世界之中。他所说的“包括在”的意思即必须被确立的东
西。换句话说,利息率和生息资本的流通可以以自主的、独立的方式运动,因为它们是由不
断变化的供给、需求以及竞争所决定的特殊性。借用《政治经济学批判序言》中的话,
\textbf{是否存在这样一种方式,通过这种方式,能使这些特殊性反过来以决定的而非偶然
的方式来影响生产的一般性?如果存在,那么当这些特殊性自由运行时资本运动的一般规律
又是如何作用的呢?或者说这些特殊性会在某种程度上受制于资本运动的一般规律吗?}(Note)

\bigskip \hrulefill

如果生息资本的流通执行“\textbf{阶级共有资本}”的职能,那么我们怎么有可能把它从
资本运动一般规律的特定情形中排除出来?我尽可能直截了当地提出这个问题,因为不管答
案为何,一般来说它都会对我们如何理论化资本主义制度下的危机形成机制,尤其是对我们
如何利用马克思的观点去分析近来发生的事件,产生巨大影响。

第一步就是要考察与通过产业资本循环产生的剩余价值(利润)相比,生息资本是如何获得
它的自主性和独立性的。马克思首先区分了货币资本家(那些拥有货币权力的人)和产业资
本家(那些组织剩余价值生产的人)。利息率是由这两类资本家阶级之间的竞争所决定的。
于是,从历史角度——如果不是理论角度的话——来看,货币资本家和产业资本家之间的权力关
系就被放到了中心位置。

这一关系的历史有时候被从目的论的角度来解释:相对于产业资本,金融资本自1980年左右
起就不可避免地日益占据了主导地位,并且这一事实催生了一种不同的资本主义——金融资本。
与产业资本占主导地位的时期(比如马克思所生活的时代)相比,金融资本有迥然不同的运
动规律。马克思并没有(尽管在一些段落中看起来有)对其做一般性的论证。我也不打算这
么做。但毫无疑问的是,同一阶级的这两个派系之间(也包括他们和这一阶级的其他主要派
系,比如地主和商人之间)的\textbf{力量平衡从未达到一种稳定状态。霸权的转换肯定已
经发生了}。例如,在乔万尼·阿里吉的著作中,有一个貌似有理的观点:世界经济霸权的转
变(比如20世纪上半叶从不列颠到美国的转变)是在金融化阶段(在20世纪早期由希法亭、
霍布森和列宁所描述的那种阶段[4])之后发生的。从上世纪70年代开始的确定无疑的金融
化浪潮似乎预示了又一次的霸权转移(从美国到东亚?)。要理解资本主义的历史,我们需
要理解不同阶级中的不同派别之间在不同时期、不同地点实际的力量平衡,以及他们之间的
竞争所带来的一系列结果。

但是马克思走得更远。起初表现为阶级派系之间关系的东西事实上内部化到个别资本家的人
格特征中去了。所有的资本家都同时扮演着两个完全不同的角色。产业资本家必须总是以货
币的形式持有一定量的资本。这样一来,他们就可以随时选择把自己的钱用于扩大生产以生
产更多的剩余价值(和利润),或者仅仅为获取利息将其借给他人。这一决策的逻辑为资本
家提供了诱人的可能性。你会如何选择呢?是选择遭受实际生产剩余价值的一切麻烦(和爱
闹事的工人、不靠谱的机器或变化无常的市场打交道),还是选择将货币借出去,一边在巴
哈马群岛上享受生活一边吃利息?根据马克思的记录,英国的许多产业资本家的野心是不断
进行生产,直至他们的财产足以使他们成为食利者或者金融家,那么在退休后,他们单靠利
息就可以在田庄里过上舒适的生活。但马克思认为,如果每个人都试图做靠利息或者租金生
活的食利者而没有人愿意去生产剩余价值,那么\textbf{利息率就会降到零},与此同时,
再投资于生产的潜在利润会涨到一个无法预计的高度。\pagescite[][424]{capital3}在这
里,我们至少碰到了这样一种情形,即生息资本的流通必须服务于并且服从于剩余价值的生
产。(Big Note)

货币的积累是可以没有限制的。简单来说,生息资本好像拥有以复利(马克思在第一卷中比
喻道,它就像一只会下金蛋的鹅)速度增长的神奇(拜物的)力量。

复合增长会永远持续下去的幻象——马克思在这里通过引用1772年出版的一个小册子里描绘的
美妙景象而强调了这种幻象——“一个先令,在耶稣降生那一年以6\%的复利放出(大概是投
放在耶路撒冷的圣殿),会增长成一个比整个太阳系所能容纳的还要大的数目”。
\pagescite[][445]{capital3} 顺便提一句,这或许解释了为何我们不得不放弃金本位制度,
直到最后纸币的发行不再需要任何的商品基础。于是,世界货币的供给是无限的,因为它们
仅仅是数字而已。\textbf{美联储可以转眼之间立即增加约一万亿美元的货币供给(但增加
金块就另当别论了)。}虽然无限积累的观点“胜过一切信仰”,但马克思表明,它实际上
促使了1847—1848年和1857—1858年两次金融和商业崩溃的发生。借贷关系能够脱离控制并产
生越来越多的信用货币(欠条数目的增加)。这必然会带给所有的信用市场一种虚拟的性质。

因此,在这里,马克思使用了这\textbf{个非常重要但是又不成熟的概念:虚拟资本}。这
给了货币资本的拜物教性质一个更为形象可感的形状和形式。……我再次强调,拜物教正如
在第一卷中定义的那样,是真实和客观的,即使它掩盖了深层的价值关系。超市里的商品的
确是用来交换回货币的,但却是以掩盖创造了它的劳动(价值)的信息的方式。虚拟资本的
概念必须以同样的方式来理解。它并不是因为嗑了药而精神错乱的华尔街银行家创造出来的
东西,而是一种真正的资本形式——以有价格的商品形式存在的货币。虽然这种价格可能是虚
拟的,我们也不得不回应它(不管是支付抵押,为我们的存款寻求利息,还是借钱去做生
意)。

在马克思对借贷资本(为扩大生产而借的货币)和扩展到贴现票据(促进市场上价值的实现)
范围内的货币之间区分的讨论之中,有一个重要且简洁的说明。\textbf{货币资本从完全不
同的两方面介入了产业资本的流通——即循环的开始阶段和结束阶段。}同一个金融家可以一
面借钱给开发商建造住宅,一面借钱给买方购买那些住房来保证市场需求。因此货币资本同
时增加了商品的供给和需求。很容易看出这是怎么成为一个封闭循环的(比如,\textbf{在
住房的生产和实现过程中产生的资产泡沫})。这便是利息率和利润率相互交织、相互作用
的极为重要且经常是投机性的方式。(Big Note)

实际上,马克思想表达的是,在利息率的决定中存在某些不能比较的东西,所以才是不合理
的、矛盾的。

接下来,有必要提出两个相互关联的重要问题。首先,马克思的坚决主张——即对作为虚拟资
本范畴基础的利息的拜物教性质的观点——在何种程度上改变了我们对资本运动一般规律如何
发挥作用的理解?虽然商业资本分配范畴似乎可以纳入到\textbf{马克思之前建立的一般理
论框架之中},但相比于产业资本的流通,\textbf{生息资本流}通的影响似乎是另外一回事。
在我看来,马克思理论中另一个重要的“无理数”——\textbf{地租}——也\textbf{不能纳入
到}他的理论框架中,尽管他已经做了相反的说明。和利息一样,地租也是虚拟资本的一种
形式,一种真实存在并产生了实际影响的形式。当你到曼哈顿定居时,你不能说因为地租和
房价是虚拟的,你就不需要为一个虚构的东西支付任何费用。大多数人通过为房屋抵押贷款
支付利息而买到了房子——这种利息也是虚拟资本的一种形式。

那么,留给我们的问题是,使利息率服务和服从于价值和剩余价值生产的力量来自哪里
呢?……某种规训力量在起作用——它把投机性金融活动的所有幻象和谎言拉回到了实际生产
的尘世之中。但是,马克思的分析中同样存在令人不安的迹象——\textbf{金融和生产之间的
权力关系可能反过来}。

其中一个迹象来自于对马克思分析的附加说明,看起来相当有趣。当一个产业资本家以货币
形式积累资本并将其存到银行中获取利息(正如我们所看到的,当涉及固定资本流通时,这
种情况是很常见的,因为资本家必须贮藏货币资本进行固定资本的更新换代)时,利息就表
现为一种纯粹的对所有权的回报率。\textbf{这种纯粹财产所有权的消极回报与通过对生产
的组织和监督进行的积极的剩余价值创造形成了对比}。那么,为什么资本家不愿意给某些
人支付监督工资来照看生产,自己获取纯粹的财产所有权的回报?这就引出了一个资本主义
历史中非常有趣和至关重要的区别——\textbf{所有权与监督管理}。记住这些要点,让我们
开始讨论文中的细节部分吧。

\hrulefill \bigskip

\begin{quotation}生息资本却不是这样。它的特有性质也在于此。要把自己的货币作为生
息资本来增殖的货币占有者,把货币让渡给第三者,把它投入流通,使它成为一种作为资本
的商品;不仅对他自己来说是作为资本,而且对他人来说也是作为资本;它不仅对把它让渡
出去的人来说是资本,而且它一开始就是作为资本交给第三者的,这就是说,\textbf{是作
为这样一种价值,这种价值具有创造剩余价值、创造利润的使用价值}……它既不是被付出,
也不是被卖出,而\textbf{只是被贷出};它不过是在这样的条件下被转让:第一,它过一
定时期流回到它的起点;第二,它作为已经实现的资本流回,流回时,已经实现它的能够生
产剩余价值的那种使用价值。

  作为资本贷放的商品,按照它的性质伙食作为固定资本贷放,或是作为流动资本贷放。
\pagescite[][384]{capital3}(Big Note)
\end{quotation}

马克思把以商品形式进行的借贷归为生息资本循环的一般形式。然而,这样做有一个非常深
刻的含义。如果财产(比如房屋)和土地都可以被借出,那么租金和生息资本的循环之间就
一定存在一种内在联系。马克思在这里并没有提到这一点,但我在其他地方进一步讨论了这
个问题。而且,对这一问题的研究越深入,我就越觉得这是马克思政治经济学理论中一个至
关重要的缺失环节。(Big Note: 可以联系现实中的出租房屋)

\begin{quotation}借贷资本的出发点和复归点,它的放出和收回,都表现为任意的、以法
律意义上的交易为中介的运动。……作为独特的商品,资本也具有它的独特的让渡方式。因
此在这里,回流也不是表现为一定系列的经济行为的归宿和结果,而是表现为买者和卖者之
间的一种\textbf{特有的法律契约}的结果。回流的时间取决于再生产的过程;而就生息资
本来说,它作为资本的回流,好像只取决于贷出者和借入者之间的协议。因此,就这种交易
来说,\textbf{资本的回流不再表现为由生产过程决定的结果,而是表现为:好像贷出的资
本从来就没有丧失货币形式。当然,这种交易实际上是由现实的回流决定的。但这一点不会
在交易本身中表现出来。}\pagescite[][389-390]{capital3}

这种贷放就是把价值作为资本而不是作为货币或商品来让渡的适当形式。(Note)
\pagescite[][392]{capital3}

就其余的商品来说,使用价值最终会被消费掉,因而商品的实体和它的价值会一道消失。相
反,资本商品有一种特性:由于它的使用价值的消费,它的价值和它的使用价值不仅会保存
下来,而且会增加。\pagescite[][393]{capital3}

因此,“借贷货币资本的使用价值,也表现为这种资本生产价值和增加价值的能力”。并且,
“这个使用价值与普通商品不同,它本身就是价值,也就是由于货币作为资本使用而产生的
那个价值量超过货币原有的价值量所形成的余额。利润就是这个使用价值”。
\pagescite[][394]{capital3} (Note)

它通过使用才自行增殖,才作为资本来实现。但借入者必须把它作为已经实现的资本,即作
为价值加上剩余价值(利息)来偿还;而利息只能是他所实现的利润的一部分。只是一部分,
不是全部。\pagescite[][395]{capital3}

货币,商品也一样,自在地,潜在地,在可能性上是资本,它们能够作为资本出售,并且以
这个形式支配他人的劳动,要求占有他人的劳动,因而是自行增殖的价值。

其次,资本所以表现为商品,是因为利润分割为利息和\textbf{本来意义的}利润是由供求,
从而由竞争来调节的,这完全和商品的市场价格是由它们来调节的一样。但是在这里,不同
之处和相同之处一样的明显。\textbf{如果供求平衡},商品的市场价格就和它的生产价格
相一致,也就是说,这时它的价格就表现为由\textbf{资本主义生产的内部规律}来调节,
\textbf{而不是以竞争为转移},因为供求的变动只是说明市场价格同生产价格的偏离。
[31]

(哈维注:这是一个第一卷中常见的论点——在均衡时,供给和需求解释不了任何东西。甚至
连工资也是这样的:)

如果供求平衡,供求的作用就会相互抵消,工资就等于劳动力的价值。\textbf{但货币资本
的利息却不是这样}。在这里,竞争并不是决定对规律的偏离,而是\textbf{除了由竞争强
加的分割规律之外,不存在别的分割规律},因为我们以后会看到,并不存在“自然”利息
率。相反,我们把自然利息率理解为自由竞争决定的比率。利息率没有“自然”界限。在竞
争不只是决定偏离和波动的场合,因而,在它们的互相起反作用的力量达到均衡而任何决定
都停止的场合,那种需要决定的东西就是某种本身没有规律的、任意的东西。
\pagescite[][398]{capital3}

\end{quotation}

这是一段非常重要的论述:资本积累的动态变成了无规律的和任意的。这里,马克思在《政
治经济学批判大纲》中建立起来的,目前为止在整部《资本论》中用来研究资本运动规律的
一般性的交战规则(rules of engagement)的大厦似乎都要被瓦解了。这座大厦倒塌与否
取决于接下来几章的内容,因为马克思说:“在下一章,我们要进一步讨论这一点!”

这显然和《政治经济学批判大纲》中的框架产生了背离。马克思发现他无法把生息资本的流
通放到迄今一直指导他研究的假设框架里面。……我不由得想到,面对这种背离的未知结果,
马克思一定感到非常困难,压力巨大。一方面,在这几章中蕴含的紧张情绪说明把框架限制
抛于脑后的激动;但是,对生息资本的解释失去控制(这种不确定性和独立性)威胁到了他
之前所建立的整座理论大厦。

在这些段落中隐藏着另一个重点。\textbf{马克思使用了“生产价格”这个术语,而不是
“价值”}。这一术语上的改变有深刻的含义,但是我不打算在这里对其进行解释,因为在
第三卷前面的几个章节(第9章和第10章)中,在分析有着不同价值构成的不同部门之间的
竞争如何使利润率平均化时,这一问题已经出现过了。简单地说,\textbf{利润率平均化的
作用就是:使商品按照由不变资本、可变资本和平均利润率的价值形成的生产价格(c+v+p),
而不是更早的公式c+v+m所决定的商品价值进行交换。利润率平均化的结果就是,低价值构
成(劳动力比例较高)的部门中的剩余价值会向高价值构成(不变资本比例较高)的部门移
动}。(Big Note)

那么,当竞争从一个资本运动内部规律的纯粹执行者变成一个资本积累的无规律特性的积极
决定因素时,事情又发生了什么样的变化呢?

马克思一开始批评了吉尔巴特关于“生产当事人之间进行的交易的正义性”的观点。这一问
题的出现是因为利息率是一种\textbf{法律契约}而不是一种商品交换。在马克思看来,
\textbf{正义是“生产关系”的“自然结果”。虽然,这种“法律形式”表现为“当事人的
意志行为,作为他们的共同意志的表示,作为可以由国家强加给立约双方的契约”,但正义
的内容“要与生产方式相适应,相一致”}。因此,奴隶制和商品质量上的弄虚作假在资本
主义生产方式的基础上都是非正义的,而雇佣劳动却不是。\pagescite[][379]{capital3}
(Note)

他并不完全信奉柏拉图在《对话录》中赋予塞拉西马柯(Thrasymachus)的观点——正义由社
会中最有权威的人说了算(这是柏拉图为了说明更完美的正义理想而尽力去证伪的一个观
点);然而,马克思坚决反对柏拉图式的普世理念。正义是嵌入到由一定的生产方式所决定
的社会关系中的(随着资本逐渐主宰了社会关系,自由主义的正义论产生了)。“正义的”
利息率是和资本的持续再生产相一致的利息率。它明显区别于高利贷。这并不意味着在阶级
斗争过程中发挥作用的资产阶级的正义观念中不存在矛盾。但是马克思反对这样一种观点,
即认为存在一些阿基米德的支点,某种完美的正义形式和道德规范可以用这些支点评判世界。
他认为,这是蒲鲁东论证的主要缺陷。

如果我们考察一下现代工业在其中运动的周转周期——沉寂状态、逐渐活跃、繁荣、生产过剩、
崩溃、停滞、沉寂状态等等,对这种周期做进一步分析,则不属于我们的考察范围——我们就
会发现,\textbf{低利息率多数与繁荣时期或有额外利润的时期相适应,利息的提高与繁荣
转向急转直下的阶段相适应,而达到高利贷极限程度的最高利息则与危机相适应}。
\pagescite[][404]{capital3} (Big Note)

然而,\textbf{这仅仅是一种经验性的概括,而不是理论上的论述}。马克思也假设在危机
的高潮时期,国家对货币供给的干预无法使利息率下降到接近零的地步(而这正是美国自
2007年以来的情况)。我提到这点是因为,很明显,马克思试图掌握货币资本供求波动时的
供求状态,但是除了对利息率和利润率之间的变动关系做经验概括以外,似乎没有其他方法
了。

\begin{quotation}不过,利息率即使完全不以利润率的变动为转移,也具有下降的趋势。
这是由于两个主要原因:

  1,在老的富有的国家,不愿亲自使用资本的人所占有的国民资本部分在社会全部生产资
本中所占的比例,比新垦殖的贫穷的国家大。在英国,食利者阶级的人数是多么多啊!随着
食利者阶级的增大,资本贷放者阶级也增大起来,因为他们是同一些人。(拉姆塞)2,以
银行家为中介,产业家和商人对社会各阶级一切货币储蓄的支配能力也跟着不断增大,并且
这些储蓄也不断集中起来,达到能够起货币资本作用的数量。这些事实,都必然会压低利息
率的作用。
\end{quotation}

这是马克思第一次提到这个至关重要的问题:金融系统在集中用于流通的初始资本中所起的
作用(像以往一样,他保证“这一点以后还要详细说明”)。在整个资本主义的历史发展中,
金融系统扮演的调动社会各个阶级的储蓄,并将这些储蓄作为货币资本进行重新配置的角色
越来越重要。

但是,问题在于,“一个国家中占统治地位的\textbf{平均利息率}——不同于不断变动的
\textbf{市场利息率}——不能由任何规律决定。在这方面,像经济学家所说的自然利润率和
自然工资率那样的自然利息率,是没有的”。因此,“当竞争本身在这里起决定作用时,这
种决定本身是\textbf{偶然的,纯粹经验的},只有自命博学或想入非非的人,才会试图把
这种偶然说成必然的东西”。\textbf{但“习惯和法律传统等等都和竞争本身一样,对它的
决定发生作用”这一事实也削弱了竞争的影响。}\pagescite[][406-407]{capital3} 而
“这两种有权要求享有利润的人[产业资本家和放贷人]将怎样分割这种利润,本身是和一
个股份公司的共同利润在不同股东之间按百分比分配一样,纯粹是经验的、属于偶然性王国
的事情”。这和工资与利润之间的关系大不相同(按马克思的说法,地租和利润之间的关系
也是不同的):“在利息上……质的区别相反地是从同一剩余价值部分的纯粹量的分割中产
生的,”而工资和地租的情况则是刚好颠倒过来。地主提供有形商品——土地,劳动者提供劳
动力,但是货币资本家只提供货币资本——它只是一种价值的表现形式,对实际生产没有什么
看得见的贡献。(Note)

一般利润率当然是由决定剩余价值的因素(剩余价值量,预付资本量以及竞争的状况)所决
定的。这与利息形成了对比,正如我们所看到的,利息是由供给和需求决定的。但是,存在
着

\begin{quotation}在我们着重指出利息率和利润率的这种区别时,我们还撇开了有利于利
息率固定化的以下两种情况:1. 在历史上是现有生息资本,并且有一般利息率流传下来;
2. 世界市场不以一个国家的生产条件为转移而对利息率的确定所产生的直接影响,比它对
利润率的影响大得多。\pagescite[][412]{capital3}

\end{quotation}

根据我之前提到的,货币,尤其在信用的形式下,是资本可以自由飞翔的 “蝴蝶形式”。
证券市场上的利率变动报告就像“气象报告”一样,但借贷资本的价格仍然收敛于一个一般
值:
\begin{quotation}在货币市场上,只有贷出者和借入者互相对立。商品具有同一形式——货
币。资本因投在特殊生产部门或流通部门而具有的一切特殊形式,在这里都消失了。在这里,
资本是以独立价值即货币的没有差别的彼此等同的形态而存在的。特殊部门之间的竞争在这
里停止了;它们全体一起作为借款人出现,资本则以这样一个形式与它们全体相对立,在这
个形式上,按怎样的方式使用的问题对资本来说还是无关紧要的事。如果说产业资本只是在
特殊部门之间的运动和竞争中表现为\textbf{一个阶级的自在的共有资本,}那么,资本在
这里则是现实地充分地在资本的供求中表现为这样的东西。\pagescite[][413]{capital3}

\end{quotation}这是一个相当惊人的想法。在不理解货币资本是如何作为一个阶级的自在
的共有资本运作的情况下,我们究竟如何能揭示资本运动的一般规律呢?

\begin{quotation}另一方面,货币市场上的货币资本也实际具有这样一个形态,在这个形
态上,它是作为共同的要素,而不问它的特殊使用方式如何,根据每个特殊部门的生产需要,
被分配在不同部门之间,被分配在资本家阶级之间。并且,随着大工业的发展,出现在市场
上的货币资本,会越来越不由个别的资本家来代表,即越来越不由市场上现有资本的这个部
分或那个部分的所有者来代表,而是越来越表现为一个集中的有组织的量,这个量和实际的
生产完全不同,是受那些代表\textbf{社会资本}的银行家控制的。因此,就需求的形式来
说,和借贷资本相对立的是\textbf{一个阶级的力量};就供给来说,这个资本本身作为
\textbf{群体}表现为借贷资本。\pagescite[][413]{capital3} (Note)

利息对他(货币资本家)来说只是表现为\textbf{资本所有权}的果实,表现为抽掉了资本
再生产过程的资本自身的果实,即不进行“劳动”,不执行职能的资本的果实;而企业主收
入对他来说则只是表现为他用\textbf{资本所执行的职能}的果实,表现为资本的运动和过
程的果实,这种过程对他来说现在表现为他自己的活动,而与货币资本家的不活动,不参加
生产过程相对立。\pagescite[][420]{capital3}

总利润的这两部分硬化并且互相独立化了(注意此处关于独立化的主题)……好像它们出自
两个本质上不同的源泉。这种硬化和互相独立化,对整个资本家阶级和整个资本来说,现在
必然会固定下来。

资本的使用者,即使是用自有的资本从事经营,也具有双重身份,即资本的单纯所有者和资
本的使用者;他的资本本身,就其提供的利润范畴来说,也分成\textbf{资本所有权},即
处在生产过程\textbf{以外}的、本身提供利息的资本,和处在生产过程\textbf{以内}的、
由于在过程中活动而提供企业主收入的资本。\pagescite[][421]{capital3}

总利润分为利息和企业主收入这种分割,一旦转变为质的分割,就会对整个资本和整个资本
家阶级保持这个质的分割的性质……

“不管产业资本家是用自有的资本还是用借入的资本从事经营,都不会改变这样的情况,即
货币资本家阶级是作为一种特殊的资本家,货币资本是作为一种独立的资本,利息是作为一
个与这种特别资本相适应的独立的剩余价值形式,来同产业资本家相对立
的。”\pagescite[][422]{capital3} 但是,“不管他的资本在起点上已经作为货币资本存
在,还是要先转化为货币资本”。(Note)

假如大部分的资本家愿意把他们的资本转化为货币资本,那么,结果就会是货币资本大大贬
值和利息率惊人地下降;许多人马上就会不可能靠利息来生活,因而会被迫再变为产业资本
家。\pagescite[][424]{capital3}

\end{quotation}这里我们可以清晰地看到这点:生息资本的循环服从于剩余价值生产,并
由剩余价值生产支配。

因此,虽然“自然利息率”并不存在,但是这里也暗示着货币资本家和剩余价值生产活动之
间必须保持力量平衡(或者在个人层面来说,保持一些情绪上的平衡)。目前,我们仍无法
获知这一个平衡点在哪。(它仅仅是失常的和偶然的吗?)但是,这种相对于货币资本的长
期的不均衡的结果将会是它的贬值——这已经发出了明确的信号。日本于1990年开始,而美国
于2007年开始盛行的超低利率是预示这种不平衡的信号吗?(Big Note)

\begin{quotation}在利息的形式上,这种与雇佣劳动的对立却消失了;因为生息资本就它
本身来说,不是以雇佣劳动为自己的对立面,而是以执行职能的资本为自己的对立面;借贷
资本家就他本身来说,直接与在再生产过程中实际执行职能的资本家相对立,而不是与正是
在资本主义生产基础上被剥夺了生产资料的雇佣工人相对立。生息资本是作为所有权的资本
与作为职能的资本相对立的。但是,资本在它不执行职能的时候,不剥削工人,也不是同劳
动处于对立之中。\pagescite[][425]{capital3} (Big Note: 联系当前产业不振,从而金
融资本疲软外逃)

\end{quotation}

怎么强调从阶级斗争的动态过程来思考这个问题的重要性都不为过。虽然在劳动过程中和劳
动力市场上工人与执行职能的资本家之间的对抗和斗争关系是很清楚的,工人和作为所有权
的货币资本之间的关系则显得非常的抽象和不透明。发动工人去反抗货币资本的权力及其流
通方式存在更多的问题。小企业比工人群体看上去更有可能反抗银行和金融机构的权力。这
些斗争\textbf{一般很难整合到通常所说的阶级斗争里}。从历史上看,反抗货币资本家权
力(更为一般地反抗食利者)的斗争倾向于采取(并继续采取)民粹主义形式。最近,我们
在占领华尔街运动中能直观感受得到的民粹主义就是一个很好的例证。(Big Note)

生息资本给生产资本施加了生产剩余价值的压力;利息率越高,这种压力就越大。于是,生
产者就对工人们说,他们所遭受的高剥削率反映了高利息率,从而把他们的注意力从生产资
本家身上转移到银行家的贪婪和权力上。阶级斗争的动力因此被错放甚至是扭曲了。

然而,有一个更加复杂的问题。货币资本家和生产资本家这两个不同角色被内化在同一个人
身上自然而然地使执行职能的资本家把自己的企业家收入当作
\begin{quotation}同资本所有权无关的东西,宁可说是他作为非所有者,作为劳动者执行
职能的结果。因此,在资本家的脑袋里必然产生这样的观念:他的企业主收入远不是同雇佣
劳动形成某种对立,不仅是他人的无酬劳动,相反,它本身就是工资,是监督工资,wages
of superintendence of labour,是高于普通雇佣工人工资的工资,1. 因为这是较复杂的
劳动,2. 因为是资本家给自己支付的工资。\pagescite[][427]{capital3} (Note: 拜物
教)

这个利息形式使利润的另一部分取得企业主收入,以至监督工资这种质的形式。资本家作为
资本家所要执行的\textbf{特殊职能},并且恰好是他在同工人相区别和相对立中具有的特
殊职能,被表现为单纯的劳动职能。他创造剩余价值,不是因为他作为资本家进行劳动,而
是因为他除了具有作为资本家的属性以外,他也进行劳动。因此,剩余价值的这一部分也就
不再是剩余价值,而是与剩余价值相反的东西,是所完成的劳动的等价物。因为资本异化的
性质,它同劳动的对立,被转移到现实的剥削过程之外,即转移到生息资本上,所以这个剥
削过程本身也就表现为单纯的劳动过程,在这个过程中,执行职能的资本家与工人相比,不
过是在进行另一种劳动。因此,剥削的劳动和被剥削的劳动,二者作为劳动成了同一的东西。
\pagescite[][429-430]{capital3}

\end{quotation}

这样一来,“利润的一部分事实上能够作为工资分离出来”。在以更为复杂、更为细化的分
工为特点的大规模生产企业中,这种工资实际上付给了经理人。

\begin{quotation}凡是直接生产过程具有\textbf{社会结合过程}的形态,而不是表现为独
立生产者的孤立劳动的地方,都必然会产生监督和指挥的劳动。不过它具有二重性。

  一方面,凡是有许多个人进行协作的劳动,过程的联系和统一都必然要表现在一个指挥的
意志上,表现在各种与局部劳动无关而与工厂全部活动有关的职能上,就像一个乐队要有一
个指挥一样。这是一种生产劳动,是每一种结合的生产方式中必须进行的劳动。

  另一方面,——完全撇开商业部门不说——凡是建立在作为直接生产者的劳动者和生产资料所
有者之间的对立上的生活方式中,都必然会产生这种监督劳动。这种对立越严重,这种监督
劳动所起的作用也就越大。因此,它在奴隶制度下所起的作用达到了最大限度。
\pagescite[][431]{capital3}(Big Big Note: 强烈对立!)
\end{quotation}

目前,已经有许多文献证明了在不列颠被广泛运用的工厂管理技术,事实上是西印度群岛甘
蔗种植园中为管理大批奴隶所率先使用的。

“尤尔先生早已指出,”马克思继续写道,“‘我们的工业制度的灵魂’不是产业资本家,
而是产业经理。”无论事实如何,有一点是可以确定的,即“资本主义生产本身已经使那种
完全同资本所有权分离的指挥劳动比比皆是。因此,这种指挥劳动就无须资本家亲自进行
了”。\pagescite[][434]{capital3}

在合作工厂中,监督劳动的对立性质消失了,因为经理由工人支付报酬,他不再代表资本而
同工人相对立。随着信用而发展起来的股份企业,一般地说也有一种趋势,就是使这种管理
劳动作为一种职能越来越同自有资本或借入资本的占有权相分
离……\pagescite[][437]{capital3}

企业主收入和监督工资或管理工资的混淆,最初是由利润超过利息的余额所采取的同利息相
对立的形式造成的。由于一种辩护的意图,即不把利润解释为剩余价值即无酬劳动,而把它
解释为资本家自己劳动所取得的工资,这种混淆就进一步发展了。针对这种情况,于是社会
主义者提出了要求:要把利润实际地缩减为它在理论上伪称为的那种东西,即单纯的监督工
资。

但是,非技能化使得监督工资也倾向于下降,所以这一错误理论承受了更大的压力。随着工
人合作社的形成和股份公司的发展,“混淆企业主收入和管理工资的最后口实也站不住脚了,
利润在实践上也就表现为它在理论上无可辩驳的那种东西,即表现为单纯的剩余价值”。
(Note: 可以看作是马克思对资本家不创造价值的一种解释。监督工资。我感觉此处应该批
判。)

\begin{quotation}在资本主义生产基础上,一种涉及管理工资的新的欺诈在股份企业中发
展起来,这就是:在实际的经理之外并在他们之上,出现了一批董事和监事。对这些董事和
监事来说,管理和监督实际上不过是掠夺股东、发财致富的一个借口而已。
\pagescite[][438]{capital3}

\end{quotation}

马克思借用尤尔先生的观点,预言了所有权和管理权分离的潜在意义,以及管理阶级出现的
可能性。他并没有预测到它的全盛状态,部分是因为股份公司形式那时才刚刚起步。但是,
他确实预见到了后来被称为“货币管理资本主义”下产生的各种新的“欺诈”方式的可能性。

在当时流行的社会主义形式——合作社中(就像罗伯特·欧文所做的社会主义实验那样),管
理者报酬的问题也出现了。坦率地说,如果今天所有的机构和公司都以“蒙德拉贡模
式”(在前面讨论过)运作,那么我们将活在一个完全不同的世界里。美国大学校长的年工
资不会超过十五万美元,而不是超过一百万美元;而助理教师将不会只赚两万美元(如果他
们足够幸运的话),而是能赚五万美元一年。

\begin{quotation}在生息资本上,资本关系取得了它的最表面和最富有拜物教性质的形式。
在这里,我们看到的是G-G`,是生产更多货币的货币。
\end{quotation}

拜物教的极端形式——信用货币——控制了资本运动规律。它产生了虚拟的形式,这些形式神秘
化、扭曲\textbf{并最后破坏了马克思迄今为止所关注并将之理论化的资本积累的运动规
律。}这里的语言振聋发聩:(Big Big Note: 破坏!)

\begin{quotation}资本表现为利息的即资本自身增殖的神秘的和富有自我创造力的源泉。
现在,物(货币、商品、价值)作为单纯的物已经是资本,资本表现为单纯的物;总再生产
过程的结果表现为物自身具有的属性……因此,在生息资本上,这个自动的物神,自行增殖
的价值,会生出货币的货币,纯粹地表现出来了,并且在这个形式上\textbf{再也看不到它
的起源的任何痕迹了}。社会关系最终成为一种物即货币同它自身的关系。下面这一点也是
颠倒的:尽管利息只是利润即执行职能的资本家从工人身上榨取的剩余价值的一部分,现在
利息却反过来表现为资本的真正果实,表现为原初的东西,而现在转化为企业主收入形式的
利润,却表现为只是在再生产过程中附加进来和增添进来的东西。在这里,\textbf{资本的
物神形态和资本物神的观念已经完成}。在G-G′上,我们看到了资本的没有概念的形式,看
到了\textbf{生产关系的最高度的颠倒和物化}:资本的生息形态,资本的这样一种简单形
态,在这种形态中资本是它本身再生产过程的前提;货币或商品具有独立于再生产之外而增
殖本身价值的能力——资本的神秘化取得了最显眼的形式。

对于要把资本说成是价值即价值创造的独立源泉的庸俗经济学来说,这个形式自然是他求之
不得的。\pagescite[][441]{capital3}

\end{quotation}

更重要的问题是:资本家对这种拜物教形式的颠倒扭曲已经迷信到何种程度,以至于他们不
能理性地安排自己的再生产?如果说竞争的强制规律和他们所能获取的全部市场信号给他们
指明了错误的方向,那么资本除了给自己挖下更深的陷阱——即便没有自掘坟墓那么严重——还
能做什么?(Big Note: 资本自掘坟墓的必然)

\begin{quotation}
  资本作为生息资本,占有所能生产出来的一切财富,而资本迄今已经获得的一切,不过是
  对资本的无所不吞的食欲的分期偿付。按照资本的天生固有的规律,凡是人类所能提供的
  一切剩余劳动都属于它。这个摩洛赫! (Big Big Note)


  只有当利润(剩余价值)中再转化为资本的那部分,即被用来吮吸新的剩余劳动的那部分,
  可以叫作利息的时候,资本的积累过程才可以看作是复利的积
  累。\pagescite[][447-448]{capital3}

  资本拜物教的观念完成了。按照这个观念,积累的劳动产品,而且是作为货币固定下来的
  劳动产品,由于它天生的秘密性质,作为纯粹的自动体,具有几何级数生产剩余价值的能
  力。\pagescite[][449]{capital3}  (Note)

\end{quotation}

当在没有界限的货币系统内流通的生息资本上下盘旋,到达复利资产和虚拟资本价值的最上
层时,实际剩余价值生产的量的界限很快就被抛在脑后了;只有在危机来临的时候,这种限
制力量的存在才被承认。(Big Note)

\begin{quotation}
  银行家把借贷货币资本大量集中在自己手中,以致与产业资本家和商业资本家相对立的,
  不是单个的贷出者,而是作为所有贷出者的代表的银行家。 银行家成了货币资本的总管理
  人。另一方面,由于他们为整个商业界而借款,他们也把借入者集中起来,与所有贷出者
  相对立。 银行一方面代表货币资本的集中,贷出者的集中,另一方面代表借入者的集中。
  银行的利润一般地说在于: 它们借入时的利息率低于贷出时的利息率。\pagescite[][453]{capital3} 

  在获得货币的愿望之后,下一个迫切的愿望是,按照某种会带来利息或利润的投资方法,
  再把货币投放出去;因为,作为货币的货币是什么也生不出来的。因此,如果在过剩资本
  不断涌来的同时,投资范围得不到逐渐的充分的扩大,那么,\textbf{寻找投资场所的货
    币就必然会周期性地,在不同情况下多少不等地积累起来}。多年来,\textbf{国债一直
    是英国过剩财富的一个大吸收器}……在经营上\textbf{需要巨额资本并不时地吸引多余
    的闲置资本的各种企业}……至少在我国是\textbf{绝对必要}的,以便为在普通投资部
  门找不到地盘的\textbf{社会过剩财富的周期积累}打开出
  路。\pagescite[][468]{capital3}(Big Big Note)


\end{quotation}



















%%% Local Variables:
%%% mode: latex
%%% TeX-master: "../main"
%%% End:
