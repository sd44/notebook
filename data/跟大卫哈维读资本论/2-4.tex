\chapter{资本的时间与空间(第二卷 第12—14章)}

总周转时间等于资本的生产时间和流通时间之和,但是,生产时间分为劳动期间——生产价值的劳动实际上被用于商品生产的时间——以及完成商品生产过程所需要的不需要劳动投入的时间(例如,像大部分农业生产一样)。

为寻找减少处于闲置状态的资本数量的方法而产生的压力不断增加。这样,像加快周转时间和库存管理那样的技术,以及像信用制度那样的制度安排就开始发挥作用。缩短劳动期间和生产时间的竞争动力已经产生了深远的影响。

在连续生产的情况下,即使生产停止,流动资本也没有受到重大损失;而在机车制造的情况下,所有已经对象化在产品中的流动资本不是被搁置就是被白白耗费掉了,因此这意味着从事这种形式的生产具有更大风险。

\begin{quotation}
那些需要很长劳动期间,因而需要在较长时间内大量投资的企业,特别是只能大规模经营的企业,例如筑路、开凿运河等等,或者完全不是资本家经营,而由地方或国家出资兴办(至于劳动力,在较早的时期,多半实行强制劳动)。或者那种需要较长劳动期间才能生产出来的产品,只有很小一部分是靠资本家自己的财产来生产的。\pagescite[][260]{capital3} (Note)

在给私人建造房子时,私人分期付款给建筑业主。因此,事实上他是一部分一部分地支付房
屋的代价。而在发达的资本主义时期,一方面大量资本集中在单个资本家手里,另一方面,
除了单个资本家,又有联合的资本家(股份公司),同时信用制度也发展了,资本主义建筑
业主只是在例外的情况下才为个别私人定造房屋。他以为市场建筑整排的房屋或市区为业,
就像单个资本家以作为承包人从事铁路建筑为业一样。\pagescite[][260]{capital3}(Note)
建筑业主不再是为顾客,而是为市场从事建筑……以前,一个建筑业主为了投机,也许同时建筑三四栋房屋;现在,他却必须购买大块地皮……在上面建筑一二百栋房屋,因此他经营的企业,竟超出他本人财产的20倍到50倍。这笔基金用抵押的办法借来;钱会按照各栋房屋建筑的进度,拨给建筑业主。一旦发生危机,分期垫款就会停止支付,整个事业通常就会停顿;最好的情况,是房屋停建,等情况好转再建;最坏的情况,就是半价拍卖了事。\pagescite[][261]{capital3} 

举办劳动期间相当长而规模又很大的事业,只有在资本积聚已经十分显著,另一方面信用制度的发展又为资本家提供方便的手段,使他可以不用自己的资本而用别人的资本来预付、来冒险的时候。

问题在于生产资料和生活资料……分散或集中在单个资本家手中,也就是,资本的积聚已达到什么程度。信用会引起、加速和扩大资本在个人手中的积聚,就这一点来说,它会促使劳动期间从而周转时间缩短。\pagescite[][262]{capital3} 

\end{quotation}

在YouTube上搜索标题“九十小时内在中国建一座十五层的宾馆”就可以看到它。现在又有一个标题为“十五天内在中国建造一幢三十层高楼”的视频。当然,在这两个例子中,零部件是预制的,但是观察和思考劳动过程的性质也是十分有趣的。重点不仅在于协作、机械化以及对分工的协调,也在于劳动强度,这在《资本论》第一卷中逐渐成为剩余价值生产的一个主要贡献者。当然,工人仅仅获得九十小时(轮班工作)的工资。

直白地说,将生产时间缩短到技术上可能达到的最小程度的动机,是十分强烈的。马克思因
此引用了炼钢史上的进步,“炼钢法由1780年前后发现的搅拌炼钢法,变为现代贝氏炼钢法
和以后采用的各种最新方法”。虽然“生产时间大大缩短了,不过固定资本的投资也相应地
增加了”\pagescite[][267]{capital3} ——再一次强调了减速和加速之间的潜在矛盾。

非常不幸,到目前为止,空间的生产、空间关系以及地域形式(“位置”)的问题,在马克思思想的研究中不是被极大地忽视了,就是被视作显而易见从而是不值得研究的问题。

\begin{quotation}

  所以,流通时间只有从它是利用劳动时间方面的自然限制这一点来说,才决定价值。……可见,流通时间表现为劳动生产率的限制……因此,资本一方面力求摧毁交往即交换的一切地方限制,夺得整个地球作为它的市场,另一方面,它又力求用\textbf{时间去消灭空间},就是说,把商品从一个地方转移到另一个地方所花费的时间缩减到最低程度。资本越发展,从而资本借以流通的市场,构成资本空间流通道路的市场越扩大,资本同时也就越是力求在空间上更加扩大市场,力求用时间更多地消灭空间。 \pagescite[][33]{karlvol46b} (Big Big Note)

在运输工具发展的同时,不仅空间运动的速度加快了,而且空间距离在时间上也缩短了。\pagescite[][278]{capital2} 

\end{quotation}

交通运输工具的创新和投资持久地变革着资本所创造的地理景观。\textbf{空间—经济的相对空间在不
停地变换。在资本家竞争的整体景观里,由于相对区位优势的改变,整个城市的资本主义活
动都将走向衰落。大量固定资本的价值都被嵌入到了土地上,随着其他地方那些激励资本活
动的新的通讯线路和运输设施的建设,它们的价值要么得到增强,要么将受到威胁。}马克思
没有详细考察这个问题,但这些固定资本资产价值所面临的价值重估或贬值的永久威胁,是
资本主义历史中不稳定性的一个重要来源:在20世纪80年代左右,由于历时已久的全球化进
程的动力彻底转变了方向,随着生产大规模地、主要但不是唯一地向东亚转移,资本主义发
展中的许多核心地带——诸如底特律、巴尔的摩、曼彻斯特、谢菲尔德、埃森、里尔及其他老
牌制造业城市——经历了极其艰难的去工业化过程。国内的地理转移——从美国的中西部和东北
部到南部和西南部——在制造不稳定和不均衡的资本主义地理发展方面,与国际转移同样重
要。(Big Big Note)


\begin{quotation}
  首先是运输工具的运行次数有或大或小的增加,例如,一方面,一条铁路的列车次数,随
  着生产地点生产的增加,随着它变为较大的生产中心而增加,而且这种增加,是面向现有
  的销售市场,也就是面向大生产中心、人口中心、输出港等等的。另一方面,这种交通特
  别便利的情况以及由此而加速的资本周转(就资本周转取决于流通时间来说),反过来既
  使生产中心又使它的销售地点加速集中。随着大量人口和资本在一定的地点这样加速集中,
  大量资本也就集中在少数人手里。\pagescite[][278]{capital2} 

\end{quotation}

马克思在这里要表达的是一个我们地理学者称之为\textbf{相对空间关系}的理论。这一空间
的确定不是根据物理距离,而是\textbf{根据距离间的摩擦力,这个距离间的摩擦力用穿过
  物理空间所需的变化着的费用和时间来衡量。}物理空间本身与资本并无关系。资本所关心
的是运动的费用和时间,它会尽其所能地寻求费用和时间的最小化,并且减少运动的空间障
碍。为此,必须不断地从根本上变革空间关系。马克思在《政治经济学批判大纲》中说“用
时间消灭空间”指的就是这个意思。资本主义为了实现减少空间障碍和距离摩擦力这一目标
而进行的创新的历史令人叹为观止。但是\textbf{障碍不仅是物质的:它们同时也是社会的
  和政治的。减少资本运动(不一定是人的运动)的关税壁垒及其他政治障碍已经成为国际
  资本主义新秩序(一个充满矛盾并频繁成为政治冲突和社会斗争的焦点的过程)的“圣
  杯”的一部分。}但是很难想象,如果20世纪50年代左右欧洲的贸易壁垒没有被逐步打破,
资本积累会受多大的抑制。到了20世纪70年代中期,整个欧洲滞留在边境海关检查点的卡车
长队已经让人无法容忍。(Big Big Note)

\begin{quotation}
随着资本主义生产的进步,交通运输工具的发展会缩短一定量商品的流通时间,那么反过来说,这种进步以及由于交通运输工具发展而提供的可能性,又引起了开拓越来越远的市场,简言之,开拓世界市场的必要性。运输中的并且是运往远地的商品会大大增长,因而,在较长时间内不断处在商品资本阶段、处在流通时间内的那部分社会资本,也会绝对地和相对地增加。\pagescite[][279]{capital2} 

经济学家总爱忘记,企业所需资本的一部分不仅不断交替地通过货币资本、生产资本和商品
资本这三种形式,而且这一资本的各个部分不断地同时具有这三种形式,尽管这些部分的相对
量是不断变化的。\pagescite[][284]{capital2} 
(Big Note)


\end{quotation}


但马克思的发现中的“奇怪”之处确实提出了一个问题。为了反驳李嘉图的观点,马克思有时不得不将自己的剩余价值生产理论与一个事实相协调,即周转时间不同将导致年剥削率的明显不同,而且缩短周转时间确实能够提高年剩余价值率。马克思的答案是,必须区分预付资本和所用资本。

\begin{quotation}
预付可变资本,只是在它被实际使用时,在它被实际使用的时间内,才作为可变资本执行职能;而在它没有被使用,仅仅被预付,充当储备的时间内,不作为可变资本执行职能。但是,一切会使预付的可变资本和使用的可变资本的比例发生变化的情况,总起来说,就是周转期间的差别。剩余价值生产的规律是:在剩余价值率相等时,执行职能的等量可变资本生产等量的剩余价值。\pagescite[][332]{capital2} 

\end{quotation}
 $$ M' = m'n $$

 这里,我们发现了个别资本家下述行为的一个额外动机:进一步\textbf{用时间消灭空间};
 在商业策略中积极追求时空压缩。因为一旦成功缩短了劳动期间和(或)流通时间(例如,
 通过寻找让自己的商品更快推向市场的方式),他的预付资本就能获得较高的利润率(即便
 所用资本的利润不变),只要新的生产和流通策略的相关成本不会抵消掉他们更高的利润
 率。

 但还有一种间接的方式能够解决流通和周转时间的问题,即货币市场和信用制度的发展,这是本章的伏线。

在第二卷中,马克思在这里明确指出,资本主义社会必须存在大量的、随时可得的过剩货币资本,以支持生产活动的连续性。他多少附带性地提到,正是这一点使货币市场和信用制度对资本主义的正常运行来说非常必要。

在第一个周转期结束时收回的五百英镑已经由工人生产出来。所以,资本家为第二期的可变资本预付的五百英镑,事实上是工人自己生产的产品的等价物。

\begin{quotation}

  如果我们设想一个社会不是资本主义社会,而是共产主义社会,那么首先,货币资本会完
  全消失,因而,货币资本所引起的交易上的伪装也会消失。问题就简单地归结为:社会必
  须预先计算好,能把多少劳动、生产资料和生活资料用在这样一些产业部门而不致受任何
  损害,这些部门,如铁路建设,在一年或一年以上的较长时间内不提供任何生产资料和生
  活资料,不提供任何有用效果,但会从全年总生产中取走劳动、生产资料和生活资
  料。\pagescite[][349]{capital2} (Big Note)
\end{quotation}

直到此时,共产主义的观念主要局限于联合劳动者为了社会的目的而自由地控制和组织自己的劳动。但这里隐约出现了一个重要的协调问题——长期中发展生产力和扩大基础设施将吸收大量的劳动力和生产资料,而在相当长的一段时间内不提供直接好处。


它提出了一些问题,“社会”可能怎样合理地协调和“计算”总劳动分工,同时怎样在缺乏
市场信号的情况下,以鼓励而不是阻碍联合劳动者自由追求他们的共同利益的方式来管理长
期发展计划。这里的分析在《资本论》中第一次但不是最后一次地表明,在共产主义事业的
核心有一个核心矛盾。只有当建立在严格的私人财产基础上的规训机制巩固了资本主义生产
方式之后,个别资本家的自由和解放才成为可能。同样的道理,共产主义也必须在一个估计、
协调、计算的总体框架下,找到一条重新定义和保护联合劳动者的自由和解放的道路。这个
总体框架限制并规训必要的社会和物质基础设施的生产,使它们能增加人类解放的前景。(Big Big Note)


\begin{quotation}
  相反,在资本主义社会,社会的理智总是事后才起作用,因此可能并且必然会不断发生巨大的
  紊乱。一方面,货币市场受到压力,反过来,货币市场的缓和又造成大批这样的企业的产
  生,也就是造成那些后来对货币市场产生压力的条件。货币市场受到压力,是因为在这里不
  断需要大规模地长期预付货币资本。这里完全撇开不说产业家和商人会把他们经营企业所
  必需的货币资本投入铁路投机事业等等,并通过在货币市场上借贷来补偿这种货币资本。

 (这个过程给第三卷中分析的金融资本和信用制度一切“颠倒错乱的形式”和“疯狂的”行为提供了技术基础:) 

 另一方面,社会的可供支配的生产资本受到压力。因为生产资本的要素不断地从市场上被取
 走,而投入市场来代替它们的只是货币等价物,所以,\textbf{有支付能力的需求将会增加,而
   这种需求本身不会提供任何供给要素。}因此,生活资料和生产材料的价格都会上
 涨。\textbf{此外,这个时候,通常是欺诈盛行,资本会发生大规模转移。投机家、承包人、
   工程师、律师等一伙人,会发财致富。}他们引起市场上强烈的消费需求,同时工资也会提
 高。至于食品,那么,农业当然也会因此受到剌激。但是,因为这些食品不能在一年内突然增
 多,所以它们的输入,像一般外国食品(咖啡、砂糖、葡萄酒)和奢侈品的输入一样,将会增加。
 因此,\textbf{在进口业的这个部分,就会发生输入过剩和投机}。另-方面,\textbf{在那些
   生产可以急剧增长的产业部门(真正的制造业、采矿业等等) ,由于价格的提高,会发生突
   然的扩大,随即发生崩溃。这同样会影响到劳动市场,}以致把大量潜在的相对过剩人口,甚
 至已经就业的工人,吸引到新的产业部门中去。一般说来,像铁路建设那样大规模的企业,会
 从劳动市场上取走一定数量的劳动力,这种劳动力的来源仅仅是某些只使用壮工的部门(如农
 业等等)。甚至在新企业已经成为稳定的生产部门以后,从而,在它所需要的流动的工人阶级
 已经形成以后,这种现象还会发生。例如,在铁路建设的规模突然比平均规模大时,情况就是
 这样。工人后备军一寸主种后备军的压力使工资保持较低的水平一一有一部分被吸收了。现
 在工资普遍上涨,甚至劳动市场上就业情况一直不错的部分也是这样。这个现象会持续一段
 时间,直到不可避免的崩溃再把工人后备军游离出来,再把工资压低到最低限度,甚至压低到
 这个限度以下。(Big Big Note)\pagescite[][349]{capital2} 



 资本主义生产方式中的矛盾:工人作为商品的买者,对于市场来说是重要的。但是作为他
 们的商品——劳动力——的卖者,资本主义社会的趋势是把它的价格限制在最低限度。——还有
 一个矛盾:资本主义生产全力扩张的时期,通常就是生产过剩的时期;因为生产能力从来
 没有能使用到这个程度,以致它不仅能够生产更多的价值,而且还能把它实现。商品的出
 售,商品资本的实现,从而剩余价值的实现,不是受一般社会的消费需求的限制,而是受
 大多数人总是处于贫困状态,而且必然总是处于贫困状态的那种社会的消费需求的限制。(Big Big Note: 消费不足危机论)\pagescite[][350]{capital2} 

 然而,棉纱可能在印度再赊卖出去。以此在印度赊购产品,作为回头货运回英国,或把一张
 金额相当的汇票汇回英国。只要这种状态延续下去,就会对印度的货币市场造成一种压力,
 而对英国的反作用可能在英国引起一次危机。这种危机,即使在它伴随着向印度输出贵金属
 的情况下,也会在印度引起一次新的危机,因为曾经从印度的银行取得贷款的英国商行和它
 们的印度分行会陷于破产。因此,\textbf{出现贸易逆差的市场和出现贸易顺差的市场会同
   时发生危机。}这种现象还可以更加复杂化。例如,英国把银块送往印度,但是,印度的
 英国债权人现在会在印度索债,于是印度随后不久又要把它的银块送回英国。

 英国和印度之间的贸易差额,可以看起来是平衡的,或者只是显出偏向这方或那方的微小的摆
 动。但是,危机一旦在英国爆发,就可以看到没有卖出去的棉纺织品堆积在印度(就是商品资
 本没有转化为货币资本,从这方面说,也就是生产过剩) J 另一方面,在英国,不仅堆积着没有
 卖出去的印度产品的存货,而且大部分已经卖出、已经消费的存货还丝毫没有得到货款。因
 此,在货币市场上作为危机表现出来的,实际上不过是表现生产过程和再生产过程本身的失
 常。\pagescite[][352]{capital2} (Big Note: 顺差、逆差、平衡均可爆发危机)

论第二卷第17章:剩余价值的流通。本章的核心问题“不在于剩余价值从何而来,而在于剩余价值借以货币化的货币从何而来?”作为卓越的货币商品,金的生产能否提供实现剩余价值所需要的额外货币?如果不能(很明显马克思拒绝了这种可能性,尽管他并不否认金生产者的独特作用),那我们就面临一个尴尬的问题:\textbf{有效需求从何而来,以实现不断被抛向市场的剩余价值?}(Big Big Note)


\begin{quotation}
  随着资本主义生产的发展,信用制度也同时发展起来。资本家还不能在自己的企业中使用的
  货币资本,会被别人使用,而他从别人那里得到利息。对他来说,这种货币资本是作为特殊意
  义上的货币资本,也就是作为一种与生产资本不同的资本执行着职能。但是它在别人手里却
  作为资本起作用。很明显,当剩余价值的实现更加频繁,剩余价值生产的规模更加扩大时,新
  的货币资本即作为资本的货币技入货币市场的比例也会增加,其中至少有一大部分会重新被
  吸收来扩大生产。\pagescite[][356]{capital2} 

\end{quotation}

让我简单解释一下这个问题的结构。纵观《资本论》,马克思假定(至少在分析关于货币资
本和金融的章节之前)供给和需求处于均衡状态。但是我们现在遇到的情况不同,不仅供求
不均衡,而且资本家使出浑身解数来扩大供求之间的缺口。简单地说,\textbf{资本家的需
  求是生产资料(c)和劳动力(v),但他提供给市场的商品价值是c+v+s,从而商品价值
  的供给系统性地超过了需求。}此外,最大限度地获取剩余价值的欲望则把这种不一致推向
了极限。对剩余价值的额外有效需求从何而来?如果剩余价值不能实现,资本流通就会停
止。

在我们又面临这样的情况,必须分清接下来到底是马克思的一般答案,还是政治经济学家
的“似是而非的遁辞”的一般答案:“当一个x×1000镑的商品量要流通时,不论这个商品量
的价值是否包含剩余价值,不论这个商品量是否按资本主义方式生产,这个流通所必需的货
币量决不会因此有所改变。\textbf{可见,这个问题本来就是不存在的。}……如果这里存在
什么问题,那么,它和总的问题是一致的,一个国家的商品流通所必需的货币额从何而
来?”问题于是被简化为调节一个国家的货币供给,使之足以满足所有商品交换的需要的层次。
我认为马克思的意思是说,这种观点是一种最厉害、最似是而非的遁辞:它等同于马克思在
第一卷中严厉地斥为“幼稚的胡言乱语”的萨伊定律。(Big Big Note)

不过,除此以外,资本家就不再是处在流通中的货币量的起点了。可是,现在只有两个起点:资本家和工人。所有第三种人,或者是为这两个阶级服务,从他们那里得到货币作为报酬,或者是不为他们服务,而在地租、利息等形式上成为剩余价值的共有者。\pagescite[][368]{capital2} 


\end{quotation}

至于扩大再生产的情况,马克思没能找到一条明显的研究路径,这源于一个简单的事实:剩
余价值的一部分现在必须投资于生产性消费(新的生产资料和劳动力的增加),\textbf{从
  而削弱了资产阶级的消费能力。}如果资本家不得不放弃个人消费以进行更多的生产性消费,
那么他就必须得再次深入挖掘自己的货币储备,否则就不可能清除掉已经生产出的多余的剩
余价值。\textbf{认为这种储备是无底洞的想法显然是荒谬的。扩大的总需求的来源问题需
  要加以解决,但马克思做的还不够。}

我能找到的最明确的答案是资本家通过\textbf{先买(从而实现剩余价值)后付(剩余价值
  已经货币化后)}这个简单和长期存在的做法解决困难。换句话说,他们\textbf{用债务支
  撑扩张。}这涉及货币市场和信用制度,正如我们所看到的,马克思在整个第二卷不愿意
(尽管他承认了它的绝对必要性)从事这个研究。这可能就是解决方法,正如我们已经看到
的,马克思在第三卷对货币市场、金融资本和信用制度的作用的探究中已经作出了暗示。把
这一观点推向极致,这个论点表明,\textbf{通过剩余价值生产进行的资本积累必定与市场
  上实现剩余价值的债务积累同步进行。}(Big Big Note)

马克思近乎暂时性地承认了这一点。一部分剩余价值被投资于生产扩张,这就减少了可利用
的作为收入来流通用于产品实现的数量。这样就生产出了追加的剩余价值。“这里又出现了
和上面一样的问题。用以实现现在以商品形式存在的追加剩余价值的追加货币从何而
来?”\pagescite[][381]{capital2} 马克思和以前一样,研究了古典政治经济学提出的许
多解决方案,它们试图通过对货币流通的考察来解决问题,最终都诉诸金生产者的活动。除
了求助于信用的方案外,马克思对所有解决方案都心存怀疑,尽管它们至少有一些技术上的
可能性:“只要那些和信用制度一起发展的辅助工具发生这种作用,它们就会直接增加资本
主义的财富……这样也就解决了一个毫无意义的问题,即\textbf{资本主义生产按它现在的
  规模,没有信用制度,只有金属流通,能否存在。显然,不能存在。}相反,它会受到贵金
属生产的规模的限制。另一方面,我们对于信用制度在它提供货币资本或使货币资本发生作
用时所具有的生产力,也不应该有任何神秘的观念。”不幸和令人沮丧的是,他补充
道:“对这个问题的进一步说明,不属于这里的范
围。”\pagescite[][383]{capital2} (Big Big Note)

前面已经指出,必须有作为“流通基金”来发挥作用的“货币基金”,它不同于扩大再生产
需要的“潜在的货币资本”。马克思考虑了潜在的货币资本可能存在于哪里,他认为有银行
存款、公债券和股票。但是用于实现剩余价值的流通基金在哪里呢?当货币不得不被用于这
个目的,甚至为此而进行贮藏时会发生什么?不幸的是,马克思没有给出答案。

\chapter{资本的再生产(第二卷 第18—20章)}
\label{chap:reproduct}

在第二卷的第三篇中,马克思设想了一个分成两大部类的经济。第一部类为其他资本家生产
生产资料(原材料、半成品、机器和包括生产的建成环境在内的其他固定资本项目)。第二
部类生产供工人和资本家个人消费的消费资料(也包括消费的建成环境)。生产消费资料的
部类必须从第一部类购买其生产资料。第一部类的工人和资本家,必须从第二部类购买其消
费资料。这种经济要想平稳运行,两大部类之间的交换就要互相平衡。在简单再生产(没有
扩张)的情形下,流向第二部类的生产资料的价值必须与流向第一部类的工人和资本家的消
费资料的价值相等。

用代数形式,这可以表达为:

第一部类 $c1+v1+m1=w1$(生产资料的总价值)

第二部类 $c2+v2+m2=w2$(消费资料的总价值)

生产资料的总需求是$c1+c2$。消费资料的总需求是$v1+v2+m1+m2$。如果我们假定,需求和供给是均衡的,那么$W2=c2+v2+m2=v1+v2+m1+m2$

在等式两边消去同类项后,得到$c2=v1+m1$

如果这个能够确保连续、平衡的再生产的必要的价值比例能够实现,那么第二部类对生产资
料的需求必须等于第一部类对消费资料的需求。 “由此得出结论,”马克思说,“在简单再
生产中,第一部类的商品资本中的v+m价值额(也就是第一部类的总商品产品中与此相应的比
例部分),必须等于不变资本……也就是第二部类的总商品产品中分出来的与此相应的部
分。”\pagescite[][446]{capital2} 

马克思所设计的图式包含了\textbf{各类假设}——只有工人和资本家两个阶级(在第17章中简略陈述
过);只有生产生产资料和生产消费资料的两个部门(尽管在某些时候他确实又将消费资料
分成必需品和奢侈品);需求与供给是均衡的;所有产品的周转时间都是一年;没有技术变
革;所有产品按其价值进行交换——这还只是提到的一些主要的假设。尽管马克思最初承认,
他应该 “既在价值又在物质”(使用价值)形式上考察再生产过程,但事实上他仅仅在价值
形式上解决了两大部类之间的比例关系,因此要假设再生产物质上的数量需求会自动得到满
足。从这些假设中会引出很多问题,而放松这些假设会引起令人难以置信的复杂性。(Big Big Note)


在这些图式中,应该要注意,工人的消费占有“在比例上有决定意义的部分”。因此如果这个图式有一点政治指向性的话,那就是稳定工人收入的必要性,这是为了协调生产资料的总产出和对消费资料的总需求之间的关系。这与第一卷中的发现相矛盾,在那里马克思把工人阶级日渐加剧的贫困化视作是自由市场资本主义不可避免的一个结果。但是,因为第二卷并没有与“一般规律”等价的一章,马克思仅仅是暗示了这个矛盾。(Big Note)

相反地,我们需要想象一下,第二卷中与“一般规律”那章等价的章节,可能是什么样呢?
例如,为了使市场上价值实现的条件保持稳定,在很多地方,会有数量庞大的工人日益被卷
入到无止境的、越来越盲目的消费主义浪潮中去吗?再者,考虑到他们痴迷于诱人的资本主
义的消费主义的程度,这些工人是如何对社会主义革命失去兴趣的?反消费主义(这类运
动20世纪60年代的时候在许多地区确实很活跃,而且它如今是很多环境政治学的核心)在革
命运动中起着怎样的作用呢?当然,很难想象,马克思竟然会写下这么一章,并且对于多数
虔诚的马克思主义者来说,这种想法几乎肯定会被谴责为具有诽谤性。(Big Note)

但是,马克思的再生产图式很有趣的一点是,它们决不拒绝这些可能性(举个例子,这几乎
就是罗莎·卢森堡对它们的内容感到如此失望的原因)。并且,美国和其他发达资本主义国
家70\%的经济活动,都是由消费主义驱动的(相反中国只占一半,可能更接近于马克思所处
的年代盛行的情况),而且许多所谓的“富裕的”工人,确实深深地醉心于他们所处的资本
主义世界(和所有明显的缺陷)的消费主义之中。因此,在这里我们手头就有些工具可以用
来分析这一类政治经济形势。显然,这与第一卷第25章中\textbf{工人日益贫困化的论点}之间的矛盾,
带来了严重的问题。然而,虔诚的马克思主义者当然不能仅仅因为它是严重的和棘手的,就
避开这个矛盾。

但是,确实存在巧妙处理这个核心矛盾的方法。马克思在某些地方提到了我们今天所谓
的“中产阶级”存在。在当代条件下,那个阶级的主要作用是作为消费的中坚分子,并
且对资本主义民主的运行提供广泛的政治支持。

马尔萨斯首先指出了这个社会阶层在提供必要的有效需求方面的重要性(尽管他观念中的消
费者阶级更纯粹地是贵族化和寄生性的,现在不具有政治可行性——除了海湾各国中)。由于
我们早已接受了这个观点——一个主要充当经理层、管理层和服务角色,有着稳定且足够工资
的中产阶级的发展,对资本主义经济、社会和政治稳定都是十分重要的。那么,可以认为,
我们目前遇到的矛盾源自马克思的两阶级模型的假定,而不是任何现实情况。三个阶级情况
下的矛盾可能表现为第一卷中设想的对更低下的工人阶级(例如,在中国)的工资压制,以
及第二卷中设想的,收入向中产阶级(包含富裕的工人阶层和非生产阶级)消费者(例如,
在美国一些工人拥有房屋所有权,且实现了郊区化的生活方式)的流动,它足够提供必要的
有效需求。当然,在马克思的图式下,中产阶级的收入最终都必须来自价值和剩余价值的生
产,尽管在现代情况下,这毫无疑问会由\textbf{以债务驱动为基础}的国家在消费基金上的
支出和促进\textbf{中产阶级消费主义(尤其是与住房需求有关的)的信用扩张}来补充。有
意思的是,现在人们普遍认为北美和欧洲很多地区中产阶级的生活水平是岌岌可危的——一定
程度上是因为过度负债——并且这与对支撑经济的有效总需求不足的大声哀叹有着直接联系。
通过中国和其他发展中国家中产阶级的形成而可能实现的国内消费需求的增长,作为一种补
偿运动被抱以厚望。现在中国的政策制定者面临很强的内外压力,采取积极措施来刺激国内
市场。我们也能听到一些贸易盈余国家中有影响力的政策制定者的要求,例如德国,要求放
松工资压制的倾向(第一卷中的)并且刺激消费(第二卷中的)以帮助总体经济增长(迄今
为止德国拒绝了这样的要求)。我发现,倘若我们能够灵活地、广泛地加以利用,那么在再
生产图式的一般框架内考虑当前的形势是很有用的。 (Big Big Note)

与第一卷相比,第二卷还有其他显著的差异。在第一卷中,与技术细节相比,马克思
对\textbf{资本和劳动阶级关系的再生产},以及发现自己陷入了无止境积累(“为积累而积
累”)的资产阶级的“历史使命”更感兴趣。与“怎样”相比,他对“为什么”更感兴趣。
而在第二卷,对于“为什么”的关注基本上消失了。相反,他建立了资本是怎样永远地进行
积累的技术模型。读这几章的时候,记住这一点是很重要的,即阶级关系的再生产,尽管很
少提到它,始终是核心问题。

在20世纪30年代末期,瓦西里·里昂惕夫是一名俄国经济学家,在20世纪30年代先是去了德国,后来又到了美国。他为了创立后来广为人知的“投入产出分析”,详细研究了马克思的模型。图6举出了一个典型的里昂惕夫矩阵。使用这样一个投入—产出矩阵,对于一个给定的产业(例如钢铁业),可以估计提高产出水平需要多少额外的投入(例如煤炭、能源和铁矿石),还能反复地追溯增加煤炭生产所需的额外投入(例如,额外的机器和这些机器所需的额外的钢铁)。它成为中央计划的一个重要工具。

在世界上的许多地方和不同的政治环境下,对投资和劳动的合理的社会配置是公共政策十分重要的一个方面。尽管使用这种技术进行的中央计划的名声并不太好,但在公司内部,我们正在使用其更精确的版本以确定复杂的生产体系中的最佳效率。

后来的研究从不同方向对这个图式的技术特性进行了详细阐述,并且极大地提高了描述的数学复杂性。这些阐述不但没有解决,反而加深了马克思留下的奥秘。例如,安德鲁·特里格最近的一项研究说明,“如果不能清楚地说明再生产图式的目的,那么对它们是用来做什么的,它们如何同第二卷的剩余部分联系起来,以及它们如何同整个《资本论》联系起来,就不能达成一致”。总之,用马克思自己的术语解释马克思,在这个情形中近乎是不可能的。

\begin{quotation}
因此,总括起来成为社会资本的各个单个资本的循环,也就是说,就社会资本的总体来考察的循环,不仅包括资本的流通,而且也包括一般的商品流通。后者本来只能由两部分构成:1. 资本本身的循环;2. 进入个人消费的商品的循环,也就是工人用工资,资本家用剩余价值(或其中的一部分)购买的那些商品的循环。\pagescite[][390]{capital2} 
\end{quotation}

当你处理总体经济的问题时,千万不要认为可用的货币数量会有一个界限。由于现在货币已经失去了其金属基础,它能够由中央银行无限制地创造出来。非常引人瞩目的是,美联储可以宣布在任何它喜欢的时候将一万亿美元注入经济中。尽管事实上这可能存在一些政治上的限制(导致金融萧条),它们总是可以被绕过的。(Big Big Note)

\begin{quotation}
在资本主义生产的基础上,历时较长范围较广的事业,要求为较长的时间预付较大量的货币资
本。所以,这一类领域里的生产取决于单个资本家拥有的货币资本的界限。这个限制被信用制
度和与此相连的联合经营(例如股份公司)打破了。因此,货币市场的混乱会使这类企业陷于停
顿,而这类企业反过来也会引起货币市场的混乱。\pagescite[][396]{capital2} (Big
Note: 长期投资的严重问题,麻烦在于,这种投资在再生产图式中基本都被假定为不存在的。)

在\textbf{社会的生产的基础}上,必须确定前者按什么规模进行,才不致有损于后者。在社会的生产中,和在资本主义的生产中一样,在劳动期间较短的生产部门,工人将照旧只在较短时间内取走产品而不提供产品;在劳动期间长的生产部门,则在提供产品之前,在较长时间内不断取走产品。因此,这种情况是由各该劳动过程的物质条件造成的,而不是由这个过程的社会形式造成的。在社会的生产中,货币资本不再存在了。社会把劳动力和生产资料分配给不同的生产部门。生产者也许会得到纸的凭证,以此从社会的消费品储备中,取走一个与他们的劳动时间相当的量。这些凭证不是货币。它们是不流通的。\pagescite[][396-397]{capital2}(Note) 

\end{quotation}

威廉·哈维的血液循环理论,取代了盖仑统治了几个世纪的理论。在盖仑的理论中,心脏是血液生产的中心,血液从心脏流到各个器官,然后被耗尽。这是一个从生产流向消费的单方向模型。与之相比,威廉·哈维把心脏看作是使血液在全身持续循环的一个泵,血液通过与外部资源物质进行的代谢转化得到补充和净化。魁奈将哈维的概念应用到了政治经济学中,因此马克思——带着对流动性、持续性以及价值流通的强烈关注——明显被魁奈的思考方式吸引住了。

魁奈坚持认为只有农业部门才生产价值,工业生产是寄生于农业的。魁奈不敢批评凡尔赛的挥霍性消费或贵族的消费主义,因此他假称,农民和拥有土地的贵族都从事价值生产,从而掩饰了对农民榨取的剩余价值。这个“重农主义的”(主要在法国)观点与“重商主义”(当时主要在英国)形成了对比,后者将通过贸易积累金储备看作是经济政策的圣杯。

马克思反对这两个学派的思想。但是,考虑到当时法国所盛行的产业结构,魁奈的重农主义见解似乎有一些合理性,因为从农业中榨取的剩余价值,支持了主要为贵族消费(可以去参观一下魁奈生活的凡尔赛,看看那个时代所谓的工业通常生产什么东西)生产奢侈品(珠宝、高档衣服、陶器、地毯,等等)的手工业的产业结构(与马克思看到的工厂完全不同)。

在盖仑的理论中,治疗通常都是流血类的(理解为:紧缩),或者后面会伴随着输血(理解为:世界各地中央银行的量化宽松和流动性释放),这两者从马克思的理论观点来看,都是没有意义的。马克思理论中面临危机时的稳定政策,要求分析资本流动的持续性遇到的主要阻碍和障碍;同时采取措施解决所有障碍,努力使经济体系回到再生产图式所展示的也许可能的那种均衡——我强调了“也许”,因为这绝不是一个必然。(Big Big Note)

\begin{quotation}
年产品既包括补偿资本的那部分社会产品,即社会再生产,也包括归入消费基金的、由工人和资本家消费的那部分社会产品,就是说,既包括生产消费,也包括个人消费。这种消费包括资本家阶级和工人阶级的再生产(即维持),因而也包括总生产过程的资本主义性质的再生产。\pagescite[][435]{capital2} 

产品价值的一部分再转化为资本,另一部分进入资本家阶级和工人阶级的个人消费,这在表现为总资本的结果的产品价值本身内形成一个运动。这个运动不仅是价值补偿,而且是物质补偿,因而既要受社会产品的价值组成部分相互之间的比例的制约,又要受他们的使用价值,它们的物质形态的制约。\pagescite[][437-438]{capital2} 

\end{quotation}

但是这里存在一个难点。在再生产过程中,不仅价值要得到补偿,使用价值也要得到替换。
例如,如果要进行工人阶级的再生产的话,进入到劳动力的价值的特定使用价值必须以恰当
的数量生产出来。生产消费需要的特定的使用价值,也需要再生产出来。必须假定这些物质
要求与价值关系的必要再生产是相称的。但是,这并不会自动实现。在一个典型的里昂惕夫
投入产出体系的模型中,生产用于制造汽车引擎的钢铁所需的铁矿石和煤炭的数量,都可以
在一个投入产出矩阵中模型化为一个物质过程。这个模型是物质模型,是建立在使用价值基
础上的。伴随这些使用价值关系的资金流动是一个完全不同的问题。一个可能会平稳运行,
另一个可能就不会。我们以哪个为基础呢?马克思似乎两者都想选择。然而,接下来,社会
再生产过程的使用价值和物质模型,要么逐渐从我们的视野中消失了,要么被假定为会毫无
疑问地塑造价格和货币、价值流动。我们所得到的,在以使用价值初步定义了生产部类间的
一个宽泛的划分之后,是一个反映了使用价值的差异和要求的社会总资本运动的纯粹的价值
(货币)分析。价值和货币分析与物质上的使用价值流动之间的潜在矛盾,并没有得到考
察。

考虑到马克思的习惯——在《资本论》的开头就强调了\textbf{使用价值与交换价值的矛盾},
对这个矛盾的掩盖暗示了这里是产生危机的一个地方,也是我们应该寻找\textbf{再生产图
  式内的崩溃的地方。}事实上,这个分裂引起了从物质和使用价值方面解释再生产图式的学
者(一般指新李嘉图学派,包括皮耶罗·斯拉法)与从货币方面进行考察的学者(与凯恩斯主
义者类似)之间的矛盾。马克思认为要想合理地将这些图式用于社会协调,首先要废除货币
资本的作用,这暗示了图式中的基本矛盾就出现在这里——而从货币流动的立场而言,固定资
本形成的物质要求也阻碍了事物的平稳性和持续性这个事实,也暗示了一种源于与货币运动
相关的物质方面的矛盾形式。在某种意义上,我猜测,马克思可能会将新李嘉图学派与凯恩
斯主义后来在理解这个图式上的分裂,看作是将资本的内部矛盾在思想领域内外部化的经典
案例。当然,这些在文中都没有任何暗示。

\begin{quotation}
这种互相交换是通过货币流通来完成的。货币流通成为交换的中介,同时也使这种交换难于理解,然而它却具有决定性的重要意义,因为可变资本部分必须一再表现为货币形式,即表现为由货币形式转化为劳动力的货币资本,在整个社会范围内同时进行经营的一切生产部门,不论他们属于第一部类还是第二部类,可变资本都必须以货币形式来预付。\pagescite[][442-443]{capital2} 

\end{quotation}

很明显,“奢侈品生产中吸收的劳动力的数量……取决于资本家阶级的挥霍,取决于这个阶级的剩余价值的很大一部分转化为奢侈品”。但是,这对于经济形势是很敏感的。危机暂时减少了奢侈品消费,这会随之减少可变资本的花费——而这又反过来减少了对非奢侈品消费资料的一般需求。“在繁荣时期,特别是在欺诈盛行期间,情况正好相反”,这时充分就业的工人阶级得到了更高的工资,事实上可能会购买少量的奢侈品。

\begin{quotation}
认为危机是由于缺少有支付能力的消费或缺少有支付能力的消费者引起的,这纯粹是同义反复。除了需要救济的贫民的消费或“盗贼”的消费以外,资本主义制度只知道进行支付的消费。商品卖不出去,无非是找不到有支付能力的买者,也就是找不到消费者(因为购买商品归根结底是为了生产消费或个人消费)。但是,如果有人想使这个同义反复具有更深刻的论据的假象,说什么工人阶级从他们自己的产品中得到的那一部分太小了,只要他们从中得到较大的部分,即提高他们的工资,弊端就可以消除,那么,我们只需指出,危机每一次都恰好有这样一个时期做准备,在这个时期,工资会普遍提高,工人阶级实际上也会从供消费用的那部分年产品中得到较大的一份。按照这些具有健全而“简单”(!)的人类常识的骑士们的观点,这个时期反而把危机消除了。因此,看起来,资本主义生产包含着各种和善意或恶意无关的条件,这些条件只不过让工人阶级暂时享受一下相对的繁荣,而这种繁荣往往只是危机风暴的预兆。\pagescite[][456-457]{capital2} (Big Big Note)
\end{quotation}

乍一看,似乎很难调和这个叙述与第350页的脚注,那里是这样描述的,“商品资本的实现,
从而剩余价值的实现,不是受一般社会的消费需求的限制,而是受大多数人总是处于贫困状
态,而且必然总是处于贫困状态的那种社会的消费需求的限制”。事实上,马克思所说
的“同义反复”并没有否认有效需求的重要性,而只是强调,唯一起作用的需求是有支付能
力支持的需求。这再一次将我们的注意力引向,在不考虑对使用价值的真实需求的情况下,
货币(交换价值)是如何流通的。(Big Big Note)

马克思认为工人阶级的收入在危机之前会上升,我认为这个观点从经验上来看并不正确。尽
管20世纪70年代的危机是这个情况,但很难说2007—2008年爆发的危机符合这个观点。因此我
建议修改马克思认为有效需求与资本真正的内在矛盾无关的一般叙述,而认为有效需求不足
在特定情况下可能是那些内部矛盾的一种表现形式。但这只是我的个人观点,许多人当然会
不同意。

\begin{quotation}
  土地所有者(地租的承担者)、高利贷者(利息的承担者)等等,同时还有政府和它的官吏,食
  利者等等。这些家伙在产业资本家面前是作为买者出现的,而他们作为买者使产业资本家的
  商品转化为货币。他们各自也把"货币"投入流通,产业资本家则从他们手中得到这些货币。
  这时,人们总是忘记,他们最初得到并不断地重新得到的货币的来源是什么。(这个货币所
  代表的价值,最终必须来源于生产。但是,在我看来,它是来源于过去,还是预期来源于
  未来(例如,通过债务创造),似乎是一个在这里并没有得到充分阐述的重要区别
  )。\pagescite[][470]{capital2}

如果生产是社会的,而不是资本主义的,那么很明显,为了进行再生产,第一部类的这些产
品同样会不断地再作为生产资料在这个部类的各个生产部门之间进行分配,一部分直接留在
这些产品的生产部类,另一部分则转入其他生产场所,因此,在这个部类的不同生产场所之
间发生的一种不断往返的运动。\pagescite[][473]{capital2} (当然,这些就是里昂惕夫后来在其模型中所构建的投入产出关系。)

劳动力只是劳动者的财产(它将不断自行更新,自行再生产) ,而不是他的资本。劳动力是他为
了生存而能够不断出卖和必须不断出卖的唯一商品,它只有到了买者即资本家手中,才作为资
本(可变资本)起作用。\pagescite[][491]{capital2} 
\end{quotation}

第XII节:货币材料的再生产

\begin{quotation}
  既然归根结底必须把资本家阶级本身看做是投入流通的全部货币的源泉,每个资本家怎么能
  够从年产品中取出货币形式的剩余价值,也就是说,他从流通中取出的货币怎么能够比他技
  人的货币多呢?

  1.这里唯一必要的前提是:总要有足够的货币使年再生产量的不同要素进行交换。这个前
  提不会因为一部分商品价值由剩余价值构成而受影响。假如全部生产归工人自己所有,从而
  他们的剩余劳动只是为自己的而不是为资本家的剩余劳动,那么,流通的商品价值量也还是
  那么多,并且在其他条件不变的情况下,这个商品价值量的流通所需的货币量也还是那么多。
  所以,在这两个场合,\textbf{问题只是:这全部商品价值借以进行交换的货币从何而
    来?——而绝不是:剩余价值借以货币化的货币从何而来?}

  每一个开办新企业的资本家,在营业开始以后,都能把他为维持生活而用于消费资料的货币
  再捞回来,作为使他的剩余价值货币化的货币。但是一般说来,全部困难有下面两个来源:

  第一,如果我们只考察资本的流通和周转,从而把资本家也只是看做资本的人格化,不是看做
  资本主义的消费者和享受者,那么,我们固然看见他不断把剩余价值作为他的商品资本的组
  成部分投入流通,但从来看不见有货币作为收入的形式存在于他的手中,从来看不见他为了
  剩余价值的消费而把货币投入流通。

  第二,如果资本家阶级以收入的形态把一定货币额投入流通,那就好像他们为全部年产品的
  这一部分支付了一个等价物,因此,这一部分就好像不再代表剩余价值了。但是,代表剩余价
  值的剩余产晶,不需要资本家阶级花费分文。作为一个阶级,他们白白地占有和享受了这些
  剩余产品,而货币流通也不能使这件事有所改变。

  2. 每一个产业资本在开始的时候,都把用来购买全部固定资本组成部分的货币一次投入流
  通,但只是在若干年内逐渐通过出售其年产品再把它收回。所以,它最初投入流通的货币多
  于它从流通中取出的货币。总资本每一次要用实物更新时,这种现象都重复发生。

  3. 当其他资本家(撇开固定资本的支出不说)从流通中取出的货币多于他们为购买劳动力和
  流动要素而投入流通的货币时,生产金银的资本家(撇开作为原料使用的贵金属不说)只是把
  货币投入流通,而只从流通中取出商品。不变资本(损耗部分除外)、大部分可变资本和全部
  剩余价值(资本家自己手中积累的贮藏货币除外) ,都作为货币投入了流通。

  4. 一方面,固然有不是在当年生产的各种东西如地皮、房屋等等,其次,还有生产期间不止
  一年的各种产品如牲畜、木材、葡萄酒等等,都作为商品来流通。对于这种现象和其他现
  象,重要的是掌握住一点:除了直接流通所需要的货币额外,总有一定量货币处于潜在的、
  不执行职能的状态,一旦遇到某种推动就可以执行职能。这类产品的价值,往往也是一部分
  一部分地逐渐流通的,如同房屋的价值是在若干年内以租金的形式来流通的一样。\pagescite[][531-536]{capital2} 
  
\end{quotation}






%%% Local Variables:
%%% mode: latex
%%% TeX-master: "../main"
%%% End:
