\chapter{固定资本问题和扩大再生产(第二卷 第20—21章)}

\begin{quotation}

  诚然,在第一个场合(需用实物更新的固定资本比例大于只是需要补充损耗的固定资本比
  例),同一劳动可以靠提高劳动生产率、增加劳动量或增加劳动强度提供更多的产品,这
  样就可以弥补第一个场合的不足;但是发生这种变化的时候,总不免会有劳动和资本从
  第 I 部类的某个生产部门移动到另一个生产部门;并且,每一次这样的移动,都会引
  起\textbf{暂时的紊乱}。其次,第 I 部类(由于增加劳动量和劳动强度)不得不用较多的
  价值来交换第II部类的较少的价值,因而第 I 部类的产品就要跌价。

  在第二个场合则相反,第 I 部类必须压缩自己的生产,这对该部类的工人和资本家来说,
  意味着危机;或者第 I 部类提供的产品过剩,这对他们来说,又是危机。\textbf{这种过
    剩本身并不是什么祸害,而是利益;但在资本主义生产下,它却是祸害。}(Big Note)

  对外贸易既然不是单纯补偿各种要素(按价值说也是这样),它就只会把矛盾推人更广的范
  围,为这些矛盾开辟更广阔的活动场所。

  再生产的资本主义形式一旦废除,问题就归结如下:寿命已经完结因而要用实物补偿的那
  部分固定资本(这里是指在消费资料生产中执行职能的固定资本)的数量大小,是逐年不同的。
  ……在其他条
  件不变的前提下,消费资料年生产所需的原料、半成品和辅助材料的数量不会因此而减
  少;因此,生产资料的生产总额在一个场合必须增加,在另一个场合必须减少。这种情况,只
  有用不断的相对的生产过剩来补救;一方面要生产出超过直接需要的一定量固定资本;
  另一方面,特别是原料等等的储备也要超过每年的直接需要(这一点特别适用于生活资料)。
  这种生产过剩等于社会对它本身的再生产所必需的各种物质资料的控制。但是,在资本主义
  社会内部,这种生产过剩却是一个无政府状态的要素。\pagescite[][525-526]{capital2} 

\end{quotation}

简而言之,比例失调的危机是不可避免的。而且危机的深度和广度也难以确定。但马克思显
然断定即使两大部类间的交换正常进行,危机也会发生。

对此有两种解释。第一种观点认为固定资本流通造成的中断证实了无论通过何种方式,再生
产过程都无法顺利地进行,因此比例失调的危机自始至终既是普遍的又是无法避免的。第二
种解释则认为这种危机只是源自固定资本的流通。在这种情况下,通过固定资本流通的社会
化就可以避免这一危机。固定资本流通的社会化可以采取多种形式,从国家供给或干预到更
为激进的社会计划的形式,包括共产主义下去商品化的固定资本投资。但马克思并未排除,
如我们之前所看到的,在信用制度和股份公司的帮助下,也许资本家自己就能克服困难。后
一种解决方案的问题(如我们在第三卷中看到的)是它打开了潘多拉的魔盒,引发了以与固
定资本流通相关的货币运动为中心的投机性的繁荣和崩溃。尽管固定资本的一个问题得到了
解决,另一个更为严重的自发的金融危机问题又出现了。我们进一步考察这种情况。(Big Note)

但事实上,即使在资本主义制度下,再生产也并非一定是无政府状态的。许多长期固定资本
投资都是由国家承担的,因此存在合理的社会规划和设计的可能性。联合资本(股份公司)
的形成和“在资本主义生产方式中扬弃资本主义生产方式”开辟了新的、或多或少有些无政
府主义的协调模式(其不利方面是建成环境投资上的投机性繁荣,而其有利方面则是共同生
产集体的生产和消费资料)。(Note)

扩大对外贸易和建立世界市场或许能够暂时地缓和危机,但最终他们只是将资本的矛盾转移到更为广阔的地理范围内罢了。(Big Note)

“扩大的再生产”必须在商品形式下就已经存在了。所以“货币本身不是实际再生产的要
素”,\pagescite[][551]{capital2} 因为如果没有可用的剩余商品,储蓄的货币也就没有
用了。

贮藏(储蓄)的货币本身并不构成新的财富,但是它确实创造了“新的潜在货币资本”。但是,如果每个人都预期未来的扩张而进行贮藏,那现在这里就没有人会购买商品了,于是流通过程就中断了。滞销商品阻断了这一体系。只有在金的生产中,在金产品包含剩余产品,即剩余价值的承担者的时候,新的财富(可能的货币)才会被创造出来。

价值(包括剩余价值)生产和实现中的“实际平衡”要求“互相交换的商品具有同等的价值额”。

\begin{quotation}
(上述平衡要求买卖的价值额互相抵消)商品生产是资本主义生产的一般形式这个事实,已经
包含着在资本主义生产中货币不仅起流通手段的作用,而且也起货币资本的作用。同时又会产生这种生产方式所特有的、使交换从而也使再生产(或者是简单再生产,或者是扩大再生产)得以正常进行的某些条件,而这些条件转变为同样多的造成过程失常的条件,转变为同样多的危机的可能性;因为\textbf{在这种生产的自发形式中,平衡本身就是一种偶然现象。}\pagescite[][557]{capital2} (Big Note)

(第一部类Ia卖给Ib的工人的剩余价值)是用来生产Ic的生产资料,而不是用来生产IIc 的生
产资料的,是用来生产生产资料的生产资料,而不是用来生产消费资料的生产资料的。在简单
再生产的情况下,前提是第I 部类的全部剩余价值作为收入花掉,即用在第II部类的商品
上,所以,它只不过是由那种以自己的实物形式重新补偿不变资本IIc 的生产资料构成的。因
此,为了从简单再生产过渡到扩大再生产,第 I 部类的生产要能够少为第II部类制造不变资本
的要素,而相应地多为第 I 部类制造不变资本的要素。完成这种过渡往往不是没有困难的,但是,由于第I部类的有些产品可以作为生产资料在两个部类起作用这一事实,完成这种过渡就容易些。\pagescite[][559]{capital2} 

\end{quotation}

这的确是需要注意的:许多产品,最明显的例子就是能源了,无论在哪个部类都能充当生产
资料。但我认为这一论点的主旨已经造成了巨大的后果。\textbf{它支撑了在社会主义发展
  战略中长期占主导地位的观点,即必须优先考虑扩大第I部类的产出,如果必要的话可以牺
  牲消费品的生产。}其出发点是:发展重工业,投资于生产的固定资本和基础设施的固定资
本,并限制个人消费。最后当利用生产资料生产生产资料的能力达到一定水平后,开始关注
人民群众的消费需求。这是许多共产主义国家(苏联和中国)走过的典型路径。(Big Note)

然而,在后来的文本中,马克思拒绝了“积累是靠牺牲消费来进行的这种一般的说法”,认
为这种说法“不过是和资本主义生产的本质相互矛盾的一种幻想,因为这种幻想假定,资本
主义生产的目的和动机是消费,而不是剩余价值的攫取和资本化,即积
累”。\pagescite[][566]{capital2} 在一个纯粹的资本主义生产方式下,唯一的宗旨和目
标就是不断创造并巩固越来越大的剩余价值和增加资产阶级的财富、特权和权力。在这一生
产方式中,集中投资于为了生产生产资料而进行的生产资料的生产和忽略消费的策略是十分
合理的。\textbf{人民群众的消费状况与其直接利益无关。因此,在社会主义计划的实践中,
  这种由阶级强加的优先对第I部类进行投资的做法的继续存在应该受到质疑。}(Big
Note)

\begin{quotation}
已经在一个国家执行职能的生产资本(包括并入生产资本的劳动力,即剩余产品的创造者)越多,劳动的生产力,从而生产资料生产迅速扩大的技术手段越发展,因而,剩余产品的量无论是在价值方面或是价值借以体现的使用价值量方面越大。\pagescite[][560]{capital2} 

(货币)是绝对非生产的,它在这个形式上虽然和生产过程平行进行,但却处在生产过程之外。它是资本主义生产的一个死荷重。渴望利用这种作为潜在货币资本贮藏起来的剩余价值来取得利润和收入的企图,在信用制度和有价证券上找到了努力的目标。货币资本由此又以另一个形式对资本主义生产体系的进程和巨大的发展,产生了极大的影响。\pagescite[][561]{capital2} 

首先要假定最简单最原始形式的金属流通,因为,这样一来,流出和流回,差额的抵消,总之,在信用制度内表现为有意识的调节过程的一切因素,才会表现为独立于信用制度之外而存在的东西,事物才会以自然形式,而不是以后来所反映的形式表现出来。\pagescite[][563-564]{capital2} 

\end{quotation}

考虑到我们所了解的信用制度作为“阶级共有资本”所发挥的作用,不难发现信用制度远不是危机的根源。信用制度不但是消除货币流通障碍的主要机制,而且是更一般的避免危机或解决危机的主要机制,即使“整个机制的人为性质”会增加“扰乱正常的进程的机会”。因此,马克思在这些段落中频繁提到信用和银行体系就不足为奇了。但是它们的矛盾性质(像我们已经看到的那样)可能使马克思拒绝了任何将它们的作用纳入这里的系统性的尝试。考虑到马克思对作为“所有颠倒错乱形式之母”的信用制度的分析,我们将形成一个更清晰的观点:信用制度是怎样把我们从比例失调危机的煎锅中拯救出来,又把我们投入了金融和商业危机的烈火中的。

\begin{quotation}
第I部类形成潜在的追加货币资本(所以从第II部类的观点来看,就是消费不足);第II部类的商品储备搁置起来,不能再转化为生产资本(所以在第II部类出现相对的生产过剩);第I部类的货币资本过剩,第II部类的再生产不足。\pagescite[][566]{capital2} (Big Note)

第II部类作为总体来看,如上所述,比第 I 部类还有一个优点:它是劳动力的买者,同时又是
再向自己的工人出售商品的卖者。每一个工业国家都提供了十分明显的实例,证明可以怎样利
用这个优点,可以怎样在名义上支付正常的工资,事实上却一部分用实物工资制,一部分用伪造
通货的办法(也许还不受法律的处罚) ,把其中的一部分在不付相应的商品等价物的情况下再
夺回来,换句话说,再偷回来。例如,在英国和美国就是这样。(关于这一点,要列举若干恰当的
例子来加以说明。)但是,这种做法,正好是第 I点所讲的同样的做法,只不过伪装了一下,而
且是迂回曲折地进行的。因此,这种做法要和前一种做法一样被排除。这里讲的,是实际上支
付的而不是名义上支付的工资。\pagescite[][573]{capital2} 


\end{quotation}


近期,通过丧失抵押品赎回权而对美国数百万人的房屋的欺诈性剥夺,是这一观点在当代最
明显的例证。过去四十年左右我称之为掠夺式积累的全部政治活动也是如此。

增长过程产生了一个奇迹般的和谐:确实,增长和新的资本积累使得原本不平衡的地方变得和谐了。当然,马克思仔细地选择了他的数字和条件使这一结果能成立。但他因此也证明了和谐的资本积累的可能性(绝不是一种概率)。他使这一进程看起来好像能不断持续下去。

用代数术语说,第I部类的追加投资是 $C1+ \Delta C1+V1+ \Delta V1+M01$(最后一项是指追加投资用
于扩张后资本家阶级剩余的消费),第II部类是 $C2+ \Delta C2+V2+ \Delta V2+M02$。保持这一动力
机制顺利进行的均衡交换条件是 $C2+ \Delta C2=V1+ \Delta V1+M01$。

最大的问题是要达到这样的均衡状态到底需要什么。即怎样才能使两大部类之间的交换保持
正确的比例和比率,以至于不存在一个部类生产过剩而相对的另一个部类经历消费不足的情
况。显然,这一图式是不切实际的,并且马克思编造了这些数字以符合他的例子。 (Big Big Note)

当汽车产业在1914年引进五美元的八小时工作日后,亨利·福特派遣了一队社会工作者去教导工人如何审慎地和理性地消费。而所谓的“理性”是这样定义的——工人能为资本家生产的任何消费品“创造市场”。如何通过有组织的消费主义使消费的个别性理性化是一个具有挑战性的问题,马克思并没有讨论。


投资基金无法在两大部类之间流通,意味着没有实现部类间的利润率平均化的机制。由于这是与第三卷中考察的利润率下降相关的至关重要的一个方面,这里就有一个明显的理论问题,需要引起我们的关注。交换是按价值(而不是根据生产价格,就像第三卷前几章陈述的那样)进行的,且假定所有东西都实行等价交换。尽管货币资本的介入经常干扰这一问题,但流通中的货币方面并未完全纳入分析。一切都以一年为周期进行周转,固定资本形成和流通中最严重的问题大部分都被假定没了。通过地租、利息、商业利润和赋税等占有和剥削剩余价值的其他形式被忽略掉了。

但以这种方式建模的目的并不一定是要重现现实(尽管这种成功的模型或许会为最终重现现实打下基础)。马克思也许会这么说:这样的建模是为了凸显资本主义生产方式内部结构(这里指的是再生产)的核心关系——本质。那么这些图式揭示了什么呢?很简单,资本以货币、商品和生产资本三种循环的方式持续流动,通过这一持续流动的资本进行的资本积累的再生产注定是棘手的,因而具有危机的倾向,并且某种危机(由固定资本流动或更一般的比例失调引起的)或许只能以在其他地方(特别是在金融体系中)造成问题更大的危机为代价来解决。我喜欢指出在马克思的分析中,危机的趋势并没有得到解决,而是被转移了。(Big Big Note)

马克思未能回答自己的问题:“用于购买剩余产品的有效需求来自哪里?”这一问题是马克思在第17章遇到的,并试图在第20章和21章解决。这也是凯恩斯主义经济理论的中心问题。马克思的再生产图式似乎对于激发某种类型的凯恩斯主义思想和1930年以来的宏观经济增长模型,发挥了隐含的作用。

在凯恩斯看来,要实现和谐的经济增长,政府(或者许多政府和国际金融组织如IMF)必须实施恰当的财政政策和货币政策。受凯恩斯主义影响的其他经济学家表明,只有通过一条技术和组织变革(生产率c/v的演进)的独特路径,才能保持合理的比例。然而,实际的技术变革路径并不必然与平衡增长的需要相适应。偏离了平衡增长的需要的技术变革越多,比例失调的危机就越严重。(Big Big Note)

这些图式最早应用于苏联早期,那时一个叫费尔德曼的波兰经济学家为了制订经济发展的五年计划开始探索它们的功能。后来,马克思的图式被卡莱斯基(也是波兰人)和其他直接受凯恩斯理论影响的经济学家所借鉴,用以在资产阶级经济学内构建宏观经济增长模型和经济发展理论。埃弗塞·多马,广为人知的哈罗德—多马宏观经济增长模型的创始人之一,明确承认借用了马克思的图式。资产阶级经济学中所有的宏观经济增长模型都借用了这一遗产。如果他们更认真地对待马克思的图式的话,传统经济学家将避免许多麻烦,并且(或许)大约在七十年前就能开始对宏观经济模型和公共政策计划的研究。

可以这么说,目前,我们更关注反资本主义方案中的“联合劳动”的方面而不是在整个社会范围内合理配置劳动力的问题。部分是因为后者与共产主义甚至社会民主国家——目前人们不再轻易相信(在我看来这么做是对的)这个制度——的统治和镇压有关,部分是因为共产主义和社会民主计划的经验总的来说不尽人意(尽管说它彻底失败了是错误的)。但是,正如马克思在其他地方说的那样,我们不能以这些特别的污点“为借口,来排除理论上的困难”。(Big Note)


第II部类必须依靠第I部类的说法是毫无根据的。这一切之所以出现既是因为马克思独断的选择,也是因为两大部类间的关系的不平衡性质,后者是第I部类相对第II部类更高水平的贮藏的不同影响造成的。社会主义的过渡当然要彻底消除这些差异。这将完全有可能扭转两大部类的关系,使第I部类为第II部类服务。正如马克思所指出的,这在资本主义社会关系中是不可能的,因为资本的目标是积累资本,\textbf{而不是为了满足人民群众的生存和消费需求。}但是,可以肯定的是,社会主义或共产主义世界的目的将完全相反。

\chapter{反思}
\label{chap:fansi}

那么,我们能从建立了《资本论》第一卷和第二卷联系的“生产和实现的矛盾统一体”中,得出什么结论呢?

第二卷说明的是,资本循环的连续性不断地受到产生于实现过程中的限制和界限的威胁。这些界限不同于那些大多数马克思主义者熟悉的、在劳动力市场上和生产领域内的界限。但是,正如马克思在《政治经济学批判大纲》中强调的,对实现过程的各种各样的限制和界限,构成了一个对持续积累的动态的永久的威胁,并且频繁地引发重大危机。马克思甚至这样说道:“资本\textbf{不可遏制地追求的普遍性,}在资本本身性质上遇到了界限,这些界限在资本发展到一定阶段时,会使人们认识到资本本身就是这种趋势的最大限制,因而驱使人们利用资本本身来消灭资本。”\pagescite[][393]{karlvol46a} 

这些界限可以共同被看作是消费的界限以及“为积累而积累”规定的情况里的协调的界限。
然而消费这样一个过于粗略的范畴本身难以涵盖所有涉及到的问题。首先,区分生产性消费
(对原材料、能源、半成品和固定资本项目的消费)和最终消费(雇佣劳动者、资本家以
及“非生产阶级”对消费资料和奢侈品的购买和消费)是至关重要的。为创造更多的剩余价
值而对剩余价值进行的再投资行为,持续地扩大着生产性消费。但是,正如第二卷所说的,
生产性消费产生了对用于生产每一个独特的商品的特定的使用价值的需求。……与此同时,
新的需求和欲望(例如,最近的手机)的创造要求一个范围更广泛的商品投入,并且这些投
入必须在资本需要它们时已处在准备好的状态。正如马克思在他对再生产图式的研究中所阐
释的,虽然通过市场机制的供给,资本完成对所有这些需求的一种理性的协调不是不可能的,
然而没有太多扰乱而完成均衡增长的可能性无疑是十分低的,这从而预示着比例失调(使用
价值对一系列给定的生产过程的需求来说过多或过少了)的周期性危机。对均衡的波动性偏
离是一回事,由于这个或那个原因而对均衡的单调的偏离却是一个完全不同的命题。

但是,不仅物质形态的使用价值的流动需要协调,货币(和价值)流动也需要与均衡增长的
目标相一致。作为劳动的社会性的物质代表,货币与使用价值的性质毫不相关,货币量的流
通必须在这样一个环境中保持均衡:不同劳动分工中的货币协调极易发生严重错误。正如马
克思令人信服地指出的,这个问题不在于货币的总量可能相对不足,因为有许多能适应商品
交换增加的货币机制(比如,对计算货币的使用)。问题在于在一个错综复杂的交换模式中,
以不挫伤在每个交易点上实现利润的可能性的方式动员有效需求(有支付能力的需求)。

当任何错误发生时——它必然会发生,我们很可能目睹生产过剩危机(就像马克思在第二卷前
四章中所阐述的),其可能的表现形式是产生闲置的货币资本、闲置的生产能力和无法以可
获利的价格售出的过剩的商品。结果是一场资本贬值的危机。危机会持续多久和影响多深,
取决于各种实际情况。(Big Big Note)

然而,资本家之间围绕生产资料商品进行的复杂的交易,最终也是有条件的,它取决于商品在最终消费领域的实现。

我们迄今为止所概述的都没考虑到周转时间差异的影响(劳动期间、生产时间、流通时间)。特别是,我们根本没有关注恼人的固定资本流通问题(和类似的拥有很长生命周期的固定项目,比如包括在消费基金中的房屋)。第二卷不遗余力地理论再现了这些流通过程如何运行并塑造了资本积累的时空而没有——这是关键——求助于信用体系。正如第一卷所提到的,结果是更大数量的货币资本贮藏在不起作用的非生产性状态。持有货币储备的必要性在于应对周转时间差异和周期性的固定资本更新。资本主义生产体系越是错综复杂,就越需要贮藏货币。这种贮藏日益成为积累扩张的限制。这也使得创建一个适当的货币市场和成熟的信用体系愈发显得必要。结果是资本从根本上改变了自己存在的地点,这样“在生产过剩的普遍危机中,矛盾并不是出现在各种生产资本之间,而是出现在产业资本和借贷资本之间,即出现在直接包含在生产过程中的资本和在生产过程以外独立(相对独立)地作为货币出现的资本之间”。\pagescite[][397]{karlvol46a} 

在关于货币和金融的章节中,有很多对马克思的思想彻底重建的迹象——尽管放到他全部著作的背景中,这更应该被看成是一种对马克思最初思想的深度挖掘,而不是一种根本性的背离。这就是为什么我如此强调他对拜物教概念的重提和从这种拜物教概念到虚拟资本概念的转化的原因。马克思对货币资本的幻觉和幻象,对任何收入流资本化的幻想,以及对由此产生的过剩货币资本(IMF通常称为流动性过剩)的无限积累的鞭辟入里的揭露,使他坚信:“如果我们设想一个社会不是资本主义社会,而是共产主义社会,那么首先,货币资本会完全消失,因而,货币资本所引起的交易上的伪装也会消失。”

马克思预想的大纲:
\begin{quotation}
“一、 (1) 资本的一般概念。(2) 资本的特殊性:流动资本,固定资本。(资本作为生活资料,作为原料,作为劳动工具。(3) 资本作为货币。二、 (1) 资本的量。积累。(2) 用自身计量的资本。利润。利息。资本的价值:即同作为利息和利润的自身相区别的资本。(3) 资本的流通。(α)资本和资本相交换。资本和收入相交换。资本和价格。(β)资本的竞争。(γ)资本的积聚。三、 资本作为信用。四、 资本作为股份资本。五、 资本作为货币市场。六、 资本作为财富的源泉。资本家。在资本之后可以考察土地所有权。然后考察雇佣劳动。以所有这三者为前提,价格运动作为在流通的内在整体性上被规定的流通来进行考察。另一方面,三个阶级作为在生产的三种基本形式上和流通的各种前提上来看的生产。其次是国家。(国家和资产阶级社会。——赋税,或非生产阶级的存在。——国债。——人口。——国家对外:殖民地。对外贸易。汇率。货币作为国际铸币。——最后,世界市场。资产阶级社会越出国家的界限。危机。以交换价值为基础的生产方式和社会形态的解体。个人劳动实际转化为社会劳动以及相反的情况。”\pagescite[][219]{capital2} 

\end{quotation}

结果是留下了一个未完成的理论大厦,它对于资本主义可能采取的各种历史地理结构有着很强的解释力,但是对解释现实情形却不是这么有帮助的。这种现实情形包括对纯粹状态的偏离、纯粹状态的不完善以及对其进行的政治干预,以及那些金融的特殊性或者消费主义离奇的个别性等居于支配地位的情形。最重要的是,商业、金融危机和已确立的充满矛盾的资本运动规律之间的关系仍然没有得到充分发展。

我的博士论文研究了19世纪肯特郡种植啤酒花和水果的行为,研究中我发现19世纪40年代在
肯特郡中部的自耕农和西印度糖料种植园主之间形成了一个奇怪且不太可能的联盟。两个集
团都叫嚷着要减少对糖征收的关税。对于水果种植者来说,这意味着更便宜的糖料和对用来
制作果酱和蜜饯的水果的更大需求。这段时期是自由贸易思潮在英国最为盛行的时代,这个
思潮是由\textbf{希望通过低廉的食物价格来降低劳动力价值,}从而增加他们可以占有的剩
余价值的曼切斯特工厂主引导的。这股思潮主要关注面包的价格,工人们需要在面包上加点
东西。富含糖分的果酱(连同甜茶叶)可以及时为长时间工作的工人提供所需的能量。因而,
正如悉尼·明茨在他的著作《甜味与权力》中指出的,产业利益促进了工人对这种可以及时补
充能量的含糖食物的消费(因而茶歇时间一直是英国工人阶级生活中不可或缺的东西)。
《资本论》第一卷对与劳动力价值和强度(在关于“工作日”的章节)有关的贸易政策的分
析为这些工人阶级消费形式的推广确立了背景。

资本为这种看起来独特的文化习惯的形成和流传创造了某种“可能性条件”。住宅所有权和“美国梦”也是显而易见的例子。


















%%% Local Variables:
%%% mode: latex
%%% TeX-master: "../main"
%%% End:
