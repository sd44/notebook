\part{大萧条与斯大林主义}

\chapter{马克思主义经济学与大萧条}
\section{资本主义的最后一次危机}
大萧条是资本主义历史上最严重的一次经济危机。1930年美国的国民生产总值下降
了9.9\%,1931年又下降了7.7\%,1932年下降了14.8\%。1932年,德国和美国的工业生产不
及1929年的47\%(对其他资本主义强国来说,这次大崩溃不是全面的,但仍然是严重的)。马克
思主义经济学家对于这次大灾难提出了两个问题:它对\textbf{资本主义制度的未来和社会主
义的前途}意味着什么?如何用\textbf{马克思的危机理论}对它作出始终如一的解释?这两
个问题都不简单。马克思的资本主义危机理论并没有完整地得到概括,甚至没有形成有体系的
结构,而20世纪30年代的政治史又要求它有新的发展。其中最突出的现象就是\textbf{法西斯
主义}在德国的得势、\textbf{美国的“新政”以及国家干预}在其他方面的增强。它们再
次提出了国家在现代资本主义中的作用问题,以及民众支持倒退或改良、而不支持革命的基础
问题。

马克思主义理论家对这次危机的本质有着不同的看法。起初,至少是德国社会民主党的经济学
家并不感到兴奋。这次大萧条“\textbf{既不是‘扬格危机’(Young-crisis),也不是合乎理
性的危机,不是资本主义制度的彻底崩溃,也不是世界革命的来临}”。弗里茨·纳夫塔利
在1930年写道:“像每一次危机所表现的那样,这次危机\textbf{仅仅是具有历史特性的资本
主义制度的典型危机}”。1931年,卡尔·考茨基驳斥了德国社会民主党认为只有社会主义才
能结束这次危机的激进观点:“这种看法使我联想到那些生活在凉爽而湿润的夏天的人们,他
们认为这样的天气永远不会变暖,一个新冰川时代已经来临。”像先前所有的危机一样,复苏
是不可避免的;它会增强工人阶级的经济和政治力量,足以使人们相信当前的危机是这类危机
的最后一次。考茨基的结论是:“我们有充分的理由期待未来的繁荣将把人们带入一个持久富
裕的、持久安全的、并迅速把生产过程转向满足工人阶级需要的时代,我们应把这个时代的特
征归为无产阶级革命。”与这一结论相反,法西斯主义在德国得势,同时其他地方的经济复苏
缓慢而不彻底。但不管是考茨基还是德国社会民主党的主要理论家鲁道夫·希法亭,似乎都没
有修正关于萧条问题的自负的观点。

可是,大多数马克思主义者还是同意法兰克福研究所弗里德里克·波洛克的观点,他认为这次大
萧条比以前的危机更深刻、范围更广、持续时间更长,甚至与1873—1879年的危机相比也不逊
色,因此需要作出专门的解释。有少数人认为,资本主义发展中存在\textbf{“长波”现象}。
除了已经认同的\textbf{7—10年的商业周期外},还有\textbf{以半个世纪为周期的经济波动}。
这一长波的下降阶段与其上升阶段相比,其周期性的扩张更为脆弱,衰退更为强烈,持续时间更
长。这种观点最早是在第一次世界大战前由俄国马克思主义者\textbf{亚历山大·格尔方
德(即帕尔乌斯)}提出来的。在20世纪20年代,引起了德国经济学家沃尔夫和瓦格曼的注意,也
引起了苏联前社会革命党人Н.Д.康德拉季耶夫的注意,\textbf{50年的周期便以康德拉季耶
夫的名字命名}。大致情况是,上升阶段被确定为1851—1873年和1896—1914年(或1920年),相
应的下降阶段为1826—1850年、1873—1895年及1920年之后。弗里茨·纳夫塔利尝试用长波理论
解释这次大萧条,奥地利马克思主义者奥托·鲍威尔则更为自信地推进这一解释,他们都强调农
业价格下跌在下降阶段的作用。但是,这种假想几乎无人赞同,例如,\textbf{波洛克把长波理
论斥为“形而上学的”,认为它建立在孤立环境中的“不可靠的泛化”的基础上。}如果这
次危机既不表示一次周期性商业循环的下降趋势,也不表示这一衰退由于受到康德拉季耶夫长
波的影响而得到加强,那么期望实现积累率或产出和就业的持久复苏似乎是毫无根据的。所
以,人们可以认为资本主义注定要陷入持久的停滞,或者濒临经济崩溃。一般说来,严格意义上
的崩溃的观点——断言由于纯粹经济方面的理由,可获利的经济增长很快会变得不可
能——在1914年以前比其后得到更为广泛的认同,当然,亨里克·格罗斯曼在华尔街崩溃前夕曾
极为热情地鼓吹这种观点。可是,在\textbf{20世纪30年代中期,斯大林在第17次党代会上宣
布的代表苏联官方的观点称其为“一次特种的萧条”,会有“有限的复苏”。这就是说,它
既不是真正的复苏,也不是陷入低谷。}用\textbf{共产国际首席经济发言人尤金·瓦尔
加}的话来说,大萧条已经“引起资本主义制度的极大紊乱,开始了资本主义总危机的一个新
的、更高的阶段,\textbf{致使伴随危机的革命的客观条件日趋成熟}”。左翼反对党的前理
论家叶夫根尼·普列奥布拉任斯基在20世纪30年代一时得到人们的重新拥护,他同意这种说
法,即“假如它没有引起世界大战,或者没有被一场技术革命打断,在垄断制度下的总体性经济
危机肯定会突破其经济结构,并成为一场整个资本主义历史制度的总体社会危机”。在资本主
义世界的一些地区(如美国)可能重新走向经济高涨,但这是以牺牲其他国家(尤其是英国和法
国)为代价的。

越来越多的独立的马克思主义者形成了相似的看法。持不同意见的美国共产党人刘易斯·科里
在承认“\textbf{衰退不是崩溃}”的同时,仍然把大萧条看作是经济不稳定和停滞趋势不断
发展的证明,它预示着资本主义制度“\textbf{最后的、永久性的危机}”。

在德国,纳塔莉·莫斯科斯卡把日益严重的危机,看作是永久性危机的危险信号,而永久性危机
表明资本主义即将垮台。她的同胞弗里茨·斯滕伯格认为,有三个理由可以相信复苏将是极其
困难的:不再有新的海外市场可以利用;工薪者、文职雇员和独立的中产阶级前所未有的贫困
化,已经毁掉了一个重要的内部稳定器;大规模的失业和实际工资的削减,使国内需求的复苏异
常困难。

少数马克思主义者甚至更进了一步。托洛茨基和议会共产主义者保罗·马蒂克把这次大萧条看
作是资本主义的“\textbf{垂死挣扎}”。马蒂克在1933年世界产业工人联合会的纲领中写道:“在资本
主义的最后阶段,剩余价值第一次不再能支撑足够的工资水平和必要的积累”。这不仅提供了
无产阶级革命的客观经济条件,而且还促使人类在“共产主义和野蛮状态”之间作出抉择。托
洛茨基持相类似的观点,他把这次大萧条看作是他的以第一次世界大战为开端的“时代概
念”的有力证明,第一次世界大战表明欧洲生产力的进一步发展受到发达资本主义民族国家结
构的制约。他以列宁《帝国主义论》的语言风格表达了自己的观点,他写道:
\begin{quotation}
我们时代的垄断资本主义贯穿着一连串的危机。每一次危机都是一场灾难。人们必须通过
关税壁垒、通货膨胀、增加政府支出和债务等方式,从这些局部性灾难中摆脱出来,这反过
来又为新的、更深刻的、范围更广的危机创造了条件。对市场、原材料、殖民地的争夺使
战争灾难不可避免。总之,它们为革命性灾变创造了条件……毫无疑问,“\textbf{崩溃理
论}”战胜了“\textbf{和平发展理论}”。
\end{quotation}

显然,这里有一些模糊的地方。“资本主义总危机的新的、更高阶段”是什么意思?不断恶化
的不稳定性是否与停滞相一致?“永久性危机”究竟是什么?对这些问题,理论家们并没有作出
令人满意的回答。尤其是托洛茨基,他仰仗着辨术和武断的主张,也没有在经济上有力地论证
他的无产阶级革命新时代开始于1917年的观点(参见本书第一卷第13章)。琼·范·海詹劳特
在20世纪30年代的大部分时间里都是托洛茨基的秘书,他写道,可能“\textbf{从托洛茨基方
面看,他对命令性经济缺乏信心}”,斯滕伯格关于他1934年与托洛茨基的一些非正式讨论的
叙述,也证实了这种观点。


到目前为止,\textbf{最严谨的停滞理论}来自\textbf{保罗·斯威齐}(在1942年)的论述。斯
威齐强调群众性消费不足和投资机会减少的作用,这既归功于J.M.凯恩斯和阿尔文·汉森等自
由主义理论家,也归功于马克思,同时也利用了希法亭和列宁的许多成果。像许多马克思主义
著述者一样,斯威齐对经济的质上的变化比纯粹的量上的变化更感兴趣,他指出\textbf{资本
主义发展新阶段已经到来}。他\textbf{指责}希法亭和列宁在他们的“金融资本”概念中,错
误地把20世纪的资本主义从总体上概括为银行资本家统治工业的特定的、短暂的阶段(参见
第5章和本书第一卷第14章)。在与保罗·巴兰合作之前,斯威齐就提出“\textbf{垄断资
本}”的术语,概括这一时代的大公司、价格刚性、边际利润增长、投资不景气和销售成本
上升,以及为抵销消费不足引起的萧条趋势。纳塔莉·莫斯科斯卡的“晚期资本主义”的概念
与此极为相似。

与欧洲的马克思主义者相比,斯威齐的看法是“\textbf{经济主义的}”,他\textbf{较少关注
国家日益增长的经济作用}。奥托·鲍威尔根据德国当时的实践,把大萧条看作是新
的“\textbf{官僚主导的垄断资本主义}”的征兆。\textbf{鲍威尔}认为,欧洲在1932年以后
的有限的工业复苏,是建立在不断膨胀的军事开支基础上的,并伴随着国家对外贸易、国内价
格、工资构成的严格控制。倒闭的银行被国有化,政府通过特殊的就业手段获得了对作为“军
事储备队伍”的失业者的控制。鲍威尔相信,\textbf{回归到自由竞争和自由贸易的资本主义
是不可能的,国家经济权力的提升是不可逆转的。新的制度只提供了经济计划的可能性,而
不是现实性;它只可能抑制资本主义的基本经济矛盾,而不可能克服这些矛盾。一场新的世
界大战将不可避免。}

\textbf{尤金·瓦尔加}依据列宁关于战时资本主义的论述(并含蓄地依据布哈林的论述),得出
与鲍威尔相类似的结论:\textbf{人为地克服危机的主要后果(以及危机期间所有的资本主义
政策的后果)是国家对经济生活各个方面的干预,以有利于整个统治阶级,尤其是垄断资本和
大农场主。垄断资本利用其对国家机器的控制,影响国民收入体系向有利于他们的方向转
变,以各种方式和各种借口洗劫国库。“国家资本主义”趋势迅速增长。正如列宁对第一次
世界大战时的资本主义的称呼一样,垄断资本主义在某种程度上正向“战时国家垄断资本主
义”转变。}

“事实上,资本主义当前的处境非常类似于世界大战期间的景况……为下一次世界大战准备条
件,这越来越多地支配着国家所有的经济政策的核心。”这样,“战争国家垄断资本主义”的
突出特点,就在于经济政策中的军事因素占据支配地位,以及\textbf{国家经济权力的极大增
长}。像鲍威尔一样,瓦尔加坚持认为资本主义仍然具有深刻的矛盾。刘易斯·科里交替使
用“国家资本主义”、“垄断资本主义”和“国家垄断资本主义”这些术语,他否定该制度代
表一个新的社会秩序:
\begin{quotation}
与国家资本主义相联系的“国家计划”,从任何真正意义上说都不是计划,因为计划依赖于
私人利益关系间无政府状态的消亡;对产量实现计划控制只是对资本主义工业的零碎补
缀,是为了阻止这一制度的彻底崩溃和工人的暴动。国家资本主义是旧的和新的社会秩序这
对不可调合的矛盾的产物(对需要一种新的社会秩序的否定表述),它使不断衰退的资本主义
的矛盾更加恶化。
\end{quotation}

正如我们已经看到的那样,托洛茨基尽管与鲍威尔和瓦尔加存在其他一些分岐,但也存有相似
的观点。因为资本主义基本矛盾使生产力已经超出了民族国家的界限,任何通过国家经济管理
来解决危机的办法都是不可行的。如果不爆发无产阶级革命,新的帝国主义战争将不可避免。


弗里德里克·波洛克阐述的是一种完全不同的观点。\textbf{波洛克起初强调资本家会反对计
划,因为计划将使他们成为领取年金的人,暴露其寄生的本性,威胁其特权地位的合法性。但
是,波洛克不久就以罗斯福“国家产业复兴法案”为主要例证,开始强调国家经济作用在不
断增强。}他指出,不论是新的世界战争还是经济的彻底崩溃,都不能认为是不可避免的。假
如国家管制进一步增强、政治制度发生相应转变,\textbf{有计划的、稳定的资本主义经济完
全是有可能的。权力越来越集中在经济寡头手中。中间阶级将失去其独立性,而技术性失业
和劳动力市场的分割将挫败工人罢工,毁灭无产阶级抵抗的意志。}波洛克在1933年得出结
论:“\textbf{走向终结的不是资本主义,而仅仅是其自由主义阶段。无论在经济上、政治上
还是在文化上,大多数人在将来拥有越来越少的自由。}”议会型政府将让位于公民投票式
的独裁,从心理上对大众控制的机构不断发展,将导致国家机构独立于所有阶级,使它取得超脱
于社会之上的自主地位。


1941年之前,波洛克一直在探讨一种\textbf{新的国家资本主义}。在这种新的国家资本主义
中,市场不再控制生产和分配,经济规律已经消失。国家具有了以“虚拟市场”为工具来调
节经济生活、保证充分利用资源的功能。国家资本主义“意味着从经济主导的时代向实质上
的政治时代转变......权力动机取代了利润动机”;利润成为次要的,资本家降为靠年金生
活的人。波洛克区别了\textbf{新制度的两个变量}。在民主的国家资本主义中,\textbf{国
  家由民众掌握};而在极权主义国家中,它成为高层工商管理人员、国家主要官僚(包括军
队中的官僚)、(单一)政党的最高集团组成的\textbf{“新统治集团”的工具}。波洛克预期
后者将取得胜利。与乔治·奥威尔《1984》中提出的反乌托邦相似,\textbf{他认为首要
  的是战争和为战争做准备,因为该制度不能允许存在大规模失业,它还必须阻止民众生活
  水平的提高,唯恐增加工人闲暇时间会为“批判性思维提供更多的空间,革命精神会由此
  发展起来”。}在这些条件下,国家资本主义不会面临不可克服的经济障碍,因为“我们无
法发现任何内在的经济力量,以及能够阻止国家资本主义发挥作用的各种新的和老的‘经济
规律’……我们甚至可以说,在国家资本主义中,经济学作为一门社会科学已经失去其对
象。”

与此密切相关的是法西斯主义问题。\textbf{卡尔·考茨基从来不理解法西斯主义对大型工商
业的吸引力,在他看来,法西斯主义是经济衰退的非理性的产物,}它适合意大利而不适合德
国的环境。\textbf{波洛克则从另一个极端,把法西斯主义看作是新的社会秩序的一个范
例,在这里政治超过经济而处于至高无上的地位,国家已经开始统治经济生活。}这是法兰克
福学派大多数人的观点,但是波洛克同事中的少数人,尤其是赫伯特·马尔库塞和弗兰茨·诺伊
曼,则坚持正统马克思主义的观点,认为\textbf{法西斯主义是资本主义专制的一种形式}。在
颇有影响的《极权主义国家经济》一书中,诺伊曼认为“国家垄断资本主义”是一个矛盾的术
语,并把当时德国经济描述成\textbf{极权主义的垄断资本主义}。这一观点与保罗·斯威齐关
于纳粹德国本质的看法“从根本上说是一致的”,与希法亭的“极权主义的国家经济”概念极
其类似,这也类似于官方共产主义的观点。托洛茨基的分析更为精细,他认识到法西斯主义的
小资产阶级基础,但其结论则是一样的。正如季米特诺夫在共产国际第七次世界代表大会上指
出的,对大多数马克思主义者来说,法西斯主义是“最反动的、最具沙文主义的、最具帝国主
义的金融资本的公开的、恐怖的专制”。按刘易斯·科里的说法,“法西斯主义仅仅是一种旧
的社会秩序,是一种除弃了该社会结构以前所具有的进步性的社会秩序。”与波洛克相对立的
这种观点,并没有得到更好的阐述。

\section{大萧条的原因}

由于人们对大萧条的本质众说纷纭,因此没有一个得到普遍认同的关于大萧条原因的理论就毫
不奇怪了。\textbf{保罗·斯威齐}在1942年的一部著作中区分了\textbf{两类危机:由于利润
率下降而形成的危机和由于剩余价值实现的困难而导致的危机。}在斯威齐的论述
中,\textbf{每类危机又有两种明显的危机理论},共提出四种危机理论。\textbf{利润率下降
而形成的危机,是由于技术进步推动的资本有机构成提高快于剥削率增长造成的,如马克思
在《资本论》第3卷中所强调的;或者是由于资本积累率提高过快导致失业大军枯竭和工资
提高使剥削率下降造成的,如马克思在《资本论》第1卷和第3卷中所提示的。剩余价值实
现的困难而导致的危机,既可能源于不同生产部门的比例失调,也可能源于消费不足导致的
总需求不足。}这些理论不一定是相互排斥的:\textbf{斯威齐本人赞成第二种理论(用于解
释短期波动)和第四种理论(作为解释长期停滞的理论),而否定第一和第三种看法。然而,这
四种理论并不是完全孤立的},在以下的论述中将明显地看到这一点。但是,作为对萧条的马
克思主义的解释,它们之间的区别极为显著,其适用性也曾引起很大的争论,对此需要单独进行
讨论。

在1929年以前,对利润率下降是由于技术进步推动的资本有机构成提高快于剥削率增长造成的
论证,是普遍地被接受的,但是人们只是将其作为一种严格的危机理论。一些处在马克思主义
边缘的经济学家确实拒绝对其作出全面分析,认为资本家(在不存在工资上升的条件下)只有在
利润率提高时才会进行技术革新。在一些正统的马克思主义政治经济学家中,一般的做法是承
认马克思的分析正确地揭示了技术变革的后果,但根本不将其与资本主义危机联系起
来:\textbf{罗莎·卢森堡是最极端的的例子},当然还有其他一些人。希法亭确实把利润率下
降与危机的发生联系起来,但他没有强调这一点(在更大程度上,他用比例失调来解释危机)。
除了极少的例外,对利润率下降的这种忽视,在20世纪20年代重复地发生,如苏联经济学家如瓦
尔加,在随后的几十年里都一直坚持这种观点。理查德·戴曾查遍了关于大萧条的当代俄文文
献,仅仅发现一篇关于利润率下降的参考资料。1929年,随着\textbf{亨里克·格罗斯曼}关于
大崩溃理论著作的出版,上述情况发生了变化。格罗斯曼的数学模型以奥托·鲍威尔1913年的
一篇论文为基础,论证了技术进步需要资本家在一个周期内固定资本以10\%的速度增加,可变
资本仅以5\%的速度增长。他假定剥削率是固定的,很容易得出积累最终(在35个周期后)将由
于缺乏进行积累所必需筹措的剩余价值而无法实现的结论。在他的数学例子中,\textbf{当资
本家的消费降至零时,危机就爆发了,而这时的利润率(尽管下降)仍然是正的}。正因为如
此,\textbf{从严格意义上说,把格罗斯曼定为用利润率下降分析危机的理论家可能是不正确
的,但其分析的内在机制与《资本论》第3卷是一致的。}格罗斯曼受到了共产主义者、社会
民主党和独立的马克思主义者等的严厉批评,但在随后的几年里,资本有机构成提高开始在一
些马克思主义危机理论的分析中占有十分突出的位置。\textbf{首先进入该领域的是刘易斯·科
里},他引用官方统计数据证明,美国的有机构成不论在长期(从1849年到1914年)还是在大萧
条准备期都在\textbf{持续增长}:“1923—1929年间,制造业中的固定资本增长为可变资本
的4倍,即为24.4\%与5.7\%。”科里承认这可能会被剥削率的增长所抵销,利润率只有在非常
例外的情况下才不会下降。事实上,在1923年到1931年间,它从9.2\%下降为负数(当然,像我们
将看到的那样,科里把这种下降部分地归为实现的困难)。\textbf{在英国,首先是约翰·斯特
雷奇然后是莫里斯·多布,将利润率下降引入他们带有折衷主义色彩的危机理论中。同时,在
欧洲大陆,奥托·鲍威尔从战后资本主义生产迅速实现合理化中得出类似的结论。}

但是,也有不少人对这种观点提出异议。主流马克思主义中的著述者,第一次对马克思论述的
合理性提出质疑。颇为奇怪的是,没有一个人站出来反对科里,指出他的经验性论据是不可信
的,因为它涉及的是\textbf{价格而不是劳动价值量}。相反,这一规律从理论层面上受到质
疑。\textbf{莫斯科斯卡指出,仅仅是利润率下降这一事实还不能为《资本论》第3卷的分析
提供充足的论据,因为利润率下降可能是有极不相同的原因引起的(例如,由实现的困难造成
的)。}

她还批评马克思的结论,其根据是——如杜冈—巴拉诺夫斯基和博特凯维茨曾经提出的——与技
术进步相关的生产率的提高,将会导致利润率上升,除非实际工资的增长足以保持剥削率不
变。\textbf{她认为,这就使得谈论“剥削率上升的规律”同谈论“利润率下降规律”是一回
事。}马克思本人曾讨论过与利润率下降趋势相反的抵销趋势。\textbf{“固定资本要素的
贬值”会降低有机构成,同时必要劳动时间的降低会提高剥削率。}在斯威齐和更为游移不
定的多布看来,这些因素使人们不可能得出任何关于利润率长期趋势的确定论断,也导致人们
怀疑马克思提出的规律作为危机理论根据的合理性(参见本书第7章)。

\textbf{有更大把握的是,在经济繁荣的情况下,由于实际工资上升,剥削率可能下降,由此利
润率也可能下降。这种“过度积累”理论可以追溯到马克思,奥托·鲍威尔}于1913年使这一
理论重新焕发活力(参见本书第一卷第6章)。它沿着以下思路,为人们提供了一个清晰而合理
的周期波动模型:积累过程开始时,仍然存在大量失业后备军、较低的实际工资、较高的剥削
率和利润率;尽管固定资本积累比可变资本积累快得多,对劳动力的需求还是在扩大;失业后备
军缩小,工人的实际工资开始上升;实际工资很快就超过劳动生产率的增长,压低了剥削率,从
而降低了利润率;这又阻塞了投资,使积累陷入停滞;失业因此增加,实际工资下降,剥削率得到
恢复,利润率又上升了,从而使整个循环得以周而复始地进行。\textbf{莫斯科斯卡在她的第
一本著作中,曾经提出过这些概念,尽管她在以后的著作中放弃了这种观点。斯特雷奇以商
业周期高涨时实际工资上升,作为反对消费不足理论的决定性证据,而斯威齐和多布则将过
度积累看作是马克思危机理论的基本要素(当然,他们没有运用这一术语)。}


过度积累作为一种理论上的可能性是难以否认的,\textbf{但它经验上的可靠性则是另一回
事}。\textbf{科里和瓦尔加}引用美国的统计资料证明,当时失业率已经居于高位,实际工
资处于停滞状态,终于导致1929年的大危机。他们以此证明大萧条的基础是\textbf{消费不
足,而不是过度积累。}但是,在考察这一点——这是马克思主义关于衰退的最一般的解释——之
前,有必要阐述一下与实现理论不同的比例失调问题,因为它有时同消费不足理论难以区分。


\textbf{长期以来,马克思主义者认为资本主义生产的无政府状态是导致经济危机的主要原
因。}因为投资是由彼此独立的私人资本家分散进行的,没有一个总体计划指导他们的决策
或使他们彼此协调,个别部门的生产过剩几乎是不可避免的,并可能扩散到其他工业部
门,导致\textbf{总体上的生产过剩。但对引起这种情况的机制却从未作过专门的阐述。}马
克思主义经济学家和正统理论家一样没有关于有效需求和乘数过程的准确概念。但是,他们比
前凯恩斯主义正统理论家更接近这些概念,\textbf{比例失调还使他们认识到危机在资本主义
经济中的功能。危机有疏泄的作用,即消除不合理的投资,根据“价值规律”恢复恰当的比
例关系(即形成一种使所有产业部门都趋于获得平均利润率的资源配置)。}

奥托·鲍威尔和弗里茨·纳夫塔利在20世纪20年代阐述资本主义合理化时,提到过比例失调理论
的一些内容。\textbf{纳夫塔利和斯滕伯格则把它与垄断的增长联系起来},进一步阐述了这
一观点。他们认为,经济形势的进一步恶化,是由于对最初赢利的垄断部门的过度投资的刺激,以
及把必要调节的全部重担向竞争性部门转嫁。按照\textbf{弗里德里克·波洛克}的观点,国家
对陷入困境的垄断者提供援助,会进一步弱化该体制的自我调节能力。它在这里引入了一
种“\textbf{受保障的资本主义}”的形式,在这种形式中,\textbf{通过使不成功的企业遭受
毁灭性损失的竞争监管权力已经不再有效。}这是大萧条为什么难以驾驭的最重要原因之一。
波洛克相信,这也表明《通论》中存在一个严重缺陷。\textbf{凯恩斯对投资品和消费品部门
的总体分析,忽视了不同部门间的比例失调问题,进而对资本主义经济混乱的规律作出错误
判断。}

\textbf{普列奥布拉任斯基对危机的解释,是以比例失调在资本主义新的垄断阶段变得越来越
严重这一命题为根据的。}在竞争条件下,通过价格机制的激励,资源能够迅速地从一个经济
部门转移到另一个经济部门,总产量水平迅速地对总需求的增长作出反应。但是,\textbf{在
垄断资本主义条件下,资源流动受阻,需求的变动带来非对称的后果}:需求减少时产量下降,而
需求增加却使物价而不是产量上升。当需求增加时,投资增长为何特别缓慢的原因是多方面
的。\textbf{垄断资本家掌握着大量的剩余生产能力,这会阻碍新的投资。进入市场壁垒使新
企业的创建更为困难。消除无效率的生产单位需要更长的时间,保守的官僚主义的工会运动
具有越来越严重的缺陷,使得创新的动力减弱,而这种动力原来是由工资的不断增加来提供
的。所以,与资本主义历史上的自由竞争阶段相比,危机更为严重,复苏更为缓慢。甚至由非
资本主义需求增长(普列奥布拉任斯基在这里引用的是罗莎·卢森堡的观点)所带来的暂时喘
息的余地,在垄断资本主义条件下也将不复存在。}

但是,自20世纪初以来,比例失调理论在马克思主义经济学家中引起激烈的争论,它似乎带有修
正主义色彩。\textbf{如果危机源于无政府的个人主义,它们应该可能通过资本家本身的集体
积累的计划得到克服,这种计划或者是私人性质的,或者是与政府相联系的。资本主义在很
大程度上摆脱危机的这一前景首先吸引了爱德华·伯恩施坦,继而吸引了1914年之后“有组
织的资本主义”的理论家,同时也抵制了从卢森堡到列宁和瓦尔加等革命的马克思主义者的
观点。}与此有些不一致的是,纳塔莉·莫斯科斯卡反对用比例失调说明萧条的理论,反对该
理论关于在垄断资本条件下个别计划增强和通过价格机制继续发挥有效规制作用的观点;她甚
至引证哈耶克关于后者的观点。莫斯科斯卡对其所谓的新、旧比例失调理论进行了比较。新
的理论强调工资与利润之间、消费与储蓄之间、进而投资品产业与消费品产业之间的不平
衡。“假如说旧的理论是在生产中寻找危机产生的原因,那么新的理论则转向分
配……\textbf{低工资与高利润削弱了消费能力,促进了积累。”对莫斯科斯卡来说,比例失
调意味着消费不足。}


莫斯科斯卡的危机理论可以概括如下:\textbf{资本家之间的竞争要么绝对地降低实际工
资,要么降低相对于利润的实际工资。而剥削率随之上升,资本家越来越难以寻找到实现其
产品中包含的剩余价值的充分的消费需求。只要实际工资增长远远落后于劳动生产率,劳
动力市场就会失衡。这就会导致商品市场的失衡,进而导致生产与消费之间的比例失调进一步
发展。结果是,流通成本增加,因为资本家无助地试图运用各种促销方式来创造需求。}

在垄断资本主义条件下,即使货币工资保持不变,由于价格竞争的约束,消费不足的压力也会随
着剥削率的上升而越来越强。\textbf{莫斯科斯卡否认这样一种观点,即消费不足会导致停滞
而不是经济活动的剧烈波动。}她认为,在每次循环的下降阶段,生产针对消费所作的调
整,是通过\textbf{三种内在稳定器的作用暂时取得的:维持失业工人和非资产阶级的消费支
出;固定成本越来越重要,这意味着生产率比收入支出下降更快;强化营销和非生产性工人工
资的相应增长。}

奥托·鲍威尔提出了一个更为成熟的消费不足理论,清晰地阐述了莫斯科斯卡分析中许多不甚
明了的地方。\textbf{鲍威尔首先从资本家的储蓄倾向大大高于工人阶级的储蓄倾向(它趋于
零)的假定开始:}
\begin{quotation}
群众消费的发展与社会生产发展之间的关系,取决于\textbf{工资与利润之间的比例关
系}。\textbf{工资总量增长越慢,利润总量就增长越快;那么群众消费也就增长越
慢,社会生产部门也就增长越快。}
\end{quotation}
因此,所有的事情都以工资与利润的相对份额为转移。如果利润增长快于工资增长,剥削率就
会上升,储蓄就会比产量增长得更为迅速(而消费的增长则更为缓慢)。

鲍威尔在马克思主义著述者的第一个消费不足理论的数学模型中,把这些观点加以形式
化。\textbf{他把积累定义为净产出与消费之间的差额。}如果剥削率上升,积累将会加速增
长。这会带来生产能力的真正提高。但是,\textbf{要求达到的生产能力的提高,通过一个系
数(它“取决于技术发展的普遍提高程度”)与消费的增长密切相关,而对凯恩斯主义宏观理
论中的加速系数反应迟缓。}鲍威尔得出的结论是,只要消费的增长落后于收入的增长,实际
积累就将超过必要的积累,因此“社会的固定资本的增长就会超过满足消费增长而进行的生产
所需的固定资本;消费落后于生产能力”,最终是一场消费不足危机的爆发。

共产党的理论家从\textbf{本本的权威和政治的权宜之计}出发,对这些观点作了批判。首先,他
们认为,马克思本人曾把关于危机的消费不足理论,斥为“\textbf{纯粹是同义反复}”,并指
出\textbf{实际工资的增长通常发生在生产周期的高涨阶段,并引起过度积累的危机。}其次,他
们把消费不足理论斥为反革命的异端,因为\textbf{它使人们相信危机可以通过提高工资而不
消灭资本主义这一改良主义的方式得到克服}。正如一本党的教科书中所宣称的,它是“适
应社会民主党的实践任务”的教义。与此相反,人们认识到马克思自己的观点就有不明确的地
方。

他同持消费不足理论的人一样,否定了“萨伊定律”,相信在“资本主义条件下消费落后于生
产发展”就会发生危机,并以此阐述了“\textbf{生产的社会性与资本主义占有方式之间的矛
盾}”。所以,\textbf{保罗·斯威齐}这样一个同情共产主义运动的无党派学者,完全赞同鲍
威尔的模型,\textbf{科里和瓦尔加}则以类似的但略微粗糙的方式赞同这样的观
点。\textbf{科里把技术进步、资本有机构成的提高、劳动生产率的提高与剥削率的不断提
高联系起来},即:
\begin{quotation}
剥削率的提高限制了工人的购买力和消费。\textbf{工资总是落后于利润,工资从相对量上
看也落后于产出和利润。}这大大限制了市场的发展,使生产资料和消费资料失衡,从而推
动周期性危机和崩溃。
\end{quotation}
\textbf{瓦尔加坚持认为},资本有机构成提高所造成的失业上升,导致了\textbf{工人阶级的
绝对贫困},而不仅仅是相对贫困:“因此,资本主义社会消费能力的相对下降限制了生产资
料的销售……社会消费能力的限制,人民大众所处的无产者的地位,是所有真正的生产过剩危
机的原因”。这是苏联官方对大萧条的解释。

\section{结论}

瓦尔加是非常坚定的,其后只有莫斯科斯卡和他一样信奉消费不足理论,排除马克思关于危机
理论的其他所有变体。\textbf{多数学者}把在前一部分讨论的四个因素中的两个或两个以上
的因素\textbf{结合起来},莫斯科斯卡甚至在其1929年的一本书中,毫不费力地从消费不足论
滑向我们称作的“过度积累”论,但却使用消费不足这一术语对这两种分析进行了颇为艰难的
综合。鲍威尔和科里把具有《资本论》第3卷风格的利润率下降理论的内容加入到他们的消费
不足理论中,而弗里茨·纳夫塔利则利用了消费不足论和比例失调论。多布和斯威齐是同等的
折衷主义者。斯威齐仅仅彻底否定了利润率下降的解释,多布在其精妙而又多少令人迷惑的论
述中,把所有四种危机理论揉合在一起。

一些著述者试图进行明晰彻底的综合。\textbf{斯威齐的综合也许是最有条理的,他把作为短
期波动根源的过度积累,同作为长期停滞根本原因的消费不足合起来进行说明,普列奥布拉
任斯基}的贡献也给人以深刻的印象,他\textbf{超越了含混的比例失调理论,构建了一个垄
断(与自由竞争相比较)资本主义条件下的令人信服的投资不足理论。}另一个极端来自斯特
雷奇的令人迷惑的结论:“当代资本主义的困境的实质是消费者的需求一方面太低,以致于不
能提供一个市场;另一方面又太高,以致于无法进行有利可图的生产。这就是马克思所说的资
本主义的矛盾。”人们不可能接受有效需求太高同时又太低的观点。但是,可能发生的首先是
需求在通常情况下要么太高,要么太低;其次需求均衡水平是一把“双刃剑”,它会不断地自动
偏向任何一方。\textbf{奥托·鲍威尔在阐述官僚统治下的垄断资本主义时代的政府干预
时,更加接近一种宏观经济不稳定的模型。}鲍威尔认为,国家不可能克服资本主义的基本矛
盾。\textbf{如果国家干预是为了提高剥削率,消费不足就会发生;如果国家的活动旨在降低
剥削率,利润率的降低就会引发危机。}但是,鲍威尔并没有试图把这种直觉加以形式化,把
后人才能取得的理论成果加进他的著作是不明智的,也是不必要的。


尽管马克思主义经济学家们存在着分歧和疑虑,但是与新古典理论家相比,他们无论在概括性
地论述两次世界大战期间的发展问题方面,还是在专门阐述大萧条的具体问题方面,都做得相
当出色。其原因不难理解。\textbf{正统的经济学家还没有整体的社会理论,这种理论超
越19世纪流行于世的自由主义理论,因此世界大战和法西斯主义完全处于他们的范式之
外。}与此相反,马克思主义者却具有理解这些现象的较为成熟的理论框架。

在两次世界大战期间,对商业周期的正统的非马克思主义分析,在深度和范围上取得了很大进
展。但是,\textbf{非马克思主义的正统派拘泥于有效需求短缺不可能发生的观点。那些赞成
举办公共工程的经济学家,由于其直观性的观点而备受人们推崇,但在1936年之前他们的这
些直观性观点还没有一套严密的理论阐述。}马克思主义者所面临的分析上的障碍要小得多。
危机是其整体观点中的一个部分,马克思经济学提供了许多难以解释的问题的出发点。如果说
马克思主义也存在问题,那就是他们与非马克思主义所面临的问题正好相反:\textbf{理论丰
富的困境}。分歧也由此产生。

但是,马克思主义者对大萧条的分析论证是有缺憾的,其根本原因类似于资产阶级经济学对大
萧条的分析:他们\textbf{缺乏一个完整的有效需求理论。马克思本人既运用需求这一新古典
经济学的概念,又对其进行了出色的批判,表明它并不具有广泛的适用性。他还构建了工人
对消费品需求的数量限制的概念,但是他既没有把这一概念拓展到其他类型的需求上,也没
有始终如一地接受任何形式的消费不足观点。20世纪30年的马克思主义者未能克服这些缺
陷,但是居主导地位的比例失调论和消费不足论表明,他们正朝这个方向努力。}从这种意义
上讲,琼·罗宾逊(在1942年)得出的结论是正确的:“马克思给自己提出发现资本主义运动规律
的任务,但并没有概述它的细节问题,如果人们还希望推动经济学前进,就必须运用学术的方法
来解决马克思所提出的问题。”当然,罗宾逊所说的“学术的方法”就是\textbf{凯恩斯的方
法}。

本章所提到的战后对其他一些问题的讨论,将在本书第二篇进行阐述。第4章将概述马克思主
义经济学家试图理解现代资本主义结构变化的方法。第6章阐述有关垄断资本主义和消费不足
的一些理论,因为这些理论经过修正后用来解释1945年后出现的“长期繁荣”。第7章对利润
率下降理论的发展作了评析,最后第8章分析马克思主义关于军事支出重要性的有关文献。其
中的一些问题在本书第三篇中还会出现,在那里它们经过修正用于解释资本主义在第三世界明
显没有得到发展的事实,在第16章又被用于解释20世纪70年代所谓的“第二次衰退”。但
是,我们首先还是转向马克思主义政治经济学在20世纪30年代所关注第1章马克思主义经济学
与大萧条的其他一些主要问题:苏联生产方式的本质以及斯大林主义者宣称它所代表的真正的
社会主义


\chapter{斯大林的政治经济学}
\section{斯大林体制}

\textbf{正当大萧条席卷西方资本主义经济时,苏联经济的增长率急剧提高。}1928至1937年
间,工业生产增长3倍,从不足国民生产的$\sfrac{1}{3}$发展到接近$\sfrac{1}{2}$。1937至1953年间,工业再次增长两
倍多,到斯大林去世时接近于总产出的60\%。\textbf{这一切只有通过大规模投资才会成为可
  能。}每年平均有20\%以上的产出用于积累,\textbf{工人与农民的消费显著下降};工人的
实际工资直到20世纪50年代初才再度达到1928年的水平,而农民的生活水平下降得更多,需要
更长的时间恢复到原有水平。

随着向前的“大飞跃”,一场“自上而下的革命”建立起一种新的生产方式。始于1927年末
的\textbf{谷物危机,使战时共产主义的一些收购措施得以恢复},1928年后继续推出旨
在\textbf{“消灭富农阶级”和迅速实现农业集体化}的政策。这在1934年前取得了实质性进
展,直到30年代末才真正完成。\textbf{随之带来了对农产品分配的必要的政治控制,清除了
  曾在20年代大部分时间中实施的对工业高速扩张的约束。}快速的工业化又反过来使30年代
发生了意义深远的\textbf{农业机械化}。\textbf{在集体化的同时,非农部门再度实现国有
  化,行政分配在很大程度上取代了市场交易。}一系列的五年计划成为大部分生产的指导力
量。


但是,这种“命令经济”只是不完善的计划。特别是在早期,弥漫着唯意志论和战争意识;在这
里,“没有任何堡垒…… 布尔什维克不能占领”。\textbf{产量目标的制定不受生产能力的
  制约,超额完成任务的压力极为强烈,“突击战术”被应用于具体的规划。}比例和平衡观念
被斥为“资产阶级偏见”,科学的标准由于不能充分表达“群众的热情”而被视为不合时宜。
斯大林以帝国主义进攻临近的威胁和国内阶级斗争的激化,为经济的高速度辩护。\textbf{无
  法实现产量目标的现象集中于非优先发展的部门,因为如果重工业产量跌到计划目标以
  下,就把非优先发展部门的投入转到重工业。}

社会政策的制定是为了促进经济转轨和政治统治。工人对剩余工业品的支配权完全被取消,取
而代之的是“\textbf{一人管理}”,新的劳动纪律也是严厉的。\textbf{不仅货币收入的不
  平等在扩大,而且通过对较高层统治集团提供特殊的零售渠道和社会服务、在军队中恢复军
  衔以及提倡传统的家庭关系,使不平等扩大了。刑法体制失去其教育性,“劳动改造
  营”(即“古拉格”)成为经济体制的一个不可或缺的组成部分——其中的大部分原因是要镇
  压农民对集体化的抵制、清洗“反革命分子”和对盗窃国家财产的人实施严酷的惩罚。内
  部安全机构在规模上和权力上膨胀起来,以适应对“古拉格”的管理,以保证对遭受严重剥
  夺的人民全面监视的需要。}

被镇压的人的数量是令人惊愕的。1929至1937年间,1100万农民在清除富农和农业集体化中死
去。在1936至1938年的“大清洗”中,500万人由于政治原因被逮捕,其中至少有100万人被枪
决,200万人死于狱中。\textbf{在1929年以后的10年间,斯大林成功地清除了所有的“老布尔
  什维克”,其中包括许多斯大林主义者和20年代的各种反对派。一套内部通行证制度使农民
  受到控制,}使他们仅仅依靠私人小块土地生产维持生计,以致使农业生产关系近似于农奴制。
到20世纪30年代末,监狱中关押的人超过1000万,毫不夸张地说,他们的生活条件与奴隶差不
多。

新经济政策时期的文化多样性不复存在。艺术、自然科学和马克思主义本身都服从于党的路
线。对斯大林的崇拜与对列宁的崇拜结合在一起,并最终超过了对后者的崇拜。斯大林被推崇
为马克思主义理论领袖、万能的天才、人类进步智慧的主要源泉。大清洗毁掉了大部分20年
代圆满完成任务的经济研究机构和计划机关的人员。这是从所谓的“\textbf{遗传学派}”开
始的,\textbf{这个学派指出了过去经济发展的人为的强制性、市场关系的重要性以及急速的
  社会结构变革的代价。}其反对者是一批目的论者,这些人强调运用行政措施的国家计划在
转型中的潜力。最初他们并没有受到严厉处理,但是该政权在30年代后期相对稳定后,持这样
观点的人同样遭到被镇压的厄运。数量经济学在苏联受到猛烈的批判,甚至像瓦尔加这样一些
忠诚的理论家,如果他们的著作被认为是修正主义的,也会受到冷落。

尽管政府企图实行全面监控,一些社会团体和学术团体并没有全部被消灭。大多数后来成为马
克思列宁主义正统派的学术团体来自于下面,对立的见解只有得到官方裁定后,才能成为国家
意识形态的一部分。马克思主义政治经济学同样如此,对它的研究也从未完全废止。即使
是1936年后的“大清洗”,也有其“大众化的”一面,部分地摆脱了斯大林主义权力中心为其
设定的发展轨迹。来自上层的社会、经济和文化上的转变,常常得到下面热情的支持,也常常
受到一系列空想的鼓舞。

这些情绪在\textbf{斯大林1935年宣称苏联已经实现社会主义}时得到认可,一年以后的新宪
法再次加以肯定,这部宪法宣称是“世界上最民主的法律”。上述二者都不是偶然发生的;斯
大林在其理论著作中对社会主义建设道路、社会主义建设的成功及其巩固都作了阐述。到目
前为止,我们阐述的是苏联对其社会成就的自我意识;虽然没有论述与官方结论同时存在的反
对派的观点,但还是为我们提供了对其含义与重大意义的一种新的理解。

\section{斯大林的政治经济学}
相当奇怪的是,斯大林从来不宣称自己对马克思主义理论作出了创造性的贡献。\textbf{他提
  出的总是真正的列宁主义的思想,}这些思想并不需要进一步创新。而且斯大林还不断地以
执行列宁的“嘱托”为其1923年后的政策辨护;而列宁的“嘱托”正是斯大林所致力的布尔什
维克主义的事业,这隐含着他本人的思想是得到列宁肯定的。这种忠诚能说明斯大林某些著作
中明显存在的教条主义迹象的原因,而他的地下职业革命者的背景无疑有助于树立这样一种形
象。同样奇怪的是,\textbf{尽管在斯大林统治下发生过巨大的经济变革,但他的政治经济学
  却偏重于“政治”,而不是“经济”。}苏联其他的马克思主义者也是如此,与德国马克思主
义者(参见本书第一篇)相比,他们的经济学总是集中在政治问题上。在斯大林已经出版的著作
中尤其如此,它们实际上并没有讨论中央计划的方法、没有讨论投资评估和技术选择的标准、
没有讨论应对不确定因素和非均衡因素的措施,也没有讨论信息传递问题和非理性的价格评估
问题。许多苏联经济学家当然意识到由命令经济产生的困难,被迫采取一些措施,但是,他们工
作的极权主义氛围自然而然地阻碍了深入的研究和公开的讨论。西方马克思主义经济学家如
莫里斯·多布,确实出版了有关这些问题的极有价值的著作,而其他一些对斯大林主义没有什么
好感的马克思主义者(以及强烈反对马克思主义的经济学家)突出了斯大林主义生产方式的深
刻矛盾。但是,因为这些观点直到斯大林去世后才对苏联产生影响,并且常常成为对马克思主
义进行更为广泛的自由主义攻击的一部分,或者是斯大林的马克思主义的对手的观点,所以把
对这些观点的考察推到后面第3章和第18章,\textbf{本章将集中讨论斯大林的马克思主义观。}

斯大林对列宁主义的解释既不是无懈可击,也不是他自己的创造。他广泛地吸收其他人的观
点。\textbf{在新经济政策期间,他大量地引用布哈林的观点;1928年与右派决裂之前,他又利
  用了源于左翼反对派的一些思想。}他的《列宁主义问题》比1924年以后的其他著作更能确
立他的理论地位,但这本书可能归功于\textbf{卡冈诺维奇梳理列宁思想}的尝试。斯大林甚
至\textbf{会采用托洛茨基对俄国革命事业的某些最初看法},并把这些看法强加给列宁。因
此,斯大林的马克思主义观是\textbf{高度折衷主义}的,它既掺杂着列宁的思想(这是斯大林
常常忽视的),又揉合了被他击败的反对派的看法,他把理论仅仅作为获取个人权力的斗争工具。
但是,还是不应该轻易否定斯大林的理论探索。他重视理论的原因同其他的马克思主义者是完
全一样的。理论为人们提供“确定行为方针的能力,认识周围环境的内部联系……只有它能帮
助实践认清…… 阶级如何运动并朝着什么方向运动”。另外,在20世纪20年代早期,列宁本人
思想的地位已经显著衰落,斯大林把列宁的思想重新编纂为一个整体,这就为适应由一连串危
机所引起的政策上的剧烈变化,提供了充分有力的逻辑上的一致性。\textbf{斯大林的绝大部
  分著作是通俗易懂的,显示了他对列宁著作的深刻理解。他对包括托洛茨基在内的对手们的
  批判,常常击中其真正的弱点,时时展现斯大林论战的技巧。尽管斯大林与不断革命论的理
  论家相比是理论上的矮子,但他绝不是托洛茨基所说的无知之辈。}

尽管他在不同的时期明显地改变其政策,但关于他在理论上偏离了通常所理解的列宁主义、马
克思主义的原则,或者其观点与列宁本人或与布哈林、季诺维也夫、加米涅夫以及其他布尔什
维克领导人的理论相比是否更为摇摆不定,这些都是可以讨论的。即使1953年斯大林去世
后,斯大林对列宁主义的解释仍然是\textbf{苏联意识形态的基础},当然,他的后继者在其他
领域实行了“非斯大林化”。

斯大林具有的俄国党的国内组织者的背景,使他与西欧的流亡者集团相隔离,这有助于理解与
他的名字相联系的马克思主义为什么出现得较晚。\textbf{它作为一个明确的理论结构
  在1924年才开始出现,这时,世界革命遭受一连串的失败,列宁去世后又出现了接班人的问
  题。}所有的问题都集中在列宁主义的定义上,即“帝国主义时代的马克思主义”上。斯大
林批评那些强调列宁的思想完全基于俄国情况的布尔什维克,也批评那些认为列宁的原创新的
贡献就在于把农民问题引入革命理论的布尔什维克。根据斯大林的观点,列宁主义从整体上对
马克思主义政治经济学作了真正的发展。斯大林认为,由于垄断资本主义是新近的现象,马克
思和恩格斯都没有意识到由帝国主义所引起的变化。只有列宁才解决了经典马克思主义需要
解决的问题,这些问题关系到全世界的工人阶级,而不仅仅关系到俄国的无产阶级。

斯大林对解决列宁的帝国主义理论中(参见本书第一卷第13章)中含糊不清之处和存在的缺陷,没
有做出任何贡献。但从对斯大林的非批判的观点来看,\textbf{他极有理由地将资本主义世界
  体系的不平衡发展理论置于这一学说的核心。由此得出,“重新瓜分世界的战争”意味着资
  本主义在总体上已经成为一种与全球性危机相联结的倒退力量。}但是,\textbf{斯大林认
  为,正是由于这种不平衡,不可能指望资本主义在所有地点同时崩溃,而马克思和恩格斯在资
  本主义发展到帝国主义阶段以前的著述中认为是可能的。}随着“最薄弱的链条”的突
破,资本主义只可能逐渐地被推翻。这些不必发生在经济上最发达的国家。斯大林强调,世界
经济对革命的制约呈现出更为复杂的形式,但他并没有比列宁更多地用一般性的用语详细说明
什么决定着“薄弱的”链条。

\textbf{可以利用帝国主义内部的分化来保证苏联有一个暂时的和平时期,在这一和平时期可
  能在本国基础上建立起社会主义。斯大林认为,这并不排除社会主义最终胜利需要世界范围
  的革命,确实把国际革命放在适当的位置。}他认为,\textbf{遭受资本主义包围所面临的各
  种矛盾,可以分为内部与外部两个层面。俄国内部的小资产阶级生产关系对革命产生威
  胁,但是沙皇时代发展的特殊性,已经使这个国家的大工业掌握在无产阶级政府手中,允许较
  大程度上的自给自足的发展,可能克服由落后引起的各种内部矛盾,从而最终达
  到“完全”建立社会主义。但是,社会主义的“最终”胜利只能以世界革命为保证,它将消
  除资本主义势力对一国国内成功的社会主义(不论多么发达)的武装干预。因此,苏联绝不会
  放弃国际无产阶级,但也不会因被封锁隔离而停滞不前。}

事实上,斯大林认为相反的情况也是存在的。\textbf{由于苏维埃俄国只有通过西方无产阶级
  牵制帝国主义的侵略,才能成功地得以巩固,1917年俄国革命实际上是一场国际革命。}它打
破了世界经济的整体性,全球被分为两大集团:一方面是帝国主义力量,另一方面是以苏联为首
的反帝国主义的力量。\textbf{扩大反帝国主义力量的最有效方法是证明社会主义可以成功
  地在苏联建立起来,这同时也将削弱社会民主党对工人阶级的影响。其中蕴含的重要结论就
  是,世界无产阶级的利益同苏维埃俄国的利益是一致的。}按照斯大林的观点,它们之间不存
在分歧。这样,斯大林就改变了人们过去对俄国和世界革命之间关系的认识,把苏联的民族主
义(有时表现为俄国的民族主义)与世界社会主义事业融合在一起了。\textbf{与其说他使共
  产国际服从于苏联国内的利益}——正如人们经常指责他的那样——\textbf{不如说他把苏联
  国内利益等同于无产阶级的整体利益。}在这里也同他的所有其他理论观点一样,都是通过
引述列宁的章句而使之合法化的。

按照斯大林的观点,社会主义的建立应该遵循两个原则:\textbf{必须增进党的团结统一,必须
  快速而自主地实现以重工业为主的工业化。党的领导地位等同于无产阶级专
  政(在1936年“建成社会主义”之后,则等同于劳动人民专政)。}在这里,斯大林不过表达了
所有布尔维克主义者共同信奉的准则。但是,他确实特别强调这样的信念,即孤立和落后引起
左派与右派的“分化”,因此清洗成为自卫的持续的必要的条件。还有其他一些威胁党的纯洁
和国家管理的因素。干部可能沾上官僚主义惰性,受权力诱惑而滥用他们的地位。另外,由于
对资产阶级专家的依赖,政权利用了旧社会的可能背叛的阶层。因此,建立党内的组织纪
律、“来自下面的批评”,以及建立“红色专家队伍”,都是建立社会主义的至关重要的因
素。

斯大林关于政党政治和行政管理问题的这些言论并不局限于20世纪20年代,当时这些言论有助
于维护他本人的小集团的统治。这些言论延续到30年代,诠释了1934年大清洗背后的某些动
机,一批在过去十年间给予他支持的人大部分遭到迫害。这些言论也与思想意识相关。斯大林
认识到“来自下面的批评”的重要性,并将它与列宁的《国家与革命》联系起来,\textbf{他
  利用人们对“大知识分子”的普遍反感,这表明毛泽东的文化大革命尽管有其新特点,但却
  完全是以斯大林主义的实践为基础的。}

“\textbf{一国社会主义}”要求实现特定形式的工业化。斯大林认为,由于具有国际上的象
征意义以及国内的社会基础,苏联不会照搬资本主义的剥削方法。他从来不提供评价一种特别
的社会主义工业化成功与否的标准,但是,\textbf{他在20年代用以反对普列奥布拉任斯基社
  会主义原始积累论的观点,成为反对左翼反对派的有效力量。但是,因为斯大林“左转”后
  还多次重复他的观点,所以使这一观点所包含的内容超出了宗派斗争的需要。}他特别强调
独立发展的必要性,相信任何试图把苏联生产融入国际市场的做法,都会产生一
种“\textbf{依附}”的条件。这正是他的经济政策区别于托洛茨基的地方,托洛茨基认为广
泛利用世界经济对达到较高的劳动生产率至关重要。

\textbf{这种对自给自足的强调与布哈林新经济政策下的社会主义发展战略相吻合,斯大林
  在1928年以前对这种战略予以广泛支持。但是,在1925年以后,他与布哈林的渐进主义开始
  分裂,抛弃注重消费品生产的看法,相反宣称必须迅速实现工业化,并且集中于重工业。}

战后重建于1926年完成,1927年以后又出现“战争恐慌”,这些都强化了上述看法。

但是,在整个这一时期,斯大林未能把实现工业化的看法与新经济政策固有的制约因素结合起
来,结果导致1928年和1929年的\textbf{谷物危机}。他的著作表现出极大的混
乱,例如,他把\textbf{农村消费品市场和重工业}都看作是现代化的“基础”。而斯大林在论
述重工业的重要性时,的确意识到一些\textbf{重要的比例关系},后来\textbf{费尔德曼
  在1928年发表的数学增长模型中概括了这些关系,现在人们一般将它看作是对马克思再生产
  模式的成功的概括和拓展。}费尔德曼的两部门模型建立在非常严格的假定基础上,阐述的
是一种封闭经济,\textbf{在这种封闭经济中,资本存量是限制增长的主要因素。}他证
明,\textbf{第\Rnum{1} 部类相对于第\Rnum{2} 部类的资本存量比例越高,经济的总体增长率就越高。所以,经
  济的加速增长需要投资集中于资本品部类。}费尔德曼还证明,一旦达到一个理想的增长率,就
可以通过两大部类之间按照已有的资本存量的比例进行投资来维持这一增长率。\textbf{那
  时,消费品的增长将与资本品的增长相等。由此得出一个明显的自相矛盾的结论:消费品增
  长率的平稳提高,要求将现有投资集中于资本品生产部门。}

1928年以后苏联经济的发展似乎符合费尔德曼的模型,这也是斯大林有意识地提出的相关政
策的结果。投资被集中在第\Rnum{1} 部类,1952年斯大林又重申投资集中于重工业部门的必
要性,直至苏联的人均收入超过西方资本主义国家的人均收入。\textbf{但是,
  在20世纪20年代末,斯大林完全没有看到,强调发展重工业的任何观点所固有的“杜冈主
  义”,与布哈林用以反对左翼反对派的消费不足论相矛盾,因为布哈林认为维护农村市场
  是经济发展的真正基础(参见本书第一卷第9章、第10章和15章)。}另外,费尔德曼的模
型\textbf{并不完全适合}于分析斯大林的经济体制,\textbf{因为它假定比例关系保持不
  变}。1928年以后的“高速、绷紧的计划”政策,几乎不符合费尔德曼的均衡观点;也不符
合满足防务的需要。\textbf{费尔德曼对第\Rnum{1} 部类优先权的辩护是建立在消费的更大
  增长的基础上的,而这是通过把资源向资本品生产部门的最初转移实现的。}费尔德曼的模
型为苏联的热情支持者如莫里斯·多布提供了一个重要论据。与对斯大林命令经济的理想化论
述相配合(参见以下第三篇),多布运用这一模型展示了社会主义工业发展与资本主义相比具
有的优越性,他的分析有着广泛的影响。

苏联的工业化还表明它\textbf{偏重于资本密集型生产技术。这反映了“赶超”资本主义发
  达国家的目标,也反映了传统马克思主义的大规模组织生产具有更高效率的观点。}但是,多
布再次试图使这个问题人性化。他认为,在可用于投资的剩余的规模成为经济更快增长的制约
因素的情况下,选择这种使\textbf{剩余最大化}的技术是合理的,而这些技术可能比那些最大
限度地提高当前产量和就业的技术更倾向于资本密集型生产。\textbf{随着投资集中于
  第\Rnum{1} 部类,任何放弃消费的初始代价都是暂时的,而且可以通过后来的更高的消费来弥补。
  这是正确的,但正如米哈尔·卡莱茨基所指出的,按照多布的标准来选择技术,可能会在经济
  增长的早期阶段使产量降低和就业减少;而建立在失业和资源浪费基础上的发展战略是不受
  欢迎的。卡莱茨基还认为,技术变化的过程会使剩余最大化的技术与就业最大化的技术之间
  的差距缩小,这样多布的观点在实践上变得越来越不重要。}

\textbf{最后的一个观点是技术变化采取的确切形式的问题。它绝不会削弱多布关于投资使
  剩余最大化的原则,但是也提出了在两个时间段之间进行选择的难题:当前的产量、消费和
  就业越低,将来的产量、消费和就业就越高(可能非常高)。}事实上,就苏联的发展而言,多
布和卡莱茨基的分析都不十分贴切。\textbf{人们难以证明,可投资的剩余规模严重阻碍了斯
  大林统治下的苏联的发展。}实际上,当时苏联出现的是\textbf{相反的情况:投资超过了经
  济的吸纳能力,不必要的加紧剥夺,进而引起大量的无效率的生产(参见以下第18章)。但推
  进剩余产品的增加不会降低产量或引起失业。}

1927年的“\textbf{战争恐慌}”使斯大林对帝国主义理论作了修改。\textbf{他背弃了布哈
  林关于资本主义已经稳定的观点,而赞成新卢森堡学派用消费不足所导致的对新市场的持久
  需求来解释西方的侵略。}这就成了整个30年代的斯大林主义的教条。尽管它与斯大林的苏
联工业化观点相吻合,但它在20世纪20年代进一步使斯大林与布哈林主义右翼分道扬镳,而与
左翼反对派更为接近。然而,他对于资本主义稳定为什么不复存在的详细理论阐述,主要还求
助于瓦尔加。

至于国内谷物危机所引起的内部不稳定,斯大林似乎一无所知。他认识到,\textbf{更快的工
  业发展,需要更大的农村市场,但是他像大多数左翼反对派一样,乐观地相信这一切可以通过
  建立在长期机械化基础上的自愿的集体化来实现。}托洛茨基非常正确地指出,斯大林笨手
笨脚地引发了危机,他解决危机的激进方式纯粹是“\textbf{经验性的}”。只有在斯大林从
实践上彻底抛弃新经济政策后,才使其“自上而下的革命”得以合理化,并对“一国建成社会
主义”的教条提出了与其在20年代占主导地位时不同的内容。

但是,这种教条的形式符合斯大林的马克思主义观中的一个不变准则:\textbf{他不是将急
  剧的变化解释为错误的后果,而是将其解释为新阶段发展的要求。}斯大林运用这一原则阐
述列宁的理论发展,认为列宁主义早在1905年(如果不是更早的话)就已经完全形成;其后的变
化是很明显的,革命发展到新的阶段都会揭示出那些曾经是潜在的东西,每一阶段都是前一
阶段的必然结果。至于谷物危机,斯大林认为这正是社会主义乘胜前进的代价。\textbf{在
  新经济政策条件下,向社会主义迈进已经“使阶级斗争激化”,因为作最后挣扎的资产阶
  级分子试图保持其阶级地位。}由于社会主义“上层建筑”的作用就在于建立其“基础”,
国家已经通过“自上而下的革命”彻底介入阶级斗争。不仅富农“作为一个阶级被消灭”,
而且工业部门的许多资产阶级分子通过国家化和对那些在“社会主义不断前进”中进行“破
坏”的专家进行起诉而得到清除。因此,一个新时代诞生了,“一国社会主义”在农业集体
化和计划经济的基础上飞速前进。

\textbf{“阶级斗争激化”的教义已经成为一种笑柄,}因为它掩盖了政策上的严重倒退,并
且被广泛地用来镇压斯大林的反对者。但是,它并没有被托洛茨基主义者轻易地抛
弃。\textbf{最初提出不断革命论,是为了巩固无产阶级权力而不是革命本身,它要证明落
  后的俄国最困难的是:无产阶级专政推翻了农村古老的社会制度,但农民却强烈地抵制推
  动其进一步发展所必需的措施(参见本书第一卷第12章)。}在新经济政策时期,托洛茨基修
正了他的观点,降低了无产阶级与农民之间对抗的必然性,认为假如左翼反对派的政策能够
奏效,就\textbf{有可能在自愿的基础上实现集体化}。他或者其他左翼反对派\textbf{没有
  足够地重视布哈林的观点,即“超工业化”需要第二次革命,并且这场革命只有通过全面
  的恐怖统治才能成功。}

在这一方面,斯大林并不否认其现行政策在很大程度上与左翼反对派在20年代的政策主张相
吻合,但他解释说,左翼反对派的观点不适合早期发展阶段。另一方面,斯大林认为,布哈
林和“右派分裂分子”没有认识到新经济政策阶段已经发生转变,还试图保持过去的政策,
同样犯了时代的错误。因此,斯大林愿意在左翼反对派成员承认错误的情况下,恢复他们的
党员资格,他也同样接受右派的公开认错,当然这一切都是以卑躬屈节的方式进行的。

\textbf{直到20世纪30年代中期,(国有)工业部门的产量与农业部门的产量大致持平(苏联据
  此宣布为“工业国”)};绝大多数农民被集体化;生产资料私人所有制已经变得无足轻重。
如果斯大林相信社会主义能够在一个国家建成(参见第三篇),那么宣布苏联建成社会主义就
是理所当然的了。资本主义的包围被用来解释社会主义存在的问题的性质,特别是国家的继
续存在、残酷镇压的必要性,反映了“资产阶级意识”残余支撑的反革命势力仍在垂死挣扎。
斯大林表达了后来与\textbf{毛泽东}联系在一起的一个教义,\textbf{宣称阶级斗争不会随
  着“一国社会主义”的成功而废止,只有到社会主义在全世界“最后”胜利才可能停止。}

但是,\textbf{苏联社会主义的“完全”胜利,把(在单一国家里)向共产主义过渡提上日
  程。}斯大林从\textbf{斯达汉诺夫工人的较高的劳动生产率}中洞察到共产主义出现的一
缕曙光,1935年以后,他不时就通向人类解放更高阶段的正确道路作出指示。纳粹的入侵以
及战后的重建,使这一问题直到20世纪40年代末才引起广泛讨论。其中的很多问题都同布尔
什维克统治一开始就引起的一个争论联系在一起的,\textbf{这个争论就是价值规律在社会
  主义中的地位。}许多理论家认为,社会主义中价值规律不再起作用,社会主义意味着商品
生产的终结,因此也是价值形式的终结。与这一观点相联系的是,人们相信苏维埃国家在社
会主义经济关系的转变中是不受制约的。20世纪30年代期间,当“没有任何堡垒…… 布尔什
维克不能攻破”的思想占统治地位的时候,上述观点赢得了广泛的支持。但
是,\textbf{1943年《在马克思主义旗帜下》杂志发表的一篇未署名的文章提出,这一切都
  是错误的,价值规律只有在共产主义来临的时候才会停止对经济发展的制约。}尽管这篇文
章并不深刻,条理也不清楚,但它在西方引起一些震动。这篇文章第二年在《美国经济评论》
中用英文再次发表,引起了广泛讨论。马克思主义者对这个问题的理解相距甚远。一方
面,\textbf{拉亚·杜娜耶夫卡娅}认为,苏联经济学家接受价值规律表明一种新的非社会主
义的社会秩序已经巩固,\textbf{它要求实现意识形态上的变革,以便使广泛存在的不平等
  现象合法化。}另一方面,\textbf{艾萨克·多伊彻}在后来几年的著述中认为,\textbf{它
  意味着已经变质的工人国家的矛盾日趋成熟},这一点正如托洛茨基所预见到的那样(参见
第3章和第18章)。

\textbf{1952年,斯大林在干预苏联的这一争论时,驳斥了乌托邦的唯意识论,选择了经济
  渐进主义。}他认为,\textbf{苏联社会主义受客观规律——包括价值规律——的制约,这些
  规律可以被自觉地运用,但在社会主义条件下不能被超越。}斯大林没有对如何识别这些规
律作出论述,但是这些规律可以认知和被有意识利用的事实表明,社会主义的自由与资本主
义条件下盛行的盲目的无政府状态形成鲜明对比。商品关系(当然是非资本主义的商品关
系)在消费产品的分配中,在国营工业和集体农庄的关系中,仍然发挥着作用。尽管苏联可能
在不久后会实行实物交换,以取代货币交易,但要经过一个时代才能实现从集体农庄到国营
企业的转变,同时才能消灭商品关系。这同所有其他新的过渡措施一起,被社会主义国家自
上而下地引入,但斯大林忽略了集体农庄的市场目标和过去20年相关的“交换”率都由中央
确定的事实。斯大林认为,由于苏联在总体上还落后于资本主义发达国家,只有到经济上超
过西方国家、工业能够集中于消费品生产的时候,苏联才能向共产主义迈出实质性步伐。斯
大林断言,到那时社会主义积累规律还是有效的。\textbf{第\Rnum{1} 部类的增长率必然超过
  第\Rnum{2} 部类的增长率,否则“国民经济不可能持续发展”。这明显地是对费尔德曼增长公式
  的歪曲,但它后来却被苏联的政治经济学教科书以类似的形式不断重复。}

事实上,斯大林的这些观点类似于他在新经济政策早期所持的观点。苏联的经济关系并不存
在进一步的革命性的变革;苏联将“发展到”共产主义的更高阶段,社会主义的不完美将“消
失”。但是,不祥的迹象表明,他正准备像20世纪30年代晚期那样,通过新的清洗来实施其
保守主义,以对抗官僚机构中的激进分子。对那些急躁的反对斯大林观点的人来说,非常幸
运的是,那位大独裁者在把这些想法大规模实施之前,就于1953年3月去世了。

\section{斯大林主义与马克思主义}
国际共产主义运动及世界范围内的追随者,在四分之一世纪还多的时间里,都把斯大林看作
是马克思主义的\textbf{最高理论家},把苏联看作\textbf{社会主义方案的生动体现}。苏
联以外的主要马克思主义经济学家如\textbf{莫里斯·多布和罗纳德·米克,也持这种看法},
当然他们比在斯大林统治下的典型的共产主义者更严谨,更富学究气。多布从未参与“个人
崇拜”,并且事实上对“个人崇拜”还作过一定程度的批判,特别是对第一个五年计划的高
速度作过批判。他的著作是有价值的,受到非共产主义经济学家的好评。多布从大萧条和第
二次世界大战的角度,用通俗易懂的方式清楚地描述了命令经济的状况。但是,他显然未能
把马克思主义作为一种批判的理论,他对苏联的总体描述与苏联极权主义的现实相去甚远。

多布在一些关键问题上没有明显地脱离斯大林主义的传统认识。在20世纪20年代末期,他与
苏联官方观点相呼应,赞成布哈林的经济发展战略,反对普列奥布拉任斯基的战略,并宣称
新经济政策完全符合列宁主义。在“自上而下的革命”发生后,多布用同斯大林相同的方式
使之合理化,也对斯大林反对者的观点予以驳斥。他对反对派的批评常常是斯大林已经提出
过的。多布对于在加速工业化中存在的多方面的镇压,要么视而不见,要么认为这仅仅是信
息不灵通的结果,并以战争威胁和反革命活动为其辨护,或用技术性的术语来对待它。多布
对工人管理或政治民主没有兴趣,并把社会主义等同于国家主义和中央计划。他对苏联存在
的不平等和特权的严重程度估计不足。简而言之,多布轻描淡写地谈到这些事实上绝不平静
的事件。

罗纳德·米克的大部分著作具有文质彬彬的亲斯大林主义的特点。但是,与多布不同,米克更
关注斯大林的思想,声称《苏联社会主义经济问题》是对马克思主义的重大贡献。米克欣然
认为,斯大林坚持社会主义条件下存在商品关系和价值规律的观点,与马克思和恩格斯是不
同的,这正是斯大林对马克思主义的贡献。俄国革命是在落后条件下取得成功的,这在19世
纪是不可想象的。这意味着商品生产和价值关系仍然起支配作用。斯大林由于认识到这一点,
并阻止了更具空想性的经济学家冒险地推动苏联前进而受到称赞。这显然是对斯大林真实思
想的曲解,实际上也与米克本人的思想不相符合,这也勾勒了苏联后来对价值理论的争论(参
见以上第2节)。

像多布一样,米克没有充分阐述斯大林主义的恐怖,但也没有像其典型的追随者那样制造粗
糙的辨护和狂热的谎言。与共产主义者不同,\textbf{这些追随者忠于苏联是与斯大林掌握
  政权相一致的,他们将斯大林上台视作科学的社会管理对布尔什维克激进主义和阶级暴力
  的胜利。}他们几乎不承认自己是马克思主义者,也不对苏联做任何的经济分析,而马克思
主义者则不同。他们大多数是社会民主党人、自由主义者、基督徒和和平主义者,对苏联社
会主义敬而远之,不参加当地的共产党,满足于他们自己的资产阶级的生活条件。与苏联宣
传机器相呼应,他们的表演几乎是丝丝入扣,着重渲染权威主义情绪,这种情绪有时会有支
持启蒙运动的价值。

最具有分析性和批判性的,是马克思主义中的斯大林反对者(参见以下第3章)。他们不肯严肃
地对待斯大林的理论,将其视为对马克思主义的荒谬的歪曲,并常常将其看作是对革命理想
的背叛。他们表达的观点和斯大林主义对其的反驳,不断激起关于斯大林主义与马克思主义
的关系、斯大林的独裁是不可避免还是偶然的激烈争论。

形形色色的观点都有着丰富的论据。列宁、布哈林和托洛茨基的成熟理论与斯大林的粗陋形
成鲜明对比;20世纪20年代的独裁形式比30年代的形式要温和得多;\textbf{可以有把握地认
  为,斯大林上台的原因是他的反对者犯有明显的错误,而不仅仅是反对者们对斯大林的野
  心和能力估计不足。}另一方面,正统的马克思主义关于社会主义的观点总是倾向
于\textbf{集权主义,尽管在马克思本人思想中明显地存在着自由主义成分。}另外,按原文
加以注解,使人把斯大林的分析与列宁的分析联系起来,并且实际上还与布哈林和托洛茨基
的思想联系起来;历史学家可以辨别出早期布尔什维克主义的冷酷无情和独裁主义特点;人们
很容易认为经济发展中的问题排除了与斯大林所采取的十分不同的政策措施。这些说法,可
以变换来为斯大林主义的不同理论提供辨护或予以驳斥。这些含糊不清的说法的根源,在于
历史分析常常存在方法论上的、本质上的问题,但也包括马克思主义本身所存在的缺陷。

实践是用以解决这些问题的一个可能的标准,特别是斯大林本人充分准备接受结果的检验。
他指出了铸铁上的显著成就、生产资料私有制已被实际根除、苏联从一个农业社会转变为工
业社会,以及国家不仅取得了反对纳粹侵略的胜利,而且同时把社会主义扩展到周边地区。
但是,\textbf{批评者合理地指出了存在着顽固的野蛮因素、持续的国家镇压和斯大林发动
  自上而下革命后大量不平等的增长(不只是不平等的维持)。}然而按照社会主义的经典概念,
这些解释中的每一条都存在问题。马克思主义者为了证实自己的观点,传统上是把社会主义
定义为包括斯大林所援引的说法,而排除了批评者所强调的内容。所以,苏联的发展引出了
社会主义新模式,不论是斯大林主义者还是其马克思主义的反对者,都不能够断然面对这样
一个难题,即\textbf{这本身是否与历史唯物主义相悖。}

马克思主义的批评者有时采取不同的方针,把斯大林的\textbf{成就与“罪恶”区别开来},
认为前者标志着其成功之处。尽管这种看法从自由人道主义的观点可能具有合理
性,\textbf{但它在马克思主义自身的范围内却不是一个连贯的见解。}就马克思主义者对道
德标准的关注而言,\textbf{他们倾向于使手段的重要性最小化而集中关注其结果,}他们非
常清楚发展表现为否定。

另外,斯大林的恐怖很难与他的积极的成就分开。尽管镇压带有随意性成分,意识形态的统
治对科学的进步产生很大损害,但这两者也解决了经济发展的现实问题:敌对势力的抵制、
劳动纪律的惊人松弛、民族文化的匮乏,以及官僚集团内部的离心倾向。这些都由于斯大林
及其随从缺乏了解而恶化,部分原因是行政管理机构的效率极度低下,对国家的控制有时极
其松散。

通过\textbf{把理想的自由社会与阶级斗争相联系},马克思主义者\textbf{把合法的革命行
  为降为确定无产阶级利益的问题。马克思似乎相信这是一个明显的社会现象,}从而对这个
问题没有作深入探讨。从卢卡奇在《历史与阶级意识》中提出的相当成熟的具有黑格尔理论
印记的观点,到斯大林将无产阶级的事业与苏联重工业的扩张等同起来的粗陋的理解,后来
的马克思主义者对此的理解存在很大差别。还有一些理解处于这两个极端之间,包括列宁的
《怎么办?》中的观点。这一问题还与无产阶级利益的其他三个相关方面的问题联系在一起:
\begin{quotation}
  \textbf{(1)无产阶级的特殊利益;(2)非无产阶级的被压迫者的阶级利益;(3)实行阶级专政
    的统治形式。}还没有一个理论家解决了这些问题(参见以下第3章)。所以,\textbf{人
    们无法断定或否认斯大林的政权组织和经济发展战略符合无产阶级的利益。}
\end{quotation}

\textbf{斯大林利用了马克思主义的这些缺陷,使他的统治合法化。}但在关键时刻,他似乎
决定冒大风险。在20世纪30年代中期,他指责布尔什维克中的反对者是自觉的反革命,同帝
国主义势力的情报机构相勾结。托洛茨基被刺杀,基洛夫、加米涅夫、布哈林及其他多数老
布尔什维克,都以上述借口被判死刑。证据被罗织出来,通过酷刑使他们招认,而历史档案
则被篡改。这些都具有\textbf{热月政变的特征},而许多马克思主义者却毫无缘由地无视这
一点。(参见以下第3章)

但是,甚至这些事件都有其正统的理论的合理性,这些时时被一些较为敏感的斯大林主义者
暗示过,但他们仅仅认为布尔什维克反对斯大林主义客观上就是反对革命,尽管并不认为这
些反对者是自觉的帝国主义的代理人。不管他们的动机如何,他们政策的后果将会损害苏联,
从而损害国际无产阶级;因此对他们的指控是有根据的。这些主张的表达方式是与人们文化上
的落后相适应的,所有布尔什维克人——斯大林主义者和反斯大林主义者——都十分清楚地认
识到这一点。\textbf{这要求使用说教性语言,把客观结果转变为主观目标。用相似的原则
  使官方对历史作出的歪曲、对列宁著作进行的删改,以及对生活水平提供错误的统计数据
  等做法合法化。}

所有这些都表明,马克思主义在其发展历程中,\textbf{经历了从批判的理论到国家意识形
  态的转变和扭曲。}在取得统治权之前,常常\textbf{假设真理会与无产阶级的发展一同前
  进。}马克思主义者从来没有想到忠实理论本身可能会威胁他们的统治。但在早期,正好在
斯大林上台之前,这种矛盾确实出现了。1921年的\textbf{克琅斯塔得叛乱}作为反革命事件
被镇压,克琅斯塔得的水手们因要求恢复布尔什维克自身在1917年所坚持的政策,而被谴责
为“白匪”。这并不表明布尔什维克主义就是斯大林主义,或者斯大林主义就是布尔什维克
主义发展的唯一形式。但它确实意味着,斯大林主义可以归为马克思主义的一种形式,其依
据与布尔什维克主义的其他形式大致相同。试图指导和控制社会主义实践的是一系列的理论。
在一个不宽松的氛围中,获取权力的动机渗透到理论中,并在斯大林的统治下不断地蜕变。
但是,斯大林从来无法垄断马克思主义,理论上的贫乏有利于斯大林主义自称为马克思主义,
也使另一种选择——甚至其他形式的布尔什维克主义——成为颠覆斯大林主义统治的学说。我
们在下一章将转入讨论与斯大林同时代的马克思主义。至于后斯大林的发展,将在以下第9章
第2节和第18章进行讨论。



\chapter{苏联生产方式}
\section{预期}

\begin{quotation}
  如果资产阶级实行阶级统治的经济条件没有充分成熟,要推翻君主专制也只能是暂时的。
  人们为自己建造新世界,不是如粗俗之徒的成见所臆断的靠“地上的财富”,而是靠他们
  垂死的世界上所有的历来自己创置的产业。他们在自己的发展进程中首先必须创造新社会
  的物质条件,任何强大的思想或意志力量都不能使他们摆脱这个命运。(马恩全集第一
  版,第四卷,P332)

  对于激进派的领袖来说,最糟糕的事情莫过于在运动还没有达到成熟的地步,还没有使他
  所代表的阶级具备进行统治的条件,而且也不可能去实行为维持这个阶级的统治所必须贯
  彻的各项措施的时候,就被迫出来掌握政权。他\textbf{所能}做的事,并不取决于他的意
  志,而取决于不同阶级之间对立的发展程度,取决于历来决定阶级对立发展程度的物质生
  活条件、生产关系和交换关系的发展程度。他\textbf{所应}做的事,他那一派要求他做的
  事,也并不取决于他,而且也不取决于阶级斗争及其条件的发展程度;他不得不恪守自己
  一向鼓吹的理论和要求,而这些理论和要求又并不是产生于当时社会各阶级相互对立的态
  势以及当时生产关系和交换关系的或多或少是偶然的状况,而是产生于他对于社会运动和
  政治运动的一般结果所持的或深或浅的认识。于是他就不可避免地陷入一种无法摆脱的进
  退维谷的境地:他\textbf{所能}做的事,同他迄今为止的全部行动,同他的原则以及他那
  一派的直接利益是互相矛盾的;而他\textbf{所应}做的事,则是无法办到的。总而言之,
  他被迫不代表自己那一派,不代表自己的阶级,而去代表在当时运动中已经具备成熟的统
  治条件的那个阶级。他不得不为运动本身的利益而维护一个异己阶级的利益,不得不以空
  话和诺言来对自己的阶级进行搪塞,声称那个异己阶级的利益就是本阶级的利益。谁要是
  陷入这种窘境,那就无可挽回地要遭到失败。(马克思恩格斯文集,人民出版
  社2009年,第三卷,P303-304)
\end{quotation}

马克思和恩格斯曾经警告过,\textbf{“过早地”夺取政权将会使无产阶级政府根基不稳。
  由于无法实现自己的纲领,无产阶级将被迫为了生存而成为资产阶级的工具,完成资本主
  义的历史任务。}但这究竟包含着怎样的蜕化过程,并没有得到详细的说明,他们简短的论
述几乎没有表明苏联后来的发展。尽管人们可能会认为,斯大林主义的不断发展所完成的是
历史唯物主义指出的资本主义的职能,但这种政权按传统的定义已明显不再是资产阶级的工
具。但是,一个无阶级的社会是否已经在苏联出现就更不清楚了,如果还没有出现,那么这
个新的统治阶级又可能是什么呢?

马克思和恩格斯的担忧极大地影响了他们的后继者。\textbf{普列汉诺夫、孟什维克、“合
  法马克思主义者”和全世界的社会民主党,都认为在俄国只有资产阶级民主革命才是可行
  的。}在1917年之前,\textbf{列宁}也持这种观点,也许除了与\textbf{布哈林}保持密切
联系的人之外,所有的布尔什维克人也都坚持这种观点。\textbf{持各种各样意见的正统马
  克思主义者都相信,俄国革命可能具有特别激进的特点},但在第一次世界大战之
前,\textbf{只有托洛茨基和他的追随者}相信,苏联可以马上进入社会主义阶段(参见本书
第一卷第2篇)。
  
但是,在托洛茨基的早期著作中,既有革命激进主义,也有对布尔什维克主义畅言无忌的反
感。\textbf{托洛茨基}接受了孟什维克提出的要建立一个\textbf{结构松散的群众性政
  党}的观点,宣称列宁在《怎么办》一文中所描绘的蓝图是\textbf{把雅各宾主义渗入社会
  民主党中。列宁已经用政党代替了阶级。}他的思想逻辑就是,“政党组织以自己代替这个
政党,中央委员会以自己代替这个组织,最终‘独裁者’以自己来代替中央委员会”。不管
托洛茨基预言性批判的性质如何,他不可能相信它适用于未来任何一个无产阶级政府,哪怕
这个政府只从布尔什维克那里获取理论指导。在1905年至1917年间,他认为“不断革命”是
一个必需的过程;不管这个革命政党的纲领和结构如何,它们最终将被真正的无产阶级力量的
发展所淹没,这些政党要么将被迫支持无产阶级,要么将被搁在一边不被理
睬。\textbf{1914年前后,托洛茨基也宣称,从世界资本主义经济占优势地位的角度来看,
  俄国的无产阶级统治绝不是“早熟”的。}

托洛茨基的全球观点最终渗透到布尔什维克主义中,该党的所有理论家(包括一些左派孟什维
克党人如尤利·马尔托夫)在1917年后都\textbf{否认俄国革命是冒险行为}。按照他们的观
点,\textbf{帝国主义战争表明资本主义已不再是一种进步力量,他们自信地预测俄国革命
  之后会爆发一系列的无产阶级革命。}但是,无论是托洛茨基还是列宁、布哈林,都没有清
楚地说明,在发达国家推翻资本主义将如何扫除落后的俄国进行社会主义建设面临的障碍。
这种遗漏带来了很大的麻烦,\textbf{因为普列汉诺夫早在19世纪80年代就认为,如果没有
  国际的支持,使俄国农业实现现代化的企图将导致任何社会主义政府变质。但他没有详细
  阐述这将采取何种形式},并且他的理论是针对民粹派而不是针对马克思主义者的(参见本
书第一卷第 140 页 )。

作为一个无政府主义者攻击马克思主义的组成部分,\textbf{巴枯宁}在较早的时候就提出了
类似的观点。按照他的观点,马克思主义关于革命过程的整个概念都是令人怀疑的;\textbf{正
  是“科学社会主义”概念隐含着精英统治论,通过集权国家来实施无产阶级专政的思想与
  人类自由是不相符合的。尽管在对巴黎公社进行阐述时,马克思曾含蓄地接受了许多无政
  府主义的论点,但在行动上他一直反对无政府主义。}在正统马克思主义的全盛时期,马克
思主义者对无政府主义批评家表示轻视,对他们\textbf{不顾客观条件限制}进行持续不断的
解放运动表示蔑视。大多数无政府主义者并不清楚沿着国家社会主义道路前进的具体结果,
这一事实无疑巩固了上述状况。

\textbf{波兰激进主义者简·麦克哈吉斯基}是一个例外,他在19世纪90年代把粗略的历史唯
物主义应用于马克思主义自身。他认为,\textbf{社会主义知识分子典型地利用了工人运动,
  达到他们自己的小资产阶级的目的,所以,成功的革命将导致新的阶级统治形式,而知识
  分子将成为统治阶级。这种思想肯定会在马克思主义中找到自身的位
  置。}早在1918年,\textbf{考茨基}就指出,\textbf{一个“新阶级”在苏联发展起来},
尽管他并不打算确定它的具体特征或者将之归为哪一种新的生产方式。但在20年后,唯物主
义对这种特权的新形式作了详细分析,从而为麦克哈吉斯基的思想提供了一个较为坚实的基
础(参见本章第3节、第4节)。

马克斯·韦伯论述官僚统治的著述可以用来论证这些思想,尽管大多数马克思主义者对此犹豫
不决,因为\textbf{韦伯认为广泛的官僚化是公有制和计划经济的必然结果,马克思设想的
  社会主义根本是不可行的(参见以下第18章)。}韦伯的学生\textbf{罗伯特·米歇尔斯}则更
进一步,他在研究了德国社会民主党的结构后,于1911年详细阐述了“\textbf{寡头统治铁
  律}”。

考茨基于1908年试图含蓄地对这些思想进行反驳,布哈林也于1920年试图从正面迎接这个挑
战,但是他的观点在布尔什维克统治下尤其不能令人信服。布哈林后来对官僚化予以关注,
但他的论述没有形成体系,而他关于国家资本主义的思想确实产生了较大的影响(参见本章
第4节)。韦伯的思想可能影响了\textbf{克里斯琴·拉可夫斯基},他在20世纪20年代是左翼
反对派的主要成员。\textbf{他超越了托洛茨基关于蜕变的分析(参见本章第2节),认为即使
  在最宽松的环境下,官僚化都会真正地威胁到社会主义理想的实现,他开始把苏联的官僚
  制度与新的统治阶级联系起来。}但是,拉可夫斯基没有能够使他的分析系统化,1933年在
政治高压下,被迫宣布放弃了这些思想,5年后成为斯大林大清洗的牺牲品。

\textbf{革命的彻底性}从一开始就引起党内各方面的关注。布列斯特-里托夫斯克和约表现
出的向帝国主义的妥协、政党独裁的发展、雇用“资产阶级专家”和沙皇官吏,以及战时共
产主义使劳动军事化的方案,所有这一切都引起对党内左翼反对派的疑
虑。\textbf{到1922年,列宁甚至明确承认已经出现了官僚主义蜕变,并发现它在斯大林的
  独裁行为中得到体现。}列宁把出现上述情形的根本原因归为俄国文化的落后,面对这一相
当普遍的现实,\textbf{他有时认识到用行政措施来限制腐败是软弱无力的。}

在20世纪整个20年代,党内成员一直存在着不满情绪,其中的一些人同意被流放的孟什维克
党人的观点,\textbf{认为复辟势力不断增长,并正在改变布尔什维克的意识形态和党的组
  织。}但最不可想象的是,反对派的思想倾向在1923年后不断升级。利夫·托洛茨基在国内
战争期间达到其权力的巅峰,他既是所有布尔什维克领导者中最大的权威,又是其贬损者的
最猛烈的批判者,后来却成为坚持\textbf{革命蜕变论}的主要理论家。直到1940年被刺杀前,
他一直对苏联制度抱有敌意,是最坚定的反对者之一,而他的支持者则投向了斯大林主义。

\section{一个蜕变的工人阶级国家?}
托洛茨基花费近20年的时间试图阐明苏联革命的蜕变问题。由于这一时期发生了巨大变化,
托洛茨基的思想也毫不奇怪地随之发生了变化。但是,某些基本特征自20世纪20年代中期以
后一直没有发生变化。在对他自1923年至1940年的理论演化史阐述之前,有必要考察这些保
持不变的特点。

像大多数布尔什维克人(包括列宁和斯大林在内)一样,托洛茨基相信\textbf{世界资本主义
  经济处于划时代的危机中}。也像他们一样,他对危机潜在的经济原因总是含糊其词。但是,
他明确承认他的整个分析取决于这种观点的合理性。他承认,\textbf{如果国际资本主义有
  可能实现长期稳定,那么他的整个理论(包括适用于苏联的理论)的基础将被削弱,他对苏
  联蜕变的分析也将随之衰落。}

对社会主义革命不受客观经济条件制约的信念,使托洛茨基在两次大战期间的马克思主义观
具有极强的政治特点。他在解释无产阶级斗争的延迟、失败和蜕变时,将其归为错误的领导。
他从1923年起领导的一个组织,在1938年最终组成为\textbf{第四国际},这个组织被当作纠
正上述情况的手段。该组织\textbf{试图在世界范围内重复列宁的布尔什维克在俄国所完成
  的一切。}托洛茨基把他的不断革命理论等同于列宁主义,并视其为唯一的真正革命的马克
思主义。

根据托洛茨基的观点,苏联的情况仅仅是\textbf{无产阶级领导权总体上不适应}的一种特例。
蜕变在相当程度上\textbf{集中于上层建筑},对经济关系并没有造成很大的影响。这种生产
方式既不是社会主义的,也不是资本主义的,而是\textbf{过渡性}的。但是,只有当这一革
命\textbf{扩展到西欧}(参见本书第一卷第15章),\textbf{这一过渡才能成功地到达社会主
  义}。尽管“无产阶级专政”——或者如托洛茨基通常所说的“工人阶级国家”——继续存在,
但它在政党工具的支配下已经\textbf{官僚化}了。他逐渐承认这种官僚制度具有统治阶级的
某些特征,\textbf{因为它把特权和权力延伸到了职能上非必要的领域。}但是,他坚持认为,
从马克思主义的观点看,它仅仅构成一个社会的“\textbf{特权阶级}”,\textbf{对无产阶
  级是一种非剥削的、但属寄生性的关系。}在这里,托洛茨基主要求助于马克思本人的观点,
即\textbf{官僚制度自身并不构成一种社会力量,但它依附于一个或几个真正的阶级。}这在
资本主义生产方式条件下是有意义的,可托洛茨基从来没有阐述为什么在苏联也普遍存在着
同样的制约因素。因此,除了求助于马克思在不同条件下得出的公式外,托洛茨基并没有明
确说明如果官僚主义不是一个统治阶级又是什么的问题。他提出的“特权阶级”在一定程度
上具有永久性,它与两次大战期间的动荡形势严重不符。

托洛茨基认为斯大林只是同辈中最优秀人物中间的一个,认为他是在思想上背离了列宁主义、
并\textbf{用官僚主义意识形态来反对真正的布尔什维克主义}(即托洛茨基主义)的那种人。
因此,官僚政权的统治不仅损害了苏联无产阶级专政,而且损害了国际工人阶级革命的潜能,
增加了资本主义复辟的可能性。“一国社会主义”是一个导向完全错误的教义,\textbf{前
  进的唯一道路就是在不断革命理论指导下在国际范围内推翻资本主义。否则,苏联中间性
  的社会形式永远不可能稳固,它将不断遭受危机的破坏并最终被危机所吞没。}

托洛茨基在解释官僚化问题时摇摆不定,依次解释为与世隔绝、物质匮乏和非人道化等原因。
但是,从总体上看,他趋向于\textbf{循环因果关系}的论述,其中每一个起作用的因素都证
明了其他的因素。苏联的官僚化源于国际革命的失败和国内战争期间工人阶级力量的消耗,
而官僚化又强化了后两者。共产国际受到错误的领导或遭受破坏,使苏联更加孤立。同时,
官僚政权镇压革命马克思主义批判家(即托洛茨基主义者),又损害了党的领导作用,使工人
阶级非人道化,因而进一步使共产国际破产。官僚机构的无效率和特权,使苏联的经济发展
同样被遗弃,反过来又证明了别处的“倒退”,并使苏联维持着不合时宜的经济政策。根据
托洛茨基的观点,可以通过恢复真正的布尔什维克主义来切断这种循环因果关
系。\textbf{他似乎从未怀疑过真正的民主集中制能够统治党,他理想化地想像这个党由无
  产阶级“哲学王”构成,这些“哲学王”与落后条件下必然产生的国家管理机关中的任何
  官僚阶级相隔离。根据这个设想,党将构成这个阶级真正的先锋,理论上生机勃勃,为世
  界革命和迈向社会主义的经济发展制定正确的战略与策略。}

托洛茨基关于世界资本主义经济的观点在1914年以前已经明显成形,而他对布尔什维克主义
的许诺在国内战争期间得到兑现(参见本书第一卷第12章)。但是,\textbf{列宁却早于托洛
  茨基觉察到党内的官僚化。}从总体上看,托洛茨基遵循了列宁的最初理论,但即使在早期
阶段,他们也存在判断上的差异。他们虽然都把官僚化看作是一种“蜕变”,不认为存在一
个明显的官僚阶层,\textbf{但托洛茨基认为它是一种新的现象,不像列宁把它看作是经济
  文化落后的产物。}1923年年底,托洛茨基把它看作是政党组织与国家机器不恰当地结合在
一起所产生的问题,其中难免有一些官僚化现象。这种融合使党的政治任务和理论任务从属
于行政任务,通过损害党内民主来排斥有效的批评。官僚化加剧了由与世隔绝和新经济政策
造成的危险。

在20世纪20年代余下的时间,托洛茨基集中探讨的是\textbf{党的官方政策存在的复辟危险}的
问题。在这个时期,他对官僚化的分析\textbf{越来越具有唯物主义特征},他开始把官僚化
与非无产阶级的压力联系起来,把官僚政权看作是由个人组成的明确的阶层,这些个人的出
身背景和利益使他们可能与在新经济政策下成长起来的资产阶级分子联合起来。与此同时,
托洛茨基开始把斯大林“一国社会主义”的教条,看作是适合于官僚政权希望稳定的一种意
识形态。他也看到官僚政权内部的分化问题。斯大林集团被归为“中间派”,它对反对阶级
相互冲突交替的影响作出反复无常的反应,而另一方面却更为坚定地亲近富农。

最初,托洛茨基把斯大林在1928年至1929年对新经济政策的背弃和1929年后“自上而下的革
命”说成是“左倾”。党的领导者\textbf{听任资产阶级势力增强},致使形势处于反革命的
边缘。但是,\textbf{左翼反对派的压力已经使官僚机构分化,迫使中间派为对抗富农和维
  护其传统利益而向左转,中间派的传统利益与真正革命的利益是相对立的。}在这种估计的
背后,托洛茨基一直拒绝承认官僚统治能够成为历史上独立的力量。尽管他开始认识到官僚
统治自身的特权依赖于非资本主义的经济关系,\textbf{“一国社会主义”的教义已经抛弃
  其早期所带有的布哈林主义的痕迹,但他认为,用官僚统治本身来维持那么一个既非资产
  阶级又非无产阶级的发展前景是不可想象的。}他相信,对左翼反对派的持续镇压,预示着
不久的将来必然会向右转。

由于深信官僚统治根本没有真正的理论,托洛茨基对上述观点作了强化。\textbf{没有党内
  民主,就缺乏与无产阶级的紧密联系,党只能根据当前形势的压力“经验性地”行事。}所
以,\textbf{官僚统治无法预见到由其疏忽和残酷策略所引起的灾难。}由于集体化是在缺乏
先进的技术基础的情况下推行的,它也就不可能终止农村的两级分化,富农肯定会重新出现。
托洛茨基坚持认为,冒险主义者的工业化速度不符合按比例发展的要求,是不可能持续下去
的。

因此,斯大林发动自上而下革命的企图,受到各种矛盾的困扰,政权也奄奄一息,行将结
束。

托洛茨基的立场观点还有一个没有得到承认的特点,这就是\textbf{他提出的经济政策与布
  哈林更为接近}。尽管托洛茨基毫不含糊地支持斯大林反对右派,但同时他又同布哈林一样
对正在出现的斯大林体制提出了许多批评。托洛茨基和布哈林都相信,\textbf{行政措施无
  法摆脱经济落后,计划需要市场关系来补充,从长远来看,资产阶级成分只能在显示出社
  会主义关系优越性的基础上通过“经济”手段来消灭。}因此,布哈林和托洛茨
基\textbf{都赞成恢复新经济政策},将其作为与过渡相适应的经济形式,这也使斯大林在某
种程度上相信,右翼和左翼反对派结成了反革命联盟。

尽管托洛茨基从未与激进派达成一致,接受斯大林自上而下的革
命,\textbf{但在20世纪30年代中期,他还是承认这一革命最终取得了部分成功。}同时,他
相信国际革命的命运要求对苏联政权的性质作出重新评价。这时,他区分了官僚化的内部作
用与外部作用,试图更准确地理解其特征。纳粹在德国夺取政权似乎使托洛茨基的观点开始
发生变化。他一直认为莫斯科对共产国际的控制是无力的。但在1933年,托洛茨基向前迈了
一步,\textbf{承认苏联的蜕变已经使官僚统治发展到有意损害国际革命的程度},这有利于
帝国主义的生存并维护其自地位,而这一地位可能受到其他地方成功的社会主义革命的威胁。
因此,苏联的官僚统治成为一个外在的反革命力量,共产国际破产了,需要建立一个新的国
际组织。

从1933年起,托洛茨基开始把斯大林主义者描绘成\textbf{热月党人}。\textbf{以前他
  用“热月”这一概念指的是取得成功的资产阶级反革命的第一阶段},并否认苏联实际已经
出现这种情况。现在,他运用这一术语只是描述保守派的稳定性,认为苏联的热月政变早
在1923年就已经开始。

斯大林所代表和领导的官僚统治,现在起到了内部的双重作用:\textbf{它保卫了后资本主
  义财产关系,但却是以保留自身特权的方式来实现的。因此,尽管官僚统治已经背叛革命,
  但还没有颠覆革命。}

这成为托洛茨基对作为工人国家的苏联进行辩护的关键所在。\textbf{1933年之前},他曾为
支持苏联寻找理由,\textbf{认为这种蜕变可以通过改革加以纠正}:左翼反对派可能在党内
获取领导权,如果这样,就\textbf{能以和平方式铲除权力机器}。\textbf{现在},托洛茨
基抛弃了这种观点,认为这个党就像共产国际一样,\textbf{是不可能获得新生的。有必要
  发动一次革命,以布尔什维克组织取代二者。}但是,托洛茨基还认为,这场革命必
然\textbf{是“政治性的”,而不是“社会性的”}。工人国家的经济基础还没有变化,同以
前一样,发生蜕变的\textbf{仅仅是上层建筑}。因此,他的无产阶级专政概念现在只涉及
到\textbf{财产关系}。他承认苏联政权与法西斯主义政府一样采取独裁和恐怖手段行事,从
这里可以清楚地看出托洛茨基沿着上述方向走了多远。但他又坚持认为,从阶级实质上看,
由于苏联政权和法西斯主义政府所赖以存在的财产关系不同,因此它们又属于不同的两
极:\textbf{一个是蜕变的工人国家,另一个是垄断资本的独裁统治。}为了证明这种解释,
必须从狭义上理解财产关系,而托洛茨基最后把他的理论观点建立在纯粹法律意义的财产关
系概念上。这种形式主义在他的著作中并不典型,但它确实反映了他的马克思主义观不仅与
列宁和斯大林有相同之处,而且与第二国际的大多数理论家也有相同之处。他们都倾向于认
为现实中的社会关系基本上不受成功的无产阶级革命的影响。可能有更好的工作条件和更高
的报酬,但工厂中的生产技术和管理结构与资本主义在很多方面是相同的。

\textbf{托洛茨基对工人国家的定义是,它不只意味着工人实际上控制着这个国家,而且实
  际上还排除了那些掌权者成为一个新的阶级。官僚阶层是一个在无产阶级内部成长起来的
  寄生者。}为了证明这一论点,托洛茨基断言官僚阶层在生产中不发挥独立的作用,它所代
表的蜕变仅仅涉及到分配关系和权威的实施。即使是在这些方面,官僚阶层也不具备把他们
的特权传给其继承人的阶级特征。\textbf{托洛茨基没有排除官僚阶层通过建立私人所有制
  而成为一个新资产阶级的可能性,他倾向于认为官僚阶层的利益是建立在国家化所有制基
  础上的一个独特层面。}他的分析的关键现在就是\textbf{资本主义在苏联的复辟只可能通
  过外部干涉来实现。}

所以,托洛茨基非常接近于承认构成斯大林“一国社会主义”基础的内部和外部矛盾的二分
法(参见以上第2章)。他与斯大林的区别仅仅在于如何解决这个矛盾。托洛茨基相信,内部矛
盾不可能通过官僚统治得到彻底解决,官僚统治的对外政策也不会增强工人阶级国家的安全。
事实上,他坚持认为官僚统治处在国际前线会迫使其处于威胁自身长期生存的境地,苏联就
是如此。它的利益使它抛弃国际革命事业,但在这样做的同时,官僚统治也增强了帝国主义
的力量。与此同时,国内政策也限制了经济发展。\textbf{到20世纪30年代中期,托洛茨基
  承认工业化是迅速的,计划经济已经在实践中得到证实。但他又坚持认为,官僚统治妨碍
  了它全部潜能的发挥。中央试图控制一切,对镇压的依赖以及把资源转移到优先发展的部
  门中去,所有这些都表明生产力受到官僚关系的限制。尽管这一观点与托洛茨基所说的官
  僚统治在生产中不能发挥独立作用的观点不相一致,但他以此坚持生产力与生产关系之间
  的矛盾是大清洗的原因,它表明这个政权处于危机之中。}

托洛茨基从来没有明确解释经济基础中的对抗是如何与上层建筑的混乱联系起来的。他相信
对反对派的镇压处于恐怖统治的核心,但他也正确地认识到,对反对派的镇压也延伸到了形
形色色的斯大林主义者中间。托洛茨基长期以来一直坚持认为,单个的独裁者可以克服内部
分化从而有利于官僚统治。由于官僚同盟似乎正在崩溃,他倾向于把大清洗看作是导致了官
僚统治的内部争斗。但他没有清楚说明这一观点如何与他关于官僚组织阻碍了生产力发展的
观点联系在一起,后一观点是他长期所坚持的官僚统治不可能稳定论调的核心。

不管如何令人难以置信,托洛茨基因此而得以重申他的观点,即\textbf{只有实现社会主义
  革命的扩展,并伴随真正布尔什维克的政治实践,才能最终拯救俄国的工人国家。}但即使
在这个问题上,他的论述也是\textbf{矛盾}的。\textbf{他对布尔什维克主义的看法不仅过
  于理想化,而且还与他在1917年期间及以后对布尔什维克行为的批判相悖。}托洛茨基认为,
如果没有列宁,这个党就无法掌握政权,而在列宁生病和去世后,布尔什维克的许多领导者
在“热月事变”中都发挥了积极的作用。这样,托洛茨基对布尔什维克主义的信奉最终归结
为对\textbf{“领导者”的信奉问题},而列宁去世后,作为蜕变的核心事件,他自己也落到
失去权力的地步。他不断否认这是事实,但是他自己的思想逻辑却证明了它。另
外,\textbf{在20世纪30年代中期,托洛茨基隐约地(虽然是半心半意地)承认,即使在列宁
  的领导下,布尔什维克主义也是有欠缺的,因为党在1921年禁止派系斗争曾促进了官僚化
  的发展。但是,由于他还相信这种禁止是必要的,这并不影响他在立场观点上的一致
  性。}1936年,他又补充说\textbf{一党制政权可能不是无产阶级专政的适宜形式},并准
备考虑允许其他政党发挥作用的可能性。\textbf{但托洛茨基把这种自由局限于“苏维
  埃”政党},并且也没有扩展到这些政党有权利实际执政的地步。

托洛茨基只是在其\textbf{一生的最后两年}才考虑对布尔什维克主义进行\textbf{激进的重
  新评价的可能性}。但即使在这时,他对布尔什维克主义本质要点的忠诚仍然是很显然的,
因为他把它与马克思主义本身的有效性联系在一起。托洛茨基感到置疑的原因是\textbf{布
  鲁诺·里齐}在1939年提出了\textbf{官僚集体主义理论}。里齐认为,苏联形成了一种新的
生产方式,这种生产方既不是资本主义的,也不是社会主义的,\textbf{更不是过渡性
  的}(参见本章第3节)。它也正在法西斯国家内部出现,在资本主义国家的集体主义趋势中
也十分明显。令人惊奇的是,尽管托洛茨基以前用相似的思想对付其他批判者,但他并没有
立即抛弃里齐的这种观点。相反,他承认官僚集体主义在理论上是可能的。

人们并不清楚托洛茨基对待官僚集体主义的思想有多认真。他在1939年和1940年重申了他
在20世纪30年代中期已经讲清楚的立场观点,并且1939年发生的一些事件使他再次确信他的
分析是合理的,这样做又使他陷入自相矛盾之中。\textbf{在希特勒—斯大林条约之后,波兰
  东部的苏维埃化表明官僚主义确实是一种外部的革命力量。它不仅在国内继续保护国家化
  所有制,而且还把其生产方式向苏维埃国家以外推进。托洛茨基以此证明斯大林主义仍然
  代表着无产阶级专政,尽管这削弱了他早些时候把官僚主义称为国际反革命力量的观点。}

托洛茨基在这一时期还写下\textbf{遗嘱,声称他对共产主义理想的信念不可动摇,并写下
  了对辩证唯物主义的教条主义辩护,}反对詹姆斯·伯纳姆和马克斯·沙奇托曼的复辟主义思
想。他尽可能相对温和地对待官僚集体主义,这反映了里齐还没有对第四国际在帝国主义进
攻的情况下无条件地为苏联辩护的立场提出置疑。当其他人这样做时,托洛茨基毫不留情地
对他们的思想予以批判。里齐也没有批判辩证法,他对马克思主义的信奉显得模棱两可,但
他暗示官僚集体主义也可以表现得与历史唯物主义相一致,某一个阶段仅仅会延缓一些国家
努力走向社会主义。托洛茨基显然不愿意迎难而上,并以不甚乐观的态度看待官僚集体主义
出现的可能性。他坚持认为,苏联最终是一个蜕变的工人国家,还是一个新的官僚集体主义
生产方式的问题,这将会在第二次世界大战结束后自行解决。这样,\textbf{托洛茨基就制
  定了一个判断其关于苏联的理论是否有效的标准},这一标准同他对整个帝国主义时代的分
析的判断标准一样:\textbf{如果战争没有引发新的社会主义革命,或者如果它巩固并扩展
  了苏联的现有体制,那就表明无产阶级无法成为一个新的统治阶级。马克思主义者到那时
  将被降低到仅从人道主义立场出发反抗极权主义奴隶制,共产主义解放将没有任何希望。}

\section{官僚集体主义}
布鲁诺·里齐在1939年与托洛茨基决裂之前的许多年中,对托洛茨基关于苏联的分析是赞同的。
究竟是什么促使他这样做还不清楚,但他关于\textbf{国际无产阶级无力夺取政权——或者,
  更准确地说,在革命后无力将政权握在自己的手中}——的信念已经明显包含在其著作《La
Burearcratisation du Monde》之中,并且这也同1914年以来工人阶级政治斗争的历史相一
致。里齐当然不怀疑资本主义处于腐朽的过程之中,他也没有对托洛茨基关于苏联发展的事
实提出质疑。另外,里齐并不孤立,其他许多托洛茨基主义者也得出同样的结论。詹姆斯·伯
纳姆于1941年发表了与里齐极为相似的论文,他还和马克斯·沙奇托曼于1940年一起辞去了在
第四国际美国分部的职务。

《La Burearcratisation du Monde》的中心内容是\textbf{对托洛茨基关于“无产阶级专政
  唯一的必要条件是国家要捍卫国家化所有制,政治组织的确切构造仅仅反映了无产阶级统
  治的纯粹程度或蜕变程度”的观点进行有力的批判}。里齐正确地指出,\textbf{这些命题
  依赖于托洛茨基没有进行充分论证的信念,即在当今时代只有无产阶级或资产阶级才能掌
  握政权。}一旦这一点受到怀疑,就不可能保证无产阶级专政能够实现,除非工人阶级自己
实际行使着权力。\textbf{第三种力量也可能掌握政权,它既非无产阶级也非资产阶级。}

里齐把这一观点与他对财产所有制概念的考察结合在一起,\textbf{他把财产所有制看作是
  一种权力形式。}由于苏维埃国家的控制权掌握在官僚手中,而官僚阶级与无产阶级在利益
上存在对抗性——这两点托洛茨基都承认——因此\textbf{集体财产是官僚的财产,而不是无
  产阶级的财产}。苏维埃的官僚于是成为\textbf{它并不象托洛茨基所说的那样仅仅是一种
  产自于并依附于无产阶级的寄生物。}从其财产的本质上看,\textbf{官僚在生产中有独立
  的地位}。他们根据一整套明确的非无产阶级利益来编制五年计划,确定价格和工资,作出
投资决策以及消费决定。\textbf{苏联工人因此受到剥削,他们的剥削形式是历史上所没有
  的。}官僚统治并不运用资产阶级的方法。相反,这种关系\textbf{直接是“阶级与阶
  级”的关系};统治阶级不象在资本主义中那样被分成单个企业。\textbf{官僚统治运用国
  家本身去榨取剩余劳动,然后按照合乎自己利益的特定规则进行分配。}

尽管苏联因此代表了一种独特的生产方式和一种相应独特的阶级结构,但在里齐看
来,\textbf{它还是一种更一般现象的高级形式}。他觉察到,在法西斯主义和美
国“新政”中,都具有同一集群的胚胎形式。\textbf{里齐坚持认为,马克思曾正确地认识
  到资本主义不可能容纳现代生产力,它们的进一步发展确实要求实行计划经济和财产国有
  化。}但是,\textbf{马克思也错误地相信这就是社会主义的组织形式和无产阶级的统
  治。}如果这些确实实现了——\textbf{里齐认为它们最终会实现——也只能是在后集体主义
  社会才能做到;后资本主义社会就是官僚集体主义。}

里齐在1939年的一篇论文具有不容置疑的逻辑。它与大萧条的经历、工人阶级的一系列挫折、
苏联的巩固和纳粹德国的迅速军事化等相吻合。即使在战后时期,也有充分的证据证明里齐
著作的普遍冲击力。的确,法西斯主义已经被击退,但苏联是完成这一任务的主要力量,与
此同时西方资本主义经济中国家的作用明显增长。当然,里齐的观点也还有重要的缺陷,其
他一些马克思主义者通过指出这些缺陷,不仅能够修正官僚集体主义的思想,而且能够更好
地理解苏联生产方式。

\textbf{里齐根本没有探索苏联走向官僚集体主义的历史路径。}他在一些地方重复了托洛茨
基的观点,在另外一些地方他又偏向于麦克哈吉斯基和拉可夫斯基。\textbf{更为重要的是},
纳粹主义的溃败,\textbf{西方资本主义的蓬勃发展},以及苏联经济模式向东欧、亚洲和加
勒比海的发展,\textbf{这些都表明官僚集体主义迄今已经成为一种普遍的现象,它与里齐
  的看法存在差异。}里齐从来没有解释苏联为何完善到他所认为的最发达的生产方式,这使
得对他的批评得到了加强。一些马克思主义者自然而然地把官僚集体主义与落后联系起来,
而不是将它与现代性联系起来。

另一方面,对于苏联来说,里齐理论的一个方面更适用于经济成熟时期,而不是斯大林主义
转变时期。鲁道夫·希法亭于1938年通过对阶级概念是否真正适用于苏联的置疑,提出这一问
题。\textbf{希法亭认为苏联的生产方式既非资本主义又非社会主义;相反,它与法西斯主义
  相似,}并且他相信它已经\textbf{颠倒了历史唯物主义内在的因果关系结构}。在“专制
的国家经济”中,与“有组织的资本主义”一样,\textbf{政治支配着经济}。希法亭宣称官
僚并不是一个统治阶级,\textbf{它是以斯大林为核心的精英的政治工具},它通过清洗成功
地控制了国家机器,并运用它为自己的目标服务。希法亭并不清楚这些目标是什么,仅仅提
到掌握政权的愿望,但他的观点即使没有这一点也是有效的。\textbf{里齐忽视了恐怖用作
  反对官僚统治的手段的作用,而只是把它看作官僚统治用于对另一些人进行攻击的工具。}因
此,他认为官僚统治不仅具有相似的结构性,而且还表现出相当程度的凝聚力,事实上在斯
大林的统治下\textbf{却缺乏这种凝聚力}。换言之,在里齐看来,官僚统治不仅构成一
个“\textbf{自在的阶级}”,而且构成一个“\textbf{自为的阶级}”。\textbf{事实上,
  只是在20世纪60年代勃烈日涅夫宣布“稳定干部队伍”的政策时,官僚主义似乎才具有了
  自己的阶级意识和自治组织。}这一过程是从50年代非斯大林化开始的,当时创立了某种近
似于“法律准则”的东西来规范官僚统治内部的分歧。但是,即使在赫鲁晓夫的统治下,单
个官僚也存在着职位上的不稳定性,而巨大的行政组织重组打破了已有的惯例。只有到勃列
日涅夫统治时期,这种混乱状况才明显减缓。因此,里齐关于官僚统治的概念,以及从里齐
的著作中得出的“新阶级”的理论,在斯大林去世后形成的更为稳定的状态下才最清晰地显
现其有效性。\textbf{斯大林本人倾向于分裂官僚统治,其目的是为了避免出现托洛茨基所
  警告过的、也是里齐所宣称的那种阶级的形成。}

后来,\textbf{蒂克亭}对希法亭的观点作了概括,并且也为这一观点提供了\textbf{唯物主
  义的基础}。蒂克亭接受了苏联既非资本主义又非社会主义的观点,认为正是其组织结构表
明了在官僚统治的核心和外围之间存在着结构性划分。苏联模式的运行规则依赖于核心层追
求更高产量的压力,\textbf{而下层官僚却由于自身处境的内在原因对此进行抵制}。因此,
蒂克亭认为,\textbf{没有任何主观或客观的标准可以将作为一个整体的官僚统治看作是苏
  联发展的某个阶段上的一个阶级,}不论是在斯大林时期还是在后斯大林时期都是如此。这
是一种极端的立场。\textbf{马克思主义者一直认为,任何一个统治集团内部都可能发生分
  裂,而且认为整体统一性并不是统治阶级存在的必要条件。}蒂克亭没有解释为什么苏联官
僚统治内部的斗争如此重要以致于会阻碍官僚统治形成为一个阶级。另一方面,马克思主义
理论家根本没有为解决这个问题提供一个一般的标准。然而,更为重要的是,\textbf{蒂克
  亭的观点在更深层次上切断了里齐的思想,因为他阐明了官僚统治的内部斗争怎样使苏联
  体制产生巨大的浪费。这个方面完全为里齐和伯纳姆所忽视(但托洛茨基和布哈林并非如
  此)。}它并不是一个偏离了正题的问题,因为他们正是求助于官僚集体主义的所
谓\textbf{效率},试图把它解释为一个最终将取代资本主义的世界力量。这更使人们认为斯
大林主义是落后的产物,而不是现代性的产物。

里齐关于再生产的动力的观点也不甚清楚。他不仅假定官僚集体主义形式能够维持自身发展,
而且它将\textbf{迅速成长}。里齐一度认为这是任何一种生产方式的所有制关系,他的这一
观点显然是错误的。在别的地方他又追随托洛茨基,认为生产力发展的基础正是官僚统治扩
展其特权的利益所在。这一解释更为合理些,但它没有说明苏联集中发展重工业的原因,而
这正是苏联经济发展的突出特点。\textbf{希法亭关于苏联经济的关键在于权力政治的观点
  是有启发性的,尽管还不够确切,因为他没有具体阐明权力运用的目的是什么,以及掌权
  者攫取权力的动力是什么。}

\textbf{只关注一种生产方式的最高点,不能完全理解这种生产方式的特点。从马克思主义
  的观点看,从属阶级的本质至少同样重要。}但是,里齐几乎把注意力完全集中于新统治阶
级,而\textbf{没有注意到被榨取剩余价值的被压迫阶级}。在这方面,伯纳姆、希法亭以
及“新阶级”理论家都是追随他的。实际上,形形色色的政治分析一直存在着一种学习苏
联“自上而下”方式的趋势。里齐关于斯大林时期苏联主要附属阶级的说明并没有很大的启
发意义。\textbf{他否认苏联工人组成了一个无产阶级的观点,相反,他相信奴隶制是他们
  生活条件的更准确的描述。}

这不仅与他对社会主义未来的看法相左,而且这只是从表面意义上看待集中计划和集权主义
思想。\textbf{但是,即使是在斯大林时代的鼎盛时期,由于对劳动力的大规模过度需求和
  企业之间为各种类型的投入而展开竞争,产业工人仍然保留了一些经济上的自由。这表明
  苏联模式中主要生产阶级的成员更接近于雇佣劳动者,而不是奴隶,并且这也表明该生产
  方式本身可能并非自成一类,相反,它是一种特殊类型的资本主义。}


\section{国家资本主义}
马克思和恩格斯不止一次认识到\textbf{资本主义会以各种各样的形式出现。同时,他们认
  为这种生产方式的特殊运行机制会导致迅速的结构变化。}在对此进行深思以后,恩格斯甚
至声称马克思的观点防止了僵化的概念性定义,避开了过程的结构问题。\textbf{他这么做
  暗示了通过求助于已有的教条来判断马克思主义新著是不恰当的。}更为重要的是,恩格斯
的这些话表明,要对一个变化的生产方式进行经济分析是极其困难的。

尽管第二国际的马克思主义者存在明显的教条主义特征,但他们的行动常常与其理论观点保
持一致,如同本书第1卷所印证的。\textbf{布尔什维克主义者}当然也是如此,\textbf{他
  们为自己在1917年发动革命辨护的理由是资本主义已经进入了新的帝国主义时代,它所提
  供的国家资本主义制度促进了向社会主义的过渡(参见本书第一卷第13章)。列宁甚至认为
  过渡本身就将涉及国家资本主义的一种修正形式,它只能逐渐地让位于社会主义的组织形
  式(参见本书第1卷第15章)。}这给他的左派批评者提供了弹药,他们要求同过去作出更为
激进的决裂,常常把“国家资本主义”这一术语当作对革命成就失望的象征来使用。与此类
似,许多孟什维克党人声称俄国的落后排除了资本主义发展形式之外的任何形式。这再一次
表明斯大林把左倾和右倾反对派等同于一个集团并不完全是凭空想像的(参见以上第2章)。

但是,\textbf{托洛茨基对任何形式的国家资本主义从来就不感兴趣,他特别批评了那些认
  为俄国是国家资本主义的人。他的一些批评是相当合理的;这一术语常常被用作一种反对的
  口号,而不是一种分析。}但是,在托洛茨基反对的背后,存在着某种更为根本的东西。他
仅仅把国家资本主义看作是资本主义发展的产物,而不是无产阶级革命的结果。希法亭在嘲
弄地把这一概念运用于苏联时,揭示了类似的观点。

然而,苏联是国家资本主义的观点最终确实破坏了战后的托洛茨基主义运动。\textbf{20世
  纪40年代,拉亚·杜娜耶夫卡娅和C.L. R.詹姆斯都离开第四国际,抛弃了托洛茨基关于苏
  联是蜕变的工人国家的分析,相反,提出了一种国家资本主义的观点。他们这么做是受到
  了“青年马克思”的《1844年经济学哲学手稿》的影响。他们根据异化劳动来分析苏联真
  实的生产关系,这成为他们证明其理论科学性的主要依据。}其他的马克思主义者——其中
的许多人最初还是托洛茨基主义者——持有极为相似的立场,\textbf{重振早期无政府主义反
  对国家社会主义的主旋律,认为正是布尔什维克主义导致了苏联的国家资本主义。尽管俄
  国革命不是一场资产阶级革命,但它可以被看作是由先锋队政党扮演了资产阶级代理人的
  一场资本主义革命。}毫不奇怪,托洛茨基主义者由于表示要忠于布尔什维克主义,他们有
时被认为是“忠诚的斯大林主义反对者”,甚至是“流亡的斯大林主义官僚”。另外,从这
一观点出发,苏联仍将是国家资本主义,即使苏联以议会政治的方式实现了民主化,并以列
宁在《国家与革命》中所阐述的最低纲领派的方式实施了工人阶级的管理。依据这一观
点,\textbf{雇佣劳动制度是资本主义的规定性特征。任何对自我管理方式的拙劣修补,与
  变更财产所有权的法律形式一样,都是与划分生产方式问题无关的。}社会主义意味着与前
社会主义方式的彻底决裂,它只能在革命的过程中被创造出来,不存在长期的过渡。毫不奇
怪,\textbf{这样的理论家倾向于认为西方的资本主义与东方的国家资本主义仅仅是程度上
  的区别。}在任何意义上,苏联都不是有计划的;商品关系无所不在;国家甚至在似乎是“私
人”资本主义的领域也是主导经济的力量。\textbf{法兰克福学派的大多数学者
  在20世纪30年代至40年代也得出相似的结论,尽管他们认为计划经济无论在苏联还是在西
  方资本主义国家都是一个现实。}对波洛克来说,苏联、纳粹德国和西方的罗斯福“新政”,
都是同一类型的不同例子。他的观点在实质上与希法亭、里齐和伯纳姆的观点没有什么区别,
尽管对国家资本主义的命名法有所不同。但是,\textbf{法兰克福学派的思想家们不断地向
  这一观点靠拢,认为工具理性的支配是使所有现代社会受到压抑的真正力量,不管这一现
  代社会的形式如何。}这种观点最著名的版本是由赫伯特·马尔库塞在20世纪60年代提出的,
但这种思想早在30年前就已成形。\textbf{至于苏联马克思主义,马尔库塞在1956年就揭示,
  它以促进工业社会的发展代替有人道的世界,它也在这一过程中发生转变。}但是,他没有
对苏联生产方式的具体本质,以及相应的阶级结构进行深入的分析。这与法兰克福学派的一
般观点相一致,认为这是次要的问题。

事实上,\textbf{这些理论家都没有提出具体的国家资本主义政治经济学,}以详细阐述其特殊的经济
动力、矛盾和所有制关系的转变等问题。在斯大林主义时期,只有托尼·克利夫和后来的社会
主义工人党所提出的国家资本主义理论,试图从根本上详细研究这些问题,但这一研究存在
大量的缺陷。从总体上看,这一理论是有启发性的,但它从来没有向外延伸,没有结合斯大
林主义和后斯大林主义发展的复杂性进行研究。

在20世纪40年代末之前,克利夫也是一个正统的托洛茨基主义者,但后来他像其他马克思主
义者一样,认识到苏联生产方式在地理上的拓展是有悖于托洛茨基的一般观点的。他在修正
自己的观点时,也许受到过里齐的影响,尽管在认为苏联是国家资本主义问题上,克利夫与
第四国际完全不同。50年代,克利夫把这一理论运用于分析全部所谓的共产主义社会,包括
中国在内,并且也采纳了一种军备生产的分析方法,解释了西方资本主义长期繁荣的原因(参
见以下第8章)。因此,在所有坚持国家资本主义的理论家中,他试图效仿托洛茨基宽广的视
角,对这个时代从整体上作出经济上的理解。但是,克利夫的经济思想更接近于布哈林的早
期思想,而不是接近于托洛茨基的思想,他的基本思想可以追溯到布哈林的《帝国主义与世
界经济》一书。

\textbf{克利夫仅仅用两个特征来描述资本主义:竞争;生产阶级与生产资料的分离。}克利
夫假定苏联生产方式中存在雇佣劳动,但他没有赋予其特别的意义。相反,所有一切都以西
方的军事威胁为转移,这种军事威胁最终迫使与世隔绝的、落后的俄国以资本主义的方式进
行积累。斯大林的“自上而下的革命”是一种新版本的原始积累,生产者在那里被完全剥夺
了对生产资料的控制权,而官僚统治却受到保护。从那时起,官僚统治通过快速积累从俄国
无产阶级身上榨取的剩余价值,承担起典型的资产阶级的职责。

\textbf{对克利夫来说,苏联是世界资本主义经济中一个巨大的资本单元},并且他坚持认为,
脱离这一现实就不可能正确地理解关于苏联生产方式的经济学。例如,\textbf{苏联显然没
  有价值规律据以产生的内部结构}:商品生产十分微弱,价格和利润仅仅是计算的工具,资
源配置通过行政命令来实现。只有从世界经济的角度,这些制度才表现为资本主义内部竞争
的替代物,\textbf{间接地执行着“价值规律”}。克利夫似乎得出以下结论:\textbf{如果
  苏联生产方式不遵从价值规律,它就不能抵御私人资本主义的压力,也不能建立起充分强
  大的武器库来抵御帝国主义的侵略。}

\textbf{苏联生产方式在第二次世界大战中以胜利的姿态出现,接着又将西方资本主义置于
  自己的威胁之下,两种形式的资本主义都被迫进行扩军备战。}在克利夫看来,这最终挽救
了西方资本主义制度,防止重新回到20世纪30年代的衰退状况,并在1945年后维持了“长期
繁荣”。所以,\textbf{列宁曾错误地认为帝国主义是资本主义的最高阶段},它只是资本主
义倒数第二个阶段。\textbf{“持久军备经济”所蕴含的矛盾,最终将开创世界范围的社会
  主义革命(参见以下第二篇,特别是第8章)。}克利夫直到那时还认为,落后资本主义的矛
盾只能导致国家资本主义结构的形成。这些与先前的观点相比可能是进步的,但它们并不代
表无产阶级专政。其中的原因与列宁主义的原则无关,而与缺乏足够强大的无产阶级队伍有
关,与西方资本主义(或其他国家资本主义)对落后社会的压力有关。

\textbf{克利夫关于西方的军事威胁对于苏联经济发展具有重要意义的核心思想},至少
自19世纪80年代以来成为马克思主义持续关注的主题(参见本书第1卷第二篇)。波洛克
在1941年明确地提到这一思想,希法亭在20世纪30年代也很了解这一思想。另外,一些后来
的马克思主义者既不接受里齐对国家资本主义的特殊概念,也不接受里齐从他的分析中得出
的大胆论断,他们一直强调苏联的军事化。而且,大多数反对里齐的官僚集体主义理论的批
判者,也运用了克利夫的国家资本主义理论。克利夫的理论也存在其他一些特有的缺陷。最
有说服力的反对意见是,\textbf{他的理论高度抽象化,对历史特性没有予以足够的关注。}例
如,\textbf{欧内斯特·曼德尔有力地证明,资本主义的竞争是一种只可能发生在商品生产体
  系中的竞争形式,它的后果超出资本主义之外是不可想象的。在人类历史之初,竞争就以
  各种形式存在着,但它只有在现代条件下才发展为主要是资本主义的形式。}

还有一个进一步的相关问题,即\textbf{确定苏联反革命的确切发生时间}。克利夫本人论述
列宁的巨著指出,\textbf{在国内战争结束之前或者更早,官僚化作为一种结构,在列宁和
  托洛茨基的领导下已得以确定。}但是,克利夫按照他本人的布尔什维克的观点(这种观点
与托洛茨基的观点并无二致),则声称官僚统治或国家资本主义宗派,到20世纪20年代末期才
取得统治地位,最终成为一个“自为的阶级”。\textbf{第二个问题是为什么大清洗吞噬了
  那么多的新阶级,而不仅仅是他们的反对派,并且为什么官僚统治的主要成员即使
  在1938年后仍然感到没有保障。}从这一点看,托洛茨基的分析即使存在一些局限性,但确
实是胜人一筹的,因为\textbf{托洛茨基至少隐隐约约地感觉到斯大林主义恐怖统治的复杂
  性,而克利夫忽视了这些。}

许多重要的官僚自己也成为斯大林恐怖统治的牺牲品,这也成为20世纪50年代非斯大林化背
后的一股主要政治力量。这里也有重要的经济原因,克利夫对此进行了较为详细的研究。但
是,令人惊奇的是,他们几乎根本没有对克利夫关于斯大林统治下的苏联和国家资本主义的
观点作任何修正。即使20世纪80年代末期最激进的措施,也被认为与20世纪50年代的做法相
似;它们仅仅是重建官僚统治和苏联生产方式的手段,以便更好地迎接其他资本主义国家的竞
争,并且这些激进的措施将具有更为重大的意义,只要官僚统治中的改革派为了赢得广泛的
支持,不经意地发动自下而上的革命。否则,这些变革将是无关紧要的,从其内部结构上看,
苏联生产方式的动力显然没有什么变化。

克利夫和他的同事们没有对这一戏剧般的结论进行一般性的理论阐释,并且这一结论似乎明
显是非马克思主义的。但是,在20世纪70年代期间,伊曼纽尔·沃勒斯坦的世界体系理论提出
了一个概念性的总体框架,试图为它辩护并且还声称它的结论符合马克思主义的观点。这些
思想作为新帝国主义理论和第二次世界大战后出现的全球经济学的一部分,以下第9章将对此
进行讨论。第18章将再次论述这一问题,讨论社会主义是否仍然是苏联的可行方案。但是,
我们先要转而考察马克思主义对1945年以后西方资本主义的分析。


\part{长期繁荣}

\chapter{“资本主义发生变化了吗?”}
\section{正统马克思主义的反思}

\textbf{第二次世界大战结束后,}许多马克思主义(以及一些非马克思主义)经济学家预测,
一场经济衰退很可能像大萧条一样,在短暂和平后会马上发生。\textbf{12年以后,原先预
  想的危机并没有发生,发达资本主义国家仍然迅速而平稳地进行着资本积累,马克思主义
  者面临对其整个政治经济学进行重新审视的压力。}关于马克思主义与凯恩斯主义理论之间
的关系这一重要问题,将在下一章进行分析。在这里论述的是与资本主义本质变化相关的一
系列广泛的问题。马克思本人已经预见到资本主义的经济结构在一定时期会发生变化,但是
他对这一体制演变的界限(如果存在这一界限)很少予以论述。

20世纪50年代中期,爆发了一场\textbf{新的“修正主义论战”} (参见本书第一卷第4章和
第14章)。莫里斯·多布在1957年写道:“新费边主义著述者已经声称,资本主义或者已经进
入一个新的、变革的阶段,这一阶段与19世纪的资本主义有巨大的区别;或者甚至已经不再是
资本主义,它早已变为另一种东西。”在这些“新费边主义者”中,有许多杰出的政治家
如C.A.R.克罗斯兰和约翰·斯特雷奇。克罗斯兰是英国工党中“修正主义”思潮的思想领
袖,\textbf{这一思潮试图清除马克思主义对党的章程和行动计划的任何影响}。德国正在发
生类似的思想运动(其中,\textbf{反马克思主义的“巴德—哥德斯堡纲领”在1957年被采纳},
它反映出这些思想运动取得的成功,爱德华·伯恩施坦的思想重新焕发活力),其他所有发达
资本主义国家几乎都有一个重要的工党或社会民主党。

如同多布所承认的,马克思主义经济学确实要对这一情况作
出回答。资本主义在战后开头十年的持续繁荣,不可能完全归结
为战后的恢复:
\begin{quotation}
  至于所谓由战时状态和战时灾难导致的被抑制的需求,在20世纪50年代维持工业活动中的
  作用急剧下降。需要解释的是,特别是北美、西德等国在后来的三、四年中(即自美
  国1953—1954年在衰退中崛起以后),在面临着原先巨额军费开支逐渐停止、利率不断升高
  和信贷紧缩等情况下,个人投资为什么能持续增加。这种事实越突出,就越迫切地需要作
  出解释,由于在两次战争期间来自理论和实践的所有的东西都会使我们非常容易想到垄断
  资本主义。垄断资本主义越发展,工厂和设备的过剩生产能力以及投资和经济增长率的停
  滞趋势就越严重。
\end{quotation}
多布继续指出,新费边主义者对这些现象作了三个方面的解释。第一是“\textbf{管理革
  命}”,认为管理革命使工业的控制权从资本家阶级那里转移出来,掌握在\textbf{新的管
  理精英}手中,他们的行为使投资决策波动更小,从而刺激了私人投资的增长。第二是近几
十年的所谓的“\textbf{收入革命”},大大降低了所有发达工业国家中的经济不平等,提高
了平均消费倾向,从而刺激了总需求。最后是\textbf{国家经济作用的显著增强},极大地促
进经济的稳定性。\textbf{多布认为,前两个因素很容易被否定,因为资本所有者还保留着
  对其财产的控制,所谓的管理革命也是不合逻辑的,而收入平等化的进展极其缓慢。相反,
  自1939年以来的“国家垄断资本主义的大规模扩张”却是实实在在的,国家支出(尤其是军
  事支出)的扩张,在支撑过去12年的工业高产出和高就业方面起着重要的作用。}

\textbf{多布提出另外两个因素也促使了战后的复苏。}一个因素是“\textbf{内部积累}”。
它的资金来源是\textbf{企业保留利润},而不是银行提供的外部基金;这有助于鼓励大公司
的投资,使大公司的决策不再需要得到外部融资者的认可。另一个因素是\textbf{不断加快
  的技术创新},它同促进技术发生革命性变革的工业“自动化”进程相联系,提高了投资率,
降低了投资率在面对需求变化时的波动。\textbf{多布得出结论,第\Rnum{1} 部类扩张的结果抵销
  了消费不足的倾向,不然这种倾向可能对第\Rnum{2} 部类产生相反的作用。}但这并不意味着“上
升阶段像一些新费边主义者认为的那样,\textbf{可以无限制地进行下去},将一个创新‘阶
段’(我认为我们必须承认它的存在)转变为一个崭新的‘发展时期’”。\textbf{他在两年
  后指明,这个新阶段是“危机更为频繁,但也是更为短促和浅度的”阶段。没有证据表明
  资本主义的矛盾已经被克服。这些矛盾只是以新的通货膨胀的形式表现出来,另一次经济
  萧条当然并没有被排除。}

在苏联,尤金·瓦尔加在1945—1946年间写了一本书,这本书出版于1946年9月。他在该书中认
为,由于战争导致欧洲、中国和日本经济衰竭,主要资本主义国家至少在10年的时间里要承
担经济复苏的使命。\textbf{瓦尔加预计,国家的经济实力将会持续增加,如果立即爆发一
  场严重的经济危机,国家的强大作用就会体现出来。有意识的计划正在取代市场的无政府
  状态,国家有一定程度的决策权;仅仅用金融寡头统治已经不可能概括它了。}但是,瓦尔
加的理论被斥为异端邪说,因为\textbf{他的理论暗含着资产阶级国家能够超越商品生产的
  内在规律,把国家资本主义的发展趋势提升到一个新的、不再发生经济危机的资本主义发
  展阶段。}

他在1949年宣布放弃这种思想,而苏联在斯大林时代和赫鲁晓夫时代的官方观点仍然坚
持1939年以前的看法。\textbf{“垄断资本主义”自第一次世界大战以来存在着根本上的连
  续性。}在“资本主义制度的总危机”下,世界被分为“帝国主义”阵营和“社会主义”阵
营。资本主义不可能长期稳定发展,相反其矛盾变得空前尖锐,军国主义猖獗一时,公民的
自由权利连续受到威胁,无产阶级革命的可能性在不断增长。苏维埃阵线内也存在着模棱两
可的认识,特别是在资本主义暂时稳定的可能性、国家资本主义倾向不可抗拒地增强、帝国
主义阵营内部冲突的持续性等方面。然而,总的来说,苏联官方认为资本主义的本质没有发
生变化。

\textbf{非殖民化}也对传统共产主义观点提出挑战。帝国主义国家在1945年以后的20年中放
弃对亚洲、非洲和加勒比海等大部分地区的形式上的控制,乍一看这与希法亭—列宁的帝国主
义理论很难统一,因为根据后者的理论,瓜分和重新瓜分世界,是资本主义最高阶段的本质
特征之一。如果由列宁主义正统观点所证明的促使帝国主义扩张的经济矛盾曾发挥过作用,
那么殖民地得到解放难道不是清楚地证明,1914年之前的经济矛盾实质上已被克服了吗?(见
本书第一卷第13章)瓦尔加确实把印度的独立看得很重要,同时,新修正主义最有影响的著
作——约翰·斯特雷奇的《帝国的结束》认为,“霍布森—列宁”的帝国主义理论是建立在消费
不足基础上的,与20世纪宗主国资本主义国家中实际工资的巨大增长不相符。斯特雷奇认为,
老牌帝国主义强国不再从殖民地榨取利润。西德没有海外财富,但它在战后比法国恢复得更
快,法国为延续其帝国主义的虚荣而付出了沉重的代价。

马克思主义理论家对这种状况作出了各种解释,这些解释相互之间不总是一致。大多数人否
认欧洲强国已自愿放弃了它们的殖民地。英国共产党人R.帕姆·达特在1953年写道,“西方帝
国主义的破产”反映了以前的殖民者再也不能通过以前非常成功的把暴力和分而治之结合起
来的手段维持它们的统治。英国资本家阶级尽管不热衷于企图至少在某种程度上维持其帝国,
然而还是导致“经济和军事力量的过度扩张”,从而削弱了该国战后的复苏,大部分殖民强
国很快放弃了这种努力。如同保罗·巴兰在四年后所说的:“两次世界大战剥弱了帝国主义力
量,它们再也抵御不了殖民地的民族解放运动的压力,被迫向不可抗拒的历史潮流低头,承
认那些反帝力量最强大的、不可能指望进一步维护其殖民统治的国家,在政治上实现独立”(参
见以下第9章)。

早在20世纪60年代初,托洛茨基主义者\textbf{欧内斯特·曼德尔}从极不相同的角度阐述了
这一问题。曼德尔承认,\textbf{非殖民化是“宗主国家资产阶级向殖民地资产阶级作出的
  一个不可避免的让步”。但这一过程是同核心国家与外围国家之间经济关系的重大变化相
  一致的,其中,生产资料的输出比以前更为重要(同时消费品输出的地位降低)。}曼德尔认
为,\textbf{(前)殖民地资产阶级现在被看作是一个消费者,而不是一个竞争者,因此可以
  被允许在行动上有更大的自主权。前殖民地通过政府干预促进重工业的建立,这只能有利
  于西方机械工业的发展,西方也可以通过给贫穷国家提供经济“援助”得到补贴。}

\textbf{曼德尔、达特和巴兰一致认为,形式上的独立掩盖了(西方国家)对前殖民地国家经
  济政治生活的持续的非正式的控制。列宁本人就以土耳其、埃及和中国为例,证明纯粹名
  义上的国家主权掩盖了实际的依赖关系。}爱尔兰和伊拉克在两次大战期间加入了这个行
列;这个行列的名单从1945年起显著增加。因此,达特认为,英国金融资本继续占据印度的重
要经济部门,从中获得贡物,并且美国资本也在迅速发展。达特得出结论认为,帝国主义并
没有绝迹,它只不过采取了一种不同的伪装。\textbf{“新殖民主义”或者曼德尔所说
  的“新帝国主义”的新阶段已经出现(参见以下第9章和第10章)。}

这就是西方学者的观点。作为吸引新独立国家的官方意识形态,苏联马克思主义不得不更加
小心谨慎。所以,Я.А.克隆罗德在1961年一次国际讨论会的发言中,把非殖民化看作既是
一场真正的解放,也是对资本主义核心层的经济稳定的主要威胁:
\begin{quotation}
  由于世界范围内殖民体系的崩溃,前殖民地持有者已经丧失或正在丧失对其殖民地的物质
  资源进行不公正的、有利于自己的再分配的基础。这些国家工业结构的急剧变化是绝对必
  要的,因为它们失去了对殖民地的特权,被迫在“机会均等”的基础上参与世界市场上孤
  注一掷的竞争。另外,它们原来的殖民地正迅速地转变为世界经济竞技场上新的工业竞争
  力量。按这种理论,非殖民化加剧了发达国家资本主义的矛盾,这与列宁的《帝国主义论》
  的总论题是一致的。
\end{quotation}

\section{“资本主义发生变化了吗”}
克隆罗德是苏联出席1958—1959年度一次国际讨论会的代表,讨论的成果由日本经济学家都留
重人于1961年编辑出版。\textbf{都留重人在战前与保罗·斯威齐合作研究消费不足理论的发
  展问题,并以试图综合凯恩斯主义与马克思主义的宏观经济学而闻名。}都留重人在讨论的
议题中提出这样一个问题,即资本主义是否“经过不断进化,已经足以避免像1929——1933年
那种类型的大萧条”。美国已经实现20年的经济增长,而没有爆发一次经济危机。

\textbf{这不可能完全归因于战争和备战,}因为繁荣早已经受过1945—1947年、1953—1954年
军费支出的大幅度缩减。正如一些马克思主义者所认为的,\textbf{资本主义避免了萧条,
  不可能用新的技术进步的浪潮加以解释,}这一浪潮同康德拉季耶夫50年周期的上升阶段相
联系(见以上第1章第1节)。新的“科学—产业革命”最早从1954年开始,当时在所有先前的康
德拉季耶夫繁荣阶段都存在着传统类型的10年周期。都留重人承认,\textbf{经济政策的转
  变确实是其中的部分原因},因为1946年“美国就业法案”、银行改革、农产品价格支持以
及内部财政稳定器的增长,都限制了有效需求不足的程度。制度也发生了变化,其中最突出
的是美国工会争得的收入分配平等的增长,提高了平均消费倾向,只是部分地被固定价格的
寡头垄断部门边际利润增加所抵销。

都留重人本人强调军费支出在维持需求中的重要性,假设美国个人投资与国民收入之比最高
可能达到16\%。其他一些更为短暂的因素也发挥了作用,尤其是\textbf{美国的出口盈余(现
  在正在下降)和消费信贷难以维系地迅速扩张。}都留重人相信,政治上对政府增加民用支
出的反对,将阻止它增长到足以抵销军事部门的剧烈衰退。\textbf{繁荣要求高投资,进而
  要求高利润。但是,资本家会抵制任何侵占私人投资的危险,不管它是提高工资、增加福
  利,还是低成本的公共住房规划。}都留重人认为,\textbf{如果确实要维持经济增长,那
  么它只能以巨大的浪费为代价,其中有加速折旧、巨额广告费用和持久的军事化,这将有
  助于维持有效需求。}

\textbf{在改良的社会主义者中,普遍流行着一种资本主义与社会主义之间的区别越来越模
  糊的观点,}应该如何看待这种观点呢?都留重人坚决反对资本主义与共产主义之间会慢慢
地但又是不可逆转地趋同的观点。\textbf{一种生产方式的性质,是由谁控制剩余产品来决
  定的。}都留重人认为,\textbf{资本主义的基本特征包括四个方面:利润是经济活动的动
  力;利润由私人资本控制;利润在很大程度上用于积累;经济人有持续的压力,通过售卖商品
  而实现利润。}这些基本特征一个也没有发生显著的变化。大公司尽管存在着所有权和控制
权的分离,但仍然追求安全的、长期的利润最大化;\textbf{国家通过公司所得税只能获得一小部分剩
余产品;不断升高的提留比率降低了利润的消费倾向;销售面临的压力空前强烈。}都留重人的
结论是:“至少对美国来说,资本主义生产方式的基本特征仍然存在”。

都留重人的结论得到保罗·斯威齐和保罗·巴兰的支持和充实。\textbf{斯威齐强调,在垄断
  资本主义条件下,技术进步与投资之间的联系在不断减弱。}他认为,现在新技术的引入能
够从公司的折旧储备中得到资金支持,但对任何有效需求不产生任何刺激,\textbf{因此一
  个迅速的创新速度与经济停滞是完全一致的。}至于收入不平等的缩小,这完全是1945年以
前的事情;平等的增长不存在固有的或长期的趋势。最后,斯威齐驳斥了一种过分简单化的观
点,即不认为由于不断扩大的国家支出将有利于美国资本主义,所以它将不可避免地发
生。\textbf{这种观点既忽视了“意识形态上的眼罩”,这种“眼罩”使资本家错误估算真
  正的长期利益,也忽视整个资本家的长期利益和部分资本家的短期利益之间存在冲突。}在
斯威齐看来,\textbf{资本主义国家既不是中性的调解人,也不是联合的统治阶级的驯服的
  工具。经济政策的制定是不断的斗争的主题,}至少从美国的情况来看,增加公共支出的反
对者总的来说占据了上风。\textbf{保罗·巴兰}对消费不足理论作了更长的、更彻底的和更
为详尽的重新阐述。巴兰认为,\textbf{必须把消费不足看作是一种趋势,它可能被一些相
  反的力量抵销。}自1870年以来,美国工人的劳动生产率比他们实际工资的增长快得多。结
果作为总产出一部分的经济剩余有了巨大增长,并且越来越集中到数目逐渐减少的大公司手
中。所以,\textbf{存在着消费不足的趋势,不论是资本家的消费还是投资,都不能提供足
  以吸纳经济剩余持续增长的有效需求。为了抵制由此产生的停滞的压力,必须增加非生产
  性和浪费性支出,尤其是以私人部门的产品差异与广告费用,以及国家军事开支的形式出
  现。}因此,利润现在仅仅代表经济剩余的一部分;吸纳经济剩余的浪费性支出则是其余的
部分。这意味着\textbf{作为总产出的一部分,不论是利润没有上升还是消费没有下降,都
  不会否定消费不足理论。资本主义由于消费不足仍然趋向于停滞(参见以下第6章)。}

Я.А.克隆罗德、莫里斯·多布以及法国共产主义者夏尔·贝特兰代表了马克思主义的较为传
统的体系。克隆罗德的分析重复了尤金·瓦尔加在两次战争期间的观点(参见以上第1章)。克
隆罗德认为,尽管资本主义国家的作用大大增强,但是世界范围的生产过剩的危机仍不可能
长期地得以避免。\textbf{西方资本主义发生的所有的结构变化都是矛盾的。垄断的增长引
  起生产发展的速度大大快于市场扩大的速度,而国家支出的增加又是通货膨胀性的,它趋
  向于降低实际工资,进而限制工人阶级的购买力。“非生产性”(劳务)活动的增加仅仅暂
  时缓解了消费不足的压力。}贝特兰的观点在本质上与上述观点相似,否认资本主义国家能
够抑制资本主义基本经济规律的作用,并把1945年以后没有爆发经济危机归因于浪费性的销
售费用和军事开支的增加。

正如我们在上一节所看到的,\textbf{多布}的观点更为谨慎。他\textbf{在给都留重人编辑
  的专题文集撰写的论文中指出},\textbf{有两个极端的观点都是错误的:断言资本主义没
  有发生任何变化是错误的;认为资本主义已经进入一个全新的制度同样是错误的。}国家作
用的增强,技术的加速进步,以及金融资本的崩溃都是重要的发展,但是这些不能“在任何
意义上证明‘新阶段’言论的有效性,或者在任何基本方面改变我们对资本主义作为一个制
度的评价和对其未来的估计”。

为《资本主义发生变化了吗?》撰稿的马克思主义者,\textbf{没有一个人提到世界范围的
  协作},它最初曾受到诸如\textbf{马歇尔计划}的推动,并通过一些\textbf{较为固定的
  组织机构,如国际货币基金组织、世界银行以及关税与贸易总协定}而得到促进。直到20世
纪80年代中期,国际经济秩序显得日益脆弱,马克思主义者这时才意识到它的重大意义(参见
以下第16章)。也没有一个马克思主义者试图评析马克思的并非粗劣的教条与实际工资在一个
时期中稳步增长之间的相关性。这一工作留给了罗纳德·米克,他在1962年对马克思主义的工
资理论提出的尖锐批评,但从来没有受到更为正统的马克思主义著述家的有效反击。

在这次讨论中,J.K.加尔布雷思和约翰·斯特雷奇两人发表了两种不同的见解。加尔布雷思强
调经济集权的作用,但他自负的新凯恩斯主义的乐观,减弱这一观点对当代马克思主义思想
的影响。他的文章对其《美国的资本主义》一书并没有什么发展,他在该书中描绘了大公司
蓬勃发展的垄断力量如何与国家新的“抵销力量”、工会、农场主组织、消费者协会和“集
体配售” (连锁商店和超级市场)相配合。在总的方面,\textbf{斯特雷奇}的文章内容较为
充实,充分保留了战前的马克思主义观点(参见以上第1章),\textbf{他断言消费不足仍然存
  在:高利润既是充分就业的必要条件,(也因为它们抑制了消费需求)也是充分就业的毁灭
  者。寡头的增长使这种矛盾更加尖锐(而非缓和),因为价格刚性扩大了边际利润,使不平
  等更加恶化,并降低了市场的自我调节能力。}斯特雷奇认为,资本主义最终无法逃避这种
困境,但是,\textbf{如果工人阶级运用国家机器,通过贫富之间的收入再分配,通过增加
  对社会有用的民用工程的公共支出,这种困境就会得到一定的缓解。}西欧自战争以来,确
实在相当大程度上出现了上述情况。斯特雷奇认为,“新政”取得了部分的成功,美国资本
家最终将认识到他们真正的长期利益。停滞和增加浪费不再只是唯一的选择。

\textbf{欧内斯特·曼德尔}对上述论断给予了有保留的支持。他没有参加都留重人的讨论,
但他在20世纪60年代初出版了大量关于马克思主义政治经济学方面的著述。曼德尔撰写的编
年体著作《马克思主义经济理论》于1962年用法文第一次出版,他在该书中将始
于1900年的“\textbf{资本主义衰退时代}”作了这样的描述:\textbf{在这个时代,国家日
  益通过负责承办非赢利的基础产业,对私人资本提供直接或间接补贴,以及提供利润担保
  等措施来维持垄断利润。军事支出为重工业部门的产品提供了“替代市场”,从而有助于
  第\Rnum{1} 部类的稳定发展,而工会所争得的国家福利待遇和稳定的工资收入,又维持了对
  第\Rnum{2} 部类的需求。尽管现代资本主义经济仍然存在着重要的停滞力量,但是,国家的干预
  已经能够阻止1929年大灾难的重演。}曼德尔在两年以后出版的《马克思主义经济理论导论》
中宣称,国家干预的增强已经导致其所谓的“新资本主义”。\textbf{“新资本主义”的显
  著特征是经济计划以及对有组织的工人阶级采取容忍和妥协的政策。新资本主义从某种程
  度上说是一种暂时的现象,它与康德拉季耶夫长波的高涨时期联系在一起,并且很可能随
  着这个高涨期的结束而结束。但它反映了资本家避免第一次大萧条重演的迫切愿望,反映
  了他们认识到没有规制的市场机制是不可行的,新资本主义达到的程度就会到此为止。}然
而,它仍然是资本主义:

据此,我们可以得出以下结论:\textbf{国家干预经济生活、管制经济、经济规划、指导性
  计划,从社会的观点来看,这些绝不是中性的。它们是由资产阶级或资产阶级统治集团掌
  握的干预经济的工具,绝不是资产阶级和无产阶级之间的公断人。这一点更清楚地体现在
  国家试图实行工资计划方面。收入政策是资本主义的一项旨在降低社会产品中工资所占的
  份额来保护利润的措施,工会也被纳入这个制度内。}

\section{法兰克福学派与非经济分析的萌芽}
\textbf{对无产阶级溶合这一主题的研究,成为“批判性理论家”战后著作的特点。这些批
  判性理论家就是法兰克福学派以及受其影响的人。}1945年后的20年间,在积极活动的马克
思主义理论家中,批判性理论家从当代资本主义变化中得出最为激进的结论。\textbf{法兰
  克福学派深受法西斯主义和斯大林主义作为一种压制性的国家权力的思想体系而产生的影
  响,它抛弃了那些将自然科学作为范例的变形的马克思主义所包含的宿命论的、悲观的思
  想成分。}批判性理论家反对独裁的和官僚的列宁主义先锋队政党所持的人类自我解放的自
由意志论观点。\textbf{他们反对机械的经济崩溃论的观点,主张对社会制度、意识形态和
  政治之间的相互作用进行更为细致的分析。}正如博特莫尔所指出的,法兰克福学派改变的
正是“政治”这一概念,将它拓展到劳动的分化、科层制、文化、家庭、财产所有权和国家
机器等领域。\textbf{经济的决定作用相应被降低。批判性理论强调“政治”与“经济”不
  断综合的问题,计划的增长和官僚控制是以牺牲市场为代价的;强调社会生活的不断合理
  化;以及劳动分化的增强,它使工人阶级的任务支离破碎,把工人阶级分裂为个体,使工人
  阶级不再理解并且组织起来反对自身的异化。}

在对苏联马克思主义的研究中,赫伯特·马尔库塞引出了马克思主义政治经济学的某些内涵。
希法亭和考茨基论证了一个稳定的、有组织的世界经济如何可能代替资本家之间的相互争
斗(见本书第一卷第14章)。\textbf{马尔库塞认为,1945年以后,面对要与苏联展开竞争的
  迫切需要,这些可能性变成为现实。在东西方冲突中,帝国主义列强之间原来的对抗消失
  了,取而代之的是基于计划而不是市场无政府状态之上的“洲际政治经济学”}:

\begin{quotation}
  为动员整体的物质力量与精神力量的需要,必须取消经济和文化生活中的自由放任,对政
  治过程有条理的控制,以及在真正的经济实力等级下对国家进行重新组合——不惜以牺牲
  宝贵的国家主权为代价。\textbf{西方社会整体的压倒一切的利益改变了国家和阶级的利益。}
\end{quotation}

其结果就是类似于希法亭的“\textbf{总的卡特尔}”,受美国这一超帝国主义盟主的支配。
对于劳动者,\textbf{马尔库塞接受了阿诺德·汤因比对“内部”与“外部”无产阶级的区
  分。}对于后者,即居住在前殖民地地区和西方国家中被隔离的少数民族聚居地的人来说,
没有发生什么变化;而前者则被溶入资本主义体系,并且作为对其忠诚的奖赏,它的生活水平
有所提高,并具有少许政治影响力。马尔库塞认为,如同列宁主义理论认为帝国主义之间不
可抑制的竞争将导致战争的观点一样,马克思关于发达资本主义国家爆发无产阶级革命的期
待,也已经成为乌托邦。

弗里德里克·波洛克对生产自动化影响的研究,进一步发展了上述观点。\textbf{波洛克预期新技术会
拉大业已存在的差距,即以一小部分高素质的“管理者”、工程师、专家为一方和以大量靠
工资生活的工人为另一方之间的差距。其中的原因,不仅在于两个群体之间的个人素质,而
且在于他们的技术和管理培训程度。}“手”的活动,现在一般被限制在完成相当初级的操作,
或者按简单的指令操作,他们没有必要理解这样操作的真正目的。

波洛克隐晦地指出,自动化的长期影响可能包括演化为一个建立在独裁主义和军事原则基础
上的“社会新形式”,在这个社会里,\textbf{高素质专家组成的“经济总参谋部”,将毫无挑战地
统治该社会,资本家将失去其经济作用。}

对马尔库塞和波洛克来说,战后资本主义的变化已经使工人阶级变得破碎和被动。这种基本
上悲观的观点,在1961—1962年遭到一位居住在巴黎的希腊经济学家的反对,这位经济学家与
法兰克福学派没有正式的联系,但深受它的影响。保罗·卡登(在他的《科层资本主义》中)认
为,传统马克思主义所指出的“科层资本主义”的经济矛盾已经被成功地克服。卡登认为,
马克思对利润率下降的分析有着致命的缺陷,同时,随着工人的生活水平不断提高,消费不
足也已经避免了。国家干预把周期性波动限制在很窄的限度内,结果是“实际上已经永久实
现充分就业。不论是体力劳动者还是脑力劳动者,只要他们适应环境,就能够面对永无止境
的就业前景。除了细小的波动外,生产年复一年地以相当高的百分比扩张。”实际工资也以
大致相同的比例上升。由于投资和政府支出的稳定增加,这意味着“市场问题从根本上得到
了解决”。卡登据此得出结论:“1929—1933年程度的危机在今天是不可想象的,现在已经不
可能突然爆发集体性的非理性的行为,并同时影响那么多的资本家及其经济顾问。”

但是,对卡登来说,这既不意味着资本主义已经摆脱了所有矛盾,也不意味着工人阶级注定
要被动地接受现状。这两点是紧密联系的。\textbf{卡登认为,马克思政治经济学的根本错
  误是,它假定资本主义社会中的代理人完全是“客观具体的”(即将它降为由人力无法控制
  的经济规律支配的客观存在)。}不论是马克思的剥削理论还是他对经济危机的分析,都假
定资本家和工人都不可能对经济运行产生影响。但是,这与资本家有能力组织国家管理的、
不存在危机的积累不相符合,也与工人阶级在资本主义经济的所有方面——从工资水平和投资
的节奏,到生活结构和技术变化的性质——所展开的持续斗争不相符合。生产中阶级冲突的存
在,证明具体化是有界限的。卡登所认为的现代资本主义真正的基本矛盾是:它需要借助于
工人的参与(没有工人的参与,就不可能进行能够获取利润的生产),同时又需要限制这种参
与(唯恐他们组织起来废除资本主义本身)。卡登得出结论,资本主义社会的真正动力就是阶
级斗争的动力,它表现在工人为控制生产和生产发展速度而展开非正式罢工的次数不断增
多。

尤根·哈贝马斯是第二代“批判性理论家”中最突出的一位,他(以颇为不同的方式)详细讨论
了类似的问题。尽管他最重要的著作《合法性危机》直到1973年才出版,但是它与本章主题
的联系是很清楚的。\textbf{哈贝马斯否认“有组织的或国家控制的资本主义”已经能够完
  全消灭经济危机。利润率下降趋势的基本规律——在这一点上,哈贝马斯与除曼德尔外的其
  他参与争论的理论家们不同——仍然在起作用。但危机已经改变了表现形式,通货膨胀、国
  家财政经常面临的压力、个人富裕与公共贫穷之间不平等状况的恶化等取代了传统的生产
  萎缩和就业下降。阶级关系已经具有政治性,以致于“经济过程不再可能被看作是经济制
  度的内在运动”。由国家提供公共产品,尽管这对降低固定资本价格和提高剥削率至关重
  要,但是,这也对劳动价值论提出了质疑,它不能被运用于教育、技术或科学等领域。工
  资的决定也与价值规律严重背离,它成为准政治性的,表现为通过集体谈判而达成的阶级
  妥协。}

哈贝马斯的思想与“国家垄断资本主义”概念有许多相似的地方。\textbf{“国家垄断资本
  主义”是东德的官方意识形态,认为国家是一个资本家集团,它运用集中计划取代市场力
  量的自发作用。哈贝马斯认为,这样一种把国家仅仅看作是垄断资本家的代理人的观点过
  于简单,也夸大了国家合理制定计划的能力。}但他的确接受了国家垄断资本主义思想中的
最重要内容,即\textbf{经济活动的政治本质日益公开化。只有把经济危机背后的压力转移
  到社会生活的其他领域,经济危机才可能避免。阶级冲突和剥削仍然具有基础性价值,但
  现在它们被用来表述政治和社会的、而不仅仅是经济的不稳定。}

这一结论把意识形态问题提到政治经济学的核心位置。哈贝马斯坚持认为,\textbf{国家干
  预增长的后果之一是“资产阶级公平交易的基本意识”瓦解了,取而代之的是正式的民主
  政治观念,以及计划者中的技术精英的实际统治。这增加了爆发两种类型的政治危机的可
  能性:一种是“合理性危机”,这源于该制度不能兑现它向民众所承诺的成功的经济管理,
  一种是“合法性危机”,即它不再能保持公众对它的忠诚。在社会文化领域中,这将导
  致“激励危机”,因为该体制不能“产生必要数量的行为激励意图”。}

\textbf{卡登从基层工人的不满意中看到了基层群众革命运动的萌芽}(这在法国1968年5月的
革命中得到部分证明),而哈贝伯马斯的结论就不那么深刻了。哈伯马斯宣称他回答了“人们
还没有作出满意回答的‘资本主义已经发生变化了吗?’这一问题”,但是,他的答案还远
不够清楚。\textbf{资本主义已经克服了传统的经济矛盾,但面临着工人阶级信任度消退的
  前景。哈贝马斯似乎认为,这会引起一场“持久性危机”,威胁资本主义的持续生存。只
  是这种威胁可能如何成为现实却在某种程度上仍然是模糊不清的。}

\section{没有结束的繁荣,无止境的工商业}
我们当然可以同意哈贝马斯的看法,即都留重人提出的问题还没有得到令人满意的回答。资
本主义已经发生了变化,但是,它对马克思主义政治经济学在何种程度上、赋予了什么含义
仍然存在着激烈的争论。\textbf{资本主义发展的阶段性,判断资本主义发展新阶段的标准,
  在方法上与新“时期”的区别,这些问题并没有得到解决,}它们在未来的几十年里还会被
提起。\textbf{关于马克思与凯恩斯之间的关系这一核心的分析性问题,是人们在许多现实
  问题上存在分歧的根源。}持有各种不同观点的经济学家对此展开了非常广泛的讨论,这是
下一章要讨论的主题。

人们在1965年后对许多还没有得到解决的问题展开了热烈讨论,当时,战后经济繁荣的脆弱
性开始显现。巴兰和斯威齐的消费不足理论在美国影响很大,人们围绕这一理论展开了一场
争论;我们将在第6章阐述他们对“垄断资本”的分析。《资本论》第3卷阐述的利润率下降的
趋势,通常被当作是消费不足的替代,当然它偶尔也与经济危机的复合模型联系在一起。这
些是第7章所要讨论的内容。对马克思主义关于发达资本主义的理论来说,一个很重要的问题
是军事支出对经济的影响或者是促进资本积累,或者是(像达托所认为的)成为资本积累的沉
重负担。第8章将讨论“持久性军事经济”的内涵。最后,第三世界中的前殖民地国家显然没
有实现战后经济繁荣,这在20世纪60年代成为人们讨论的突出问题。马克思主义和新马克思
主义的经济发展的文献,在20世纪60年代和70年代纷纷涌现,这些文献远远超过了早期对非
殖民化所作的平淡的分析。整个第三篇(第9—11章)主要阐述新出现的帝国主义理论。

在所有这些讨论中,几乎所有的参与者都理所当然地把资本主义经济矛盾放在首位。这不是
责备他们是庸俗的经济决定主义者;巴兰和斯威齐特别对政治、意识形态和文化问题予以极大
的关注。但是,“批判性理论”的洞察力没有被严格地运用于分析经济问题,直
到20世纪70年代中期以来,解释经济危机的更为传统的方法所存在的缺陷越来越明显。在近
期的许多经济危机理论中,工人阶级的自觉行动、立法和动机的作用占据了首要地位,如同
我们将在第16章看到的那样。



\chapter{马克思与凯恩斯}
\section{凯恩斯与马克思}
\textbf{大萧条最重要的理论成果是1936年出版的凯恩斯的《通论》,这本书并非完全首创,
  它不具有一个完全连贯一致的和独立的理论体系。}书中的许多观点已由更早的著述者(包
括凯恩斯本人)所预言,而且正如我们将会看到的,\textbf{《通论》中还有重要的分析上的
  漏洞。甚至《通论》所开的政策处方,也比人们通常所料想的更缺乏新意和可争辨性。}此
外,凯恩斯从未与正统理论彻底决裂,而是把大量新古典的重要理论观点掺和进自己的论著。
尽管如此,这本书依然是迄今为止出自一位受人尊重的主流经济学家笔下,\textbf{对大量
  失业是无序的资本主义经济的正常结果这一论点所作的第一次系统的和完整的表述,它的
  出现是马克思主义理论家们不能忽视的一个重要事件。而且凯恩斯理论提出了一种经过改
  良的、消除了经济危机的资本主义的前景,在这一点上,马克思主义经济学变得缺乏时代
  性,从而面临着直接的政治上的意义。}本章的大部分内容涉及马克思主义政治经济学如何
回应凯恩斯主义者的挑战。然而,我们在开始时却要问一个相反的问题:凯恩斯是如何看待
马克思的?

《通论》只在\textbf{三处}提到马克思,其中一处\textbf{只是简单地承认马克思是“古典
  经济学”这一术语的首创者。}第二处,\textbf{凯恩斯描写了1820年以后李嘉图经济学怎
  样成功地排除了总需求不足这一观点}:“它只能偷偷摸摸地存在于下层,生活在卡尔·马
克思、西尔维奥·格塞尔和道格拉斯少校这些不入流社会中。”因为格塞尔和道格拉斯是在理
论上没有什么地位的货币异教徒,这并不是在奉承马克思。但接下来的情况更
糟。\textbf{凯恩斯断言,与马克思不同,格塞尔已经明确地否定了“古典假设”(即萨伊定
  律)。格塞尔这样做就使得马克思本人对古典经济学的批判变成多余的了;}因此在他的论著
中包含了“对马克思主义的回答”。凯恩斯以并不十分肯定的口气得出结论:“后世从格塞
尔那里得到的,将比从马克思那里得到的多。”凯恩斯并非如上述最后一段提到的那样,对
马克思一无所知,也不总是对马克思持蔑视的态度。在1920—1921年或1921—1922年,莫里斯·多
布还是一位研究生,他曾在凯恩斯的房间里读到一篇论述马克思与剑桥的政治经济学俱乐部
的论文。多布回忆道,凯恩斯很赞许这篇论文,因为“他年轻时在一定程度上喜欢非正统思
想”。凯恩斯在20世纪20年代以来的文章中,对马克思主义说了一些坏话,包括著名的关
于“一个如此不合逻辑,如此空洞的教条怎么能对人的思想从而对历史事件有如此强烈和持
久的影响”的质问。但是,到1933年,大概在皮罗·斯拉法的影响下(参见以下第13章和
第15章),凯恩斯开始对马克思采取比较赞许的态度,\textbf{在他关于古典货币理论的演讲
  中,婉转地提到马克思对实现问题的阐述,而且发现了马克思和马尔萨斯在有效需求问题
  上的密切相似之处。}

《通论》的第一稿,也写于1933年。在该稿中,凯恩斯从
未如此接近于正确理解马克思。有关段落值得全文援引如下:
\begin{quotation}
  合伙经济和企业家经济之间的区别,同卡尔·马克思所作的大量观察有某种关系,——尽管
  其后他对这一观察的利用是相当不合逻辑的。他指出,在现实世界中,生产的本质并不像
  经济学家们通常所认为的那样,如\textbf{W-G-W'的情形,即把商品(或劳务)换成货币是为
    了获得另外的商品(或劳务)。}这也许是私人消费者的观点,但不是商家的看
  法,\textbf{后者认为是G-W-G'的情形,即抛出货币换取商品(或劳务),是为了获取更多的
    货币。}
\end{quotation}
凯恩斯接着指出,这一观点的意义在于:\textbf{企业家对劳动力的需求,依赖于生产预期
  的可获利性,而不取决于对人类需求的直接满足。}

在一个长脚注中,凯恩斯作了进一步阐述。
\begin{quotation}
  G'超过G的余额,是马克思的剩余价值的源泉。令人不解的是,在经济理论史上,那些数百
  年来以这种或那种形式用G-W-G'反对古典公式W-G-W'公式的异教徒们,或者倾向于相信G'总
  是并且必然超过G,或者倾向于相信G总是并且必然超过G',这要取决于他们生活的时期哪一
  种思想在实践中占支配地位。马克思与那些相信资本主义制度必然具有剥削性的人断
  言,G'余额是不可避免的;然而,相信资本主义内在地具有通货紧缩和就业不足发展趋势
  的霍布森、福斯特、卡钦斯或道格拉斯少校则断言,G余额是不可避免的。但当马克思补充
  说G'持续增加的余额,将不可避免地被一系列日益猛烈的危机或者企业倒闭和未充分就业所
  打断时,马克思正在逐渐接近不偏不倚的真理,可以推测,在这种情况下,G一定会有余额。
  如果能够得到承认,我自己的观点至少可以有助于使马克思的追随者们和道格拉斯少校的
  追随者这两派达成和解,而不去理会那些不切实际空洞地相信G与G'总是相等的古典经济
  学家们(按照凯恩斯对这一术语的独特用法,古典经济学家就是指萨伊定律的支持者)。
\end{quotation}

撇开把马克思与道格拉斯相提并论有些不恰当这一点不论,凯恩斯这时已离真理不远了,我
们在下一节中将会看到。但是,凯恩斯对马克思的调情卖俏没能持续多久。第二年,凯恩斯
又重操旧论,嘲笑马克思对资本主义世界毫无希望的描述不适合当代现实,并嘲笑他正是那
个已被推翻的19世纪正统学派的栋梁。凯恩斯1934年11月对一位广播电台听众说:“如果李
嘉图经济学说破产了,马克思主义理论根基的一个主要后盾也将随之瘫塌”。这一时期,在
与乔治·肖伯纳的通信中,凯恩斯仍坚持自己的观点,\textbf{重申马克思主义理论是建立在
  李嘉图学说(即萨伊定律起作用)基础之上的。}他不屑一顾地把《资本论》比作《古兰经》,
认为它们都是无用的教条,并抱怨说人们对于《资本论》的争议是“乏味的、过时的和学究
气的”。凯恩斯的结论是:“《资本论》在当代的经济价值(排除一些偶然的但却非建设性的
和不连贯的思想火花以后)是零。”这表明《通论》对马克思的蔑视是充分的,但却丝毫没有
减弱马克思主义者对“庸俗”经济学家推理的固有的敌意。

\section{马克思主义者论凯恩斯:最初的回应}
甚至在《通论》问世之前,刘易斯·科里就对凯恩斯的《货币论》作了激烈的抨击。在本书
第1章我们已经看到,\textbf{科里对大萧条的解释,综合了马克思危机理论中缠在一起的利
  润率下降和消费不足这两个因素。}毫不奇怪,科里而后对凯恩斯的攻击也集中在这两个论
题上。科里坚持认为,\textbf{投资取决于利润率的变动,而不是(像凯恩斯所讲的那样)取
  决于利息率;因此,凯恩斯对萧条期货币的解释混淆了原因和结果。}“利润率下降并非像
凯恩斯想象的那样是平稳过渡到一个既‘是’资本主义而又‘不是’资本主义的‘新社会秩
序’的手段。\textbf{利润率下降只表明经济衰退,而且是激烈的阶级斗争、社会暴乱和战
  争的前兆。}”至于第二个因素,\textbf{科里认为凯恩斯“轻视了消费”而夸大了投资的
  重要性。}“\textbf{过度储蓄是循环过程中的一个因素。因为它并不造成资本投资(和生
  产)的不足,而是把本该进入消费的货币转化成了投资,从而造成消费不足。}”最后,科
里作了方法论上的批判:凯恩斯的理论“\textbf{强调了交换这一次要因素,而没有强调生
  产这一首要因素。}”这成为马克思主义者对凯恩斯的回应中的一个反复不断的话题。

马克思主义经济学家对凯恩斯的《通论》作过广泛的评论。现在流亡在纽约的前德国社会民
主主义者\textbf{埃米尔·莱德勒},指出了凯恩斯与马克思的三点相似之处。\textbf{第一
  点是凯恩斯采纳了劳动价值论,包括在“工资单位”概念中用熟练劳动力对非熟练劳动力
  的折算来说明。第二点是他关于利润率下降的观点,这一点常以资本边际效率下降的形式
  作伪装。第三点是凯恩斯承认消费和投资之间有必要保持比例,从而承认了马克思关于
  第\Rnum{1} 部类和第\Rnum{2} 部类之间的比例关系。莱德勒对《通论》的反对是方法论
  上的和政治上的,而不是狭义的经济上的。}他认为,凯恩斯赞成个体的心理因
素,\textbf{忽视了阶级的存在},并对资本家政治行为的合理性表示了盲目的乐观主义的表
述:

凯恩斯作了一个\textbf{不言而喻的假设},即\textbf{“(资本家的”)决策受理论洞察力的
  指导}。一种理论见解即使意味着社会地位的毁灭,也能被资本家所接受,他认为这是理所
当然的。这个从18世纪继承下来的观念,使凯恩斯的观点变成了一种乌托邦——在那些还没有
学会和解艺术的国家更是如此。莱德勒得出结论,\textbf{除非受利益和感情驱动,否则思
  想不能改变任何事物}:“这样一个合理的分析并不是一种原动力……凯恩斯提出了要求进
行一场权力和财产革命的思想,但是我们难以相信,通过劝说的方式能够使人类接受一种新
的经济社会制度。”

对于法兰克福学派(其本身也被驱逐)而言,\textbf{库尔特·曼德尔鲍姆和弗雷德里克·波洛
  克}甚至更挑剔。他们认为,\textbf{既然货币的错位是危机的症状而不是原因,那么凯恩
  斯对于流动偏好的强调就用错地方了。凯恩斯夸大了利息率变动对投资的影响,同时忽视
  了技术进步对利润率进而对投资的影响。}《通论》没有谈及\textbf{不同经济部门投资比
  例失调}所造成的影响。事实上,凯恩斯的这本书少有新意。书中对萨伊定律的驳斥,马克
思早已讨论过了,而凯恩斯那“天真的、长期被人反驳的消费不足主义”,揭露了他在与资
产阶级思想决裂方面的无能,而他对主观倾向的依赖则表明其分析上的肤浅。甚至他的自由
主义也只是表面的:\textbf{他对重商主义的赞成和对通货膨胀导致实际工资下降的肯定,
  都暴露了他的集权主义倾向。}曼德尔鲍姆和波洛克断言,“凯恩斯的修正程度超过了古典
教条,但他没有指出更光明的前途,而是指向更暗淡的未来。”

凯恩斯思想中的保守成分也被\textbf{约翰·达雷尔}所强调,其观点发表在新创刊的美国马
克思主义季刊《科学与社会》上。达雷尔承认,凯恩斯与马克思在某些问题上,特别是
在\textbf{利润率下降}方面,有着密切的相似之处。但从总体上看,凯恩斯对危机和资本主
义长期发展趋势的说明,同马克思主义的观点是不可调和的。尤其是凯恩斯的\textbf{个人
  主义心理}是错误的。追随杰文斯、瓦尔拉斯和门格尔,凯恩斯研究的是自私的个人,并认
为他们的偏好、他们的消费倾向、他们对利润的预期是既定的。因此,达雷尔得出的结论是,
《通论》是与主观价值论相符的。它是一本经济学著作,而不是政治经济学著
作,\textbf{它未能揭示资本主义生产的内在规律}。创建于1938年的《现代季刊》是《科学
与社会》杂志在英国的副刊。在该刊物的第一期,\textbf{埃里克·罗尔}批评凯恩斯把注意
力放在了\textbf{交换}上而不是放在\textbf{生产}上,\textbf{并忽视了那些与试图改变
  交换机制相联系的政治问题}。罗尔断言,\textbf{《通论》的言外之意是,自由主义在政
  治上和在经济上都破产了。}凯恩斯的理论已经在法西斯意大利和纳粹德国得到实施。尽管
凯恩斯本人真诚地信奉进步的观点,但他的著作的确有被反动政客所借用的危险。英国的马
克思主义者们对凯恩斯的反应也不一致,约翰·斯特雷奇在《现代季刊》上发表的三篇对《通
论》的赞赏有加的论文就证明了这一点。\textbf{斯特雷奇}指出,在关于利润率这一中心问
题上,凯恩斯的结论与《资本论》第3卷的结论有明显的相似之处。斯特雷奇承认,凯恩斯的
确忽视了技术进步和利润率之间的关系。然而凯恩斯和马克思两人都认为,\textbf{要增加
  就业就需要扩大投资,而且都认为资本积累导致了利润率的下降}(按照斯特雷奇的说法,
利润率这一概念与凯恩斯的资本边际效率最接近)。因此,\textbf{利润率下降趋势是“凯恩
  斯先生背后的战车。非常有趣的是,一位一流的资产阶级经济学家竟把利润率下降趋势重
  新确立为他所描绘的资本主义社会的核心。”}像科里和达雷尔一样,斯特雷奇对这两种理
论的鉴别已走得太远了,但是20世纪30年代末期,另外一些同情马克思的著述者在学术期刊
上发表的文章,强调马克思与凯恩斯的这种或那种相似之处。在《经济研究评论》中,J.
D.威尔逊像斯特雷奇一样,认为凯恩斯的资本边际效率同马克思的利润率密切相关,但是凯
恩斯以长期因素作为比马克思更为成功的分析的中心论据。这份杂志还发表了署名樊弘的一
篇重要论文,该论文用代数公式比较了《通论》和《资本论》中关于有效需求、利润率、货
币和利息等理论。\textbf{樊弘认为,凯恩斯错误解释了马克思的再生产图式。这些解释没
  有接受萨伊定律的合理性,但相当详细地论述了与凯恩斯的有效需求分析相容的宏观经济
  均衡条件。}在《经济实录》中,澳大利亚人E.E.沃德强调了马克思与凯恩斯之间在消费不
足观点上的相似,但也指出了《通论》(在方法论意义上)的相对肤浅和主观主
义。\textbf{沃德把马克思对资本主义发展过程(日益增加的失业、垄断增强、危机、国际扩
  张)的成功预言,同凯恩斯缺乏对这些内容预言进行了对照,凯恩斯理论缺乏预言性内容是
  他未能将制度变化融入其理论中的不可避免的结果。}波兰马克思主义者奥斯卡·兰格早就
说过,这是所有资产阶级经济学的主要局限。

有一种学术观点出人意料地保持沉默。\textbf{1936年,莫里斯·多布是英国共产党的主要经
  济学家,也是以英语为母语的国家中最卓越的马克思主义经济学家};他也以剑桥大学为自
己的根据地。然而他似乎从未参加过任何有关《通论》这本书的讨论。多布本人把这
种\textbf{令人吃惊的缺憾},归因于埋头从事反法西斯主义和其他政治活动。1977年,他在
剑桥的同龄人奥斯汀·罗宾逊进一步证实,多布是“性格相当孤僻的人……当时他没有像现在
这样处于剑桥争论的中心,在围绕凯恩斯和凯恩斯理论争论最起劲儿的时候,他在剑桥正是
处于这种情况。”

但是,这不能说明全部情况。多布曾发表过对凯恩斯的《货币论》表示敬意的书评,并寻机
发表了一个重要的理论读本(1937年),并在三年后,在与米哈尔·卡莱茨基进行讨论后,修改
了关于危机的一章。1940年版的《政治经济学和资本主义》有七处提到凯恩斯,其中仅三处
有一些实质内容,而且没有一处提出对于《通论》的主要观点的批评意见。多布似乎一开始
就被这本书弄糊涂了,同时他也不便对曾经是他彭布罗克大学时代有恩于自己的这个人进行
正面攻击。多布早期对凯恩斯的尊重多是由于《和平的经济后果》一书中有原则性的国际主
义,同时凯恩斯也安排出版了多布的《论工资》一书,并对多布的《俄国经济发展》一书的
手稿进行了有益的评论。对多布来说,应该不会有来自党内同志的要求他谴责《通论》的巨
大压力。因为20世纪30年代末反法西斯主义的共产主义人民阵线运动,本身就包括有效地放
弃任何反对改良主义经济学的思想运动。这样,多布就能够通过罗宾逊和卡莱茨基的文章慢
慢地和凯恩斯的思想达成妥协,而不必被迫对他们的理论成就表明自己的坚定立场。

概括地说,马克思主义者对凯恩斯的最初反应是复杂的。在重要的相似的思想,特别是在资
本主义危机趋势的本质、利润率下降和货币与利息理论方面,引起了关注。尽管如此,凯恩
斯的主观主义,他对交换的表面现象的关注,\textbf{他不能与自由主义微观经济学决裂,
  以及他对于政治前景的天真(或者更糟),还是遭到了激烈的批评。}某些遭到反对的观点,
甚至比其他观点更牢固地成为凯恩斯经济学的基础。特别是莱德勒和罗尔对凯恩斯《通论》
的非自由主义政治见解提出的疑惑,的确揭示了凯恩斯理论的重要缺陷。\textbf{但是,科
  里的批评却没有触及实质,不论是科里还是曼德尔鲍姆和波洛克,他们似乎都认为凯恩斯
  更为强调的是资本边际效率表本身的变动,而不是沿着资本边际效率表的运动;达雷尔也明
  显地误述了凯恩斯关于利润预期的观点。凯恩斯的工资单位与马克思的劳动价值之间的任
  何相似之处都纯粹是形式上的,而资本边际效率表是一个主观概念,它比马克思的利润率
  下降理论所涉及的时间维度要短得多。}马克思主义者对凯恩斯的批评\textbf{缺乏明确性,
  在很大方面是因为他们未能看到凯恩斯主义经济学属于短期分析,在短期条件下股本和技
  术被认为是固定不变的。}1945年以后,马克思主义者对凯恩斯的反对得到了加强,而且更
系统化了,凯恩斯的短期分析特点得到了明确的承认。与此同时,早期凯恩斯主义者正在重
新考虑他们自己的处境。

\section{凯恩斯主义者与马克思}
《通论》出版以后,凯恩斯没有对马克思表示更多的兴趣。然而,\textbf{他的确鼓励他的
  一个最积极的追随者琼·罗宾逊对凯恩斯主义和马克思主义之间的思想关系进行不懈研
  究。}凯恩斯作为《经济学杂志》的编辑,发表了罗宾逊有关马克思的最初两篇文章(一篇
是评论斯特雷奇的《资本主义危机的本质》的文章),并“宽厚地接受”了她的《论马克思主
义经济学》。在写于1935年的第一篇文章中,她得出了“\textbf{马克思事实上是一位古典
  经济学家这一明显地自相矛盾的观点}”,这是按照凯恩斯对这个词带有贬义意义上(即最
为萨伊定律的支持者)使用的。罗宾逊之所以得出这个结论,\textbf{是因为“(马克思的)理
  论的核心是:除非消费下降否则投资不能增加”。对萨伊定律的否定将——这里她使用了凯
  恩斯的说法——摧毁“马克思经济分析的绝大部分基础”。}罗宾逊进而\textbf{舍弃了作
  为“一种形式主义”的劳动价值论,并舍弃了通过“完全的循环论证”推导出的马克思对
  利润率下降的说明(即使利润率作为日益增长的资本—劳动比率的结果,的确是下降的。)}

\textbf{所有这些都得自于斯特雷奇的而不是马克思的著述。}罗宾逊在《经济学杂志》的一
篇文章中,得出同《资本论》相似的一种新的见解,尽管她对《剩余价值理论》并不熟悉(这
是重要的),因为\textbf{那时《剩余价值理论》还未被译出},并且可能——由于纳粹焚
书——在德文原版中缺少这一内容。凯恩斯似乎也没有读过《剩余价值理论》,马克思在《剩
余价值理论》中对萨伊定律作了比《资本论》中严厉得多的批判。然而,罗宾逊对《资本论》
第3卷有关利润率下降的分析也没有表示赞同,她在对纳塔莉·莫斯科斯卡的回忆文章中(参见
以下第7章),把利润率下降规律歪曲成可以用剥削率上升规律来重新表达,认为“换个说法
也是一样的”。关于萨伊定律,\textbf{她首先指责马克思假定资本家用他们积蓄的所有东
  西进行投资,从而忽视了有效需求问题。}她认为,\textbf{马克思所信奉的在萧条期间工
  资减少会提高剥削率,从而会提高利润率的观点是错误的,因为它忽视了实际工资减少对
  工人消费,从而对总需求的消极影响。}但是,罗宾逊接着援引\textbf{米哈尔·卡莱茨基}的
话,以表明“\textbf{马克思的方法为有效需求分析提供了基础,而学院派经济学家,由于
  他们对马克思的轻视,已经浪费了大量时间为自己重新找到这一方法。}”这必然包含着对
凯恩斯的含蓄的批评,也包含着对她自己早期著述的含蓄的批评。

\textbf{在卡莱茨基的帮助下},罗宾逊已在马克思那里找到了消费不足主义者的因素,并把
马克思的危机理论解释成第\Rnum{1} 部类和第\Rnum{2} 部类之间的一种比例失调,也是\textbf{消费和投
  资之间的比例失调}。事实上,“工人不能消费,而资本家不愿消费。\textbf{消费品产业
  因此为投资提供了一个狭窄的领域,而资本品产业则依次遭受了需求的限制。在这里萨伊
  定律终于被推翻了,而马克思似乎预见到了现代有效需求理论。}”因此——她用马克思的
一个关键的观点结束了她的这篇文章:“\textbf{资本主义生产的障碍是资本本身}”。

对这些问题,罗宾逊在《论马克思主义经济学》中作了更大篇幅的阐述。\textbf{她再次舍
  弃了利润率下降理论,认为它代表了“一个虚假的线索”,而且对劳动价值论的敌意更明
  显了}(罗宾逊从未给人留下劳动价值论与凯恩斯的“工资单位”有明显相似之处的印象)。
现在她把马克思说成是一个老牌的消费不足论者,在马克思看来,工人的贫困和资本家对积
累的贪婪制约了对消费品的需求。结果就限制了对于投资品需求的增长率,并造成有效需求
不足。但是,罗宾逊认为,\textbf{马克思还没有认识到萨伊定律的基础作用,并试图建立
  一种即使萨伊定律成立也能适用的危机理论。}这就把马克思本人和后来的理论家混淆起
来(可以推测,也和凯恩斯、罗宾逊两人混淆起来)。然而,罗宾逊坚持认为,\textbf{经过
  适当修订后的马克思主义政治经济学,同凯恩斯《通论》的基本观点是完全一致的。马克
  思和凯恩斯甚至拥有某些共同的缺陷,其中包括他们都缺乏完整的收入分配理论和投资引
  诱理论。}

罗宾逊历时近40年,不断地写有关这些方面问题的文章,尽管她对资产阶级经济学不断地进
行批判,并更多地接受马克思的一些观点,但她仍相信自己在1942年得出的总的结论。其他
那些激进的凯恩斯主义者们几乎无一例外都在她的引导之下,并常常被认为是形成了一
种\textbf{凯恩斯主义的马克思主义形式}。\textbf{亨利·史密斯}确实可能是这样做的第一
人。早在1937年,他就指出,\textbf{马克思的商业周期理论与投资支出的波动是相联系的,
  而这恰恰是凯恩斯在《通论》中得出的结果。S.S.亚历山大}发现凯恩斯与马克思的相似之
处在于,\textbf{他们都把收入分配作为消费倾向的决定因素,他们对货币储蓄和流动偏好
  的分析,以及(再一次)对利润率下降的长期趋势的研究。}通常自称为凯恩斯主义者的亚历
山大,在这里重复了马克思主义者对凯恩斯的某些批评。战后L.R.克莱因用参数值不同的经
济计量模型来表述马克思主义和凯恩斯主义体系,并对它们进行经验检验(由此马克思的理论
显现出相当的优势),从而扩充了樊弘的著述。一旦有效需求的长期分析由经济增长理论来解
释,\textbf{许多凯恩斯主义著述者又论证了哈罗德—多马模型和马克思增长模型的紧密相似
  之处。}多马本人直率地承认,他受惠于20世纪20年代俄国经济学家,并通过他们而受惠于
马克思。最后,在20世纪80年代,老牌的凯恩斯主义者\textbf{达德利·迪拉德把马克思和凯
  恩斯的主要成就重新表述为将货币理论整合进一般经济分析。}在以下第15章,我们将再次
论及马克思主义和凯恩斯主义的理论关系。

\section{马克思主义者论凯恩斯:第二次和第三次思考}
可以预料到,迪拉德的评价受到了来自马克思主义阵营而非其凯恩斯主义同伙的挑战,因为
凯恩斯的新古典价值论、技术统治论和改良资本主义的国家主义形式,与马克思的政治经济
学是根本不相容的。在最初的几十年里,这样的异议不断地出现。冷战初期,意识形态的对
立处在最紧张状态,很多马克思主义者都认为,苏联明显地面临着日益迫近的来自资本主义
世界的军事进攻的危险,这时反对意见常常相当富有进攻性。因此,尽管保罗·斯威齐
在1946年的评价中承认,《通论》对有效需求分析的许多观点,马克思主义者也能得出,但
他把凯恩斯(并非不公正地)说成\textbf{终究是新古典理论培育的一个“囚犯”。}斯威齐认
为,\textbf{凯恩斯从来没有把资本主义制度看作是一个整合了资本主义经济、政治、技术
  和文化的总体。因此,凯恩斯认为阶级斗争不过是“令人讨厌的混乱”,而忽略了资本主
  义国家的阶级的作用,把国家只看作是一个在紧要关头突然出现以扭转局面的角色。他忽
  视了技术进步对增加失业的影响,他把这看成是经济机制中可以矫正的缺陷,而不是保证
  资本家控制劳动力市场的手段。最后,凯恩斯甚至比他的一些新古典理论派的同行们更少
  关注垄断的力量,丝毫没有论及垄断在宏观经济中的意义}(这一缺陷后来由“后凯恩斯主
义”理论家如卡莱斯基,以及巴兰与斯威齐这样的马克思主义者们努力加以校正:参见以下
第6章)。斯威齐断定,在这方面,希法亭的《金融资本》还没有资产阶级的对手。

1950年,莫里斯·多布终于和凯恩斯清算了理论总账。多布强调了《通论》与新古典正统理论
的决裂及其对传统的经济均衡假设的否定。但多布认为,凯恩斯一直想改革资本主义,因此
不能把他看做是“民主社会主义”理论家;他提出的“投资社会化”(从未作过明确的定义,
但可能包含金融部门的国有化),可以理解为对生产社会化的替代。多布坚持认为,凯恩斯的
经济方法过于宏观,因而导致他\textbf{忽视了不同部门之间的必要的均衡,并进而忽视了
  计划的必要性。凯恩斯同情生产资本家,反对借贷资本家,他也不同情工人阶级,他把资
  本主义国家的特点描述成一个中立的仲裁者,从而使他看不到政府经济政策的政治局限。}多
布断言,\textbf{军费开支大概是资本家乐意接受的政府介入反萧条的唯一形式,}而且在资
本主义制度下充分就业是个乌托邦式的理想。(我们对“持久的军事经济”将在以下第8章讨
论)。多布在其晚期著作中,对《通论》的敌意减少了,但他对凯恩斯的本质上还属于新古典
主义的价值理论和分配理论仍进行严厉的批判。他认为,\textbf{凯恩斯建立在投资边际效
  率思想上的利润分析,是整个《通论》的“最庸俗的观点”。}

正如斯威齐指责凯恩斯忽视了垄断的宏观经济含义一样,多布在这里也说出了很多真理。但
是,尽管他们作了这些批判,\textbf{斯威齐和多布还是没有否定《通论》在关键的分析上
  的信条,即有效需求不足是经济危机的原因,}而且财政政策能够(至少在原则上)使情况恢
复正常。他们对凯恩斯的\textbf{反对主要集中在政治上的而不是狭义的经济上的局限},并
似乎大多归因于米哈尔·卡莱斯基的一篇有影响的文章《政治商业周期》。斯威齐公开承认自
己是消费不足论者(参见以下第6章),多布很少承认这一点,但他极为同情消费不足论者的观
点。他们都没有太多的时间考虑,作为《通论》的余波,大多数马克思主义者对危机的解释,
即《资本论》第3卷对利润率下降的分析,不久将发生什么变化。\textbf{不管利润率下降理
  论的优点和缺点是什么(对此将在以下第7章分析),但它的确从萧条的原因、从为防止与缓
  和危机进行政府干预两个方面,为主流凯恩斯主义提供了一种可供选择的明确的分析方法。
  按照这一理论,技术进步将比提高剥削率更快地提高资本有机构成,从而强制降低利润率
  并抑制投资,}私人资本家——甚至整个资本主义国家——经历一次需求下降的衰退。但
是,\textbf{把需求不足看作根本因素,就误把症状当成了起因。资本主义经济的核心是生
  产,而不是交换。既然政府支出只是构成利润源泉的剩余价值的“排水渠”,因此,它不
  能提供长期的解决危机的办法。资本主义具有内在的和不可避免的危机趋势。}

对凯恩斯主义的这种直截了当的反对,根源于\textbf{亨利克·格罗斯曼}的(凯恩斯主义之前
的)著作,并由他的弟子\textbf{保罗·马蒂克}孜孜不倦地宣传了近半个世纪。\textbf{今天,
  它已被马克思主义经济学家们普遍地接受。}这能用马克思的资本循环公式得到最好的解释,
这一公式在凯恩斯1933年的手稿中曾被不确切地引用过(参见以上第1节),\textbf{它忽视了
  劳动价值转化为生产价格(参见以下第12—14章)以及固定资本(参见以下第13章)问题。}在
简单商品生产或前资本主义商品生产中,循环采取W-G-W的形式:私人生产者把一种商
品(W)交换成货币(G)是为了购买另一种与之等价的不同的商品(是W,而不是凯恩斯所说的W')。
在资本主义生产条件下,循环公式是W-G-G'-W' (不是凯恩斯所说的W-G-W')。这样,资本家首先
用一笔货币(G)去交换与之等价的生产手段和劳动力(W)。这些商品接着被投入生产中使用,
在使用的过程中剩余劳动完成,剩余价值创造出来。在生产过程结束时,资本家是新商品的
所有者,新商品的价值(W')大于他最初投入的价值,W'与W的差额表现为剩余价值。一切都很
顺利,资本家以劳动价值卖出新商品,并得到等价的货币(G'),这里G'超过G的余额既代表生
产出来的剩余价值,也代表资本家得到的利润。利润率被表示为(G'-G)/G,相当于(W'-W)W。

那时,反对凯恩斯主义的马克思主义者认为,\textbf{危机产生的原因是未能生产出足够的
  剩余价值,以至于(W'-W)的增长慢于W的增长,从而引起利润率下降。危机并不是实现阶
  段(W'-G')的困难所致。}既然政府本身不是资本家,它不生产任何剩余价值,\textbf{而
  且政府活动又构成对私人盈利部门生产劳动所创造剩余价值的挥霍。政府干预减少了私人
  资本得到的剩余价值的数量,而且只会使事情变得更糟。}这同凯恩斯经济学含义正相反。
乍一看,这是李嘉图谬误的一个突出例子,\textbf{凯恩斯曾用马克思本人的观点对此加以
  指责:如果假设W'=G'(而且W=G),表面看来是坚定地以萨伊定律为基础,并且否定了有效
  需求不足的可能性。}事实并非如此。凯恩斯主义反对者的主张\textbf{并没有排
  除G'<W'的可能性。它确实很可能是利润率下降造成投资缩减的结果,而这又会减少有效
需求,并导致产品价格下降;紧接着它会成为使危机普遍化和深化的一个重要因素。}但
是,\textbf{需要反复强调的是,}对于反凯恩斯主义者而言,它仅仅是一个结
果。\textbf{危机的更深层原因在其它方面,在于剩余价值的生产,而不在于实现剩余价值
  的困难。}

重要的是,要搞清楚\textbf{凯恩斯主义的反对者}已经得出了什么结论和没有得出什么结论。
完全撇开马克思对利润率下降分析(参见以下第7章)的正确性不论,\textbf{他们已经证明,
  即使不存在剩余价值实现的困难,危机也可能根源于剩余价值生产。他们没有得出的结论
  是,所有的危机都必定以这种方式产生。}在(W'-W)/W比率固定不变甚至提高的条件
下,\textbf{实现了的利润率(G'-G)/G也会下降,并且如果W'<G'是由于有效需求不足造成的,
  那么经济危机就可能接踵而来。反之,危机也可能产生于过度的有效需求。}在这类情况下,
生产资料和劳动力的短期市场价格就上升到它们长期生产价格(我们已假设它们与劳动价值相
等)之上。因此,资本家无法买到与W等价的投入,结果是G<W。

\textbf{受卡莱茨基影响的马克思主义者},也探讨了有效需求不足的问题,虽然在方法上略
有不同,但并非完全不一致。他们认为剩余价值的生产是不存在问题的,因此一开始就提出
在均衡条件下总收入必须等于总计划支出的命题。排除政府和对外贸易因素,我们可以将其
表述如下:
\begin{gather}
工资+利润=工人的消费+资本家的消费+投资 \label{e:fifth1}
\end{gather}
这个公式十分接近一个真理,即认为\textbf{工人消费掉他们的全部工资收入,因为他们在
  工作期间的所有储蓄(比如通过养老基金形式),会被他们退休后的不储蓄所抵消。}因此,
公式(5.1)可以简化为:
\begin{gather}
  利润=资本家的消费+投资 \label{e:fifth2}
\end{gather}

这与凯恩斯大名鼎鼎的“寡妇”取之不竭的坛子的总利润模型是相同的,\textbf{在这里,
  资本家的收入决定于他们的支出,特别是(因为资本家的消费压力很小)决定于他们的投资
  支出。}按照马克思的循环公式,\textbf{这意味着(G'-G)的大小由资本家的投资决定;在
  不考虑剩余价值生产(由W'-W决定)时,它将对利润总量构成一个最大的限制值。}然而,需
要注意的是,\textbf{卡莱茨基主义者通常不能做到这一点,}因为他们只是从\textbf{生产
  过程}中得出总结。\textbf{相反的情况也是对的:即如果不能生产出足够多的剩余价值,
  资本家的投资计划也将受挫。}在\textbf{卡莱茨基}看来,只要制定一些有关\textbf{税
  赋}的假设,\textbf{政府支出也能被加以考虑。假设工人不承受任何负担},公式(5.2)可
被重写为:
\begin{gather}
  净利润 + 税收 = 资本家的消费 + 投资 + 政府支出 
  \intertext{或者\ }
  净利润=资本家的消费+投资+(政府支出-税收) \tag*{(5.3a)}
\end{gather}

这样,在\textbf{有效需求不足}的情况下,\textbf{预算赤字将提高净利润,并刺激剩余价
  值生产。}然而,这是\textbf{以私人支出没有任何“挤出”为先决条件的;也就是说,它
  认为资本家对政府日益扩大的作用不通过减少投资来作出反应,}就像反凯恩斯主义的马克
思主义者所认为的那样。\textbf{所以一切都取决于投资这一决定因素},对此(像我们已经
看到的)马克思和凯恩斯都没有提供令人满意的解释。以这个很重要的条件为前提,卡莱茨基
主义的重新表述,的确为马克思主义和凯恩斯主义理论的整合提供了基础。\textbf{在凯恩
  斯主义理论中},剩余价值的实现取决于有效需求的水平,而\textbf{抛开了剩余价值生产
  中存在的问题}。如果马克思主义经济学家在当时得知这样的重新表述,他们对大萧条的反
应就将可信得多(参见以上第1章)。

\section{结论}
事后看来,《通论》是资产阶级经济学的分水岭,但更是马克思主义政治经济学的分水
岭。\textbf{对传统经济学的“凯恩斯革命”,在很大程度上是流产了,凯恩斯结论中的任
  何激进主义,都首先被同新古典理论的综合所窒息,并接着被新古典主义的反革命所击
  溃。}凯恩斯本人对传统微观经济学的不懈坚持和其短期分析的局限,为上述两个过程助了
一臂之力。

然而,\textbf{只有当凯恩斯的理论被修订为长期分析,并在斯拉法的古典政治经济学复兴
  的情况下,凯恩斯的观点对马克思主义所具有的真正的重要性},才变得明显起
来。20世纪50年代和60年代的这些理论发展,将在以下第13章和15章讨论。然而\textbf{在
  马克思主义者中,不久就发生了很大的分歧,很多人认为卡莱茨基-斯拉法版本的马克思主
  义,不过是“凯恩斯主义左派”。与凯恩斯观点不同,他们强调马克思经济思想方法论的
  独特性,与交换明显不同的生产的中心作用,利润率下降所具有的压倒一切的重要性,并
  且明确地站在反凯恩斯的立场上。}这两派的鸿沟在加宽,包括在价值理论上,“凯恩斯左
派”和卡莱茨基主义者在面对由\textbf{劳动价值论}产生的许多问题时,比他们的对
手\textbf{更乐于放弃这一理论}。这些理论斗争在以下的一些章节中还会碰到,特别是在
第14章和第15章中。


\chapter{垄断资本}
\section{引言}
正如我们在第1章所看到的,马克思主义经济学家主要使用四种类型的危机理论来解释大萧条:
比例失调、过度积累、利润率下降和消费不足。其中,\textbf{比例失调论很快就失去了它
  的所有影响,而过度积累论几十年来不再得到支持。1945年以后,人们乞灵于其他两种理
  论,用以解释“长期繁荣”的原因,}并用它证明“长期繁荣”不可能无限地持续下
去。\textbf{战后,一派理论家认为,因为各种反作用趋势的力量,利润率下降的趋势已经
  暂停了,但是它将逐渐恢复,并结束持续的繁荣。}这在第二次世界大战以前是极少数人的
观点,但到了20世纪60年代可能已被大多数马克思主义经济学家所接受,至少在欧洲是如此。
参与讨论的\textbf{另一派观点以消费不足理论为基础},在北美影响较大,他们通过
对“\textbf{垄断资本}”的分析,认为战后浪费性(如军事的)开支已经刺激了总需求。对于
保罗·巴兰、保罗·斯威齐和他们的追随者们来说,资本主义不仅趋于停滞和危机,而且对它
的生产力作出进一步的不合理的使用。

本书的这一篇结构如下。本章考察垄断资本的理论,评价它在形式上的合理性及其说服力。
接着的第7章,要考察自1883年以来对利润率下降理论的批判史。这两章,我们重点研究
到“长期繁荣”结束时期——严格说是到1973年——的理论发展过程,把其后争论的细节问题
放到本书最后一篇的第16章。在第8章,我们将通过对有关军费开支和战后西方资本主义经济
成就之间关系的讨论,对本篇作出结论。

\section{《垄断资本》的起源}
大萧条之前,马克思主义经济学在美国几乎没有什么影响。阶级意识在美国比在欧洲要淡得
多,工人运动表现为更多的机会主义成分,而大学对于激进思想可能更为保守和更不能容忍。
对布尔什维克革命的反应在北美也更严酷,增强了本来就很有力的文化与政治的一致。制度
主义者索尔斯坦·凡勃伦是资本主义的犀利批评者,他的所作所为在某种意义上确实像传播社
会主义观念的渠道。但是,除了路易斯·布丁的《卡尔·马克思的理论体系》(出版于1907年的
一本著名的但主要是注释性的书),20世纪30年代中期刘易斯·科里关于危机理论的论著问世
之前,美国没有出现任何有特别重要意义的马克思主义的教科书。

\textbf{科里主要依靠自学,他是行动主义者而非学者。他的论著被他的折衷主义所毁损,
  而且不久就完全抛弃了马克思主义。}事实证明,保罗·斯威齐倒是位重要得多的人物。在
哈佛大学和伦敦经济学院学习时,\textbf{年轻的斯威齐}就发表了关于英国煤炭工业经济史
的论文,并在主流学术杂志上撰写大量文章。\textbf{最初他是作为正统的“古典”经济学
  家(按照凯恩斯的理解),接受萨伊定律的合理性,支持通过降低工资来减少失业,}并在论
证这一观点时表现出了不可忽视的数学才能。\textbf{然而到1938—1939年,}斯威齐已经改
变了自己的立场,显然变得更加激进了,\textbf{认为就业决定于产品需求,商品供应垄断
意味着在一定变化的范围内具有零工资弹性的一条连续的劳动需求曲线;因此降低工资不能刺
激就业的回升。这个结论是他的著名的折弯需求曲线分析的必然结论。}

严格说来,斯威齐什么时候成为一位马克思主义者是不太清楚的。洛里·塔希斯是斯威齐的同
时代人,他在回忆中首先把斯威齐看作是“\textbf{一位热心而好斗的哈耶克的辩护者}”,
而后在1937年,又把斯威齐看作是\textbf{一位坚定的凯恩斯主义者}。第二年,斯威齐作为
一本相当有影响的小册子《美国民主制度的经济纲领》的合作者出现,这本小册子“以新政
的美国人的服饰表现了凯恩斯的《通论》的样式”。\textbf{斯威齐本人的回忆与此不同},
他在1981年写道:“当我完成伦敦学院的学业(于1933年)重返美国时,是一个坚信但却非常
无知的马克思主义者”。可是,斯威齐在20世纪30年代发表的作品,并没有使他与非常胜任
的、有独到见解的凯恩斯主义者区分开来。\textbf{他的消费不足论部分地也要归功于英国
  激进主义著述者J·A·霍布森},斯威齐在1938年发表的一篇书评中赞扬了霍布森的著作。另
外,斯威齐对垄断势力的批评态度,同他的很多自由主义和社会主义同时代的人,如斯图尔
特·蔡斯和加德纳·米恩斯所持的立场是一致的。

无论怎样,那时\textbf{哈佛大学}是学习马克思主义经济学的好地方。研究生中有日本的马
克思主义者\textbf{都留重人和斯威齐的另一位未来合作者保罗·巴兰(在1939年以后)。社会
  主义讨论小组}像细胞分裂一样迅速增加。其中教授有约瑟夫·熊彼特和瓦西里·里昂惕夫,
他俩是研究马克思的专家,而次一级的专家爱德华·S. 梅森,则聘用斯威齐作为他的社会主
义经济学这门课程的助教。斯威齐为这门课写的讲义,奠定了他的第一本重要理论著作
《\textbf{资本主义发展理论}》的基础,该书出版于1942年,\textbf{甚至到今天它仍然是
  对马克思本人经济思想作了最好介绍的著作之一},而且也是对直到20世纪30年代用英文写
作的马克思主义危机理论的最全面的概括。正如我们在第1章中看到的,斯威齐把周期波动的
过度积累分析,同对长期停滞趋势的消费不足理论分析结合起来,在这里我们关心的是后者,
因为它与斯威齐的垄断资本学说有着十分紧密的联系。

\textbf{在对停滞问题的分析中,斯威齐大量引证了奥托·鲍威尔关于消费不足的正式模型},
该模型与由英国凯恩斯主义者罗伊·哈罗德3年后即1939年单独表述的增长模型有着密切关
系。\textbf{鲍威尔—斯威齐模型是有缺陷的},但是其理论实质表述起来很简
单:\textbf{消费作为总产品的一个比例趋于下降,因为资本家在利润不断增加的情况下没
  有足够多地花费,以保证储蓄率不变,同时工人(花掉一切所得的人)得到的收入不断减少,
  因此,如果把所有储蓄都用作投资,股本就会比消费品的产量增长得更快。“适当”的股
  本——也就是使利润达到最大化的股本——与消费水平紧密相关,因此,如果积累有超过消
  费支出的危险,那么,投资将被削减,而可以获得利润的增长将宣告结束。}简而言之,这
就是鲍威尔—斯威齐的消费不足模型(对此的批评,见以下第4节)。

\textbf{斯威齐推断,在一些相互抵消的因素的作用下,长期萧条是资本主义发展趋向的正
  常情况。投资会由于新产业的创立而得以增加,如19世纪中叶铁路的兴建,略微次要的原
  因是那些不增加盈利性生产能力而只促进需求的错误投资,也会引起投资的增长。人口增
  长和非生产性支出的扩大,能够刺激私人消费。最后,政府支出会增加。}斯威齐提
出,\textbf{迄今为止最重要的反作用趋势,是新兴产业部门的发展和人口的增长。他坚持
  认为,这两个因素在近几十年里已相当弱化了,而使政府的非生产性消费和需求创造,成
  为需求不足的重要障碍。}在这里,斯威齐显露出凯恩斯主义观点和他哈佛同行\textbf{阿
  尔文·汉森著作的影响}。汉森是后期的但却是热心的改变原来的信仰而追随凯恩斯的
人,\textbf{他的著名的“停滞论”强调人口增长下降和技术创新率下降。}在《资本主义发
展理论》中,虽然没有特别提到这两个人,但斯威齐在悼念凯恩斯时对他的赞扬已经说明了
这一点(参见以下第5章第6节),迟至1954年,斯威齐仍在赞扬汉森强调的停滞的外因问
题。\textbf{对于斯威齐来说,非生产性消费和政府支出是造成需求不足的主要障碍。}在
《资本主义发展理论》中,有关经济学的内容相对而言极少,更多的却是关于国家的政治理
论。然而,斯威齐关于非生产性消费的观点显而易见是来自于他对\textbf{垄断资本}的论述。
希法亭和列宁的观点曾被斯威齐肯定地加以引用,斯威齐与他们一样,\textbf{确信垄断增
  强代表了资本主义发展的一个新阶段},对这一阶段——这里按他自己想法提出——的资本主
义发展的运动规律必须进行再思考。他同意马克思的观点:\textbf{垄断者的超额利润主要
  是靠牺牲其他资本家的利益而获得的,而且这又造成了不同经济部门利润率的长期不同。}马
克思对产品竞争价格的分析所依据的假设是:所有产业部门的利润率都是相同的。在垄断资
本主义条件下,这一假设是难以成立的。可见,\textbf{垄断价格没有任何一般规律:只能
  说,与自由竞争条件相比,它的产量更低,价格更高。}斯威齐继苏联专家\textbf{普列奥
  布拉任斯基}的观点(斯威齐并不知道他)之后,\textbf{认为垄断也有宏观经济后果:投资
  更低,分配成本更高。}因为竞争性资本家能力太小,他们不能影响市场价格,他们的投资
决定只反映了他们希望从新积累的资本中得到的利润,而忽视了对他们已经使用的资本的影
响。\textbf{可是,垄断者也必须考虑新投资对现存资本盈利能力的影响,现存资本盈利能
  力将会由于工业生产能力的增长而下降,结果引起价格下降。其他情况是相同的,垄断者
  将比竞争性资本家投资更少,因而减少了有效需求并产生了向萧条发展的趋势。}反其道而
行的办法是\textbf{增加商业和分配上的支出},这样能够提高单个垄断者的需求,
并\textbf{扩大整个经济的消费}(通过减少储蓄的方式)。但这些支出是\textbf{非生产
  性}的,垄断资本因此具有了\textbf{提高浪费水平}的特征。

最后的结论已由普列奥布拉任斯基和纳塔莉·莫斯科斯卡两人同时作了阐述,但对斯威齐来说,
这一结论显露了\textbf{当代资产阶级理论的影响}。尽管\textbf{爱德华·张伯伦}本人从他
们的激进的(特别是他们的宏观经济)含义中退缩回来了,\textbf{但产品差别和销售成本是
  张伯伦的“垄断竞争”模型的基础,}他在1926年的博士论文中,对此作了初次阐述。他和
琼·罗宾逊两人,按照罗宾逊在1933年对“不完全竞争”的非常相似的分析,提出
了\textbf{单一市场均衡的可能性。这样,资本家将使用过剩的生产能力,因为他们会发现
  通过削价来消除这种过剩是无利可图的。}斯威齐关于垄断资本主义条件下投资水平更低的
观点,确实是对上述提法的一种动态扩充。因为张伯伦—罗宾逊坚信:\textbf{在不增加新投
  资的情况下,过剩的生产能力可以满足增长的需求。}当然,斯威齐不是唯一一位在实
施“新政”的10年中,把垄断势力和宏观经济萧条联系在一起的经济学家。

\section{介绍保罗·巴兰}
在《资本主义发展理论》中,斯威齐感谢保罗·巴兰对他的帮助。巴兰1939年已到了美国,在
接下来的四分之一世纪中,他成为斯威齐的主要合作者。巴兰在第二国际和第三国际已为自
己奠定了坚实的马克思主义基础,最早在他的出生地俄国,接着在德国,他与法兰克福学派
的弗里德里克·波洛克一起工作过。\textbf{在某些方面,巴兰的兴趣是对保罗·斯威齐的补
  充。}首先,巴兰对\textbf{落后地区的经济增长问题}表现出深切的关注。斯威齐在他的
书中,已否认来自发达国家的资本输出能够使不发达地区产生和谐的平衡发展的观点,但几
乎没有提及\textbf{不发达国家的依附性和欠发达理论},巴兰后来的论著对这一理论作了阐
述(参见以下第9章)。巴兰的第二个贡献是他在法兰克福时期就已获得的有关批判理论的知识。
巴兰远远超过他同时代的大多数马克思主义者的是,\textbf{他不仅仅关心资本主义在狭隘
  的经济意义上的生存能力,更关心资本主义作为一种社会生活方式的合理性。}巴兰的影响
使“垄断资本”理论少了些它本应具有的经济主义的含义,并对北美的马克思主义经济学
在\textbf{劳动、教育和家庭生活的文化、意识形态方面}给予了特别关注。

第三,巴兰对斯威齐关于资本主义国家的相当简略的分析作了扩充。他在发表于1952年的一
篇有关经济计划的重要论文中,分析了\textbf{可能抵消消费不足的政府支出的6种形式。其
  中4种形式——完善的社会服务、海外援助、生产企业的投资和直接的消费支出——将会引起
  工商界的反抗行为,他认为,这是因为它们对资本主义意识形态的统治造成了威胁。在巴
  兰看来,只有军费开支和非生产性的民用项目(如“拾树叶”)才是不受这些反对意见影响
  的。}因此,像社会民主主义政治家们所信奉的那样,\textbf{计划不能提供与和平的、自
  由—民主资本主义相一致的充分就业。}为了“使劳动者各得其所”,除了社会主义,其他
可供选择的无非是军事法西斯主义统治或干脆抛弃充分就业。

\textbf{巴兰和斯威齐合作的最后的、也是最重要的贡献},就是关于\textbf{经济剩余}的
概念,这一概念是巴兰1957年出版的《\textbf{增长的政治经济学}》一书的核心(参看以下
第9章)。严格说来,这里有三个不同的概念:\textbf{计划的剩余、实际的剩余和潜在的剩
  余}。\textbf{计划的剩余是社会主义经济的最适度产量与最适度消费之间的差额。}“适
度”是根据人们“对于由理性和科学所引导的社会主义社会的判断来定义的”。它与资本主
义没有关系,并且\textbf{和传统的马克思主义剩余价值理论也没有任何直接关系。实际的
  剩余是现实产量与现实消费之间的差额。}在封闭经济中,如果不考虑国家因素,这个差
额\textbf{与当前的储蓄是相等的,并且它小于剩余价值,它与剩余价值的差额等于资本家
  的消费减去工人的全部储蓄。}一般说来,实际的剩余能够从常规的国民收入账
目——用\textbf{市场价格、而不是劳动价值——中计算出来。}


巴兰把潜在的剩余定义为“\textbf{在既定的自然和技术环境中,借助于可利用的生产资料
  所能够生产出的产量,与可以被认为是必需的消费之间的差额}”。它有四个组成部
分:\textbf{首先,是上层阶级和一部分中层阶级的消费超过由社会所决定的能够被接受的
  最低消费水准的差额;其次,被放弃的产量,因为要雇佣非生产性的社会成员:军火制造者、
  奢侈品和炫耀性商品的生产者,政府和军队官员,牧师,律师,广告代理商,中间人,商
  人和投机者。(第一类和第二类有重叠的地方,因此增加了重复计算的可能性。)第三,由
  于“对现存生产机构的不合理的和浪费的管理”,包括正常的(即非萧条时期)剩余生产能
  力、被放弃的经济规模、无意义的产品差别和为了保护现有的特权和利润而对技术进步进
  行抑制所丧失的产量。最后,是那些由于总需求不足而从未被生产出来的产量。}因此,巴
兰的潜在的经济剩余是一个混合概念,它既与当时的资本主义现实相联系,又与他关于有理
性的未来社会主义思想相联系。它的积极作用在于“\textbf{预见到要对社会产品的生产和
  分配进行或猛烈或温和的重新调整,并暗示了社会结构的深刻变化。}”

在《增长的政治经济学》中,巴兰并不想计算潜在的剩余,他也拒绝预言它作为总产出的一
部分的长期发展趋势,他只是在一个单独的脚注中说明了它与传统剩余价值的关
系。(\textbf{与剩余价值不同,潜在的剩余包括由于未充分就业或对生产资源的滥用而丧失
  的产量,但不包括资本家的必要消费和在公共管理方面的必要支出。})然而,这一概念对
于他的整个理论体系在两个方面具有决定性的重要意义。首先,对现存的非社会主义经济制
度,包括落后的和发达的经济,进行批判地分析提供了一个有力的工具。\textbf{如果潜在
  的剩余得以实现,并被引导到社会生产活动中,那么,增长率将会提高,失业将会减少,
  生活水准和生活质量将得到极大改善。}事实上,\textbf{“浪费”}构成了巴兰反对发达
国家垄断资本主义的\textbf{主要观点},并用他的潜在的剩余概念对它进行了定义。第二,
潜在的剩余概念\textbf{为消费不足理论提供了一个新的更有辩护力的系统表述。}消费不足
论的批评者们反对如下观点:\textbf{这一理论所必须的是,作为国民收入的一部分,利润
  没有得到增加,储蓄也没有得到增加;换句话说,储蓄率和利润份额从长期看仍是相当稳定
  的。}巴兰认为,这将混淆现实性和可能性。就收入而言,现实的利润和实际的储蓄不会增
长,但它们的\textbf{潜在能力的确大大增长了。通过过剩的生产能力和非生产性消费,现
  实与潜在能力之间的差额已经被填充了——即剩余已被吸收。}因此,\textbf{浪费增长是
  为消费不足论辩护的最明显而又可能的证据。}如果没有浪费增长,停滞和危机将是显而易
见的。

《资本主义发展理论》一书的结束语是相对乐观的,斯威齐期待着社会主义在西欧的早日胜
利,以及美国从资本主义向社会主义(较长时期的)和平转变的可能性。15年之后,《增长的
政治经济学》描述了一个总体上较为暗淡的前景。\textbf{从根本上说,垄断资本是敌视充
  分就业的,并抵抗政府公共支出的扩张。这就使私人消费和政府用于帝国主义统治方面(特
  别是军队方面)的支出,成为解决潜在的剩余的唯一重要的出路。}巴兰因而得出结论:这
一制度在经济、政治和军事方面的稳定是相当危险的。\textbf{如果要能够避免萧条,就只
  能以麦卡锡式的病态和持续不断的战争威胁为代价。}

\section{《垄断资本》}
有关垄断资本理论的大部分部件现在被装配起来:斯威齐对于消费不足和垄断企业的分析,
巴兰对于潜在的剩余和限制政府干预的论述。留给《垄断资本》的就是找到剩余的部分拼进
来以完成这一理论。这剩余的部分就叫作“\textbf{剩余增长规律}”。\textbf{如果能够证
  明潜在的剩余确实随着时间的推移有明显的增长趋势,那么,被修正的消费不足理论就将
  得到极大加强。}发达资本主义(至少在北美)的历史将被重写,\textbf{它将不仅仅能够解
  释1945年以后长期繁荣的原因,也能预言它的日益迫近的消亡。}

巴兰在1959年发表的一篇文章中,第一次对剩余增长规律作出暗示,紧接着在与斯威齐合作
发表的两篇文章中提到剩余增长规律,这是他们长达20多年合作后首次共同发表文章。它们
是《垄断资本》核心章节的初稿,该书最后出版于1966年,也就是巴兰去世后的两年。《垄
断资本》从第一次的尝试性写作大纲到最后出版花费了近10年的时间。该书的出发点
是“\textbf{大公司}”,\textbf{大公司在小企业纯粹起被动作用的制度中,将作为主要发
  动机取代私人资本。}在巴兰和斯威齐看来,大公司同传统企业一样具有追求利润最大化的
强烈的动机;股票持有人与管理者的利益是一致的,都为了寻求更高的利润,这增强了公司的
规模和实力,并为它的管理者们提供了最大的安全和最好的晋升前景。\textbf{公司确与对
  手竞争,但不采用削价的方式,因为市场供应垄断者认识到这样做只能是自我拆台,自找
  失败。取代价格战的是心照不宣的互相勾结和共谋。这样,价格和产量水平经协议达成一
  致,并接近每一位垄断者都认为自己将最有利可图的那一价格和产量水
  平。}这一“\textbf{共同利润最大化}”模型很无耻地从当代微观经济理论中产生出来。
巴兰和斯威齐指出,激烈的竞争仍在继续,但使用的是\textbf{非价格武器},如产品差别、
产品创新和销售成本。\textbf{垄断资本阻碍价格下降要甚于阻碍创新。}既然生产成本继续
下降,同时价格又是固定的,那么,\textbf{利润就增加了}。这是微观经济学原
理。\textbf{从宏观经济上看,它反映相对剩余和绝对剩余的增长趋势。}

在《垄断资本》中,巴兰和斯威齐把\textbf{剩余定义为“一个社会生产的产品和生产这些
  产品所费成本之间的差额”}。并请读者参阅《增长的政治经济学》对此所作的进一步的分
析。在一个附录中,他们介绍了\textbf{约瑟夫·菲利普斯的统计数值}。菲利普斯是按照下
列几项来计算\textbf{剩余数量的:财产收入、浪费性支出、政府支出和促销渗透于生产过
  程引起的成本}(如不必要的模型和设计改变引起的成本)。这既不是巴兰在他的书中所定义
的实际的剩余,也不是他的潜在的剩余(\textbf{菲利普斯的剩余与巴兰的剩余不同,前者并
  不想把事先决定的产量归因于失业和过度的生产能力})。菲利普斯的观点已遭到激烈批评,
我们在本章的下一节将会看到。现在把他的结论记录在此就已足够了:1929到1963年间,美
国国民生产总值中剩余从46.9\%上升到56.1\%。1964年财产收入总计不到总剩余的13\%。

\textbf{怎样使增长的剩余得以吸收,}这是《垄断资本》的核心问题。巴兰和斯威齐的结论
是:\textbf{资本家的消费不能解决这一难题,资本家的投资也不能解决它。由于公司利润
  中用于支付红利部分的比例下降,股票持有者的消费明显减少,因而剩余中寻求投资的部
  分增加了。}消费不足论者的理由是,即便在垄断条件下对投资的刺激比竞争资本主义更强,
投资与产量之比率也决不会持续增长。\textbf{巴兰和斯威齐也否认了列宁的下述观点:资
  本输出到落后地区最终是为了吸收剩余。}美国从它在穷国的资产中\textbf{得到的收入,比它在这
些国家的投资中得到的收入更多}:\textbf{剩余正在被转移,但按照错误的方向在转移。}

剩下的问题是销售费用和政府支出。在非马克思主义读者中,《垄断资本》以\textbf{把“促销”描
写成解决剩余的一个出路而闻名:}
\begin{quotation}
  可能今天\textbf{广告已具有支配作用},为了生产者和消费品销售者的利益,\textbf{广
    告正在为反对节约和支持消费而进行不懈的战斗。而实现这一任务的主要手段是引导时
    尚变化、创造新的需要、设置新的社会地位标准、推行新的礼仪规范。}广告在实现上述
  目标方面的无可非议的成功,已极大地加强了它作为抵抗垄断资本停滞趋势的力量的作用,
  同时它也被称作为著名的‐美国生活方式”的主要缔造者。
\end{quotation}



\textbf{至于政府,来自私人既得利益的意识形态上的偏见和异议肯定了不是民用支出而是
  军事开支支配着政府对剩余的吸收。}《垄断资本》对民用消费的讨论,没有对巴兰书中的
有关问题提供什么更新的东西,它对军费开支的论述将在本书的下一章讨论。巴兰和斯威齐
最后认为,\textbf{正统的共产主义提出的“国家垄断资本主义”的术语是不恰当的,因为
  国家不是一种独立的力量,而且它在经济上的作用没有发生任何质的变化。}他们认为,垄
断资本大约可以追溯到1870年,它是随着大公司在美国经济中作为有决定性的影响因素的出
现而产生的,而且剩余增长规律从那时起就已存在。巴兰和斯威齐认为,\textbf{停滞一直
  是垄断资本的正常状态,除非它被世界大战或划时代的发明创造,如铁路(1870—1900年)或
  内燃机(20世纪20年代)的出现而抵消。}停滞的第一次真正明显标志,在1907—1915年的长
期萧条中就已显现;而20世纪30年代的大萧条,“不应该被看作是特例,而应该被看作是美国
经济体制运行的正常结果”。\textbf{1945年以后,促销的发展和军事开支持续不断的增长
  掩盖了停滞趋势,但到1963年这一趋势日益迫近地表现出来,长期繁荣的末日已不远了。}

值得注意的是,该书并没有在此结束。\textbf{关于种族歧视的一章把落后的第三世界革命
  和资本主义心脏地带那些被排斥的人群联系在了一起}(参见以下第9章)。接着,巴兰和斯
威齐又剖析了垄断资本主义条件下的生活质量。\textbf{垄断资本主义既没有满足人类的需
  要,也没有使人们快乐。}他们在回忆马尔库塞和加尔布雷思的一章中,对这些问题的论述
并不比马克思少。\textbf{因为人们日益迷失方向,冷漠和绝望,加之诸如不断地挣扎于贫
  困之中、恶劣的居住环境和由于市郊的延伸及公共交通与教育的衰退等问题,价值危机已
  经产生。}《垄断资本》的最后一章,主要运用批判理论评价了这一制度的合理性问题。巴
兰和斯威齐提出,\textbf{大公司能确保美国社会局部合理性,但不能阻止整个社会日益增
  长的非理性。}尽管算计和虚伪已渗透到日常生活的各个方面,但资产阶级思想本身正在分
化瓦解之中。由于人们受到精细的劳动分工所造成的非人性的影响,越来越疏于工作,在消
费方面也是如此,甚至“闲暇”也成了折磨。垄断资本这只有害的手,深入到了个人领地,
影响着家庭生活和性的满足。

对于巴兰和斯威齐来说,最重要的结论是,\textbf{再也不能依赖发达国家的无产阶级作为社会变革
的代理人}:

\begin{quotation}
  \textbf{产业工人成为美国工人阶级中日益减少的少数人,}而且\textbf{基础产业中工会
    的头领们,已作为消费者和意识形态上适应于这一社会的成员而融入了这一制度。}与马
  克思时代的产业工人不同,他们不是这个制度的特殊的牺牲品,尽管他们与其他阶级和阶
  层一起,遭受了这一制度的基本厄运和非理性——但却\textbf{比上不足,比下有余}。
\end{quotation}

无论如何,垄断资本将被第三世界的革命战争所推翻(参见以下第9章)。

\section{《垄断资本》及其批评者}
从《增长的政治经济学》出版之日直到现在,有大量文章批评巴兰和斯威齐。其中一部分认
为他们的方法论是非马克思主义的,而其余的则指出了他们在理论和证据上的根本缺陷。至
于前者,\textbf{正统的马克思主义者认为,《垄断资本》忽视了劳动过程和工资的决定;强
  调了剩余价值的实现而非剩余价值的生产,因而是“凯恩斯主义左派”而非马克思主义;用
  模糊的、与历史无关的和道德主义的“剩余”概念代替马克思的科学的剩余价值概念;几乎
  没有论及美国的工人阶级(而且书中所说的内容是明显不讨人喜欢的);依赖第三世界农民阶
  级的而非西方无产阶级的革命热情;否定利润率下降理论;抛弃了列宁主义对帝国主义分析
  中的关键因素。最重要的是,劳动价值论在巴兰和斯威齐的理论中没有起丝毫作用。}

虽然一种观点是真理还是谬误,同它和马克思一致还是不一致是没有关系的,但上述批评意
见多数没错。对于最后一项批评,斯威齐的回答是:\textbf{他和巴兰从未否定马克思的劳
  动价值论,}反而把它看作理所当然是成立的。在竞争资本主义条件下,\textbf{劳动价值
  转化成生产价格,而垄断资本需要第二次转形,即从竞争价格向垄断价格的转化。}马克思
忽视这个过程是可以理解的,《垄断资本》对此进行了研究。\textbf{这个相当站不住脚的
  辩护},使人想起\textbf{萨缪尔森讽刺地把转型描写成用橡皮擦去并加以替换的过
  程。}斯威齐可能得到忠告,\textbf{承认垄断资本和劳动价值论没有关系,把劳动价值论
  看成是决定价格的数量论,而且对马克思的价值量分析具有异化和拜物教理论的永久意义
  进行辩护。}斯威齐一直强调劳动价值论的这个方面的意义,这与巴兰对法兰克福学派的支
持是完全一致的。

巴兰和斯威齐\textbf{对劳动和工人有点忽视},后来哈里·布雷弗曼在《劳动和垄断资本》
中对此作了校正,以下我们就要谈到这个问题。在这一节稍后一点,我们将评论他们对于剩
余和剩余价值的分析。在其他方面,无论是从方法论上还是从理论上,斯威齐都坚决否认异
端的指责。\textbf{马克思本人对唯物主义思想的分析,要求理论要随着社会实践的变化而
  不断地得到修正},因此在资本主义发展新阶段改造政治经济学并没有什么不合适。况且
《垄断资本》所抛弃的传统马克思主义经济学的那些内容,特别是利润率下降理论和列宁对
于资本输出的论述,都是公认的理论中最薄弱的环节。而消费不足论在第二国际和第三国际
却拥有一个悠久的而又相当著名的渊源(参见本书第一卷和以上第1章)。当巴兰和斯威齐写作
该书时,谁能料到在美国将会发生无产阶级革命呢?

除了这些“原教旨主义”的观点外,对垄断资本理论有\textbf{五种实质性的反对意见}值得
进行慎重讨论。它们包括:\textbf{消费不足一般理论的正确性;“剩余增长规律”中表现的
  特殊形式,在概念和经验上的地位;缺乏工资理论以及与过度积累有关的问题;对政府支出
  的分析;(最关键的是)现代资本主义是否在本质上是垄断的而不是竞争的。}所有这些都和
一个真正关键的问题有关系:《垄断资本》真的能解释战后的繁荣和繁荣的逐渐消退吗?在本
世纪早期,\textbf{杜冈—巴拉诺夫斯基指出了消费不足理论的关键:资本主义是由利润,而
  不是由劳动人民的消费需求驱动的。}总体来说,资本家在很大程度上是\textbf{他们自己
  的主顾},以至于对一种\textbf{生产品的需求},常常依赖于其他种类\textbf{生产品}的
产量。杜冈观点的通俗意译就是:“\textbf{生产工具的工具生产工具}”(参见本书第一卷
第9章)。在资本主义经济中,资本与消费的比率没有理由不无限增长,尽管从一个假设的、
具有合理计划的社会主义组织的观点看,它是多么的荒谬。然而斯威齐的哈佛同
事\textbf{E·S·多马}提出了一个解救消费不足理论结论的办法,\textbf{即用一个恒定不变
  的资本—产出比率(在经验上更有辩护力)代替资本与消费之间想象上的不变关系。多马指出,
  这实际上强化了斯威齐的结论,因为这时会出现有效需求不足——在某些条件下——如果储
  蓄率有上升的趋势,即使一般消费倾向没有丝毫下降,也会如此,只有加速增长率才能避
  免需求不足。}不管多马模型是一种消费不足论还是一种无需讨论的观点,最要紧的
是\textbf{他揭示了停滞或危机的极大的可能性。}

第二种批评是关于经济剩余和它的所谓的增长趋势。如我们在前一节看到的,《增长的政治
经济学》中的\textbf{潜在的剩余概念,是社会批判的工具,它和马克思的剩余价值概念只
  有松散的联系。}《垄断资本》中对潜在的剩余的定义是很直白的,考虑到这样一个事实:
即非生产性劳动的使用只是吸收剩余价值,并不创造剩余价值,因此“一个社会生产的产品
和生产这些产品所费成本之间的差额”,被认为与马克思的剩余价值是很接近的。然
而,\textbf{对于生产性劳动和非生产性劳动的精确区分仍是相当有争议的。}特别是由于巴
兰和斯威齐把所有的政府支出都看成是非生产性的或吸收剩余的,并忽视了很多政府职能在
生产上的需要,因此受到了批判。\textbf{由于菲利普斯使用价格而不是价值标准,混淆了
  剩余产出和收入的定义,而且由于有关的重复计算,}有人提出了对他的计算方法的进一步
反对意见。这些缺陷都无关紧要,其中大多数已由\textbf{E·N·沃尔夫}在1977年的一项非常
仔细的研究作了纠正,该研究发现:“\textbf{调整后的剩余价值率(作为国民生产总值的一
  部分,最接近巴兰和斯威齐的剩余),在1947年以后的20年间已经明显上升了,这主要是由
  于非生产性活动增加造成的。}

自由主义批评家们永远关注的问题是,\textbf{在这一时期很少有或根本没有停滞的迹象。
  失业和过剩生产力水平都没有明显上升,经济增长率虽没有西欧和日本快,但还是相当可
  观的。}显而易见,\textbf{对剩余的吸收并没有像巴兰和斯威齐所说的那么困难。}对垄
断资本理论的第三种批评意见补充了上述观点。在\textbf{《资本主义发展理论》中,斯威
  齐根据接近繁荣鼎盛期失业减少,实际工资增加而造成的剥削率下降来解释危机(参见以上
  第1章)。在《垄断资本》中丝毫没有发现这样的观点,该书根本就没有工资理论。}这就引
出一个问题:在斯威齐早期著作中所讨论的“过度积累”是否不可能成为具有如此强大的反
萧条力量的经济的半永久特征。哈里·布雷弗曼在《劳动和垄断资本》一书中,强调
了\textbf{美国工人屈从于泰勒制的“科学管理”},虽然这是很片面的,但却给人以深刻的
印象。尽管如此,该书对工资理论并没有什么建树,而且对增加实际工资可能成为剩余吸收
的主要形式这种可能性,也没有得出什么结论。

《垄断资本》对政府论述的一些薄弱之处,增加了对基本理论分析正确性的怀疑。至少有一
位马克思主义评论家发现,\textbf{想象中的限制军事支出增长是令人难以置信的。}其他批
评家抱怨巴兰和斯威齐\textbf{极大地夸张了扩大民用政府支出的障碍}。19世纪末,对于私
人资本积累来说,这些政府支出很多在事实上是必要的。一些人也反对\textbf{琼·罗宾
  逊}所描述的巴兰对于\textbf{公共财政}的草率的论述,\textbf{否定在长时期由减税造
  成的预算赤字能够明显地刺激有效需求,这是很没有说服力的。}

然而,对巴兰和斯威齐最不利的指责是:他们完全歪曲了当代资本主义的本质,资本主义过
去一直(现在仍然是)充满激烈的竞争。\textbf{在一些批评家看来,《垄断资本》严重地夸
  大了大企业不受价格和利润的竞争压力影响的程度。}从许多大企业生产多种产品这一特点
看,仅有规模并不必然具有垄断势力。\textbf{拥有相关技术和过剩生产能力的企业,可以
  通过股份公司的形式潜在地或实际地跨入另一个行业,}在这方面的机会是无限的。在这里,
第二个条件必须要加以考虑。\textbf{由于战后贸易和资本流动的自由化,加之国际竞争的
  持续和不断升温,很多行业的“垄断程度”应该从全球而不是仅从一国范围内来把握。}关
于这些问题,“垄断资本”是一种幻觉。

\section{评价}
对于垄断资本理论的第三、第四和第五种批评意见,即关于工资、政府和国际竞争程度方面,
还有很多话要说。事后看来,很显然,\textbf{巴兰和斯威齐的观点完全是他们所处时代和
  所处国家的产物。}他们写作于20世纪50年代和60年代初期的美国,把美国资本主义当做其
他资本主义国家能从中看到自己未来的一面镜子,恰如马克思用英国(希法亭用德国)作为范
例一样。一些同代人对他们的观点感到不安,因为它们鲜明地反映了第二次世界大战的直接
后果是美国取得了经济上的霸主地位。戴维·霍罗威茨很遗憾地指出,“在《垄断资本》中,
有过于\textbf{依赖美国作为典型垄断资本主义社会的倾向,而忽视了国际体系的相互依
存”}。莫里斯·多布怀疑这本书对西欧的现实意义;詹姆斯·奥康纳甚至问道:“美国给戴高乐
主义的法国展示它的未来了吗?或者它是在走另一条路吗?”

到20世纪70年代初,这个问题开始变得不那么离谱了,因为美国原先的经济霸主地位已经削
弱,在一个开放的、竞争的、静态的和通货膨胀的,并伴有周期性的令人讨厌的工人运动的
资本主义世界经济中,美国只是几个经济强国中的老大。因此,《垄断资本》在西欧几乎没
有什么影响并不令人吃惊,在\textbf{米哈尔·卡莱茨基和约瑟夫·斯坦德尔}的影响下,那里
已经发展了另一种垄断资本理论。这个在20世纪80年代由\textbf{基斯·考林和马尔科姆·索
  耶等所代表的欧洲流派,对于垄断资本所作的分析比巴兰和斯威齐的分析要严密得多,但
  是,它对历史唯物主义概念的运用也并不明显,由于这个原因,事实上它更像后凯恩斯主
  义,}而不太像马克思主义。

一旦长期繁荣难以为继,大多数北美马克思主义者更通常的做法是从《垄断资本》和消费不
足理论转向其他危机理论。\textbf{巴兰和斯威齐的观点仅在两个方面对全世界的马克思主
  义经济学有更久远的影响:一是他们对军事支出的论述,}该内容将在以下第8章讨
论;\textbf{一是他们关—于第三世界的消费不足理论,这将在第三篇阐述。}


\chapter{下降的利润率}
\section{下降的利润率:1883—1918年}
以考察早期发展情况作为论述现代问题章节的开端,是有一个简单的理由的。在1945年以前,
利润率下降理论中几乎所有的关键问题都已被提出来了,而且其中不少问题也得到解决。因
此,很显然,\textbf{大多数参与辩论的人并不知道,战后的很多论战不过是过去几十年前
  发生的辩论的再现,}并没有什么新的内容。在早期文献中,能够发现两个问题都涉及最近
辩论中的论点和主张。首先,对利润率的长期趋势作出明确的预言,这种理论在逻辑上站得
住脚吗?第二,它为严密的马克思主义危机理论提供根据(或者甚至就是这一个根据)了吗?
进一步提出的两个问题就是:利润率确实下降了吗?如果是这样的话,为什么下降呢?这两
个问题在某种程度上是被忽视的。

马克思在《资本论》第三卷第三篇对一般利润率作了详细论述。他的观点可以用简单的代数
式表示。社会资本有机构成可写成$k=c/v$,是总不变资本与总可变资本的比率;剥削率
为$e=m/v$,是总剩余价值与总可变资本的比率。利润率是$r=m/(c+v)$。分别用v去除分子式
的分子和分母,并做适当的代换,可得到$r =e/(k+1)$。因此,利润率随$e$的提高而提高,
随$k$的提高而下降。最简单的概括就是,马克思认为,在资本主义发展的“现代工
业”或“机器制造”阶段,\textbf{$k$倾向于比$e$增长得更快}。尽管一些“反作用趋
势”在起作用,但利润率一定会逐渐下降。\textbf{这是和更猛烈的周期性危机趋势相联系
  的——至于如何联系,马克思并不清楚。}

有充分的理由说明为什么马克思要把\textbf{利润率下降}理论的阐述,\textbf{同对劳动价
  值转化为生产价格的分析放在一起,因为一般利润率和生产价格是同时决定的}(参见以下
第12—14章)。但是,甚至还在1894年《资本论》第三卷出版之前,这一理论就处在争论之
中。“有奖论文竞赛”中一个叫乔治·C·斯蒂贝林的人提出,较高的资本有机构成与较高的剥
削率是有关系的,因此,当资本有机构成提高时,利润率可能会保持不变。他的观点惹怒了
恩格斯。斯蒂贝林从美国工业统计数字中引用典型事例来说明自己的观点,但是他分析中的
错误是严重的。

很难对1899年单独出版的\textbf{本德图·克劳斯和米哈伊尔·杜冈-巴拉诺夫斯基}的观点忽
视不理。\textbf{克劳斯}用意大利文写道,\textbf{技术进步提高了劳动生产力,并因此在
  其他条件不变的情况下降低了不变资本的价值。他断言,这将提高利润率,而不是降低利
  润率。}杜冈的观点与此相似,他强调\textbf{技术革新在使不变资本各要素价格低廉和在
  提高剥削率这两方面的影响。}但是,与克劳斯为解决利润率上升的意义相反,\textbf{杜
  冈既断定马克思的危机理论的一个分支是有问题的,也断定马克思的剥削理论已被摧毁。}如
果\textbf{利润率事实上的确下降了,它可能是由于别的什么原因,而不是因为马克思所依
  据的那些原因,如实际工资的增长、工作日长度的缩短、租金的增加,或者是利润税负的
  增加。}然而杜冈的批评缺点不少。他的大量例子令人相当难懂,因为\textbf{他经常把实
  物量和价值量混淆起来},而且对这个问题没有一般的代数上的分析。他\textbf{假设在不
  同的部门有相同的资本有机构成,}这也就意味着他的分析在逻辑上被局限在一个单部门的
经济中。而且,像拉迪斯拉斯·冯·博特凯维兹之后不久所发现的,\textbf{杜冈并没有证明
  有机构成和利润率之间的任何联系;他只是指出马克思关于这种联系的观点是有缺陷的。}

然而,博特凯维兹本人却同意杜冈的结论,他努力使该理论精确地建立起来。\textbf{博特
  凯维兹认为,如果技术发明可以提高一般利润率,那么也只是资本家使用的那些使过去由
  手工完成的生产过程实现机械化的技术发明。马克思的错误是让他的资本家去计算价值而
  不是价格。}仅仅在能使利润率提高的条件下,对现存生产过程的改进将再次被利
用。\textbf{利润率下降需要至少一个部门劳动生产率下降,或实际工资的增加(马克思不同
  意这种观点)。}在博特凯维兹看来,马克思得出了另外的结论,因为他忽视了劳动生产率
增长对剥削率的影响。

相隔多年之后,马克思主义者才开始研究博特凯维兹的复杂的数学模型。然而,杜冈的观点
理解起来要容易得多,尽管有点令人吃惊,德国的修正主义者对该理论无任何建树,只是对
该理论作出的反应更加正统。\textbf{在对杜冈第一本书的早期评论中,卡尔·考茨基}以马
克思已经酌情考虑了劳动生产率增长对资本有机构成的影响为依据,为马克思作了辩护。考
茨基认为,\textbf{仅仅在不变资本价值不超过可变资本价值这种特殊情况下,杜冈的大量
  例子才是站得住脚的。在任何情况下,杜冈犯了构成上的错误:技术发明对单个资本家是
  有利的,但却使整体资本家的利润率下降。}虽然考茨基对马克思理论分析的辩护是迅速
的,\textbf{但考茨基的危机理论是消费不足论},而且和利润率下降趋势没有任何关系。在
这一点上,考茨基是他同时代马克思主义者的典型。甚至强调利润率下降延缓资本积累速度
的路易斯·布丁也采取了相似的立场。利润率下降只构成鲁道夫·希法亭解释危机原因的一个
次要因素,罗莎·卢森堡在她的《资本积累论》中措辞含糊,但举了一个利润率在扩大再生产
中严格保持不变的数学例子。在她的《反批判》中,她嘲笑一位评论者关于利润率下降会导
致资本主义崩溃的观点:
\begin{quotation}
  人们并不太清楚这些可怜的家伙们究竟怎样面对这种情况——\textbf{资本家阶级是否会由
    于对低利润率感到绝望而在某个时刻自杀},或者他们是否会因此宣告商业是如此糟糕,
  以至于根本不值得人们为它付出太多,并因此而把钥匙交给无产阶级呢?然而,这种安慰
  不幸被马克思的一句话驱散了,这句话就是“\textbf{大多数资本家会通过大批量生产来
    弥补由于利润率下降所造成的损失}”。因此,\textbf{在利润率下降引起资本主义灭亡
    之前,还有一段时间,粗略说来要直到太阳烧毁。}
\end{quotation}
事实上,俄国数学家\textbf{乔治·冯·查洛索夫}已经在他1910年出版的《马克思主义体系》
中回答了卢森堡的反诘。他指出,\textbf{利润率对资本积累率构成了最大限制。}如果工人
花费掉他们的\textbf{全部收入},所有的积蓄都是由资本家提供的,那么很容易得
出:\textbf{积累率一是资本家的储蓄倾向,二是利润率的结果。}用现代符号表示就
是 $g=s_c \times r$ ,这里$g$是\textbf{储蓄(假设和投资相等)和资本的比
  率};$s_c$是\textbf{储蓄和利润的比率};而$r$是利润率,\textbf{是利润和资本的比
  率}(参见以下第15章)。查洛索夫认为,\textbf{$s_c$趋向一致,以至于利润率下降会引
  起积累率下降。}而且他通过马克思\textbf{对资本过度生产}的分析,\textbf{把利润率
  下降和周期性危机的原因直接联系起来。查洛索夫似乎是自马克思以后第一位认为资本有
  机构成提高会降低最高利润率的人,他认为,如果工资是零,$r=m/c$,它是死劳动与活劳
  动之比率的倒数。}

尽管如此,查洛索夫依然是\textbf{马克思定律的强有力的批评者},他进一步阐述了杜
冈—巴拉诺夫斯基的反对理由,并增加了一些他自己的观点。他认为,\textbf{均衡利润率不等
于$m/(c+v)$;资本家不会采纳降低利润率的发明;而且甚至那些提高第\Rnum{1} 部类有机构成的技术
上的改进,也能够使不变资本的各要素便宜到足以降低第\Rnum{2}部类的有机构成,并增加两大部
类的剩余价值生产。只有当李嘉图的报酬递减开始,或者实际工资提高的时候,利润率才会
下降。}查洛索夫得出结论,马克思的利润率下降理论“不是什么定律,而是一个明显的错
误……按照资本主义经济规律,利润率从来也不会下降”。该理论中的这些弱点,足以摧毁
整个马克思主义政治经济学的目标。然而,除了奥托·鲍威尔的简短的不屑一顾的评论之外,
查洛索夫的书几乎没有引起人们的注意。在传统的俄国马克思主义者(普列汉诺夫、列宁、布
哈林和托洛斯基)的作品中,缺乏对利润率下降的分析,这表明他们在这个问题上与罗莎·卢
森堡的观点是完全相同的。

因此,\textbf{到1918年对利润率下降理论的一般评价是:马克思低估了技术进步对劳动生
  产率的影响;这些影响有助于降低不变资本价值和提高剥削率;而且作为技术进步的结果,
  利润率不但不会下降,反而可能会上升。}传统的马克思主义者和马克思本人一样,说话是
留有余地的。然而,大多数马克思主义者还是否认在长期中反作用趋势足以阻止利润率下降,
而且很少有人对技术进步确实将提高利润率这种观点妥协。没有人把利润率下降理论看成是
马克思危机理论中很重要的一部分,也\textbf{没有谁(除了斯蒂贝林)引证任何一点经验证
  据}。

\section{1918—1945年}
第二阶段所发生的批评家们的争论,先是由\textbf{纳塔莉·莫斯科斯卡}、接着由日本杰出
的经济学家\textbf{柴田敬}的努力得以大大增强。自相矛盾的是,这也是一些马克思主义者
开始用利润率下降理论来解释经济危机的一段时间。然而,这两方面的进展主要发生
在\textbf{国际马克思主义的外围}。传统的社会民主主义者,如考茨基和希法亭,并没有进
一步关注这些讨论,而且在共产主义经济学中这种忽视是相同的。1945年以后,利润率下降
日益被看作列宁的“腐朽”资本主义概念的分析基础,列宁用它来解释资本输出、日益激烈
的帝国主义竞争和战争。但是列宁的《帝国主义论》对这个问题却是模棱两可,而且绝大多
数列宁主义(和后来的斯大林主义)著述者在两次世界大战之间都把他理解成是消费不足论者。
因此,1935年,第三国际的主要理论家\textbf{尤金·瓦尔加把利润率下降趋势看成是资本有
  机构成提高的一个偶然结果},资本有机构成提高的主要影响,是由于\textbf{使用的可变
  资本减少而降低了工人阶级的购买力。然而,有机构成将会提高,而利润率下降被看成是
  理所当然的。}

\textbf{利润率下降与危机的联系首先}由德国的大学教师\textbf{埃里克·普雷泽尔}所强调,
对于他来说,\textbf{利润率下降是马克思危机理论的基础,它可用于解释商品的生产过剩
  和激烈的竞争战。}普雷泽尔摒弃了以前把马克思看成是消费不足论和比例失调论理论家的
做法。他认为,这涉及到一个方法论上的错误,因为“黑格尔的历史哲学并不比利润率下降
定律更令人明白”。五年后,这一点被\textbf{亨里克·格罗斯曼}所接受,他认
为,\textbf{为了与马克思的唯物主义历史观相一致,资本主义的经济矛盾必须追溯到剩余
  价值的生产,而不是剩余价值的实现。}格罗斯曼从马克思的\textbf{周期波动理论}和更
惊人的\textbf{经济崩溃的预言}推论:\textbf{剩余价值相对于所用资本的下降,而这将逐
  渐使资本家们不再可能保持原来的积累速度,并且不可能再维持他们自己的消费支出水
  平。}与法兰克福学派有联系的独立的马克思主义者格罗斯曼,由于\textbf{忽视了技术进
  步对劳动生产力的影响,并在不考虑技术进步对资本家利润的影响的情况下假定资本家会
  继续积累},而遭到了来自各个方面的猛烈批评。社会民主主义者汉斯·尼斯尔断言,即使
格罗斯曼的分析是对的,仍不能在利润率下降和经济危机来临之间建立联系。甚至当利润率
下降时,积累还会继续,只要\textbf{利润率仍是正}的;而且,\textbf{那些以牺牲自己的
  对手为代价来提高自己赢利能力的成功的资本家,当然会继续扩大他们自己的生产能力。}

1936年,对于\textbf{商业循环和利润率下降之间联系的第一次确实有道理的解释,来自奥
  地利社会民主主义者奥托·鲍威尔,}那时他还被流放在捷克斯洛伐克。鲍威尔的出发点是
处于\textbf{周期谷底}的经济,该经济受到了一种(无法解释的)外部刺激。由于同样的(未
充分利用的)工厂和设备能生产出更多的产品,那么生产力的利用效率将提高。因
此,\textbf{会雇佣更多的工人,这样会降低资本有机构成并提高利润率,从而促成新一轮
  的资本积累。}这种情况进一步推进着经济繁荣。逐渐地,新的投资速度加快到足以提高资
本有机构成。\textbf{现在,所有的事都依赖剥削率。}鲍威尔认为,\textbf{如果这一上升
  趋势足以使资本家保持现在的利润率,随利润变动而变动的工资下降就可能消除消费不足
  的危机。如果剥削率滞后于有机构成,利润率会下降。公司对这种情况的反应是削减它们
  的股息或红利;金融市场会发生崩溃,直接导致投资下降,并因此造成经济衰败。}尽管鲍
威尔的分析\textbf{缺乏规范的周期模型和能带来繁荣的内生机制},但是,在当代众多利用
利润率下降分析周期波动理论和引入有效需求问题的种种努力中,他的分析是最有说服力的。
在某种程度上,鲍威尔的分析也预示了战后如欧内斯特·曼德尔这样的理论著述者们更具雄心
的综合。

当格罗斯曼和鲍威尔正在构建危机理论时,\textbf{纳塔莉·莫斯科斯卡已回到杜冈—巴拉诺
  夫斯基对这个问题的分析方式上:在不提高实际工资的条件下,技术进步真的能降低利润
  率吗?}莫斯科斯卡以马克思用来衡量技术进步的标准作为她分析的起点。\textbf{马克思
  认为,为了对资本家有利,一台新机器必须至少节省和它所花费的成本相同的有酬劳动。}莫
斯科斯卡用大量的例子说明:这个判断标准引出另一个标准。在马克思看来,\textbf{一项
  能节约劳动的技术发明——如其他条件都相同,实际工资保持不变——也能提高利润率。在
  有限制的条件下,资本家并不关心是旧技术还是新技术,因为劳动价值的净节约是零,利
  润率将不会改变。}因此,杜冈是对的。\textbf{能降低利润率的新技术,不能满足马克思
  的节约劳动费用的判断标准。}

把莫斯科斯卡的有限制条件下的例子稍加修改,就可以用来说明她的观点。在一个单一部门
的模型中,我们假设,同样的工人使用同质的生产资料(谷物),生产大量的相同商品。旧技
术用实物单位可表示为:
\begin{gather}
  170谷物+340劳动 \to 510谷物 \label{e:510gu}
\end{gather}

该公式表明:在每个生产阶段,340个工人把170吨谷物转变成510吨的总产出。净产出
是340吨;\textbf{劳动生产力被定义为每个工人的净产出,是340比340等于1;而1吨谷物的
  价值是劳动力价值的倒数(即,也是340比340等于1)。}如果我们假设,\textbf{每个工作
  日的一半支付了报酬,}另一半没有付酬,那么,劳动力的价值是二分之一,而\textbf{剥
  削率是百分之百。这也就意味着净产出的二分之一给了工人,每个工人消费$170 /
  340=1/2$吨谷物;剩余的170吨谷物代表剩余产品并由资本家获得。}记住:1吨谷物的价值
是1,这个公式用价值形式可表示为:
\begin{gather}
170c+170v+170m=510
\end{gather}
因此,利润率是$170/(170+170)=50\%$莫斯科斯卡认为,新技术可以表示为:
\begin{gather}
340谷物+340劳动 \to 765谷物
\end{gather}
可以看出,这个公式正好满足马克思的判断标准。如果规模收益不变,765吨的总产出能在旧
技术条件下通过分别增加50\%的谷物和劳动的使用而生产出来,得出:
\begin{gather}
  255谷物+510劳动 \to 765谷物 \tag{\eqref{e:510gu}$a$} \label{e:510suba}
\end{gather}

比较(7.1a)和(7.3),很显然,新的生产过程多使用了85吨谷物,价值是85,而且少使
用170个工人,其劳动力价值是二分之一,这表示\textbf{节约了85个单位的直接劳动的劳动
  力价值}。因此,当使用原有技术的劳动价值进行比较时,\textbf{包含在新生产资料中的
  额外劳动恰恰等于新技术所允许的有酬劳动的节约。}

使用新技术,净产出增加到($765-340=425$);每个工人使用的生产资料已经翻了一
倍(从$170 / 340=1/2上升到340/340= 1$);而且劳动生产力提高了$25\%$(以前每个工人的
净产出是$340/340=1,现在是425/340=1.25$)。因此,谷物的单位价值已经下降,
从$340/340(=1),下降到340/ 425(=0.8)$。 \textbf{如果实际工资没变,每个时期每个工
  人的工资是1吨谷物的二分之一,}工人得到170,而剩下的$425-170=255$吨构成由资本家
所有的剩余产品。那么,价值量能够像前面那样计算:不变资本是$272(340 \times 0.8)$,
可变资本是$136(170 \times 0.8)$,而剩余价值是$204(255 \times 0.8)$,得出:
\begin{gather}
  272c+136v+204m=612 \label{e:612}
\end{gather}
剥削率等于$150\%(204 \div 136)$,利润率没变,仍是$204 \div (272 +136)=50\%$。如果
剥削率保持不变,还是100\%,利润率确实会下降:
\begin{gather}
272c+170v+170m=612  \tag{\eqref{e:612}$a$}\label{e:612suba}
\end{gather}
利润率$r=170 / (272+170)=\mathbf{38.5\%}$。但是,\textbf{这会涉及到实际工资的上
  升(马克思在论证中排除了这一点)},因为170个单位的可变资本代表$170 \div
0.8=212.5$吨的谷物,每个阶段每个工人将能够消费$212.5 \div 340=0.625$吨谷物,而不
再是原来的0.5吨。\textbf{劳动生产力提高不超过25\%也能使利润率趋于下降。}但正如我
们已看到的,\textbf{按照马克思的节约劳动标准,这项技术发明对于资本家来说是不能接
  受的},因此,这并不代表对其理论的批判。然而,如果劳动生产力\textbf{提高25\%以上,
  在工资保持不变的条件下,利润率将会提高。}

莫斯科斯卡从中得出以下理论:
\begin{quotation}
  这是一个\textbf{动态的,而不是一个历史的规律。}它没有表达一个历史事实——利润率
  下降,而\textbf{只是表达了两个变量之间的相互依存},即:
\begin{enumerate}
  \item 当剥削率不变时,利润率下降。
  \item 当利润率不变时,剩余价值率上升。
\end{enumerate}
因此,该规律只表达了一种机能性的关系,而且像“利润率下降趋势规律一样”,也可以把
它叫做“剩余价值率上升趋势规律”。
\end{quotation}

莫斯科斯卡的观点与杜冈的问题是相似的。\textbf{按照劳动价值而不是生产价格来说明}这
些观点,而且是在\textbf{单一商品经济},即只生产一种商品的经济范围内来研究马克思的
观点。\textbf{然而,资本家是按照价格量,而不是劳动价值来作出决定的,}而且在生产多
种商品的经济中,资本家只能这样做。因此,莫斯科斯卡在综合批评的论证中有两个缺陷:
首先,马克思判断一项发明对资本家是否有利的标准,必须按照价格术语来加以具体说明,
并且要求新增不变资本所增加的成本要大于工资成本的降低,而且他的观点必须以此为根据
进行评价。第二,这种评价应该在多商品的经济条件中进行。

莫斯科斯卡的书出版5年之后即1934年,\textbf{柴田敬}发表的一篇文章证明,杜冈—博特凯
维兹—莫斯科斯卡的利润率上升原理\textbf{的确可以建成既能用价格,也能用劳动价值来加
  以详细说明的模型。}柴田敬使用的只是\textbf{循环资本的三部门模型,为了简便起见,
  开始时模型中每个工人生产资料的数量和其他部门是相同的。}稍微与众不同\textbf{(柴
  田敬没有命名这三种商品},而且以很不同的方式开始论述他的观点)的是,他的模型可以
写成:
\begin{align}
8/3钢+4/30劳动 &\to 4钢 \notag \\
2/3钢+1/30劳动 &\to 1谷物 \notag \\
2/3钢+1/30劳动 &\to 1金
\end{align}

如果每个时期每单位劳动的实际工资相当于5吨小麦,这个模型就是\textbf{简单再生产}。
钢的产量恰好等于三个部门所使用的钢的数量(4吨),\textbf{全部谷物产量用来养活工人},
因为$(6/30)(5)= 1$;\textbf{金的全部产量都归资本家所有}。通过对以下方程组的解释,
可以得到三种商品的劳动价值($\lambda_s,\lambda_c,\lambda_g$)

\begin{align}
8/3 \lambda_s +4/30 &=4 \lambda_s \notag \\
2/3 \lambda_s+1/30 &=\lambda_c \notag \\
2/3 \lambda_s+1/30 &=\lambda_g
\end{align}
由此得知,$\lambda_s=\lambda_c=\lambda_g=1/10$,而价值关系是
\begin{gather}
8/30c_1+2/30v_1+2/30m_1=4/10  \notag \\
2/30c_2+1/60v_2+1/60m_2=1/10  \notag \\
2/30c_3+1/60v_3+1/60m_3=1/10
\end{gather}
在每个部门中资本有机构成是4,剥削率是100\%,利润率是20\%。用金来表示前两种商品的
价格$p_s和p_c$,用$w和r$分别表示工资率和利润率,通过解下面的方程组可以得
到\textbf{相应的价格量}:
\begin{equation}
\begin{aligned}
(8/30p_s+2/30w)(1+r) &=4/10p_s \\
(2/30p_s+1/60w)(1+r) &=1/10p_c \\
(2/30p_s+1/60w)(1+r) &=1/10p_g
\end{aligned}
\end{equation}

这里我们已经知道,$w=5p_c,而且p_g=1$。这些方程式表明,每种商品的均衡价格必须能够
使资本家收回他们的成本并得到一般的\textbf{平均利润率r}(参见本书第一卷第3章和以下
第12—14章)。解方程组(7.8)得$p_s=p_c=1,而且r=0.2$。

现在,柴田敬介绍了几种技术变化,其中,每一种都包括\textbf{一点生产资料使用的增加
  和一点活劳动使用的减少}。在第一种情况下,他增加了第\Rnum{1} 部类和第\Rnum{2}部类中每单位产出
中\textbf{钢}的投入量,从2/3提高到401/600,相应地减少了\textbf{劳动投入},从1/30减
少到199/6000。金业的生产条件没有改变。他在求解(7.6)和(7. 8)方程式的有关变量时指出,
两大部类的有机构成已经如所预料的那样上升了(从4上升到4.03),同时价格和利润率已改变
了:$p_s=p_c=1.001,而且r=0.1988$。但是,这对马克思的利润率下降规律没有任何帮
助。\textbf{两种商品的价格已经上涨。因此,这种技术创新不能节省成本};有理性的资本
家都不会引进这种创新,所以这与当前讨论的问题是没有关系的。

第二种情况是关于\textbf{在不改变价格的条件下提高有机构成}。(柴田敬在前两个部门中
选择新的投入系数,这样做没有产生不同于等式(7.5)所描述的第一种情况下(7.8)等式所计
算的价格。)取得这种结果是由于提高了钢的需求量,\textbf{每单位钢和谷物产出需求}的
钢提高到$4100/6006$,而且在每一部类直接劳动需求减少到$199/6006$。柴田敬使用作了适
当修改的方程式(7.6)的变形发现,\textbf{有机构成与剥削率以相同的比率上升了,但利润
  率的价值率没有改变。}新方程组(7.8)更证实了这一点,从(7.
8)得出$p_s=p_c=1,r=0.2$。这个例子与莫斯科斯卡的“限定条件”是很相似
的:\textbf{成本没有减少,利润率没有变化。资本家对于技术的新旧是无所谓的。}

现在分析第三种变化,即在前两个部门中,单位投入的需求量改变
为 $401/601$和$199/6010$。这一新的价值关系表明,\textbf{有机构成、剥削率和利润率
  都得到了提高}。价格方程式更证明了这一点,该方程式给$p_s=p_c=0.999, r=0.20080$。这
项创新\textbf{减少了成本},因此,对资本家来说是可以接受的。\textbf{然而,只有当实
  际工资增加时,才不会导致利润率的上升。}

1934年,柴田敬明确地运用生产价格直接探讨马克思的观点,他的分析比莫斯科斯卡的要深
入。但是,\textbf{他没有研究真正的多种商品经济的复杂性。在他的文章中,生产钢的部
  门和生产谷物的部门的资本有机构成始终是相等的。}因此,这两种商品总是在相同的技术
条件下被生产出来的,而且从分析问题的角度看,这两种商品可以作为一种商品来讨论。当
然,\textbf{柴田敬允许在金的生产上有不同的资本有机构成,但博特凯维兹在1907年就已
  指出,这样的“奢侈品”生产部门对利润率,或对非奢侈品价格不会有任何影响。}

因此,柴田敬的观点只适用于前两个部门,而且还可以减少到一个部门。尽管讨论的只是表
面的现象,但柴田敬并没有研究莫斯科斯卡文章中的第二个局限性。

然而,柴田敬1939年发表了一篇更具概括性的论文,专门分析研究了这个问题。事实上,他
提供了一个\textbf{里昂惕夫投入-产出模型的早期例子},从这个例子(已知实际工资),可
以推导出\textbf{生产价格和利润率。这篇论文在有关转形问题的讨论中起着里程碑的作用},
以下第12章将对此作详细分析。这里需要注意的是,柴田敬不辞劳苦地提出三个命题,这三
个命题\textbf{在1945年以后都正式得到了进一步的证实。}第一个命题是,\textbf{已知实
  际工资,在不参考第\Rnum{3} 部类的劳动价值或生产条件的情况下,通过第\Rnum{1} 部类和第\Rnum{2} 部类的
  投入系数与资本周转期的统计资料,可以计算出该系统的利润率。第二个命题是,只要实
  际工资保持不变,无论是第\Rnum{1} 部类还是第\Rnum{2} 部类的降低成本的技术创新都会使利润率提
  高。}后来在由冲盐伸夫做出精确证明之后(参见以下第4节),这一结论也就
以“\textbf{冲盐定理}”闻名。第三个命题是,\textbf{如果假设不变资本随着时间的变化
  呈线性贬值,那么这些结论不会因为不变资本的引入而受到影响。}

因此,柴田敬在证明马克思的利润率下降理论的虚假性方面做出了重大贡献。虽然他的著述
只在日本出版,但却是用英语写的,而且,不久以后得到了保罗·斯威齐的注意,大概是通过
斯威齐的哈佛大学的同事都留重人,都留重人曾向柴田敬的第一篇论文的结论发起过挑战,
但没有成功。\textbf{斯威齐在1942年出版的《资本主义发展理论》中,引用了柴田敬的观
  点,但显然他并不理解利润率上升原理的重要性,从而否定了这一点。}然而,斯威齐的确
否定了作为一般规律的传统利润率下降理论,\textbf{因为他不能说明有机构成一定比剥削
  率提高更快的原因。}斯威齐断定,如果利润率真的下降了,那更可能是实际工资提高的结
果,或者是有利于劳动者的国家干预的结果。

这时期的英国的理论著述者好像既不熟悉莫斯科斯卡,也不熟悉柴田敬,但是,他们对于马
克思的分析的一般合理性的保留意见显而易见是相似的。莫里斯·多布在20世纪40年代初认为,
利润率的变动趋势取决于技术变化、劳动生产率提高和剥削率的关系。利润率可能会逐渐下
降,但这是根据情况而定的,而且可能被长期延迟。琼·罗宾逊更进一步,她得出了与莫斯科
斯卡很相似的结论。罗宾逊认为,人们可能像谈论利润率下降一样也谈论剩余价值率上
升,“换个说法也是一样的”。她对于鲍威尔把有效需求引入分析之中产生共鸣,但却认为
马克思对利润率下降的分析,是“一个错误的线索……(它)没有说明任何问题。”

\section{1945—1973年}
战后的论战在三个层次上展开。首先,在本世纪50年代末和60年代初,英国、美国和日本的
经院经济学家们继续研究利润率下降理论的逻辑统一问题;他们都发现在这方面或多或少是
有缺陷的。紧接着,在70年代初长期繁荣的后期,\textbf{对亨里克·格罗斯曼的重新发现,
  激起人们对这一理论在马克思主义政治经济学中作为替代凯恩斯主义和消费不足论的影响
  的理论的有力辩护(参见以上第5章)。}同时,有人正努力为马克思《资本论》第三卷的分
析提供经验上的证明,并把它和当代资本主义经济的现实发展联系起来。

最初的贡献在1956年来自\textbf{H·D·迪金森,他使用新古典经济学的分析工具来揭示资本
  有机构成和剥削率之间的关系。}在实际工资保持不变的条件下,\textbf{迪金森使用一个
  科布—道格拉斯生产函数把资本的增长和产出的增长联系起来。}他认为,只有在极特殊的
条件下,利润率才会持续下降;否则,一开始它将随着资本有机构成的提高而提高,只有当
资本积累超过某一临界点时,利润率才会下降。因此,迪金森断言,\textbf{虽然利润率下
  降是无法逃脱的,但它可能被延迟到“遥远未来的某个时间”。}

这是对马克思经济学和新古典经济学的有独创性的、被后人广为模仿的,\textbf{但最终是
  失败的一次综合。}对于\textbf{把资本积累、劳动生产力和剩余产品的增加联系起来,一
  般而言没有什么异议},杜冈、莫斯科斯卡和柴田敬都已经这样做了。\textbf{虽然不变资
  本不创造价值,但是从提高商品产量这个意义上说,它又是生产性的,马克思本人非常了
  解这一点。}迪金森论证的\textbf{问题在于,他使用了已经被证明具有严重缺陷的新古典
  主义思想。只有非常特殊种类的技术才能按照总生产函数来加以表述。}

1960年,\textbf{罗纳德·米克}在一篇文风朴实,但却很有影响的文章中避免了这种易犯的
错误。米克\textbf{没有使用代数归纳法,而完全依靠大量的似乎合理的统计数字来说明问
  题,在这些例子中技术进步既提高了有机构成,也提高了利润率。}他的结论与迪金森的很
相似。米克断定,“如果我们从很低的有机构成开始,那么我们就可以这样说,根据马克思
的前提,利润率的变动‘趋势’是先上升,\textbf{经过一段时间以后开始下降。}”最初的
上升越大,并且向下的转折越迟,那么,最初的剥削率就越低,与有机构成的既定提高相联
系的劳动生产率的增长就越大,而随第\Rnum{1} 部类而转移的第\Rnum{2} 部类的劳动生产力增长就更
快。\textbf{这最后的结果有些令人吃惊,因为第\Rnum{1} 部类劳动生产力的增长使不变资本各要
  素更廉价,并因此抵消了假定的有机构成的提高。这指出了米克方法中的一个缺陷。}像莫
斯科斯卡、柴田敬和迪金森一样,\textbf{米克把资本有机构成看作是一个随着时间的推移
  具有内在增长趋势的参数。然而,它应该被看做是一个内生变量,它的价值是从一个与技
  术变化特点有关的假设推导出来的。}事实上,\textbf{如果不变资本的单位价值下降得足
  够快,技术进步和有机构成的下降可能是联系在一起的。}

\textbf{第一个模型只能追溯到1972年,它似乎是荷兰经济学家阿诺德·希尔吉的模型。}在
该模型中,有机构成的内生性已得到明确表述。\textbf{希尔吉主要依靠萨缪尔森的那些深
  受里昂惕夫和冯·诺伊曼的开创性分析影响的早期作品。}正如一位传记作家所声称的,诺
贝尔奖得主保罗·萨缪尔森虽不是将现代活动分析应用于马克思的第一人,但他无疑是最杰出
的,而且也是最有毅力的。在第14章我们将介绍萨缪尔森在\textbf{20世纪70年代}与劳动价
值论的长期对峙。\textbf{1957年},他的论文“\textbf{工资与利息:对马克思经济模型的
  现代剖析}”,与这里讨论的问题是有关系的。在这篇论文中,萨缪尔森假设:没有连带生
产或稀缺的自然资源,而且实际工资保持不变,那么,如果一项技术革新确实被资本家所采
用,利润率一定会上升。如果利润率不能上升,资本家使用旧技术日子也能过得不
错。\textbf{假定已知资本家的行为是有理性的,那么不可能同时存在一是技术进步、二是
  实际工资不变,三是利润率下降。}因此,\textbf{如果技术进步没有增加实际工资,那么
  它一定提高了利润率。这是对柴田敬结论的概括和更好的表述。}萨缪尔森在其文章的数学
脚注中很大胆地说明了这一观点。然而,日本经济学家\textbf{置盐信雄}接着提供了精确的
证据,这个观点也就以置盐的名字被命名为置盐定理。到20世纪60年代初,经院派经济学家
们对于利润率下降趋势提出了严格的保留条件。在迪金森和米克看来,凡事依赖于有机构成、
劳动生产力和剥削率之间的关系,而且他们相信,利润率在开始下降之前,会在相当长的时
期呈上升趋势。\textbf{按照萨缪尔森和冲盐的观点,技术改进引起利润率下降这种后果只
  有同时在实际工资上升时才会出现。}但是,马克思认为,机械化也取代了工人,而且由此
造成的失业只会“妨碍工资的增加”。因此,用萨缪尔森的话说,\textbf{马克思似乎“太
  依靠一匹马了”。}因此,科尼利厄斯·卡斯托瑞安迪斯并不是惟一一位批判利润率下降,
并用它来否定任何不是由剩余价值实现困难造成的资本主义危机理论的社会主义者。

这些疑问由于吉尔曼自斯蒂贝林以来对这一定理进行的首次严肃的经验上的研究而得到加
强。\textbf{约瑟夫·吉尔曼在1957年使用19世纪末20世纪初的美国官方资料,发现在1919年
  前后利润率变动趋势发生明显的断裂,}1919年之前,资本有机构成明显上升,超过了剥削
率的上升幅度,造成预料中的利润率下降。\textbf{1919年以后,这三个比率几乎都没有发
  生变化。}

吉尔曼断定,或者马克思的定理只适用于资本主义发展的早期阶段,或者必须对马克思的定
理重新进行表述。\textbf{他指出了垄断资本主义条件下技术改进具有“节约资本”的性质,
  尤其是指出了非生产性支出的快速增长,如销售成本和管理成本的增长,这在巴兰和斯威
  齐的《垄断资本》中是很重要的(参看以上第6章)。}他认为,这些从剩余价值得出的推论
与马克思的观点是一致的。\textbf{在不考虑可变资本的条件下,如果把非生产性支出表示
  为$u$,那么实际利润率是$(m-u)/c,而不是m/c$。}利润率不仅依赖于剥削率和有机构成,
而且也依赖于$u/v$,也就是说,\textbf{依赖于非生产性支出和生产工人工资之间的比
  率。}吉尔曼认为,如果以此为基础重新加以判断,\textbf{在1919年以后利润率的确已经
  下降,但它是由于$u/v而不是c/v$的提高而引起的。}

吉尔曼的一系列分析使用的都是\textbf{市场价格而不是劳动价值},所以并不能用它们来估
量马克思的$r=m/(c+v)$,这个公式的三部分都是用劳动价值定义的。然而,\textbf{他分析
  中的主要问题是假设全部生产能力都得到了充分利用。}保罗·巴兰在给吉尔曼的书所做的
书评中强调了这一点:
\begin{quotation}
  的确,看待这个问题的方式是自相矛盾的,而且我怀疑吉尔曼本人充分认识到了它的含义,
  因此可以得出,\textbf{在缺乏吉尔曼所说的非生产性支出的情况下},美国资本主义现在
  仍会生活很好。\textbf{利润量和利润率都将更高。显然,为了保持繁荣,这些数量可观
    的利润将必须在投资(在国内和国外)和(或)资本家的消费方面寻找必要的出路,以免被
    普遍的萧条和失业所“抵消”(和消灭)。}既然吉尔曼不可能认为萧条和失业是资本主义
  制度的幸事,那么\textbf{他一定相信较高的利润率会刺激投资,并使投资自动增加到维
    持适度就业所需要的水平。}

  \textbf{对于投资的利润率的弹性充满信心与美国经济普遍(和日益加强的)垄断化几乎就
    是矛盾的,吉尔曼正确地看到并强调了这一点,很显然,他并没有被这种矛盾所扰乱。}他
  似乎对商业循环以毫不掩饰的沮丧反作用于政府非生产性支出的减少没有什么深刻的印
  象,\textbf{或对由华尔街每一次“和平恐慌”爆发所引起的痛苦也无动于衷。}
\end{quotation}

\textbf{巴兰断言,在总需求不足的条件下,非生产性支出可能会通过使更多已经生产出来
  的剩余价值得到实现来增加利润。}

在\textbf{欧内斯特·曼德尔}的著作中,\textbf{利润率下降和消费不足危机模式之间的紧
  张关系是相当明显的,曼德尔是这一时期引用赢利能力变动趋势的经验证据的惟一一位马
  克思主义经济学家。}曼德尔在早期作品中,对这一理论重视不够。在他1962年的《马克思
主义经济理论》中,与比例失调和需求不足危机理论相比,利润率下降所占的篇幅要少得多。
在5年后出版的《马克思主义经济学导论》中,强调的重点仍然是有效需求,而不是利润率。
直到《\textbf{晚期资本主义}》1972年用德语、1975年用英语出版,\textbf{曼德尔才从帕
  尔乌斯和康德拉季耶夫那里借用50年周期或资本主义发展的“长波”概念(参见以上第1章
  和以下第16章),强调利润率下降是理解长期繁荣和繁荣消退的关键。}在曼德尔看
来,\textbf{积累是利润率的一个函数。}在\textbf{长周期中},经济景气是由那些能
够\textbf{降低有机构成和提高剥削率}的“触发因素”引起的。曼德尔认为,\textbf{战后
  的繁荣}就是这种情况,其明显特点就是表现为\textbf{长期有力的资本积累高涨和短期微
  弱的经济不振}。在1945年以后发挥作用的特有触发因素包括:使不变资本的各要素变得廉
价的关键技术改进,以及\textbf{缩短流通时间的交通和通讯的改进;这两个因素抵消了有
  机构成上升的趋势。}同样重要的是,欧洲工人阶级反法西斯主义的失败导致了剥削率的实
际上升。随着“第三次技术革命”能量的耗尽,以及工人阶级对自身力量与自信的觉
醒,\textbf{长期繁荣将让位于一个新的康德拉季耶夫下降趋势,}而资本主义将面临古典的
经济危机的重新开始。

曼德尔用朴素的语言,从正统经济学著作和官方的统计资料这些无所不包的百科全书式的内
容中引用大量的经验性的详细材料,讲述了一个吸引人的故事。然而,\textbf{他的机敏解
  释只会隐藏他分析中的真正缺陷。}在《晚期资本主义》中\textbf{引用的资料掩盖了
  他“顽固拒绝面对现实”的立场,特别是顽固拒绝承认资本—产出比率——基本上代表马克
  思的有机构成——几乎一个世纪都保持不变或一直在下降,因此,利润率的波动主要是由于
  剥削率的改变。}关键因素是\textbf{阶级斗争},而不是技术变化。从理论上
说,\textbf{曼德尔并没有正视冲盐定理和迪金森—米克的利润率上升可能是一种普遍现象而
  不仅是长周期现象的观点。}而且,由于他不能否定他的早期凯恩斯主义观点,这必然使他
模棱两可。正如一位评论者所断言的:
\begin{quotation}
  曼德尔是否认为资本主义具有\textbf{生产过剩的内在趋势,并周期性地以利润率下降的
    方式表现出来,或者生产过剩本身是由利润率下降引起的,这从来是不清楚的。}因此,
  他经常提到的需求和剩余价值实现有点像存在于真空中,而且使人不明白的是,它们与他
  的资本主义发展基本理论如果说有点什么联系的话,那么这种联系是什么呢?
\end{quotation}

坚决的反凯恩斯主义成为一种日益有影响的思想潮流的明确特征,这是以亨里克·格罗斯曼的
观点为依据的,并得到他的弟子\textbf{保罗·马蒂克(美国)和罗曼·罗斯多尔斯基}(德国)的
宣传。在众所周知的\textbf{“资本逻辑”学派的大量的夸张性的文献中},最有代表性的例
子是\textbf{戴维·耶夫}1973年发表的被广泛阅读的一篇文章,该文完全忠实于格罗斯曼的
理论精神(除了没有忠实于他得出的经济崩溃结论的数量模型)。\textbf{耶夫开篇就警告人
  们要提防对马克思的人道主义解释,像法兰克福学派的解释一样,它们存在于经济以外的
  资本主义矛盾中,}存在于意识形态、技术和政治领域,\textbf{而且要提防很多自封为马
  克思主义经济学家的凯恩斯主义者(参见以上第4章和第5章)}:“如果资本主义生产方式在
具备或不具备政府干预的条件下,都能保证经济的持续发展和充分就业,那么,为社会主义
革命理论辩护的最重要的带有目的性的观点就会不攻自破。”在《资本论》第一卷中,马克
思集中研究“\textbf{资本一般}”或“\textbf{资本的内在属性}”,着手于对资本主义的
分析,\textbf{而“资本一般”是从“许多资本”的竞争行为的作用中抽象出来的。}从这点
看,他已经证明,有机构成提高“\textbf{不纯粹是一个主张,而是从逻辑上得自资本概念
  本身}”,\textbf{因为机械化和由它造成的死劳动代替活劳动是必需的,只有这样才能保
  证资本对生产过程的统治。}

耶夫在文中继续谈到,\textbf{由于存在反作用趋势,利润率下降“不是呈线性的},但是在
某些时期它只以潜伏的方式发作,而在其他时期则表现得或强或弱,并\textbf{以一个危机
  周期的形式表现出来”}。一旦发生“\textbf{绝对过度积累}”,\textbf{再增加积累并
  不会增加所生产的剩余价值总量},经济增长就会停止。在耶夫看来,这就是马克思危机理
论的全部。从逻辑上说,它与竞争和有效需求是没有关系的,因为:
\begin{quotation}
  \textbf{在不考虑竞争的条件下,我们已指出了资本主义本身所具有的生产过剩和危机趋
    势。}到此为止该论述\textbf{一直假设所有的商品实际上是以它们的价值出售的,不存
    在价值实现上的困难};这样得出的是\textbf{资本主义危机趋势},\textbf{而资本的
    生产过剩在没有如此条件的情况下也能推导出。}
\end{quotation}

\textbf{竞争在解释如何克服危机时的确是有用的};\textbf{资本“重构”}——一个令人想
起20世纪70年代的词——\textbf{保证只有在效率更高、获利最大的条件下,资本才能存活下
  去}:“从这个意义上说,\textbf{资本主义危机}可以被看做是与\textbf{利润率下降的
  长期趋势}相对立的\textbf{最强有力的反作用趋势}……由于受相反趋势的影
响,\textbf{‘崩溃’停滞}的趋势因此采取了\textbf{循环}的形式,\textbf{而现实中的
  危机不过是这一相反趋势的极端情况。}”

在文章的结尾,耶夫再次\textbf{指责马克思危机理论中他认为不正确的大量观点},这些错
误观点“\textbf{把流通过程从资本主义生产过程这个整体中分离出来}”。其中\textbf{最
  严重的错误是消费不足观点,它误把结果当做原因}:“资本过度积累是\textbf{商品生产
  过剩}的原因,而后者\textbf{并不是资本主义生产过程的限制因素。}”由此可见,政府
活动不可能阻止利润率下降趋势,因为\textbf{它本身是非生产性的}。向政府出售商品(如,
武器)的资本家所获得的利润是以\textbf{牺牲其他资本家的利益}为代价的,\textbf{因为
  这只是“已经生产出来的剩余价值的再分配”}。因此,“混合经济并没有从根本上改变传
统资本主义制度的矛盾”,它仍具有危机趋势。

这是对利润率下降理论的重新表述,\textbf{并不是一种合理的辩护}。同曼德尔一
样,\textbf{耶夫只是提供了对冲盐定理的一种批评意见。}他确实把他恰当地称作的“积累
过程的内在趋势”,看成是剥削率上升的因素。但是,\textbf{随着技术进步的继续,减少
  必要劳动时间变得越来越困难(这是由马克思首先声明的)},正是根据这一点,\textbf{耶
  夫又否定了他的上述观点。}米克勉强承认这个观点,但他的统计资料还是表明利润率最初
上升,而且可以想像它能持续相当长的时间。耶夫完全忽视了这一可能性。\textbf{他也没
  有努力去说明长期繁荣的原因,}或者去解释几乎长达30年没有发生由积累过度造成的危机
的原因,这些危机常被认为是资本本质的组成部分。就此而言,他的分析比曼德尔的要缺乏
一定的说服力。

\section{结论}
到20世纪70年代中期,利润率下降理论的辩护者们的议程开始分为两部分。在理论上,他们
将不得不同\textbf{冲盐定理}进行斗争,还击\textbf{迪金森—米克}的非难,而且以一种不
同于同义反复的方式来系统表述他们的观点,这种\textbf{同义反复是“或者利润率下降,
  或者如果反作用趋势足够强大,利润率不下降。”}在经验上,他们必须更加小心谨慎地拿
出证据,根据劳动价值标准来计算有机构成和剥削率,并利用这些资料说明\textbf{利润率}的
运动,这是一个\textbf{包括价格在内的比率}。最后,他们将必须完全清楚创造剩余价值的
生产性劳动和吸收剩余价值的非生产性劳动之间的界线。在第16章我们将会看到他们是多么
成功。


\chapter{持久的军事经济}
\section{引言}
\textbf{战前与战后世界最显著的(而且当然是最不祥的)不同},\textbf{可能就是战胜国军
  事开支水平的变化。}第一次世界大战以后,裁军并没有完成,但是像现在这样规模的军事
开支在资本主义和平时期是绝无仅有的。1950年,英国的军费开支占国民生产总值的6.6\%,
法国占5.5\%,美国占5.1\%;10年以后,这三个国家的军事开支,已分别占国民生产总值
的6.5\%,6.5\%和9. 0\%。对这种\textbf{持久军国主义}的一种解释,是自由主义经济学
家\textbf{詹姆斯·托宾后来所描述的军费开支的“天真理论”,}该理论把军费开支简单地
看作是“\textbf{对世界事件的反应}”。然而,这并不像听起来那么天真。毫无疑
问,1945年以后有严重的政治障碍阻碍裁军。法国和(不太绝望的)英国卷入了殖民战争,美
国正在保卫它刚刚赢得的“不成熟的帝国”,而且上述三个国家都在从事与苏联和中国的长
期冷战对抗。

可是,马克思主义者不可避免地发现了“天真理论”的不足,因为它忽视了这一问题的几个
重要方面,而且\textbf{特别是它没有提到高额军费开支可能给世界资本主义制度及其各个
  民族国家带来的实际利益。}首先是\textbf{意识形态上的利益。}军国主义可以促
进\textbf{民族团结},反对可能的\textbf{外部威胁},并有助于\textbf{缓和内部阶级对
  抗}。我们将在本章的最后再讨论这个问题。其次是\textbf{两类间接的经济上的利
  益。}最重要的经济利益是,\textbf{运用帝国主义势力通过在殖民地和新殖民地投资与贸
  易,获取超额利润。}在这种场合,军费开支对于\textbf{帝国主义}来说是一
个\textbf{支柱},而不是一个单就自身实力而言的重要的现象。(我们把军国主义经济学从
帝国主义政治经济学中分离出来,从某种意义上说是人为的,不熟悉后面问题的读者,应参
考以下第9章、第10章和第11章。)另一个很重要的经济利益,是\textbf{从军事研究和开发
  获得了民用的“副产品”},“从圆珠笔经过计算机到核能”。稍后我们就将进一步介绍这
样的副产品。

然而,我们这里主要关心的是\textbf{军国主义可能产生的直接的经济利益。}得到军事合同
的公司是有利可图的(事实上,是非常有利可图的)。但是,军费开支如何才能提高整个制度
的利润率呢?像大多数自由主义者一样,很多马克思主义者否认这种可能性。一些人认为,
军费开支降低了利润率,并因此对资本形成一个净负担。然而,其他马克思主义著述者则认
为,军费开支由于下述两个原因之一而是直接有益的。\textbf{首先是刺激总需求,因而相
  应地减弱了否则会阻碍剩余价值实现的消费不足趋势。}这种观点是(并不局限于)巴兰和斯
威齐以及他们的追随者们分析的中心问题(参见以上第6章)。\textbf{其次是有关剩余价值生
  产而非剩余价值实现。军费开支被认为是抵制利润率长期下降趋势的一个最重要的力
  量。}20世纪50年代和60年代初,英国新托洛茨基分子迈克尔·基德隆对这一观点做出了最
有说服力的表述,\textbf{直到现在这一观点仍是英国社会主义工人党的官方立场。}

本章的其他章节结构如下。下一节,我们要对古典马克思主义有关军费开支的稍嫌简略而未
有定论的观点作出概要总结。接着,我们要考察消费不足论者的观点,并特别注意用经验的
方式检验它的可能性。第4节将分析军费开支与利润率之间的关系。然后,解决有关军国主义
的经济成本、“军工联合企业”的作用,以及认为军国主义是一种经济寄生形式等问题。最
后,我们将讨论“持久的军事经济”论题的方法论意义。

\section{1939年之前的马克思主义和军费开支}
马克思没有任何关于军费开支经济学的著述。尽管恩格斯在他生命的最后10年,对于战争的
技术方面和欧洲军国主义的实质性增长具有浓厚的个人兴趣,但他做得比马克思只略多一点。
恩格斯写于1893年的小册子《\textbf{欧洲能否裁军?}》,号召通过谈判和用民兵替换常备
军的方式来解决国际争端。\textbf{在恩格斯看来,军费开支给资本主义国家带来的既不是
  直接的利益,也不是间接的利益,而只是财政毁灭的前景。}德国和俄国的马克思主义者倾
向于忽视这整个问题。像恩格斯一样,卡尔·考茨基的确指出了殖民扩张以及与此相关的军事
支出所花费的成本,并警告说德国在开销巨大的陆军基础上若还想建设一支强大的海军,则
将导致经济上的灾难。当世界战争隐约出现时,\textbf{考茨基}认为,\textbf{虽然军国主
  义确实能促进消费需求,但它(像其他非生产性支出一样)是前后矛盾的。}资本家确能从军
费开支中获利,但从\textbf{军费开支来源于向利润征税}这个角度讲,资本家必然会抵制它。
结果,军费支出增长肯定会受到限制,这样就不可能使这一制度克服它的\textbf{长期的消
  费不足趋势}。考茨基认为,当他的社会民主党左翼批评家们把军国主义看作资本主义存在
的一个必要条件时,他们是弄错了。虽然军事支出有经济上的原因,但它不是不可缺少的,
而且裁军具有真实的可能性,尽管这种可能性是靠不住的。

在所有古典马克思主义者中,只有罗莎·卢森堡对军事开支的经济含意表现出最浓厚的兴趣,
但是她在《资本积累论》的简短章节中对此所作的分析显然是很难懂的。卢森堡这本书的主
题是在剩余价值得以实现的前提下对国外(即非资本主义)市场的需求(参见该书第一卷第6章)。
她似乎把军事开支看成是对在海外殖民地和新殖民地开拓的更重要的外国市场的补充,但是
她关于包含军事部门的扩大再生产的数学例子,只会使她的观点更加难懂。\textbf{事实上,
  关键在于军事支出的扩张如何获得资金上的支持。}这里有三种可能性。\textbf{向工人增
  加税收}只会改变产出的构成(\textbf{更多的枪炮,更少的黄油}),\textbf{却不能影响
  总利润或有效需求水平;对利润征收较高的税能否增加资本的全部盈利,取决于它们先前
  在不征税的情况下的用途;只有在需求不足以实现在一切生产能力都得到充分利用时生产
  的全部剩余价值时,赤字财政才对需求起刺激作用。}从消费不足论者的观点看,军事支出
有明显的好处,它“\textbf{增加生产能力但并没有引起更严重的问题(更不用说资本主义国
  家在以武器彼此敌对后,由重复建设产生的巨大的新的投资机会)}。”

但是,这只表明可以用凯恩斯的术语来理解卢森堡,并不意味着她本人是早熟的凯恩斯主义
者。这可能给了她过多的同时也是过少的赞誉。说过多,是因为\textbf{《资本积累论》的
  确没有对有效需求问题做出必要的清晰的分析};说过少,是因为这样就忽视了她的广泛的
贡献。\textbf{卢森堡强调},资本主义从军国主义得到了\textbf{社会的和意识形态上的好
  处},包括在国内的(\textbf{缓和阶级矛盾,增加必要的强制力量})和在国外
的(\textbf{强迫自然经济转变为商品经济,把资本和雇佣劳动迅速引进到落后地区})好处。

\textbf{布尔什维克的经济理论观点,是不会支持卢森堡关于军事开支得到经济好处的看法
  的。}虽然列宁和托洛斯基都没有特别研究过军费开支本身的影响问题,这与他们对帝国主
义的分析明显不同,布哈林也是这样做的。按照马克思的再生产图式,布哈林在《过渡时期
经济学》中认为,军事生产卷入剩余价值并妨碍了扩大再生产。\textbf{的确,如果军事产
  品价值超过了总剩余价值,该制度就将进入“负的扩大再生产”。}布哈林相
信,\textbf{这个过程可能被货币价格的通货膨胀所掩盖,但这样的增长是虚幻的。布哈林
  用含蓄的而不是明显反驳的方式指出}:卢森堡对军事经济的研究与实际情况正相反对;军
事经济的主要影响是毁灭价值生产,而不是有助于剩余价值的实现。以下我们将对此进行讨
论。尽管布哈林的观点更具有辩护力,但事实证明到目前为止,卢森堡的观点在马克思主义
政治经济学中影响最大。

在已知所有主要资本主义国家大幅度裁军的条件下,就很难认为军事开支是造成20世纪20年
代相对经济稳定的原因。\textbf{只是在1933年以后,随着希特勒统治下的德国军国主义(和
  德国经济)的惊人复活,人们才严肃地把注意力投向卢森堡的结论。}尤金·瓦尔加在写
于1937年的文章中,否认军事开支能够保证繁荣。然而,\textbf{瓦尔加观点的要旨明显的
  是凯恩斯主义的,并具有浓厚的卢森堡的色彩。他认为,如果军事支出靠向工人阶级征税
  来获得资金支持,那么总需求将是不变的,发生变化的只是总需求的构成。另一方面,只
  要“利用的是以前的闲置资本”,贷款支出会扩大需求。也就是说,在一定程度上过剩生
  产能力已被消灭。}两年后,瓦尔加在一个附录中态度更为明确。他写道,\textbf{军事支
  出的增长已经“导致德国消灭了失业”,“军备为资本主义提供了一个巨大的而且几乎无
  限的市场”,它不会以牺牲民用项目为代价的,除非充分就业已经实现后军事支出还继续
  增加。}瓦尔加是以他背后的第三国际的充分权威来写作的,这就清楚地说
明,\textbf{到1939年,消费不足论、军费支出“吸收剩余”的观点可以看作是斯大林主义
  的正统理论,}以至于斯大林能够免于纳塔莉·莫斯科斯卡对他的指责:他和列宁都未曾努
力研究“战时资本主义”这一新现象。

然而,可以从一个完全不同的角度来看待军国主义的复活。如我们在第1章中所看到
的,\textbf{弗里德里克·波洛克(在1941年)认为,}在西方,新的国家资本主义不能忍受大
量失业,害怕群众反抗。\textbf{但提高人民生活水平也是危险的,因为这意味着更多的闲
  暇,更多的反省时间,以及自觉的和革命的反抗这一更大的危险。}波洛克的措辞很像战后
乔治·奥韦尔的小说《\textbf{1984}年》中的用语。波洛克总结说,资本主义不可能经受
住“和平经济”:“只要一国的国家资本主义还没有征服整个地球,\textbf{然而总是有大
  量的机会为永久增长的和在技术上更完善的军事花费掉大部分过剩的生产能力(超过最低生
  活标准需求的余额)。”}

\section{军事支出和有效需求}
在《资本主义发展理论》中,保罗·斯威齐把军国主义描写成“\textbf{抵消消费不足趋
  势}的一个日益重要的力量”,它“为整个资产阶级提供了\textbf{不断增加的资本投资获
  利机会}”。但是斯威齐对它的经济影响的分析仅占一页的篇幅;只是到了战后,他和保
罗·巴兰才把军事支出置于他们的垄断资本理论的中心(参见以上第6章)。斯威齐在1953年认
为,\textbf{军事经济的一个不利之处就是高税收。}

其他方面都是有利的。\textbf{正是最大的垄断者,最直接地从军事支出中受益};与私人企
业没有任何直接或间接的竞争;\textbf{和军备相伴的仇恨与偏狭的气氛}——以莫须有罪名
进行的政治迫害、侵略主义、\textbf{赞扬暴力}——营造了一种气氛,有产阶级发现在这种
气氛下最易于控制工人、农民和下层中产阶级的思想与活动。

自由的和平运动者所提出的可选方案没有一个能够成功。资本家们将抵制政府民用支出的扩
张,而且任何以牺牲利润为代价,通过提高工资来增加消费的企图,都将受到强烈的抵
制。\textbf{事实证明,对军费支出和税收同时并均衡地加以减少并不奏效}:“利润如此之
大,以至于使消费增长局限在狭窄范围内,\textbf{如果这些利润被大量地投资于扩大民用
  生产能力,那么结果很快将是一场超额生产能力和生产过剩的危机。}”

可能是由于朝鲜战争结束时美国军事支出相对下降,保罗·巴兰在这4年后出版的《增长的政
治经济学》,很少强调军事支出的稳定作用。然而,\textbf{军国主义在}《\textbf{垄断资
  本}》中的作用却重大得多,此书\textbf{把它与“促销”一起作为两个主要的剩余吸收
  者。}到1970年,“军事工业联合企业已经消灭了长期停滞的幽灵”,已变成了美国马克思
主义者的一个信条。巴兰和斯威齐更加慎重,\textbf{认为能够创造需求的军事支出受到军
  事的和经济的限制。武器生产的高度资本密集型特点,使得它严重地影响了就业,而且核
  军备竞赛的致命的非理性在军事组织内越来越得到了公认。}巴兰和斯威齐认为,这些限
制“\textbf{预示着永久繁荣能够通过无限扩大军事预算获得保证这一幻想的破灭}”,而且
预示着停滞和萧条将卷土重来。\textbf{严酷的事实是,经济上和军事上的压力都没有像巴
  兰和斯威齐所预料的那样限制美国的军事支出。}军事预算(按当前价格)从1961年的470亿
美元增长到1984年的2650亿美元,1989年达到2990亿美元;就是在扣除了由于通货膨胀所造
成的物价上涨之后,在越南和里根时代军事支出的大量增加已是很明显的了。

从理论上说,\textbf{消费不足论者的观点提出了两个重要问题。}第一,\textbf{军事支出
  是否作为一种经济政策手段而被有意识地执行。}第二,\textbf{军事支出实际上是否已经
  起到了巴兰和斯威齐以及他们的追随者所主张的刺激作用。}就第一个问题来说,要把军事
支出有意识地作为一个反周期或反萧条的方法\textbf{持续地使用下去是十分困难的}。要这
样做的话,下面的条件是必要的:首先,美国统治阶级意识到了需要这样的手段,并同心协
力决心使用它;第二,从宏观经济考虑,军事支出的变化在时间和数量上受到他们的支配。
这两个主张似乎都不只是有一点儿的强词夺理(参见以下一些很不确定的证据)。上述两个
观点,巴兰和斯威齐都不赞成,他们使用更规范的术语解释战后军事支出的增
长:\textbf{它是美国在政治和经济上支配一个日益敌对世界的不可避免的先决条件。}这不
是托宾有关军国主义的“天真理论”,因为\textbf{它依据的是从帝国主义的霸权主义中增
  长起来的间接经济利益。}然而,任何得自军事支出的直接经济利益,都仅仅是美国超级帝
国主义地位的副产品。

这些利益是实在的还是表象的呢?\textbf{对该问题的研究常常被概念上和技术上的问题所
  干扰。}首先,\textbf{硬把辩证的马克思主义理论纳入适于正规统计检验的均衡模型,很
  难说就是合情合理的。}这样做不可避免地会使马克思主义理论的研究范围\textbf{狭窄
  化},因为众所周知,经过人为加工的大量历史事件,比精确且少量的经济学假
设\textbf{更难于进行经济计量方面的检验。}其次,发表的数据总是\textbf{以市场价格而
  不是以生产价格来计算的,更不用说以劳动价值来计算了,}从正统的马克思主义理论观点
来看这些检验,其确切性就值得怀疑了。第三,关于分析的\textbf{层次},是分析单个的国
家还是整个资本主义世界?第四,关于一大堆经济计量上的问题,特别是\textbf{同时性的
  问题}。仅仅在\textbf{军事支出和生产能力的利用度之间}建立统计关系,丝毫不能说明
两者之间\textbf{何为因、何为果},可能前者是因,后者是果(“装备新的武器促进需求”),
或者后者是因,前者是果(“经济增长为更多的军事支出创造了条件”);当然,\textbf{也
  可能它们之间没有任何因果关系。}

\textbf{从马克思主义观点出发,对军事支出进行严格的经济计量学研究的只有两个人。首
  先是阿尔·西曼斯基,}他研究了从1950年至1968年间18个最富的资本主义国家的军事支出、
失业和经济增长之间的关系。他最初的观察结果是:\textbf{军事支出很小――除了英国、
  以色列和美国外,}其他国家都低于国民收入的4\%,以致对军事支出的经济影响产生了严
重的怀疑。(\textbf{然而它更多依赖于国际乘数效应的力量,}有关内容将在下面讨论。)紧
接着,西曼斯基发现,\textbf{除了美国之外,以人均收入水平测定的发展阶段与军事支出
  份额之间没有任何相互关系。}如果比较富裕的国家的确生产了相对更多的剩余,并遭受了
更严重的消费不足,这并没有从它们的军事支出中反映出来。第三,与\textbf{巴兰和斯威
  齐}的预料相反,\textbf{西曼斯基发现军事支出和经济增长率之间具有负相关关系。}只
通过一项验证该理论就成立了:西曼斯基真的发现了\textbf{军事支出较高而失业率又较低
  的国家。}总的来说,\textbf{剩余吸收方法还是和所观察的资本主义现实不一致。}

另一个是\textbf{罗恩·史密斯},他对消费不足论产生了进一步的\textbf{怀疑}。他认
为,\textbf{由于军事产业投产准备阶段长,加之大多数军备产品的高度资本密集型特点,
  军事支出不适合作为一项反周期政策。他没有发现这种迹象,即美国增加军事支出的时间
  选择是受失业趋势影响的。}史密斯还发现,1970年以前,发达资本主义国家军事支出的变
化同它们相对的失业率没有关系。和西曼斯基一样,他提供了\textbf{军事支出与经济增长
  之间负的横截面关系。}史密斯也认为,消费不足论者对军国主义的解释缺乏说服力。

可能有人要提出\textbf{反对意见,认为横截面关系抓不住检验的要点。如果能够证明美国
  的军事支出已刺激了日本和西欧的需求,那么,较低的军事支出与消费不足论者的观点将
  没有关系。}然而\textbf{有充分的理由说明事实并非如此。}首先,仍处在猜测之中的是:
没有任何确凿的证据证明,其他资本主义国家的失业率或生产能力利用度指数,同美国军事
支出的变化呈线性移动。其次,对于这样的国际需求外溢来说,至少在20世纪50和60年代,
美国进口的边际倾向非常低,无法起到扩张经济的主要发动机作用。第三,具有决定意义的
是,\textbf{在上述20年中,美国军事支出的资金来源主要靠增加税收,而不是靠赤字财政。
  因此,其扩张作用是相对小的。假定军事支出中没有一点用于购买进口商品,那么军事支
  出本身就会提高美国的国民收入(根据凯恩斯经济学的平衡预算乘数原理),但对国际经济
  不会有任何影响。}如果其中一部分军事支出用于购买外国商品,将会产生\textbf{国际乘
  数效应}。然而在这种情况下,该效应\textbf{会被美国国内经济中需求扩张的缩减而抵
  消。}

这种观点必须\textbf{加以限定}。\textbf{美国私人投资很可能也存在加速器效应,}可以
想象这可能已溢出到其他西方国家(比如美国工程技术公司向日本订购机床)。\textbf{这样
  会依次刺激全世界的商业信心(并因此刺激投资),虽然很难对之进行经验上的检验。}最后,
关键是要注意到西曼斯基和史密斯\textbf{都没有研究1970年以后的情况,此间美国经济对
  于进口渗入开放得多了,在这段时间里(特别是从1981年以来),庞大的预算赤字一直是军
  事扩张的伴生物。}因此,他们的结论只适合于早期阶段。然而,这些限制条件没有一项能
够弱化这两个本质上否定的结论。\textbf{军事支出的需求创造效可能没有以巴兰和斯威齐
  所说的方式引起1945年以后的“长期繁荣”;20世纪70和80年代下降的增长率及提高的
  失业率也不是“改剑铸犁”的意外结果。}

\section{武器和利润率下降}
比巴兰和斯威齐更进一步,\textbf{迈克尔·基德隆把军事支出看做是1945年后资本主义成功
  的关键(参见以上第3章第4节)。}在基德隆看来,\textbf{它是一个高就业、快速增长和经
  济稳定在因果关系上起作用的时期,正如20世纪30年代停滞和不稳定相互作用一样,这些
  因素也在以同样的方式相互作用。}基德隆否定了解释这一变化的几个流行的理
由。\textbf{政府计划以高水平的经济活动为前提条件,}如果不具有这个条件,那么,政府
计划将是不必要的了。至于\textbf{战后贸易自由化和技术革新}的快速发展,基德隆认
为,\textbf{两者都是在已经提到的因果关系内部起作用。要解释长期繁荣,需要一个自发
  的外在因素。}

对于基德隆来说,\textbf{这个外在因素就是军事经济。}就向利润征税为军事经济提供资金
来源来说,\textbf{军事支出剥夺了资本主义本该用于生产性投资的资源。它们被用于武器
  生产,引起民用经济中资本有机构成增长速度放慢,因此,明显阻碍了本该出现的利润率
  下降。}基德隆的分析\textbf{依据的是一个极为重要的命题}:\textbf{军事生产中有机
  构成的提高,对更广义的经济领域的利润率没有任何影响。}武器装备不是供工人消费的,
它们也不能直接或间接地作为工资品产业的生产资料来使用。从这点看,它们倒像由资本家
消费的奢侈品。它们既不属于第\Rnum{1} 部类,也不属于第\Rnum{2} 部类,\textbf{而应属于第三部门,
  即马克思所说的\Rnum{2} b(奢侈消费资料生产)和博特凯维兹(在这点上,后来的多数作者都追随
  他)所划分的第\Rnum{3} 部类。}大卫·\textbf{李嘉图早就认为:这种产业生产条件的改变,只会
  影响相关商品的相对价格,不会影响一般利润率。}19世纪和20世纪之交,博特凯维兹通过
一个简单的三部门简单再生产模型\textbf{证明上述观点是正确的}。1960年,皮罗·斯拉法
用更加概括的方法,进一步证实了这一观点。基德隆由此得出结论:\textbf{永久的军事经
  济是利润率下降的最新的、很可能也是最有力的反作用趋势。}它构成了资本主义发展新阶
段的基础,取代了列宁的帝国主义,事实证明帝国主义只是“次高级阶段”。

马克思本人对于李嘉图有关奢侈品生产的分析的合理性\textbf{游移不定},偶尔也对之作出
断然否定。这样就使得那些不愿意批评自己老师的马克思主义者们,如欧内斯特·曼德尔,拒
绝基德隆的观点,认为这一观点是错误的。\textbf{曼德尔确实指出,军事生产中相对较高
  的有机构成是一种迹象,它表明军事经济已经加速了利润率下降,并助长了不稳定因素。
  然而,这是完全错误的。}博特凯维兹的简单模型证明,一般利润率与第\Rnum{3} 部类的生产条件
是没有关系的,而且这一结果的潜在逻辑是容易理解的。\textbf{第\Rnum{1} 部类和第\Rnum{2} 部类的生
  产不需要来自第\Rnum{3} 部类的投入,从这个意义上说,第\Rnum{1} 部类和第\Rnum{2} 部类的生产是自给自足
  的。}这两大部类构成一个\textbf{独立的子系统},而且它们内部的生产条件单独决定它
们本身的利润率。如果整个社会经济的利润率是均衡的,那么,第\Rnum{3} 部类的利润率必须因此
调整到由第\Rnum{1} 部类和第\Rnum{2} 部类构成的子系统的一般利润率。\textbf{第\Rnum{3} 部类固定资本的过
  度集中,可能使利润率保持在过低的水平,以至于阻碍了利润率均衡地实现,但是,它不
  会影响第\Rnum{1} 部类和第\Rnum{2} 部类的利润率。}

最近,\textbf{相同的观点已经在n个产业的条件下得以建立。皮尔·斯拉法证明,只有“基
  础”产业,即生产工资品和生产生产资料的产业,对利润率有影响。而武器生产在技术意
  义上显然是“非基础”生产活动。}

基德隆研究存在的真正问题是很不一致的。从理论上说,他的分析与马克思的利润率下降理
论成败与共,我们在第7章看到,他的分析是有严重缺陷的。因此,\textbf{基德隆关于如果
  没有持久的军事经济,日益下降的利润率本该使战后资本主义趋向危机的说法,没有任何
  理论意义。}他也没有像人们所希望的那样,使自己的观点得到任何经验上的证明。例
如,\textbf{军事支出在英国产出中所占的比例,}从1952年的12\%\textbf{锐减}到1970年
的5.5\%,\textbf{但并没有引起利润率相应的下降}。我们将在下一章看到,\textbf{利润
  率发生下降完全可以用其他因素来加以说明。}这并不是否定军事支出可能影响利润率。例
如,\textbf{技术上的副产品}可以使民用工业的固定资本要素变得廉价,同时,\textbf{军
  国主义的思想影响}可能会诱使工人接受更低的工资和更高的剥削率。同样,下一节也会涉
及这个内容。有些军事支出方法的使用可能会减少利润。但是没有一个能为基德隆的论题提
供证据,因为\textbf{军事支出影响经济的方式是很不同的。}

\section{军事支出的成本}
西曼斯基曾得出这样的结论:“从根本上说,军事支出\textbf{不是导致经济增长的原因,
  而实际上表现出至少是相对停滞的一个原因。}”事实上,\textbf{起码有三种军事支出方
  式能够阻碍增长:军国主义可能“挤掉”生产性工业的投资;可能使科学家、技术员和辅
  助资源从非军事科研与开发项目中转移出去;而且在军事部门内部和更广泛的经济领域培
  养了骄傲自满和无效率。}尽管从军事支出中获得了特殊的既得利润,但\textbf{如果这些
  因素起作用,它是以牺牲总资本为代价的。}

\textbf{随机观察支持了军事支出会减少生产性投资这一观点。}极端的例子就是,战后英国
和美国军事负债沉重,它们比日本的投资率和增长率低得多不可能是巧合;直到最近,日本
宪法规定的军事支出占国民生产总值1\%的限额仍被看作是有约束力的。\textbf{从时间序列
  和横截面的资料看,史密斯的经济计量分析,揭示了军事支出所占的国民收入份额与民用
  投资之间的一种强的负相关关系。}这大概反映了军事生产集中在资本品工业(工程技术、
造船和电子),相伴随的是有效需求的严重不足和对由于扩大军事支出造成的私人和公共消费
下降的抵抗。

\textbf{军国主义不仅影响投资的数量,也影响投资的质量,而投资质量主要依赖于技术革
  新。}再一次引用极端的例子,1945年以来整个英国研究与开发资源的一半以上被用于军事
事业。除非有大量的和不断的技术上有用的副产品,否则这一定是以牺牲民用工业的进步为
代价的。然而,\textbf{几乎所有的著述者都承认,军事创新日益变得专业化和秘密化,给
  非军事资本主义提供的利益越来越少。而且和直接用于民用的研究费用相比,军事上有用
的副产品对经济增长几乎没有太大的影响。}

\textbf{最后,可能最重要的是,军事支出对民用经济的效率、动力和灵活性产生了无形的
  有害影响。}正常的竞争压力一点也不能、或者说仅仅以一种微弱和歪曲的形式作用于军事
生产,因为这类产品的唯一大主顾是政府。\textbf{利润率高得超乎寻常;衡量管理成功的
  标准是成本最大化(因此也是津贴最大化),而不是成本最小化。}当军事需求支撑本该滑坡
的产业,而且像电子等新兴部门都服从军事目的时,\textbf{结构刚性}就产生
了。\textbf{松散的和浪费的行为在承包公司的各层次中都会得到奖赏,}当它们不能在民用
市场生存时,就更加依赖军事生产。在更高度军事化的国家中,国内竞争的整个基础会被严
重削弱。

那么,为什么身处这种经济中的资本家会支持军事经济呢?\textbf{简单的答案可能是,他
  们并不支持军事经济。}克拉伦斯·劳对美国人在1948和1953年间(麦卡锡主义时代)态度的
仔细研究发现,由于税负、通货膨胀和政府控制,\textbf{工商界对于增加军事支出的反对
  是很强烈的。}装备新式武器的计划总是排在前面。当然,强烈支持一个居高不下并还在增
长的军事预算的,是那些从中直接受益的公司。对于军国主义的自由批评家们,从科布登和
霍布森到梅尔曼,曾抨击既得利益构成了“\textbf{军工联合企业}”。对这一点,退休总统
艾森豪威尔(他本人从前是位将军)在1960年就曾警告过美国人民。第一次世界大战这一血战
刚刚结束,A ·C·庇古就写文章斥责军火生产商提供了“爆炸物,并由此点燃了战争的火焰”;
而且他是从比喻,也是从字面的意义上说这些话的。\textbf{很多马克思主义者也把军国主
  义看成是一种经济寄生现象,这里少数人的资本利润是由社会中的其他人(包括大多数资本
  家)来付款的。}用詹姆斯·奥康纳的话来说,国防物资的承包人好像“已经在联邦预算上建
了一个永久的流出口,并因此在军备竞赛中下长期的赌注……因此,大军火商都毫不犹豫地
参与国防项目,而不考虑从全国的资本利润来看,这些工程是否具有合理性。”\textbf{军
  国主义强有力地证明现代资本主义国家经济管理上的失败。}

\section{资本主义与和平}
回到本章第1节曾使用过的概念,\textbf{军事支出的直接经济效应可能是负的}(或者说直
到20世纪70年代末其效应是负的:对美国里根总统执政时期的“军事重新工业化”还有待于
进行充分的评价)。接受这个结论,也就意味着\textbf{是供给约束而不是需求不足严重地制
  约了战后资本主义生产的增长},或者由裁军造成的有效需求缺口至少应该能够很快地由增
加民用(包括政府)支出加以弥补。\textbf{这种对凯恩斯观点的驳斥有时也被称为“供给派
  马克思主义”(参见以下第16章),}这一称谓并非完全可笑。对于美国来说,\textbf{军事
  支出的最初的间接经济效应,由于超过了它的直接成本从而可能是正值,因为它的军事实
  力的确有助于美国在1945年以后统治世界经济达四分之一世纪之久。}然而,\textbf{这又
  是一对矛盾,因为接下来的平衡预算赤字以及生产投资的挤出效应,都将极大地促进由国
  防军事支出开始的经济霸权的毁灭。}

对于像英国和法国这样在军事支出上具有实力的其他西方国家来说,\textbf{平衡表绝对是
  负的。因此,持久的军事经济不是长期经济繁荣的原因,军事支出的削减也不意味着繁荣
  的消失。}根据日益增强的社会凝聚力和资本主义阶级关系的日益合法化,可以看出,军国
主义在意识形态上仍有影响。这些效应不能量化,而且在某些时刻可能是负的(特别是在越南
战争期间)。假定它们总体上是正的,就会提出一个重要的方法论上的问题。从军事支出使资
本主义运行更顺畅这一事实出发,并以此认为这就是1945年以后军事支出如此高涨的原因。
这样的推理合适吗?作这种机制上的解释,马克思主义的分析可以接受吗?或者说因果陈述
必须按照个人的理性选择来构建吗?在这些问题上存在着大量的分歧(如在国家理论和劳动过
程理论上的分歧),有关内容将在以下第17章谈到。

另一个与此联系紧密的问题对马克思主义理论中一个比较传统的话题有影响。这就
是\textbf{高额军费支出在多大程度上是由经济因素造成的,而不是由“相对自治的”政治
  的、战略的和其他上层建筑的决定因素造成的。}\textbf{玛丽·卡尔多}认为存在一
种\textbf{武器崇拜}。在这种武器崇拜中,武器系统作为彼此独立的金属器具的一小部分,
表现为受它自身的消费和生产模式所支配,并把战争模式的军事和工业成分结合在一起。可
能是由于我们对炸弹的畏惧,使我们成了\textbf{武器崇拜的牺牲品,使我们不能辨别它在生产它的
社会系统中的意义,并因此使我们在现代军国主义急速发展的势头面前显得无能为力。}

没有必要走得像\textbf{E·P·汤普森那样远},在他看来,\textbf{自动推进的“灭绝主
  义”已取代处于世界历史中心的资本主义和社会主义,以实现军事竞争确实支配着的致命
  的运动。这引导我们到终点,大概也是最重要的一点。}现代军国主义\textbf{毕竟是资本
  主义}的产物,不论其生产过程有多么复杂和矛盾。那么,它在资本主义范围内能被克服吗?
这是第一次世界大战前夕使卡尔·考茨基感到烦恼的问题。他小心谨慎以免把军国主义是资本
主义的创造物(他所接受的)与军国主义是资本主义存在的必要条件(他所否定的)这两种说
法混淆起来。因此,考茨基认为,在资本的逻辑中,没有任何力量能够阻止裁军与和平的出
现。鉴于资本主义生产方式具有很强的适应力,我们只能希望考茨基是对的。






%%%LocalVariables:
%%%mode:latex
%%%TeX-master:"../../main"
%%%End:
