\chapter{马克思主义政治经济学和剩余经济学}  

\section{斯拉法马克思主义}

从本质上讲,面对挽救马克思价值论的各种尝试,\textbf{斯拉法经济学家}仍旧保持着镇静,
而且依然继续强调本书以上第13章概括的批判是有效的,即\textbf{重新设计的劳动价值范
  畴,从最好的方面讲是多余的,从最坏的方面讲是引起争议的或错误的。}同时,斯拉法主
义者声称,他们对马克思的分析是建设性的,他们认为斯拉法和马克思同属于一
个“\textbf{剩余范式}”,具有相同的视角和方法论。因此,他们认为他们对马克思的批评
是属于“内部的”批评,马克思主义政治经济学实际上得到了加强,因为其最初形式中的缺
陷被暴露出来而且被证明与支持它的更一般的方法是不相关的。因此,斯拉法的著作提供了
一个牢固的基础,使剩余范式得以发展并使马克思主义的真正洞察力重新得以建立。

这些著述者对剩余所下的定义是不相同的,但\textbf{基本的观点}总是相同的:\textbf{剩
  余形成可处置的资源。任一经济体系的净产出被分成两部分:产品再生产所需要的部分,}它
代表生产或再生产的必要成本,其中\textbf{包括必要的劳动成本};\textbf{净产出的余留
  部分代表剩余。}不管剩余如何使用,这一经济的产出都可以得到保证,虽然剩余的使用会
影响到该体系的动态性(参见以上第6章和第9章)。\textbf{剩余收入没有对应的等价物的交
  换,不代表生产成本。因此,财产收入——通常被认为是剩余在阶级社会里分配的主要形
  式——很容易被认为是剥削的结果},许多斯拉法主义者明确地将其作为一种合适的描述而加
以接受。

在资本主义社会,利润是财产收入的主要形式,\textbf{在竞争中利润的分配通过生产价格
  的形式,生产价格使利润率平均化同时使经济的再生产得以进行。因此,不像新古典主义
  经济学中所说的那样:利润并不表示任一“生产要素”的报酬,价格也不反映相对稀缺
  性。}在这种剩余观点中,存在着一个\textbf{用商品生产商品的“循环过程”,这与新古
  典主义观点中的“从‘生产要素’到‘消费品’的单程路径”相反。}而且在新古典主义理
论中不需要描述\textbf{再生产和剩余}的概念,或者如果附加地提到,再生产和剩余的概念
也没有什么重要的意义;商品可以被生产出来,但没有被看作是再生产的投入和超过生产所
需投入的剩余价值之和。进而对那些坚持剩余范式的人来说,\textbf{正是生产和再生产的
  需要在决定经济行为中起着支配作用,它不能被准确地界定为是自主经济人的选择}(参见
以下第17章)。相反,依据在剩余生产和分配中的作用界定的是\textbf{阶级}。

凭借通常所指的\textbf{作为“长期状态”或“引力中心”的竞争性均衡}这一特殊概念,剩
余理论家们对价值进行了分析。不同部门中的利润率和同类型劳动的工资率分别被设想成是
相同的,同一单位商品以同一价格交易。这些特征在第13章所概括的斯拉法体系中是很明显
的,马克思主义对转型问题的论述很典型地遵循了这一均衡概念(参见本书第一卷第3章和以
上第12章、第14章)。斯拉法主义者声称对这些经济状态的特征曾进行过严格的检验,但马克
思主义者却没有这样做。他们对马克思主义的批判是这一结果,而对新古典主义理论的攻击
则是另一结果(参见以上第13章)。

对这些斯拉法主义者来说,马克思本人价值理论的重要性纯粹是\textbf{历史方面}的:它是
李嘉图经济学衰微之后剩余范式得以保留和扩张的主要媒介。然而,由于诸如德米特里夫、
里昂惕夫和冯·诺伊曼等经济学家后来超越马克思而进一步发展了剩余方法,而且通过这样做
揭露了马克思著作中的重要局限。因此,斯拉法主义者声称,如果马克思主义的真理要保留
的话,成为一个修正主义者是至关重要的。然而,\textbf{剩余范式的现代理论家们,对这
  一范式到底需要什么并未达成一致的观点。}加里格纳尼、伊特韦尔和米尔盖特喜欢直接建
立在《用商品生产商品》基础上的一种方法,而森岛通夫、古德温,可能还有帕西内蒂,特
别是冯·诺伊曼的著作,则认为线性生产理论更有说服力。琼·罗宾逊和后凯恩斯主义者则更
喜欢把卡莱茨基解释的凯恩斯著作作为发展的主要模式(参见以上第5章),而马尔格林和哈里
斯的观点更加具有综合性(参见以下第4节)。所有这些观点的共同特征是:把马克思的观点和
其他经济学家的观点混合起来,特别具有\textbf{冲淡马克思主义成分}的效果,以至于现代
马克思主义政治经济学到底是什么变得模糊不清了。以下事实则进一步加剧了这一倾向,即
对剩余范式特征的界定有时与新古典主义理论相对立,而且对于到底什么才是新古典主义的
本质也存在着不同的看法。

下一节将要概括历史理论家们如何用\textbf{斯拉法的观点来解释经济思想的发展,特别是
  价值理论的发展。}他们强调,自从资本主义出现以后,就存在着两种主要的分析立
场:\textbf{剩余传统和供求理论。}第3节探讨马克思主义者对这一观点的评论,认为把马
克思并入到同时也包括象李嘉图和斯拉法这样的资产阶级思想家的政治经济学的流派中是完
全不合适的。第4节接着对现代剩余理论家们如何试图超越《用商品生产商品》而扩展他们研
究经济学的方法作了概括,第5节则探讨对他们的努力所作的批评。

\section{经济思想史中的剩余传统}

罗纳德·米克和莫里斯·多布很快就察觉到在马克思和斯拉法之间有某些相似之处。他们各自
及时地根据斯拉法的著作重写了经济学史,认为至少从18世纪开始,经济理论就在两种传统
中发展:供求方法和剩余范式。多布坚持认为,亚当·斯密是一个关键性的人物,可以从他的
理论中同时引出这两种传统。李嘉图和马克思在19世纪发展了剩余方法,而德米特里夫、博
特凯维兹、里昂惕夫、冯·诺伊曼和斯拉法则是20世纪发展剩余方法的主要人物。米克的着重
点稍有不同,他对重农主义的广泛了解,导致他抬高了亚当·斯密在现代剩余理论范式的公式
化方面的重要作用;他对历史唯物主义起源的研究加强了这一结论。米克认为,在斯密之前,
没有一个经济学家正确地构思资本主义社会的阶级结构,他们也没有意识到资本主义阶级关
系积累的重要性。此外,尽管米克没有对李嘉图著作在剩余传统中的中心问题提出质疑,但
是他对斯拉法本人对李嘉图经济学发展的解释不应受到指责这一点并不充满信心。在编辑
《大卫·李嘉图著作及通信集》中,多布与斯拉法进行过广泛的合作,因此毫无疑问,多布在
李嘉图理论的确切性质方面受到了斯拉法观点的影响。

米克和多布在运用斯拉法的著作重写经济思想史时,没有遇到什么困难,因为这与他们已有
的观点十分一致。\textbf{1960年之前,他们各自都曾试图把马克思严格地界定在古典政治
  经济学的传统中,认为古典分析不能被简单地看作是新古典主义经济学的萌芽形式(象许多
  非马克思主义的思想史学家所做的那样)。}另外,在斯拉法的批判超出把转型问题看作是
一个“复杂的迂回”的逻辑变得很明显之前,他们都\textbf{已过世}了——多布1976年去世,
米克1978年去世(参见以上第13章)。因此,他们的理论史著促进了那些继续为经济思想史
的“双重传统”解释进行辩解、同时也接受斯拉法对马克思批判的全部力量的更为忠诚的斯
拉法主义者。\textbf{加里格纳尼非常仔细地考察了古典的和马克思主义的价值理论的结构,
  认为这两种理论在逻辑上和历史上都与凯恩斯的有效需求理论相一致。}米尔盖特认为,凯
恩斯可以被看作是一个“长期”理论家,他的分析可以不受20世纪60年代在“资本争论”中
遭到破坏的残余的新古典主义因素的影响。确实,\textbf{米尔盖特和伊特韦尔}曾强调指出,
新古典资本理论的逻辑缺陷完全颠覆了凯恩斯著作的其他解释,而且\textbf{要求把凯恩斯
  理论重新界定在剩余范式之内。}

供求理论的进展也引起了斯拉法主义者的关注。加里格纳尼、伊特韦尔和米尔盖特都曾指
出,\textbf{在20世纪30年代,在希克斯和哈耶克的影响下,新古典主义均衡理论放弃了价
  值理论的传统“目标”——长期均衡。}很显然,瓦尔拉斯关于经济关系的观点从一般意义上
看与一般利润率的存在不相一致。由于这一原因,\textbf{包括资本物品供给价格上的不均
  等收益率在内的内部暂时均衡和短暂均衡的观点,开始代替了长期的重力中心,而这一中
  心以前曾在两种传统中支配着价值理论家的分析。}结合着对新古典主义理论其他形式的致
命批判,\textbf{斯拉法经济学家由此声称,剩余范式是政治经济学唯一连贯的方法。}

多布本人就是这一观点的拥护者。在《用商品生产商品》之前,他就曾用古典政治经济学和
马克思主义政治经济学的优点来攻击新古典主义理论,但他却从来没有对供求分析的逻辑连
贯性展开全力攻击,只是与米克一起声称,它从概念上讲是错误的,从意识形态上讲是有动
机的。在受到斯拉法的影响之后,多布的关注点有了一个重大的转移。多布现在则声称新古
典主义理论的科学地位已完全被削弱;这也就意味着,马克思本人完全击溃庸俗经济学的目
的已达到了。

\section{马克思主义对剩余传统的批判}

\textbf{许多马克思主义者反对把马克思并入到剩余范式中,经常声称斯拉法主义者及其前
  辈本身都是庸俗经济学家。}相反,他们认为马克思主义政治经济学是独一无二的,它
与\textbf{新古典}分析之间有着明显的\textbf{分岐};把马克思主义政治经济学并
入\textbf{斯拉法}的思想源流中,不可能不对马克思主义政治经济学的统一性造成无法弥补
的伤害。这些评论家无需就斯拉法批判马克思价值论的逻辑一致性进行质疑;确实,有些人
承认它在形式上是有一定道理的,但是他们拒绝接受批判马克思时所使用的术语,仅仅接受
了对新古典主义理论的攻击,尽管它也是庸俗经济学(参见以上第14章)。

在为多布的《亚当·斯密以来的价值和分配理论》所写的评论中,\textbf{保罗·斯威齐}提出
了一种温和的批判形式。
\begin{quotation}
  马克思的理论确实是建立在\textbf{李嘉图理论基础之上}的,并且在多个方面\textbf{发
    展}了李嘉图的理论。\textbf{但与李嘉图完全不同,马克思认为他的任务是建立对整个
    资本主义秩序全面的、不妥协的批判,}包括对整个资本主义秩序的运动方式竭尽全力的
  全面的、不妥协的批判。为了完成这一任务,他开辟了一个全新的领域,建立了一
  种\textbf{既反对古典经济学又反对新古典主义经济学的传统…}…在多布看
  来,\textbf{斯拉法就是他所说的传统在当代的化身,}斯拉法著作的书名本身就与马克思
  的方法截然不同。马克思没有着重去考虑“\textbf{用商品生产商品}”,他的主题
  是\textbf{用人类劳动生产商品。}
\end{quotation}
本着同样的精神,其他马克思主义者也曾指责多布和米克两人都贬低了马克思对古典政治经
济学批判的意义。

在这里,有一点是真实的。马克思把一系列的质的特征和量的内容并入到他的价值范畴中。
他阐述了一个“劳动价值理论”和一个“价值劳动理论”。换句话说,\textbf{马克思的概
  念试图用古典政治经济学中所缺乏的一种明确的方式来揭示商品生产体系的社会关系。}我
们可以很公正地指责多布实质上忽略了马克思对古典经济学的批判,指责米克不合逻辑地把
只有在马克思的著作中才清晰地出现的质的特征硬塞进了斯密和李嘉图的著作中。由于这样
做,他们轻视了不连续性,贬低了马克思下述合理的观点,即他的古典主义先辈们的经济学
中渗透着的\textbf{拜物教}观点。马克思的经济学不仅仅是古典政治经济学的线性延续,而
且不论斯拉法对马克思价值理论的量的批评是否有道理,这都是真的。然而,马克思主义评
论家却忽略不谈米克和其他人在质的分析中,试图运用的是同马克思最初建立的框架相并行
的斯拉法框架。更重要的是,他们未能意识到马克思价值范畴的神圣本质使得理论整体结构
的诸方面都存在一种缺陷。\textbf{在经希法亭、比德里和鲁宾到斯威齐和罗斯多尔斯基的
  马克思主义思想史中,发现这样一种只强调价值理论的质的方面、从实质上排斥量的方面
  完全不同的传统,确实是有可能的,}而斯拉法主义者则与作为庸俗经济学的新古典主义理
论家混在了一起。

马克思主义对斯拉法的批判没有提到这一极其基本的要点,但他们提出了两个更为深刻的问
题。第一,斯拉法的批判采用了一种特殊手法,由此决定联合生产方式生产的商品之间具体
劳动的分配,马克思没有发表过这种见解。批评家们认为,这样做的结果就是斯拉法主义者
把太多的意义硬塞到了他们自己的结论中,而最后并没有表明他们对马克思价值的量的分析
的理解是恰当的。第二,\textbf{科学分析仅仅是马克思主义努力成为以转换现存秩序为目
  标的批判性理论的一个基本元素。}在斯拉法的框架中,没有什么东西能够足以弥补由于排
除对\textbf{剩余价值}概念的所有考虑而造成的\textbf{思想政治损失}。

然而,这些论证却无法为“不管怎样继续进行下去”的策略进行辩护,这是某些马克思主义
者面对斯拉法的批判所采取的策略。\textbf{马克思价值理论在量的方面的含糊不清,被反
  对斯拉法的马克思主义者所误用。}第13章中所概括结论中的自负,远甚于在阐述其过程和
假设中的自负。斯蒂德曼的批判家们也总是经常退回到对不确定的和难以理解的价值—价格关
系的“真实”性质的描述中,当然他们也没有对《资本论》第3卷不同于《资本论》第1卷的
价值理论进行演绎的、形式的描绘。\textbf{在《资本论》第3卷中,可以辨认出马克思是处
  在李嘉图传统中},而且这一点也通过恩格斯对“有奖论文比赛”的探讨而得到了进一步证
明(参见本书第一卷第2章和第3章)。而且,劳动价值论的政治—思想作用依赖于它自身的科学
真理。\textbf{科学与政治的分离会威胁到马克思方案的整个基础,正如在斯大林统治下把
  马克思主义用作国家意识形态所展示的那样(}参见以上第2章第3节)。

反斯拉法的马克思主义者从捍卫马克思转到攻击斯拉法经济学的刻板结构时,有着更强有力
的根据。然而即使如此,他们也错误地把斯拉法著作中没有包括的特性硬塞进斯拉法的著作
中。象斯蒂德曼所说的那样,《用商品生产商品》本身没有包括对劳动和循环过程的分析这
一事实并不意味着它无法内在地解决这些问题。指责斯拉法的理论仅仅局限于交换关系,把
生产看作是一个技术问题和并入了拜物教观点等等,是毫无意义的。这一结论是以下述事实
为依据的,即\textbf{斯拉法主义者试图在斯拉法原著的基础上建立一种全面的政治经济
  学,}正如我们将在下一节所展示的那样。

\section{超越价值理论}

剩余理论家们采用了多种多样的研究方案以扩展和丰富他们研究政治经济学的方法。并不是
所有的人都认为《用商品生产商品》是最合适的出发点;有些人更倾向于里昂惕夫和冯·诺伊
曼的模型。但是\textbf{所有的人都认识到,凡是与马克思的观点有联系的地方,必须把这
  些观点并入到非马克思主义理论家的著作中,}包括象凯恩斯和熊彼特这样的反马克思主义
者的著作中。不过,这种修正主义形式不自觉地比许多早期的看法\textbf{更欠积极性},并
且只是把增强马克思主义对资本主义发展的洞察力作为一般目标,而不是对之提出进一步的
质疑。

\textbf{加里格纳尼、伊特韦尔和米尔盖特}不仅强调了斯拉法价值论述的独特优点,而且还
声称该论述准确地反映了作用于资本主义经济的因果关系的总体结构。\textbf{价值理论
  的“目标”——长期均衡,允许把分析集中在通过竞争发挥作用的“持久而系统”的力量上,
  并且把它们从复杂和短暂的失衡影响中抽象出去。}决定均衡价格和其他分配变量的价值理
论的论据(技术、产出和分配的一个变量)反映了在产出、分配、价格和技术进步的实际变化
之间\textbf{缺乏任何一般的因果关系}。其结果是,这些论据中每一决定性要素大部分都相
互分离,因此政治经济学要求采用一种包括对每一类型变量进行不同解释的分割的方法,而
不是象新古典主义理论家所尝试的那样,用一套\textbf{单一的论据}来同时决定所有的变
量。\textbf{斯拉法主义者声称,古典的和马克思主义的经济理论的方法的标志,是对每一
  类型的变量,或一组系统上内部相关的变量(如利润率和均衡价格)分别进行解释,接着是
  考虑它们的相互作用。}这一点与供求传统中运用的\textbf{同时决定}的方法论形成鲜明
的对照。剩余理论的发展过程反映了这样一种信仰,即尽管在资本主义经济中所有现象不断
地相互作用,但是也存在着\textbf{相对自动的关系子集}。因此首先可以把它们看作是独立
的,而它们与其他现象(从经验上它们自身也是变量)的相互作用问题则可以推迟到下一阶段
进行研究。

从目前的发展来看,斯拉法的观点存在着\textbf{实际的弱点}。加里格纳尼、伊特韦尔和米
尔盖特更多关注的是理论构建的规则,而不是真正地去建立超越斯拉法的任何理论。而在他
们试着去这样做的地方,却又总显得与他们方法论的限制不相一致。他们没有提出这样的问
题,即李嘉图和马克思的“每次一个问题”的步骤是否代表有关因果关系的一个真正的观点,
或者只是简单地反映了在处理系统的内部关系方面的分析上的不足。他们也没有检验预先假
定的资本主义竞争的“持久而系统”的力量是否会导致经济收敛于长期均衡。其他人这样做
了,但从整体来看,他们得出的结论并不令人鼓舞;\textbf{非集中的调节路径极为可能,
  但由此则削弱了把长期均衡作为“趋势中心”的观点。}

毫不奇怪的是,剩余范式内的许多现代理论家都把他们的\textbf{注意力限定在稳定状态增
  长中的长期均衡的序列上,}这是马克思在其再生产模型中首先开始分析的内容。如果给
《用商品生产商品》加上\textbf{规模收益不变}的假定,其结果就可以\textbf{并入到包括
  里昂惕夫和冯·诺伊曼的线性生产理论中。}而且,有可能把对生产的这一分析与第5章中讨
论的\textbf{卡莱茨基的有效需求分析}连接起来,这样做在一定程度上补充了斯拉法的价值
理论。

以上第5章方程\eqref{e:fifth2}可以重新写成:
\begin{equation}
P=(1-s_c)P+ I  \label{eq:fifteen1}
\end{equation}
其中$P$代表利润,$I$ 代表投资,$s_c$是资本家阶级的储蓄偏好,用$K$(代表股本)除以等
式两侧,再重新确定术语,方程\eqref{eq:fifteen1}则变成:
\begin{equation}
  \label{eq:fifteen2}
P \times K=(I \times K) \times s_c
\end{equation}
\textbf{对于稳定状态增长的路径,}$P / K$是均衡利润率($r$),积累率($ I / K$)等于产
出增长率($g$), 那么方程\eqref{eq:fifteen2}可以写成:
\begin{equation}
  \label{eq:fifteen3}
r=g \times s_c
\end{equation}
\textbf{因此,利润率由积累率和资本家阶级的储蓄率决定。}把这一利润率\textbf{代入斯
  拉法的分析数据中,长期均衡的所有内生变量都可以得到决定。}卡莱茨基的分配理论和斯
拉法的价值理论融为一个\textbf{连贯的整体}。

\textbf{卢格·帕西内蒂}已表明,方程eq\ref{eq:fifteen3} \textbf{并不是基于工人的储蓄
  是零}的假设基础上【方程\eqref{eq:fifteen1}是以工人储蓄为零为基础的】。他还运用方
程\eqref{eq:fifteen3}来支持\textbf{剩余范式},以对抗马克思主义的批评家。
\begin{quotation}
  马克思和庞巴维克的时代距今已有一个多世纪了,经济理论家们一直在争论,\textbf{利
    润率是否归因于资本的“生产力”}……但\textbf{现在}已开辟了新的地平线\textbf{。
    从长期看——利润率决定于……增长率除以资本家阶级的储蓄偏好,它不依赖于资本的任
    何“生产力”……而且它实际上不依赖于任何其他因素。}
\end{quotation}
很明显,\textbf{这一点与马克思政治经济学的一般观点相吻合,因为重要的是积累过程和
  阶级活动,而不是通过交换关系而相互决定的“生产要素”的“神圣的三位一体”。}然而,
稳定状态增长模型只是第一次接近于理解现实资本主义中积累的干扰条件。\textbf{帕西内
  蒂}则比他们走得更远,从而开辟了与马克思主义理论的其他问题直接相关的途径。

在《\textbf{结构变迁和经济增长}》一书中,他考察了具有\textbf{不稳定技术变迁率},
并逐渐形成\textbf{消费模式}的经济的特性,他还特别地指出了许多马克思主义者一直看作
是资本主义发展基本矛盾的消费不足倾向(参见本书第一卷和以上第1章和第二
篇)。\textbf{马克思主义危机理论的过度积累观点,也由于理查德·古德温把失衡并入到冯·诺
  伊曼的增长模型中而得到了促进。}他的结论很大程度上处于奥托·鲍威尔和保罗·斯威齐的
传统之中(参见本书第一卷第6章和以上第1章)。但是古德温的周期分析更严格些,而且该分
析还并入了熊彼特和马克思的分析因素。

剩余理论的另一个组成部分则形成了\textbf{积累和危机模型},这一模型更加折衷但却明显
地展示出马克思的影响,并围绕马克思主义政治经济学来探讨问题。\textbf{琼·罗宾逊的
  《资本积累论》是第一部,也是迄今\CJKunderdot{最著名的积累和危机理论}},当然还有
其他的理论,特别是斯蒂芬·马尔格林、唐纳德·哈里斯和后凯恩斯学派的理论。\textbf{对
  新古典主义理论的敌意是对后凯恩斯主义著述者的促动力量,其重要性可能同有马克思主
  义倾向的著述者是一样的。但是,这两种影响趋于相互补充,即使两者在逻辑上并不是相
  关联的(参见以上第5章)。} 从这一意义上讲,克里斯托弗·布利斯非常正确地表明:现代
经济分析的高级理论中的\textbf{基本冲突是马克思主义和“庸俗经济学”之间的冲突。}

然而这一冲突并不是单方面的。《用商品生产商品》以供求理论家能够理解的方式使他们感
到痛苦,他们于是组织了反攻。紧接着新古典理论在“资本争论”中的\textbf{失败},出现
了萨缪尔逊在20世纪70年代早期对转型问题的批判,但这一批判很快就被从剩余理论传统中
出现的对正统马克思主义所进行的更深层次的批判\textbf{所超越}(参见以上第13章和
第14章)。\textbf{从那以后,萨缪尔逊和其他新古典主义理论家瞄准了更具有实质性的目标,
  而且他们的一些观点在反对斯拉法的马克思主义者中间引起了共鸣。}正统经济学家和正统
马克思主义者无疑是陌生的同伴,但是他们却共同拥有\textbf{使斯拉法的大部分论断中立
  化}的目标,共同拥有\textbf{把斯拉法的著作作为一个“特例”吸收到新古典主义理论框
  架中去}的目标。然而,是供求理论家起了带头作用,他们对剩余经济学的剖析揭露出了严
重的缺陷,它使得马克思自己的政治经济学也难免遭受损失。

\section{斯拉法分析的局限性}

把古典政治经济学和马克思主义政治经济学定位于\textbf{剩余传统},已引起了来
自\textbf{某些经济思想史学家的连珠炮式的批判},如塞缪尔·霍兰德就认为,自斯密以来
只有一种传统,即供求传统,这其中也包括了马克思。然而,这仍然只是少数派的主张,它
甚至遭到一些新古典经济思想史学家的拒绝,\textbf{这些史学家对剩余理论的现代观点几
  乎不持赞同态度,但却承认用这种方式来解释古典政治经济学和马克思主义政治经济学会
  更有意义。}因此,斯拉法、多布和米克关于19世纪思想史的观点得到了广泛的认可,而不
是仅仅局限于剩余传统内的理论家。

然而,在20世纪,这一判断并没有频繁地扩展到里昂惕夫和冯·诺伊曼的著作中。\textbf{萨
  缪尔逊}确实指责过,在斯拉法的影响下,\textbf{经济学家们误解了冯·诺伊曼的增长模
  型,该模型一点都没有冲击新古典一般均衡理论的任何一种主张。}萨缪尔逊还认为,他自
己的著作同阿罗、德布鲁和库普曼的著作都已经证明,\textbf{里昂惕夫和冯·诺伊曼在现代
  供求理论}的发展中是如此关键。萨缪尔逊还声称,冯·诺伊曼的框架比斯拉法的框架更为
一般,剩余传统内的一些理论家也持这一观点。

\textbf{哈恩}进一步巩固了这些观点,断言在《用商品生产商品》中不存在与新古典一般均
衡理论的结论相矛盾的命题;\textbf{认为新古典理论自身可以得出斯拉法的结论,}并进一
步揭示了推广这些结论所必需的\textbf{限定性假设条件};因此把斯拉法经济学看作是新古
典主义理论的一个特例是有道理的。\textbf{特别是,只有在特殊的资本存量初始禀赋和最
  终需求结构的条件下,长期均衡才取决于新古典主义一般均衡模型。}哈恩认
为,\textbf{一般地说,利润最大化行为与斯拉法主义者所想象的均衡,即包括资本品再生
  产价格上利润率的平均化,是不一致的。}竞争性均衡涉及所有资产都有一个\textbf{相同
  的回报率},\textbf{但是新古典主义理论认为,只有在特殊的条件下,资产价格才会与它
  们的生产成本相等。}

在新古典对斯拉法主义理论家的这一攻击中含有冷嘲的成分,当然,斯拉法主义理论家对之
进行的回击也是如此。新古典批评者的立场在\textbf{形式上}与斯拉法主义者在同马克思经
济学关系中采取的立场是\textbf{一致的}。\textbf{新古典主义者完全同意斯蒂德曼的观
  点,}即他对马克思的评价代表着“\textbf{一个逻辑学的观点}”,而且“\textbf{任何
  人要想对它进行挑战,他们要么必须发现一个逻辑错误……要么必须明确而一贯地拒绝基
  于其上的一个或更多的假设。}”但是,新古典主义者还运用现代一般均衡分析,把这一观
点扩展到他们自己对斯拉法理论的批判中。另一方面,斯拉法主义者对自己经济学的捍卫,
模仿了许多反对斯拉法的马克思主义者的方式——他们接受了对逻辑一致性的批判,但却拒绝
接受其对手概念化的合适性。

\textbf{反对斯拉法}的马克思主义者经常与新古典主义者站在一起,没有意识到他们的批判
对马克思本人的分析是如何地有害。\textbf{现代供求理论揭露了所有剩余理论形式中存在
  的重要缺陷,而这并不依赖于对新古典主义世界观的承认。}斯拉法假设中存在的一个难题
就是“一致性原则”,\textbf{在“一致性原则”中,不但在所有过程中工资和利润率是相
  同的,而且每一种商品不论是作为投入还是产出,其价格是一致的。假定竞争条件不保证
  价格是固定的。}例如,可以采用下面的体系:
\begin{align*}
  a_{11}p_1^1(1+r) + l_1w  &= p_1^2 \\
  a_{21}p_1^1(1+r)  +l_2w &= p_2^2
\end{align*}
\textbf{价格的上标是指时间,如果价格不受限制地趋于一致的话(因此,$p_{11} \not
  p_{12}$是可能的),那么,指定$r$和货币兑换率标准,既不足以决定价格,也不足以决定
  工资。}

这一无法确定的特殊原因,也威胁着利润和剩余之间的联系。\textbf{产出价格与投入价格
  不等可以表现为如下方式,即一个正的剩余对于正的利润来说,既不是必要条件也不是充
  分条件。}因此,把经济体系看作是含有\textbf{再生产和剩余榨取},其作用是令人怀疑
的,\textbf{因为一旦接受了非固定价格的可能性,这两个概念与交换现象的联系不再存
  在。}结果,\textbf{剩余理论中因果关系的设想(这一理论认为 \CJKunderdot{生产和特
    定的分配关系完全决定交换关系}),不再是非令人相信不可的理论了。}反过来,它也对
剩余传统内的许多理论家为了把他们的经济学并入一般社会理论而对历史唯物主义概念所做
的独特应用提出了质疑。

\textbf{即使}交换关系像斯拉法经济学家设想的那样,\textbf{使不变价格向量普遍通行,
  他们对生产结构的论述也还存在一个难题,即要求生产过程的数量与商品的数量相等。}这
一假定从经济学上看似乎很随意,因为设想联合生产的经济中存在的过程数量少于其生产出
的商品数量,既是可能的,也是有道理的。\textbf{在这样一种情况下,内生变量是由以斯
  拉法为基础的经济学范围之外的因素决定的,除非再指定额外的假定条件。}这又一次“一
点也没有支持马克思来反对斯拉法:\textbf{在过程数量少于产品数量时,马克思的劳动价
  值甚至无法计算”}。

斯拉法在论述交换和生产时的这些局限性,对通过设置\textbf{标准商品}以重建马克思的剥
削理论来说也有着重要的影响,这一点在以上第13章第5节作了概括。\textbf{在缺乏不变价
  格、过程数量与商品数量不相等的情况下,既无法确立标准商品,}也无法把标准商品作为
整个经济体系的替代物。每一件可能发生的事都意味着,第13章第5节的观点只有在特定的技
术和交换关系类型的条件下才会有效,它并不是普遍有效的。

\textbf{剩余理论家之间的内部争论是具有毁灭性的。特别是琼·罗宾逊,}严厉批评了斯拉
法的均衡方法论。她声称,从马克思那里学到的一个教训就是“\textbf{必须依据历史来思
  考,而不是依据均衡来思考。”经济学从一个“不可改变的过去”转向“未知的将
  来”,“生产的无政府状态”使得协作的地域性失败。}在她看来,\textbf{分析长期均衡
  的最基本的理论基础只能是马克思的再生产模型:}它们能够用来阐明这些条件是如何地脆
弱,由此揭露了现实经济所遭受的各种干扰。反斯拉法的马克思主义者提出了类似的反对意
见,而且他们确实发现,新古典经济学家间存在着广泛的一致性意见,这些新古典经济学家
对均衡模型的应用,同样也不能反映他们与现实紧密一致的信念。

因此,斯拉法主义者、反斯拉法的马克思主义者和新古典主义者之间的争论,有时使得人们
很难准确地理解它们之间的分歧。当然,\textbf{现代供求理论}与马克思所相信的供求理论
的早期形式相比,确实更少具有“庸俗性”。例如,这一理论\textbf{认真思考了是什么条
  件威胁着均衡的存在,防碍着市场力量从集中向均衡发展,从而揭露了资本主义经济中危
  机和不稳定的潜在原因。}而且,在20世纪80年代期间,随着“\textbf{理性选择马克思主
  义}”的出现,使得思想学派之间区别不清的状况进一步加重了。“理性选择马克思主
义”既接受斯拉法对马克思的批判,也接受新古典主义者对斯拉法的批判,继而他
们\textbf{拒绝把剩余范式作为马克思主义政治经济学进一步发展的一种合适的方式。}相反,
他们接受了新古典主义方法论,用供求分析的工具来分析马克思主义中的问题。我们在以下
第17章将讨论他们的著作;然而,我们首先转向分析危机理论问题。

\part{当前的争论}

\chapter{“第二次衰退”: 1973年之后的危机理论}

\section{长期繁荣的终结}

\textbf{70年代初,有明显的迹象表明,战后的长期繁荣已经结束。}在1945年之后的四分之
一的世纪中,整个发达资本主义世界经历了经济快速增长的“\textbf{黄金时期}”,(大部
分国家)实现了实质上的、持续的充分就业,真实工资稳步提升,但并没有很快达到威胁资本
盈利能力的程度。但是,1973年之后,积累速度明显放慢。四个主要欧洲国家(英、法、西德
和荷兰)和日本的GDP,1950~1973年的年均增长速度为5.6\%,1973~1984年只有2.1\%;同
期,单位小时劳动产出率的年增长速度从5.3\%降为2.8\%。美国的衰退只是稍轻一些:GDP及
单位小时GDP产出的增长速度分别由3.7\%降到2.3\%、由2.5\%降到1. 0\%。失业的增加更引
人注目。例如,英国男性的失业率,1953~1970年间在1.1\%~3.6\%之间波
动,1971~1980年升到在3.6\%~8.7\%间波动,\textbf{1981~83年则在13.7\%~16.7\%间
  波动。}其他国家的情况即使没有这么严重,也基本类似。利润率及其在净产出中的份额
在70年代都降低了。\textbf{通货膨胀——在这之前的几十年,它与其说是个问题,还不如说
  是经济发展的刺激因素——令人注目地攀升。}一些经济学家,包括正统经济学家和马克思主
义者,都认为这些新情况为\textbf{帕尔乌斯和康德拉季耶夫首次认定的经济增长“长
  波”}(见以上第1章第1节)提供了深层证据。根据这些理论家的观点,\textbf{资本主义
  以50年的周期波动为特征,在它之上又叠加着人们更为熟悉的7至10年为周期的商业循
  环。}在一个长的上升阶段,\textbf{平均的资本积累率比较高,}周期性的上升阶段强劲
有力,而衰退相对微弱,1945年之后的25年属于这种情况。\textbf{但在一个长的衰退阶段,
  微弱的经济景气与相对深刻而长久的衰退组合在一起造成缓慢的增长。}战争期间可以看成
一次康德拉季耶夫衰退,1970年之后是另一次。

不是所有的马克思主义理论家都被“长波”的存在所说服,不管怎样“长波”本身还需要加
以说明。大萧条同70年代和80年代温和得多的、更象是通货膨胀的危机之间,存在着明显的
差别,以致于作任何简单的类比都没有说服力。尽管如此,马克思主义者还是在利润率有明
显下降,以及它是资本积累速度放慢之后非常重要的作用过程这一点上,达成了普遍的一致。
在解释\textbf{利润率下降的原因}时,马克思主义经济学家利用了上一代马克思主义者解释
大萧条的四种理论中的三种:\textbf{第3卷中的“有机构成提高模型”、消费不足理论和巴
  兰—多布—斯威齐的过度积累分析(参见以上第1章第2节)。}只有第四种认为\textbf{不同生
  产部门比例失调}的观点,在1970年之后缺乏拥护者。\textbf{而1973年的石油危机恰恰反
  映了制造业的商品产出与能源供给之间存在严重的比例失调。}可是,它被马克思主义经济
学家普遍地当成\textbf{更深层的经济问题}的结果,而不是盈利能力下降的根本原因。

\textbf{这保留下来的三个思想线索,经过不同方法润色之后,单独地或者以某种方式组合
  起来,构成了所有严肃的马克思主义者探讨长期繁荣终结问题的基础。}他们互不相同的主
张可以用数学形式加以概括。沿着\textbf{托马斯·韦斯科普夫}的思路,我们把利润率写成
如下的恒等式:
\begin{equation}
  \label{eq:weiji1}
  r \equiv \frac{P}{K} \equiv \frac{Y}{Z} \frac{Z}{K} \frac{P}{Y}
\end{equation}
在这里,$P$代表总利润;$Y$是净产出;$K$代表股本;$Z$表示潜在产出(股本被充分利用时
的产出水平)。于是,\textbf{利润率是生产能力利用率($Y / Z$)、潜在产出—资本比率($Z
  / K$)和利润份额($P /Y$)的乘积。}考虑到等式\eqref{eq:weiji1}是用\textbf{市场价
  格}而不是劳动价值表示的,\textbf{所以,它能够将这三种竞争的理论明显地区别开
  来。}\textbf{首先,利润率下降可以是由消费不足引起的,消费不足引起利润实现上的困
  难,同时降低生产能力的利用率($Y / Z$);或者,它可能是由有机构成的提高引起的,这
  反映在潜在产出—资本比率($Z / K$)的下降上;最后,利润率可能会因为利润份额($P /
  Y$)的下降而下降,这是由过度积累(它被韦斯科普夫称为“劳动力量增强”命题)带来
  的。}虽然在实践中存在很多实质性的困难,但在原则上,经验性的研究应该能够确定这三
个因素中的哪一个决定了1970年之后利润率的下降。

我们在第2节研究消费不足论者的解释,下一节转向有机构成提高的分析方法,第4节和第5节
讨论过度积累理论的两个变量,由政府经济职能强化引出的特殊问题是第6节的议题。最后一
节,我们将得出一些实质性的和方法论上的结论。

\section{对消费不足理论的重新讨论}

前面各章使我们想起,\textbf{现代马克思主义的消费不足理论}虽然在细节问题上存在很大
的区别,但却存在一个\textbf{共同的核心},这就是认为\textbf{供工人阶级消费的工资增
  长得太慢},赶不上\textbf{产出扩张}的步伐,因此造成\textbf{有效需求不足}。在解
释70年代及80年代早期的情况时,消费不足论者的解释遇到了\textbf{两个明显的困难。第
  一个与收入分配有关。}无论是在1973年开始的危机之前还是在它的最初阶段,工资、薪
金\textbf{在纯收入中的份额都在增长},与20年代劳动份额的下降(至少在危机开始爆发的
美国是这样的)形成鲜明的对比;\textbf{第二是关于“滞胀”的问题。}与\textbf{1929年}之
后各资本主义国家\textbf{价格水平都剧烈下跌}相反,\textbf{1973年之后的危机遇到的是
  迄今为止在和平年代前所未有的高通胀率。}消费不足论在1970年之后拥护者甚少,其中大
部分原因就在于这些异常情况的出现。

马克思主义消费不足理论的老前辈\textbf{保罗·斯威齐}力图对付这些异常情况。虽然他的
合作者保罗·巴兰已于1964年去世,但他与哈里·马格多夫一起作为《每月评论》的联合编辑
及经济专题主要的撰稿人,保持着活跃的状态。斯威齐没有修订再版《垄断资本》,但从他
后来的著作中可以看出,他强调的重点已经发生了\textbf{重要变化}。80年代初,他仍然坚
持“停滞的直接原因在今天与在30年代是一样的,是储蓄倾向过于强烈而投资倾向过于微
弱”。尽管斯威齐还显然地坚持他对\textbf{消费不足}理论的信仰,\textbf{但他不再
  把“剩余增长规律”作为垄断资本的根本矛盾,而改为赞成卡莱茨基的过度投资模
  型:“资本家的投资是经济增长的发动机,这是真实的;而投资趋向于引起资本过度积累、
  反过来导致周期性的经济危机,这也是真实的。”}“投资灾难”——正如卡莱茨基所说——在
于它\textbf{同时提高了有效需求和生产能力:“强烈的投资动机使投资迅速增长,反过来
  它又会破坏投资的动机”}。1973年之后的衰退,就是这样一种“投资阻滞”,它是由老的
产业过去的过度积累造成的,这些老的产业没有能在美国新兴产业迅速而有效的扩张中被抵
消。

\textbf{关于通货膨胀},斯威齐把\textbf{货币主义}的因素与\textbf{查尔斯·L·舒尔
  茨}1959年提出的\textbf{结构主义}结合在一起。根据舒尔茨的理论,\textbf{经济垄断
  部门向下的价格刚性特征,意味着需求结构的变化在本质上只会带来通货膨胀:价格在需
  求增加的产业上升,在需求下降的产业却不能下降。}斯威齐注意到,\textbf{这种类型的
  结构性通货膨胀,与实质性的生产能力过剩及失业往往同时存在。}因此,价格上涨的罪魁
祸首是\textbf{垄断者}而不是行业工会。欧洲著述者们夸大了阶级斗争的通货膨胀效应的重
要性,在很大的程度上,这与工会势力比较微弱的美国没有什么关系。\textbf{斯威齐认为,
  垄断组织的强大说明1973年之后凯恩斯主义需求管理政策失败的原因。凯恩斯理论以自由
  竞争为先决条件。}而在垄断的情况下,需求的增加却会引发价格上升、利润增加、成本增
加,最终使\textbf{工资上升(这是生活费用增加的结果)而不是产出扩大}。结果就是普遍的
通货膨胀:“\textbf{简言之:停滞越严重,反作用的财政金融手段就越严厉,通货膨胀就
  越恶化。}”\textbf{事实上,这次长期繁荣是建立在私人债务和公共债务持续增长的基础
  上的。}1945年之后的“金融爆炸”,打开了有利可图的\textbf{不动产业和建筑业}的投
资机会,刺激了以螺旋上升的利息支付的奢侈品的消费。\textbf{但金融部门的这种“过度
  膨胀”显然是“病态的和寄生的”。它是“战后的长期繁荣和70年代作为回报的经济停滞
  的奥秘。}随着繁荣逐渐消失,人们在很多年中,用越来越多的\textbf{债务}创造、越来
越疯狂的\textbf{投机}和越来越严重的\textbf{通货膨胀与停滞}做斗争。”现在
是\textbf{清算}的时候了。

对斯威齐的分析有几方面\textbf{反对意见。首先,}他从来没有成功地、仔细研究过60年代
末、70年代初的“\textbf{利润挤压}”,这与他在《垄断资本》中宣称的\textbf{“剩余增
  长”很难统一起来。}依照斯威齐早期著作的精髓,他坚持认为,\textbf{表面上剥削率的
  下降是非生产性活动持续增长造成的统计幻觉。}但他既没有沿着这条线索继续研究,也没
有试图纠正或更新菲利普斯对经济剩余做的估计。\textbf{第二方面}更严厉的批评关系到斯
威齐的\textbf{不充分的投资理论,}这与《垄断资本》的分析也不一致。\textbf{仅仅提到
  投资的两面性,并不能证实停滞趋势的存在。}而且,这种两面性在不接受斯威齐观点的资
产阶级经济学家当中,已经得到普遍认可,并且在哈罗德—多马增长模型的初期就一直存在。
就此而言,建立一个更精确的专门模型是有必要的,也需要对斯威齐早期消费不足分析持有
异议的观点进行反驳(参见以上第6章第4节),并提供一些证据,表明这个模型可以真正用于
分析与长期繁荣的终结相关的情况。可是,斯威齐对此没有提供任何东西。

\textbf{第三,}他对通货膨胀的解释不能令人满意。\textbf{舒尔茨的结构性理论只对50年
  代价格总水平的轻微上涨提供了一个似是而非的解释,但部门间需求的变化没有达到70年
  代两位数通胀率这样剧烈的程度。}就所涉及的货币因素而言,斯威齐的看法也不完全一致:
有时把它们看成是第二重要的因素,而在别的场合,又把它们看成是通货膨胀的主要原因。
此外,在货币和通货膨胀的联系问题上所存在的问题,\textbf{比斯威齐所意识到的更严重}。
沿着以下的思路,得出\textbf{马克思主义的货币理论}是有可能的:\textbf{商品货币的价
  值即黄金,与任何其他商品的价值一样,由生产它所花费的社会必要劳动量来决定。因此,
  黄金的价值只会随着金开采业劳动生产率的变化而波动。但在商品流通中,黄金被纸币、
  银行存款和信用所取代,纸币、银行存款和信用扮演着黄金中包含的劳动的代表或“符
  号”的角色。每一盎司黄金所对应的“符号”的量越大,它们所代表的劳动量就越小},因
为,更多的“符号”要用来代表与黄金中、同时也是用它们购买来的商品中包含的劳动量相
等的劳动量。既使是为了作为论据,而接受了这种劳动价值论的正确性(参见以上
第12-15章),\textbf{但用它说明通货膨胀理论仍然存在巨大的困难。没有理由能够说明为
  什么是增长的货币数量(而不是商品数量)应当被看成是通货膨胀的根本原因。}事实上,
把\textbf{货币的膨胀解释成真实经济失调}——如在工资、利润收入分配问题上产生的阶级冲
突——所造成的价格水平全面上扬的后果,与马克思对这一问题的分析一致的程度比现代金融
机构的行为方式与之一致的程度要高得多。

\textbf{因此,斯威齐的货币理论把货币与价格之间真正的关系搞颠倒了。}他又一次没有详
细阐述他的根据,只是\textbf{简单地、想当然地认为货币增长与价格上升有关。}这同斯威
齐总体分析上的第四个,也是\textbf{最后一个缺点}有一些关系,即引导他的是\textbf{未
  经证实的、矛盾的、有时甚至是完全错误的预言。}这样,他1981年所做的金融灾难就要来
临的预言就没有分析的基础,事实上,4年后他就抛弃了这一预言;他曾预期的里根政府大规
模军备开支有“强烈的通货膨胀倾向”也是如此,它所促成的持续、有力的经济恢复如此巨
大,使之绝不仅仅是“一次正常的存货调整”。总而言之,斯威齐对滞胀的论述不能令人信
服。

\section{再一次提出利润率下降}

\textbf{利润率下降}理论用于1973年之后发展问题的研究工作,可以在三个标题下作一评
价:\textbf{第一,一些理论著作对置盐信雄定理的意义的抨击;第二,对资本有机构成发
  展趋势及其与利润率的关系进行的(极少量的)经验分析;最后,一些著述者既没提供新的
  分析模式也没提供任何相关证据,只是用纯粹方法论的理由捍卫马克思《资本论》第3卷的
  理论。}

回顾一下可以知道,\textbf{置盐信雄定理表明,对于一项降低成本的创新来说,如果实际
  工资没有提高而利润率却下降了是不可能的。这与马克思有关论述的中心思想是不一致的,
  因为有机构成的提高使利润率下降,也就会使失业增加,因此会阻止实际工资持续增长(参
  见以上第7章)。}有人对此的回答是,\textbf{把它作为经济处于长期均衡状况且仅仅使用
  流动资本(这也是置盐信雄为其分析所作的限制)情况下的一种特例,进而否定这个定理的
  正确性。在短期内,各个企业固定资本的生产能力可能互不相同,因此而产生这种可能
  性(马克思曾仔细研究过),}即一种创新活动能提高引进企业的利润率、也能降低整体资本
家的利润率。早在1967年,苏联经济学家A.A.库伊斯就已经提出置盐信雄论证\textbf{能否
  用于使用固定资本的经济}这一问题。1979年又出现了两篇彼此独立的、煞费苦心地详细阐
述这个定理实际上能够用于固定资本的文章。但不久就遭到了\textbf{N.萨尔瓦多}的反对,
他认为置盐信雄定理\textbf{不适用于联合生产的一般情况}。这里是存在一些例外,而马克
思的观点可能是正确的。因为它只举了一个例子反驳一个一般性的定理,必须承
认:\textbf{在原则上,技术进步、失业增加和利润率降低,在一个联合生产的系统里毕竟
  是不矛盾的。但是,这些结论对解释长期繁荣的终结所具有的价值仍然是不清楚的。}当然,
马克思的利润率下降理论并不是在所有的情况下都正确,在联合生产的各种情况下也是如
此。

抨击置盐信雄定理的第二条理论线索是\textbf{安瓦·赛克}1978年提出来的,\textbf{他认
  为,在最早的流动资本模型中,利润率($利润/资本$)与边际利润($利润/成本$)是一致
  的。}在赛克看来,\textbf{事实上,置盐信雄证明了技术变化只会提高边际利润,并使实
  际工资保持不变。赛克观察到,在一个更一般的模型中,这与利润率的下降是完全一致的。
  这是对的,但并不切题。}因为对资本家来说,\textbf{边际利润不是一个容易察觉出来的
  目标变量。对资本家来说,采纳一种以降低利润率为代价的革新手段来提高边际利润是不
  明智的,因为它会减少既定的资本量能带来的利润总量。}最后一种理论上的回应
是,\textbf{在形式上接受置盐信雄定理的正确性,但拒绝承认它对资本主义经济的重要
  性,}因为在这样的经济中,无论是否存在大规模的失业,随着技术的进步,\textbf{实际
  工资都不可能保持不变。}在这样的经济中,\textbf{如果劳动生产率的增长不能快到抵消
  工资增长的程度,利润率就会下降。}把利润率写成如下的表达式,就可以很清楚地看出这
一点:
\begin{equation}
  \label{eq:weiji2}
  r \equiv \frac{P}{K} \equiv \frac{P}{Y} \frac{Y}{L} \frac{L}{K}
\end{equation}
在这里,我们\textbf{忽略了实际产出与潜在产出的区别。}$P / Y$是净产出中的利润份
额;$Y/ L$是平均劳动生产率;$L /K$是资本-劳动比率的倒数。如果我们用更现实
的\textbf{假设——实际工资与劳动生产率同步增长——来代替置盐信雄的假设——单位劳动力的
  实际工资不变,$P /Y$将保持不变。那么,就只有在技术进步带来的单位工人产
  出($Y/L$)的增加,小于它所带来的单位工人资本量占用量($K /L$)的增加时,利润率才会
  降低。这就需要提高大体上可以看作是马克思的资本有机构成的资本—产出比例($K /Y$)。}

现在看来,这第三种方法与解释这次“\textbf{新的衰退}”最相关。\textbf{如果免去逻辑
  上的质疑,它还指出了支配利润率发展趋势的重要变量:实际工资、劳动生产率增长和资
  本—净产出比率的发展趋势。}我们在本章的下两节研究第1个和第2个变量。第3个呢?有什
么证据能够证明1973年之后盈利能力的下降与大体上可由资本—产出比率代表的资本有机构成
的提高有关呢?

值得注意的是,在捍卫\textbf{利润率下降}分析的人当中,感到有必要号召大家来回答这个
马克思本人认为是核心的问题的人是如此之少。\textbf{埃内斯特·曼德尔}算是一例。他曾
再三强调,\textbf{这次长波的下降阶段——他确定,这一阶段从60年代后期就开始了——与资
  本有机构成的急剧提高刚好重合。他断言,这是由“第三次技术革命”的衰竭造成的。}在
这次革命中,电子工业和核能工业的进步,使\textbf{固定资本的构成要素变得便宜,这在
  很大的程度上抵消了技术构成持续提高的效果。}我们可以把核能是否使除了人生命之外的
其他物品变得便宜这个问题放在一边,也可以撇开1965年之后电子工业生产力的进步是否逐
渐衰减这一有争议的论断,只需注意到曼德尔《第二次衰退》中的57个统计表
格,\textbf{没有一个涉及有机构成。}

\textbf{韦斯科普夫}对美国经验所做的严谨而仔细的研究,没有为1975年之前\textbf{潜在
  产出—资本比率(即\eqref{eq:weiji1}式中的$Z/K$)的急剧下降}提供任何证据,这同时是
利润率正在下降的时期。\textbf{但史密斯证明,70年代一些发达资本主义国家的资本—产出
  比率是上升的,}尽管如此,他还是注意到\textbf{利润份额(即\eqref{eq:weiji1}式中
  的$P / Y$)的下降对它起的作用,比对利润率下降起的作用要大,尤其是在制造业中。}哈
格里夫斯·希普也确认了资本—产出比率的上升,并把它与原材料价格的相对上升联系在一起。
最后,\textbf{利匹兹宣布,1973年之后,法国的资本—产出比率在经历了20多年的平稳不变
  或下降之后有明显的提高,}并引用了西德、英国(50年代中期以后)、日本以及美
国(从60年代中期开始)也有类似的上涨的证据。利匹兹还认为,\textbf{利润份额的下降有
  同样重要的意义。}

对一些\textbf{马克思主义经济学家}来说,这些(毋宁说是无确切结果的)研究在根本上是一
种\textbf{误解},因为\textbf{马克思的“利润率下降趋势的规律”不是用来对利润率长期
  发展趋势进行经验性预测的。}用马克思的话说,\textbf{他只是提出“一个资本固有的、
  同时又在发展过程中不断地被超越的界限”,它是资本“内在矛盾发展趋势的一个暴
  露”。}这意味着马克思的分析永远不会错,因为\textbf{它对经验分析没有用处},仅仅
是用来把影响利润率向不同方向变化的重要因素分离出来的一种分类学上的工具,并不打算
揭示现实资本主义制度某个既定时间哪种因素起支配作用。无论人们怎样看待这个论
点,\textbf{很显然,它与说明“第二次衰退”的原因无关,也不能用来反驳那些对马克思
  的论点及变量进行的经验分析。}


\section{“过度积累”与利润挤压}

随着\textbf{资本有机构成的提高对利润率的降低至多只起到微弱的作用}这一点变得日益明
显,人们的注意力\textbf{开始转向剥削率,它可以用净产出中的利润份额来反
  映。}\textbf{早期}马克思主义者探讨这个变量,\textbf{主要是希望剥削率的长期增长
  也许会抵消有机构成提高的趋势}(参见以上第7章)。\textbf{但70年代初,两位英国著述
  者指出,在过去20年中,盈利能力受到明显的挤压,其中利润份额的下降对利润率的下降
  起到主要的作用。}安德鲁·格林和鲍勃·萨特克利夫把这一点归因于联合战斗能力的增强,
它使货币工资直线上升。面对激烈的\textbf{国际竞争,英国资本家不能以涨价的方式把增
  加的成本完全转嫁出去,因此边际利润下降。}1975年,拉福特·博迪和詹姆斯·克罗蒂证明,
同样的过程对美国利润率的急剧下降也负有责任。类似的主题在\textbf{埃内斯特·曼德
  尔}的著述中出现得也越来越频繁,他一直认为\textbf{欧洲工人阶级力量的薄弱}(先是法
西斯主义的镇压、然后是第二次世界大战造成的)\textbf{是1945年之后经济扩张的重要基
  础。}到1980年为止,曼德尔一直在\textbf{“绘制” (比喻)一条“阶级斗争曲线”},强
调像阶级斗争这样的“主观因素”的“相对独立性”,并且否定了新一轮长期增长的可能
性,\textbf{除非有组织的工人阶级首先受到决定性的挫败。}过度积累理论的特殊说法在日
本\textbf{宇野学派}的危机理论中也能发现。

% Please add the following required packages to your document preamble:
% \usepackage{booktabs}
% \usepackage{graphicx}
\begin{table}[htbp]
  \centering
  \caption{美国的利润率及其决定因素,1949-1975}
  \label{tab:weiji1}
  \resizebox{\textwidth}{!}{%
  \begin{tabular}{@{}ccccccc@{}}
    \toprule
    \begin{tabular}[c]{@{}c@{}}变量\end{tabular}
    &\begin{tabular}[c]{@{}c@{}}整个时期\end{tabular}
    &\begin{tabular}[c]{@{}c@{}}第1周期\\1949.4-1954.2\end{tabular}
    &\begin{tabular}[c]{@{}c@{}}第2周期\\1954.2-1958.2\end{tabular}
    &\begin{tabular}[c]{@{}c@{}}第3周期\\1958.2-1960.4\end{tabular}
    &\begin{tabular}[c]{@{}c@{}}第4周期\\1960.4-1970.4\end{tabular}
    &\begin{tabular}[c]{@{}c@{}}第5周期\\1970.4-1975.1\end{tabular}
    \\ \midrule
    \begin{tabular}[c]{@{}c@{}}利润率\\$r$\end{tabular} 
     & 12.1 & 13.7 & 12.0 & 11.4 & 13.1 & 9.4  \\
    \begin{tabular}[c]{@{}c@{}}利润份额\\$P/Y$\end{tabular} 
    & 19.2 & 21.6 & 19.7 & 19.1 & 19.1 & 15.5 \\
    \begin{tabular}[c]{@{}c@{}}实际-潜在\\产出比率\\$Y/Z$\end{tabular} 
    & 83.6 & 85.0 & 83.3 & 79.8 & 84.7 & 82.3 \\
    \begin{tabular}[c]{@{}c@{}}产出能力-\\资本比率$Z/K$\end{tabular} 
    & 75.5  & 74.7 & 73.0 & 75.0 & 78.0 & 73.2\\ \bottomrule
    \multicolumn{7}{l}{资料来源 : Weisskopf. “Marxian Crisis Theory”,Table 2 ,p351}
  \end{tabular}%
}
\end{table}

美国的利润受到\textbf{挤压的经验性证据}是由\textbf{托马斯·韦斯科普夫}于1979年提供
的。他分解了美国从1949年到1975年战后五个经济周期中利润率变化的情况,
表\ref{tab:weiji1}是其概要。可以看到,利润率在第1周期、第3周期之间下降了,然后恢
复,在第4周期、第5周期之间又一次下降。\textbf{利润份额($P/Y$)的下降可以说明其下降
  的大部分原因,没有证据能够清楚地显示潜在产出—资本比率或生产能力利用程
  度($Y/Z$)的变化趋势。}后者实际上在第3周期、第4周期之间明显上升,这可以说明利润
率在此时的上升。确立了他的“劳动力量增强”命题之后,韦斯科普夫继续区分它
的\textbf{“进攻性的”和“防御性的”}的方面:
\begin{quotation}
  利润率从1949到1975年的长期下降,\textbf{几乎完全起因于实际工资份额的上升,}它表
  明劳动者力量的增强。但是,这种上升在很大的程度上是防御性的。工人阶级不能成功地
  使实际工资所得与实际劳动生产率同步增长,他们只能比资本家阶级稍稍成功地使自己免
  遭长期贸易条件恶化的伤害。
\end{quotation}
特别是1973年,“劳动力量的增强”使实际工资免受欧佩克石油价格上涨的冲击。


韦斯科普夫的分析只同\textbf{1975年之前的}情况有关,且仅限于\textbf{美国}。但利润
挤压似乎是一个世界性现象,并至少持续到这个10年的后期。\textbf{全面的文献是由格林
  和他的两个同事在1984年提供的},他们还详细地阐述了一个\textbf{一般的过度积累理
  论}。
\begin{quotation}
  过度积累的基本思想是:\textbf{资本主义有时候会产生一个比它所能支撑的积累率更高
    的积累率,这样,积累率最终会跌落下来。}在战后繁荣的后期,积累与劳动力供给之间
  的不平衡,导致劳动力日益严重的短缺。对劳动的过度需求使旧机器更快地变成废物。实
  际工资被提了上去,旧设备变得无利可图,\textbf{这使工人更快地转向新设备。在原则
    上,这个过程可以平稳地进行:}随着盈利能力的下降,积累率平稳地降到可以支撑的水
  平。\textbf{但是,资本主义制度不具备在这种情况下保证它平稳过渡的机制。}60年代后
  期,过度积累的最初结果是一段时期的过热增长,同时伴有工资、物价的迅速\textbf{上
    升},以及\textbf{对迅速致富的热切渴望}。这一切暂时掩盖了、却不能阻止盈利能力
  的恶化。资本家的信心遭到破坏,投资崩溃,大规模的破产随之发生。\textbf{过度积累
    带来的不是增长率的适度下降,而是一种典型的资本主义危机。}
\end{quotation}
这种情况往往与通货膨胀相伴随,因为\textbf{资本家为了抵消工资成本的上升会抬高物价,
  通过货币的膨胀(它本身是商业信贷增加的结果),价格的提升得以实现,并造成继此之后
  的货币工资的新一轮的提高。}

阿姆斯特朗、格林和哈里森的分析与格林与萨特克利夫10年前的分析有所不同,最明显的
是\textbf{把加快了的工资膨胀速度主要归因于增加了的投资对劳动力所产生的需求,并降
  低了工人阶级在生产中产生的战斗精神的自主性影响。}事实上,他们的处理恰恰在这一点
上易于受到\textbf{严厉的批评}。例如,他们提供的世界失业数据只与1965年之后的10年有
关,这就造成一种错觉,\textbf{好象在此前没有地方实现过充分就业}。但只需举一
个\textbf{反例}:50年代英国劳动力短缺的程度比后来任何时期都严重,这样,
就\textbf{搞不清楚}利润挤压为什么一直推迟到60年代末才出现。

\textbf{一定要用这种方式来说明过度积累命题是没有道理的。一个更似是而非的说法是持
  续的高就业与工人们对提高生活水平热切渴望的增强结合在一起所产生
  的“\CJKunderdot{滞后效应}”起了作用。}这种热切的渴望是在前一代工人阶级体验了挫
败之后,伴随着他们\textbf{集体自信心及战斗精神的复兴}而形成的。在某一个时期,当劳
动生产率增长的速度下降,加之贸易条件恶化,国家又通过\textbf{累进所得税}在平均收入
中占有越来越大的份额时,如果工资压力上升,利润就会被挤压。在这种解释下,“过度积
累”变成了卡莱茨基“\textbf{政治性商业循环}”中的一个变量。

与过度积累理论相关的一个深层次问题,涉及假设中的从60年代中期开始的\textbf{投资支
  出加速,这完全是由日本在世界经济中的份量的提高造成的,而实际上,积累在任何单个
  的资本主义国家都没加速。}在投资增加与利润挤压之间的确切关系问题上,也存在一些疑
虑。阿姆斯特朗、格林和哈里森\textbf{部分地用原材料成本的上升}(最明显的是1973年之
后石油价格的上涨)、\textbf{部分地用与积累率的加快恰好重合的国际竞争的增强}来解释
这个问题。但最重要的是,70年代劳动生产率的增长速度放慢。我们可以从下式中看出这种
世界性现象的重要性。
\begin{equation}
  \label{eq:weiji3}
\frac{W}{Y}≡\frac{W}{L} \frac{L}{Y}
\end{equation}
在这里,$W$代表总工资,$W/ L$是平均工资,$L/Y$是平均劳动生产率水平$Y/L$的倒
数。\textbf{这里提出了一个更宽泛的、工资和劳动生产率都在其中起一定作用的利润挤压
  概念。}阿姆斯特朗、格林和哈里森从多个方面解释劳动生产率放慢的原因:\textbf{资本
  投资能力降低,这反映在资本—产出比率的上升上};美国与欧洲和日本的竞争者之间的技
术差距缩小,来自于追赶北美工业的\textbf{“相对落后优势”随之消失};由于多少可察觉
出来的\textbf{工人对提高速度、技术细分和提高劳动强度的抵制,使来自劳动重组的收益
  日渐枯竭。}但是,他们\textbf{并不把}这看成是使盈利能力下降的\textbf{最重要}的因
素。正如我们在下一节将看到的,这与法国和美国马克思主义经济学家的观点形成鲜明对比,
他们把这最后一个因素放到比阿姆斯特朗、格林和哈里森在过度积累分析中所强调的因素都
重要的位置上。

\section{情感、意志和制度的积累}

\textbf{对劳动生产率下降的分析,支配着70年代法国“调节”学派和之后10年美国马克思
  主义经济学家的著述。}法国最著名的著述者\textbf{米歇尔·阿格莱塔}的《资本主义调节
理论》一书出版于1976年,三年后被译成英文。这本书的标题——连同这个学派的名称——容易
引起误解。通过“\textbf{积累统治}”、“\textbf{调节模式}”这样的用语,这位法国经
济学家实际上描绘了资本主义生产的\textbf{一系列发展阶段},每个阶段有其特殊的劳动组
织方式,以及满足消费需求的方式。阿格莱塔这本书的主题是\textbf{揭示最近的或者
  说“福特主义”阶段的危机。}

福特主义之前的阶段是“\textbf{泰罗主义}”,它主要指\textbf{大规模地采用技术手段对
  生产进行“科学的管理”}。在\textbf{福特主义}阶段,\textbf{半自动化的生产线可能
  带来的大规模生产,与更高的实际工资以及社会福利所带来的群众消费的扩张相匹配。}因
此,像对大萧条负有责任的\textbf{消费不足的危机,在1945年之后的20多年中得以避免。}但
阿格莱塔强调,60年代末以来,福特主义的\textbf{局限性}变得越来越明显:\textbf{首先,
  工作的速度和强度提高,}这使工人阶级身体困乏、精神疲惫,降低了劳动生产率,使旷工
率增加;\textbf{其次,由于计件工资不适应福特主义的管理制度,因此激励工人变得越来
  越难。}这反过来又影响到劳动生产率的增长,并由于工厂中\textbf{阶级冲突}的加剧而
恶化。这些因素综合作用的后果,就是经过几十年的相对下降之后,实际工资开始超过劳动
生产率的增长速度。为解决随之而来的“\textbf{盈利能力危机}”,就需要\textbf{对劳动
  进行根本性的重组,}资本家对丰富工作内容、建立“半自治性”工作小组的兴趣不断提高,
也许预示着这种发展方向。用克里斯琴·帕洛克斯的话来说就是:“\textbf{新福特主
  义}”产生了。

用这样枯燥的语言概述阿格莱塔的理论是不公正的,因为他的论证要精致和复杂得
多。\textbf{由于包含了资本主义国家的经济职能这样严肃的内涵(参见以下第6节),它能够
  加以扩展,既用来考察“新兴工业国”的“外围福特主义”,也用来考察集体消费成本的
  上升。}正如我们在前一节看到的,阿格莱塔的分析\textbf{还能在利润率降低的理论框架
  内}重新加以解释。它能够说明,\textbf{如果劳动生产率的增长赶不上实际工资的增长,
  利润率就会下降。}阿格莱塔还重复了巴兰和卡斯托瑞安迪斯的观点,详细地描绘
了\textbf{福特主义的隐蔽成本:“个人使用商品的强度在不断增加,而人与人之间的非商
  品关系却引人注目地枯竭。”}换句话说,调节学派提供的不是狭隘的经济学理
论;\textbf{福特主义的危机是多层次、全方位的。}

在英语国家中,这些问题已经引起了广泛的兴趣。\textbf{英国马克思主义者是在“劳动过
  程争议”这一令人生畏的题目下讨论这些问题的。}但最初他们并不倾向于将其与劳动生产
率的下降或经济危机联结在一起,而且劳动组织一直属于马克思主义社会学家、产业关系理
论家和劳动历史学家研究领域的问题。狭义地说,它对经济的重要意义是由\textbf{杰夫·霍
  奇森}在1982年发表的一篇被不恰当地忽视了的文章中提出来的。霍奇森引证了许多制造业
方面有代表性的研究资料,表明\textbf{英国工人的劳动生产率水平只达到西德、法国和美
  国工人的$1/2到1/5$}。他也提到虽然简明但却颇能说明问题的1974年\textbf{3天工作周}案
例,当时,\textbf{希思政府对能源使用实施的限制,使单位小时的劳动生产率突然提高
  了50\%(工作小时数降低了40\%,而产出只降低了10\%)}。霍奇森得出结论:\textbf{劳动
  生产率是一个变量,而不是既定技术条件下的恒量。劳动力的使用价值不可能在实际使用
  之前被提前确定。}

正如我们在以上第6章看到的,北美洲的马克思主义政治经济学在几近30年的时间里,颇受保
罗·巴兰和保罗·斯威齐极具特色的思想的影响,而巴兰和斯威齐则深受法兰克福学派批判理
论的影响。虽然60年代中期勃然兴起的“激进”很快废黜了《垄断资本》作为头等权威的教
科书的地位,仍由此产生了影响,\textbf{美国的马克思主义经济学家比其在欧洲的大多数
  同行,更可能接受“非经济的”或“上层建筑的”变量所具有的经济学意义}。因此,他们
研究\textbf{劳动的组织形式、劳动力市场上的不平等和歧视、劳动的家庭及性别分工、教
  育政治经济学、国家理论,以及所有对经济危机的传统分析。}

直到80年代初,塞缪尔·鲍尔斯、戴维·戈登和托马斯·韦斯科普夫还一直主张,这些因素是危
机理论的必要组成部分。他们对\textbf{长波}的解释强调他们称作的“\textbf{社会结构积
  累}”的作用,这是\textbf{一个由劳动管理、国际货币运行机制和原材料供应网等因素组
  成的系统。}提供经济的、政治的稳定的社会结构积累,能够在一次正常的周期性衰退的后
期,使令人满意的预期利润得到恢复,使资本积累继续进行。因此,这是一种\textbf{“生
  产性的”周期。然而在一种“非生产性的”周期中,社会结构积累不能恢复盈利能力,}要
想恢复持续的、有益的增长,就需要\textbf{对基本制度进行变革}。第一次非生产性周期是
大萧条之前1926~1929年的周期,第二次是1969~1971年,以及1973年之后带来的长期下降
趋势。

法国人的“\textbf{制度积累}”概念与鲍尔斯-戈登-韦斯科普夫的“\textbf{社会结构积
  累}”学说之间存在的强烈的\textbf{同族相似性}的印象,\textbf{被后者对劳动生产率
  的下降所进行的有影响的“情感和意志”分析所强化。}他们认为,劳动生产率下降的原因
在于\textbf{商业创新衰减及工作强度下降的共同作用}。他们认为,工人劳动投入的下降可
以从“马克思效应”那里得到解释:\textbf{劳动生产率是劳资斗争的产物},因此,它是由
工人受激励的程度、劳动对抗的范围和(相反地)雇主对劳动控制的效率决定的。\textbf{从
  资本家的角度来看,所有这些因素在60年代后期都恶化了,相当重要的原因是,随着福利
  津贴的提高和平均失业时间的缩短,工人失去工作的代价大大降低。}这为1973年之后的危
机打下了基础,而这次危机又因为生产能力利用率下降和贸易条件向不利的方向逆转而大大
地加剧了。

鲍尔斯、戈登和韦斯科普夫以引人注意的《\textbf{超越困乏}》为题,出版了一部很长的通
俗读物来表达他们的观点。这本书在两个方面有新的进展:\textbf{首先,它企图测算“法
  人权限的成本”},它反映在美国资本主义制度\textbf{经济浪费}的程度上。他们对浪费
的定义在某些方面与《垄断资本》的分析相类似,但在其他方面有根本的不同,具体包
括:\textbf{由于生产能力过剩而废弃的产出;劳动被错误配置及未被充分利用产生的结果;
  用于军事、医疗健康、能源、犯罪控制和超过了理性组织的经济所必需的市场活动中的过
  量的资源。}他们得出结论:1980年,有用的产出可能比实际达到的高出50\%。

《超越困乏》的第二个特点是,\textbf{它对解决危机所需要的激进政策作了详细的分析}。
鲍尔斯、戈登和韦斯科普夫强调,必须形成\textbf{一种新的社会结构的积累,以大大加强
  劳动人民反对法人资本的力量。}他们\textbf{提出了一种能在更高的实际工资的水平上恢
  复充分就业,并使经济得以恢复的工资引导策略,}它将增加有效需求、鼓舞工人士气,因
而会提高劳动生产率,并迫使资本家在其工厂中实行机械化,否则就会在竞争中灭亡。为保
证它的有效实施,\textbf{还必须给工会以更强大的法律保护,对工厂及自然环境施以大规
  模的民主管理。}虽然对潜在的、强大的反资本主义动力,它只表现一种动员手段,但在本
质上这是一种\textbf{平民主义}的方案,与70年代意大利及西班牙的“欧洲共产主义”所主
张的“历史折衷主义”非常相似。

虽然鲍尔斯、戈登和韦斯科普夫关于资本主义行为方式的模型及其对劳动生产率的下降所作
的基本假设,并\textbf{没有计量经济学的证据支持,但许多正统经济学家却发现对他们是
  相当合适的。}毫无疑问,正是这一切构成了\textbf{迪瓦恩反对鲍尔斯、戈登和韦斯科普
  夫“既不批判资本主义也不拥护资本主义”}的原因。但是,他们的观点是否正统与它的真
实价值是绝对无关的。这些是易变的,因为《超越困乏》的诊断比其开出的\textbf{药方}要
有力得多。\textbf{在所有的马克思主义者当中,只有消费不足论者阿尔·西曼斯基严肃地要
  求对劳动生产率的下降作出解释。}在利用\textbf{国际数据}的基础上,西曼斯基断
言,\textbf{劳动生产率增长的速度会因为就业保障的改善而提高(}不是降低)。事实上
是\textbf{经济停滞阻碍了劳动生产率的发展,而不是相反。其他批评者}更有说服力地提出,
鲍尔斯、戈登和韦斯科普夫\textbf{没对危机的国际范围给予足够的重视,而这一点就将排
  除任何一个资本主义国家单独实行高工资改革政策的可能性。它的政治前景也不乐观},尤
其在罗纳德·里根执政的美国。

\section{国家与经济危机}

鲍尔斯、戈登和韦斯科普夫可能被指责为犯了所有改良主义者\textbf{共有的重要错误,即
  过份倾向于把国家当作能被工人阶级及其联盟所控制,并为其谋利的中性的工具。}正如我
们在以上第5章所看到的,对这种基本上属于自由主义或社会民主主义的国家概念的反对,是
马克思主义者对凯恩斯作出回应的最显著的特征之一。但是,直到最近,马克思主义经济学
家在这方面,除了正统的习惯定则之外,并没有找到什么合适的东西。\textbf{简单地断言
  国家是“为资本利益服务的”,就等于对它在经济活动中的决定性作用或影响什么也没有
  说。}既使战后最深刻的评说,即巴兰和斯威齐对美国扩大民用的政府开支的障碍的说明,
也几乎完全没超出这些无用的泛论(参见以上第6章第3节)。

到70年代初,越来越明显的是,对危机管理的性质及其局限需要国家政治经济学的进一步的
发展。\textbf{与巴兰和斯威齐相反,有些著述者找到了政府扩大非军事性支出可以有利于
  资本积累的方法。最有意义的是那些由国家提供教育、医疗健康和社会保障的支出,并降
  低劳动力价值的方式。}随着生产变得越来越复杂,技术变化越来越快,社会对熟练、健康、
可流动、富有灵活性的工人的需求日益增长。因此,\textbf{“福利国家”被解释为适应了
  资本增长的需要。}

但是,这种\textbf{简单的功能主义}的解释,并不比传统马克思主义者的观点高明多
少,\textbf{对回应批评者们关于国家只是垄断统治阶级的囚徒的观点几乎不起什么作用。
  其他的马克思主义者更为激进,他们认为国家是阶级冲突的场所,}这种观点展示的前景就
是:一方面工人阶级在反对资本利益的斗争中赢得的某种妥协,另一方面国家本身在这些斗
争中获得了某种程度的“\textbf{相对自主}”。这样,国家的内部结构、人员构成和实践活
动的自身权力就变得重要了。资本家之间的不可避免的利益冲突,也必须由国家调
停。\textbf{国家的支出在许多方面是矛盾的}:通过提高劳动力的质量,能够提高劳动生产
率;使生产活动的重要领域免于竞争、鼓励浪费(参见以上第8章第5节),又降低了劳动生产
率。\textbf{它通过保护无效率的资本家、削弱群众失业者的有纪律的力量,刺激经济从危
  机中复苏,但同时又破坏了经济复苏的内在机制。而且,在积累需求——可支配的利润越高、
公司税越低——与合法的需要之间,总存在着固定的张力。}工人已经对正常的社会福利习以为
常,只对持续增加的供给品作出积极的反应。这种“\textbf{棘齿效应}”是对国家支出水平
上升造成压力的主要原因,也是造成“国家财政危机”的重要因素。与制造业相比,落后的
服务业的劳动生产率是造成财政危机的另一个重要因素。

尽管如此,有关国家政治经济学的许多最重要的问题,仍然有讨论的余地。\textbf{首先要
  考察的是税收负担的问题,只有假设税后的实际工资已经达到不可削减的最低水平,它才
  会完全落在资本身上。}只要这不是事实,资本家就可以通过制定更高的价格或支付更低的
工资,把部分税收负担转嫁到工人阶级身上,使自己有钱可赚。\textbf{但马克思主义经济
  学没有理论能够说明,用这些方法能使多大比例的财富在阶级之间转移。}涉及到政府支出
的效果时,情况稍好一点。\textbf{有些著述者强调类似于正统宏观经济学的“挤出”效应
  的影响;另有些人把国家支出看成是剩余资本的吸收器,它减弱了消费不足或利润率下降
  的趋势。}从某种程度上说,国家提供了一种“\textbf{社会工资}”,它可以在不影响雇
主支付的私人工资的情况下,提高工人整体的生活水平;或者它允许工资有所下降,但把教
育、子女抚养、健康和社会保险的成本完全社会化;\textbf{最后,一部分政府支出可以直
  接或间接地成为私人资本的补助金。}这些因素的相对而言的重要性是存在争议的;与之密
切相关的问题就是,哪些政府雇员的劳动可以看成是生产性劳动,哪些是非生产性劳动。

这些极为基本的问题上观点的不一致,使\textbf{马克思主义经济学家}在国家对长期繁荣的
终结所起的作用以及可能产生的影响的问题上发生\textbf{分歧}就是不可避免的
了。\textbf{大部分人与罗恩·史密斯一道认为,美国经济霸权的突然衰落使有效的国际管理
  更困难,它加剧了这个体系整体上的潜在的不稳定性,破坏了资本家对积累收益率的信心。
  许多人接受史密斯关于1973年危机的冲击力由于政府的干预而得到缓解的判断:“虽然协
  调得不恰当,却是有价值的,国际金融体系的完整性}在1974~1975年保存下来。如果那时
没有干预,1929年那样规模的恐慌很可能已经发生。”同样,这对理解1987年的股市崩溃来
说也是正确的。\textbf{无论在哪种情况下,对金融市场的不当配置都会带来巨大的现实危
  机。这些进展的确切意义还不清楚。}“由政府保证金融体系不会崩溃,不再出现成为30年
代大萧条前兆那样的普遍的通货紧缩,在长期中意味着什么?”这个由保罗·斯威齐和哈
里·马格多夫提出的问题等着人们去回答。

值得注意的是,\textbf{充分就业除了作为里根大规模军备扩张的副产品,在北美洲这个有
  限的范围内得到实现外,在别的地方都没得到恢复。在西欧的大部分地区,人们用紧缩的
  财政、金融政策来降低通货膨胀、削弱工人阶级力量、增加失业和扭转利润挤
  压。}\textbf{削减福利支出}是这个过程的核心,\textbf{并日渐增加地伴随着出售国有
  企业、放松对私人企业的管制、在累进税原则上实施大退却等。}几乎没有马克思主义著述
者预料到\textbf{经济自由主义}会在80年代复活。例如,晚至1978年,\textbf{埃里克·奥
  林·赖特还曾预言,政府干预将会继续扩大,会把“国家垄断资本主义”转变为羽翼丰满
  的“国家资本主义”,并强化这种制度的合法性危机。}既使是米歇尔·阿格莱塔,尽管承
认存在着政府撤销一些公共物品和服务供应的可能性,\textbf{但他的结论还是:“生活条
  件的不断的大规模的社会化,将会摧毁作为自由主义意识形态基础的私人企业制度”,创
  造出一个集权化的国家资本主义。}然而玛格丽特·撒切尔却带来了巨大的(非常令人不快
的)\textbf{震惊}。

\section{第二次衰退的教训}

马克思主义经济学家们\textbf{曾在解释和说明30年代大萧条的问题上发生过严重的分歧。}大
部分人赞同\textbf{消费不足}的解释,也有相当的人支持\textbf{比例失调论}和《资本论》
第3卷关于\textbf{利润率下降}的分析。他们在危机的预后问题上存在的差别也是显而易见
的:一些人断言资本主义生产方式的基本性质没有改变,另一些人却看到了不同性质的国家
资本主义制度正在酝酿的迹象(参见以上第1章)。其实,\textbf{80年代}马克思主义经济学
家之间的分歧比30年代还要严重。\textbf{大部分马克思主义者用过度积累理论中的一些变
  量说明长期繁荣的终结,但消费不足论者和利润率下降理论的支持者则对此持保留意见,
  而过度积累论者本身又在许多重要的细节问题上意见不一。}只有很少的人提出需要进行理
论的综合。总之,争论跟以前一样激烈。\textbf{有两个分歧明显的问题特别突出。一是经
  济基础—上层建筑的两分法,在危机理论中的运用明显弱化。对大萧条所作的绝大部分的现
  代解释,在本质上是经济学的,依赖于几乎是想象的资本积累模型},该模型与意识形态、
阶级冲突或劳工组织是无关的。这是一种变化,\textbf{而且第二国际年代马克思主义中教
  条的经济学的科学主义几乎不再有拥护者。}这就产生了一种丰富得多的(有必要在自信和
精确上有所欠缺的)政治经济学,产生了一种与\textbf{“西方马克思主义”的本质特征}相
符的政治经济学。

\textbf{二是本质上更直截了当的方法论的问题。}20世纪30年代几乎没有理论家会对功能主
义者观点的正确性提出质疑,\textbf{在功能主义者看来,经济的驱动取决于资本的(互相冲
  突的)“需求”或者(充满矛盾)的“逻辑”,以及在很大的程度上互不相关的个人决策动机。
  现在不再是这种情况。}例如,正如我们在生产组织问题争论上所看到的,\textbf{在新古
  典主义经济思想和反功能主义的科学哲学家的共同影响下,马克思主义政治经济学家开始
  探究,认为“资本主义社会的运动规律”是建立在单个资本家和工人理性选择基础之上
  的。}由此提出的问题是我们下一章研究的主题。


\chapter{理性选择马克思主义}

\section{理性选择马克思主义的实质}

理性选择马克思主义作为一种独特的思想流派,源起于20世纪70年代,在20世纪80年代获得
迅速发展。许多没有参与它的创立的马克思主义者,现在成了它的“反对者”;而那些不断
对它进行批判的马克思主义者,则对它趋于尊重。这一流派在总体上显示出三个主要特
征。\textbf{首先,}理性选择马克思主义者对\textbf{马克思主义理论的精确和清晰}显示
出非同寻常的关注。因此,这一学派有时又被称为“\textbf{分析马克思主义}”。它极其关
注的是概念的准确含义,导出结论的推演过程,以及这些结论能够在多大程度上支持传统的
马克思的命题。通过提出问题的方式:它的含义是什么、什么样的含义是最合理的、它们在
多大程度上是实际的真理等,对历史唯物主义关于生产关系制约生产力发展的论断作出批判
性评价。毫无疑问,通过这一方法,唯物史观的强势和弱点就清楚地显现出来了。

\textbf{第二,}在对马克思的著作进行分析时,\textbf{非马克思主义的概念和观点,特别
  是分析哲学的、建立数理模型的、现代心理学的和新古典主义经济学的概念和观点,发挥
  了出色的作用。}因此,理性选择马克思主义显然是修正主义的,尽管20世纪20年代以
来“资产阶级”观念向马克思主义的广泛“注入”,使这一点看上去似乎不成为它的新特征。
但是,就分析马克思主义而言,修正主义的特征最明显地体现在方法论中,而不是实质分析
上,这就完全改变了关于马克思主义最薄弱之处的典型观点。虽然理性选择马克思主义也对
马克思主义的核心观点提出批评——有时甚至是严厉的批评,但是它还是一再声称,马克思的
许多分析都是正确的,并且可以用非马克思主义的理论加以论证。

\textbf{第三,}理性选择马克思主义存在着一种\textbf{显著的倾向,即以决策者的理性行
  为推演出马克思关于社会-经济体系的命题。}正是这一特征,使得分析马克思主义同时成
为理性选择马克思主义。也正是这一特征,使得理性选择马克思主义的观点被看作是对阿尔
都塞结构主义(在20世纪60年代,它在马克思主义理论界居统治地位,参见以上第11章)和斯
拉法的政治经济学(它同时试图把马克思主义转换为以结构主义原则为基础的剩余经济学,参
见以上第13章和第15章)的反应。不过,理性选择马克思主义从来没有系统地尝试直接卷入其
中任何一种思想,也没有显示任何认为值得如此做的意向。实际上,\textbf{理性选择马克
  思主义很少对马克思之外的马克思主义者的著作进行探讨,他们给人以对马克思之后的大
  部分马克思主义著作都不很尊重的印象。}

关注马克思主义经济学的理性选择马克思主义者,主要有\textbf{格里·科亨、罗伯特·布伦
  纳、乔恩·艾尔斯特和约翰·罗默。}他们的著作不仅在所思考的主题上不同,而且在致力
于\textbf{个人主义形式}的解释上,在强调\textbf{理性选择重要性的必备条件}上,以及
运用\textbf{新古典主义经济学}的推理上也\textbf{都有所不同}。\textbf{科亨}对历史唯
物主义的技术观点重新进行系统阐述(这在以上第11章已作过探讨)只是强调,\textbf{理性
  是人类行为的决定力量},在稀缺条件下,它将以发展生产力的形式发生作用。科亨认
为,\textbf{对理性行为的偏离无需排除在外;并且,有关解释不是必定要采用关于行为的
  术语——理性或非理性,采用功能性解释就相当合理了。}\textbf{布伦纳}提出了另一种不
同的历史唯物主义观点,理性选择在这一观点中发挥了更为出色的作用,这在以上第11章已
作过探讨。但是,布伦纳的著作\textbf{对经济人面临的选择结构,几乎没有进行正式分析。
  而且,他描述的通常是阶级的而不是个人的理性行为。}只有\textbf{罗默和艾尔斯特}坚
决遵循“\textbf{微观基础}”的方法论,把个人的理性选择作为理解整个社会经济现象的基
础。

因此,理性选择马克思主义表现为各种不同纯度的理论观点。由于\textbf{科亨和布伦纳的
  观点已经在以上第11章作过探讨},因此,\textbf{本章要探讨的是理性选择马克思主义的
  更为激进的形式:罗默的经济学和艾尔斯特对罗默的经济学方法论的维护。}本章第2节概
述罗默对马克思主义经济学的看法,第3节则对罗默的观点进行批判。第4节和第5节探讨微观
基础的逻辑,以及艾尔斯特和罗默在多大程度上始终如一地维护这一观点。最后一节以探讨
他们对马克思主义政治经济学的主要贡献结束。

\section{罗默对马克思经济学的看法}

\textbf{罗默}广泛地研究了马克思的经济学,但是他的最独到的批判是:\textbf{马克思理
  论在传统上假设社会现象和经济现象是联系在一起的,但这些假设事实上只在概念上是确
  定的,在现实历史中可能是各自独立的。}在他看来,严密的方法要求理论家们推演出事物
同时具备的多种特征;这些特征应当被系统地阐述为定理而非公理。更具体地说,罗默坚持
认为,生产力的分布、具体的生产关系、阶级地位和剥削形式,原则上是可以相互独立的。
因此,如果它们实际上被联结起来,就需要按照相关个人的理性选择加以解释。为了支持和
解释这一观点,罗默还提供了一系列抽象的理论模型。罗默指出,\textbf{即使没有剩余的
  生产和阶级的存在(在商品生产系统中由其所处的市场地位所决定:雇主和雇员,贷款者和
  借款者),马克思用劳动价值定义的剥削也可能存在。另一方面,他还认为,没有剥削,也
  可能存在阶级划分};剥削者事实上也不见得比被剥削者更富有;在不存在劳动力市场的情
况下,剥削可以通过商品市场和信用市场而实现。

然而,逻辑上性质截然不同的现象,在实际中往往是相互关联的。罗默对重要的历史事件非
常了解,\textbf{他所建立的与历史无关的抽象模型的目的之一就在于:阐明存在“阶级—剥
  削对应”、“阶级—财富对应”和“剥削—财富对应”所必须具备的条件。}证明这些源自特
定条件的结合,将搞清楚阶级、剥削和财富交结在一起的实际历史状况的决定因素;而且也
将指出,那些在实际历史中本该发生的事情,在特定方式下是完全不同的。

我们可以通过解释剥削在简单商品生产条件下是如何发生的来说明罗默的观点。\textbf{设
  想存在一个由独立手工业者和农民组成的社会,}这些手工业者和农民\textbf{各自}使用
非熟练劳动、利用已知的技术条件进行生产。虽然所需的工艺技能因生产物及所需投入的不
同而不同,但是,\textbf{由于不存在联合生产,不同商品的劳动价值可以轻易地被计算出
  来。}生产者也因各自拥有的据以进行生产的\textbf{商品存货的不同}而各不相同。在每
一生产阶段,每位生产者都力图使本阶段已用完的商品恰好得到替换,销售所得的净收益则
全部用于消费。\textbf{消费方式将在个人偏好、相对价格和最初的资产禀赋基础上,由生
  产者的最优化行为决定。在瓦尔拉斯均衡中(在这一均衡中,价格决定于市场出清状况),
  并没有保证生产者所消费的劳动价值将等于他或她所生产出的商品的劳动价值。}因
此,\textbf{一些生产者可能被剥削,而另一些生产者则成为剥削者,这取决于他们所生产
  的商品的劳动价值和他们所消费的商品的劳动价值之间的差额。}但是,由于不存在劳动力
市场,也就不存在雇佣劳动;由于全部生产者都是小资产阶级成员,从而也就没有阶级的存
在。那么,很显然,这种生产者有着古典意义上的充分自由但已涉及剥削的商品生产系统,
并不必然与资本主义经济体系具有同等含义。

同样,罗默用另一个例子阐明,\textbf{资本主义剥削也不是必定包含雇佣劳动。}设想现在
有一个由两类不同禀赋的人——\textbf{富人和穷人——组成的经济体。}在现有可资利用的技术
条件下,穷人使用其贫乏的资源不能够生产出自己所需的全部消费品。为了生存,他们被迫
把自己的劳动力\textbf{出卖}给富人,而富人则把这些劳动力连同他们更加充足的其他生产
资料储备一起用于生产过程,结果就产生了作为雇主的资本家阶级和无产阶级。在一般均衡
条件下,当所有经济人都追求经济行为的最优化、全部市场都处于出清状况时,前者通常会
剥削后者。这显然没有什么特别之处。但是,\textbf{不涉及雇佣劳动的另一种结果也是可
  能产生的。穷人不再在劳动力市场上受雇于人,而可能租用富人拥有的生产资料。}这样,
他们就可以自行负责生产,或独立经营,或采取联合经营的方式;\textbf{通过向富人租用
  生产资料,剥削经由信用市场而产生。}除了体制上的这种改变,其他的与第一种情况完全
相同。在这里,存在一个租出资本的资本家阶级,和一个租用生产资料的无产阶级。实际上,
罗默证明了这两种情况之间存在着严格的一致性:\textbf{不论是穷人租用资本,还是资本
  家雇佣穷人,剥削都可以在同样程度上发生。这说明,雇佣劳动和资本主义剥削并不是必
  然联系在一起的。}

从上述例证中,罗默并未推演出资本主义剥削不重要的结论,也没有得出雇佣劳动是次要的
看法。相反,\textbf{他认为,资产的不平等分配是产生剥削的主要因素};劳动力市场之所
以比信用市场显得更为突出,是因为资本家通过诸如提高劳动强度这样的剥削方式能够获得
一定利益,而这在我们刚刚讨论的情形中则不明显。但是,罗默的确指出,\textbf{马克思
  经济学太过于夸大劳动过程的重要性。特别是许多马克思主义者都错误地认为,生产过程
  对劳动力的这种支配是产生剥削和阶级对抗的主要原因。}在罗默看来,\textbf{生产资料
  所有权的不平等,在每一种情形中都是主要因素。}它制约着经济人的理性选择,以致这些
选择几乎总是引发包含着剥削和矛盾冲突的资本主义阶级关系,即便不存在对劳动过程的统
治权的情况下也是如此。

罗默的观点存在一个\textbf{明显的局限,即它是建立在以劳动价值定义的剥削概念的基础
  之上的。}正如我们在以上第四篇看到的,\textbf{具体的劳动系数可能确定不出来,或者
  可能是反常的信号,总体而言,它们在逻辑上并不优先于价格因素。}罗默对与劳动价值有
关的令人困惑的问题阐述了自己的看法。他对阶级、剥削和市场所作的分析,并不表明他坚
持劳动价值论,也不表明他忽视劳动价值论存在的缺陷。他以劳动来测度剥削,是由于这一
方法简便易行,也因为他试图就剥削产生的根本原因与其他马克思主义者进行辨
论。\textbf{他还颇感兴趣地向新古典主义经济学家指出,贸易互利的存在,并不意味着剥
  削的消失;相反,在市场经济中,剥削正是通过贸易产生的。}这样,新古典主义的理
论“工具”,如\textbf{瓦尔拉斯的价格理论,转而成为新古典主义经济学的辩护性观点的
  反面论据。}

然而,与劳动价值概念联系在一起的这种分析上的缺陷,连同其他考虑一起,的确促使罗默
用不同的思想方法系统阐述剥削理论。\textbf{新的理论——通常被称作研究剥削的财产权方
  法——是用博奕论的概念清楚地加以表述的,}它可以概括如下:“设想存在一个由$N$个人
组成的社会(不论其类型如何),这一社会对财产所有权分配有着清晰的界定。\textbf{假设
  存在一种可以选择的可行的财产权分配方式,}这种分配在某些形式的财产权方面是平等的,
在所有其它形式的财产权方面至少不是不平等的。在这种新的分配方式下,\textbf{如果全
  部$N$个社会成员中的$S$个离开现存社会能够改善他们的经济状况,则这个由$S$ 个经济
  人组成的社会子系统就被定义为被剥削者。如果这一离开,使剩余的(N-S)个人的经济状况
  变坏,则他们就被归为剥削阶级}——因为他们的利益部分地依赖于实行一种逊于“平等”的
财产分配方式,而这种分配方式相对于更加“平等”的分配方式会给$S$带来损
失。\textbf{既然假设可以选择的可行方式为数众多,}$N$个人中S和($N-S$)的人数划分也
可以多种多样,那么,就可能存在\textbf{多种不同的剥削类型,以及多种剥削集团和被剥
  削集团。}罗默集中论述了封建剥削(其产生的基础是对他人的劳动力拥有所有权)、资本主
义剥削(除劳动力因素外,主要产生于可让渡的生产资产在分配上的不平等)和社会主义剥
削(在财产社会化条件下,因每个人拥有的不可转让的技能的不同而产生)。但是,他的理论
也可以用于解释对妇女和少数民族的剥削。罗默还指出,前面在特定背景下使用剥削的劳动
价值定义的例证,是剥削的财产权概念的特例。

罗默的这一理论构架,对其他马克思主义者的著作有着相当广泛的影响。例如,\textbf{埃
  里克·奥林·赖特应用这一理论阐明了现代资本主义社会中阶级地位的复杂性,特别指出中
  产阶级是如何被确切地定位的,以及他们所处的中间位置所固有的紧张状态。}不过,罗默
的兴趣似乎主要在于对一些规范问题进行探讨。在这一点上,他对自己的一般理论并不满意。
他对剥削的劳动价值概念和他自己提出的剥削的财产权定义,能否用于解释分配的道德问题
提出质疑。虽然传统马克思主义论述存在的种种缺陷和不足,促使罗默重新对剥削理论进行
系统阐述,但是,\textbf{他认为,即使是他创立的以财产权为衡量标准的剥削理论,提供
  的也是有缺陷的不公平指数。他认为,在许多情况下,资本主义剥削可能是公正的,或者
  至少不是不公正的。}另一方面,社会主义剥削则提出了一个\textbf{复杂的问题,即人们
  从其生而具有的优越的天赋能力中获益,在多大程度上是合理的。}

\section{罗默的马克思主义观的某些局限}

罗默强调指出,在生产资料供给稀缺的情况下,\textbf{资本主义剥削的主要根源在于生产
  资料所有权的不平等。他尤其注意反驳那种认为剥削只能产生于劳动过程的观点,并对那
  些把劳动过程看作至关重要的马克思主义者提出了批评。}罗默的这些不同看法,并不直接
针对马克思本人的观点。罗默攻击的是其他马克思主义者,而不是马克思。然而,按照罗默
的标准,马克思本人的某些论述也存在着许多缺陷。\textbf{马克思毫无疑问是轻视利润交
  换理论的,他暗示市场本质上与对剥削的理解无关;在他的简单商品生产模式中是不存在
  剥削的(参见以上第10章和第11章)。《资本论》特别强调劳动和劳动力的区分,这表明,
  在马克思看来,劳动过程确实与资本主义剥削有着极为特殊的关系。}

这些论述可以理解为,马克思是不会接受罗默的剥削理论的,这一理论不仅极具原创性,而
且是对马克思的强力批判。不过,\textbf{这样的解释是不正确的。马克思对利润交换理论
  的批判,是建立在对重商主义学说的评价之上的,是一种对纯粹的利润交换理论的批判。
  马克思认为,贸易所获利润的价值必定为零;一方之所得恰恰等于另一方之所失。为了获
  得纯利,必须使生产中创造的价值大于生产所使用的投入的价值。马克思的论证是正确
  的,}而罗默关于剥削的许多例证,在根本上把生产包含在内,并没有得出与马克思相反的
结论。因此,\textbf{马克思和罗默的意见是一致的:没有生产,就不会有剥削。}

但是,\textbf{在马克思看来,生产过程的存在并不足以产生剥削。在他的简单商品生产模
  式中,价格与劳动价值是一致的,}每一位生产者通过交换获得价值与自己生产的价值恰好
相等。因此,\textbf{这与生产者本身拥有的天赋是无关的——不论他们是平等的还是不平等
  的都无关紧要。只有在资本主义生产关系条件下,剥削才会产生。}这一看法正
是\textbf{罗默所反对的。他认为,剥削可以在简单商品生产条件下产生,而不必要求资本
  主义生产关系的存在。}

不论是马克思还是罗默的论述,\textbf{都没有逻辑上的错误。}他们得出的结论之所以不同,
是由于他们\textbf{应用不同的价格理论}。马克思相信,\textbf{李嘉图}的劳动价值理论
适用于简单商品生产;罗默的模型则建立在\textbf{瓦尔拉斯}的供求理论之上。当价值与价
格相符时,没有一个生产者可以得到少于或多于他或她所生产的价值。当价值与价格不相符
时,生产者得到的价值则可以少于或多于他或她所生产的价值。\textbf{从历史来看,马克
  思和罗默的论证基础都较薄弱。与实际商品生产系统(其中,生产者拥有自己的生产资
  料)相联系的极不完善的市场表明,不论是李嘉图还是瓦尔拉斯的价格理论,与之都没有太
  大的关系(参见以上第14章第5节)。因此,马克思和罗默的结论都缺乏经验基础。}

按照马克思的看法,\textbf{作为剥削产生根源的资本主义阶级关系,是原始积累过程的产
  物}。这一过程正如马克思所理解的那样,\textbf{同时也是生产者被剥夺的过程。因此,
  资本主义阶级关系的发展是与财产权利的日益不平等的发展同步的。}马克思还清楚地表
明,\textbf{他确信分配变化是这种关系变化的原因}。因此,他的观点与罗默的观点即便不
是完全一致,也是\textbf{极其相似}的。所以,根据这两位理论家的论述,\textbf{正是生
  产力所有权的不平等导致了资本主义剥削的产生。}

这一观点在马克思关于商业资本、制造业和现代工业的论述中得到进一步论证。\textbf{虽
  然马克思依据雇佣劳动来定义资本主义生产方式,但他承认,商业资本家可以通过信用市
  场剥削生产者,正如在生产内部系统中发生的一样。因此,马克思察觉一种重要的历史状
  况:在这种状况下,资本主义剥削与雇佣劳动无关。}从而他并不认为劳动力市场是产生剥
削的基本要素。在这一点上,马克思和罗默又是一致的。马克思的确没有把信用市场和劳动
力市场看作类质相同的市场,但他区分二者的理由,与罗默解释劳动力市场之所以在资本主
义制度中居支配地位的理由非常相似。马克思认为,\textbf{劳动力市场最先是与制造业相
  联系的。在制造业中,生产的技术基础不变,只是劳动分工大大扩展了。}资本主义剥削仍
然存在,并且发生在现在所谓的资本主义生产方式之中。\textbf{然而,由于技术水平象手
  工艺技术一样发展停滞,从而限制了“相对剩余价值”的生产,这意味着“资本总是被迫
  与工人的反抗作斗争”。资本家理想的要求是改变劳动过程的继承形式。}依据马克思的理
论,“\textbf{现代工业}”的出现已经使之成为可能。这种生产方式的特点不仅在于它是一
种工厂生产,而且在于它是一种使用动力机械的生产。\textbf{它使得工人的生产技能水平
  降低,使资本家通过提高劳动强度加强了对工人的剥削,进而加强了资本家对工人的统
  治。}这正是“劳动过程”理论家们所着力强调的(参见以上第16章)。把这些理论家的著作
置于马克思经济学的整体背景之中,则他们的观点与罗默把财产权不平等置于首位的主张并
不冲突。\textbf{依据马克思的政治经济学,剥削只是在资本主义劳动过程中得到加强,而
  不是产生于劳动过程。}罗默自己则承认,资本家对劳动过程的支配确实提高了剥削程度。

用马克思经济学来审视\textbf{罗默}关于剥削的论述,则这种论述的原创性就不那么显著了。
在他认为财产权关系优越于劳动过程并对其他任何强调相反看法的人提出的指责是正确的限
度内,他所做的不过是建议马克思主义政治经济学“回归马克思”。这一点,只是由于他构
建自己理论的方式不同于马克思的做法而被掩盖了。\textbf{在这里,马克思的真正有力之
  处,没有被罗默所“复制”。}罗默构建其理论模式的核心几乎完全以\textbf{逻辑关
  系}为基础。他使用了标准的新古典主义经济学的\textbf{比较静态方法,但又把它用于分
  析制度形式的变化以及为新古典主义经济学典型地忽视的量的变化。这一方法要求外生因
  素必须相互独立,从而使它们在其他情况均相同的条件下发生独立的变化,进而使这种变
  化引起内生变量变化的作用效果相互独立。}与罗默不同,\textbf{马克思使用的是“逻
  辑—历史”分析方法。}依据这一方法构建的模式更接近于马克思确信的历史中实际存在
的“特定情形”。同时,这种方法忽略对任何纯粹假设的世界进行考察。因果关系通过比较
不同历史结构模式而得到评估;但是,\textbf{这些历史结构不只在一个方面,而是在好几
  个方面都是不同的。}因此,在其他情况都相同的条件下,不可能产生一种独立的变化,也
不能估价这种变化产生的因果作用。

罗默的做法显然更具“普遍”性,因为它既适用于真实的经济社会,也适用于假想的经济社
会,还可以在各个参数纠结在一起的历史状况下,单独地改变这些参数。但也正由于这个原
因,\textbf{它忽略了那些被视作外生的因素之间的实际上的依赖关系。相反,马克思的方
  法则明确地试图抓住这些依赖关系,并对历史上发生的真实变化的重要意义予以评估。}马
克思对这种方法的使用并未能免除责难(正如我们在以上第14章第5节所看到的),但这种方法
本身比罗默的新古典主义方法更适合于马克思主义政治经济学。

\textbf{马克思(正确地)认为,个体的特性(如他们的偏好)、技术的精密程度和财产权都具
  有高度的相互依赖性。}\textbf{罗默}的新古典主义方法论(\textbf{与罗默实际信奉的相
  反)则认为它们是互不相干的。}这样,每一种因素都可以独立地发生变化,内生变量(如剥
削程度)的变化,可以视为外生因素变化的结果。因此,罗默关于信用市场和劳动力市场的同
质定理,假设制度结构的变化不会引起剥削指数的变化,同时这一定理又要求生产过程不受
任何市场运作类型的影响。然而,我们已经了解到,马克思(正确地)认为,\textbf{劳动力
  市场的引入最终会导致生产过程的革命。}与此相类似,\textbf{罗默确信财产权分配而非
  资本主义劳动过程才是资本主义剥削产生的原因,是基于这样一种分析:劳动过程不因其
  在资本主义或非资本主义生产方式中进行而有所变化。}在这个问题上,罗默与马克思
的“距离”几乎不能说是比较远,他的理论模型甚至对“劳动过程”理论家们提出的问题毫
无反应。

罗默的非历史方法不是偶然的。它反映这样一个事实,即他主要关注的\textbf{根本上就不
  是历史。他的理论意图主要受到伦理学的支配。}在有所保留的条件下,他是一
个\textbf{激进的平等主义者。}并且,罗默确信这是马克思主义者自然而然会采取的立场。
但是,这决不是显而易见的。马克思清晰地表述了他的道德观,其中一部分是对不平等现象
的抗议。然而,这些信念在马克思的成熟的历史唯物主义理论中的地位远不是那么清楚
的。\textbf{对马克思原著的任何具体的解释得出的结论都必须承认,马克思主义的局限性
  可能意味着现代社会主义对修正的伦理原则的需要。}如果把这种伦理原则与社会主义者在
历史上曾经认为是重要的道德原则结合起来,就不是绝对的平等主义。在社会主义思潮中,
理性主义和自由主义是至少同等重要的主题。这些理想的实现可能会制约平等的实现。

在罗默关于剥削的一般理论(其建立的基础是财产权的可行的重新分配)中,罗默经济学方法
的\textbf{缺陷}和论题本身的缺陷是交织在一起的。在对什么是“可行的”下定义时,罗默
并不很清楚应当把什么样的变量包含在内。\textbf{罗默无视特定分配形式的激励效果}——尽
管他并不是不清楚它们对于效率的重要作用。罗默还认为,\textbf{规模经济能够把他认为
  不公正的分配形式与那些可归入剥削的分配形式划分开来。}另外,他对转移成本加以抽象,
从而提出了评价可选择的分配方式后果的时期阶段问题。这些后果本身问题重重:一个联合
体的全部成员都得从把他们定义为被剥削者的选择方案中受益吗?可以认为剥削者的损失是
完全不相干的吗?\textbf{罗默乐意把经济人的偏好作为判断标准,这就要求对被剥削集团
  的成员满足帕累托标准。}对于判断任何现实的以及大部分假设是可行的再分配方式来
说,\textbf{这是在不充分范围内的一种片面的安排。同时,罗默又不愿漠视剥削者的财产
  权}——如果他们过去在道德上是清白的话,这又与整个剥削理论的关键之处相矛盾。罗默自
己把这一点看作剥削之所以是不公正的不完善指数的原因,这反过来又提出一个问题,即为
什么他继续相信马克思主义是道德理论的适当媒介?

\section{微观基础和马克思主义}

罗默的伦理观点并不是他的\textbf{个人主义}(这可能与马克思主义相对立)的唯一表现。在
他的著作和在艾尔斯特的著作中,与伦理观点同样鲜明的是方法论上的个人主
义。\textbf{这一方法要求以个人行为为依据来解释一切社会现象;整体的联合特征应被递
  次分解为经济人的个体选择;宏观现象应当存在着微观基础。}大部分马克思主义者都敌视
这些说法。例如,在20世纪初希法亭和庞巴维克之间的争论中,以及在50多年后阿尔都塞结
构主义者的责难中,表现得非常明显。然而,这种个人主义的方法深深植根于新古典主义经
济学中;自边际革命以来,这种方法的自负的理论家们一直为运用这一方法而取得的成果感
到自豪:构建了不同类型的决策者模型;显示了个人选择或这些选择的外生决定因素(偏好、
技术水平或禀赋)如何可能导致经济体系的运行。\textbf{20世纪60年代初以来,一些凯恩斯
  主义经济学家也日益认同这一方法,这部分原因是新古典主义理论家对宏观经济学结论的
  产生缺乏个体行为模型作了成功的批判。如果马克思主义也屈服于同样的攻击,那么它就
  可以在很体面的理论团体中占据一席之地。全部社会科学都使用同样的个人主义方法求得
  发展,这正是罗默和艾尔斯特所向往的。}那样,各种思想流派之间的分岐就会缩减为质的
分析上的分歧,而“交易工具”将成为共同的财产,使真正的对话得以产生。

虽然卢卡奇认为马克思主义的原创性和独特性在于它的方法,但是,很显然,上述现象的出
现并不会完全驱逐马克思主义。马克思和恩格斯本人曾经有过看起来是与方法论上的个人主
义相一致的论述。然而,\textbf{乔恩·艾尔斯特坚持认为,马克思在这一点上并非始终如一。
  在马克思的论述中,科学的解释有时为粗糙的功能主义和源自历史思辩哲学的目的论所代
  替。}

马克思以后的马克思主义者,如果与马克思稍有区别的话,在于他们更少受个人主义方法论
规则的约束。他们的某些观点,从艾尔斯特的社会科学观来看是有缺陷的。不过,揭示出其
他理论家在具体论点上的缺陷,并不等于确立了支持简化论的一般方法论立场。并且,理性
选择马克思主义提出的其他用以支持微观基础方法的论据也不是决定性的。

“\textbf{简化}”将是一个无休无止的过程,因为物质世界的任何因素都能够进一步分解。
\begin{quotation}
  任何在本质上都不是不可分割的事物,证明都有一个结构,是一个能够分解为构成因素的
  复杂体,这些构成因素本身是依据那些有待于发现的规则而相互联系着的。\textbf{发现
    一种最终不能进一步分解的简单实体,可能是某些科学领域的梦想,但是,对它们的探
    寻,例如在理论物理领域,是极其不成功的,}因为物理学发现的每一种简单实体,在更
  加深入的研究下,结果都证明是一个出乎意料的复杂体。旧时被认为是不可分割的原子,
  不可能被任何更小的被假想成物质“建筑材料”的东西所代替。
\end{quotation}

此外,“\textbf{简化”也未必能够完善对事物的解释或说明。把一种解释或说明的组成部
  分分解为更加基本的成分,事实上可能会削弱对事物的解释或说明。}
\begin{quotation}
  社会是个人的集合体,正如个人是细胞的集合体一样;社会现象是个人行为的结果,正如
  个人行为是构成个人的细胞的行为结果一样。然而,至少对于个人行为来说,本体论的可
  分性(不留任何残余的分解性)显然不能等同于解释或说明的可分性。对个人行为的最好解
  释或说明,本质上无需涉及细胞水平上的行为。

  在上述意义上,第二次世界大战仅仅是处于运动中的亚原子微粒的聚合。但是,对这种亚
  原子微粒的全部了解,根本无助于我们了解第二次世界大战的原因。我们可以合理地认为,
  即使战争能够完全用物理术语描述出来,这些描述在原则上不可能是这些现象的最好说
  明。
\end{quotation}

总体而言,\textbf{罗默总是在个体水平上停止分解},并且没有说明拒绝进一步分解的理
由。\textbf{艾尔斯特认为,罗默的这种截断是合适的,因为人类本质上是复杂的决策者,
  从而是以其他动物所不能运用的方式进行最理想的选择。}但是,艾尔斯特对罗默的这种维
护是\textbf{不中肯的}。撇开动物是否能尽可能完善地进行选择的问题,彻底的分解法意味
着:所有的\textbf{社会理论},包括与个人主义方法论的要求相一致的主张,\textbf{必须
  被看作是处于“次优”状态,只有在其被吸收进入生物科学领域之后才能得以完善。}但是,
正如前面的例子所揭示的,这些科学领域并不总是能够适合于所要求的解释类型;即使适
合,\textbf{进一步分解的可能性也是无限的},因为人类至今还未发现不可分解的“简单
体”的存在。\textbf{这说明,分解法并不是普遍适用的方法论规则,从而也就不能据此证
  明个人主义方法论是合理的。}

罗默和艾尔斯特\textbf{都没有始终如一地坚持个人主义的解释。}艾尔斯特认为,作为对所
研究问题的基本认定,非个人主义的因果解释是可接受的。他还注意到,社会具有
与\textbf{达尔文的自然选择(这种选择构造了个体决策)}相似的过程。罗默运用瓦尔拉斯的
经济理论说明马克思的目的,但\textbf{他没能指出,在他的公理式的系统阐述中,并未要
  求经济人应当是个体,而只是要求有定义明确的决策者,而这个决策者可以是集合体。}而
且,\textbf{尽管罗默遵循通常的瓦尔拉斯程序,集中探讨了均衡的结构,但却忽视了把均
  衡作为终极状态的决策和再决策过程。}此外,\textbf{非唯一性问题}也被一起忽视
了。\textbf{通常的情形是:特定均衡与个体行为的不同综合相一致,个体行为的任何一种
  综合又会达到许多各不相同的均衡。}

当然,罗默和艾尔斯特都可以声称,所有这些例证都\textbf{取自社会科学发展不充分}的情
况之下,因此他们对最好的解释是真正个人主义的解释的主张毫不怀疑。但是,这样一种立
场在其他情况下是站不住脚的。\textbf{只要存在经济再生产的先决条件,并且特定经济类
  型再生产自身有更具体的要求,就可以直接求助于这些条件。它们告诉我们经济现象之成
  为经济现象所可能发生的某些情况,我们无需为此考察个人的行为或动机。}相类似的是,
对具有某些现象的社会与不具有这些现象而其他方面与前一社会都极其相似的社会进行比较
的\textbf{历史比较法},提供了这些现象发生原因的信息——不论这种原因是否被个体行为的
调查所补充说明。这两种解释形式都存在着一个系统,\textbf{假如它们被微观基础的解释
  所代替,则这种系统就会消失:也就是说,如果用个人主义方法进行系统阐述,则它们就
  会更加复杂,也更不雅致。}而这种性质又不能为理性选择马克思主义者自己所贬低,因为
在他们的理论模型中,\textbf{个人在事先即是优秀的理性行为者,而经济的行为是全部理
  性行为的必然特征。}如果他们的这一看法是正确的,那它对社会科学家与其他决策者都同
样适用。

\textbf{乔恩·艾尔斯特}特别研究了个体理性的确切含义和特征,他的著作的许多部分
都\textbf{似是而非地显示了理性选择范例的局限。}他的看法有两个方面尤其值得注
意。\textbf{第一,个人被看作“社会”人;他们的偏好、可行的环境条件和信念受到社会
  条件的制约。第二,为了使选择完全理性化,他们必须遵循极其严格的、任何一个经济人
  未必可能完全达到的要求,在某些情形下,还会出现对理性的明显偏离。}因此,艾尔斯特
的观点比大部分新古典主义经济学家的观点要更加复杂精细得多,并且同“\textbf{边缘理
  性}”的理论家们,如赫伯特·西蒙的观点更为接近。

罗默也接受个人的社会性的观点,并同意标准的新古典主义选择模型存在着严重的局限性。
但是,他的看法远不及艾尔斯特的那样广泛而详尽。

\textbf{任何经济人的选择都可以用两种因素来说明:基于因果联系的偏好,以及确定选择
  得以做出的那些因果联系的可行条件。}艾尔斯特认为,这两种因素\textbf{可能不是相互
  独立的},每一种因素的内容会因社会地位不同的个人而不同。根据艾尔斯特的看法,经济
人基于偏好做出选择,并且这些偏好具有一贯性,还不足以使其被归类为理性。此
外,\textbf{还更多地需要选择结构中的每一因素,}都达到相应的理性程度。

\textbf{偏好很可能是适应性的;它们随着时间的推移、环境的变化以及经验的积累而改
  变。}因此,对个人来说,认识这一点并考虑一下他或她希望成为什么是理性的;这样,理
性的人就能有意识地确定自己的偏好。即便偏好不受环境和经验的影响,\textbf{那些“强
  烈地渴望得到他们不能得到的东西”的个人也“将不会感到幸福”,从而他们的这种愿望
  是“非理性的”。理性的愿望是那些“最令人满意地适应现实可行的条件”的愿望,因此,
  经济人有意识地改变自己的偏好也是理性的。}但是,艾尔斯特也认识到,同时起作用的还
有\textbf{非理性的力量},例如,无意识地缩减不协调的机制,也会使偏好发生改变。“愿
望癖好”在任何情况下都是理性选择的现实威胁因素。

正如艾尔斯特在对马克思的意识形态理论探讨中所认为的,关于什么是切实可行的环境条件
的看法可能也偏离了理性观念。\textbf{马克思主义在这方面的“有价值的核心”看法是认
  为“一种意识形态包含着从部分观点来看的对整体的理解”}。艾尔斯特本人探讨了导致阶
级成员把他们的特定利益与社会总体利益混淆起来,从而不适当地\textbf{把局部有效的关
  系推广到更广泛的范围之内的机制。}此外,偏好的形成和对切实可行的环境条件的认识,
都有赖于对信息的掌握。为了使自己的行为具有理性,经济人必须在任何情况下都具有适当
的信念。\textbf{由于个人不能够了解他或她没有掌握的信息的价值,因此,为信息收集提
  供一个理想的标准是不可能的。}

毫不奇怪的是,艾尔斯特认为,\textbf{个人行为可能会单凭经验而不是遵循严格的新古典
  主义规则。}由于这个原因,\textbf{或者由于社会条件的制约,新古典主义关于囚徒困境
  的说法在经验上是不正确的。}个人并不象所能预料的那样频繁地采取搭便车行为,他们在
个人选票的作用无足轻重的情况下,也仍然参加大的选区中的选举这类代价高昂的活动。

因此,\textbf{个人主义方法论的教条是不具说服力的,特别在增加了始终包含着理性选择
  要求的解释时更是如此。个人概念本身就是“历史的”;存在许多无需求助于选择的有效
  的解释类型;也存在许多理由,使人们确信个人不会总是理性地行为——尤其是在狭窄的新
  古典主义意义上。}这说明,理性选择马克思主义实际上比他们所宣称的离新古典主义经济
学要更远些,而更接近于结构主义。

\section{结论}

艾尔斯特和罗默认为,\textbf{个人都是社会的人,完全理性在他们看来可能是困难的、甚
  至是不可能的。}这些看法促使我们就理性选择马克思主义的激进观点的现实重要意义,作
出以下几点结论。

首先,尽管在罗默和艾尔斯特使用的模型中,\textbf{经济人的行为是新古典主义理论家们
  所假设的那样,}但是,他们的论述经常显示出,\textbf{他们认为这些模型过于粗糙和简
  单,}需要进一步完善。然而,他们又不总是一贯地坚持这一看法,他们所强调的以理性选
择为基础的马克思主义的重要性的主张,也没能反映出新古典主义方法的局限,以及他们本
人所承认的理性本身的限制条件。\textbf{这种认识对于斯拉法主义者拒绝把政治经济学建
  立在个人选择基础之上的做法是一种支持。}在他们看来,正如新古典主义理论所描述的那
样,\textbf{个人的选择是不稳固的,因为它们包含着对不可预知的未来的猜测;个人选择
  又是难以确定的,因为它们建立在不能评价其是否适当的主观假设的基础之上。}据此,斯
拉法主义者认为,新古典主义关于个人选择的看法,对任何试图系统阐述经济体系运行规律
的理论来说,都是不够可靠的基础;严密的社会科学,包括马克思主义在内,都应该
是\textbf{结构主义}的。对于这一看法,不论是罗默还是艾尔斯特,都没能作出有力的反驳。
而且,罗默对马克思经济学的许多批判,例如他对置盐信雄理论的支持(参见以上第7章),与
斯拉法主义者对马克思经济学的批判惊人地相似。这表明,结构主义和微观基础方法不一定
是对立的。事实上,罗默和艾尔斯特的微观基础理论,与结构主义经济学的大部分内容显然
是不相矛盾的。

罗默和艾尔斯特用以描述他们的方法论的\textbf{术语是不恰当}的,\textbf{因为他们确实
  把个人看作具有社会性的人。}这样做的结果是,他们没有把整体的特征分解为个人选择的
结果,因为这些特征被保留在进行选择的个体的素质之中。\textbf{除了宏观行为的微观基
  础,还存在着微观行为的宏观基础,任何一个人都不能被视作与另一个人相对立。}严密的
结构主义者的立场是:他们对“个人”的概念并无敌意,他们反对的只是那些不准确地
对“个人”予以定义的观点。\textbf{把新古典主义准确地描述为方法论上的个人主义,这
  是正确的;}因为它名副其实地把对经济现象的解释分解为对经济人决策的解
释;\textbf{在对微观基础的探求中,它的正统理论否定了其对宏观基础的需要。}确实如此,
在新古典主义经济学中,个人的偏好和制约条件被视为外生因素。但是,正如我们已经了解
到的,不论是罗默还是艾尔斯特都认为新古典主义的观点不具有说服力。他们使用建立在新
古典主义原则之上的理论模型,在此基础上发表关于马克思主义的看法,但是他们并不认为
这些原则是正确的。

\textbf{把选择和结构对立起来造成了一种错误的对分法,这种分法只有在把理性选择马克
  思主义归于他们并不赞成的结构主义立场时,才是令人信服的。}罗默和艾尔斯特都坚持认
为结构主义的标志在于:认为个人选择的限制条件严重地制约着切实可行的环境条件,以致
选择要么被彻底消除,要么由于受到压制而显得无关紧要。这是对结构主义观点的一种漫画
式的描述,它强调由于系统特征的运作建立在偏好和制约条件的基础上,因此个人的决策是
由\textbf{系统特征}“决定”的。这一看法事实上非常接近于艾尔斯特和罗默的观点。他们
并不强调(或者确信这种强调并不重要)在个人选择中存在着唯意志论的巨大因
素。\textbf{至少是更温和的理性选择马克思主义者如科亨和布伦纳,采取了类似于结构主
  义的决定论立场。}事实上,艾尔斯特和科亨之所以会就功能解释发生争论(参见以上
第11章第4节),是因为\textbf{艾尔斯特}不仅坚持认为马克思的解释应当以个人行为为依据
进行系统阐述,而且更强硬地认为,\textbf{没有什么关于社会现象的描述可以被归类为一
  种解释——除非它实际上提供了一种把个人主义的解释要素与待解释的社会现象联系起来
  的机制。科亨正确地抵制了这一说法,}他认为,尽管必须存在一种机制,但\textbf{一种
  解释不必起源于这种机制,}因此,艾尔斯特的这一观点不符合建立在科技哲学家解释之上
的逻辑规则,并且也不是艾尔斯特本人一贯所持的立场。然而,科亨并不否认,功能解释
的“详尽阐述”是合乎需要的,马克思主义者有时没有意识到这一点,或者,他本人所作的
详尽阐述是不完善的,需要进一步加强。

非个人主义解释的详尽阐述,无需限定在对有意识行为的看法的系统阐述上,甚至个人主义
的解释,也无需限定在理性选择的范围内。从马克思主义观点来看,\textbf{艾尔斯特}对理
性的研究可以看作是关于理性的能力、激励和压力如何随着历史的发展而改变,以及如何随
具体环境的变化而变化的概述。就此而言,他的著作是非常有启发的。例如,\textbf{他对
  布伦纳关于前资本主义社会的静态特征源自理性经济人的决策(参见以上第11章)的观点提
  出质疑。惯性行为模式可能会产生与理性选择的结果相一致的结果;但是,既然这些行为
  模式是习惯性的,它们在不断变化的环境中就仍然保持惯性;如果选择确实是理性的,就
  无所谓变化的环境。}这样,我们就可以更好地理解封建社会的经济人在可以获得效率更高
的技术时,为什么不迅速调整其生产行为。对艾尔斯特著作的这种理解,还可以解释为什么
科亨在借助于理性时,并不清楚明确地借助于理性选择(参见以上第11章)。\textbf{科亨只
  认为存在着理性行为的趋势,而不认为行为总是理性的,或者理性趋势总是明显地达到同
  一程度。}

\textbf{把理性行为看作强度不定的历史力量的认识,也是看待新古典主义理论的一个适当
  的角度。}当联系到\textbf{偏好和制约条件都是历史的}这一事实时,它意味着新古典主
义经济学能够为马克思主义提供的主要服务是作为一种\textbf{反事实的推理模式}。这一方
法的推理过程是:\textbf{首先构造一种模式,在这种模式中,人们竭力去解释的存在着的
  某种现象,事实上并没有发生;接着,把模式中的假设条件与看来实际上在现实中适用的
  条件进行比较,就可以得到是什么在事实上导致了所讨论的现象发生的线索。}因此,新古
典主义的模式可以在不被相信的情况下有效地使用。如果这是一个正确的论断,那么从反事
实角度理解时,罗默运用新古典主义理论解决阶级、剥削和财产权问题就是有强大说服力
的。

从马克思主义史的视角来看,罗默的分析实质上比其方法更不具有原创性。但
是,\textbf{罗默的方法实质上是新古典主义经济学的方法。}因此,从经济思想史的整体发
展来看,其新颖独特之处在于把\textbf{不同传统的因素联系在一起}:非新古典主义的问题
和新古典主义的方法。这是正统经济学范围内的关于广义“帝国主义”发展的一部分,其中
使用了既有的新古典主义分析方法来分析传统上被认为是新古典主义理论范围之外的领域。
可能正是从这一优点而不是从马克思主义的观点来看,罗默的著作受到高度评价。但是,这
一评判尚需有一个重要方面得到说明。过去100年来,马克思主义政治经济学产生了大量观点
和不同的分析形式。除了一些明显的例外,它们的一个共同的特点是对经济模型的严密的系
统阐述持漫不经心的态度。这并不是任何理性选择马克思主义者都可受到责备之处,至少罗
默本人以及理性选择理论的精密性,可以证明是对马克思主义的有益影响。尤其是我们下一
章将要探讨的社会主义政治经济学,在今天尤其需要一些严密的思维习惯。



\chapter{社会主义政治经济学}

\section{引言}

在马克思主义理论史上,\textbf{社会主义政治经济学}由于人们对资本主义生产方式所给予
的过多关注而相形见绌,只有后资本主义经济结构问题作为一段小插曲,走到了理论关注的
前沿。即便如此,\textbf{它也常常是在非常具体的环境条件下出现的,而这种具体的环境
  条件又制约着可能得出的结论的普遍性。}本章第2节和第3节将解释这种对社会主义政治经
济学的忽视是如何内在地存在于马克思原创观点中的。接着的一节将揭示:为什么
自20世纪20年代以来,尤其是20世纪70年代以来,社会主义政治经济学被提高到一个较为显
著的位置。正当“实际存在的社会主义”危机的深度开始明显之时,1989年的剧变预示
着20世纪90年代,这一主题将成为马克思主义政治经济学的主导。本书最后一章对社会主义
政治经济学所作的单独探讨,恰如其分地反映了马克思主义政治经济学自身的历史。

我们进行探讨的时间跨度很大。从下一节探讨马克思和恩格斯的观点开始,接着在第3节论述
第二国际著名理论家的观点。对苏联早期实践的反思构成第4节的主题。第5节则探讨两次大
战期间在西方经济学家中进行的著名的“社会主义计算的争论”。在马克思主义和非马克思
主义经济学中,主要作为战后现象的斯大林主义和非斯大林化的政治经济学,则在第6节进行
概述。这为20世纪80年代进行的关于社会主义可行性的讨论提供了背景,本章第7节将对此进
行论述。

不过,在开始探讨马克思和恩格斯的观点之前,我们应当了解一个事实,即\textbf{关于后
  资本主义社会的大部分讨论都是不确定的。}首先,理论家们都慎重对待设想中的社会主义
的存在环境问题,其观点相去甚远。许多理论家绝对地把社会主义构想成一个\textbf{孤立
  的系统},而其他理论家则承认\textbf{社会主义将与资本主义共同存在,并不得不与它发
  生联系并使自己适应于它。}但是,马克思主义者有时不清楚他们的观点是否与过渡时期有
关,或者它们是否也扩展到了一种已经稳定的新的社会结构内。

第二,效率问题没有得到充分的讨论。马克思主义者一贯认为,社会主义经济结构比资本主
义经济结构更能有效地满足人们的需要。但是,他们在对其含义进行详细说明时却遇到了困
难。\textbf{依据唯物史观,人的禀性——意识,兴趣,抱负和能力——随社会关系的变化
  而变化,因此,社会主义社会中人的需要也将不同。但是,准确地确定它们如何不同,必
  然是一种理论上的推测活动,从而对社会主义经济制度效率的任何评定也是理论上的一种
  推测。}当假设\textbf{后资本主义社会}在很长时期内仍可能保持发达资本主义通行的动
力和需求结构时,这个问题就好解决得多了。然而,尽管许多马克思主义者都确信这是着手
研究社会主义经济的最合理的假设,但是,他们常常不能面对这一假设的言外之意,
即\textbf{社会主义只有在满足同样的需求方面优越于资本主义,而不仅仅是由于它可以更
  好地适应于满足新的需求才能赢得支持。}因此,对现代资本主义——而不是马克思时代的
资本主义——为什么可能失败进行正确的研究,是一个迫切的问题。总体而言,马克思主义
者一直倾向于回避这一问题,他们一再重复传统的论据,诸如资本主义制度下发生危机的倾
向、对资源的挥霍和浪费以及非人的残酷条件,而迄今为止,这些论据并没有使发达国家中
的更多人相信资本主义需要被新的制度所代替。

第三,\textbf{社会主义者传统上认为,一旦后资本主义社会得以建立,压迫就会显著减少,
  甚至可能彻底消亡。这一信念得以建立的基础是假设所有人的需要将会在社会主义条件下
  得到更好的满足,因此,它有赖于社会主义的效率。}但是,这一看法并没有充分地认识
到,\textbf{在非压制关系居统治地位的情况下,效率必须在微观结构中得以普及,而不只
  限于宏观经济。}否则,“得到准许的成年人之间的资本主义行为”就有重新出现的趋势,
社会主义将从内部受到威胁。因此,\textbf{在提高自由程度仍然是社会主义事业的主要促
  动因素的范围内,为社会主义提供蓝图的任何企图都必须对这一问题特别敏感。}然而,几
乎所有正式的社会主义经济模式都忽略了这一点。例如,20世纪30年代始于奥斯卡·兰格的著
名蓝图(参见本章第5节),以及市场社会主义的大部分现代观点的妙想,都没有试图研究这一
问题。\textbf{兰格只是想当然地认为,把生产资料的私有权排除在外的“社会主义法
  则”将得到坚持。}

第四,\textbf{“实际存在的社会主义”显然不是富有效率的,也不是非高压统治的,它们
  的历史在多大程度上与实现真正的社会主义相关,是一个难以确定的问题。}可以认为,在
马克思主义传统内,苏联和东欧的所谓“社会主义”的明显失败,与真正获得解放的社会的
可行性问题并没有什么联系。\textbf{马克思本人}在他的大部分理论研究生涯
中,\textbf{一直坚持认为社会主义既需要生产力的高度发展,也需要广大的工人阶级群众
  有意识的支持和参与。1917年落后的俄国则不具备上述任何一项条件,}当时,俄
国\textbf{无产阶级的力量很微弱,并且很快就屈从于布尔什维克的专政。}第二次世界大战
结束时,在红军坦克的炮筒之下输出的苏联模式,也不能构成对马克思的社会主义纲领可行
性的公正检验。正如我们在前面第3章所了解到的,几个持不同见解的马克思主义派别广泛地
关注苏联生产模式,它或者被视为阶级社会的一种新形式,或被看作过渡社会的一种歪曲的
反常类型,其崩溃也被广泛地预测到。

同时,\textbf{如果仅仅因为社会主义产生的经济困难几乎必然地被任何社会主义实验所经
  历,而不论进行这一实验的环境条件多么有利,就由此坚持认为斯大林主义的实践与社会
  主义是不相干的,这种看法是错误的。}社会主义的批判者认为,就社会主义本质而言,任
何把社会主义付诸实践的尝试都必定导致效率低下,因而只能通过高压统治来维持。然
而,\textbf{苏联模式社会主义产生的历史性的危机说明,必须谨慎地对待现实的社会主义
  实践。}这在\textbf{两方面}都说得通。就粗略的增长率而言,\textbf{苏联
  在20世纪60年代之前的迅速发展,很大程度上依赖于从发达资本主义国家引入技术,其他
  社会主义国家不大可能模仿这一点。另一方面,面对虎视眈眈、掠夺成性的资本主义,在
  军事上赢得均势的迫切需要,以可能的方式影响着苏联的社会-经济结构(参见以上第2章和
  第3章)。}

上述四点说明了一个问题,即资本主义政治经济学和社会主义政治经济学需要密切地联系起
来。\textbf{正是由于社会主义被看作是一种后资本主义的社会结构,在这种社会结构中,
  人们在资本主义社会所获得的自由和解放将得到进一步扩展,因此,社会主义经济学只有
  建立在对资本主义进行彻底了解的基础上才有说服力。}本书前面的章节已经指出,在这方
面,马克思主义可能存在着\textbf{真正的局限}。马克思主义经济学家在分析资本主义时频
频出现错误,这种错误必定会渗入他们的社会主义概念之中。不过,在1929年之前,资本主
义和社会主义的密切联系确实得到过关注。特别是马克思和恩格斯的本质分析法尽管存在着
严重缺陷,但他们的方法论依然为如何构建社会主义经济学提供了一种模式。

\section{马克思和恩格斯论后资本主义社会}

社会主义的理想在马克思主义出现之前就存在了很久。马克思主义者对早先的关于如何组织
社会主义的观点提出的批评已被发现是错误的,这特别是由于他们没能就自己关于后资本主
义社会的制度安排的观点提供更深入的解释。当然,马克思和恩格斯在对他们之前的社会主
义蓝图大肆奚落时,并未感到有什么限制,但他们自己却不愿详细说明他们关于未来社会结
构的看法。并且,当他们确实这么做时,其说明通常是对当时发生的事件的一种反应,而不
是就主题本身进行全面的讨论。事实上,这种立场是他们理论性质的逻辑上的必然结
果。\textbf{马克思和恩格斯对社会主义传统思想的首要贡献是一种科学的观点,从这种科
  学观点来看,社会主义因素作为资本主义自身发展的结果产生于资本主义内部。他们认为,
  这一点使得离开现存社会可能发生的事情去系统阐述社会主义理想的企图受到限制。}“科
学社会主义”的大胆预言,总是无不震惊朋友和敌人,但在这种震惊背后的却是\textbf{世
  俗的假定,即任何社会主义方案的价值都有赖于它的实践。}

因此,我们可以相对容易地确定马克思和恩格斯所认为的后资本主义经济的主要特征。这些
特征在逻辑上来自于他们所认为的日益制约着资本主义充分满足人们需要的那些因素。这里
的\textbf{核心概念是异化。后资本主义社会最终将是一个不存在异化的社会:人类劳动的
  产品将不再统治它们的创造者,而是处于人类自觉的控制之下。马克思和恩格斯认为,这
  一特征要求消灭商品生产和生产资料私有制。因此,为了达到这些目的,未来社会的所有
  形式从一开始就要削减市场活动。}

此外,社会主义解决资本主义没有能力满足人们需要的方案,也直接来自于异化。在马克思
看来,异化的条件与历史唯物主义的领域是统一的。他认为,\textbf{在资本主义条件下,
  生产力和生产关系的矛盾附着于这样一个事实,即生产力日益要求新的社会主义关系的产
  生。}《资本论》中论述的所有具体的“运动规律”——消费不足的危机,利润率下降以及不
断扩大的失业后备军——都源自于此。由于这些“规律”产生的严峻后果影响着工人阶级的生
存条件,因此,\textbf{马克思和恩格斯把无产阶级看作是推翻资本主义的主要力量。由于
  这个原因,他们认为,尽管社会主义自身将普遍地提高人们的生活水平,但是,后资本主
  义社会首先将采取工人阶级国家的形式。}

在马克思和恩格斯看来,无产阶级的革命活动将导致人类的普遍解放,因为作为革命基础的
生产力和生产关系之间的矛盾,同样也在朝着社会主义方向修正着资本主义生产关系。因此,
无产阶级只是使资本主义内部目前已经萌芽的事物成为现实。这里,\textbf{至关重要}的是
马克思和恩格斯的这一看法:\textbf{资本主义的发展包含着生产资料积聚和集中的显著趋
  势,这种趋势会缩减商品关系的范围,从而为不存在交换的使用价值生产提供基础。正是
  这一关键点,使马克思和恩格斯确信,社会主义经济将是有计划的。}

从这一历史唯物主义观点出发,马克思和恩格斯批评空想社会主义者把理想与现实发展分离
开来、并经常与现实发展相悖,是有其道理的。而且唯物史观也解除了他们为未来社会提供
详细蓝图的义务。
\begin{quotation}
  无论哪一种社会形态,在它们所能容纳的全部生产力发挥出来以前,是决不会灭亡的;而
  新的更高的生产关系,在它存在的物质条件在旧社会的框架里成熟以前,是决不会出现的。
  所以人类始终只提出自己能够解决的任务,因为只要仔细考察就可以发现,任务本身,只
  有在解决它的物质条件已经存在或者至少是在形成过程中的时候,才会产生。(马恩文
  集,第2卷,P592,《政治经济学批判 序言》)
\end{quotation}

据此得出的必然结论是:\textbf{对后资本主义社会进行描述可望达到的详细程度,取决于
  资本主义发展的成熟程度。}因此,指责马克思和恩格斯本身是空想社会主义者是完全不正
确的,由此认为历史唯物主义可以与乌托邦主义的新形式相结合则是误入歧途。马克思和恩
格斯关于资本主义发展的看法当然是错误的,从而他们对任何社会主义的未来形式的看法当
然也是错误的。但是,他们解决这一问题的方式依然是合理的。

然而,\textbf{他们错误地认为,资本主义自身具有的发生危机的趋势,对于说明社会主义
  的优越性至关重要。即便他们已经为其关于资本主义经济危机将会日趋严重的观点提供了
  一个可信的基础,这也不能说明社会主义更可取,}或者说明社会主义是广大工人阶级更愿
选择的一种制度。一个浪费的和低效的资本主义制度(资源在这种制度下令人羞耻地得不到充
分利用)可能仍然比任何其它可行的选择都明显地要更好些。而且,\textbf{即便资本主义制
  度下利润率下降、失业上升和消费不足变得更加严重,也不能排除资本主义关系紧接着反
  资本主义革命之后,在微观层次上重新建立起来。}个人就如何最有效地生产出特定商品进
行计算时,可能不会把资本主义生产方式的宏观经济后果考虑在内。因此\textbf{,社会主
  义经济结构只有在每一个生产领域都具有更高的效率,才能保证后资本主义社会不会倒退
  回资本主义。}

这些看法表明,马克思和恩格斯用以论证他们赞同社会主义的另外两个观点的重要
性。\textbf{第一,资本主义自身通过不断扩大的资本积聚和集中,使生产关系不断社会化。
  第二,商品生产的消除越来越与人们深切感受到的(如果只是不完全明确地表达的话)自我
  实现和克服异化的需要相一致。}但是,这两个观点都不易得到证明。马克思和恩格斯关于
资本主义将逐步消除全部商品关系的预测是错误的。尽管他们关于资本积聚和集中趋势的看
法是正确的,但是,这并不能对在非商品原则基础上组织经济生活的观点予以解释和说明。
各个大型生产单位之间居于统治地位的关系仍然是货币关系;各个公司的内部结构是对外部
竞争的一种适应;这些结构通常建立在准市场原则基础之上。并且除了像保健供应这样的特
定领域,人们在满足其消费需求方面也几乎没有显示出对市场的不满。然而,在马克思政治
经济学史上,理论家们却对马克思关于资本集中的观点、以及他对市场关系敌视的观点上显
示出固执的忠诚。这在任何方面都不比第二国际时代更正确。

\section{社会主义和第二国际}

马克思的科学社会主义是与他对\textbf{革命政治的巨大热情}联系在一起的。因此,在他的
著作中,\textbf{一方面是对社会主义取得成功所要求的条件进行慎重判断,另一方面是对
  资本主义的焦躁,二者呈对立“拉张”之势(参见本书第一卷第7章)。}马克思有时也对历
史赋予超越科学判断的“含义”,认为后资本主义社会代表着社会哲学的全部理论问题得到
真正的解决,并由此构成人类对自身探索的终结。但是,德国马克思主义者并不很认同马克
思科学观的这两个限制条件。他们中的所有马克思主义著述者大多强调马克思著作的科学方
面,许多人认为社会主义是一个“实际的”事情。在这里,马克思成熟著作的过度影响是显
而易见的,同样显而易的是马克思主义变成了工人阶级政党被迫为“面包和黄油”利益服务
的理论(参见本书第一卷第1章和第4章)。俄国马克思主义则不同。面对沙皇专制统治造成的
落后状况,它不得不以更剧烈的方式提出革命问题。但是,俄国的杰出理论家们同德国马克
思主义者一样,并不违背科学社会主义的宗旨和原则。就社会主义经济学而言,甚
至\textbf{在1914年第二国际解散之后,列宁和布哈林仍旧是希法亭的信徒,而布尔什维克
  的先锋主义,正如列宁在《怎么办》一书中所指出的,是在考茨基的著作中找到根基的。}

\textbf{考茨基坚持认为,社会主义需要资本主义的充分发展,并通过现存制度内议会民主
  力量的增长得以实现。}与许多现代马克思主义者的评价不同,在第一次世界大战之前,考
茨基的这种观点通常被他的同时代人认为是合理的。确实,他关于通过议会民主下的激进改
革实现社会主义的观点,在当时几乎没有遇到批评。(\textbf{考茨基认为,暴力革命可能是
  需要的,但它仅仅是对付抵抗历史必然性的保守力量的一种防卫措施。})普列汉诺夫关于
资产阶级民主制度下的恰当策略的观点,在本质上与此相同。许多孟什维克竭力主张仿
效1905年革命后的德国政党制度。\textbf{列宁}对此持反对态度,但是,他\textbf{反对的
  理由主要集中在俄国缺乏可靠的政体条件,}而不是1914年之后逐步形成,这出于对考茨基
主义的反对。

在社会主义经济结构方面,考茨基显然忠实于马克思和恩格斯的看法,认为在未来的社会主
义社会中,社会所有权居于支配地位,资源分配是有计划的,权力机构完全的民主。考茨基
并没有试图对确切的制度结构进行详细说明,出于同样的原因,马克思也没有写出“未来的
食谱”:\textbf{详细的方案不能够先于历史发展提出来。}然而,\textbf{考茨基承认,资
  本主义的许多因素在工人阶级掌握政权之后的一段时间内可能会继续存在。}马克思已经承
认过这一点,但考茨基的论述要更为深入,也更加明确。这可能反映了修正主义论战的影
响(参见本书第一卷第4章)。但是,考茨基只是笼统地表明了自己的观点。尽管在未来的社会
主义社会中,将有一个扩大得多的政府供应来满足工人阶级的住房、教育和社会保障需要,
但是,生产资料私有制(尤其在农业和智力活动中)仍将继续存在;需要以不平等来维持对工
作的激励,也不会立即停止。此外,考茨基还认识到:
\begin{quotation}
  迄今为止,货币是最简单的手段,它使现代生产力这样复杂的机制可能存在……保证商品
  的流通……\textbf{货币手段使每一个人根据个人偏好来满足自己的需要成为可能……作
    为流通的手段,货币在更好的替代品被发现之前是必不可少的。}
\end{quotation}

在考茨基的著作中,资本主义的进步要求生产资料不断进行集中的看法依然非常明显。因此,
考茨基从未明确地设想过一种成熟的社会主义经济会不需要充分的社会化的和计划。随后的
社会主义理论和实践的发展进一步论证了这一预见,最显著的是希法亭的《金融资本》和第
一次世界大战的历程。\textbf{希法亭的这部著作说明了国家资本主义经济的有组织的性质,
  他明确地把这一点与社会主义的实现联系起来。}
\begin{quotation}
  金融资本把社会生产的支配权越来越集中到少数最大的资本集团手中,使生产的经营权同
  所有权相分离,生产社会化达到资本主义范围内所能达到的限界。……\textbf{金融资本
    的趋势是建立对生产的社会控制……(这)……使战胜资本主义变得非常容易。…… 社会
    通过其自觉的执行机关——被无产阶级夺取的国家——占有金融资本,就足以立即获得
    控制权……}剥夺根本不必延及大量农场和小企业,因为通过对农场和小企业长期依赖的
  大型产业的剥夺,这些农场和小企业就间接社会化了。因此,这也使剥夺过程缓慢地达到
  成熟成为可能。
\end{quotation}
《金融资本》对马克思主义经济理论产生了非常重大的影响,它强化了这样一个观念,即社
会主义植根于“真正的发展”之中,只有“真正的发展”才使社会主义成为必然的未来(参见
本书第一卷第5章)。

对这一观点的进一步支持,出现在1914年之后的德国战时经济结构中。它对\textbf{布哈林
  和列宁}产生的显著影响,在本书第一卷第13章中已经论及。他们的论述中有许多新颖独到
的因素,以下我们将对此作出评论。但是,他们为捍卫关于社会主义将完全地延续现代资本
主义有组织的特征的观点所作的努力,得到广泛认同。计划的实践也促使奥地利社会主义者
和一位博学者\textbf{奥托·纽拉斯提出了社会主义经济的第一个马克思主义蓝
  图。}在1916年之后的一系列文章中,他勾划出了一个\textbf{社会化、有计划和无货币居
  支配地位的经济可能实际运行的轮廓:}
\begin{quotation}
  [通过计划和]……普遍的统计……[经济的]总体结构……能够提高……经济效率。经济计
  划将由一个把整个国民经济看作一个巨大企业的特殊机关制订。货币价格将不再重
  要……因为在计划经济的框架内,这些价格——只要它们还继续存在——是由一种基本上
  是主观的方式决定的……我们最终必须把自己从过时的偏见中解脱出来,把大规模的实物
  经济看作一种充分有效的经济形式。
\end{quotation}

纽拉斯的著作,以及同时出现的俄国革命后的战时共产主义现象,促使\textbf{路德维
  希·冯·米塞斯对社会主义将富有效率的观点进行驳斥(参见以下第5节)。}但是,在此之前,
奥地利学派的其他成员就已奠定了基础性工作。在19世纪的最后几十年中,作为对马克思价
值理论批判的一部分,\textbf{维塞尔和庞巴维克认为,边际主义理论的各种类型具有普遍
  的意义,且不受历史上具体的资本主义经济结构形式的限制。相对价格是指在稀缺条件下
  反映机会成本的替代率,利润则代表“迂回”生产过程的额外生产力,同时保持当前消费
  对未来消费的时间偏好。因此,现代经济的计划者们不能够省去这些范畴而希望保持效率。}

这一观点在1908年由\textbf{恩里克·巴罗纳}进一步加以论证。他从\textbf{瓦尔拉斯的视
  角}出发进行论证,并受到帕累托福利经济学的巨大影响。他证明,\textbf{资源的有效配
  置所要求的条件,在任何经济社会都是一样的,因此,它与特定制度结构无关:“……集
  体主义均衡的方程组正是自由竞争均衡的方程组。”}巴罗纳认为,一个社会化的、集中计
划的和取消货币的经济,在原则上可以是有效率的;但是,\textbf{最优化条件的复杂性使
  他相信,不利用市场体系,这些条件在实际上是不可能实现的。}社会主义的\textbf{看得
  见的手}注定要明显劣于“生产的无政府状态”。在1920年米塞斯发表其见解之前,没有任
何形式的“庸俗经济学”对马克思主义关于社会主义经济学的讨论产生过影响。然而,来自
社会民主派内部的批评的确开始削弱正统观念的“铜墙铁壁”。19世纪90年代,德国和俄国
的修正主义者都对马克思描述的资本主义发展的“运动规律”包括资本集中提出了质疑。他
们还从奥地利学派的分析角度,对马克思的价值理论进行了批判。但是,对于已被接受的社
会主义经济结构和现代资本主义关系的看法,主要的修正主义者并没有明确地提出质疑。他
们更关注于改变政治策略和为社会主义伦理观提供一个可供选择的基础(参见本书第一卷
第4章和第10章)。

马克思主义关于社会主义的理论受到的震动,还在于\textbf{1905年和1917年革命中工人委
  员会}的成立,并在第一次世界大战中遍及大部分欧洲国家,\textbf{最后遭夭折}。这使
得自由主义者重新恢复了对无产阶级专政的解释,恢复了对迄今为止实际上一直为无政府主
义者和工团主义者所垄断的公社国家的解释。\textbf{社会主义革命包含对现有国家机器取
  得民主控制的正统观点,现在受到了来自马克思主义左派的攻击。}他们认为,必须“打
碎”已有的国家机器,要在与以往不同的原则基础上建立新的政权。尽管这些观点很大程度
上建立在马克思和恩格斯对巴黎公社的看法之上,因而能够得到马克思的信任状,并且没有
被马克思主义者用作质疑那些已接受的社会主义经济学观点,但是,它们同社会主义形成于
发达资本主义结构的科学假设这一总体观点是不统一的。\textbf{如果有组织的资本主义是
  社会主义经济的先兆,历史唯物主义就将指出中央集权的、官僚体制的国家将是不可避免
  的,如马克思、恩格斯、考茨基和希法亭有时承认的那样。}列宁和布哈林都反对这些说法,
但是他们的解释都不具说服力。布尔什维克革命以后,他们被迫修正了自己的观点(参见本书
第一卷第6章、第13章和15章)。

\section{对斯大林主义之前的苏联实践的反思}

1917年以后,西方大多数社会民主主义者仍然忠实于社会主义经济的正统观点。发达的资本
主义日益成为“有组织的资本主义”,并提供了未来社会所必需的经济基础。主要的变化将
涉及那些已经明显扩展的事物,并将由于民主的扩展而成为现实,这将只能导致无产阶级的
政治统治(参见本书第一卷第14章)。\textbf{无怪乎许多人把十月革命看作布朗基式的乌托
  邦主义,认为它在极端落后的状况面前注定要失败。}考茨基认为,布尔什维克的恐怖主义
方式反映了这一矛盾,而\textbf{缺乏民主制的结果则是国家资本主义而不是社会主义的产
  生。}这与他以前的观点是一致的,因为他一贯把缺乏充分民主条件下日益强大的国家控制
和国有化,看作完全是在资本主义制度内的发展。列宁和托洛茨基认为,第一次世界大战揭
开了革命的新纪元,并以此为基础强烈遣责考茨基。但是,考茨基和大部分其他西方社会主
义者并不接受这一点(参见本书第一卷第5章、第12章、第13章和第15章,以及以上第3章)。

\textbf{布尔什维克左派、斯巴达克同盟的成员如罗莎·卢森堡和议会共产主义者,从类似的
  但更为赞同的立场驳斥了党的专政理论。}他们赞成列宁和布哈林在第一次世界大战期间所
拥护的公社国家制度。但是,他们都没能为后资本主义经济如何以这样的方式实际运行提出
令人信服的说明。因此,马克思主义对苏联早期实践的所有的批评,都强调政治权力的上层
建筑的重要性。\textbf{路德维希·冯·米塞斯——这是一位毕生坚决反对各种形式社会主义
  的人——把目标对准了一个更为核心的问题。}他于1920年提出的最初观点结构松散,其主
题现在已有不同的解释。然而,他进行批评的形式很易于理解。\textbf{米塞斯认为,没有
  私有产权就没有交换;没有交换,就不能对不同资源进行合理估价,也就不能够经济地使
  用这些资源,从而只能导致混乱或无秩序状态。其选择(用罗莎·卢森堡自己的警句来反驳
  她)要么是资本主义,要么是野蛮状态。}
\begin{quotation}
  如果社会主义思想的支配地位维持不变,那么,在短时间内,欧洲几千年建立起来的整个
  共同的文化体系就会动摇……所有实现社会主义的努力都只会导致社会的毁
  灭……[和]……退回到封闭的自给自足的家庭经济……陷入……满足自己需要的勉强糊口
  的生产。
\end{quotation}
这一论断是对战时共产主义实践的反思,而不应被视为米塞斯观点的完整形式。米塞斯集中
对纽拉斯和布哈林概述的\textbf{计划化、社会化及取消货币}的经济形式的效率问题提出了
质疑。

随着\textbf{新经济政策}的实行,米塞斯的并不算极端的论述得到了明显的确认,
而\textbf{许多马克思主义者把新经济政策说成是迈出了复辟资本主义的第一步。}然而,从
理论观点来看,\textbf{更有意义的是一些苏联经济学家得出的结论。1920年期间进行了一
  次大规模的讨论,主题是取消货币的经济和最近建议(包括纽拉斯的建议)实施的脱离市场
  进行计划的方案存在的种种问题。}新古典主义理论家们揭示的困难问题得到了\textbf{承
  认,特别是关于没有货币作为总的成本和受益的共同标准,合理的计算在实际上就是不可
  能的问题。}与米塞斯不同的是,人们接受的是,\textbf{取消货币的经济可能运转得不错,
  但在实现计划者的目标和满足消费的需求方面不会有效率。}布尔什维克理论家们是否开始
接受这一观点不得而知。20世纪20年代,党内各派都接受了新经济政策,但是,没有确切的
证据表明,布尔什维克党的领导人都认为,社会主义经济将成为传统设想之外的什么事
物。\textbf{对新经济政策的支持,是把它作为一种过渡模式,而不是作为达到革命目标的
  一种社会结构(参见本书第一卷第15章,以及以上第2章)。}

\textbf{即便20世纪70年代初以来布哈林的观点成为那时试图重构苏联生产方式的一种模式
  赢得了很高声望,}也需要恰当地认识到,它们\textbf{至多只是略微触及了主题。}相反,
尽管托洛茨基在内战中表现出的革命狂热和冷酷无情,使他成为大部分改革者不大喜欢的人
物,但他关于社会主义政治经济学的观点要敏锐得多。布哈林用以论证他对新经济政策看法
的消费不足经济学,以及他对自给自足政策的选择,是完全错误的;他对斯大林主义的批评,
也只是在20世纪20年代末才提出的,而且使用伊索寓言式的语言表述。托洛茨基正确地提出
了存在的主要问题,他的认识与传统马克思主义的前提要贴近得多。
\begin{quotation}
  势不两立、相互对抗的两种社会体系——资本主义和社会主义——的长期斗争,其结果将
  最终\textbf{取决于每一种体系中的相对劳动生产力水平}……只有与\textbf{世界经
    济}的现有技术水平、生产成本、产品质量和价格不断地接近或同步……才能捍卫目前正
  在建设的社会主义经济。
\end{quotation}
并且,托洛茨基明确地和再三地承认,市场关系和货币价格在纠正计划产生的失误和判断计
划成败上的作用。他提出所有这些论据,以一种现在看来是预言的方式,揭露斯大林指令性
经济的不合理性及其缺陷。\textbf{托洛茨基批判中存在的真正问题,就是它准确地描述了
  已被证明是指令性经济的体制性问题,因而似乎非常接近于支持、至少部分地支持米塞斯
  的观点。}

\section{兰格对米塞斯的回答}

两次世界大战之间的整个时期,西方社会主义者为避开米塞斯对社会主义的攻击作了各种尝
试。有的重申资本主义正迅速地超越市场结构的正统观念;或指出中央计划当局解决像巴罗
纳所详细说明的效率分配等式问题,是有可能性的;他们偶尔也象纽拉斯一直在做的那样,
对完全有计划的、社会化的和取消货币化的经济可能如何运行作出概述。其他一些人更加激
进。\textbf{他们承认米塞斯揭示了一个真正的问题,设想社会主义所有权可以与商品生产
  相联系;这样,单是公有制就可以完全表达社会主义的含义。}

直到\textbf{1936年},\textbf{奥斯卡·兰格}才为反驳米塞斯的观点提供了一个深奥精微的
理论论证,这一论证\textbf{保留了超出法定所有权形式的后资本主义社会的概念,发现了
  计划的位置,使平等主义的价值观得以贯彻。}有迹象表明,兰格对“有组织的资本主
义”可能提供克服市场关系基础的观点给予某种信任,但是,他对米塞斯作出的著名回答并
不求助于此。\textbf{1935年,他明确承认,新古典主义理论与社会主义经济学的关系,要
  比马克思政治经济学(他认为,马克思政治经济学的力量,在于从整体上理解资本主义的历
  史演进)密切得多。次年,兰格开始在新古典主义基础上指出,生产资料私有制的消失并不
  意味着市场交换的终结,市场交换的存在也不排除计划和重要的社会主义价值观的实现。}

兰格的社会主义蓝图假设所有的生产资料实行公有,但是允许消费品私有权的存在;在消费
品和职业选择上,允许自由的市场选择。由中央计划局确定全部国有资产项目的价格,并保
证生产单位能够在此价格上买卖任何数量的商品;管理的指导原则是在给定价格条件下获取
最大利润;当需求或供给过剩时,中央计划当局将以瓦尔拉斯的拍卖方式调整价格,直至市
场出清(尽管兰格似乎也允许非均衡贸易的存在);外部效应和公共产品供应的容差,能够很
容易地结合在这一过程中。这样,兰格就能够声称:价格将是“合理的”,资源配置也是有
效率的。他认为,由于经济不平衡、市场权力以及与外部因素有关的不适当政策的存
在,“现实世界”中的资本主义不能保证这一点。兰格还指出,计划当局也可以通过确定利
率来达到任何理想的积累率,可以通过租借国有资产给生产单位的形式和税收形式进行融资。
此外,国有资源还可以用来抵销自由劳动市场运作中产生的收入不平等。

兰格方案的具体细节同他的总体原则相比,就不是那么重要了。\textbf{他的总体原则就是
  使用价格体系对经济进行计划,而不是把计划看作一种可选择的资源配置方式。反过来说,
  这种可能性取决于把财产所有权看作可以授权给特定经济人的一系列权利,并且不承认社
  会所有权包含着对全部经济行为的具体的集中控制。}这样,兰格在批驳米塞斯的同时,也
含蓄地批评了马克思和其他大部分马克思主义者。在兰格看来,财产所有权形式并不紧密地
依赖于生产关系,商品生产的效果则取决于支配市场运作的体制。事实上,\textbf{理解兰
  格推理的一种方式就是,试图使用维塞尔和庞巴维克关于价值关系普遍性的主张来反对马
  克思和米塞斯、并支持社会主义。}

\textbf{多布认为,快速增长在改善福利方面的意义,远远大于资源配置效率的意义(他把资
  源的有效配置看作新古典主义经济学的主题)。}在《政治经济学和资本主义》一书中,他
对古典的-马克思的传统经济理论(这种理论集中于发展背后的客观因素)和由边际革命引入的
静态主观主义作了比较。\textbf{在多布看来,米塞斯和兰格都错误地被均衡所困扰,而忽
  视了积累和结构变迁。同时,他还认为,他们都过分估计了中央计划,特别是投资计划对
  信息的要求。}实际上,依据多布的看法,对分权体系中——不论是资本主义的还是市场社
会主义的——投资者的忽视,是这些经济类型的主要缺陷。自主经济人的行为,仅仅建立在
对不可确定的未来的猜测基础上,从而导致周期性的经济波动和其他资源的浪费,这反映了
调节只是发生在事后的事实。就消费活动而言,多布承认存在着利用市场的情况。但是,对
于现有生产资料的配置及其未来扩充,最好还是留给计划者去完成,他们可以集中相关信息,
预先做好事前调节的计划。多布还认为,\textbf{无论如何,消费者偏好不能构成判断经济
  政策的理性标准,因为它们是内生的因素。这一点被福利经济学忽视了,并且极大地削弱
  了其原则的重要意义。}由于帕累托最优状态并不是唯一的,因而这些原则无论如何都不具
有充分说服力。

尽管多布从未像保罗·斯威齐走得那么远,斯威齐没有考虑兰格的中央计划局是“价格确
定”机构,多布对\textbf{各种形式的市场社会主义}所作评论的要点是:\textbf{它们将以
  极其近似于资本主义的方式来运行。}仅仅在一点上,他才认为兰格的观点对于社会主义建
设具有一些积极的价值。如果价格调整机制是一种真正的tatonnement,以致于市场关系可以
被限定于影子价格,则计划的效率就可能得以提高。这样,多布率先提出了斯大林去世之后
出现在东欧许多文献中的“\textbf{最优化计划}”的观点。但是,多布仍然坚决地认
为,\textbf{任何把市场作为生产资料配置和积累的真正手段的试图,都将逐渐削弱社会主
  义的优越性}(参见以下第6节关于多布在斯大林去世后对市场社会主义观点的概述)。

多布还注意到,中央计划的地位在经济理论的发展中正在得到加强。他提
到\textbf{投入─产出分析},它代表了一种比苏联的物质平衡体系更加精确的用以调节的工
具。当时,他还不了解其他正在发生的具有同样潜力的理论进展,特别是\textbf{冯·诺伊曼
  的增长理论和康特罗维奇发现的线性增长理论。}20世纪50年代,\textbf{巴兰提出了不发
  达理论,这一理论极大地增强了苏联经济发展战略的声望}(参见以上第9章)。与此同时,
经济分析正朝着背离奥地利学派的方向发展。凯恩斯的《通论》显然是最重要的事件(参见以
上第5章)。但是,新古典主义经济学家正在日益采纳能够容易地用来巩固市场社会主义观点
的瓦尔拉斯范式。甚至在奥地利学派内部,也出现了背叛。\textbf{1942年,在著名的《资
  本主义、社会主义和民主主义》一书中,约瑟夫·熊彼特经过理论上的漫长跋涉,重申了古
  典马克思关于社会主义为什么不可避免的观点。}

然而,\textbf{米塞斯和哈耶克}实质上并未作任何退让。相反,他们开始着手对米塞
斯1920年提出的最初观点的内容进行澄清、重新作出系统而详尽的阐述。在此过程
中,\textbf{他们断然与新古典主义理论决裂,并强调那些激进的凯恩斯主义者一贯认为至
  关重要的变量:未来的不确定性和投资者的预期。}他们甚至根据多布用来论证苏联计划合
理性(即具有产生经济高速增长能力)的理由,提出有利于资本主义的论述。多年来,他们的
观点一直不为人所承认,但是,到20世纪70年代,有明显的迹象表明,他们对所持观点的执
著逐渐得到了回报。

\textbf{新奥地利主义的关键论点是,新古典主义者把事实上的内生变量看作是外生的。偏
  好和技术系数,并不是可以被决策者在不同环境中以同样方式使用的数据,它们随时间的
  变化而变化。}哈耶克认为,变化的主要促动力是从经济人的竞争行为中产生的新知识。
在\textbf{哈耶克看来,竞争更多的是一个过程,而不是一种具体的经济结构,价格则是使
  经济人得以在不断发展变化着的环境中,调节其行为所需的最新信息的最有效的传导者。
  这通常是正确的。但是,}哈耶克认为,\textbf{只有在以私有制为基础的自由市场经济中,
  才能取得最大化受益的结果。}对交换的任何限制,都会制约价格的信息功能,而财产权的
社会化则削弱了损益的激励和制约效应。\textbf{经济进步要求失误要付出代价,能力则将
  得到报偿。}

从奥地利学派的这一观点出发,兰格的论文被看作是对巴罗纳而不是对米塞斯的驳
斥。\textbf{巴罗纳和兰格都假设偏好和技术条件是既定的,经济人在由政权当局规定的游
  戏规则中像程序化的受动装置一样自动地实现最优化。奥地利学派的学者认为,}作为新古
典主义经济学家,\textbf{巴罗纳和兰格从未真正地与古典政治经济学的虚假的客观主义特
  征(多布在论证中央计划的合理性时曾诉诸于此)相决裂。}

奥地利学派对新古典主义经济学的批判,为一些新古典主义经济学家所接受。过去20年来,
他们越来越多地注意到\textbf{不完全信息的重要性,包括含有道德风险和搭便车可能性的
  委托人——代理人关系。}其结果就是,\textbf{对一些前资本主义经济关系形式,如分成制,
  比以前更加赞许,}其理由是:它们是对它们存在于其中的不确定的环境的有效反应(参见
以上第11章)。但是,这种新的观念,导致对社会主义更具批判性的看法。在这里,具有重大
影响的是\textbf{阿尔钦和德姆塞茨}的著作。\textbf{他们认为,资本主义生产单位的等级
  专制结构植根于对效率的考虑,而不仅仅源自财产关系的特定类型。}没有监督,工作的努
力程度就会下降;而没有业主获取剩余收入(利润)的权力,就不会有进行有效监督的动力。
他们认为,正是这些因素,解释了为什么是“资本家雇佣工人”,而不是“工人雇佣资本
家”。

但是,\textbf{奥地利学派的观点也遭到了批评。}其他新古典主义经济学家同斯拉法主义者
和凯恩斯主义者一起指出,\textbf{奥地利学派对模型详细说明的力度极为缺乏,并揭示了
  该学派在理论推演中的错误,为其理论推论提供了假定的反面例证。而且,资本主义国家
  的经济史,也几乎没有支持过于浮夸的奥地利学派的主张;更为经常的是,国家资本主义
  具有更大的活力(参见以上第11章)。}大萧条表明,价格的信息功能并不总是运作良好(参
见以上第1章),特别是与斯大林主义下达到的\textbf{惊人的增长率}相对比更是如此(参见
以上第2章及本章第6节)。然而,东欧计划经济的实际历程,为奥地利学派对社会主义的攻击
提供了某些证据。这至少是那些不仅研究过、而且在“实际存在的社会主义”中工作过的主
要经济学家的结论。

\section{斯大林主义经济学和非斯大林化}

莫里斯·多布和保罗·巴兰对苏联模式发展的某些看法,看起来是经得起苏联经济增长数据检
验的。\textbf{1928年至1985年间,“苏联国民生产总值(GNP)的年均增长率……为4.2\%。
  如果把第二次世界大战的五年排除在外,则达到4.7\%……在此后的这些年中,苏联的增长
  纪录依然保持在世界最高之列”。兰格}没有忽略这种增长现象,在20世纪50年代和60年代,
他更加\textbf{赞赏地看待中央指令性计划}。同时,兰格在波兰的同事米哈尔·卡莱茨基确
实对严格的计划和过分强调投资的做法提出了批评,赞成南斯拉夫“自治”体制的某种分权
形式,但他也没有发现苏联经济在总体方法上存在的重大问题。致力于研究第三世界的西方
发展经济学家,以及许多关注工业化的资本主义经济的加速增长问题的主流分析家也确信,
对世界的拯救可能存在于国家计划之中。\textbf{1961年,十年内赶超美国写进了尼基塔·赫
  鲁晓夫领导的苏联共产党的纲领,当时,许多西方人士认为存在着实现这一纲领的有利的
  机遇。}

\textbf{赫鲁晓夫还力图通过大幅度削弱恐怖机构的权力和提供少量的公民自由权来改革斯
  大林主义。}他本人对过去年代的观点是托洛茨基关于上层建筑蜕变分析的一种高度保守的
变体:全部问题都产生于斯大林本人的高度集中的权力。这种所谓的对列宁主义集体领导的
偏离,被委婉地称为“\textbf{个人崇拜}”,由此产生的对党的“犯罪”和领导人的“失
误”,在1956年的“\textbf{秘密报告}”中作了概略描述。尽管这一举动对国际共产主义运
动产生巨大的冲击,但赫鲁晓夫并没能解释斯大林专政是如何产生的,为什么它特别具备恐
怖主义的特征,以及这种蜕变为何没有触及向社会主义或向社会主义生产方式的过渡,这都
被认为是与列宁主义学说相一致的。然而,实际实施的非斯大林化与赫鲁晓夫的“理论”是
一致的,它主要关注的是\textbf{消除官僚主义。赫鲁晓夫及其继任者没能发现的是替代恐
  怖而控制国家机构的机制。由此,腐败如雨后春笋般地滋生漫延,加剧了苏联领导人
  从20世纪60年代开始日益关注的经济难题。}

非斯大林化部分地是对这些经济难题的回应。\textbf{经过30年的快速增长,成本低廉的大
  量劳动后备军和可利用的自然资源急剧减少,而许多资本设备包含的技术同知识前沿连接
  在一起,这阻止了未来的增长还以同过去相同的借鉴西方经验的方式进行。为了走上集约
  增长的道路,必须在更大范围内发挥人们的主动性和积极性,这就要求更好的保证条件和
  不断增长的消费品供给,以便提高激励性和为更加开化的专政提供合法依据。}但是,这本
身就使得指令性计划中已经明显存在的一系列问题更加严重了。\textbf{这种体系的运作曾
  是以不惜一切代价保护优先发展部门为基础的。}当军事和重工业计划受到威胁时,投入就
从非优先部门转向这些部门。\textbf{从而,统计错误的责任就集中在了其他领域,如消费
  品生产领域。在削弱不同“部类”之间地位差别的过程中,对持续的计划的压力增强。}这
种计划已经被过去发展中产生的日益增强的经济复杂性强化。因此,实际经验证
明,\textbf{巴罗纳对有效的中央计划可能性的疑虑是核心问题,至少对于相对成熟的经济
  来说是如此。}

经济发展也掩盖了主张\textbf{指令性经济}的马克思主义理论家们所没有预见到的严重问题。
苏联由于\textbf{过度消耗资源,疏于保护环境,对基础结构(例如,道路和仓库容量)投入
  不足,已经预支了未来的增长来支撑现在的增长。}这表明,正如多布和巴兰确信的那样,
新古典主义经济学家关于资源配置的观点并不限于静态领域。\textbf{为执行严格的计划而
  施加给生产单位的增加产出的强大压力,虽然使得目前的增长提高,但同时也对产品质量
  产生了灾难性影响。}这又是那些倾向于把每种商品类型都看作同质实体的大部分马克思主
义者所料想到的。斯大林去世后,苏联开始改革时所面临的重要问题恰恰产生于过去发展的
成功之中。而对这一点,通行的理论几乎没有论及。

\textbf{通过更广泛地应用市场和给予生产单位更多的自主权来分散决策的尝试,开始
  于20世纪60年代。}这是合乎理性的,它同两次大战期间市场社会主义者的理论广泛地相一
致。事实上,他们的观点得到了加强,这是由于对价格体系稳定性特征理解上的进步,也是
由于最理想的计划制定者为实施计划目标,在计算部门之间比例和合理价格时对数理规划技
术的运用。但是,\textbf{不论是在苏联还是在东欧国家,改革的尝试都遭到了利益受到威
  胁的官僚机构的顽固而有效的抵制。况且已经实施的改革还没有证明是成功的。增长率继
  续下降:苏联的经济增长率从20世纪50年代的5.7\%下降到80年代初的2\%左右。东欧“实
  际存在的社会主义”发现自己甚至连“追随者”的地位都难以维持,}更不必说成为“赶
超”西方经济的领先者了。正是这种令人沮丧的形势,最终导致20世纪80年代中期以来就已
明显的危机。当统治者意识到,他们不能“以不变的形式维持自己的统治”,以及他们的内
部分裂使得\textbf{“分歧产生……通过这些分歧,被压迫者的不满和愤
  慨……[可能]……爆发”}时,\textbf{这种危机就形成革命的态势。}

\textbf{莫里斯·多布}从1956年直至1976年去世之前,跟随东欧各个中央委员会的领
导,\textbf{轻松自如地适应了分权的需要,}事实上他已经预见到这一点。\textbf{他现在
  更加明确地把斯大林主义的经济看作通过粗放式增长战胜落后的一种手段,}认为它的成功
带来了对日益增长的灵活性的需要。\textbf{他对坚持过时的经济管理形式而导致微观经济
  缺乏效率,作了某些准确的分析。}但是,多布对新古典主义福利经济学的批判依然是尖锐
的。他认为,中央对投资的控制仍然是关键的。\textbf{他建议按照市场规则重新制定当前
  产出的有关决策,而计划者则保留对积累的控制。不过,多布承认这是一个人为的特征,
  很难付诸实施。}

西方的其他许多马克思主义者如\textbf{艾萨克·多伊彻,赞成把非斯大林化和分权化作为迈
  向“真正的社会主义”的开端,}这一观点一再地得到重申。但是,\textbf{在夏尔·贝特
  兰和保罗·斯威齐及其追随者看来,东欧的改革可能会使特定利益制度化,而不是维护社会
  主义整体。他们像反对官僚极权主义一样反对引入市场的社会主义(参见以上第9章)。}这
样一种观点促使瑞典社会民主主义者\textbf{阿瑟·林德贝克}对这些马克思主义者提出质
询:\textbf{他们认为如何可能在实际上形成对经济活动的调节,进而如何才能有效地进行
  这种调节。}他认为,\textbf{在与中央指令性计划必然相伴的官僚体制和更广泛地利用市
  场实行分权之间,不存在第三条道路。}林德贝克的质询没有得到满意的回答,那些反复提
出这一问题的人也没有得到满意的回答。的确,贝特兰和斯威齐对毛泽东主义的支持,对他
们放弃这一难题大有帮助。他们忽略了毛泽东统治下的灾难性的经济措施,并用一种非常奇
特的方式解释文化大革命,因而完全反对“经济主义”,赞成通过持续的阶级斗争摧毁既存
的社会关系。斯威齐甚至进一步询问:\textbf{为什么“东欧国家和苏联非得陷入与资本主
  义世界的激烈竞争之中”}。

更有分量的分析是东欧经济学家们自己作出的,尤其是\textbf{亚诺什·科尔奈,他力图解
  释“实际存在的社会主义”(不论是被改革的、还是保留了中央集权的指令性经济)中十分
  明显的规律。}随着时间的推移,科尔奈找到了更深层次上存在的主要缺陷。20世纪50年
代,“\textbf{过度集权}”被认为是社会主义诸种问题产生的根本原因;到70年
代,“\textbf{急速增长}”成为最重要的因素;80年代,社会主义“\textbf{失灵}”的基
本原因又被归结为“\textbf{软预算约束}”;而到1990年,国家所有制在支撑软预算约束中
的作用越来越多地得到强调。\textbf{由于缺乏对企业的融资约束,从而有效地排除了任何
  形式的破产,这使得东欧企业不像大部分资本主义企业那样是“需求约束”型的,而
  是“资源约束”型的。这就造成了它们对投入贪得无厌的需求,导致劳动纪律涣散、生产
  资料囤积、产品短缺、质量低劣以及对价格反应迟钝的行为、强劲的通货膨胀趋势和腐败
  现象等。}这些问题致使分权的尝试阶段性地倒退,改革总是难以完善,在扭转经济相对下
滑状况上也缺乏有效性。

\textbf{塔玛斯·鲍威尔应用这些观点来解释投资周期,}东欧经济文献对此有很好的记载。
在他看来,\textbf{投资具有增长到一定水平即难以为继的固有倾向,最终迫使采取紧缩措
  施,从而导致周期性的循环增长。}“危机”并不采取与资本主义社会的危机完全相同的形
式——特别是失业率的波动要小得多——但是,很显然,\textbf{多布一直声称的社会对积
  累控制的优越性,肯定成了某种不可知论者的观点。}越来越多的西方经济学家也认为,主
要资本主义国家经济中存在的诸多无效率和失灵现象,实质上也具有与科尔内
用“\textbf{软预算约束}”表示的症状完全相同的综合病症。这样,由社会民主主义政策引
出的难题,就被看作同缓和“生产无政府状态”尝试相伴随的一种比较温和的变体。所有这
些,对新奥地利学派的观点给予了大力支持,因为它对任何既要保持微观经济效率、又要实
现传统社会主义价值观的持续的尝试表示质疑。但是,它本身必须由以上第1节提出的相关问
题,以及以下第7节就经济效率展开的更广泛的讨论所证明。

\section{可行的社会主义}

“实际存在的社会主义”的实践,对许多社会主义者产生了显著的影响。20世纪70年代以来,
出现了大量批判早期马克思主义者的观点的著作。这些著作对非资本主义的经济结构形式的
论证和分析作了评论,提出了被认为是理想的和切合实际的社会主义蓝图,并指出了它们可
能如何实现的途径。\textbf{1989年的剧变有力地提出了两个问题:在斯大林主义和资本主
  义之间是否存在第三条道路,如果存在,它确切地由什么构成。}对这些问题缺乏具有说服
力的回答本身,有助于解释一些东欧国家以压倒之势普遍支持恢复资本主义的现象。不过,
从20世纪60年代初开始,波兰、匈牙利、捷克斯洛伐克(在杜布切克任内)和南斯拉夫的马克
思主义经济学家,就已经对社会主义进行了多方面的再思考。\textbf{南斯拉夫把真正独立
  于莫斯科与官方的“自治”思想结合在一起,}表达了这样的主张:这样的\textbf{第三条
  道路}在那里已经出现。\textbf{东欧修正主义者如奥塔·锡克强调市场在确保资源有效配
  置和削弱国家官僚主义权力方面的作用。他们还强调了工人自主管理企业的一些不同形式
  所带来的好处:}促使工人关心积累,缓解当前消费的急速增长造成的通货膨胀压力;此外,
也使生产过程更富有人性,并且改善了经济异化现象。南斯拉夫的实践引起了许多西方非马
克思主义经济学家的关注。20世纪70年代,出现了大量关于自治经济的新古典主义著作,它
们从正统微观经济学的角度对工人自治企业的行为作了分析。

\textbf{在西方,社会主义派别中关于市场社会主义的最重要的著作是亚历克·诺夫的《可行
  的社会主义经济学》},这部著作之所以重要,是因为它是作者通过对\textbf{东欧经
  济}的终生研究而形成的,并且,它对本章前几节提及的观点都作了严肃认真的研
究。\textbf{他倡导一种真正符合其词义的“混合”经济,在这种经济中,公共组织掌管基
  础设施,工人合作社负责中小规模的生产活动,私营企业和家庭企业则主要经营诸多的服
  务行业和零售批发业。}诺夫的观点典型地代表了目前被现代马克思主义者广泛认同的观
点:\textbf{如果社会主义要继续作为非乌托邦设想而存在,它就必须把市场、预算硬约束
  和某种程度的生产资料私有制联系起来。}

这种观点遭到\textbf{抵制}是不足为怪的。特别是\textbf{欧内斯特·曼德尔,他继续提出
  有吸引力的和现实的社会主义要建立在集体所有制、平等、合作和计划,并且排除了市场
  和官僚机构的基础之上。}(作为一个托洛茨基主义者,曼德尔也更经常是\textbf{工人自
  治的坚定支持者}。)诺夫本人的“可行的社会主义”方案,曾是曼德尔重申传统马克思主
义的社会主义经济学的主要攻击目标。但是,根据1920年米塞斯最早对马克思主义提出批评
以来的学术进展,诺夫在揭示马克思主义的“瑕疵”方面没有有遇到困难。\textbf{事实上,
  曼德尔的原教旨主义甚至比诺夫认识到的更脆弱。他试图把公社国家的概念和作为一个巨
  大企业运行的经济结合起来,确信工人阶级政治和技术发展趋势共同作用以实现二者。事
  实上,并没有迹象表明市场关系的作用有削弱的显著倾向。}曼德尔脱离这一事实而设想的
确定社会利益的投票程序,可能恰恰展示了新古典主义社会选择理论家们着重指出的那些无
效率现象和悖论。因此,即便不考虑经济业绩,用曼得尔设想的方式达到真正的公共控制也
是不可能的。

实际上,\textbf{诺夫的可行的社会主义模式与其他现代修正主义者的模式只有一点是真正
  不同的:}尽管他赞成在任何可能的地方实行合作制,但是,\textbf{工人的自主控制在他
  的方案中并不是处于中心位置。对经济民主可行性的疑虑是有充分理由的,}它也是马克思
主义内部的非列宁主义者和反斯大林主义者,如从20世纪20年代初的议会共产主义者,到像
保罗·马蒂克、拉亚·杜娜耶夫卡娅和科尼利厄斯·卡斯托瑞安迪斯(他恰恰是出于这些原因而
和托洛茨基主义决裂)这样的\textbf{自由的马克思主义者}不断提出的命题。\textbf{关于
  自治的理论和实证都揭示:无效率和不平等这样的重要问题会由此产生。特定工人团体的
  特殊利益可能被制度化,并可能导致一种“集体资本主义”,其运作比建立在私有产权基
  础上的资本主义要差得多。}事实上,这些缺陷可以被看作贝特兰和斯威齐反对各种形式的
分权改革的“合理的内核”。

\textbf{然而,不论是自治的理论还是实证都不是结论性的。从理论视角看,自治的后果对
  特定的结构形式非常敏感:工人管理制度的权力同其他团体的权力,比如消费者权力的关
  系;市场竞争的程度;预算约束的“刚性”。经验性的实证包括成功的生产合作社的例
  子,}这些合作社的成员积极评价他们参与合作社事务的能力,在与资本主义企业的竞争中
能够坚守阵地而不被击败。因此,其他社会主义者有理由把自治看得比诺夫所认为的更重要。
许多人认为,当前迈向经济民主的运动,构成了超越资本主义的最合理的战略。对此辩护时,
他们着重突出了诺夫在《可行的社会主义经济学》一书中的一个重要局限。

尽管\textbf{诺夫}一贯指责马克思主义是乌托邦主义,但是,他本人也可以受到同样的指
责。\textbf{他无从处置代理机构据以形成的特定利益,诺夫的模型是要由代理机构加以实
  施的,而不是依靠其观点的合理性来吸引大众的支持。}与此相对照,那些强调民主参与重
要性的人认为,这些特定利益不仅构成了自由社会主义的实质,而且也是形成过渡战略的主
要出发点。塞缪尔·鲍尔斯和赫伯特·金蒂斯指出,自从专制主义时代以来,西欧和北美几乎
所有的抗议运动都使用了民主自由主义的用语。它们的不同之处只在于哪类权利将会通
行:\textbf{财产的权利抑或个人的权利。投身于社会主义的运动,可能会发现自己在某种
  程度上是建立在前马克思主义理想之上},这种理想是大部分古典马克思主义者所不承认
的。\textbf{就其实质而言,资本主义不能充分贯彻自由、平等和博爱的价值观,而这些价
  值观传统上是被用来维护资产阶级社会的。资本主义结果就是“搬起石头砸自己的脚”。}

\textbf{进而还可以认为,经济生活中民主参与的扩展将会提高效率水平。决策权向实际生
  产者的扩展,将开掘出知识资源和生产力资源,这些资源被资本主义的专制的雇佣劳动结
  构压制着,同时还可以降低阿尔钦和德姆塞茨声称的被资本主义最小化的监督和强制成
  本。}换言之,尽管鲍尔斯、金蒂斯和霍奇森接受了奥地利学派对中央集权指令性计划所作
批判的许多观点,并反对市场社会主义的非民主形式,但他们依然坚持认为,这些批评家的
论据可用作民主社会主义的论证理由。并且也可以应用科学社会主义观点来支持这些主
张。\textbf{通过利润共享、共同所有权、共同决策,也通过磨平等级制度的尖利棱角,发
  达资本主义经济内部的许多制度已经得到修正。}这些并不是都得到显著的表现,但是已经
取得的进展表明,社会主义关系孕育于资本主义之中的观点,可能会被赋予新的涵义。

与此相对照,\textbf{资本主义的真正力量在于它的创新能力,这一点马克思在《共产党宣
  言》中已经指出。它为那些乐意承担创新风险的人,提供了相对开放的获得生产资料的途
  径,也为那些依然依附于现存技术的人展开经济战争提供了限制最少的环境。}熊彼特强调,
在“\textbf{创造性破坏}”(他认为这是在马克思那里发现的)的经典辩解中,\textbf{认为
  这个制度是残酷的、不公正的和紊乱的,但是,它的确提供了商品,然后毁掉这一切,这
  些商品是……[人民]……需要的”。}正是在这里,传统马克思主义仍然保持着对资本主义
的批判力量;在使残酷、不公平和紊乱最小化的同时保持发展的动力,既是合理的,也是经
济的。这是能够做到的,这一事实已为社会民主主义的成功改革所证明。许多现代马克思主
义者确信,伴随着激进的民主化,这些改革能够得到更深更远的扩展,它将使经济进程的社
会化远远大于资本主义可能达到的程度。\textbf{激进的民主化将不可避免地削弱资本主义
  制度下某些已经建立的个人权利。由于这些权利只关注财产权的应用,而忽视财产权对非
  所有者的影响,因此,与基本的公民自由权相比,它们在道义上的优越性要小得多。}而且,
在假设净效果是有益的情况下,某一领域内的权利收缩被看作只是其它方面的权利扩张的结
果。\textbf{民主参与可能对经济效率造成的负面影响,能够通过渐进主义方式达到最小化。
  依据资本主义原则组织起来的企业之间的竞争,可以用来限制民主化可能导致的停滞趋势。
  实际上,一些现代马克思主义者承认,保留某些资本主义制度形式是合理的},以便为那些
赞成雇佣劳动的人提供选择机会,因为他们的自我发展属于政治学和经济学范围之外的问
题。\textbf{这一战略的主要薄弱之处在于它的改良主义性质,它假设——正如第二国际时
  代的修正主义理论家们所做的那样——能够通过和平的方式夺取和改造既有的国家机构。}如
果能够得到强有力的普遍支持,这也许能够做到。1989年的东欧剧变揭示,如果既存秩序中
的危机已经足够深刻,而过去的暴力革命又没有提供明显的更具优势的选择,那么这就不是
一个空洞的希望。

\textbf{无论如何,现代马克思主义者显然力图保持科学社会主义的说服力,同时又承认马
  克思和恩格斯的原创论述中存在的巨大局限性。}在此过程中,后资本主义社会的概念经历
了剧烈变化,反社会主义的批评家则可以宣称取得了理论上的重大胜利。但是,在寻求可行
的社会主义理论的任何方案中,对这些有效的批判作出回应是不可避免的。\textbf{退回原
  教旨主义是无益的,因为,目前既不存在古典马克思主义设想的社会主义的经济基础,也
  不存在可能努力实现它的任何机构。}市场社会主义是否将成为人类历史的最终阶段,或者
是否只是向马克思原先提出的未来的“自由生产者联合体”进行漫长过渡的一部分,这仍然
是一个有待回答的问题。








%%% Local Variables:
%%% mode: latex
%%% TeX-master: "../../main"
%%% End:
