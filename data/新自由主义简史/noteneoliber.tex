
市场和贸易的自由保障个人自由,这一前提是新自由主义思想的核心特征,并长期以来在美国对待世界其他地方的姿态上占据主导地位。美国试图借助武力强加给伊拉克的显然是这样一种国家机器,其根本任务是为国内外资本的高盈利资本积累创造种种条件。我把这种国家机器称为新自由主义国家。这种国家机器所体现的自由,反映的是私人财产所有者的利益、企业利益、跨国公司的利益、金融资本的利益。\pagescite[][7-8]{davidneoliber}

再分配效果和不断增长的社会不平等已经成为新自由主义化过程中的必然特征,亦是整个计划的结构性因素。杜梅内尔和列维经过仔细重组数据后总结说,新自由主义化过程从一开始就是一项旨在重新恢复阶级权力的计划。

我们可以将新自由主义化解释为一项\textbf{乌托邦计划}——旨在实现国际资本主义重组的理论规划,或将其解释为一项政治计划——旨在重建资本积累的条件并恢复经济精英的权力。\pagescite[][19-20]{davidneoliber}

墨西哥的例子表明在自由主义实践和新自由主义实践之间存在根本区别:在前者那里,债权人承担错误投资决策的损失;而在后者哪里,债务人受到政府和国际力量的压迫,承担债务偿还的一切费用,不管这将给当地人民的生计和福利带来多大影响。……从世界其他地区摘取高额收益率。美国及其他发达资本主义国家的经济精英或上层阶级的权力重建,凭借国际流动和结构性调整实践,从世界其他地区成功榨取了大量盈余。\pagescite[][31]{davidneoliber}

简而言之,新自由主义化就是将一切都金融化。这一过程深化了金融,后者从此不仅掌控其他一切经济领域,而且掌控国家机器和——如兰迪·马丁(Randy Martin)所说——日常生活。这一过程还使全球交易关系发生剧烈波动。无疑存在着一股从生产过程转向金融领域的势头。\pagescite[][34]{davidneoliber}

然而,资本实际拥有者(股东)的权力在某种程度上被削弱了,除非他们能获得充分的投票权益以影响公司政策。

如果被放在1944年由卡尔·波兰尼(Karl Polanyi)所做的相反论述(就在朝圣山学社建立之前)的背景下考察,就会非常有趣。波兰尼指出,在一个复杂社会中,一旦自由成为咄咄逼人的行动剌激因素,自由的意义就会变得自相矛盾、歧义丛生。他注意到存在两种自由,好的和不好的。就不好的自由方面,波兰尼列出的有“剥削他人的自由,或获得超额利润而不对社会做出相应贡献的自由,阻止技术发明用于公益事业的自由,或发国难财的自由”。但是,波兰尼接着指出:“这些自由所推动的市场经济同样产生了我们所当珍视的自由:良心自由、言论自由、集会自由、结社自由、个人选择工作的自由。”虽然我们或许会“因这些自由自身的价值而珍惜它们”——而且我们许多人也向来是这么做的——但这些自由在很大程度是“市场经济的副产品,这同一种经济也要为那些恶的自由负责”①。波兰尼对问题的两面性给出的回答,从新自由主义思潮占据主流的背景下看,似乎有些奇怪: 市场经济的逝去可以开启一个拥有前所未有的自由的时代。法律上的自由和实际的自由能比以前任何时候都更广泛而普遍;管理和控制不仅能为少数人争取自由,也能为所有人争取自由。自由不是作为从源头上就腐败了的特权的附属物,而是一种远远超出了政治领城狹隘界限、延伸至社会自身内部组织的时效权利。由此,老的自由和公民权利融入新的自由之中,这种新的自由产生自工业社会为所有人提供的闲暇和安全。这样一个社会能够同时承担起公正和自由。①

自由的理念由此“堕落为仅仅是对自由企业的鼓吹”,这意味着“那些其收入、闲暇和安全都高枕无忧的人拥有完全的自由,而人民大众仅拥有微薄的自由,尽管他们徒劳地试图利用自己的民主权利来获得某种保护,以免遭那些有钱人的权力的侵害”。但是——事情往往如此——如果“没有权力和压制的社会是不存在的,强力不发挥作用的世界也是不存在的”,那么维持这种自由主义乌托邦前景的唯一办法就是靠强力、暴力和独裁。在波兰尼看来,自由主义或新自由主义的乌托邦论调注定会为权威主义甚或十足的法西斯主义所挫。①好的自由已经丧失,而坏的自由横行霸道。\pagescite[][38-39]{davidneoliber}

个人自由的价值和社会正义并不必然相容;追求社会正义预设了社会团结和下属前提:考虑到某些更主要的、为社会平等或环境正义进行的斗争,需要压抑个体的需求和欲望。

新自由主义需要实际策略的支持,即强调消费者选择的自由——不仅选择特殊产品,而且包括生活方式、表达方式和一系列文化实践的选择。新自由主义化需要在政治和经济上建构一种以市场为基础的新自由主义大众文化,满足分化的消费主义和个人自由至上主义。\pagescite[][45]{davidneoliber}

1975年一个强势的投资银行团体(由花旗银行的沃尔特·里斯顿(Walter Wriston)领衔)拒绝延长偿债期限,把纽约市推向了技术性破产边缘。随后的资金援助促使建立了接管这座城市预算管理的新型机构,他们首先要求用城市收入还清债务:剩下的资金用于基本服务。造成的结果是限制纽约市有力的地方协会的报复,施行工资冻结并削减公共职务和社会供给(教育、公共医疗、运输服务),强征使用费(学费第一次被引进纽约城市大学系统)。最后一项耻辱行为是要求地方协会将养老基金投资城市债券。于是,这些协会或者节制了自己的要求,或者面临当城市破产时失去他们养老金的前景。\pagescite[][47]{davidneoliber}

但是,纽约的投资银行家们并没有离开这座城市;他们牢牢把握机遇,按照他们的计划重建城市。创造一个“良好的商业氛围”是当务之急,这意味着\textbf{利用公共资源}建立起适合商业的种种基础设施(尤其是电信方面),而商业则与资本主义企业的津贴和税收刺激相联系。企业福利取代了人民福利。城市精英的机构被用于推销城市形象——作为文化中心和旅游目的地。统治精英支持向各色国际潮流开辟文化领地。自恋式地探索自我、性、身份成为城市布尔乔亚文化的主题;受城市强势文化机构推动的艺术自由和艺术破格实际印象了文化的新自由主义化……(还有很多)\pagescite[][49]{davidneoliber}

并将个人最高税率从78\%降到28\%,这些都明显反应出重建阶级力量的意图。最糟糕的是,公共资产毫无阻碍地就流向了私人腰包。(国家支持医疗研究和专利,但利益被公司拿走)\pagescite[][55]{davidneoliber}

新自由主义国家很可能是一个不稳定且矛盾的政治形式。

新自由主义国家应该支持牢固的个人财产权、法治以及令市场和自由贸易得以自由运转的制度……国家因此必须利用其对暴力手段的垄断来不惜一切维护这些自由。\pagescite[][66]{davidneoliber}

新自由主义理论家对民主抱有极大怀疑。他们认为,多数人的治理会对个人自由和宪政自由带来潜在威胁。民主被视为奢侈品,只有在相对富足而且存在一个强大的中产阶级以保障政治稳定的条件下,民主才有可能。所以,新自由主义者偏向专家和精英的统治。政府强烈偏向行政命令和司法判决,而不是民主和议会的决策。\pagescite[][68]{davidneoliber}

张力和矛盾。首先,如何理解垄断权力就是一个问题。竞争通常会导致垄断或寡头垄断的局面,因为强势企业会挤掉弱势企业。大多数自由主义理论家觉得这不是问题(他们说,这会使得效率最大化)……第二个聚讼纷纭的议题是关于市场失灵。权力或信息不对称……拥有更多信息和力量的一方可以轻易借此良机获得更多信息和相对更大的权力。此外,知识产权(版权)的建立鼓励了“寻租”\footnote{rent seeking,指一种不具有生产性,但能给行为主体带来利益的经济行为。}行为。……技术改革的拜物教信仰……根本性政治问题,一方面是诱人但异化的占有性个人主义,另一方面是渴望有意义的集体生活,这两方面之间产生了矛盾。……新自由主义者为抵抗他们最担心的事物——法西斯主义、共产主义、社会主义、暴民专制,甚至多数的统治——不得不为民主治理设置很大限制,转而依靠不民主和不负责任的结构(诸如联邦储备局或国际货币基金组织)做出关键决定。这造成的悖论是,在一个认为国家不该干预的世界,国家和政府却通过精英和“专家”忙于干预活动。国际竞争和全球化可以被用来规训各个国家内部反对新自由主义安排的运动。如果这一武器失败了,那么新自由主义国家必须求助于劝说、宣传,必要时也求助于赤裸裸的强力和政策力量,来镇压反对新自由主义的声音。自由主义的乌托邦计划,最终只能靠权威主义来维持。\pagescite[][71]{davidneoliber}

很难描绘处于新自由主义化时代的国家总体特征。不平衡发展、国家特性。特别有两种情况,重建阶级力量的冲动在实践中歪曲,有时甚至颠倒了新自由主义理论。第一种情况出于为资本主义创造“良好的商业或投资环境”的需要……这些偏颇体现在仅仅将劳动力和环境视作商品的时候。在发生冲突时,典型的新自由主义国家将站在“良好投资环境”一遍,或者反对劳动者的集体权利(或生活质量),或者反对新自由主义国家典型地倾向于维护金融体系的信誉和金融机构的偿还能力,而不是维护大众幸福或环境质量。\pagescite[][72]{davidneoliber}

共产主义崩溃后的中欧和东欧,状况就很特殊……以“休克疗法”为名进行的快速私有化过程。发展中国家在与新自由主义化过程相吻合的另一面,这些国家在建设基础设施以培育良好商业环境这方面却属于积极的干预主义国家。但与此同时,新自由主义化也为阶级构成创造了条件……提升它们在国际竞争中的地位……而随着这一阶级力量的壮大,它就倾向于摆脱国家权力并对其进行重新调整。\pagescite[][74]{davidneoliber}

货币主义被当作国家政策基础……吊诡的是,这意味着新自由主义国家无法承受大规模财政赤字,哪怕正是金融机构本身的错误决定构成了赤字。政府不得不介入并用自己认为的“良”币替代“劣”币——这解释了央行的压力,央行必须对国家货币的稳定性保持信心。国家权力经常被用来给予公司经济援助,或转移财政失败。

国际货币基金组织、世界银行……协调债务免除的权力……这一实践很难按照新自由主义理论来解释,因为投资者原则上应该自己承担错误。……就国际范围而言,这意味着从贫穷的第三世界人民那里榨取剩余,以便偿付国际银行家的债务。斯蒂格利茨讽刺道:“这是多么古怪的世界啊,反倒是贫穷的国家在补助最富裕的国家。”


通过金融机制榨取贡金,属于老式的帝国主义活动,但却证明对于重建阶级力量助益良多——

贷方应为自己鲁莽的投资造成的损失埋单,但国家却很大程度上帮助贷方免于损失,反倒是借方要不及社会成本负责偿清。

受到节俭措施的约束,这些国家的经济陷入周期性经济停滞,其偿还债务的期望就时常推迟到遥远的未来。

新自由主义体制改革这颗毒药。1998年墨西哥比索危机,1998年巴西货币危机,2001年阿根廷经济全面崩溃……

新自由主义国家需要某种民族主义来延续自身的存在。救治以往在新自由主义影响下遭遇瓦解的社会团结的纽带。\pagescite[][74-89]{davidneoliber}

在克林顿任期内,华尔街——国际货币基金组织——美国财政部三位一体在经济政策中占据主导……美国成功的真正秘密是它现在可以从其驻扎于世界其他地区的金融和企业机构(同时包括直接投资和证券投资)抽取高额收益。\pagescite[][95]{davidneoliber}

对于1997-1998亚洲经济危机……(哈维不同于国际货币基金组织、美国财政部的意见是)激烈的财政松绑和未能对放任和投机的证券投机建立充分管制是造成问题的根本原因……那些还没有放开其资本市场的国家和地区——新加坡、台湾地区和中国大陆——受到的影响远远小于那些开放了资本市场的国家,诸如泰国、印尼、马来西亚和菲律宾。

斯蒂格利茨:国际货币基金组织正是“反映了西方金融共同体的利益和意识形态。”但他忽略了对冲基金的作用,而且,他经常哀叹日益加剧的社会不平等,认为这是新自由主义化的副作用,却从未想过这种现象可能正是新自由主义化存在的理由。\pagescite[][100-101]{davidneoliber}

金融危机可能是由资本冲击、资本逃逸或金融投机引起的,或者,金融危机是精心设计出来协助掠夺性积累的。

 邓小平四个现代化,改革力图引入市场力量,在内部支撑起中国经济。其理念是刺激国有企业之间的竞争,并希望借此促进创新和发展。市场价格机制被引入,但比这更重要的是,中央政治权力迅速下放到各个区域和地方。这最后一步被证明尤为精明。与北京权力中心的传统对抗得以避免,地方的积极性可以为新的社会秩序开疆拓土,创新失败很快就被忘掉了。为配合这一努力,中国还向外国贸易和海外改革开放(虽然是在严格的国家监督下),由此终结了中国对世界市场的孤立。

 绝非低估……如果没有相应的改变——阶级关系、私有权、以及所有其他制度安排,用以典型地奠定市场经济发展的基础——市场本身无法改变经济。

 中国没有采取“\textbf{休克疗法}”——这是后来1990年代由国际货币基金组织、世界银行和“华盛顿共识”强加给俄罗斯和中欧的道路,令其快速私有化——所以\textbf{成功避免了困扰这些国家}的经济灾难。\pagescite[][124-126]{davidneoliber}

 对外国证券投资设置的壁垒有效限制了国际金融资本在中国的势力。中国\textbf{不愿意允许国有银行(如证券市场和资本市场)之外的金融调节形式},这就令资本丧失了其面对国家权力时的\textbf{重要武器}。\pagescite[][127]{davidneoliber}

城乡二元对立,户籍 \pagescite[][130]{davidneoliber}

1978年到1984年间,农村收入惊人地以每年14\%的速度增长,产量也类似地增长了。从此以后,\textbf{农村收入停滞不前甚至实际有所下跌(尤其是1995年以后)},只有一小部分特选的生产领域和生产线不受影响。城乡收入差距明显扩大。\pagescite[][131]{davidneoliber}

中国目前是“\textbf{有史以来世界上发生最大规模的人口迁移}”的地方……。

1983年,\textbf{合同工}。国有企业的不良贷款大幅增长。国有企业部门进一步改革势在必行。所以,1993年国家决定“把指定的大型和中型国有企业改为有限责任制或股份制企业”。……\textbf{进一步的国有企业民营化/转制浪潮发生于1990年代末},到2002年,国有企业仅占全部制造业就业人口的14\%,1990年这个数字是40\%,。国家最近的动作则是把乡镇企业和国有企业完全开放给外资所有。\pagescite[][134-135]{davidneoliber}

1995年后,中国政府实际上把整个国家开放给任何形式的海外直接投资。1997-1998年,席卷乡镇制造业的破产浪潮,设计主要城市中心的许多国有企业,成为一个分水岭。

自1998年起,中国人就设法通过以下方式解决这个问题:贷款投资大型工程项目(南水北调),改造物质基础设施。\improve[inline]{注意1997年建设部受到重视,1998年住房改革与此联系。另外,国家的固定资产投资方面可联系大卫哈维读资本论中 关于固定资产 的理论。}

快速城市化提供了一种渠道,可以吸收大量集中在城市的农村劳动力。\pagescite[][132-137]{davidneoliber}

只有到邓小平1992年“南巡”之后,中央政府才全力推动向海外贸易和海外直接投资的开放。例如,在1994年,通过官方汇率变质50\%,双重汇率制(官方和市场)得以废黜。虽然这次贬值在国内激起了些许通胀危机,它却为贸易大幅增长和资本大规模流入铺好了道路,而后两者使中国如今拥有世界上最有活力和最成功的经济。\pagescite[][141]{davidneoliber}

劳动力工资优势。低增值生产部门领域(如纺织品)……1990年代期间,中国大陆开始转向高增值生产,有韩日台马新等国寻求价廉技高能力——中国大学系统培养出来的……。

中国在一个方面\textbf{显然偏离}了新自由主义的轨道。中国有大量剩余劳动力,如果要实现社会稳定和政治稳定,就必须要么\textbf{吸收},要么\textbf{压制}这些剩余劳动力。如果采取前一种方案,那么中国只能\textbf{靠债券融资大规模开展基础设施计划和固定资本}形成计划(2003年固定资本投资增加了25\%)。潜在的危险是,有可能发生\textbf{固定资本积累过度的严重危机(尤其是在建筑环境方面)}。有大量迹象表明存在生产能力的过剩(比如在汽车制造和电子技术方面)。而在城市投资中已经发生了爆发与亏损的循环。但是,这一切都要求中国政府偏离新自由主义教条,而像凯恩斯主义国家那样行事。这就要求中国保持对资本和汇率的控制……

让整个凯恩斯主义的战后\textbf{布雷顿森林体系崩溃}的条件之一,正是由于\textbf{美元逃离了美国的货币监管},致使欧洲美元市场形成。

购买美国债券,从而美国可以方便地\textbf{吸收这些国家的生产过剩}。但这使得美国很容易受到亚洲中央银行家们奇思异想的影响。相反,中国经济动力却受美国财政和货币政策掣肘。美国目前也以\textbf{凯恩斯主义}的方法行事——巨额增加联邦赤字和消费债券,同时坚持其他任何人都必须遵守\textbf{新自由主义规则}。这不是一个\textbf{可持续}的立场,如今美国有许多强有力的声音指出,这样做可能带来一场严重的金融危机。对中国来说,这可能导致从劳动力吸收的政治转向公开压制的政治。\pagescite[][148]{davidneoliber}

中国应该有资格算作新自由主义经济,虽然是“有中国特色的”新自由主义经济。\pagescite[][157]{davidneoliber}

国家和农民工都拒绝“工人阶级”一词,也否认“阶级是建构他们集体经验的话语框架”。他们也不把自己视作“资本主义现代性理论”一般所假定的契约主题、法律主体或抽象劳动主体”,……相反,他们一般会诉诸毛泽东传统的观念。……任何大众运动的目标都是要求中央政府实践自身的革命纲领,反对外国资本家、私人利益和地方权威。

它实现了经济快速增长并缓解了很多人的贫困问题,但同时也使得大量财富积聚到社会上层精英手里。……政党和经济精英正在不断合并,而这种现象在美国已经太普遍了。\pagescite[][156-157]{bibID}

反讽的是,在一个被认为由新自由主义规则统治的世界中,这两个国家的行为方式都好似凯恩斯主义国家。美国恢复了军事主义和消费主义的大规模赤字投资,而中国利用不良银行贷款的债券资助大规模基础建设和固定资本投资。忠实的新自由主义者无疑会提出,经济衰退表明新自由主义化还不充分或不完善。\pagescite[][159]{davidneoliber}

对于资产阶级上层精英而言,这将意味着撤回他们在过去三十年内积累的某些特权和力量。……他们唯一担心的是政治运动,因为可能有没收财产和革命暴动的威胁。

新自由主义化在何种程度上成功刺激了资本积累?真实数据表明,简直一无所获。……新自由主义化唯一能宣称的全面成功在于缓和和控制了通货膨胀。\pagescite[][162-163]{davidneoliber}

那么一些地方的成功恰恰遮蔽了以下事实:总体而言,新自由主义化无法刺激经济增长或提
高人民生活。第二,从上层阶级角度出发,新自由主义进程而非其理论确实是巨大的成功:
它要么重建了统治精英的阶级力量(如美国和某种程度的英国),要么为资产阶级形成创造
了条件(如中国、印度、俄罗斯等等)。……简言之,不管出什么问题,都是因为缺乏竞争
力,或因为个人、文化、政治上的缺陷。这样的论述宣称,在一个达尔文主义的新自由主义
世界里,只有适者才应该也能够生存。\pagescite[][164]{davidneoliber}

技术性变革远离生产和基础设施建设,转入市场导向型金融化所要求的轨道,而这是新自由主义化的标志。

关于哈维对掠夺性积累的论述也请看\pagescite[][166-168]{davidneoliber}

掠夺性积累包含四个主要特征:

1,私有化与商品化。……各种形式的公共设施(自来水、电信、交通)、社会福利供给(社会住房、教育、医疗卫生、养老金),公共机构(大学、研究室、监狱),甚至战争某种程度上都已经在资本主义世界之中和之外(如中国)被私有化了。 医药专利。

2,金融化。投机和掠夺的姿态。……日均金融交易流通总量……支持国际贸易和生产投资流量……经济松绑使得金融体系可以通过投机、掠夺、欺骗、偷窃,成为再分配活动的主要中心之一。存货促销、庞氏骗局、借助通胀进行的结构性资产破坏、通过合并与收购进行的资产倒卖、债务责任等级提升(甚至……全体人民变成债务奴隶),更别提通过信贷和证券操纵所进行的企业欺骗和资产掠夺了(借助证券和企业破产来掠夺和撤销养老金)——所有这些都是资本主义金融体系的核心特征。

3,危机的管理与操控。“债务陷阱”……首要方式。世界范围内的危机制造、危机管理、危机操控,已经发展成为一场精心的再分配表演,将财富从贫穷国家转移到富裕国家。…… 工人后备军……

4,国家再分配。国家一旦新自由主义化,就成为再分配政策的首要行动者,颠倒资金从上层阶级流向下层阶级的过程——这一过程发生于New liberalism时期。新自由主义国家还通过税法改革进行财产和收入的再分配,以利于投资收益而不是工资收入,增加税法中的累减原则(如销售税),强征使用费(如今在中国农村地区广为流行),为i企业提供一系列补贴h而免税政策。

假定市场和市场信号可以最好地决定一切资源配置决策,相当于假定任何事物原则上都可以被作为商品对待。市场被假定为一切人类行动的合适指引(一种伦理规范)。\pagescite[][173]{davidneoliber}

对资本家来说,此类个体不过是一个生产要素,虽然不是无差别的要素。

对劳工的总体打击从两个方面进行。工会和其他工人阶级的力量在各自国家内受到限制或消除(必要时会采取暴力),弹性的劳动力市场被建立起来。

另一方面的打击使得劳动力市场的时空坐标发生了转变。虽然人们可以利用“竞次”\footnote{王钦:竞次指一国利用降低社会福利、破坏自然环境、剥削劳动人民等方式,来获取国际竞争中的优势地位。}来获得最廉价和最顺从的劳动力,地理上的资本流动却使之能够统治全球劳动力,而后者自身的地理流动性有限。由于移民受到限制,被虏获的劳动力数量相当多。为了避免这样的壁垒,只有借助非法移民(产生很容易剥削的劳动力)或短期合同。

对于劳动力市场上如鱼得水的人而言,最终也只是和欲望嬉戏,个人带来的满足从不超过购物中心的有限身份认同,以及(对女性而言)美貌或物质财务方面的身份焦虑。伪满足…… 表面上激动人心,内心空空如也。

一支任由摆布的劳动力大军不可避免地会转变为其他制度形式,借此建立社会团结并表达集体意愿。从小型网络、毒品走私网络、黑手党、贫民区头目,经由社会、底层、非政府组织,到世俗性仪式和宗教团体,各种形式纷纷扩张。这些都是为填补社会空隙应运而生的替代性社会形式,因为国家力量、政治党派以及其他制度形式都被积极拆除,或干脆被当作集体努力和社会纽带的中心而扫除了。\pagescite[][179]{davidneoliber}

撒切尔,环境保护也有被利用来来将关闭煤矿和破坏矿工联盟的行为正当化。

不论我们多么盼望权利的普遍性,但推行权利的始终是国家。如果政治力量不愿意,那么权利观念仍只是一纸空文。所以,权利衍生于公民身份,也以公民身份为条件。\pagescite[][189]{davidneoliber}

有些之前的热情拥护者(如经济学家界福利……)和参与者(如索罗斯)如今都转向批判立场,甚至建议某种程度上回到改良版的凯恩斯主义,或回到更为“制度性”的方式以解决全球问题。\pagescite[][196]{davidneoliber}

凯恩斯,“食利人安乐死”

新自由主义化在国内引起的经济矛盾和政治矛盾,\textbf{只有通过金融危机才能遏制}。迄今为止,些金融危机被证明在当代破坏严重,但在全球范围内却可以得到控制。当然,可控性依靠的是相当程度上远离新自由主义理论。全球经济的两大动力源——\textbf{美国和中国——完全依靠财政赤字筹措资金},无疑就是有力的迹象,表明新自由主义作为一项确保未来资本积累的可行理论指导,就算没有实际死亡,也陷入了重重困难。这并不影响新自由主义理论继续被用来当作维持精英阶级力量之重建创建的巧言辞令。

在上层金融实力劫掠整个国家经济之前,频繁发生的金融危机通常表现出长期的经济不平衡。典型迹象是不断高涨且无法控制的\textbf{国内财政赤字、收支平衡危机、快速货币贬值、国内资产股价不稳定(例如在 房产和金融市场)、通货膨胀上升、失业率上升而工资下降、资本逃逸。}

巩固新保守主义式的权威主义……潜在的应对方案……新保守主义保持了建立不对称市场自由的新自由主义动力,但通过转而依靠\textbf{权威主义、等级制,甚至军事手段}来维持法律和秩序,从而将新自由主义的反民主面向凸显出来。……\textbf{民族主义}。

替代性方案,首先要开启一个政治进程……主要有两条路可走:我们可以参与到众多实际存在着的抗议运动之中,并试图从这些运动的激进主义中提取出一种具有广泛基础的对抗性方案的精髓。我们也可以诉诸理论和实践分析,探讨我们目前的状况(如我在此做的工作)并设法依靠批判分析推导出替代性方案。……我们的任务是……\textbf{加深集体理解}。\pagescite[][208]{davidneoliber}

美国民主的伪民主 \pagescite[][215]{davidneoliber}




















将自由化约为企业自由。

规训强力的地方工会,如以提倡个体劳动者的个人自由神圣不可侵犯为名,立法和治安策略,用以驱散或镇压反对企业力量的集体组织形式。削弱(如英美)、绕过(如瑞典)或暴力摧毁有组织劳工的势力,是一个必要的前提。去工业化,空间转移

秃鹰资本

阿根廷经济危机\pagescite[][108]{davidneoliber}


汪晖 《中国“新自由主义”的历史根源》http://wen.org.cn/modules/article/view.article.php/c8/2560 http://www.aisixiang.com/data/40003.html http://history.sina.com.cn/his/zl/2014-01-13/175379935.shtml


联合国HDR报告:缩小。过去几十年,全球相对不平等程度稳步下降,相对基尼系数从1975年的0.74下降到2010年的0.63,其主要推动因素为:经济飞速增长(主要是中国和印度)引起的国家间不平等程度的下降55。

 17亿71。中国等国家的贡献是全球中产阶级规模迅速扩大的主要原因,在中国,中产阶级家庭(年收入在11,500–43,000美元)从2000年的500万增加到2015年的2.25亿72。但不同国家对中产阶级的定义各不相同(虽然都采用将收入和支出与社会平均值进行比较的方法)73。

阶级已不是一个稳定的社会形态



出口导向型,进口替代型。不均衡地理发展


新自由主义赋予市场交换以如下地位:市场交换“本质上具有伦理性,能够指导一切人类行为,代替所有先前的伦理理念。”;就此而言,新自由主义强调市场中契约关系的重要性。新自由主义认为通过将市场交易的达成率和频率最大化,社会公益会因此最大化;新自由主义试图把一切人类行为都纳入市场领域。\pagescite[][3]{davidneoliber}


大卫·哈维在《新自由主义简史》一书中,认为中国的新自由主义转向


%%% Local Variables: %%% mode: latex %%% TeX-master: "../main" %%% End: