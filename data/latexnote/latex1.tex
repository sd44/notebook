\chapter{字体应用}

\section{中西文字号似乎不一样的解答}
\subsection{满庭芳的回答}
\url{http://bbs.ctex.org/forum.php?mod=viewthread&tid=155168}

解释:这里中西文字号其实是一致的,因为字号(或者在与后边提到的“字面”相对时,称
为“字身”)其实是一种名义上的东西,而且 ctex 宏集默认不会对中西文字号比例做调整
(相对地,日本的一些宏包、文档类,出于习惯,默认是调整的(日文里也有汉字,而且假
名跟汉字也比较类似),xeCJK 文档里作为例子的那个数值“0.962216”,其实就是日本常
用的一个缩放比例)。例如,对于汉字而言,通常都是正方形的(拉长、压扁的一般视为美
术字/美工字/艺术字之类,很少用于正文),那么汉字的小四号(zihao = -4)指的其实
是这个名义上的字身正方形的边长为 12 pt(现在外边一般用 PostScript 标准的“pt”单
位,但\textbf{在 TeX 里“pt”还是英美传统标准,这里 ctex 实际上设置的
  是 TeX 里的“12 bp”,等同于 PostScript 标准的“12 pt”,而非 TeX 的“12
  pt”)},至于实际上的汉字(字面)在这个名义上的字身里边能占多大比例(字面率),
各个字体根据其自己的风格、需求等等,不尽一致,而且,一般不会占到 100\%,因为字面
完全占满字身时,汉字之间就几乎没有空隙了,对于正文而言,阅读效果很不好(但对于特
大号的海报、广告等场合,这种做法有时可以起到很好的艺术效果)。对于西文,也类似
(其实西文至少还有个视觉尺寸(optical sizes)的问题(参
考 \url{http://bbs.ctex.org/forum.php?mod=viewthread&tid=154961},如果考虑的话更
复杂一点,这里忽略之)。在中西文字体搭配的问题上,如果比较讲究的话,字面率确实是
一个比较重要的因素。楼主选择的字体,比较巧的是,思源宋体、思源黑体是字面率比较大
的汉字字体,方正新楷体字面率虽不算大(事实上因为楷体自身的特点,字面率不可能太
大),但相较于旧的方正楷体(也就是方正现在宣传所谓“免费”的那个版本),也是略有
扩大的;反之,Libertinus、STIX2、XITS 这三种西文字体的字面率都略小,所以跟汉字配
合在一起时,就显得不是很一致。(事实上,虽然有人推荐拿 Libertine(Libertinus 的前
身)跟思源宋体搭配,但我个人并不赞同,字面率是一方面,更重要的是,Libertine 在西
文字体里属于“过渡风格”(Transitional),跟中文的 *稍微* 老一点的宋体风格,例如
方正书宋、方正新书宋(但新书宋可能有点粗了),配合会更好,而思源宋体的风格相当现
代,跟西文字体的“现代风格”(Modern / Didone)字体,例如 LaTeX 默认的 CM / LM 风
格的字体,配合会更好(但 CM / LM 字体比较细,按我个人的偏好,思源宋体
用 Light、SemiBold,思源黑体用 Normal、Medium 分别去配合 CM / LM 衬线体、无衬线体
的常规体、粗体比较合适)。至于 Libertinus Sans,前身是 Biolinum,前前身是 Optima,
是模仿古罗马碑刻的字体,跟汉字的黑体(尤其是风格现代的思源黑体)配合可能不太好,
我自己以前用方正北魏楷书试过(需要缩放汉字或西文),无论从风格还是来源、时代上,
都至少是强过通常的黑体的(即使是有“喇叭口”的比较老的黑体,看似跟 Optima 类字体
的笔画末端类似,其实放在一起时,风格并不一致)。)


事实上,比较新一点版本的 LaTeX 内核,在 XeLaTeX / LuaLaTeX 下已经默认使
用 OpenType 格式的 LM 正文字体了。

我自己的话,如果我自己有掌控的权力并且想用 CM / LM + 思源字体时,我会把 CM / LM 的无衬线体换掉,改用 Helvetica 的克隆版本:

\url{https://ctan.org/pkg/nimbus15}

\url{https://ctan.org/pkg/tex-gyre}

\url{https://ctan.org/pkg/gnu-freefont}

或者类似风格的,例如 Roboto 等:
\url{https://ctan.org/pkg/roboto}

原因是:CM / LM 风格的无衬线体,样式有些不全,而且粗体的笔画末端是圆的,粗体笔画
粗细变化也很明显,与汉字黑体配合不大理想。即使是常规体,x 字高(x-height)也比较
小(相对 CM 衬线体),风格上也有些独特(CM 的无衬线体貌似不是或不完全是 Knuth 设
计的,根据某篇文章,似乎是 Richard Southall,不过我暂时还没时间看太多的字体方面的
资料)。CM / LM 无衬线体的一些再衍生版,例如 CM Bright,也或多或少有上述不利于与
汉字配合之处。

Helvetica 比 CM / LM 无衬线体稍微粗一点,跟衬线体的区分度更大一些,不过这时候思源
黑体就要改用 Regular 和 Bold 字重了,而且 Helvetica 的字面率实在有点太大了,我是
把它缩放到 95\% 使用的。

不过 CM / LM 无衬线体也有其优势:sansmathfonts 字体包
(https://ctan.org/pkg/sansmathfonts)是目前为数不多的比较完整的无衬线数学字体之
一,尽管是 Type 1 格式,而且设计上也有很多不足,但对于幻灯片而言,可能还是比较好
的选择了。CM Bright 虽然也是无衬线体、也有数学字体,但样式不太全,而且有些过于纤
细,用在幻灯片上效果未必好;Helvetica 其实也有数学字体,但是是商业的,不是自由字
体。最近新出了一个 gfsneohellenicmath 字体包
(https://ctan.org/pkg/gfsneohellenicmath),是与 gfsneohellenic 正文字体包
(https://ctan.org/pkg/gfsneohellenic)相配的(GFS 是有名的非营利的希腊文自由字体
制作组织,部分字体也包含拉丁字母,但我个人不太喜欢它的设计理念),这 *很可能* 是
目前(2018-06-06)唯一的(无论商业还是自由)OpenType 格式、Unicode 编码的无衬线数
学字体了,只是风格更为独特,中文用户估计不太习惯。

\subsection{刘海洋关于无衬线数学字体的回答}

\url{https://www.zhihu.com/question/46196562}

iwona 与 kurier 是文档中介绍的另两种相近的字体,有完整数学符号。风格比较特别,我没用过。

beamer 默认用的是它自己拼凑的 CM 字体,数学字母部分是用 CM 的无衬线正文字体拼的,
没有无衬线数学符号。

cmbright 是较细的无衬线 CM 字体,没有无衬线符号。

符号做得比较全面的是 arev 字体包。主体风格是 Vera 字体(arev 是 vera 倒过来写),
字体比较宽大。

iwona 与 kurier 是文档中介绍的另两种相近的字体,有完整数学符号。风格比较特别,我
没用过。

不要用 Comic Sans。

当然这篇文章之后也有少量其他免费字体发布,还有一些商业字体也有无衬线的。

    sansmathfonts 字体包是对 CM 数学字体无衬线加强版,与 beamer 原来风格一致,但有无衬线的数学符号、AMS 符号,效果好些。

    kpfonts,默认是衬线体,但可以用 \mathversion{sf} 切到无衬线。