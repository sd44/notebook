
凯恩斯主义经济学的兴起、发展及演变的过程,反映了国家垄断资本主义经济的发展趋势,
反映了资本主义社会各种矛盾加深和国家垄断调节制度的变化。……尽管他们承认资本主义
经济存在周期波动,存在生产过剩的经济危机,存在非自愿失业,也提出相应的解决问题的
政策主张。然而,由于不能深入到资本主义制度内部去寻找真正的根源,是不能根本解决
上述资本主义经济问题的。

其理论模型的主要特色仍是非市场出清模型,其政策主张的基调仍然是政府干预调节经济。
这些理论与政策主张,在一定程度上反映了资本主义经济的现实。因为,现代资本主义国家,
显然是不能实行纯自由经济政策,也不能实行纯政府干预政策。\pagescite[][36-37]{guojiaganyu}

储蓄与投资的差额是吸引投资和扩大就业的动力。

1936年《就业、利息和货币通论》……主张国家调节资本主义经济,提出了走出严重经济危
机的方案:通过增加财政支出来兴办公共工程,从而达到增加社会需求,增加就业岗位、缓
和危机、缓和失业的目的。

1933年罗斯福新政,政府在“反危机”的名义下,建立了各种“调节经济”的结构,举办
“公共工程”以扩大就业,通过贷款、补贴、减税、订货、收购剩余产品等办法来扩大垄断
资本的实力。在英国和法国,实行了某些部门(英国的电力输配网、法国的铁路)的“国有
化”。

《通论》 主要研究就业理论,研究什么因素决定总收入从而决定总就业量。他提出在短期内,
供给不会有大的变化。所以,一国的就业水平是由总需求或有效需求决定的。而有效需求取
决于三个基本心理因素的作用,也就是取决于“消费倾向”、“对资本未来收益的预期”以
及对货币的“灵活偏好”的作用。……他认为,自由竞争的机制不能自动地达到充分就业的
均衡,所以主张政府干预经济,通过政府的决策,特别是通过财政政策来刺激消费和投资,
达到有效需求的提高,最终充分就业。\pagescite[][38-39]{guojiaganyu}

他将“社会结构”及分配关系抽象掉,这样,他就不从本质上来研究社会失业等经济问题了。
然而,《通论》也有自己的独到之处,这就是基本上采用了短期的比较静态的总量分析方法。

所谓总量的分析方法,就是分析整个社会经济的总供给、总需求和总价格的均衡关系。

不自愿失业,就是指由于社会对商品的需求不足,以致不足以使生产吸收愿意工作的人去工
作,又称为需求不足的失业。也就是说,即使工人愿意接受比当前实际工资更低的工资,也
仍旧找不到工作。\pagescite[][43]{guojiaganyu}

有效需求就是商品的总供给和总需求处于均衡状态时的总需求,或是总供给价格和总需求价
格相等时的总需求价格。

凯恩斯认为,在短期,收入和就业水平是由总需求决定的。……消费需求和投资需求……生
产过剩、工人失业。\pagescite[][47]{guojiaganyu}

凯恩斯认为,一个社会的消费量是受以下几种因素的影响,即:1.该社会的收入;2.客观环
境;3.主管需要、心理倾向、习惯和收入分配原则。

“一般而论,当所得增加时,人们将增加其消费,但消费只增加,不若其所得增加之
甚。”……他指责以前的消费不足论者太着重于增加消费而忽略了扩大生产的另一途径——增
加投资。……双管齐下 \pagescite[][50]{guojiaganyu}

凯恩斯认为,随着收入的增加,消费将越来越不足,所以需要以投资来弥补在收入和消费之
间的缺口,而且,由于消费倾向是比较稳定的……

$K = \Delta Y / \Delta I$ 或 $\Delta Y = K * \Delta I$

初级效应:投资增加,投资品工业的生产、就业量、收入增加……

刺激效应:以上引起消费品工业生产的发展……就业和收入增加。次级效应的结果,是就业
总量大于投资品工业中的就业的增加量。所以,增加投资支出的次级效应是乘数原理的关
键。

$ K=1/(1- \Delta C/ \Delta Y)=1/(1-消费增量/收入增量) =1/(1-边际消费倾向)=1/边际储蓄倾向$

边际消费倾向越大、边际储蓄倾向越低,则乘数越大。……这样,乘数原理就成了凯恩斯鼓
吹国家刺激投资政策的理论依据。

但是,边际消费倾向只是一个主观的心理法则,资本主义生产的现实不一定按照这个法则来
运行。资本主义生产的扩大并不直接依赖于消费的增长,生产的直接目的与动力,是获取最
大限度的利润,生产和消费是最终反映出来的。同时,资本主义生产的扩大也不导致消费的
相应增加。因而,乘数原理的应用有一定的局限性。

凯恩斯提出在应用乘数原理时,要考虑下列一些限制条件。即:1.如果在投资品的生产部门
中,增加的收入用于偿还债务,则乘数会缩小;2.如果增加的收入用来购买消费品的存货,
则乘数也将缩小;3.如果增加的收入虽然是用来购买消费品,但是,由于生产条件的限制,
消费品生产不出来,或由于缺乏原料、燃料、生产工具等,这是货币的国民收入会增加,但
实际的国民收入不能同比例的增加。4. 如果增加的收入是用来购买外国的消费品,乘数也
将缩小。这样,凯恩斯认为,由于经济周期各个阶段发展情况不同,乘数的实际值可以不同。
由于边际消费倾向总是小于1,所以,成熟不会是无穷大。他估计乘数值大约为3左右,成熟
虽大于1,但不会很大。这样,新的投资增加后,引起的收入增加可以大于投资的增加,可
以促进就业增加,但仍然不能使就业水平达到充分就业的地步。

资本边际效率:基本心理法则二。

生产资料——资本品——资本资产……投资支出就是用于获得以上这些项目的支出。……企业购
买一项资本资产,实际上是购买了一系列的未来收益的权利……未来收益的权利……预期收益,而非实现了的收益。\pagescite[][53]{guojiaganyu}

\[ V = R_1(1+i) + R_2(1+i)^2 + \cdots + R_n(1+i)^n \]

V是通过贴现的方式计算出来的,贴现是复利的逆运算。……资本资产的供给价格是指足以
引诱企业再增产一单位该种资产的价格。这个价格不是现有的该类资产的市场价格,而是重
新再生产这种资本所花费的成本,它又叫“重置成本”。从投资的角度看,企业家只有在一
项资本资产的现值至少等于它的供给价格时,才打算购买,

假定一项资本资产的供给价格为C,它的一系列预期收入的现值 $V=C$,则:
\[ C = R_1(1+i) + R_2(1+i)^2 + \cdots + R_n(1+i)^n \]

r代表了一项资本资产的预期收益和它的供给价格之间的关系,凯恩斯将r称为“资本边际效
率”。

“我之所谓资本之边际效率,乃等于一贴现率,用此贴现率将该资本资产之未来收益折为现
值,则该现值恰好等于该资本资产之供给价格。”实际上,所谓资本边际效率就是使用该资
本资产的预期利润率。

资本边际效率是一种预期收益,它的特点是不确定性,有下降趋势。……利用贷款进行投资,
是要支付利息的。使用自有资金来投资,现行市场利息率是使用自有资金投资的机会成本。
当企业家进行投资决策时,往往将利息率与资本边际效率作比较。

流动偏好:基本心理法则之三

流动性是指,一种资本在不损害其原有价值的条件下,变成现金的难易程度。 而流动偏好,
就是指人们总喜欢手头上保留一部分现金,以便灵活地应付各种需要的一种心理因素。

货币需求是由以下动机决定的。其一,交易动机。其二,谨慎动机。其三,投机动机。第一
、二种和货币充当流通手段的职能有关,凯恩斯将它们合称为交易性货币需求。投资动机是
和货币充当储藏手段的职能有关。

债券市场价格的变动和利息率有着密切关系,二者之间存在着反向变动的关系。流动偏好法
则,就是指在货币数量不变时,利息率决定于流动偏好。

利息率是企业家投资需求的极限。货币数量是由银行体系控制的。在货币数量不变时,利息
率就决定于流动偏好,但是利息率总会保持一定的高度。这是因为,人们不仅对货币的流动
性具有偏好,在心理上也偏好保持货币,他必须取得一定的利息才肯贷出货币。所以,利息率
总会保持一定的水平。流动偏好这一心理法则就削弱了对投资的引诱力,阻碍了企业家继
续扩大投资,从而会造成投资需求的不足。

由于消费倾向是一个比较稳定的函数,在收入不变时,消费需求也是相对稳定的。所以,凯
恩斯特别强调投资的作用,把投资看作是填补总需求与消费需求之间缺口、扩大就业量的主
要因素。但是,在资本边际效率与流动偏好这两个基本法则的作用下,资本预期利润率与利
息率之间的差额,不足以刺激企业家继续扩大投资,投资需求不足使得有效需求不足,有效
需求不能达到充分就业所要求的均衡水平,因而,失业必然存在。 三个基本心理法则就是
如此构造了凯恩斯就业理论的框架。。

\subsection{通货、利率}

在凯恩斯之前,传统的货币理论是以自动的充分就业均衡为前提,是采用了“两分法”,即
货币论与价值论相脱离,认为货币是中性的,即货币对经济过程不起任何影响。

在凯恩斯的经济理论体系中,以资本边际效率为主要内容之一的投资理论,需要货币将企业
家的未来预期利润同现在的投资决策联系起来;以流动偏好去为主要内容的利息理论,需要
通过货币数量的变化来影响利率,从而影响投资;物价理论同货币不可分离。总之,凯恩斯
将货币理论与价值论结合在一起。\pagescite[][58]{guojiaganyu}

$费雪方程式 MV=PT,在T达到最大限度的情况下,V是一定的。$

马歇尔将分析的重点由货币的供给转向了货币的需求。除购买商品和劳务外,还希望持有一
定的现金。$M = KTP$,K代表一国居民希望持有的货币余额占国民收入的比例。无论是费雪
方程式还是剑桥方程式,都是以货币仅仅是流通手段来谈的。

凯恩斯的流动偏好这一心理法则,就使货币的职能不仅是充当流通手段,而且还充当了储藏
财富的手段。

凯恩斯认为,三大动机中投机动机是最重要和复杂的一个,与货币充当贮藏财富的手段有关。

交易性货币需求, $ M_1 = L_1(Y)$,$M_1$代表为满足交易性需求所持有的现金数,
以 $L_1$代表交易性货币需求函数,$L_1$主要取决于收入水平$Y$。那么$L_1$是否会受利
息率的影响呢?凯恩斯认为,二者之间的关系是间接的,如果利息率降低了,它刺激了收入
上升,则$M_1$也会增加。虽然二者增加比例不一定相同。

投机性货币需求,与利息率有密切关系。当高利息率时,货币需求量低,反之则高。在达到
最低利息率水平时,货币需求量成为完全有弹性。如以$L_2$代表投机性货币需求函数,它主
要决定于当前利率与当前预期状态的关系。$M_2$代表为满足投机动机所持有的现金数,
则$M_2 = L_2(i)$。

传统货币理论认为,一般价格水平只随货币数量作同比例的变动。凯恩斯认为,这个理论是
建立在三个假设前提上,即假设宏观经济处于静态的均衡中,假设充分就业,假设货币流通
速度不变。所以,在达到充分就业后,物价水平是和货币数量同比例的提高。这时,传统的
货币数量论是正确的。

那么,未充分就业呢?凯恩斯认为,货币数量的增加对经济,从而对一般物价水平的影响是
有个过程的,这首先取决于货币数量的变化对有效需求的影响如何。$货币数量增加
\rightarrow 满足投机性货币需求的数量增加 \rightarrow 利率下降 \rightarrow 刺激投
资需求、增加有效需求 \rightarrow 收入、就业、产量相应增加 $。然后,由于以下的原
因,必将使得价格上涨:

1.工人或机器设备的效率不一致,会导致短期内的报酬递减。……在还没有达到充分就业之
前,物价就开始上涨。

2. 生产要素的不完全替代性。当就业量增加时,如果有的技术工人已经充分就业,而其他
生产资源还有闲置,那么,要继续增产,就会遇到“瓶颈”现象——一定的技术水平的工人的
供给在短期内是完全无弹性。供不应求,技术工人工资上涨(其它生产要素如机器设备等也
有类似“瓶颈现象”)。一般物价水平上升。

3. 由于就业增加,使得不少工种,尤其是技术工种的工人的议价地位得到加强。

但是,如果不存在以上几种原因,如果将社会情况简化为:1.所有失业资源,就生产效率来
说,是完全相同的,可以互换的。2.边际成本中的各生产要素,只要还没有全部就业,就不
要求增加货币工资。那么,只有还存在着失业,工资单位就不会变动,生产报酬就不会变动。
因此,货币数量增加时,物价水平不会受影响,而有效需求会随着货币数量的增加而增加,
就业量随着有效需求的增加而同比例的增加。以后,当达到了充分就业以后,货币数量增加,
物价就上升。货币对物价的影响,充分就业是“最后的临界点”。这就是凯恩斯改良后的货
币数量论。

凯恩斯的这个理论表面,相对于货币数量的变动,已无增加产量作用,产量的供给弹性降到
零时,才会出现“真正的通货膨胀”。在产生真正的通货膨胀之前,可能出现“半通货膨
胀”。凯恩斯指出,现实中的情况很复杂,例如:1,当货币数量改变时,有效需求不一定与
它同比例的改变;2,生产要素的性能并不一样,所以当就业量逐渐增加时,报酬将递减,而
不是不变。3,资源并不可以互换,所以,当有些商品已经达到了供给无弹性,而有些商品还
有闲置资源可供生产之用。4,在充分就业之前,工资单位有上涨的趋势。5,边际成本中的
各生产要素的报酬,不是以同一比例改变的。……当就业量增加时,一方面是就业量增加,
产量增加;另一方面又引起成本增加,从而物价上涨,物价上涨不及有效需求增加得快,这
种现象,凯恩斯称作“半通货膨胀”。只有达到充分就业后,在货币数量增加时,物价将随
之同比例的上升。

\subsection{利息理论对吗}

凯恩斯理论体系中,利息理论占有重要位置。利息率既是资本边际效率基本心理法则的组成
要素,也是流动偏好基本心理法则的组成要素,利息率是沟通资本主义经济中的商品领域和
货币领域的关键因素。

传统经济学把利息看作是储蓄的报酬,认为利息率具有伸缩性,所以,把利息率看作是使投
资和储蓄趋于平衡的因素。凯恩斯不同意以上观点。他认为,利息是在一特定时期内放弃流
动偏好的报酬,而不是储蓄的报酬。……利息率是由既定情况下的流动偏好和货币数量两个
因素决定的,也就是货币的供求情况决定了均衡利息率,而不是由储蓄的需求(即投资)和
储蓄的供给这两个因素决定的。当货币数量不变时,利息率取决于流动偏好;当流动偏好不
变时,利息率取决于货币数量。利息率的高低,首先决定于流动偏好程度的大小。其次,它也
决定于货币供给量大小。

凯恩斯认为,利息率是一种价格。均衡的利息率,使得人们愿意以现金持有的财富(即投机
性货币需求)恰恰等于现有的现金量。如果市场利息率低于均衡水平,现金脱手所得的报酬
减少,投机性货币需求将增加,将超过现有的货币供给量。反之……利息率的波动又因货币
的供求关系而趋向于均衡水平。利率下降,货币需求大于货币供给……

在分析利息率变动对货币供求的影响时,凯恩斯提出了“流动陷阱”现象。这就是指,由于
市场利息率越是低于正常的利息率,则作为财富增加的货币需求越是增加,当利息率下降到
一定水平时,因利息收入太低,所以几乎人们都愿意持有现金,而不愿意持有债券。出现的
这种现象就是“流动陷阱”(又称为“凯恩斯陷阱”)。出现这种现象时,相对于利息率来
说,作为财富持有的货币需求变成完全有弹性,利息率的微小下降,会使这种货币的需求无
限大。

流动性陷阱是凯恩斯提出的一种假说,指当一定时期的利率水平降低到不能再低时,人们就
会产生利率上升而债券价格下降的预期,货币需求弹性就会变得无限大,即无论增加多少货
币,都会被人们储存起来。发生流动性陷阱时,再宽松的货币政策也无法改变市场利率,使
得货币政策失效。\pagescite[][66]{guojiaganyu}

\subsubsection{储蓄和投资的关系}

传统经济学认为储蓄支配投资,因为投资来源于储蓄。

凯恩斯不同意上述看法。他认为,投资与储蓄是分属于不同经济主体的不同动机的经济行为。
这就是说,投资是公司、企业家的行为,其目的是为了获得利润,它受资本边际效率和流动
偏好这两个心理法则的制约。而储蓄是居民、企业进行的经济行为。对居民的储蓄来说,利
息的调节作用不如企业那样显著。同时,利息率又具有下降的刚性。这样,储蓄和投资不能
通过利息率的自动调节来达到均衡。不仅如此,因为储蓄和投资是不同经济主体的经济行为,
所以,居民储蓄的增加,并不意味着投资的增加,只意味着他的消费的减少,从而,影响就
业的增加。只有投资增加了,才能使就业、收入增加,从而因收入的增加而增加储蓄。因此,
凯恩斯提出。有效储蓄的数量乃决定于投资数量。

\subsection{工资有伸缩性吗}

传统经济学认为,工资率具有充分的伸缩性,所以,劳动市场可以自动地趋向充分就业的平
衡。

凯恩斯认为,现实生活中,货币工资具有下降刚性的特点。工人反对,特别是工会力量增强
的情况下。另外,降低了货币工资,对个别企业来说,虽然能起到刺激投资的作用,但对全
社会来说,都是降低了消费需求,从而降低了总需求,这会导致市场萎缩、投资减少、失业
增加。在总需求下降的同时,产品价格会上涨,如果货币工资是与物价下降同步,那么,实
际工资也不可能下降,或者下降幅度较小。因此,凯恩斯认为,不能以降低货币工资来增加
就业。

当然在经济衰退时,应当通过降低工资的办法来提高资本边际效率,达到刺激投资、摆脱困
境的目的。但是,这是不应直接降低货币工资,而是通过通货膨胀,提高物价的办法,使实
际工资下降。这样做,一是可以刺激投资,二是可以因工人产生了“货币幻觉”,而不起来
反抗。

在工资理论上,凯恩斯与过去的传统经济学也有共同的点,这就是都以牺牲工人的利益为代
价,来刺激经济的发展。

\subsection{“恐慌会发生”}

凯恩斯把经济的周期性发展称之为“商业循环”,他提出资本主义经济发展经历了繁荣、恐
慌、萧条、复苏、再到繁荣等阶段。他将经济危机称之为“恐慌”。

为什么有商业循环?资本边际效率是周期性发展的主要因素。

凯恩斯是从危机的来临来分析商业循环运动的。在繁荣扩张时期,投资迅速增加,对未来信
心十足,资本边际效率上升,就业逐渐增加。而由于乘数的效应,新投资每增加一下,都刺
激了消费需求,结果收入更加增加。但是,高的资本边际效率这时会受到两个不同方向的压
力:1,由于原料或劳动力不足以及瓶颈状态的扩大,使新资本资产的生产成本增加。2,由
于资本资产迅速完成,产量逐渐增加而使收益降到预期以下。i这两种压力都会使资本边际
效率趋向下降。

怀疑……悲观……危机

危机发生后,如果利息率下降,这有助于经济复苏。但是,在实际上,由于资本边际效率已
大幅度的下降,以致利息率无论如何降低都不足以使经济复苏,这时,经济处于萧条阶段。

资本边际效率的崩溃是经济衰退乃至危机的主要原因,资本边际效率的恢复是经济复苏的必
要条件。资本边际效率是个心理法则。

将资本主义经济的周期性、经济危机爆发的原因,主要用心理法则来解释,是不科学的。凯
恩斯对企业家心理因素、预期的分析观点,成为他提出国家干预经济的理论依据。

\subsection{关于国际贸易的观点}

凯恩斯从增加国内就业的角度出发,来评述重商主义。他认为投资引诱可以来自对国内投资
和对国外投资(这包括贵金属的累积)这两方面,它们组成了总投资。……在自由竞争条件
下,政府关心贸易顺差是必要的,因为它有助于增加国内就业量。

但是,重商主义政策不能推行过度。因为,一国顺差增加了,使国内利率下降,投资额增加,
就业量增加。当就业量增加超过了一定的临界点,就会使工资成本增加;同时,国内利息率
下降如低于国外的水平,就会刺激对外贷款,超过顺差额,就可以引起贵金属外流。这两种
情况,都会使贸易顺差产生相反的结果,不利于国内经济发展。它所带来的好处,“只限于
一国,不会泽德全世界”。

总之,在涉及国际贸易的有关问题上,凯恩斯的基点是反对经济自由。这和他对国内经济分
析得出的政策主导思想—— 主张国家干预经济是一致的。

\section{解决失业的处方}

\subsection{指导思想:国家调解与干预经济生活}

在凯恩斯看来,当失业情况严重时,整个经济发展是处于衰退、恐慌阶段之中,这时候,凭
借市场机制的自由调节是无法将社会经济带出谷底的。因为就业量取决于消费需求和投资需
求。消费需求取决于消费倾向,边际消费倾向趋于递减,这样,就业量就主要取决于投资需
求。投资需求是由利息率和资本边际效率决定,其中,因流动偏好这个心理法则作用,利息
率的降低呈现刚性,他有一个最低限度。因此,投资量的大小(需求的强弱)就主要取决于
资本边际效率。资本边际效率取决于重置成本(资本资产的供给价格)和预期收益。由于人
们的预期很不确定,资本边际效率也就不稳定。……既然是这样,显然,靠经济自由的任何
政策都无法使经济摆脱困境。

“世界上不能再长久容忍失业现象”,他认为,20世纪的失业问题,“是和今日之资本主义
式的个人主义有不解之缘的”。

“因为要使消费倾向与投资引诱二者互相适应,故政府机能不能不扩大。”国家干预经济恐
怕是对个人主义的极大侵犯,但是,凯恩斯认为“这是唯一切实的办法,可以避免现行经济
形态之全部毁灭,又是必要条件,可以让私人策动力有适应经济形态之全部毁灭;又是必要
条件,可以让私人策动力有适当运用。”

主要方针。“国家必须用改变租税体系、限定利率以及其他方法,指导消费倾向。”并且,
要“把投资这件事,由社会来综揽”。但是,他指出,这样做并不是要实现国家社会主义,
将社会上大部分经济生活都包罗在政府的权限以内,而是要求“国家能够决定(a)资源之
用于增加生产工具这,其总额应为若干”;(b)持此种资源者,其基本报酬应为若干”。
做到这些,国家就已尽其职责了。他一再强调,消费倾向与投资引诱这两项,“必须由中央
统制,以便二者互相配合适应”,除此之外的经济活动,仍由私人策动,在私人策动的范围
内,“个人主义之传统优点还是继续存在”。……

由国家的权威和私人的策动力量互相合作,凯恩斯的“公私合作”的思想,后来发展为萨缪
尔的“混合经济”思想。

\subsection{政策的重心:财政政策}

国家对经济生活的调节和干预具体是通过财政政策和货币政策的实施来实现的,对这两种政
策,凯恩斯偏重于财政政策。因为货币政策是通过调节利息率来起作用的。从他的理论体系
分析来看,利率对实现充分就业不是起关键的作用。由于流动偏好的存在,使利息率的降低
有其极限;由于资本的边际效率是不稳定的,这样,有效需求很难达到充分就业所要求的水
平。……尤其是出现经济衰退、经济危机时,资本边际效率突然崩溃,它所下降的程度,使
得利息率无论如何降低,因对未来丧失信心,都不能刺激经济达到复苏的地步。“就我自己
而论,我现在有点怀疑,仅仅用货币政策操作利息率到底会有多大成就……理由是:各种资
本品之边际效率,在市场估计办法之下,可以变动甚大,而利息率之可能变动范围太狭,恐
怕不能完全抵消前者之变动。”

凯恩斯的财政政策的主要特点是实行举债支出。所谓举债支出,是指政府用举债方式进行投
资事业和弥补其他预算项目的赤字。凯恩斯认为,政府用举债方式来举办投资事业,这能增
加投资;弥补其他预算项目的赤字,是负储蓄,这能增加消费倾向。所以,实行举债支出的
财政政策,能增加有效需求,能增加就业量。运用这个财政政策,在经济衰退时期,对经济
走出低谷是有重要作用的。

一般在经济衰退时,在财政政策上,应扩大政府开支和实行减税,以刺激投资与消费的增加。
凯恩斯认为,税收的变动……但是,在短期内消费倾向是稳定的,这样,单靠税收政策就不
足以使有效需求扩大。另外,扩大政府支出固然能增加需求,但是,政府支出扩大的来源如
果是靠增加税收,就又会使私人投资和私人消费减少,这就达不到扩大有效需求的目的。所
以,政府开支的扩大,只能通过举债支出,实行赤字财政政策了。凯恩斯说“举债支出虽然
浪费,但结果到可以使社会致富。”凯恩斯的这种财政政策思想,打破了自亚当斯密以来的
传统观念——保持国家的预算平衡(财政支出等于财政收入)。

在凯恩斯的财政政策主张中,包含有“浪费致富”的想法。他说,借口采金,在地上挖窟窿;
建造金字塔,都可以增加财富。他甚至认为,“地震、战事等天灾人祸,都可以增加财
富。”由于浪费财富,凯恩斯主张政府扩大开支,也包括扩大非生产性的支出……已达到有
效需求的增加,解决失业问题。

\subsection{辅助政策:货币政策}

货币政策的主要内容是,通过中央银行调节货币供应量,以影响利息率变动,从而间接影响
社会总需求。在经济衰退时,中央银行增加货币供应量,货币的总供给增加以后,可以使一
般为满足投机需求的货币(M2)增加,其结果是利率下跌。利息率下降了,可以刺激投资,
投资增加后可以使收入加倍增加。随着收入的增加,交易所需的货币量(M1)也增加,于是,
增加的货币的总量就会有一个适当的比例分配与M1和M2之间。货币政策有多大效果,这就要
看因满足投机所需的货币M2的增加使得利润率能下跌多少,以及投资增加后又能使收入增加
多少。怎样来调节货币供应量?中央银行可以通过调整贴现率、法定准备金及公共市场业务
等措施来影响货币的供应,从而影响利息率。在经济衰退时,设法降低利息率来刺激投资需
求;在经济繁荣时,设法提高利息率来遏制投资需求。

相对于财政政策的作用,凯恩斯认为货币政策是通过利息率对有效需求起间接作用的,所以,
步入财政政策对有效需求的增加来得直接而有力,因此,货币政策的运用在国家调节干预经
济活动中,是起着辅助作用的。

在经济衰退时期,实行扩展性的财政政策和扩张性的货币政策;在经济高涨时期,实行紧缩
性的财政政策和紧缩性的货币政策。实行扩张性的政策,即增加开支,减少收入,必然要引
起财政赤字。赤字靠什么来弥补?或是靠举债支出,或是靠增加货币供给(滥发钞票),但
这样做的结果,会引起通货膨胀。但我们前面已介绍过………凯恩斯认为,当大量失业存在时,政
府实行财政赤字不必担心通货膨胀的威胁。《通论》的写作正经历了世界性经济大危机,因
而,凯恩斯是把扩张的财政政策和货币政策作为他的政策主张的主要内容。

\subsection{其他政策主张}

凯恩斯的政策主张出了财政政策和货币政策外,还提出了有关解决财富和所得分配不公的思
想,从中反映出他的社会改良主义思想。

\subsubsection{解决财富与所得的分配不公}

他认为,在今天的资本主义社会中,显著的缺点有两个,即不能提供充分就业和财富的分配
不公平合理。这两个缺点是相互密切联系的,由于分配不公,就影响有效需求不足,从而不
能实现充分就业,更何况,现在分配不均已到了无法辩护的程度了。所以,要达到充分就业,
除了前面所介绍的政策措施外,还要解决分配不公平问题。他说,如果“采取步骤,重新分
配收入,以提高消费倾向,则对资本的增长大概是有利无弊的。”

两点设想。

(1)加强对富人直接税的征收。……财富过多集中于富人之手,他们又厉行节俭去储蓄,
这必然大大降低社会消费倾向,不利于有效需求的增加……

所以,在当代情形之下,财富的生长不仅不取决于富人之节约,相反,恐怕反而受这种节约
的阻挠。

(2)降低利息率,消灭食利者阶级。第一,妨碍投资引诱的主要因素是高利率。……为降低
利息率,以促进投资要求,需要消灭食利者阶级。而且,利息降到很低的程度,仅靠利息为
生的食利者阶级就自然地消灭了。第二,从利息的性质看。凯恩斯认为,利息与地租的性质
相同,它们并不是真正牺牲的代价。资本和土地都稀少。但资本又不同于土地。土地天生供
给受限。资本不是天生,它可以通过国家举办集体储蓄来增加到免除其“稀力性”的程度。
他预计这个阶级的消除是自然死亡,并不是骤然的,因而不需要革命。

\subsubsection{对外经济政策}

贸易顺差,但不能过火,但不能引起毫无意义的国际竞争,引起别国报复。

\section{理论的变革、走向}

\subsection{理论的变革、走向}

我们认为,所谓凯恩斯革命,主要表现在以下几个方面。

第一,《通论》在理论上否定了“萨伊定律”,否定了传统经济理论的市场均衡的假定,因
而对经济危机、失业的解释取得了突破。

以马歇尔为代表的新古典经济学接受了“萨伊定律”,而新古典经济学在30年代的大危机前,
在英美等主要资本主义国家是占据支配地位的经济学理论。

凯恩斯的《通论》是与新古典经济学的理论大相径庭的。

新古典经济学的理论包含了这样一些内容:
\begin{enumerate}
\item (厂商对)劳动的供求。编辑生产率理论,供需平衡的调节。厂商对劳动的要求和真
  实工资成反方向变动。而劳动的供给和真实工资(货币工资的购买力)成正向的函数关系,
  真实工资增加,劳动的供应量也增加。……工资的伸缩性,将使劳动的供给和需求趋于平
  等。劳动供需平衡时,即充分就业量,也就是说,愿意按照平衡时的实际工资就业的人都
  已就业了。除此之外为自愿失业或摩擦性失业。


\item 所有经济活动的目的是消费,消费依赖于取得的收入,收入则来源于生产,每一生产
  行为都代表了对某种物品的需求。这样,总需求不足不可能产生,全面的生产过剩不会发
  生,只是有可能发生局部的生产过剩。过剩行业或少利行业可以转移至其他行业,非自愿
  失业不可能产生。


\item 利息率是可伸缩的,它是调节投资与储蓄趋于均衡的因素。储蓄和利息率同方向变动,
  投资和利息率成反向变动。当储蓄大于投资,利息率必然下跌,这又使储蓄减少,刺激投
  资增加。当储蓄大于投资,利息率必然下跌,这又使储蓄减少,刺激投资增加。……当人
  们进行储蓄时,总需求所存在的缺口,由于储蓄会自动转化为投资,所以,不可能产生总
  需求不足,利息率是使储蓄转化为投资的机制。


\item 货币数量论。货币数量与物价成正方向变动。M(货币供应量)V(货币流通速度)=P
  (一般价格水平)Y(真实收入)。充分就业前提下,V、Y是不变的,所以,M与P成正比
  关系。货币数量论是萨伊定律基础。由于货币只是作为流通手段,人们出售商品所得的货
  币,就只用于购买其他商品。……不会出现总需求不足的情况。

\end{enumerate}
总之,在“萨伊定律”和完全竞争的前提下,新古典经济学认为资本主义经济有自然趋于充
分就业均衡的倾向,但经济一旦处于不均衡时,市场经济的各种机制会使它恢复均衡。……
市场可以出清,资本主义经济因此不会发生总需求不足(或者生产普遍过剩)的经济危机。
1929经济危机,新古典经济学因无法解释持续存在的大量失业现象,本身也陷入了危机。

凯恩斯否定了“萨伊定律”,认为它不适用于货币经济。在货币经纪,货币出了流通手段还
具有价值储藏的职能,所以,一种出卖行为并一定是购买行为的继续。……凯恩斯认为社会
总产量不是既定的,它取决于有效需求的大小。由于三大基本心理法则作用,社会总需求存
在不足,存在着非自愿失业,存在着恐慌(生产过剩的经济危机).

所以,凯恩斯在理论上,革了萨伊的命,革了马歇尔为代表的新古典经产学的命。

第二,《通论》在研究方法上,以宏观总量分析代替微观个量分析,开创了现代宏观经济分
析。不是从个人或家庭角度考察,而是强调从国家角度。凯恩斯的《通论》是以宏观经济为
分析对象,用总量的分析方法。他研究各个经济总量——总产量或国民收入总量、总消费、总
投资、总就业等——的变动以及相互关系。他的宏观总量分析是以收入分析为基础。同时,他
的分析改变了传统的货币经济与实物经济相分离的“两分法”,而将两者结合为议题,将就
业、收入理论和利息、货币、消费储蓄与投资理论都纳入一个经济理论结构中。

第三,在经济政策主张上,《通论》以国家干预、调节经济的主张,代替了传统经济学的经
济自由放任的主张。(新)古典经济学认为资本主义经济依靠市场机制的自发调节,可以实
现充分就业的均衡,所以,他们主张自由放任、自由经营、反对国家干预经济。

《通论》认为,资本主义经济通常的状况是小于充分就业的均衡,如果没有国家干预经济是
不可能实现充分就业均衡的。这是因为,由于三大基本心理法则的作用,社会总需求不足,
要提高总需求,关键是有足够的“投资引诱”。……但是,由于企业家的心里难以控制……
所以,只有依靠国家的力量,以赤字财政政策为主加以货币膨胀的货币政策辅助,来刺激投
资,刺激消费,达到提高社会总需求,实现充分就业均衡。

它对垄断资本主义经济的发展以及西方经济学的发展,都有着巨大而深远的影响。这就是
“凯恩斯革命”的含义所在。

它的对立学派如现代货币主义、供给学派和理性预期学派是不同意这种评价的。

\subsection{凯恩斯经济学的缺陷}

凯恩斯的“有效需求”原理不仅在理论体系上超过了马尔萨斯,而且,是从整个垄断资产阶
级利益着眼的。

三个基本心理法则……主观唯心主义的分析……心理因素是一定经济条件的反映和产物……

首先,关于消费倾向法则。在一定历史时期中,各个阶级的消费状况不同。各个阶级的消费
性质和水平,主要取决于社会生产关系(一些西经学家认为,收入水平决定于财产所有权的
分配)。在资本主义社会,资本家的消费是由剩余价值所制约,工人的消费是由其劳动力价
值所制约。这是两个截然不等的量。

其次, 关于资本边际效率法则。凯恩斯指出了利润率具有下降的趋势,但这种下降趋势的
产生,并不是因为投资者的心理预期因素所决定的,而是因为,随着资本积累的发展,资本
有机构成的提高。另外,投资者的心理因素,是由经济条件所决定的,不是偶然的现象。在
经济发展不同时期,由于经济发展状况不同,使投资者对未来收益的预期心理不同……

最后,关于流动偏好法则。这个法则建立在对投机性货币需求和利息率的函数关系的分析上。
这种反应了食利者阶层的债券投机活动对资本主义经济活动的影响受到凯恩斯的重视。……
利息是来源于生产过程中所产生的剩余价值,利息能成为“一定时期内放弃流动性的报酬”,
其前提是,货币资本要借贷给职能资本家,尤其是借贷给产业资本家……通过生产过程,才
能使这笔货币资本价值增殖。单纯的流通领域的借贷,是不可能发生价值增殖,不可能产生
利息的。凯恩斯的分析抽去了利息产生的本质分析,似乎流动偏好的心理法则就能产生利息。
因此,凯恩斯的错误在于,他将人们的阶级状况抽去而谈论“一般”,又将人们的心理因素
看作是“显眼的人性”,具有永恒的性质。

利息率决定于货币的供求关系是不正确的,错误在于将货币等同于借贷资本。虽然借贷资本
开始总是以货币的形式存在,单纯的货币可以通过单纯的贷方行为,通过货币到存款的转化
而变为资本。但是,货币并不就是资本,货币并不就是借贷资本。货币转化为资本的前提是,
职能资本家要将货币去购买劳动力和生产资料。……而资本的形态很多,借贷资本只是其中
一个形态。利息只与借贷资本发生关系,利息率应该由借贷资本的供求量决定。

凯恩斯从自己的货币理论出发,认为实行扩张性的财政政策和货币政策是不会出现真正的通
货膨胀。但是,这一政策思想与事实不符,资本主义国家推行这一政策的结果,不仅没有实
现没有通货膨胀的充分就业,反而出现了爬行的通货膨胀。

还有很多局限性。短期静态分析……国内经济分析……投资只局限于自发的投资,流动偏好
只局限于持有现金与持有债券之间的选择。

\subsection{凯恩斯经济学的影响和发展}

战后一直到60年代末和70年代初,是凯恩斯主义的兴盛时期,这个时期被称为“凯恩斯主义
时代”,凯恩斯也一度被称为“战后繁荣之父”。

1944年至1945年间,由于战后西方发达国家面临着经济恢复和发展,为了防止30年代大危机
的重演才先后把凯恩斯主义奉为国策。1944年5月英国《就业白皮书》。英国工党把凯恩斯
主义作为自己纲领的最重要的支柱,鼓吹依靠国家对各种社会开支拨款来建立“无危机的资
本主义”。

美国1945年9月8日“国内充分就业法案”的塔夫脱——拉特克利修正案。1946年美国国会又通
过了“1946年就业法”。

今天,它的政策主张已被发达资本主义国家,特别是英、美两国政府所奉行,成为这些国家
经济政策的主要组成部分。

凯恩斯的门徒对凯恩斯理论的发展,是将《通论》限于国内经济的分析扩展为国际经济分析,
提出“开放经济”模型,并对其内容作了深化,主要是从凯恩斯的收入决定论即收入=消费+
投资的原理入手,他们主要将凯恩斯的消费函数理论和h投资函数理论的短期分析发展为长
期的、动态的分析。

关于消费函数理论,凯恩斯强调绝对收入假定,认为消费者是绝对收入的函数。从长期看,
随着收入的增加,边际消费倾向递减,平均消费倾向也是下降的。杜森贝、莫迪里安尼等人
提出了相对收入说、恒常收入说及生命周期说。

相对收入说认为,个人或家庭的平均消费倾向决定于他的或它的相对收入。所谓相对收入,
就是消费在收入分配中的地位。还有预期滞后等。消费的这种不对称变化,使得长期消费倾
向保持不变。

衍生出的这些学说,补充了绝对收入以外的其他因素,分析他们与消费的关系,试图说明长
期平均消费倾向是不变的,修正了凯恩斯的看法。由于他们认为在长期内,消费倾向大致是
稳定在一个比例上,而不是边际消费倾向递减,这就降低了消费倾向在凯恩斯理论体系中的
作用,突出了投资理论的重要性。

关于投资函数理论,凯恩斯在分析投资和收入的关系时,谈到乘数原理。但他只谈了投资乘
数。凯恩斯以后的经济学家将乘数原理范围扩大至总需求中任何一个组成部分,如消费、投
资或政府支出等自发变动,都可以产生比这个初始自发变动大几倍的收入变动,乘数就是支
出增量和收入增量之间的系数。

另外,凯恩斯只分析了自发的投资,凯恩斯以后的经济学,补充考察了投资的另一部分——引
致投资,即考察了由于收入增加而引起的投资的增加。

在投资与储蓄的均衡问题上,凯恩斯只分析了短期内,投资与储蓄的均衡问题。凯恩斯以后
的经济学作了长期分析,提出了经济增长的理论与模型。最有代表性的,就是哈罗德——多马
经济增长模型。这个模型从凯恩斯的投资理论出发,根据资本积累,技术进步和人口增长等
因素,研究经济持续稳定增长的条件。这个模型因而被称为标准的凯恩斯主义经济增长模型。

战后凯恩斯逐渐形成了新古典综合和新剑桥两个对立的凯恩斯学派。前者一直居于主流地位,
直到60--70年代,资本主义经济出现“滞涨并存”局面,其地位才开始动摇。

凯恩斯经济学发展成了一门新的经济学科——宏观经济学,并形成了一套较为完整的宏观经济
调节的政策措施。凯恩斯的经济学,是为垄断资本主义服务的。但他在一定程度上反映了以
社会化大生产为基础的市场经济内的一些经济现象之间的联系。

\chapter{理论的大拼盘——新古典综合派}

\section{综合中产生的学派}

《通论》存在着理论结构不完善,个别观点含糊不清,行文晦涩,不易理解等问题。

\subsection{在集大成中诞生}

萨缪尔森的“新古典综合”:只要适当地增强财政货币政策就可以使我们的混合经济不会
过分的繁荣和萧条,能够达到健全的前进的成长。这个基本点……结合新古典。

综合开始于希克斯1937年发表的《凯恩斯先生与古典学派》一文。希克斯用三个方程式与
IS--LM曲线将凯恩斯的理论综合为一般均衡理论,并寻求与新古典经济学之间的联系。认为
凯恩斯只不过是“向马歇尔的正统经济学跨回了一大步。”

凯恩斯理论的支持者阿尔文·汉森在解释、修正凯恩斯经济理论时,与希克斯一样,试图将
两者沟通起来。他认为从新古典经济学的可贷资金说中也可以推导出IS曲线。

唐·帕廷金在1956年出版的《货币、利息和价格》一书中,既运用了凯恩斯的收入--支出法
分析经济活动,又保留了新古典学派的货币数量论观点。

萨缪尔森,1961《经济学第5版》……1964年第6版中,认为新古典综合就是总收入决定论与
微观经济学经典理论的结合。

从资本主义经济实践发展的要求看。由于新古典经济学的自由竞争的市场机制能够保证资本
主义经济自动趋向均衡学说,未能经受住30年代自由市场经济完全崩溃的考验……凯恩斯国
家干预理论盛行,这促进了国家垄断资本主义的发展,当代资本主义经济逐渐演变为既存在
私人资本主义经济,同时又存在着“社会化”的公共经济的公私混合经济形态。激燃资本主
义经济是h混合型的,其理论也应当是混合的。
\subsection{五大主将}

新古典综合派的主要代表人物有萨缪尔森、托宾、索洛、莫迪利安尼、奥肯等。

\section{支点:三个模型}

凯恩斯的收入支出理论是关于社会总产量从而总就业量、总收入水平由哪些因素决定以及如
何决定的理论。新古典综合派对其修正、扩展。

\subsection{\ang{45}线模型}

凯恩斯假定一个社会只有存在厂商、居民两个部门,政府不干预经济。经济活动可以分成商
品和货币两个领域。存在商品和货币市场。一定时期社会的收入(Y)从总供给角度来看,等
于消费(C)和储蓄(S)之和;从总需求角度看,等于消费的支出(C)和用于投资的支出
(I)之和。在商品领域中,凯恩斯认为收入取决于支出,取决于销售总金额。因此,收入的
大小取决于社会有支付能力的总需求,即有效需求的大小。有效需求决定于货币倾向与投资
倾向,投资倾向又决定于资本边际效率和市场利息率,在只有预期投资等于预期储蓄的条件
下,收入与支出相等,即$C + S = Y = C + I$或$S = I$。社会经济中总需求与总供给之间
达到平衡。

新古典综合派认为凯恩斯上述分析,可以用消费加投资的曲线同\ang{45}线的交点来说明。

横轴表示收入Y,纵轴表示消费+投资,即C+I。……但均衡只是特例,如果 $C+I$ 小于潜在
收入(即实现充分就业时存在的收入),则有效需求不足。反之,则过度需求,引起通货膨
胀。

为防止经济生活中经常出现的有效需求不足和过度需求,新古典综合学派依据凯恩斯主义中
国家干预经济活动的思想,在收入支出理论的假定中加进政府部门,建立厂商、居民户、政
府三部门的经济模型,引入政府税收(T)和政府支出(G)两因素。

收入从供给角度看,$Y = C + S +T$;收入从需求角度看,$Y = C + I +G$。该式表示,通
过调整政府的收入与支出活动,可以改变总需求与总供给之间不均衡的现象。如果总需求过
度(可简单理解为生产不足,小于需求)引发通货膨胀现象,政府可以采取减少政府支出,
或增加税收,或两种措施并用,减少C、I、G,使支出减至与收入相当的水平,控制过度的总
需求,使总需求与总供给在不存在通货膨胀的条件下达到均衡。如果出现$C + I + G < C +
S + T$,即总需求不足(可简单理解为生产过剩,大于需求)引发的失业现象,政府可以采取
增加政府支出,或降低税收,或双管齐下,扩大C、I、G,使支出增长至与收入相当的水平,
持续有效需求,使总需求与总供给在达到充分就业的状态下实现均衡。

针对凯恩斯建立的是一个封闭的收入支出模型,不适应战后开放经济条件下经济活动的分析,
新古典综合派将其进一步修正为开放经济条件下收入支出模型。在开放经济条件下,新古典
综合派仅考虑出口和进口的变化对一国经济的影响。……出口活动的发生等于总需求增加;
进口行为的产生等于总供给增加。设出口为X,进口为M,凯恩斯的收入支出模型进一步扩展
为:
\[C + S + T + M = C + I + G + X\]

\subsection{IS--LM模型}

凯恩斯的收入支出理论不仅研究了商品市场的均衡问题,也花了很大气力去研究货币市场的
均衡。人们由于交易、谨慎、投机理念的存在而产生了对货币的需求,货币的供给则是由国
家的金融机构控制。当货币需求量与货币供给量相等时,货币市场达到了均衡,并确定了利
息率。

凯恩斯认为,利息率取决于货币供给量和流动偏好决定的货币需求量。人们的交易、预防和
投机动机是影响流动偏好变动的主要因素。人们的交易、预防受收入影响,而收入取决于投
资和消费,投资和消费又决定于利息率。

循环论证。问题就出在流动偏好上。影响流动偏好的三个因素中,除满足投机动机所需要的
货币量与收入无关外,满足交易和谨慎动机的货币需求量,都与收入有着紧密的联系。当收
入水平无法确定时,满足交易和谨慎动机的货币需求量就无法确定,利息率也随之无法确定。
而要知道收入,预先又得知道利息率及由此相关的投资和储蓄状况。二是凯恩斯没有考虑
如何把商品市场和货币市场结合起来,考虑均衡收入和利息率的决定问题,而是以假定——讨
论产品市场均衡时,假定货币市场已均衡;而考虑货币市场均衡时,假定产品市场已达到均
衡的形式回避了之。实际上两市场达到均衡是资本主义经济体系实现均衡的必要条件。希克
斯--汉森提出IS--LM模型,修补凯恩斯理论的上述缺陷。解决利息率的确定以及两市场的均
衡问题。

\subsubsection{IS曲线}

希克斯将凯恩斯的基本思想用三个方程式来概括。它们是$I = I(r)、S=S(Y)、M = L(r,Y)$。

由于投资是利息率r的函数,储蓄是收入Y的函数,$I(r)=S(Y)$ 同时也表示了在产品市场均
衡条件下,利息率r与国民收入Y之间的函数,即储蓄等与投资。

由于利息率、投资、收入三者之间的关系是,低利率、投资增加、收入水平提高。反之,高
利率、投资减少,收入水平降低。所以IS曲线的斜率为负,即利息率与收入呈反方向变化,
向右下方倾斜。

\subsubsection{LM曲线}

希克斯的 $M = L(r,Y)$方程式概括了凯恩斯的货币和利息理论。按照凯恩斯的说法,货币供
给(M)由政府控制,是一个外生变量。货币需求(L)包括货币交易需求(即货币的交易和
预防需求)$L_1$ 和货币投机需求 $L_2$两大部分,并且是利息率(r)和收入水平(Y)两
者的函数。货币的交易需求是收入的函数,即 $L_1 = L_1(Y)$ 或 $L_1 = KY$。K表示参与
交易需求的货币占收入中的比例。货币的投机需求是利率的函数,即 $L_2 = L_2(r)$或……表
示利率每变动一个百分点时$L_2$的变动程度。$L_2$与r呈反方向变动。在货币供给量既定
的条件下,货币市场的均衡可以通过调节货币需求量来实现。

LM曲线表示的是当货币供给既定,货币市场处于均衡时各种利息率与收入水平的对应关系。
LM曲线上任意一点利息和收入的组合点都是可以保证达到货币市场均衡状态的组合垫。在LM
曲线左边的利息率和收入的组合点,是货币需求小于货币供给的货币市场的非均衡组合点;
在LM曲线右边的利息率和收入的组合点,则是货币需求大于货币供给的货币市场的非均衡组
合点。

由于货币需求量存在与收入水平同方向变动,与利息率反方向变动的关系,因此在高收入水
平状况下,货币的交易需求量增加,货币的投机需求量减少,利息率就会提高;要么反之。
从而出现高收入与高利息率,低收入与低利息率相对应的状况,导致LM曲线向右上方倾斜。

LM曲线右上方顶端部分是垂直线,这是因为货币供给量受政府控制,在一定时期是一个定量
所致。当收入增加时,货币的交易性需求也随之增加,所需的额外货币量只能从持有的备用
现金金额中提取补充,这势必导致利息率提高。……当既定的货币供给量全部用在货币交易
需求方面时,收入水平就无法再继续提高。从而造成这一段曲线成为垂直状。

LM曲线尾段部分是水平线是由于在低收入水平时,货币的交易性需求减少。货币的投机性需
求增加,利息率下降。……随着……利息率不再降低。此段的LM曲线表现出对利息率有完全 的弹性,所以呈水平状,并被称为“流动陷阱区域”或“凯恩斯陷阱”。 \subsubsection{IS--LM模型的产生}


希克斯推导出IS和LM曲线后,将方程$I(r) = S(Y)$ 与 $M = L (r,Y)$联立,求出(r, Y)的
解。这解就是IS曲线和LM曲线的焦点。它意味着利息率与收入是可以同时决定的。……商品
和货币市场共同均衡的条件……该组合点可以使货币供给量等于货币需求量。

\subsubsection{IS或LM曲线的位移}

IS曲线的位移起因于总需求的变动,当投资需求、消费需求、政府支出、出口变化,比如增
加,将导致IS曲线右移,比较高水平的利率与收入组合。

LM曲线的位移主要来自于货币供给的变动。供给增加,LM曲线向右移动,带来了利息率降低
和收入增加的两重相反结果。

\subsubsection{IS--LM模型的运用}

依据此模型,考察政府财政政策、货币政策的效应。为此要考虑LM曲线的斜率,界定LM曲线
的三个区域。凯恩斯区域,中间区域,LM曲线斜率无穷大的区域。

政府如果在“凯恩斯区域”实行扩张性的货币政策,会使LM曲线位移,货币供给的增加既不
会提高收入水平,也不能降低利率水平。货币政策无效。如果改行扩张性财政政策,会使IS
曲线、LM曲线均会右移。收入增加,利率不变。在垂直线区域,扩张性财政政策的实施使IS
曲线上移,利率水平提高,收入水平未变,财政政策无效。如果采用扩张性的货币政策使LM
曲线右移,既降低利率又提高收入水平。货币政策有效,这一部份即古典区域。

IS--LM模型考察了商品和货币市场的均衡,但它忽略了物价水平的变动,也没有分析生产
和劳动市场的均衡。这不符合60年代后期以来物价水平上涨已普遍出现于发达工业化国家的
现实。为此,新古典综合派对LS--LM模型进行了修正,建立了总需求--总供给模型,即
AD--AS模型。

\subsection{AD--AS模型}

\subsubsection{AD曲线的推导}

新古典综合派首先对IS--LM模型的修正是,将外生变量的货币供给改为受价格水平变动影响
的内生变量。这样,IS曲线与LM曲线的交点在价格水平变动下将左右移动,从而产生总需求
曲线。

价格变动是通过两个方面影响IS曲线与LM曲线的交点,是价格导致的财富效应(庇古效应)
移动IS曲线。庇古认为价格水平变化,比如下降时,货币、存款、债务等财富的实际价值提
高,具有一定收入水平的消费者不急于进一步增加其财富的积累,反而消费欲望增强。消费
支出的增长,导致消费曲线向上方移动,消费曲线的上移意味着IS曲线向右移动。如果价格
上升,则反之。另一是利率效应(凯恩斯效应)移动LM曲线。凯恩斯认为价格变化,比如下
降时,货币的实际购买力提高,实际货币供给增加,即人们手中持有的货币量超过了他们的
需要。人们会纷纷舍弃货币而购买债券,从而引发债券价格上升和利率下降,利率的下降又
诱使投资需求增长,总需求价格大规模地提高。在其他条件不变的情况下,价格水平下跌,
货币供给增加,使LM曲线从原来的位置向右移动。如价格上升,则反之。

\subsubsection{AS曲线的推导}

在AS曲线的推动过程中,新古典综合派反对新古典经济学的货币工资具有充分自由伸缩性的
假定。新古典经济学派认为,劳动市场上劳动需求和劳动供给状况决定就业量的大小,而劳
动需求与劳动供给又取决于实际工资水平的变动。

新古典综合派支持凯恩斯……凯恩斯认为新古典经济学的假定不符合现实。因为资本主义劳
动市场处于非完全竞争状态,货币工资具有向下刚性和工人存在“货币幻觉”的倾向。当劳
动供给大于劳动需求时,就存在非自愿失业。

因此,新古典综合派认为工人会反对货币工资的下降,因而货币工资只能提高,不能下降。
所谓“货币幻觉”,第一是指工人只注意价格水平不变时,货币工资的变动。第二是指工人
不大理会货币工资不变,价格水平提高,实际工资水平下降的事实。

但新古典综合派不赞同凯恩斯的在实现充分就业之前,劳动的边际产量是一个常量的假定。
认为随着总需求的增加,生产不断扩展,持续一段时期后,产量进一步提高,……从而导致
生产总成本增加,劳动边际产量下降。

新古典综合派进一步认为,在达到充分就业后,尽管价格提高,实际工资下降,但劳动就业
量无法再增长,因此总产量不变,就业量与总产量处于均衡态势。

\subsubsection{AD--AS模型}

总需求--总供给模型(AD--AS模型)

倘若在实现充分就业后,总需求曲线$AD_2$相交于总供给曲线AS的垂直段,价格会持续上升
至$P_2$,均衡收入不变,仍是$Y^*$。此时的均衡价格与收入也是非充分就业的均衡价格与
收入。

在新古典综合派的AD--AS模型中,AD曲线与AS曲线交点决定的均衡收入一般低于充分就业时
的收入,均衡价格不是高于就是低于充分就业时的价格。这就为政府干预经济活动提供了一
个理论依据。当均衡收入低于充分就业的收入时,政府可以采取扩张性的经济政策刺激总需
求增长,扩大生产,以达到充分就业。达到充分就业后,总需求的继续增长会导致价格上涨,
出现通货膨胀。所以不管经济活动状态是否达到充分就业,政府都有必要对经济活动进行干
预和调节。

\section{奇妙之说奠定主流地位}

在以上模型基础上,新古典综合派继续阐述他们的看法。通货膨胀与失业理论、经济增长与
经济周期理论成为新古典综合派的主要理论。

\subsection{通货膨胀与就业理论}

\subsubsection{运用菲利普斯曲线修补凯恩斯的通胀与失业不会并存的观点}

凯恩斯经济理论认为,经济衰退、失业起因于有效需求不足,通货膨胀产生于过度需求。两
者均可通过政府实施扩张性或紧缩性财政货币政策来调整。因此在经济活动状态达到充分就
业之前,有效需求的扩大只会导致产量和就业量增长而不会引发通货膨胀。只有在达到充分
就业状态后,总需求的继续扩大才会出现通货膨胀、产量和就业量增长并存的局面。所以凯
恩斯经济理论认为,一般而言,需求不足、需求过度,不大可能同时出现,由此通货膨胀与
失业的现象也不大可能同时发生。

60年代末滞涨。新古典综合派将菲利普斯曲线引入自己的理论体系,修补凯恩斯的理论。

菲利普斯得出结论:在英国,失业率稳定在5\%左右,货币工资水平就会稳定。如果失业率
保持在2.5\%,货币工资增长率就会超过劳动生产率的增长率。

新古典综合派受到这个启示,认为失业率与通货膨胀之间也存在此长彼消的依存关系。当失
业率较低,单位劳动成本在单位产品成本中所占比例较大时,一旦货币工资增长率超过劳动
生产率增长率,就会导致物价水平上涨或通货膨胀提高。要么反之。

新古典综合派认为,既然菲利普斯曲线可以反映失业与通货膨胀的关系,政府就能通过宏观
经济政策的制定来运用菲利普斯曲线,规定一个临界区域,即确定一个社会经济可以认可的
通胀率和失业率。超过一定的通货膨胀率,通货膨胀将无法遏制;或高于一定的失业率,货
币工资增长率就会超过劳动生产率的增长率。超出临界区域,政府就有必要进行调节。

特别是到了70年代以后,滞涨并存的出现,使得菲利普斯曲线向右上方移动。……菲利普斯
曲线的位移说明通货膨胀率与失业率以前那种此长彼消的依存关系已有改变……只能让通货
膨胀更高,才能使得失业率比以前降低一些;或者用更高失业率的存在,将通货膨胀率稍稍
降低一些。……对于新的菲利普斯曲线而言,只能把临界区域提高到4\%的失业率和通货膨
胀的水平。菲利普斯曲线已不可能像60年代那样成为发达工业化国家指定需求管理政策的理
论依据。

\subsubsection{运用新古典学派的微观经济理论解释滞涨发生的原因}

当新古典综合派意识到修正后的菲利普斯曲线已不能解释现实中滞涨现象后,他们就转向新
古典学派,企图用新古典学派的微观经济理论补充宏观经济活动的分析,阐释“滞涨”现象
发生的原因。主要理论有三种。

\begin{enumerate}
\item 滞涨起因于个别生产部门供给突变说。

  华尔特·海勒认为,个别生产部门供给突变,比如急速增加或减少,会导致该部门产品价
  格波动,从而引起滞涨。粮食——燃料……


\item 滞涨起因于财政支出中转移支付比重加大说。

  萨缪尔森认为,由于福利国家的不断出现及国家福利制度的建立,政府财政支出中原本应
  主要用于举办公共工程等,但目前相当一部分用在转移支付支出。以失业津贴之类的转移
  支付的增加,一方面是增加了低收入家庭的收入,使其成员即使失业也不急于寻找新工作,
  造成失业率高;另一方面由于享受转移支付群体的扩大,收入普遍增长。即使在经济萧条
  时期,物价水平也居高不下,导致了高通货膨胀率。“混合经济”……


\item 滞涨起因于不完全竞争的劳工市场说。

  托宾以劳工市场均衡和失衡状况来解释失业和通货膨胀的并存。托宾认为劳工市场的过度
  需求和过度供给并存,即工作空位——有许多工作无人愿意做和失业——有许多人找不到工作
  并存,是劳工市场失衡的体现,并且是经常性的现象;而劳工市场的供求一致、劳工市场
  均衡则是偶然存在的现象。

  杜森贝利在托宾分析的基础上进一步进行研究,主要集中在两个方面;(1)产生工作空位
  和失业的原因是由于劳动供给中存在着性别、工种、地区、技术水平、劳动熟练程度的差异,
  这样劳动需求夜守上述因素限制,产生一边有人找不到工作,另一边又有许多工作无人干的
  现象。(2)个别劳工市场与货币工资增长率之间关系的分析。个别劳工市场工作空位引发
  所有劳工市场货币工资水平的提高,物价水平的上升,从而促使通货膨胀上升。
\end{enumerate}

\subsection{经济增长理论}

\subsubsection{经济增长模型的建立}

由于凯恩斯主要采用短期静态的分析方法。金考察短期内投资与储蓄的均衡并进而说明国民
收入的均衡。至于长期内投资与储蓄是否能够达到均衡,以及变动对国民收入均衡的影响,
凯恩斯的研究并未涉及。可以说凯恩斯对经济增长问题的分析是不完全的,新古典综合派认
为有必要加以补充,引入长期动态分析方法,研究国民收入在长期内动态的发展变化过程。
其次是顺应经济实践发展的要求。1929--1933年世界性经济危机的爆发及其影响,使西方经
济学者意识到靠市场调节使资本主义经济活动可以达到充分就业状态已不可能,他们开始关
注失业问题。另外西方发达工业化国家的政府纷纷采用凯恩斯的经济政策,经济均有了不同
程度的增长,在经济增长过程中也出现了一些需要解决的问题。

\subsubsection{经济增长模型的主要内容}

哈罗德--多马经济增长模型,1946--1948年间,两人提出了内容基本相同的两个经济增长模
型。哈罗德--多马模型的问世,标志着当代资本主义经济增长理论的产生。

模型仿效凯恩斯储蓄--投资分析方法,重点考察S(储蓄率,即总储蓄在总收入的比例
(S/Y));V(资本--产出比率,V=K/Y,表示生产一单位的国民收入必须投入的资本量),
$C_w$(有保证的经济增长率,即S和V既定,为了使储蓄量全部转化为投资所需要的经济增
长率。)三个变量,三变量之间的关系可以用数学公式表示,即 $C_w = S/V$。该式表明有
保证的经济增长率受储蓄率和资本--产业比率制约。将$C_w=S/V$。将此式子变换可得到一
个表示储蓄与投资,经济增长的关系式,即 $S=G_w \cdot V$。该式表明当资本--产出比率
(V)既定时,对于任何一个给定的储蓄率(S),能够实现经济均衡增长的有保证的增长率
($C_w$)只是一个唯一的值。明确了上述三个变量间的关系后,哈罗德又提出“自然增长
率”概念。自然增长率用$G_n$表示,是指一个国家经济活动所能达到的最大经济增长率。
自然增长率$G_N$取决于劳动力的增长率X和劳动生产率增长率L。$G_n=X+L$。……将有保证
的增长率和自然增长率结合起来考察,哈罗德得到达到社会经济充分就业的均衡增长调节
$G_w = Gn$或$\tfrac{S}{V}=X+L$,并认为这一条件在现实经济生活中很难实现。因为$S、
V、X、L$,也会受诸多因素制约。……当$G_w > G_n$,表示储蓄的增长率超过了劳动力的
增长率从而出现过度储蓄,引发社会经济长期停滞和萧条;如果 $G_w < G_n$,表示现存资
本设备使用已达到极限,这将刺激投资者进行新的投资,社会生产将处于高涨和繁荣时期,
就业增加,以至出现通货膨胀现象。

新古典综合派充分肯定哈罗德--多马模型,认为该模型引入时间因素,增加资本--产出比率
概念,并强调投资的两重作用,即投资增加不仅增加收入(需求),而且也增加了生产能力
(供给)。动态画、长期化,发展了凯恩斯的经济增长理论……索洛将此调节形象地比喻为
“刀锋”,意思是实现i社会经济充分就业的均衡增长条件似一把刀的刀锋那样狭窄,以致
现实经济极难按着这一条件达到充分就业的平衡。为此,新古典综合派修正“刀锋”式的均
衡增长调节,提出自己的经济增长模型。

\begin{enumerate}
\item 不变技术条件下的经济增长模型。

  假定生产技术条件不变,提出一个总量生产函数:$Y=\symit{f}(KL)$。该生产函数表面,
  国民收入量受资本与劳动投入量的制约。并显示出下列性质。(1)当L和K按同一比例同
  时变动,Y也按相同比例变动。(2)资本与劳动,即生产要素之间具有相互替代的关系。
  (3)生产也受生产要素报酬递减规律制约。

  当要素市场上资本处于供大于求状况时,利息率会下降,使用资本较使用劳动比较,前者
  比后者便宜,企业将提高资本--劳动比率与资本--产出比率,资本密集型生产……反之,
  劳动密集型生产。

  既然资本--产出比率是可变的,能够实现经济均衡的有保证的经济增长率就不像哈罗德--
  多马模型所揭示的只存在一个唯一的值。……化解了“刀锋”的存在,使国民经济可以实
  现充分就业的均衡。


\item 技术进步条件下的经济增长模型。

  索洛、米德。

  该模型表明经济增长率受技术增长率、资本增长率、劳动增长率三个因素制约。

  技术进步对经济增长的重要。

\end{enumerate}

\subsubsection{新古典综合派经济增长模型的特点}

基本特点,是用新古典学派的微观经济分析补充凯恩斯主义的宏观经济分析;用市场机制,
特别是价格机制的作用补充政府干预经济的活动,来论证资本主义社会是一个能够实现充分
就业均衡增长的社会。

\subsection{经济周期理论}

新古典综合派的经济周期理论影响很大,主要是乘数--加速数模型。把凯恩斯的“乘数理
论”与西方经济学的“加速原理”相结合,它成为新古典综合派将凯恩斯的宏观静态分析理
论拓展为宏观动态分析理论的一个范例。

加速原理的基本内容,简言之就是依照\textbf{现代化机器生产采用耐久性固定资本设备}的
生产方法这一技术特点,证明收入水平或消费需求的变动将会引起投资量更为剧烈的变动。

一般一定阶段的\textbf{投资总额是由新增投资量与用于补偿之用的重置投资两部分组成}。
新增投资量取决于收入或消费支出或消费品产量的变动量,重置投资量取决于资本设备的组
成、数量、使用期限长短等。两种投资交替在一起共同起作用,使\textbf{投资量的变动程
  度更猛烈}。

综上所述,所谓加速,就是指投资率变动的幅度,不论是增长还是下降,都强于产量或消费
支出的变动。加速的速度取决于某一年度的消费需求。如果消费需求下降,消费品产量降低,
会引发投资量猛烈的减少;即使x消费需求的绝对量不减少,但增加幅度有所下降,也会导
致投资量大幅度下降。要么反之。

萨缪尔森受汉森启发,1939年提出乘数-加速数模型。汉森--萨缪尔森模型。

\subsubsection{乘数--加速数模型}

\subsubsection{希克斯对乘数--加速数模型的发展}

主要集中在乘数理论与加速原理相互作用下引起的经济波动上下限的限定。

\section{各种政策的糅合}

新古典综合派因袭了凯恩斯的宏观需求管理思想,也就是政府应采取积极的经济政策,直接
或间接地影响需求管理的对象,如投资、储蓄、消费、政府支出、税收、进口和出口等,对
社会总需求进行适当的调整,以实现经济稳定增长的目的。但又针对不断变化的资本主义经
济活动的现实,对凯恩斯的需求管理政策进行修正与扩展,将凯恩斯主义强调的财政政策和
新古典学派注重的货币政策融合在一起。

\subsection{补充性的财政货币政策}

产生于30年代的凯恩斯学派的经济政策主张具有明显的解决经济危机的烙印。

紧缩和扩大的财政政策交替使用,即补偿性财政政策的运用。最早由汉森提出。

“一种平伏私人经济的周期性波动的调节工具”。在萧条时期,政府可以减税,增加政府支
出,如公共工程开支、政府购买、政府转移支付的扩大等形式来刺激总需求;在繁荣时期,
政府可以增税、减少政府支出,如缩小公共工程开支、政府购买、政府转移支付的形式来抑
制总需求。因此,汉森呼吁政府在确定财政预算时,不必坚持平衡财政收支的信条,不必强
求年年预算平衡,可以出现财政赤字,只要在一个周期内达到收支平衡即可。政府可以按私
人的消费和投资的数量来安排政府的预算,使私人支出数量与政府支出总量保持在可实现充
分就业的水平上。

汉森还进一步指出,可以根据经济情况的变化,提前做一些反周期波动的防范措施。

新古典综合派又依据LS--LM模型的分析,提出应该重视货币政策的运用,以便在商品市场和
货币市场同时均衡下实现充分就业。由此萨缪尔森提出了补偿性货币政策,即在经济萧条时
期,中央银行应降低利息率,增加货币供给量,以增加投资和社会总需求;在经济繁荣时期,
提高利息率,减少货币供给量,以减少投资和社会总需求。

在经济萧条时期,中央银行实行补偿性货币政策的具体做法是:1,在公开市场上买进政府
债券。中央银行用支票支付的形式向企业、个人购买政府债券。出售政府债券的企业和个人
将获得的支票存入商业银行,导致商业银行存款增加,商业银行按法定准备率将存款的大部
分用作贷款,在货币乘数作用下,各商业银行的存款、贷款都将有若干倍地增加,货币供应
量增加,利息率下降,刺激投资、总需求增加,经济得以发展。2,降低贴现率。贴现率是
商业银行向中央银行借款时的利息率。央行降低贴现率可促使商业银行向中央银行 增加借
款……货币乘数……3,降低法定准备金率。法定准备金率即银行准备金对存款的比例。法
定准备金率与银行创造货币的数量呈反向变动关系。

在经济繁荣时期,中央银行则采取与上述的相反手段来实行补偿性货币政策。1,当政府在
公开市场上卖出政府债券时,购买者从银行取出存款购买政府债券……货币乘数……2,提
高贴现率。3,提高法定准备金率。

\subsection{增长性财政货币政策}

50年代,美国政府运用新古典综合派提出的补偿性财政货币政策去调控经济活动。虽然也出
现过两次经济危机,但与第二次世界大战爆发以前爆发的经济危机相比,危机的程度不是很
猛烈。但同期经济增长速度也不块,1953--1960年,美国实际国民生产总值平均增长率只有
2.5\%。这就是被人们称为的“艾森豪威尔停滞”时期。

进入60年代以后,为了加速美国经济增长,新古典综合派的托宾、奥肯、海勒等学者提出
了“增长性的财政货币政策”。奥肯认为,在60年代以前,基本上还是预算平衡政策占统治
地位。只有在战争(如朝鲜战争)、经济危机出现(1953--1954年、1957--1958年)的年份,
才会利用预算赤字或预算盈余当作常规手段来使用,所以战后至50年代末,除了少数年份,
大多数年份中美国都未达到充分就业状态。由此,奥肯说……60年代重新制定经济政策——增
长性的财政货币政策,其特点是“\textbf{不只是以经济是否在扩张,而是以经济是否已充
  分发挥出它的潜力,作为判断经济表现的目标。}”也就是财政货币政策的运用不仅仅以熨
平经济波动为目标,还应该达到刺激经济增长的目的。因此,不管是经济衰退时期,还是经
济高涨时期,\textbf{只要实际国民生产总值小于潜在国民生产总值,即充分就业条件下的
  国民生产总值,就要采取增长性的财政货币政策,即赤字财政政策与膨胀性的货币政策来
  刺激总需求,使两者一致。}

如何确定潜在的国民生产总值,新古典综合派认为必须把失业率作为一个变量,让其代表由
于生产资源闲置对产量造成的影响,然后求出高于4\%的失业率给产量造成的损失,再考虑
实际的产量,就可得到潜在的国民生产总值。
\begin{gather*}
  g = a(u - \bar{v})\\
  潜在GNP=实际产量 \times (100\% - G)
\end{gather*}

新古典综合派特别强调增长性的财政货币政策与50年代提出的补偿性财政货币政策的重要区
别在于前者注重消除潜在的国民生产总值与实际国民生产总值之间的差距,实现连续的经济
增长。曾经担任过肯尼迪总统经济顾问委员会主席的海勒就认为,美国在50年代之所以实际
国民生产总值低于日本、西德等国,竞争力不强,就是因为美国没有创造条件让其生产能力
得到彻底发挥,导致潜在国民生产总值与实际国民生产总值之间存在一个很大的差距。这是
因为消费需求和投资需求不足。要缩小两者之间的差距,使经济持续稳定地增长,就是要刺
激总需求。为此,新古典综合派建议政府要大胆使用赤字财政政策,即减税、增加政府支出、
扩大赤字额度等。

如果政府增加购买支出,会增加产量或收入。政府购买支出增加可以直接导致整个社会投资
增长。在货币供给与货币需求一致的情况下,由于政府购买增加导致因交易动机而产生的货
币需求量增长,在货币供给量一定的情况下,利率上升,引起私人投资支出减少。为此,新
古典综合派强调在采取赤字财政政策的同时,要实行膨胀性的货币政策,即增加货币供给量
的政策。

赤字财政政策的同时,由于存在货币扩张政策的配合,赤字财政政策的实施没有提高利息率,
也没有出现排挤私人投资的挤出效应。双重扩张的政策取得了积极的效果。

从60年代初期开始,肯尼迪、约翰逊两届政府都采纳了新古典综合派的增长性财政货币政策
主张,大规模地推行了赤字财政政策和膨胀性的货币政策。……GNP增长31\%,680万个新就
业机会,物价上涨不快,而且可以为人们所容忍,每年上涨率为2\%……由此,美国政府更
加无所顾忌地推行规模越来越大的赤字财政,到头终食恶果。不断膨胀的赤字造成了60年代
末70年代初“滞涨”的出现。

{\Large 隐形通货膨胀}

\subsection{多样化的经济政策主张}

新古典综合派并不认为滞涨是增长性财政货币政策的失败,反而认为应当采取多样化的经济政策,
来处理扩张性财政货币政策的负效应。

多样化经济政策主张,包括:
\begin{enumerate}
\item 财政政策和货币政策的适当搭配。

  70年代之前,财政政策与货币政策独立使用。一般在经济萧条时期,采取所谓松,即扩张
  性或膨胀性的财政货币政策;在经济繁荣时期,采取所谓紧,即紧缩性财政货币政策。其
  结果是顾此失彼,进退两难。

  这种搭配运用有两种具体形式:一种是松的财政政策与紧的货币政策的组合。比如增加政
  府支出、减税等措施来刺激投资,扩大就业,增加总需求,发展经济;同时用政府卖出政
  府债券,提高法定准备金率,再贴现率,贷款利率等方式来控制货币供应量,防止通货膨
  胀率不断攀升。另一种是紧的财政政策与送的货币政策的组合。比如一方面采取减少政府
  购买支出,增税等措施来降低总需求,以稳定价格水平,防止通货膨胀率的不断提高;另
  一方面用降低利率,扩大信贷规模等增加货币供给量的措施来刺激投资,扩大产量和就业
  水平。

  80年代美国里根政府采取了松财政与紧货币的政策组合,持续的高利率缓和了通货膨胀的
  压力。但导致投资支出(住宅建设)锐减,1982年出现了失业率高达10.9\%的衰退。不过
  减税财政扩张的作用很快显示出来,失业率开始迅速回落,产业(收入)水平回声。这种
  政策组合存在的问题就是财政赤字逐年增长,国际收支中经常账户的赤字越来越高,增长
  率下降。……所以还要辅以其他经济政策主张。


\item 财政政策和货币政策的微观化。所谓“微观化”是指政府针对个别市场和个别部门的
  不同具体情况,分别制定不同的经济政策对待。财政政策的微观化一方面是指财政收入政
  策的微观化,主要是通过改变税收结构和个别调整税率、税种、纳税的范围等,达到有利
  于资源集中使用,经济总量增长,经济效率提高的目的。另一方面是财政支出政策的微观
  化,就是通过改变政府投资、购买的方向,调整一些行业产品的生产结构,使一些部门迅
  速发展,特别是使那些可以容纳较多非熟练劳动者的部门发展,这样既可以提高社会劳动生产
  率,又可以解决部分人的就业问题。


\item 收入政策和人力政策。收入政策是指政府宣布管制物价和工资,甚至冻结,来实行工
  资--物价管制,禁止物价和工资的上涨。二是指政府根据劳动生产率的变动状态规定工资
  和物价上涨的限度,实行工资--物价的指导,防止货币工资的增长率超过劳动生产率的
  增长率。三是指政府充当企业和工会的调节者,督促双方对某种工资和物价上涨的手段。
  对服从工资--物价管制者、指导者以减税鼓励,对违反者以增税惩处。

  人力政策是指联邦政府的就业政策和劳工市场政策。
\end{enumerate}

上述四项政策主张,前两项是宏观财政货币政策的微观化,后两项则是宏观财政货币政策的
微观补充。除此之外,新古典综合派还提出了超出传统的财政货币政策范围,但可以导致经
济增长而又不与别的政策目标相违背的一系列政策措施,如浮动汇率政策,对外贸易与外汇
管制政策,消费指导政策,能源政策,农业政策等,尽管这些措施起到一定的作用,但非常
有限。

\section{评说新古典综合派}

新古典综合派开创了宏观经济分析方法与微观经济分析方法的结合。其次,新古典综合派通
过建立一些分析模型,提出某些技术型概念分析经济现象……新的内容。最后,新古典综合
派倡导在市场机制调节作用的基础上,政府进行必要的宏观调控。自凯恩斯强调市场应积极
主动地干预经济后,不论从理论研究,还是经济实践的需要,一边倒地强调政府宏观调控的
作用,弱化市场机制调节作用的倾向非常浓厚。在这样的背景下,新古典综合派能够看到单
纯强调政府调控作用有其不利于经济发展的负效应产生,而力主以市场机制调节为基础的政
府适当干预是难能可贵的。

进入70年代以后,新古典综合派在理论观点、经济政策方面受到货币主义、理性预期学派,
供给学派以及激进政治经济学派的强烈抨击。在这种情形下,萨缪尔森1972年打出“后凯恩
斯主流经济学”的旗号,弃“新古典综合”这一旧旗号但仍以正统的凯恩斯主义继承者自居。
持续滞涨现象的存在给“后凯恩斯主流经济学” 以重创,这反映在理论的困难——以前坚持的
通货膨胀与失业不可能并存说明有悖于现实;\textbf{政策上的无能--补偿性财政货币政策
  在滞涨存在时使用会产生自相矛盾的后果。}其正统主流经济学、官方经济学的地位开始
动摇,由此开始反省并兼容并包新观点,甚至是自己论争的主要对手货币主义,理性预期学
派、供给学派的一些观点。1985年,萨缪尔森将“后凯恩斯主流经济学”更名为“现代主流
经济学的新综合”本质上是相同的,更名只是意味着该学派要摆脱不断出现的困境而已。

其衰败的原因:
\begin{enumerate}
\item 该学派的建立先天不足。新古典学派建立在综合凯恩斯主义与新古典经济学学说的基
  础上。新古典经济学的不足在于片面强调市场至上,认定市场是尽善尽美。凯恩斯主义的
  缺陷在于过分突出赤字财政政策对经济的刺激作用。新古典综合派在综合时忽视了这缺陷
  的存在。……不稳固性。


\item IS--LM模型是新古典综合派的理论基础,本身也包含致命弱点。投资、储蓄是与一定
  时期相联系的流量范畴,货币供给与货币需求是与某一时点相联系的存量范畴。……按希
  克斯自己阐述的,当LM处于存量的均衡并被维持,一方面是指在一个时期每一个时点上
  M=L;另一方面也是指将来一个时期即预期每一个时点上M=L。很明显这是用存量均衡的含
  义来界定LM曲线,这种界定就排斥了流动偏好及不确定性。不符合LM曲线的性质--灵活性。


\item 乘数--加速数原理作为新古典综合派的重要理论之一存在不少非科学内容。加速数原
  理的一些理论前提条件,比如生产能力在任何时期都可以完全充分地得到释放;资本--产
  出比率在投资过程中是固定不变的等,离资本主义经济的现实相差甚远。最根本的谬误在
  于新古典综合派撇开资本主义社会的生产关系和基本矛盾,从心理因素和纯粹技术角度去
  解释资本主义经济周期性的波动,抽象地分析资本主义的再生产,既无科学性,也无实际
  价值。


\item 新古典综合派的一些经济政策的运用受到限制,成了无用之术。比如新古典综合派用
  来补充凯恩斯主义需求管理政策的补偿性财政货币政策在使用时会产生与预期相反的结果。
  如果政府扩大赤字来增加政府预算以解决失业时,那么有效需求的增长势必会提高物价水
  平,导致通货膨胀上升。
\end{enumerate}

\chapter{破字当头,立在其中——新剑桥学派}

\section{在论战中诞生}

在西方经济学界,最早也是最猛烈地批评新古典综合派的,当属以琼·罗宾逊为首的英国剑
桥大学的一批经济学者。——新剑桥学派,或新李嘉图主义或凯恩斯左派。

\subsection{两个剑桥之争}

1853年,英国著名经济学家琼·罗宾逊发表了《生产函数和资本理论》一文,拉开了抨击新
古典综合派的序幕。琼·罗宾逊在文章中集中批评新古典综合派的生产函数和资本理论,意
在否定新古典综合派的生产理论和分配理论。

60年代达到顶峰。琼·罗宾逊为首的一批经济学家揭示出新古典综合派理论体系中的一些逻
辑错误,在一定程度上动摇了新古典综合派主流经济学的地位而告终。

新剑桥学派否定新古典综合派将凯恩斯主义与新古典经济学的结合,认为这完全违背凯恩
斯力图从传统经济学束缚中脱离出来的初衷。虽然凯恩斯本人做得不彻底,但这正是凯恩斯
所要实现的目标。由此认定新古典综合派是冒牌凯恩斯主义。

1956年,琼·罗宾逊《资本积累》。冠名新剑桥学派,意在表明他们与以马歇尔为代表的新
古典学派,即剑桥学派的彻底决裂;被称为新李嘉图主义,是因为他们宣称要返回到李嘉图
的古典经济学中;还由于他们主张比较激烈的改良主义,所以又被称为凯恩斯左派。

经济增长与收入分配相结合的观点,分别由琼·罗宾逊和卡尔多提出;回归李嘉图古典经济
学是斯拉法一直倡导的;把不完全竞争、垄断价格等因素的作用引进国民收入决定分析,强
调投资对国民收入分配有一定影响的是波兰经济学家卡莱茨基。琼·罗宾逊等人则是将大家
的看法归纳、结合、运用的代表人物。

\subsection{四大主将}

\subsubsection{琼·罗宾逊}

\subsubsection{尼古拉·卡尔多}

\subsubsection{皮罗·斯拉法}

斯拉法一生是着重批判新古典经济学的价值理论与分配理论;另一边就是倡导重构政治经济
学,回归李嘉图的古典经济学……

1960年,斯拉法对新古典学派以边际分析为前提建立起来的价值理论与分配理论大加鞭挞,
提倡回复李嘉图的古典经济学传统。“斯拉法革命”

\subsubsection{卢季·帕西内蒂}

\section{独树一帜的理论特点}

\subsection{倡导历史分析方法}

凯恩斯以前新古典经济学在社会经济行为分析过程中,强调经济人的理性行为,认为追求自
身利益最大化是i开展经济活动的目标。所以在市场机制的指导下,生产者和消费者会最大
限度地利用经济资源从事生产和交换活动,以追求最大利润或最大效用,并使经济资源达到
最佳配置。静态均衡……破坏……新的均衡

新剑桥学派的经济学者认为,凯恩斯反对新古典经济学上述的均衡分析方法。“在理论方
面,《通论》的主要论点是打破均衡的束缚,并考虑现实生活的特性——昨天和明天的区别。
就这个世界和现在说来,过去是不能召回的、未来是不能确知的”,“从理论方面来说,革命
在于从均衡观到历史观的转变;在于从理性选择原理到以推测或惯例为基础的政策问题的转
变”。新剑桥学派认为凯恩斯注重历史分析方法——从历史的角度,加进时间因素研究资本主
义现实经济活动。强调现在从事的经济行为受过去经济行为的影响,受将来经济行为预期的
支配。所以在社会经济活动中,人们一般只能根据过去的经验来判断未来,未来的不确定性
使严格的理性行为不可能存在。因此经济分析应着重不均衡状态的分析,而不是均衡状态的
建立。

新剑桥学派的学者进一步指出,新古典综合派将凯恩斯的宏观经济分析与新古典学派的微观
经济分析相结合时,承袭了马歇尔的局部均衡论与瓦尔拉斯的一般均衡论,将已被凯恩斯革
命废除的均衡分析方法重新恢复,移植进凯恩斯经济理论中。这是对凯恩斯革命的背弃,否
定均衡分析。钟摆……不存在。“一个总是处于均衡状态的世界就没有未来和过去的区别,
那里没有历史,因此也就不需要凯恩斯。”并认为资本主义现实经济活动的分析方法应从短
期,比较静态的分析方法转向长期和动态的分析方法。

\subsection{修补凯恩斯宏观分析之不足}

新剑桥学派认为凯恩斯宏观经济分析的不足是缺少微观经济基础——没有考虑价值论与分配论,
这是凯恩斯批评新古典经济学的个体分析方法不能解决国民收入与就业问题,用总量分析代
之,强调总量忽视个量所造成的。但新古典综合派竟用移花接木似的方法将新古典学派的微
观生产函数分析和市场均衡分析移植到凯恩斯的宏观经济理论中……这是不能接受的,破坏
了凯恩斯理论体系自身的统一性,使凯恩斯经济理论面目全非,庸俗了凯恩斯的宏观经济分
析。

要为凯恩斯宏观经济分析提供微观基础,就是要重视凯恩斯所不重视的价值和分配的研究。
而要建立价值论,就要否定忽视社会制度和社会经济分析的主观分析方法。琼·罗宾逊……
“社会制度的性质。经济关系是人与人之间的关系。人类同物质世界的技术关系规定了人们过
着的经济生活的条件,虽然人类社会的技术发展水平对社会中的各种关系有着重大影响,但
技术条件并不能完全决定人类社会的各种关系。”具体地说就是要彻底否定新古典经济学的
边际效用的主观分析方法,不能把价值视为“主观”的概念,而要使价值具有客观物质基
础。恢复李嘉图的劳动价值论,但要解决李嘉图所没有解决的价值标准问题。建立以斯拉法
的“标准体系”为中心的客观价值理论。并在此基础上建立分配理论。

\subsection{收入分配为第一}

新剑桥学派的经济学者认为,收入分配理论是凯恩斯经济理论的重要内容之一。《通论》第
24章关于社会哲学的论述。凯恩斯主义是20世纪30年代经济大危机的产物。在当时的经济背
景下,关注的经济热点问题只能是失业问题,不可能是收入分配趋势这样长期和动态的问题。
所以凯恩斯只是提出贫富悬殊是社会不合理的现象,富人财富的增长不是出自于节约的积累。
政府应对收入分配不均状况进行干预和调节,消灭食利阶层的观点。但凯恩斯没有展开论述,
所以不为人所重视。新剑桥学派认为这一内容很重要,不应被忽略,应大力拓展。所以,新
剑桥学派特别强调收入分配理论,并成为该学派理论的核心内容。

……收入分配问题的研究应联系经济增长。要在经济增长过程中了解工资和利润在国民收入
比重中的变化,进而研究国民收入在社会各阶层之间的分配,揭示不均等分配的根源。……
他们反对新古典经济学的“储蓄决定投资”的传统观点,强调投资决定储蓄。也就是新古典
经济学认为在任何时期,储蓄额都是一个定量,通过利率的调节,总可以使储蓄全部转化为
投资,所以储蓄活动支配投资活动。而新剑桥学派认为根据凯恩斯的看法,应该是投资决定
储蓄。“储蓄不能不受投资量(增添设备和原材料的支出)的支配。储蓄水平随收入水平而
变动。在工人失业和设备利用不足的时候,投资支出的增加会提高收入,从而增加消费支
出,又增加储蓄”。新古典综合派提出,依据居民的储蓄倾向,计算出实现充分就业时所能
达到的国民收入中的储蓄量。政府再运用财政政策和货币政策安排足够的投资来吸收这笔储
蓄,经济就可以实现充分就业均衡。新剑桥学派认为,上述观点实际上是背弃了凯恩斯的投
资决定储蓄的基本看法,又回到新古典经济学的储蓄决定投资的传统看法上去,这是对凯恩
斯经济理论的背叛。

新剑桥学派进一步通过投资率与利润率关系等的分析,导出不均等分配的根源。他们认为投
资量的多少不仅可以决定生产和就业的规模和水平,而且也决定工资和利润在国民收入中的
比重。利润率的高低又取决于投资率的高低。

在经济增长过程中,投资率会发生变动。投资率与利润率呈同方向的变化。当投资率提高导
致利润率上升时,利润在国民收入中的比重会提高,而工资在国民收入中的比重则下降。由
此新剑桥学派得出结论:资本主义经济增长必然带来收入分配不均等,贫富悬殊的现象。所
以要消除这一现象,必须改革收入分配制度。

\subsection{左派倾向和马克思主义痕迹}

实行累进所得税,高额财产税等措施对财产多和高收入的人征收d更多的税,以改善收入的
不均等;采取一些福利性措施,给低收入家庭以适当的补助,以改变他们的贫穷状况等。明
显倾向于低收入阶层,不利于高收入阶层。

另外,新剑桥学派还一直试图寻找李嘉图、马克思和凯恩斯的理论共同点,将三者结合起来。

实际上罗宾逊并非马克思主义者,她并不赞同马克思所坚持的劳动价值论。她自己也承认她
是一位资产阶级的经济学家。

\section{批判他人,再现自我}

在于新古典综合派的论战中,新剑桥学派提出了价值理论,收入分配理论,经济增长理论以
及滞涨理论。

\subsection{斯拉法革命}

斯拉法价值论的基础是李嘉图的劳动价值论。李嘉图在他的劳动价值论中,由于未区分价值
与生产价格,因而无法解释等量劳动生产等量价值和等量资本或等量利润相矛盾的现象。李
嘉图始终未能找到一个共同计量价值的单位——不变价值尺度,即这种商品的价值在劳动的投
入量不变的条件下会随着工资和利润的分配份额的变化而变化。

产量受生产的技术条件水平,变动的影响,价格取决于由资方组织和工会之间谈判所确定的
工资与利润反向变动关系的结论。

新剑桥学派对斯拉法的价值分析,新剑桥学派给予高度评价,称之为“斯拉法革命”。首先
斯拉法以区别剩余的生产与分配过程,阐明剩余价值的生产是由物质生产条件决定,是一个
客观的过程;而剩余的分配则是与社会制度,经济关系有关的过程,它涉及的是不同阶层之
间的利益关系。以此来驳斥边际效用的主观价值论。其次……剩余分配的决定,必须和商品
价格的决定,通过相同的机制同时进行。最后,……重申商品价值的源泉是劳动。在他的方
程式里商品始终是劳动和生产资料相结合的产物。

由此,新剑桥学派认为斯拉法不仅坚持了李嘉图的劳动价值论,还由于解决了李嘉图所未能
解决的难题,实际上是发展了李嘉图的经济理论。

\subsection{不同凡响的分配理论}

收入分配理论是新剑桥学派理论体系的核心,它是在批评新古典经济学建立,新古典综合派
赞同的边际生产力分配理论中逐步形成。

新古典综合派以柯步--道格拉斯生产函数为例,运用数学中偏导数求解法,确定劳动和资本
的边际生产力。

生产要素的报酬同生产要素提供的边际产品具有一致性,工资等与劳动的边际产品,利润
(利息)等于资本的边际产品。劳动和资本的边际产品决定了工资和利润在国民收入中的比
例。国民收入的分配是由边际生产力决定的。……公正、平等……即使存在一些问题,也是
合情合理,无可非议的。新古典综合派还进一步认为,\textbf{问题的中心不是收入分配,
  而是经济增长问题。}

新剑桥学派认为新古典综合派的边际生产力分配理论是完全错误的。其一,完全竞争的前提
条件设置不合理。20世纪的资本主义现实经济是垄断竞争取代了自由竞争而居统治地位的时
代,仍以完全竞争为研究背景,与现实状况相差甚远。

其二,逻辑分析陷入循环推论的怪圈中。按照边际生产力分配理论的分析思路,各种生产要
素报酬的大小取决于其边际生产力的高低,边际生产力的高低是由工资率、利润率、地租率、
利息率来衡量。新古典综合派又认为产品的价值可以分解为工资、利润、利息、地租,这样
就产生了决定工资率、利润率、利息率、地租率的各种生产要素的边际生产力。

其三,将收入分配与人与人之间的关系割裂开来,新古典综合派不考虑社会制度和各阶层之
间的利益关系,将收入分配视作纯技术性的问题,以物质技术关系来决定收入分配,认为资
本主义制度下的工资和利润的分配是公正、合理,这是对收入分配现状的歪曲。

其四,资本边际生产力是否存在值得讨论。罗宾逊认为,资本收入由其边际生产力决定的分
析既然已陷入循环推理的泥沼,这样就找不到不依赖于价格和分配关系的衡量资本的独立单
位。正因为任何商品的价格(价值)都是由利润和工资组成,所有不同种类的量通过价格体
系归结为统一整体,它们之间的比例关系的变化仅对价格和资本综合产生影响,并不涉及价
格和资本的生产率。这样就不能i联系价格(或价值)用各种生产要素的生产率去说明分配
关系。斯拉法支持……并指出生产要素的提供与衡量要受分配关系的影响。同理,资本的实
际总量测定也要受分配关系影响。另外,斯拉法还认为不同资本是无法通过加总来衡量。因
为各种生产资料是异质的,没有一个统一标准的资本根本无法在量上加总。从而否定了资本
的可衡量性。

其五,资本不存在同一性。新古典综合派认为当利润作为资本边际生产力的报酬时,利润率
会受规模报酬递减倾向的影响,与资本积累总量呈反方向的变动关系。同时资本积累规模扩
大,资本总量增多也意味着可以采用资本密集型的生产方法。新剑桥学派的罗宾逊与斯拉法
都反对这种观点。首先认为这种看法是建立在根本不存在的资本同一性前提上……现实中不
存在……实际上资本呈多样性特点。接着提出再转辙理论。再转辙理论的主要内容是,生产
函数应用来反映一切可以采用不同生产技术进行再生产的过程。当利润率很低时,采用劳动
密集型的生产技术从事生产、生产效率高;当利润率很低时,采用劳动密集型的生产技术从事
生产、生产效率高;当利润率高时,采用资本密集型的生产技术从事生产、生产效率高。这
是向前转辙的过程。如果利润率及资本存货价值均很高时,则回复到劳动密集型的生产技术
从事生产的阶段。这是向后转辙。总之,不同利润率,支配着同量商品生产采用何种生产技
术更符合生产目的。既然利润率的高低支配着资本量的运用,就不可能存在资本的边际生产
力决定利润或利息的状况。新剑桥学派另一位重要学者帕西内蒂从新工艺的运用可以导致单
位投入中资本量变动--投入较多的资本量还是较少的角度也证实是利润率的变动控制着资本
量使用状态。从而认定用演绎方法来论证资本边际生产力决定利润或利息,是不科学的。

最后,在微观层次上分析收入分配,不可能探寻到国民收入分配的本质所在。在承袭凯恩斯
的总需求决定收入和就业,而收入分配状况反过来影响总需求水平和结构的基本观点的基础
上,构建一个宏观水平的收入分配理论。

\subsubsection{罗宾逊的分配理论}

投资决定经济增长,积累是投资的来源,积累量的大小取决于利润量的大小。……随着资本
主义经济的发展,国民收入分配越来越不利于工人阶级。因此,琼·罗宾逊呼吁政府进行干
预,解决分配不均问题。

\subsubsection{卡尔多的分配模型}

卡尔多认为当经济处于充分就业状态时,如果收入(产量)变动,预期储蓄将不等于预期投
资,收入分配关系--资本家收入、工人收入在国民收入分配中的比重也会发生变化。

随着资本主义经济的不断发展,投资率的提高,投资量的增长,国民收入分配愈来愈倾向于
利润的收入者——资本家。只有通过政府加以调控才能改变之。

\subsubsection{帕西内蒂的分配理论}

以卡尔多分配模型为基础,改变卡尔多的工人储蓄倾向为零的假设,在工人存在储蓄的情况
下研究收入分配问题。

相似结论。

\subsection{标新立异的经济增长说}

哈罗德--多马经济增长模型的基础上形成。它与其它学派的经济增长理论最大区别在于,将
经济增长通收入分配问题的研究相结合。一方面是阐述经济增长同收入分配结构变化趋势的
决定作用;另一方面说明可以通过对收入分配比例的调整去实现经济的稳定增长。

在现实中,增长绝不是稳定的。

罗宾逊进一步揭示资本主义经济不能增长的原因,并指出根源在于“劳动与财产的分离”,
即资本主义的生产资料私有制。……如果高利润是垄断造成而非高积累率的结果……改变国
民收入分配,使工资随劳动生产率的提高而增长,来解决资本主义经济增长中出现的不正常
状态。很明显,罗宾逊的基本看法是通过政府的干预,如果能抑制垄断统治的力量,改变国
民收入分配不均现象,资本主义经济还可以按一定速度稳定增长。

\subsection{滞涨新解}

\subsubsection{卡尔多的滞涨理论}

滞涨形成于初级产品生产部门与第二等级产品生产部门产量增长的比例失调。

初级产品生产部门——食品、燃料以及工业生产的基本原料等。第二级产品生产部门,将原料
加工成成品,制造品,投资或消费用。服务行业或第三产业,它的经济活动主要体现在为
其他部门提供各种服务。

卡尔多认为“持续和稳定的经济发展要求这两个部门的产量的增加应符合必要的相互关系。
这就是说,这种需求的增加又是反映第二级(以及第三级)部门的增长的。”但是,“从技
术观点看,不能保证由节约土地的革新所推动的初级生产的增长率,正好符合第二级和第三
级部门的生产和收入的增加所要求的增长率。”也就是说,卡尔多认为第三级产品生产部门
影响不大……当初级产品生产部门和第二级产品生产部门生产的增长速度不符合比例,就有
可能引发经济混乱,出现滞涨。

卡尔多还以不完全竞争市场的存在为前提,进一步指出滞涨形成于农产品价格的变动。非完
全竞争市场……鉴于初级产品与制造品价格形成中的不同特点,卡尔多认为“农矿产品价格
的任何巨大变动——不论它对初级生产者是有利还是不利——对工业活动往往起抑制作
用。”……初级产品价格下降产生的对工业制造品需求增长的正效应被相反影响抵消。卡尔
多由此认为本世纪20年代世界性大危机正是起因于初级产品价格大幅度的下降。……由此,
卡尔多认为美国1972--1973年之所以发生通货膨胀是因为初级产品,主要是农产品价格上涨
带动了工资的提高。美国政府为了缓解通货膨胀的压力,采取了力度比较大的货币政策,从
而爆发了一次相当严重的经济衰退。

\subsubsection{罗宾逊的滞涨理论}

罗宾逊除与卡尔多一样,区分各类型的市场,分析控制价格的原因去研究滞涨问题外,还从
两个方面去分析滞涨。一方面是,着重从货币和资本主义经济的“不确定性”去认识通货膨
胀产生的原因。她接受凯恩斯的货币理论,认为资本主义经济活动是由商品经济和货币经济
组成。货币在经济活动中,不仅作为流通手段,还作为储藏手段。正是由于具有这些作用,
货币成为当前与不确定的将来之间的纽带,成为财富存在形式中流动性最大、风险性最小的
方式。……经济资源比较多地流向投资品生产,比较少地流向消费品的生产,结果工人的实
际工资水平将下降。……虽然在货币经济中,名义工资是以货币形式支付,并由工会与资方
协商谈判确定。但实际工资水平的高低则是由厂商的投资决策确定。因为投资回报率较高,
造成消费品供给减少和实际工资水平下降,将会产生一种被琼·罗宾逊称之为的“通货膨胀
障碍”。即“在一个工会强大的现代经济社会中,厂商要提高利润率以压低综合工资份额
(特别是它如果导致实际工资率下降的话)的企图受到坚决抵制。于是厂商提高货币工资率
以免工人罢工……这被认为是提高利润的通货膨胀障碍。”这就是说,投资率上升导致实际
工资水平下降,工人以工会形式出面与资方谈判,要求货币工资提高以弥补实际工资水平下
降而带来的利益损失,而厂商又会将货币工资增加所造成的成本提高部分加到价格上去。以
提价的形式收回成本增加所造成的利益损失。这就会引发工资--价格螺旋式上升的通货膨胀,
出现经济停止增长,失业现象严重的结果。

另一方面,罗宾逊依据波兰经济学家卡莱茨基的理论,从“政治方面的商业循环”存在与发
展状况,分析滞涨的产生。

卡莱茨基认为,厂商与政府因追求的目标不同,对失业的认识也就不同。以追求最大化利润
为生产经营目的的厂商,注重生产秩序的稳定,认为存在一些失业工人……(类似于马克思
的相对过剩劳动力)。政府则以实现充分就业,稳定物价为宏观调控目标,为此政府一般会
采用扩张性的财政政策,即减税、扩大政府支出等来刺激生产发展,增加就业。但长时期推
行扩张性的财政政策,则会遭到大厂商们及失利阶层的反对。厂商……相对过剩劳动力……
食利者阶层则认为扩张性财政政策的推行使通货膨胀率提高,他们固定利息收入的实际价值
下降,收益减少。于是,厂商与食利阶层联合起来,反对政府实施扩张性的财政政策,迫使
政府采取减少政府支出等的紧缩性财政政策来调控经济活动,从而生产萎缩,失业率提
高。……当面临下届政府选举时,政府为了取悦选民……减税、增加政府支出……结果又出
现失业和缓,通货膨胀严重的局面。……循环……“政治方面的商业循环”

萨缪尔森赞同卡莱茨基的上述分析。“经济生活随着政治节奏而摆动”……为了消除经济的
周期波动,萨缪尔森及新古典综合派建议政府实行“需求管理”政策,重点调节总需求,并
辅之以其它政策,比如工资--物价管制措施等。……在不同时期,针对总需求的变动,政府分别
实施扩张性和紧缩性政策。

罗宾逊反对萨缪尔森的上述,并主张这只能使“政治方面的商业循环”更加严重。萨缪尔森
的“需求管理”政策并非真正符合凯恩斯革命之本意。凯恩斯革命倡导实行一种稳定的、彻
底消灭失业的需求管理政策,而不是使这种政策蜕变为政治家争取民心、获取选票而使用的
工具。这样将使滞涨越演越烈。

\section{激进的改良之术}

新古典综合派认为实现充分就业、消除经济危机、稳定物价、加速经济增长四个方面政策目
标中经济增长处于首位。因此他们主张运用财政政策和g货币政策来调整总需求,并先后采
取了“补偿性”、“增长性”与“多样化”的财政货币政策,以实现国民经济的均衡增长。

新剑桥学派认为,新古典综合派上述主张不能解决问题。……许多弊端,资源浪费、生态失
衡、环境污染、两极分化。在新剑桥学派看来,新古典综合派的经济政策一方面是体现在政
府实行高额赤字财政政策,主要是连年增加军事支出。“军事支出是对付企业活动下降和失
业增加趋势的急救方案。”但过度的军事支出有其负效应。因为“军事工业的发展对经济其
它部分是不利的”。如果将一部分军事支出专用于发展经济,则有助于提高经济效率和人民
生活水平。另一方面体现在强调发挥混合经济的作用。混合经济的实质就是政府与私人企业
共同实施对经济活动的控制。具体地说就是政府出钱,私人企业去自由经营。这样做的后果
必然是私人企业生产经营决策以投资高利润率为导向,会造成经济资源不合理使用。同时,
新剑桥学派还抨击新古典综合派以工资--物流管制为中心的收入政策也是不奏效的。因为对
工资、物价的控制,实质上是控制了生产要素所有者的要素收入。这种性质的收入政策不仅
不能解决滞涨问题,反而会使现实生活中收入分配不平等的格局以行政和法律的形式固定下
来。

货币主义因为反对凯恩斯的经济理论与政策,一直不主张政府干预经济,听任市场机制发挥
作用,自动调节经济。并认为当代资本主义经济现实中出现的劳动生产率增幅缓慢、环境污
染、滞涨等现象的出现,均与国家干预经济相联系。所以力主缩小国家调控的范围力度,并
促使国家调控与市场机制有机结合。

新剑桥学派不同意货币主义的上述观点。首先他们认为,事实胜于雄辩。1929--1933年世界
性大危机的爆发已经向世人宣告市场机制可以自动调节经济,使之达到均衡的作用是根本不
存在的。市场机制是一个效率极差的调节工具。其次,新剑桥学派反对过分突出货币的作用,
并认为货币主义企图通过控制货币供给来控制通货膨胀是根本行不通的。因为货币量与生产
量之间的联系并不十分直接与紧密。减少货币供给量虽然能够影响总需求,但作用有限。因
为货币量减少时,供给超过需求的不平衡状态缓解,商品价格由高回落。但待货币收入增加,
通货膨胀提高,控制货币供给量就只能限制一部分总需求,只可以起到局部的减缓作用。

新剑桥学派认为,现实资本主义活动中的主要问题是收入分配的不平等,收入分配不平等就
是指国民收入中工人和资本家所拥有的份额不均等。这种不平等状况的形成不仅与历史的、
制度的因素有关,即现阶段社会的分配形式是在部分占有生产资料,另一部分人未占有生产
资料这一历史形成的不合理的分配制度中产生。而且与经济增长速度等因素有关。随着资本
主义经济的不断增长,在国民收入分配中资本家所拥有的比重越来越高,工人……比重则不
断下降……社会的、经济的、政治的问题。\textbf{收入分配不平等是资本主义一切问题的
  症结所在。所以必须从收入分配方面去调节资本主义经济活动。鉴于现存社会的分配制度
  是形成收入分配不均等的原因,因此不可能在目前制度下通过市场机制的调节作用来改变
  不合理的分配状况,只能借助政府的力量。收入分配的调整应是政府宏观经济政策的主要
  目标。}以下主张:
\begin{enumerate}
\item 实现收入均等化措施。一是税收制度:累进税,对高收入者课以重税;高额遗产税与
  赠与税;将税收用于社会公共目标和提高低收入贫困阶层的生活水平。二是加强完善福利
  措施,缓解“富裕中的贫困”现象。政府以财政拨款的形式,为失业者再就业培训提供费
  用,……帮助再就业;提高失业保险等给失业人员提供一个维持最低生活水平的保障;政
  府以补贴等形式关心低收入阶层,增加这一部分家庭的生活水平。


\item  对投资进行有效管制。批评新古典综合派曲解凯恩斯关于扩大支出就可以增加就业
  观点。新古典综合派不考虑政府支出的内容和资源最佳配置问题,盲目扩大政府支出,搞
  赤字财政政策,以增加就业量,结果出现了经济盲目高速增长及国民经济军事化倾向,出
  现了经济资源的大量浪费和生态失衡、环境污染的不良结果。为此新剑桥学派倡导,政府
  在解决就业问题时,不仅要注意总量的增长,而且要注重就业的内容。为克服经济盲目增
  长,mj避免经济扰动的经常出现,要对投资实行全面的管制。1.政府确定适应经济均衡增
  长的财政政策,减少财政赤字,将赤字政策的政策取向逐步转变成平衡财政政策,并根据
  均衡经济增长率来确定实行工资增长率,以改变国民收入分配中不利于工人的状况,逐步
  在经济增长过程中,去除分配的不合理性。2.政府动用财政预算的盈余去购买私人企业的
  股票。以达到控股的目的,将部分私人企业转化为国有企业,有利于政府对国民经济的调
  控。3.政府要逐渐减少军事支出,让资源从军事部门流向民用品的生产、原材料供给、环
  境保护等,让有限资源得到充分利用。4.实施进出口管制,利用本国经济资源的优势,促
  进出口商品生产规模的扩大。……外汇收入……劳动者。

\end{enumerate}

\section{评说新剑桥学派}

新剑桥学派将弊端原因归结为政府推行了新古典综合派的冒牌凯恩斯主义政策。新剑桥学派
以正统凯恩斯学派自居。

但因为新剑桥学派的经济理论及以此为基础的经济政府主张并不触动资本主义生产资料私有
制这一根本,只是企图通过改变资本主义现行的收入分配制度来缓解资本主义的基本矛盾,
促进资本主义经济的发展,这注定是矛盾的。既不为现行的西方发达工业国家政府所接受,
也未表现出与其他西方经济学派有什么根本区别,始终未处于主导地位。

正如百年前马克思评价古典经济学的状况一样,“那些把生产当作永恒的真理来谈论而把历
史限制在分配范围之内的经济学家是多么荒诞无稽。”

\chapter{新瓶装旧酒还是一场新的革命\\ ——新凯恩斯主义的勃兴}

\section{凯恩斯真的死了吗}

\subsection{霸主衰落}

60年代末、70年代初经济衰退和通货膨胀相结合的怪物——“滞涨”在历史上产生了。凯恩斯
弱点暴露,从此经济学界凯恩斯一统天下的时代一去不复还了。

凯恩斯主义对货币作用估计不足,引起货币主义的抗衡;对供给方面分析的缺乏引起了供给
学派的讨伐;而对人的理性的忽视则遭到理性预期学派的无情鞭挞。更为要紧的是,凯恩斯
主义本身一直就存在理论体系上的缺陷。凯恩斯主义一直没有建立起完整的包括微观和宏观
的经济学体系。新古典综合派将凯恩斯的宏观经济学和马歇尔的微观经济学结合了起来,但
显然没有能弥补两个理论体系上的段痕。正如斯蒂格利茨所指出的,传统凯恩斯主义教科书,
如萨缪尔森《经济学》,实质上是将经济社会中的体制分成两种。当社会的资源未能充分使
用时,宏观经济学原理可以发生作用;但是当社会的经济资源得到充分使用时,发生作用的
是微观经济学原理。凯恩斯主义“先天发育不足”。

新古典宏观学派的代表人物如卢卡斯和萨金特等也开始宣布凯恩斯主义已经死亡。因为他们
认为,凯恩斯主义理论存在两个基本的困境:一是非市场出清缺乏充足微观的基础,二是凯
恩斯主义模型中的预期的假定与最大化行为不一致。

80年代新古典经济学派对经济长期低迷处理的不力,使得新古典宏观经济学政策的可信度又
进一步受到了质疑。弗里德曼承认,新自由主义政策在实践中并没有取得突出效果,远远低
于各阶层的厚望。理性预期学派根据自然率假说认为通货膨胀无害,但事实是,当美国通货
膨胀率由1980年的10\%降到3\%时,产量却下降很大,总需求下降也很多。货币主义主张单
一的货币规则,但在1982年后的几年里,美国一方面是最快的经济增长,另一方面却出现最
剧烈的通货紧缩。在供给学派的减税政策方面,在80年代的减税过程中,美国的储蓄率却是
下降的。而且,新古典经济学派政策10年多的实行,使得美国的经济陷入困境:1989
2.5,1990 0.8,1992 -1.2,同时失业浪潮也席卷工业国家,欧共体12国失业人数今年高达
1700万,平均失业率11\%,美国的失业率7\%,失业人数超过800万。凯恩斯主义似乎又开始
“起死回生”。

\subsection{反思与修正}

凯恩斯主义学派阵营中越来越多人对传统凯恩斯主义经济学产生不满,并进行反思。凯恩斯
本人就从来没有认真考虑过经济学的微观理论基础,而所谓的凯恩斯主义主流经济学派经济
学,则牵强附会地将基于均衡分析的新古典经济学和基于非均衡分析的凯恩斯经济学捆绑在
一起。逻辑本身有矛盾。

一些人开始寻找凯恩斯宏观分析的微观基础,最初结果便导致了一般非均衡理论的产生。一
般非均衡理论具有两个弱点。从现实看,它并没有为解决当时西方所面临的“滞涨”问题提
供有效政策;从理论方面看,也没有能为解释价格和工资刚性提供有力的说明。正如新古典
经济学家所批评的,假定货币工资刚性以及更加一般的价格刚性而不解释为什么如此,那是
不能被接受的,这一批评说服了巴罗,并导致他对非均衡主义的被判。

对凯恩斯主义更为强力的挽救是新凯恩斯主义的一群年轻经济学家。他们认为,宏观经济学
必须建立在微观基础之上;他们主张应用市场出清和w经济行为始终最优化的假定基础之上
的宏观经济学理论来取代凯恩斯主义经济学……为了挽救70年代出现的“凯恩斯主义理论危
机”。

新凯恩斯主义接受新古典学派的理性预期学说,认为这不会成为凯恩斯主义结果的障碍,相
反,他们在此基础上构造了凯恩斯主义的微观基础。

对市场出清的看法不同。新古典经济学家运用完全信息条件;新凯恩斯主义则认为,预期可
能修正很慢,因此工资会调整很慢,事业也会持续很长时间。在长期中,供给曲线可能是垂
直的;但是,关键问题是“我们要等多久,才算长期”。新凯恩斯主义一词 1984 1988年出现。

\subsection{兼收并蓄}

归纳为以下几个方面。

\begin{enumerate}
\item 对凯恩斯主义主要观点的继承。
  \begin{enumerate}
  \item 产量和就业波动的性质。新自由主义特点的宏观经济学派,包括货币主义、理性预
    期学派以及实际周期理论尽管也承认经济波动,但却认为这是一种均衡的波动。劳动市
    场和商品市场总是保持供求一致的均衡状态,波动仅是由于供求曲线移动的结果,不存
    在非自愿失业,也不存在普遍的生产过剩。新凯恩斯主义则认为,实际产量和就业量波
    动是非均衡的性质,普遍生产过剩,非自愿失业。


  \item 引致波动的因素。新古典宏观经济学认为,除非货币供给量等的变化是完全随机的,
    否则就不会造成产量和就业量的波动;而凯恩斯主义学派则认为名义总需求的冲击是造
    成非均衡产量和就业量波动的主要因素。


  \item 造成经济波动的深层次原因。弗里德曼和理性预期等学派的经济学家大都从经济主
    体预期发生差错的角度,来说明总需求变化的实际效应。而凯恩斯主义则认为经济体系
    本身的不完全性(不完全市场、不完全信息等)是名义总需求冲击实际效应产生的原因。


  \item 解决变动的政策措施。新自由主义或多或少地反对政府对经济的干预,而凯恩斯主
    义经济学家则比较赞成政府对经济进行一定程度的干预。以弥补市场的不足。
  \end{enumerate}

\item 对凯恩斯主义的发展和补充。所谓凯恩斯主义复兴,并不是传统凯恩斯主义的回归。
  萨缪尔森和诺德豪斯说“\textbf{经济学本质上是一门发展的科学,它的变化反映了社会
    经济趋势的变化。}”

  新凯恩斯主义认为传统凯恩斯主义往往把产量和就业量的波动仅仅归因于名义工资和名义
  价格的刚性,即把工资--价格刚性看成是解释名义总需求变动造成实际经济变量变动的唯
  一或者至少是主要的原因。而名义的工资--价格是指以货币为单位来衡量的工资和价格。
  新凯恩斯主义的新颖之处是:
  \begin{enumerate}
  \item 它不像其他凯恩斯主义学派那样只是假定工资--价格刚性,而是试图基于这种刚性
    以合理的解释,特别是从微观的角度——在单个经济单位具有理性预期并追求最大化的基
    础行为上——来解释工资和价格的刚性。微观基础。

  \item 强调经济体系中其他方面的不完全性。如经济体中存在的实际刚性、风险和不确定
    性、不完全的市场结构、不完全的且昂贵的经济信息等等。而且,他们对于非自愿失业
    和经济波动的解释主要不是名义工资和价格的刚性,而是经济体系中这样的一些“不完
    全”因素。


  \item 供给和需求两方面的震动都是经济不稳定的潜在根源。
  \end{enumerate}


\item 之前,凯恩斯主义也吸收过其他经济学派观点。例如根据IS--LM分析,认识到货币政
  策重要性。如莫迪里安尼就曾说过,我们都是货币主义者。但是,直到新凯恩斯主义兴起,
  才大规模借鉴了其他学派特别是新古典宏观经济学的分析方法。宏观经济学之间现存分歧大多
  来自卢卡斯及其追随者的批评。


\item 采用了其他流派的一些新的分析工具。利用最新的一些分析工具如博弈论、混沌理论
  等,构建了新凯恩斯主义的一些基本模型,如集体谈判理论、局内人--局外人模型、不完
  全市场模型、多重均衡理论、太阳黑子理论等等。新凯恩斯主义还由此发展了新的经济学
  分支,如信息经济学、混沌经济学、产业组织定价经济学等。

\item 新凯恩斯主义否定了新古典宏观经济学派的一些前提假设。

  接受了新古典宏观经济学的行为最大化的界定,这样发展出了包括理性预期在内的具有新
  古典经济学风格的但结论却不同的新凯恩斯主义经济学。主要反驳新古典宏观的两个重要
  假定:市场出清和理性预期。

  关于人们是否是理性预期的。新凯恩斯主义者认为,人们虽然按照最大化采取行动,但是
  由于信息是不完全的,因此在很短时期内很难形成理性预期。虽然我们的预期最终也许会
  是理性的,但在这个最终到来以前,需要花一定时间调整。

  关于市场是否出清。一方面是,如果修改了理性预期假定,则经济的总供给曲线就不一定
  是垂直的,而可能是向右上方倾斜的。这样解释,市场是出清的,但是出现了低于充分就
  业的可能,为政府的总需求政策提供了一定的理论依据。

  其次,如果修改市场出清假定,则经济就可能不处于总需求和总供给曲线的焦点上,即价
  格水平和工资水平可能高于或低于均衡的水平。这时,“短边法则”就发挥了作用。在这
  两种情况下,产量和就业量都由较少的需求一方或供给一方决定,结果,都可能造成低于
  充分就业的均衡。

  单个人的决策导致了宏观经济的外部化,因此,决策并不一定是社会最优的结果,也就是说
  个人追求理性的结果,并不一定会自发地导致集体理性的出现。那么,如何把这种宏观经济
  的外部化问题内部化呢?新凯恩斯主义的结论是:政府的决策具有潜在的作用。

\item 新凯恩斯主义无论是在分析方法还是观点方面都具有独到新颖之处。

  首先,新凯恩斯主义的模型假定制定价格的企业是垄断企业,而不是完全竞争的企业,因
  此,他们不是价格的接受者。运用不完全竞争建立市场出清的模式。传统凯恩斯主义将价
  格视为刚性的。但新凯恩斯主义则将价格视为粘性的,即价格是可以变动的,但变动很缓
  慢。

  其次,它也认为,宏观经济学应该建立在微观经济学的基础之上,并且,理解宏观行为要
  求建立一个简单的一般均衡模型。但新古典宏观经济学是利用完全信息、完全竞争、不存
  在交易成本、完全市场体系、有代表性的经济主体等等假定来建立自己的模型,而新凯恩
  斯主义的模型则建立在不完全信息、不完全竞争等假设之上。

  再次,曼丘和罗默明确表示,新凯恩斯主义经济学意味着要对以下两个问题作出解答:1,
  这个理论违背古典派的两分法吗?它断定名义变量(如货币供应)的波动影响实际变量
  (如产出量和就业量)的波动吗?即货币是非中性的吗?2,这个理论假定经济中的实际
  市场不完善性是理解经济波动的关键吗?如不完全竞争、不完全信息和相对价格粘性这些
  思考是理论的核心吗?新凯恩斯主义对此的回答是肯定的。而且只有新凯恩斯主义对这两
  个问题都作出肯定的回答。许多较早期的宏观经济学理论(包括非均衡的凯恩斯主义模型
  在内的大多数凯恩斯主义经济学)虽然抛弃了古典派的两分法,但都只是把工资和价格粘
  性塞进另外的瓦尔拉斯式经济,或硬充作瓦尔拉斯体系。

  最后,新凯恩斯主义是建立在对传统的凯恩斯主义模型和货币主义模型所建立的基础的
  “反叛”之上,这两派的大多数学者都把经验证据视为比理论上纯正更加重要。新凯恩斯
  主义具有明显的非经验风格、微观基础。“把稀缺的研究资源投向了理论研究方面,而不
  是实证研究方面。”


\item 新凯恩斯主义还不是一个紧密性的团体。都采用了理性预期的假设,同时,又认定了
  各种市场的不完全性对宏观经济学依然是有重要含义。因此,他们一般都得出国家干预的
  凯恩斯主义结论。
\end{enumerate}

\section{中流砥柱的理论}

\subsection{工资粘性理论}

\begin{enumerate}
\item 实际工资粘性论。

  \begin{enumerate}
  \item 隐含合同理论。企业比工人承担风险的能力更强,因而它假定企业是风险中性的,
    而工人是风险厌恶者。从这个前提出发……就业关系不仅仅是劳动和工资之间一次性的
    现货交易关系,而是一种涉及较长期的合同保险关系。这种保险合同可以避免工人的收
    入的不确定性。因此,合同工资不再等于劳动的边际产品,而是相对固定。

    但是,为了减低经济波动造成的冲击,为什么企业不实行工作分摊制,而是解雇工人呢?
    新凯恩斯主义认为,工作分摊的方式比解雇工人的方式,使厂商损失更多的生产率。另
    外,政府对失业工人的补贴也鼓励了解雇工人的做法。

    早期是完全信息的隐含合同论。1983年起,不完全信息的隐含合同论。信息不完全,即
    只有企业对经济状态了解,而工人却一无所知。

    在非对称信息下,企业就有了隐瞒真实情况的动机。因此,在完全信息下的最优合同就
    很有可能没有被执行,结果导致实际的就业也不再是有效的。


  \item 集体谈判理论。非竞争性工资不再由劳动需求和劳动供给的交点决定,而取决于谈
    判双方(企业和工会)垄断的力量。极端地,工资完全由企业来决定或完全由工会来决
    定。

  \item 效率工资理论。实际工资的高低会影响工人的生产效率,而工人的生产效率又会影
    响企业的利润。

    \begin{enumerate}
    \item 生产效率的筛选。 要提高企业的生产效率,两方面途径,一是挑选高质量的工人,
      二是激励工人的工作。另外,一个人的生产效率不仅取决于他的自身能力,还与他的
      努力程度有关。

      新凯恩斯主义经济学家用夏皮罗和斯蒂格利茨建立的“怠工模型”来进行分析。在低
      工资下,个人的机会主义就非常强……偷懒……

    \item 劳动力流动模型。跳槽成本。


    \item 劳动力市场结构。一些部门是高工资,另一些部门则是低工资。在低工资部门就
      业,相对于在高工资部门的就业者来说,实质上是一种失业,就如平常的失业者待在
      家里从事“家庭行业”生产一样。


    \item  效率工资的决定。效率工资是指雇主希望支付的能是劳动成本最小的工资。而
      使劳动市场出清的工资,是使劳动供给等于劳动需求的工资,称为市场出清工资。显
      然,效率工资不一定与市场出清工资相等。新凯恩斯主义认为,通过支付比市场出清
      更高的工资,可以使劳动成本最小。

      效率工资理论也表明,工资调整的过程是缓慢的,每个厂商都不愿率先降低工资。
    \end{enumerate}


  \item 回滞理论:局内人--局外人模型。

    效率工资理论运用工资和生产效率之间的关系来解释非自愿失业的存在;集体谈判理论
    则分析低效率就业的原因。而新凯恩斯主义经济学家又发展出了兼有效率工资理论和集
    体谈判理论双重特性的回滞理论。

    回滞理论研究变量之间的非单一对互关系。回滞理论的特性是,因变量由于自变量的影
    响而发生变化;但当自变量的冲击消失之后,因变量不会返回初始状态。……黑色金属
    短暂电流冲击之后,永久磁化。

    当政府运用经济政策去降低通货膨胀时,失业率会按照菲利普斯所指示的方向上升,但
    失业率一旦上去了,就不会再沿菲利普斯向反方向运动,即使通货膨胀率明显上升了,
    失业率也不会下降。因此,回滞理论认为,自然失业率是不存在的, 而是一个与自己
    的历史有关的随时间变化的变量。

    耶伦于1984年借鉴回滞理论提出了局内人--局外人模型来解释劳动力市场僵化的过程。
    该模型认为,失业工人并不是就业工人的完全替代物,工资调整在很大程度上取决于在
    职工人而不是失业工人;而且,长期失业者对工资调整几乎没有影响。


  \item 搜寻和匹配模型。

    传统劳动力市场是基于瓦尔拉斯假设的基础上,认为工人和厂商都是同质的。所以,在
    这样的劳动力市场上,厂商不在乎失去自己的工人,因为能够以g相同的工资没有成本
    地雇佣到相同的工人;同样,工人也不在乎失去自己的工作。

    该模型假设,正的失业和正的工作可以并存,它还设立了一个匹配函数,来表示雇主招
    募新工人、工人搜寻以及雇主和工人相互评价等复杂的过程,并假定匹配函数是规模报
    酬不变的。

  \end{enumerate}


\item 名义粘性工资论。
\end{enumerate}