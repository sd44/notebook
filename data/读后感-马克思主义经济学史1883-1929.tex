\chapter{《马克思主义经济学史 卷1:1883-1929》感性的水货读后感}

\section{引子}

我们应当学习的是马克思主义史,而不是局限于马克思个人的马克思学。 正如马克思的著
作有大部分内容建立在前人基础上,然后发展并融入自己的真知灼见,马克思主义经济学史
中的诸多人物也是建立在前人的理论基础之上,并发展融入。无法想象一个前不见古人后不
见来者的超天才横空出世,提出诸多独创的深刻见解。只有在这种理论联系实际的历史考察
中,才有助于我们真正理解和加深理解马克思,并对其进行继承、发扬和批判、扬弃,这也
是马克思历史辩证法的核心所在。在这一学习过程中,我们也会对历史,国家——特别是中苏
等社会主义国家历史有着更为清晰的理解。从而明得失,知兴替。

我对马克思本人著作基本只限于《资本论》,对他的科学社会主义理论和马克思主义经济学
史的了解几乎是完全空白,本读后感仅限于这一本二手文献。并且我是初级马克思爱好者,
各方面水平有限,这更加降低了这篇读后感的价值,希望读者给予更多建议和批判。

特别感谢知乎 @克里夫托,在我向其索要最为简短的马克思书单时,他向我推荐了这本书。

\section{阅读本书的困难}

在阅读和理解马克思著作的过程中,读者有时经历晦涩平淡的低谷——特别是《资本论》第二
卷中部分内容;有时为英国当时老曼彻斯特大工业时代的工人和童工们的悲惨状况哀伤;有
时愤恨于为罪恶开脱的庸俗传教士、卫道士嘴脸,有时也会对人类社会产生绝望的情绪。但
是,即使面对这些负面情绪和低谷,在阅读量和理解程度的日益加深中,读者仍常是斗志高
昂的,经历层层叠起的高潮,批判精神与日俱增……总体来说,阅读马克思的过程仍是正面、
光明且充满力量。

而阅读马克思主义经济学史不同。马克思于1883年逝世,遗留下未完成的《资本论》二、三
卷、剩余价值论和一些未出版的著作。在1930年后,《1844年经济学哲学手稿》和英文版的
《政治经济学批判大纲》才出版,本卷书便写作于这个时间段。马克思在此成为不具本体的
幽灵。他时而浮现,体现出自身伟大价值;时而消隐,面对着各种质疑、批判,被指出明显
的不足和错误;时而强大,绘出整个人类历史中表现出来的原始性、动物性,并指导构建宏
伟蓝图;时而渺小,资本主义的自我调节能力,应用一切为自己增殖开辟道路的能力、吞噬
和改造异议的能力时常显示出资本主义的强大,并指责着马克思。


从马克思原文,一下子跳到具有较强批判和知识点密集的本书,使阅读过程并不轻松。历史
过程中,不管是外部对马克思主义经济学的批判和修正还是马克思主义者自己内部的批判、
分裂和修正都很尖锐和残酷,这中间方法论、理论稠密且彼此联系又互相矛盾,每个人都在
不断扬弃和发展自己与他人的理论,这导致阅读马原文时的总体正面激昂被跌宕起伏、阴晴
不定、前途未知所取代。我在阅读过程中对于作为业余初级民(社)科,理论水平低的自己
是否应该学习马克思产生怀疑,甚至有两三次心生退意,。但随着阅读的进一步开展,我也
就正视了批判理论,并且喜欢上了这种不断充满挑战和辩证的思维过程,勇于攀登高峰。

总的来说,阅读本书的基础,可能是需要对《资本论》三卷均有初步了解。

\section{政治经济学家八卦}

为避免太过枯燥,先插入八卦一节。

帕尔乌斯这个奇葩,几乎最早提出经济危机长波理论,并和托洛茨基联手提出“不断革命
论”,此外还有一些深刻见解,结果贪墨高尔基稿费,还成为了在各国之间大发国难财的军
火贩子,大资本家,最后还债台高筑。

奥托·鲍威尔重新演绎了卢森堡的消费不足再生产图式,认为封闭的资本主义社会内通过
\textbf{消费部类资本家和银行家}转而投资生产部类,导致总剩余价值可以实现。因其假
设资本有机构成快于剥削率,并\textbf{人为将两个比率设计成恒定增长}(这个模型的缺
陷所在),这最终将导致资本危机周期。被本书作者认为是这个时间段内尝试进行的最宏观
的动态分析,他的学说也要在10多年后才为人重视和理解。在他的学说中,有个未被他自己
和同时代其他人发展健全的理论,就是金融信贷资本的调节作用。牛的是,他的老婆海琳·
鲍威尔也是个超级猛人,在10多年后斯滕伯格和亨利克·格罗斯曼两大超级猛人批评鲍威尔
理论时,阐述了\textbf{信贷资本}的有力调节作用和自身不足。并对帝国主义侵略看法比
较超前——“海琳·鲍威尔的新熊彼特主义式的认识,即返祖性的前资本主义思想应当为帝国主
义侵略负主要责任。”虽然也带有修正主义的错误乐观色彩,但我觉得可以被认为是对军国
主义会破坏资本主义整体生产条件的一种阐述。

\section{1883-1929争论焦点}

一、主要涉及价值、市场价格、成本价格、生产价格价格的联系问题、转形问题,其中生产
价格问题又涉及一般利润率——这又涉及总价值和总价格相等、总剩余价值和总利润相等的
“两个相等”问题。

庞巴维克对马克思价值理论的批驳是有一定力度的,虽然他的理论相比马克思有更多不足。
至少在30年代美国经济危机前,主流经济学家均无法有效解释生产过剩、危机等问题。希法
亭和布哈林对其做了不完整的批判。我个人觉得,确实是要修正马克思价值观点,并采纳部
份主观价值论的观点,就本书来看,布哈林也确实采纳过一丢丢这方面的内容。

马克思在《资本论》第三卷P187-188对于生产成本的论述中,提到自己的在计算商品不变资
本时的“误差”:不变资本实际上不是等于所应用生产资料的价值总和,而是生产成本总和。
但可能因其认为这种小误差相比于一般价值规律不值一提,所以说“对我们现在的研究来说,这
一点没有进一步考察的必要。”

德米特里耶夫抓住了这点,应用瓦尔拉斯边际效用论和李嘉图劳动价值论,使用“'过去劳
动'模型取代了马克思对于不变资本和可变资本的区分”。他的分析“框架对后来的劳动价
值论和转形问题研究具有巨大的价值。”

博特凯维茨通过对德米特里耶夫的分析框架的逻辑分析,得出了马克思的因果链条是“连续
近似”谬误——马克思在论述生产成本中的不变资本和可变资本时,不变资本中的组成商品使
用的是价值计量,但结果是生产成本计量。不管是往前推还是往后推,这一整个链条都是不
断有误差的近似。这种误差的累积就叫做“连续近似”谬误。

“两个相等”的问题也因博特凯维茨因对奢侈品生产部类的研究(马克思那是IIb部类,博
特凯维茨将其扩展为第III部类)而宣告两个条件不可能同时成立。关于第III部类的特殊性
的论述在之后数十年还会掀起风浪。

二、危机理论问题。资本主义必然崩溃的崩溃论,基本上被德国的修正主义者经过考证后给
否定了,他们认为马克思从来没说过崩溃论,只说了危机理论。但马克思提到的危机理论有
\textbf{利润率下降、过度积累(由于资本积累率提高过快导致失业大军枯竭和工资提高使
剥削率下降造成的,我认为这同二十世纪七十年代主流经济学家所提出的“利润挤压”理论
是异曲同工的)、消费不足、生产资料部门和消费部门的比例失调等。}以今人视角看来,
资本主义的危机理论因马克思没有详细、系统、完整、一贯地论述,而成为马克思主义史中
永远存在的问题。例如马克思先提出消费不足理论,后又认为消费不足理论不足以说明实质
问题,只是同义反复。各类学者、专家针对危机理论各执一词,取一二危机理论加以阐述。
但是对这些问题的论述往往是最为精彩纷呈的。因论述过于复杂,碍于本文篇幅,我就不谈了。

三,生产决定论问题。我个人认为“经济基础决定上层建筑,上层建筑反作用于经济基础”
不管怎样解释,都是具有决定论性质的。事实上马克思从没直接提出过“决定”这句话,大
卫哈维也认为如此。我认为比较清晰的表达出现在《政治经济学批判 序言》:“人们在自
己生活的社会生产中发生一定的、必然的、不以他们的意志为转移的关系,即同他们的物质
生产力的一定发展阶段相适合的生产关系。这些生产关系的总和构成社会的经济结构,即有
法律的和政治的上层建筑竖立其上并有一定的社会意识形式与之相适应的现实基础。”恩格
斯进一步发展了“经济基础决定论”观点,这句话本身是由后人总结出的。关于这种决定论,
我认为“俄国马克思主义之父”普列汉诺夫的批判是合适的:“它必须寻求有机的整体因果
联系的基础。……普列汉诺夫坚信历史唯物主义,但他坚持认为,自然条件或生产力的决定作
用的发挥,采取了中介形式,即采取相对自治的社会关系的亚结构的中介形式。……在更为分
化的社会,经济基础的作用是通过阶级关系体系、政治权力和法律体系的结构发挥的。”


书中有一重要结论,\textbf{“资本主义制度的本质恰恰在于它的无政府性,资本主义(除
  了不断增殖外)没有也不需要一种目的,对资本主义进行任何形式的目的论的解释都是不
  恰当的。”}这也否定了“生产决定论”。

四,科学社会主义实践问题。这块我单独放在下一节来说吧。

\section{德国和俄国的马克思主义发展}

正如作者M·C·霍华德和J·E·金在导言中所说,“马克思主义政治经济学在这些年的发展,
同\textbf{实践性的政治问题}不可分割地交织在一起。”。我觉得书中有句话非常经典,透
露出现实社会的吊诡和无常:“\textbf{主要关注推动资本主义发展的俄国人,以破坏资本
  主义的发展而告终;期待资本主义崩溃的德国马克思主义者,却开始与资产阶级合作并维
  护资本主义。}”

\subsection{德国的科社实践}

在马克思主义发展过程中,与马克思同恩格斯有过直接接触并追随的考茨基、伯恩斯坦、施
米特、桑巴特等人在后期都走向了严重修正,甚至背离马克思而去的道路。严重修正主义者
还包括影响后世几乎所有马克思主义者,写出了《金融资本》的希法亭——即使希法亭至死也
认为自己是马克思主义者。

德国方面的修正主义道路,按书中观点源于当时资本主义危机周期远远超出了马恩预想的十
年周期,六大银行和大工业高度结合,保持稳定和快速发展,还有马克思主义德国社会民主
党当时在德国的迅速发展。另外我认为受俾斯麦式社会主义和国家主义在德国、奥地利的发
展影响,及其对科社的消融作用。“俾斯麦先生说,要粉碎社会主义, 仅仅采取镇压手段是
不够的,还必须采取种种措施以消除不可争辩地存在的社会混乱现象, 保证工作的秩序, 防
止工业危机以及其他等等。 他答应要提出为社会谋福利的这种'积极'的建议。”(马恩全
集,第一版,第19卷,第191页)

希法亭承袭自《资本论》第三卷的金融资本理论虽然是划时代意义的,影响后世所有人,列
宁帝国主义理论就深受他的影响,但我觉得还不足够成熟,暂且不表。“希法亭似乎是自马
克思以后把利润率下降和危机(而且间接地与资本输出)联系起来的第一人。他对比例失调的
分析和对马克思再生产模型的详细考察…… ”又影响了后来的卢森堡和鲍威尔。

考茨基的主要贡献在于帝国主义。他所提的关于农民的看法,我认为是符合资本主义不发达
不全面时期的农民状况的,即使在现世,我们仍可找到应用之处。“农民可能通过‘劳动过
度’和‘消费不足’来抵制资本主义农业的侵蚀(换言之,他们将比产业工人劳动更加努力而
消费得更少)。”

考茨基后期提出的“超帝国”,希法亭后期提出的“总卡特尔”均错误乐观地认为资本主义
发展的稳定将一直持续下去,资本主义将消融国家内部和国家之间的界限,寻求最大理智。
斯滕伯格对修正主义的批评比较形象生动:“修正主义就是'蜜月期理论'”。布哈林和托洛
茨基抓住了重点,布哈林直指严重修正主义的命脉:“有组织的资本主义的概念是自相矛盾
的,因为资本主义本质上是无政府性的。他认为,竞争仅仅是从国家的层面转移到了国际领域,这
使得世界经济并不比以前更加统一,和谐、和平的资本积累依然像以前一样遥远。”

这时期德国、奥地利的学者们有个突出的优点,那就是在伦理上的进步。就马克思专注于社
会性、历史性来说,伦理是依托于社会的。但这并不能表明伦理是不重要的,我们在注意社
会性局限的同时要注意伦理。单提伦理则忽视社会局限,单提社会改造则容易忽视手段正义
并采用反伦理手段,无论哪方都是偏颇,我们要批判辩证。

罗莎·卢森堡坚决的反对修正主义,她再建了马克思再生产图式,认为剩余价值的货币实现
必须经过他国消费。马克思的封闭再生产图示是无法实现全部剩余价值的。另外,她和考茨
基率先提出军国主义对于资本主义发展起着重要作用。对于修正主义改良提出了严重批评。
在诸多方面站在了马克思主义史前沿,并做了奠基。我认为书中对于卢森堡的批评也是站得
住脚的,她的理论还不成熟,多一些政见和为战而战的论战,缺乏完整的理论架构。

\subsection{俄国1917年十月革命前的科社实践}

我在看此书前,对科学社会主义实践基本抱持一种\textbf{远离}的态度,因其在自身发展过
程中出现的一些反人类伦理事件,使我主观排斥曾经的科社实践,希望将自己局限在马克思
主义中其他内容。但是看完此书后,我认为仍需要正视这段历史并加以批判。 \textbf{不管
  出于怎样的情感,俄国在马克思主义理论和实践史上的一页是无法抹杀掉的。}俄国作为中
国曾经的老大哥,也给我们提出了很多理论和实践借鉴。

以下内容基本上原封不动摘抄自本书,只是略加编辑和汇总。想要较好理解还是需要看书,
本篇读后感因我个人能力始终是“\textbf{水货}”。


1883-1929年这段历史中,相当多的学者、政治家在俄国汇集和争辩,民粹主义、孟什维克、
布尔什维克、正统马克思主义者中分离出来的合法马克思主义者和经济主义者。即使是先后
加入布尔什维克的几个人士,梁赞诺夫、列宁、托洛茨基、布哈林、普列奥布拉任斯基等人
也是相当独立并彼此吸收和批判的。但因马恩著作影响和俄国资本主义不发达的现实条
件,“\textbf{1917年之前,作为社会主义必要的前提条件,一个充分发展的资本主义是无论
  如何也不可避免的。所有的非农民政党都赞同资产阶级民主革命,认为它是落后的沙皇俄
  国唯一可能的革命形式。}”

“俄国的马克思主义之父”普列汉诺夫所提出的“革命的代数学”和“革命的算术”,使布
尔什维克和孟什维克等派别(相当于算数)在一定时间内求同(相当于代数学)存异地发展。
关于俄国革命前景,普列汉诺夫提出了资产阶级民主革命和社会主义革命两阶段论。资产阶
级革命将推翻封建制度,使生产力发展,并使无产阶级力量提高,这是社会主义革命前提。
资产阶级民主革命阶段,无产阶级成立独立政治组织保护自己利益,并与资产阶级结
成\textbf{同盟}。如果时机成熟在向社会主义革命过渡时,资产阶级抵制将会引发已经壮大
的无产阶级夺权。两阶段论中,缺乏阶级动力学的有效说明,试图复制西欧经济发展模式,
陷于僵化的历史唯物主义和一厢情愿,成为孟什维克的主旨。稍后我们也会看到他们的有力
之处。

杜冈-巴拉诺夫斯基提出了国家行为对于“\textbf{后发}”俄国发展大工业的重要性以及优
势,俄国将借住国家力量超越其后发资本主义的劣势并融入世界市场。他借助比例失调危机
理论“在分析层面支持了苏联经济学家的观点,即认为通过\textbf{限制消费增长}而不是通
过消费导致的增长来\textbf{加速工业化}”。这可能是俄国大工业理论的开端。我们在之后
还可以看到诸多政治家、经济学家对此的运用。

杜冈-巴拉诺夫斯基还认为客观主义的劳动价值论必须辅之以主观效用价值论。

列宁给我的印象,一是对于各个派别、学者的容纳吸收,民粹主义、普列汉诺夫、孟什维克、
斯托雷平、还有托洛茨基、布哈林等人。但是科社实践必然有政治性,政治中的庸俗和功
利必然导致来龙去脉的不清楚和对所吸收敌对方思想的刻意模糊,这是政治和政治家的通病。

二是对于俄国不断变化的社会现状的准确把握,在当时风云变幻的俄国他常能找到症结所在,
尤其是列宁所关注的农民、\textbf{农民家庭}问题,从19世纪末收
回“割地”,到1905年“消灭地主经济,而不是清扫地主经济”的土地收归国有化,并预料
到“农民可能\textbf{只是没收地主的土地},并把它们作为自己的财产进行重新分配。尽管
这样做不如国有化有益,但列宁还是坚持认为,这种做法也具有高度的进步性”。与已经一定
程度上背弃了普列汉诺夫的孟什维克决裂,断言“激进的资产阶级革命必须由非资产阶级来
推动”,提出了\textbf{工农民主专政}。列宁的基于现实的政治家嗅觉让我惊叹,他的理论
缺陷是这些理论主要在于俄国的“特殊性”方面——如认识到地主、资产阶级已经融入到沙皇
专制等,即使在这些方面受限于当时情况,无法得出一个很准确的结论。他对于世界资本主
义的“一般性”方面虽有建树,但我个人感觉不是很突出。

(\textbf{俄国农民问题实在太复杂},本书介绍也不是很详细。俄国农民状况风云变幻持续
几十年,\textbf{我实在是没有能力借助本书和搜到的寥寥几篇文章获得清晰和正确的认识}。
希望达人补充批评,希望读者通过自己学习,建立自己的认识,\textbf{不要受我误导}。)

托洛茨基直接否定了普列汉诺夫“两阶段论”,认为落后俄国可以将资产阶级民主革
命“\textbf{嵌入}”到“社会主义革命”,从而“后发”俄国先于西方国家实现无产阶级政
权。托洛茨基对如果只是本国内革命,那么农民阶级必然将和无产阶级产生冲突矛盾的论断
是直接也是预言性质的。在这里,我比较怀疑作为农业专家的列宁究竟是否没有预见到推翻
封建专制后的工农矛盾?如果他确实预见到了但没对此详细系统阐述那么极可能是出自政治
性的功利色彩。

托洛茨基基于“不均衡和综合发展”的“不断革命论”非常激进。俄国的不平衡和综合发展
在于俄国国家势力、外国资本占主导下,资产阶级发展受限,大工业却已初见雏形,无产阶
级数量巨大。这使落后俄国可以采取不同于西方发达国家的轨迹。他对于马恩“物质上的先
决条件”有批判也有调和,认为这种先决物质条件存在于俄国之外的工业发达国
家。\textbf{只有不断革命},将革命推至其他国家,革命党才可能成功。除此之
外,“\textbf{历史事件的逻辑将要么绕过它们,要么吞没它们}”。但托洛茨基没有对如何
将革命扩展到世界进行完整有效的理论论述。

书中认为,列宁在《帝国主义论》中的思想受益于布哈林。二人都否定了考茨基“正式的农
业区殖民地”的观点,帝国主义扩张不必受限于农业殖民地、直接的政治控制等。资本输出
可以通过经济方式来进行掠夺。

我个人认为,可能是因为《马克思主义经济学史 1883-1929》写作时间在上世纪80年代,作
者对布哈林的全球化思想没有足够的重视和介绍。布哈林已经提出了较为完整的、我们近二
十年来兴起的“\textbf{民族国家与全球化}”课题的框架,这是具有划时代意义的。我忘了
在哪里看到的布哈林的简单介绍,有句话印象非常深刻,大意是:当前资本的一般性已经是
世界范围,资本的特殊性是国家范围”,这拓展了马克思以封闭国家为范围的资本一般性论
述,并可以针对我们当前现实状况进行说明。亨里克·格罗斯曼对于资本主义反制利润率下降
的几种做法做了阐述,这有效扩展了“民族国家与全球化”课题。以书中观点来看,布哈林
相比同期一些人在政治经济学上是更为宏大的,但弱于马克思的宏大一般,同时又对特殊性
较为漠视,对资本垄断、民族国家内部资本有机等缺乏关注,这方面也招致了列宁的批评。


\subsection{十月革命之后向社会主义的过渡:1917-1929}

对于俄国1917年革命的成功,葛兰西热情洋溢地称赞其为《反<资本论>的革命》,认为其打
破了马克思历史唯物主义中物质先决条件的僵化,俄国成功的在本国资本主义不发达的情况
下建立了新的政治意识形态。1917年之前,孟什维克领导人马xxxx和列宁等针对恩格斯一段
细致的历史唯物论述进行过几次论战,均以恩格斯论点为己方观点辩护,并以此攻击对方。
以今人看来,恩格斯这段话仍是振聋发聩并具有指导意义:

\begin{quotation}
  对于激进派的领袖来说,最糟糕的事情莫过于在运动还没有达到成熟的地步,还没有使他
  所代表的阶级具备进行统治的条件,而且也不可能去实行为维持这个阶级的统治所必须贯
  彻的各项措施的时候,就被迫出来掌握政权。他\textbf{所能}做的事,并不取决于他的意
  志,而取决于不同阶级之间对立的发展程度,取决于历来决定阶级对立发展程度的物质生
  活条件、生产关系和交换关系的发展程度。他\textbf{所应}做的事,他那一派要求他做的
  事,也并不取决于他,而且也不取决于阶级斗争及其条件的发展程度;他不得不恪守自己
  一向鼓吹的理论和要求,而这些理论和要求又并不是产生于当时社会各阶级相互对立的态
  势以及当时生产关系和交换关系的或多或少是偶然的状况,而是产生于他对于社会运动和
  政治运动的一般结果所持的或深或浅的认识。于是他就不可避免地陷入一种无法摆脱的进
  退维谷的境地:他\textbf{所能}做的事,同他迄今为止的全部行动,同他的原则以及他那
  一派的直接利益是互相矛盾的;而他\textbf{所应}做的事,则是无法办到的。总而言之,
  他被迫不代表自己那一派,不代表自己的阶级,而去代表在当时运动中已经具备成熟的统
  治条件的那个阶级。他不得不为运动本身的利益而维护一个异己阶级的利益,不得不以空
  话和诺言来对自己的阶级进行搪塞,声称那个异己阶级的利益就是本阶级的利益。谁要是
  陷入这种窘境,那就无可挽回地要遭到失败。(马克思恩格斯文集,人民出版社2009年,
  第三卷,《德国农民战争》,P303-304。也可见于马恩全集第一版第7卷。)
\end{quotation}

对于1917年革命后的社会主义前景,本书提出了一个较为悲观的看
法:“\textbf{面对1917年以后的困难,布尔什维克完全可能没有任何社会主义式的解决方
  法……}革命政权继承了一个濒临崩溃的经济。”

托洛茨基1904年提出过“党取代阶级”的危险,相比罗莎·卢森堡对于布尔什维克官僚的论述
是更精准和全面的,但后来因托洛茨基看到孟什维克的资产阶级倾向放弃了这一说法,转投
布尔什维克,书中认为这也代表托洛茨基向现实妥协,弱化了不断革命论。他对于前景的预
测也继承了他的直接和有力,认为“如果世界资本主义……能够找到新的动态平衡,这就意
味着我们的基本历史判断出现了错误。这也意味着资本主义还没有完成其历
史‘使命’,而且(帝国主义)还不构成资本主义崩溃的一个阶段。”

本书将苏联在1917年革命成功后到1929年之间的苏联经济史分为了三个不同阶段:

1,1917年10月到1918年6月:农民夺取了土地,但却是以传统公社原则进行了重新分配,使
新政府颁布的正式的土地国有化法令成为多余,并降低了生产率。我认为这一点印证了列宁
的预想。实行的少数工业国有化大多是地方行为,并实行了“\textbf{工人控制}”,私人资
本家受工厂委员会和当地布尔什维克官员监督。列宁将其描述为和公社国家结合的“国家资
本主义”。

2,1918年6月到1921年初:国有化和紧缩经济。实行战时共产主义。试图征收农民全部剩余
价值。取消了公用事业、住房、铁路交通和基本食物配给的收费。工业品不通过货币而是进
行直接配置,工资以实物发放,对城市劳动力实行军事纪律。工人阶级的自治从属于等级制
的控制。对反革命分子实施“红色恐怖”。布哈林在这一阶段做了相当多的妥协,放弃了一
直坚持的完全民主化的“公社国家”主张,让位于中央集权的党的专制。

3,1921年初:列宁得出结论,“要么是经济政策的根本改变,要么是他的政府被暴力推翻”。
开始实行新经济政策。恢复了农民对于农业剩余交易的权力。商业、农业可以雇佣劳工。鼓
励合资企业和敦促共产主义者“学会贸易”。这时我们可以看到历史的吊诡和资本主义无孔
不入的强大,一个新的小资产阶级“耐普曼”产生了,它作为工人和农民之间中间商赚取高
额差价,从而使工农反受剥削。列宁将这一阶段视为“\textbf{过渡性的混合体
  制}”。1923年,爆发“\textbf{剪刀差危机}”,工业品相对于农产品来说过高,农民不
愿在市场出售粮食。这一危机在不久消除之后,1928年又发生了“粮食危机”,农产品供应
不足。

布哈林长期目标没变,但在新经济政策时期认为\textbf{应当依靠非社会主义形式的增长实
  现社会主义……国有工业取决于农民需求的增长……资本主义的长期稳定的确是在新经济
  政策下实现渐进主义的苏维埃工业化的必要的前提}。这些论点使他招致很多批评。

普列奥布拉任斯基与布哈林的均衡发展相对。他认为国有工业在经济增长中必须占据支配地
位。同时,通过向农业集体化提供资源(机械化等),可以间接产生同样的效果。同时,这
也将削弱“耐普曼”和世界经济的不良影响。提出“\textbf{社会主义原始积累}”的概念。
我认为在此书前文中所说的“杜冈-巴拉诺夫斯基的影响”开始正式显示他的强大力量(这里
说的强大力量,不是对错意义上的,而是实践意义上的)。普列奥布拉任斯基通过结合社会
主义原始积累的代数学、马克思和卢森堡的再生产图式等,得出了悲观结论,“一国社会主
义”是不可能的。

斯大林在1928年实行粮食征用,1929-1933年间实行强制的集体化。这通过毁灭一切农民独
立性的残余,缓解了快速工业化中来自农村的制约;早已存在的来自苏联无产阶级的抵制变
得不可能。斯大林主义的论述将在下一卷展开。

\subsection{我对于1929年为止科社实践的个人看法}

前文中引用的恩格斯的那段话有必要重新关注一下。

俄国科社实践,始终强烈面对着资产阶级复辟,布尔什维克被推翻的危险。理论与实践的结
合必然要考虑现实、政治,这种结合必然会造成种种冲突矛盾。不能说布尔什维克就是政治
投机或自利,他们在变幻莫测中要走一条前所未有的艰难道路。同时也要正视中间的种种问
题:例如布尔什维克的党取代阶级问题,科社实践活动中是否很可能没有任何一个成功实践
的国家是真正的无产阶级起领导作用?例如我们悲观的看到底层农民阶级始终常常面对着
困境,我个人认为这是否说明社会变革中,弱者恒弱?

